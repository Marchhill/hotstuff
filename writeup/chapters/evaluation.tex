% This is where Assessors will be looking for signs of success and for evidence of thorough and systematic evaluation. Sample output, tables of timings and photographs of workstation screens, oscilloscope traces or circuit boards may be included. Care should be employed to take a professional approach throughout. For example, a graph that does not indicate confidence intervals will generally leave a professional scientist with a negative impression. As with code, voluminous examples of sample output are usually best left to appendices or omitted altogether.

% There are some obvious questions which this chapter will address. How many of the original goals were achieved? Were they proved to have been achieved? Did the program, hardware, or theory really work?

% Assessors are well aware that large programs will very likely include some residual bugs. It should always be possible to demonstrate that a program works in simple cases and it is instructive to demonstrate how close it is to working in a really ambitious case.

% graphs, performance, ablation
% run in mininet
% https://hub.docker.com/r/iwaseyusuke/mininet/
% discuss the implications of the graph
%base on evaluation chapter of papers!

This section highlights the methods and hardware used in evaluation (Section~\ref{testingmethods}), benchmarks the performance of Cap'n Proto and the Tezos cryptography library (Section~\ref{librarybenchmarks}), and finally evaluates the performance of my HotStuff implementation (Section~\ref{hotstuffbenchmarks}).

\section{Testing methodology} \label{testingmethods}
The evaluation was carried out on the computer laboratory's Sofia server (2x Xeon Gold 6230R chips, 768GB RAM). Carrying out experiments on the server rather than my laptop helped to minimise interference from other processes on the system.

Experiments were driven by an open-loop load generator (Section~\ref{loadgenerator}) and automated using Python scripts (Section~\ref{experimentscripts}). The load generator was run for varying amounts of time in the different experiments, with a further 15 seconds after this without the load generator running to wait for any slow responses.

To reduce the effect of interference, experiments were repeated three times, and the order of experiments was randomly permuted. Where a confidence interval is shown, this shows the range of results over the three repeats.

In experiments where throughput (in req/s) is varied, it starts at 1, increases exponentially (in multiples of 2) from 25 to 200 to benchmark performance at lower throughputs, and then increases linearly (by increments of 200) up to its maximum value (which varies between experiments). Message sizes are also varied in this way.

\section{Library benchmarks} \label{librarybenchmarks}
This section presents benchmarks of the performance of the Cap'n Proto RPC framework, and the Tezos cryptography library.

\subsection{Cap'n Proto} \label{capnpbenchmark}

\begin{figure}[h!]
\centering
\resizebox{.6\textwidth}{!}{%% Creator: Matplotlib, PGF backend
%%
%% To include the figure in your LaTeX document, write
%%   \input{<filename>.pgf}
%%
%% Make sure the required packages are loaded in your preamble
%%   \usepackage{pgf}
%%
%% Also ensure that all the required font packages are loaded; for instance,
%% the lmodern package is sometimes necessary when using math font.
%%   \usepackage{lmodern}
%%
%% Figures using additional raster images can only be included by \input if
%% they are in the same directory as the main LaTeX file. For loading figures
%% from other directories you can use the `import` package
%%   \usepackage{import}
%%
%% and then include the figures with
%%   \import{<path to file>}{<filename>.pgf}
%%
%% Matplotlib used the following preamble
%%   
%%   \usepackage{fontspec}
%%   \setmainfont{DejaVuSerif.ttf}[Path=\detokenize{/opt/homebrew/lib/python3.10/site-packages/matplotlib/mpl-data/fonts/ttf/}]
%%   \setsansfont{DejaVuSans.ttf}[Path=\detokenize{/opt/homebrew/lib/python3.10/site-packages/matplotlib/mpl-data/fonts/ttf/}]
%%   \setmonofont{DejaVuSansMono.ttf}[Path=\detokenize{/opt/homebrew/lib/python3.10/site-packages/matplotlib/mpl-data/fonts/ttf/}]
%%   \makeatletter\@ifpackageloaded{underscore}{}{\usepackage[strings]{underscore}}\makeatother
%%
\begingroup%
\makeatletter%
\begin{pgfpicture}%
\pgfpathrectangle{\pgfpointorigin}{\pgfqpoint{6.400000in}{4.800000in}}%
\pgfusepath{use as bounding box, clip}%
\begin{pgfscope}%
\pgfsetbuttcap%
\pgfsetmiterjoin%
\definecolor{currentfill}{rgb}{1.000000,1.000000,1.000000}%
\pgfsetfillcolor{currentfill}%
\pgfsetlinewidth{0.000000pt}%
\definecolor{currentstroke}{rgb}{1.000000,1.000000,1.000000}%
\pgfsetstrokecolor{currentstroke}%
\pgfsetdash{}{0pt}%
\pgfpathmoveto{\pgfqpoint{0.000000in}{0.000000in}}%
\pgfpathlineto{\pgfqpoint{6.400000in}{0.000000in}}%
\pgfpathlineto{\pgfqpoint{6.400000in}{4.800000in}}%
\pgfpathlineto{\pgfqpoint{0.000000in}{4.800000in}}%
\pgfpathlineto{\pgfqpoint{0.000000in}{0.000000in}}%
\pgfpathclose%
\pgfusepath{fill}%
\end{pgfscope}%
\begin{pgfscope}%
\pgfsetbuttcap%
\pgfsetmiterjoin%
\definecolor{currentfill}{rgb}{1.000000,1.000000,1.000000}%
\pgfsetfillcolor{currentfill}%
\pgfsetlinewidth{0.000000pt}%
\definecolor{currentstroke}{rgb}{0.000000,0.000000,0.000000}%
\pgfsetstrokecolor{currentstroke}%
\pgfsetstrokeopacity{0.000000}%
\pgfsetdash{}{0pt}%
\pgfpathmoveto{\pgfqpoint{0.800000in}{0.528000in}}%
\pgfpathlineto{\pgfqpoint{5.760000in}{0.528000in}}%
\pgfpathlineto{\pgfqpoint{5.760000in}{4.224000in}}%
\pgfpathlineto{\pgfqpoint{0.800000in}{4.224000in}}%
\pgfpathlineto{\pgfqpoint{0.800000in}{0.528000in}}%
\pgfpathclose%
\pgfusepath{fill}%
\end{pgfscope}%
\begin{pgfscope}%
\pgfpathrectangle{\pgfqpoint{0.800000in}{0.528000in}}{\pgfqpoint{4.960000in}{3.696000in}}%
\pgfusepath{clip}%
\pgfsetbuttcap%
\pgfsetroundjoin%
\definecolor{currentfill}{rgb}{0.121569,0.466667,0.705882}%
\pgfsetfillcolor{currentfill}%
\pgfsetfillopacity{0.200000}%
\pgfsetlinewidth{1.003750pt}%
\definecolor{currentstroke}{rgb}{0.121569,0.466667,0.705882}%
\pgfsetstrokecolor{currentstroke}%
\pgfsetstrokeopacity{0.200000}%
\pgfsetdash{}{0pt}%
\pgfsys@defobject{currentmarker}{\pgfqpoint{1.025455in}{0.696000in}}{\pgfqpoint{5.534545in}{4.056000in}}{%
\pgfpathmoveto{\pgfqpoint{1.025455in}{0.696036in}}%
\pgfpathlineto{\pgfqpoint{1.025455in}{0.696000in}}%
\pgfpathlineto{\pgfqpoint{1.043494in}{0.866911in}}%
\pgfpathlineto{\pgfqpoint{1.062285in}{1.045837in}}%
\pgfpathlineto{\pgfqpoint{1.099867in}{1.377186in}}%
\pgfpathlineto{\pgfqpoint{1.175031in}{2.066905in}}%
\pgfpathlineto{\pgfqpoint{1.325359in}{2.939423in}}%
\pgfpathlineto{\pgfqpoint{1.475687in}{3.512575in}}%
\pgfpathlineto{\pgfqpoint{1.626015in}{3.809453in}}%
\pgfpathlineto{\pgfqpoint{1.776343in}{3.847216in}}%
\pgfpathlineto{\pgfqpoint{1.926671in}{3.764579in}}%
\pgfpathlineto{\pgfqpoint{2.076999in}{3.675952in}}%
\pgfpathlineto{\pgfqpoint{2.227328in}{3.577120in}}%
\pgfpathlineto{\pgfqpoint{2.377656in}{3.457800in}}%
\pgfpathlineto{\pgfqpoint{2.527984in}{3.339651in}}%
\pgfpathlineto{\pgfqpoint{2.678312in}{2.127416in}}%
\pgfpathlineto{\pgfqpoint{2.828640in}{1.733430in}}%
\pgfpathlineto{\pgfqpoint{2.978968in}{1.560635in}}%
\pgfpathlineto{\pgfqpoint{3.129296in}{1.443373in}}%
\pgfpathlineto{\pgfqpoint{3.279624in}{1.372537in}}%
\pgfpathlineto{\pgfqpoint{3.429952in}{1.295900in}}%
\pgfpathlineto{\pgfqpoint{3.580280in}{1.285168in}}%
\pgfpathlineto{\pgfqpoint{3.730608in}{1.245601in}}%
\pgfpathlineto{\pgfqpoint{3.880937in}{1.214375in}}%
\pgfpathlineto{\pgfqpoint{4.031265in}{1.166509in}}%
\pgfpathlineto{\pgfqpoint{4.181593in}{1.144141in}}%
\pgfpathlineto{\pgfqpoint{4.331921in}{1.136395in}}%
\pgfpathlineto{\pgfqpoint{4.482249in}{1.136476in}}%
\pgfpathlineto{\pgfqpoint{4.632577in}{1.106948in}}%
\pgfpathlineto{\pgfqpoint{4.782905in}{1.094690in}}%
\pgfpathlineto{\pgfqpoint{4.933233in}{1.068745in}}%
\pgfpathlineto{\pgfqpoint{5.083561in}{1.073059in}}%
\pgfpathlineto{\pgfqpoint{5.233889in}{1.044272in}}%
\pgfpathlineto{\pgfqpoint{5.384217in}{1.038739in}}%
\pgfpathlineto{\pgfqpoint{5.534545in}{1.043230in}}%
\pgfpathlineto{\pgfqpoint{5.534545in}{1.048870in}}%
\pgfpathlineto{\pgfqpoint{5.534545in}{1.048870in}}%
\pgfpathlineto{\pgfqpoint{5.384217in}{1.047151in}}%
\pgfpathlineto{\pgfqpoint{5.233889in}{1.069089in}}%
\pgfpathlineto{\pgfqpoint{5.083561in}{1.090028in}}%
\pgfpathlineto{\pgfqpoint{4.933233in}{1.086203in}}%
\pgfpathlineto{\pgfqpoint{4.782905in}{1.102345in}}%
\pgfpathlineto{\pgfqpoint{4.632577in}{1.123321in}}%
\pgfpathlineto{\pgfqpoint{4.482249in}{1.143024in}}%
\pgfpathlineto{\pgfqpoint{4.331921in}{1.165349in}}%
\pgfpathlineto{\pgfqpoint{4.181593in}{1.176403in}}%
\pgfpathlineto{\pgfqpoint{4.031265in}{1.207548in}}%
\pgfpathlineto{\pgfqpoint{3.880937in}{1.237441in}}%
\pgfpathlineto{\pgfqpoint{3.730608in}{1.270836in}}%
\pgfpathlineto{\pgfqpoint{3.580280in}{1.321330in}}%
\pgfpathlineto{\pgfqpoint{3.429952in}{1.329505in}}%
\pgfpathlineto{\pgfqpoint{3.279624in}{1.403639in}}%
\pgfpathlineto{\pgfqpoint{3.129296in}{1.477740in}}%
\pgfpathlineto{\pgfqpoint{2.978968in}{1.611542in}}%
\pgfpathlineto{\pgfqpoint{2.828640in}{1.774361in}}%
\pgfpathlineto{\pgfqpoint{2.678312in}{2.189400in}}%
\pgfpathlineto{\pgfqpoint{2.527984in}{3.486673in}}%
\pgfpathlineto{\pgfqpoint{2.377656in}{3.489418in}}%
\pgfpathlineto{\pgfqpoint{2.227328in}{3.797700in}}%
\pgfpathlineto{\pgfqpoint{2.076999in}{3.856439in}}%
\pgfpathlineto{\pgfqpoint{1.926671in}{4.056000in}}%
\pgfpathlineto{\pgfqpoint{1.776343in}{3.887503in}}%
\pgfpathlineto{\pgfqpoint{1.626015in}{3.935923in}}%
\pgfpathlineto{\pgfqpoint{1.475687in}{3.702859in}}%
\pgfpathlineto{\pgfqpoint{1.325359in}{2.958277in}}%
\pgfpathlineto{\pgfqpoint{1.175031in}{2.078443in}}%
\pgfpathlineto{\pgfqpoint{1.099867in}{1.383579in}}%
\pgfpathlineto{\pgfqpoint{1.062285in}{1.059959in}}%
\pgfpathlineto{\pgfqpoint{1.043494in}{0.869434in}}%
\pgfpathlineto{\pgfqpoint{1.025455in}{0.696036in}}%
\pgfpathlineto{\pgfqpoint{1.025455in}{0.696036in}}%
\pgfpathclose%
\pgfusepath{stroke,fill}%
}%
\begin{pgfscope}%
\pgfsys@transformshift{0.000000in}{0.000000in}%
\pgfsys@useobject{currentmarker}{}%
\end{pgfscope}%
\end{pgfscope}%
\begin{pgfscope}%
\pgfsetbuttcap%
\pgfsetroundjoin%
\definecolor{currentfill}{rgb}{0.000000,0.000000,0.000000}%
\pgfsetfillcolor{currentfill}%
\pgfsetlinewidth{0.803000pt}%
\definecolor{currentstroke}{rgb}{0.000000,0.000000,0.000000}%
\pgfsetstrokecolor{currentstroke}%
\pgfsetdash{}{0pt}%
\pgfsys@defobject{currentmarker}{\pgfqpoint{0.000000in}{-0.048611in}}{\pgfqpoint{0.000000in}{0.000000in}}{%
\pgfpathmoveto{\pgfqpoint{0.000000in}{0.000000in}}%
\pgfpathlineto{\pgfqpoint{0.000000in}{-0.048611in}}%
\pgfusepath{stroke,fill}%
}%
\begin{pgfscope}%
\pgfsys@transformshift{1.024703in}{0.528000in}%
\pgfsys@useobject{currentmarker}{}%
\end{pgfscope}%
\end{pgfscope}%
\begin{pgfscope}%
\definecolor{textcolor}{rgb}{0.000000,0.000000,0.000000}%
\pgfsetstrokecolor{textcolor}%
\pgfsetfillcolor{textcolor}%
\pgftext[x=1.024703in,y=0.430778in,,top]{\color{textcolor}\sffamily\fontsize{10.000000}{12.000000}\selectfont 0}%
\end{pgfscope}%
\begin{pgfscope}%
\pgfsetbuttcap%
\pgfsetroundjoin%
\definecolor{currentfill}{rgb}{0.000000,0.000000,0.000000}%
\pgfsetfillcolor{currentfill}%
\pgfsetlinewidth{0.803000pt}%
\definecolor{currentstroke}{rgb}{0.000000,0.000000,0.000000}%
\pgfsetstrokecolor{currentstroke}%
\pgfsetdash{}{0pt}%
\pgfsys@defobject{currentmarker}{\pgfqpoint{0.000000in}{-0.048611in}}{\pgfqpoint{0.000000in}{0.000000in}}{%
\pgfpathmoveto{\pgfqpoint{0.000000in}{0.000000in}}%
\pgfpathlineto{\pgfqpoint{0.000000in}{-0.048611in}}%
\pgfusepath{stroke,fill}%
}%
\begin{pgfscope}%
\pgfsys@transformshift{1.776343in}{0.528000in}%
\pgfsys@useobject{currentmarker}{}%
\end{pgfscope}%
\end{pgfscope}%
\begin{pgfscope}%
\definecolor{textcolor}{rgb}{0.000000,0.000000,0.000000}%
\pgfsetstrokecolor{textcolor}%
\pgfsetfillcolor{textcolor}%
\pgftext[x=1.776343in,y=0.430778in,,top]{\color{textcolor}\sffamily\fontsize{10.000000}{12.000000}\selectfont 1000}%
\end{pgfscope}%
\begin{pgfscope}%
\pgfsetbuttcap%
\pgfsetroundjoin%
\definecolor{currentfill}{rgb}{0.000000,0.000000,0.000000}%
\pgfsetfillcolor{currentfill}%
\pgfsetlinewidth{0.803000pt}%
\definecolor{currentstroke}{rgb}{0.000000,0.000000,0.000000}%
\pgfsetstrokecolor{currentstroke}%
\pgfsetdash{}{0pt}%
\pgfsys@defobject{currentmarker}{\pgfqpoint{0.000000in}{-0.048611in}}{\pgfqpoint{0.000000in}{0.000000in}}{%
\pgfpathmoveto{\pgfqpoint{0.000000in}{0.000000in}}%
\pgfpathlineto{\pgfqpoint{0.000000in}{-0.048611in}}%
\pgfusepath{stroke,fill}%
}%
\begin{pgfscope}%
\pgfsys@transformshift{2.527984in}{0.528000in}%
\pgfsys@useobject{currentmarker}{}%
\end{pgfscope}%
\end{pgfscope}%
\begin{pgfscope}%
\definecolor{textcolor}{rgb}{0.000000,0.000000,0.000000}%
\pgfsetstrokecolor{textcolor}%
\pgfsetfillcolor{textcolor}%
\pgftext[x=2.527984in,y=0.430778in,,top]{\color{textcolor}\sffamily\fontsize{10.000000}{12.000000}\selectfont 2000}%
\end{pgfscope}%
\begin{pgfscope}%
\pgfsetbuttcap%
\pgfsetroundjoin%
\definecolor{currentfill}{rgb}{0.000000,0.000000,0.000000}%
\pgfsetfillcolor{currentfill}%
\pgfsetlinewidth{0.803000pt}%
\definecolor{currentstroke}{rgb}{0.000000,0.000000,0.000000}%
\pgfsetstrokecolor{currentstroke}%
\pgfsetdash{}{0pt}%
\pgfsys@defobject{currentmarker}{\pgfqpoint{0.000000in}{-0.048611in}}{\pgfqpoint{0.000000in}{0.000000in}}{%
\pgfpathmoveto{\pgfqpoint{0.000000in}{0.000000in}}%
\pgfpathlineto{\pgfqpoint{0.000000in}{-0.048611in}}%
\pgfusepath{stroke,fill}%
}%
\begin{pgfscope}%
\pgfsys@transformshift{3.279624in}{0.528000in}%
\pgfsys@useobject{currentmarker}{}%
\end{pgfscope}%
\end{pgfscope}%
\begin{pgfscope}%
\definecolor{textcolor}{rgb}{0.000000,0.000000,0.000000}%
\pgfsetstrokecolor{textcolor}%
\pgfsetfillcolor{textcolor}%
\pgftext[x=3.279624in,y=0.430778in,,top]{\color{textcolor}\sffamily\fontsize{10.000000}{12.000000}\selectfont 3000}%
\end{pgfscope}%
\begin{pgfscope}%
\pgfsetbuttcap%
\pgfsetroundjoin%
\definecolor{currentfill}{rgb}{0.000000,0.000000,0.000000}%
\pgfsetfillcolor{currentfill}%
\pgfsetlinewidth{0.803000pt}%
\definecolor{currentstroke}{rgb}{0.000000,0.000000,0.000000}%
\pgfsetstrokecolor{currentstroke}%
\pgfsetdash{}{0pt}%
\pgfsys@defobject{currentmarker}{\pgfqpoint{0.000000in}{-0.048611in}}{\pgfqpoint{0.000000in}{0.000000in}}{%
\pgfpathmoveto{\pgfqpoint{0.000000in}{0.000000in}}%
\pgfpathlineto{\pgfqpoint{0.000000in}{-0.048611in}}%
\pgfusepath{stroke,fill}%
}%
\begin{pgfscope}%
\pgfsys@transformshift{4.031265in}{0.528000in}%
\pgfsys@useobject{currentmarker}{}%
\end{pgfscope}%
\end{pgfscope}%
\begin{pgfscope}%
\definecolor{textcolor}{rgb}{0.000000,0.000000,0.000000}%
\pgfsetstrokecolor{textcolor}%
\pgfsetfillcolor{textcolor}%
\pgftext[x=4.031265in,y=0.430778in,,top]{\color{textcolor}\sffamily\fontsize{10.000000}{12.000000}\selectfont 4000}%
\end{pgfscope}%
\begin{pgfscope}%
\pgfsetbuttcap%
\pgfsetroundjoin%
\definecolor{currentfill}{rgb}{0.000000,0.000000,0.000000}%
\pgfsetfillcolor{currentfill}%
\pgfsetlinewidth{0.803000pt}%
\definecolor{currentstroke}{rgb}{0.000000,0.000000,0.000000}%
\pgfsetstrokecolor{currentstroke}%
\pgfsetdash{}{0pt}%
\pgfsys@defobject{currentmarker}{\pgfqpoint{0.000000in}{-0.048611in}}{\pgfqpoint{0.000000in}{0.000000in}}{%
\pgfpathmoveto{\pgfqpoint{0.000000in}{0.000000in}}%
\pgfpathlineto{\pgfqpoint{0.000000in}{-0.048611in}}%
\pgfusepath{stroke,fill}%
}%
\begin{pgfscope}%
\pgfsys@transformshift{4.782905in}{0.528000in}%
\pgfsys@useobject{currentmarker}{}%
\end{pgfscope}%
\end{pgfscope}%
\begin{pgfscope}%
\definecolor{textcolor}{rgb}{0.000000,0.000000,0.000000}%
\pgfsetstrokecolor{textcolor}%
\pgfsetfillcolor{textcolor}%
\pgftext[x=4.782905in,y=0.430778in,,top]{\color{textcolor}\sffamily\fontsize{10.000000}{12.000000}\selectfont 5000}%
\end{pgfscope}%
\begin{pgfscope}%
\pgfsetbuttcap%
\pgfsetroundjoin%
\definecolor{currentfill}{rgb}{0.000000,0.000000,0.000000}%
\pgfsetfillcolor{currentfill}%
\pgfsetlinewidth{0.803000pt}%
\definecolor{currentstroke}{rgb}{0.000000,0.000000,0.000000}%
\pgfsetstrokecolor{currentstroke}%
\pgfsetdash{}{0pt}%
\pgfsys@defobject{currentmarker}{\pgfqpoint{0.000000in}{-0.048611in}}{\pgfqpoint{0.000000in}{0.000000in}}{%
\pgfpathmoveto{\pgfqpoint{0.000000in}{0.000000in}}%
\pgfpathlineto{\pgfqpoint{0.000000in}{-0.048611in}}%
\pgfusepath{stroke,fill}%
}%
\begin{pgfscope}%
\pgfsys@transformshift{5.534545in}{0.528000in}%
\pgfsys@useobject{currentmarker}{}%
\end{pgfscope}%
\end{pgfscope}%
\begin{pgfscope}%
\definecolor{textcolor}{rgb}{0.000000,0.000000,0.000000}%
\pgfsetstrokecolor{textcolor}%
\pgfsetfillcolor{textcolor}%
\pgftext[x=5.534545in,y=0.430778in,,top]{\color{textcolor}\sffamily\fontsize{10.000000}{12.000000}\selectfont 6000}%
\end{pgfscope}%
\begin{pgfscope}%
\definecolor{textcolor}{rgb}{0.000000,0.000000,0.000000}%
\pgfsetstrokecolor{textcolor}%
\pgfsetfillcolor{textcolor}%
\pgftext[x=3.280000in,y=0.240809in,,top]{\color{textcolor}\sffamily\fontsize{10.000000}{12.000000}\selectfont msg size (bytes)}%
\end{pgfscope}%
\begin{pgfscope}%
\pgfsetbuttcap%
\pgfsetroundjoin%
\definecolor{currentfill}{rgb}{0.000000,0.000000,0.000000}%
\pgfsetfillcolor{currentfill}%
\pgfsetlinewidth{0.803000pt}%
\definecolor{currentstroke}{rgb}{0.000000,0.000000,0.000000}%
\pgfsetstrokecolor{currentstroke}%
\pgfsetdash{}{0pt}%
\pgfsys@defobject{currentmarker}{\pgfqpoint{-0.048611in}{0.000000in}}{\pgfqpoint{-0.000000in}{0.000000in}}{%
\pgfpathmoveto{\pgfqpoint{-0.000000in}{0.000000in}}%
\pgfpathlineto{\pgfqpoint{-0.048611in}{0.000000in}}%
\pgfusepath{stroke,fill}%
}%
\begin{pgfscope}%
\pgfsys@transformshift{0.800000in}{0.688864in}%
\pgfsys@useobject{currentmarker}{}%
\end{pgfscope}%
\end{pgfscope}%
\begin{pgfscope}%
\definecolor{textcolor}{rgb}{0.000000,0.000000,0.000000}%
\pgfsetstrokecolor{textcolor}%
\pgfsetfillcolor{textcolor}%
\pgftext[x=0.614412in, y=0.636103in, left, base]{\color{textcolor}\sffamily\fontsize{10.000000}{12.000000}\selectfont 0}%
\end{pgfscope}%
\begin{pgfscope}%
\pgfsetbuttcap%
\pgfsetroundjoin%
\definecolor{currentfill}{rgb}{0.000000,0.000000,0.000000}%
\pgfsetfillcolor{currentfill}%
\pgfsetlinewidth{0.803000pt}%
\definecolor{currentstroke}{rgb}{0.000000,0.000000,0.000000}%
\pgfsetstrokecolor{currentstroke}%
\pgfsetdash{}{0pt}%
\pgfsys@defobject{currentmarker}{\pgfqpoint{-0.048611in}{0.000000in}}{\pgfqpoint{-0.000000in}{0.000000in}}{%
\pgfpathmoveto{\pgfqpoint{-0.000000in}{0.000000in}}%
\pgfpathlineto{\pgfqpoint{-0.048611in}{0.000000in}}%
\pgfusepath{stroke,fill}%
}%
\begin{pgfscope}%
\pgfsys@transformshift{0.800000in}{1.226026in}%
\pgfsys@useobject{currentmarker}{}%
\end{pgfscope}%
\end{pgfscope}%
\begin{pgfscope}%
\definecolor{textcolor}{rgb}{0.000000,0.000000,0.000000}%
\pgfsetstrokecolor{textcolor}%
\pgfsetfillcolor{textcolor}%
\pgftext[x=0.614412in, y=1.173265in, left, base]{\color{textcolor}\sffamily\fontsize{10.000000}{12.000000}\selectfont 2}%
\end{pgfscope}%
\begin{pgfscope}%
\pgfsetbuttcap%
\pgfsetroundjoin%
\definecolor{currentfill}{rgb}{0.000000,0.000000,0.000000}%
\pgfsetfillcolor{currentfill}%
\pgfsetlinewidth{0.803000pt}%
\definecolor{currentstroke}{rgb}{0.000000,0.000000,0.000000}%
\pgfsetstrokecolor{currentstroke}%
\pgfsetdash{}{0pt}%
\pgfsys@defobject{currentmarker}{\pgfqpoint{-0.048611in}{0.000000in}}{\pgfqpoint{-0.000000in}{0.000000in}}{%
\pgfpathmoveto{\pgfqpoint{-0.000000in}{0.000000in}}%
\pgfpathlineto{\pgfqpoint{-0.048611in}{0.000000in}}%
\pgfusepath{stroke,fill}%
}%
\begin{pgfscope}%
\pgfsys@transformshift{0.800000in}{1.763188in}%
\pgfsys@useobject{currentmarker}{}%
\end{pgfscope}%
\end{pgfscope}%
\begin{pgfscope}%
\definecolor{textcolor}{rgb}{0.000000,0.000000,0.000000}%
\pgfsetstrokecolor{textcolor}%
\pgfsetfillcolor{textcolor}%
\pgftext[x=0.614412in, y=1.710427in, left, base]{\color{textcolor}\sffamily\fontsize{10.000000}{12.000000}\selectfont 4}%
\end{pgfscope}%
\begin{pgfscope}%
\pgfsetbuttcap%
\pgfsetroundjoin%
\definecolor{currentfill}{rgb}{0.000000,0.000000,0.000000}%
\pgfsetfillcolor{currentfill}%
\pgfsetlinewidth{0.803000pt}%
\definecolor{currentstroke}{rgb}{0.000000,0.000000,0.000000}%
\pgfsetstrokecolor{currentstroke}%
\pgfsetdash{}{0pt}%
\pgfsys@defobject{currentmarker}{\pgfqpoint{-0.048611in}{0.000000in}}{\pgfqpoint{-0.000000in}{0.000000in}}{%
\pgfpathmoveto{\pgfqpoint{-0.000000in}{0.000000in}}%
\pgfpathlineto{\pgfqpoint{-0.048611in}{0.000000in}}%
\pgfusepath{stroke,fill}%
}%
\begin{pgfscope}%
\pgfsys@transformshift{0.800000in}{2.300350in}%
\pgfsys@useobject{currentmarker}{}%
\end{pgfscope}%
\end{pgfscope}%
\begin{pgfscope}%
\definecolor{textcolor}{rgb}{0.000000,0.000000,0.000000}%
\pgfsetstrokecolor{textcolor}%
\pgfsetfillcolor{textcolor}%
\pgftext[x=0.614412in, y=2.247589in, left, base]{\color{textcolor}\sffamily\fontsize{10.000000}{12.000000}\selectfont 6}%
\end{pgfscope}%
\begin{pgfscope}%
\pgfsetbuttcap%
\pgfsetroundjoin%
\definecolor{currentfill}{rgb}{0.000000,0.000000,0.000000}%
\pgfsetfillcolor{currentfill}%
\pgfsetlinewidth{0.803000pt}%
\definecolor{currentstroke}{rgb}{0.000000,0.000000,0.000000}%
\pgfsetstrokecolor{currentstroke}%
\pgfsetdash{}{0pt}%
\pgfsys@defobject{currentmarker}{\pgfqpoint{-0.048611in}{0.000000in}}{\pgfqpoint{-0.000000in}{0.000000in}}{%
\pgfpathmoveto{\pgfqpoint{-0.000000in}{0.000000in}}%
\pgfpathlineto{\pgfqpoint{-0.048611in}{0.000000in}}%
\pgfusepath{stroke,fill}%
}%
\begin{pgfscope}%
\pgfsys@transformshift{0.800000in}{2.837512in}%
\pgfsys@useobject{currentmarker}{}%
\end{pgfscope}%
\end{pgfscope}%
\begin{pgfscope}%
\definecolor{textcolor}{rgb}{0.000000,0.000000,0.000000}%
\pgfsetstrokecolor{textcolor}%
\pgfsetfillcolor{textcolor}%
\pgftext[x=0.614412in, y=2.784751in, left, base]{\color{textcolor}\sffamily\fontsize{10.000000}{12.000000}\selectfont 8}%
\end{pgfscope}%
\begin{pgfscope}%
\pgfsetbuttcap%
\pgfsetroundjoin%
\definecolor{currentfill}{rgb}{0.000000,0.000000,0.000000}%
\pgfsetfillcolor{currentfill}%
\pgfsetlinewidth{0.803000pt}%
\definecolor{currentstroke}{rgb}{0.000000,0.000000,0.000000}%
\pgfsetstrokecolor{currentstroke}%
\pgfsetdash{}{0pt}%
\pgfsys@defobject{currentmarker}{\pgfqpoint{-0.048611in}{0.000000in}}{\pgfqpoint{-0.000000in}{0.000000in}}{%
\pgfpathmoveto{\pgfqpoint{-0.000000in}{0.000000in}}%
\pgfpathlineto{\pgfqpoint{-0.048611in}{0.000000in}}%
\pgfusepath{stroke,fill}%
}%
\begin{pgfscope}%
\pgfsys@transformshift{0.800000in}{3.374674in}%
\pgfsys@useobject{currentmarker}{}%
\end{pgfscope}%
\end{pgfscope}%
\begin{pgfscope}%
\definecolor{textcolor}{rgb}{0.000000,0.000000,0.000000}%
\pgfsetstrokecolor{textcolor}%
\pgfsetfillcolor{textcolor}%
\pgftext[x=0.526047in, y=3.321913in, left, base]{\color{textcolor}\sffamily\fontsize{10.000000}{12.000000}\selectfont 10}%
\end{pgfscope}%
\begin{pgfscope}%
\pgfsetbuttcap%
\pgfsetroundjoin%
\definecolor{currentfill}{rgb}{0.000000,0.000000,0.000000}%
\pgfsetfillcolor{currentfill}%
\pgfsetlinewidth{0.803000pt}%
\definecolor{currentstroke}{rgb}{0.000000,0.000000,0.000000}%
\pgfsetstrokecolor{currentstroke}%
\pgfsetdash{}{0pt}%
\pgfsys@defobject{currentmarker}{\pgfqpoint{-0.048611in}{0.000000in}}{\pgfqpoint{-0.000000in}{0.000000in}}{%
\pgfpathmoveto{\pgfqpoint{-0.000000in}{0.000000in}}%
\pgfpathlineto{\pgfqpoint{-0.048611in}{0.000000in}}%
\pgfusepath{stroke,fill}%
}%
\begin{pgfscope}%
\pgfsys@transformshift{0.800000in}{3.911836in}%
\pgfsys@useobject{currentmarker}{}%
\end{pgfscope}%
\end{pgfscope}%
\begin{pgfscope}%
\definecolor{textcolor}{rgb}{0.000000,0.000000,0.000000}%
\pgfsetstrokecolor{textcolor}%
\pgfsetfillcolor{textcolor}%
\pgftext[x=0.526047in, y=3.859075in, left, base]{\color{textcolor}\sffamily\fontsize{10.000000}{12.000000}\selectfont 12}%
\end{pgfscope}%
\begin{pgfscope}%
\definecolor{textcolor}{rgb}{0.000000,0.000000,0.000000}%
\pgfsetstrokecolor{textcolor}%
\pgfsetfillcolor{textcolor}%
\pgftext[x=0.470492in,y=2.376000in,,bottom,rotate=90.000000]{\color{textcolor}\sffamily\fontsize{10.000000}{12.000000}\selectfont max goodput (Mb/s)}%
\end{pgfscope}%
\begin{pgfscope}%
\pgfpathrectangle{\pgfqpoint{0.800000in}{0.528000in}}{\pgfqpoint{4.960000in}{3.696000in}}%
\pgfusepath{clip}%
\pgfsetbuttcap%
\pgfsetroundjoin%
\pgfsetlinewidth{1.505625pt}%
\definecolor{currentstroke}{rgb}{0.121569,0.466667,0.705882}%
\pgfsetstrokecolor{currentstroke}%
\pgfsetdash{{5.550000pt}{2.400000pt}}{0.000000pt}%
\pgfpathmoveto{\pgfqpoint{1.025455in}{0.696015in}}%
\pgfpathlineto{\pgfqpoint{1.043494in}{0.868275in}}%
\pgfpathlineto{\pgfqpoint{1.062285in}{1.050937in}}%
\pgfpathlineto{\pgfqpoint{1.099867in}{1.380958in}}%
\pgfpathlineto{\pgfqpoint{1.175031in}{2.070912in}}%
\pgfpathlineto{\pgfqpoint{1.325359in}{2.949414in}}%
\pgfpathlineto{\pgfqpoint{1.475687in}{3.579870in}}%
\pgfpathlineto{\pgfqpoint{1.626015in}{3.859961in}}%
\pgfpathlineto{\pgfqpoint{1.776343in}{3.866321in}}%
\pgfpathlineto{\pgfqpoint{1.926671in}{3.918906in}}%
\pgfpathlineto{\pgfqpoint{2.076999in}{3.756244in}}%
\pgfpathlineto{\pgfqpoint{2.227328in}{3.655030in}}%
\pgfpathlineto{\pgfqpoint{2.377656in}{3.473657in}}%
\pgfpathlineto{\pgfqpoint{2.527984in}{3.433064in}}%
\pgfpathlineto{\pgfqpoint{2.678312in}{2.160584in}}%
\pgfpathlineto{\pgfqpoint{2.828640in}{1.752596in}}%
\pgfpathlineto{\pgfqpoint{2.978968in}{1.592292in}}%
\pgfpathlineto{\pgfqpoint{3.129296in}{1.462600in}}%
\pgfpathlineto{\pgfqpoint{3.279624in}{1.389807in}}%
\pgfpathlineto{\pgfqpoint{3.429952in}{1.316957in}}%
\pgfpathlineto{\pgfqpoint{3.580280in}{1.303705in}}%
\pgfpathlineto{\pgfqpoint{3.730608in}{1.258492in}}%
\pgfpathlineto{\pgfqpoint{3.880937in}{1.222642in}}%
\pgfpathlineto{\pgfqpoint{4.031265in}{1.188747in}}%
\pgfpathlineto{\pgfqpoint{4.181593in}{1.155647in}}%
\pgfpathlineto{\pgfqpoint{4.331921in}{1.147149in}}%
\pgfpathlineto{\pgfqpoint{4.482249in}{1.138700in}}%
\pgfpathlineto{\pgfqpoint{4.632577in}{1.114426in}}%
\pgfpathlineto{\pgfqpoint{4.782905in}{1.097868in}}%
\pgfpathlineto{\pgfqpoint{4.933233in}{1.076427in}}%
\pgfpathlineto{\pgfqpoint{5.083561in}{1.079440in}}%
\pgfpathlineto{\pgfqpoint{5.233889in}{1.054851in}}%
\pgfpathlineto{\pgfqpoint{5.384217in}{1.042115in}}%
\pgfpathlineto{\pgfqpoint{5.534545in}{1.046990in}}%
\pgfusepath{stroke}%
\end{pgfscope}%
\begin{pgfscope}%
\pgfsetrectcap%
\pgfsetmiterjoin%
\pgfsetlinewidth{0.803000pt}%
\definecolor{currentstroke}{rgb}{0.000000,0.000000,0.000000}%
\pgfsetstrokecolor{currentstroke}%
\pgfsetdash{}{0pt}%
\pgfpathmoveto{\pgfqpoint{0.800000in}{0.528000in}}%
\pgfpathlineto{\pgfqpoint{0.800000in}{4.224000in}}%
\pgfusepath{stroke}%
\end{pgfscope}%
\begin{pgfscope}%
\pgfsetrectcap%
\pgfsetmiterjoin%
\pgfsetlinewidth{0.803000pt}%
\definecolor{currentstroke}{rgb}{0.000000,0.000000,0.000000}%
\pgfsetstrokecolor{currentstroke}%
\pgfsetdash{}{0pt}%
\pgfpathmoveto{\pgfqpoint{5.760000in}{0.528000in}}%
\pgfpathlineto{\pgfqpoint{5.760000in}{4.224000in}}%
\pgfusepath{stroke}%
\end{pgfscope}%
\begin{pgfscope}%
\pgfsetrectcap%
\pgfsetmiterjoin%
\pgfsetlinewidth{0.803000pt}%
\definecolor{currentstroke}{rgb}{0.000000,0.000000,0.000000}%
\pgfsetstrokecolor{currentstroke}%
\pgfsetdash{}{0pt}%
\pgfpathmoveto{\pgfqpoint{0.800000in}{0.528000in}}%
\pgfpathlineto{\pgfqpoint{5.760000in}{0.528000in}}%
\pgfusepath{stroke}%
\end{pgfscope}%
\begin{pgfscope}%
\pgfsetrectcap%
\pgfsetmiterjoin%
\pgfsetlinewidth{0.803000pt}%
\definecolor{currentstroke}{rgb}{0.000000,0.000000,0.000000}%
\pgfsetstrokecolor{currentstroke}%
\pgfsetdash{}{0pt}%
\pgfpathmoveto{\pgfqpoint{0.800000in}{4.224000in}}%
\pgfpathlineto{\pgfqpoint{5.760000in}{4.224000in}}%
\pgfusepath{stroke}%
\end{pgfscope}%
\end{pgfpicture}%
\makeatother%
\endgroup%
}
\caption{Benchmarking of Cap'n Proto server maximum send goodput for varying message sizes.}
\label{sizegoodput}
\end{figure}

\begin{figure}[h!]
\centering
\resizebox{.6\textwidth}{!}{%% Creator: Matplotlib, PGF backend
%%
%% To include the figure in your LaTeX document, write
%%   \input{<filename>.pgf}
%%
%% Make sure the required packages are loaded in your preamble
%%   \usepackage{pgf}
%%
%% Also ensure that all the required font packages are loaded; for instance,
%% the lmodern package is sometimes necessary when using math font.
%%   \usepackage{lmodern}
%%
%% Figures using additional raster images can only be included by \input if
%% they are in the same directory as the main LaTeX file. For loading figures
%% from other directories you can use the `import` package
%%   \usepackage{import}
%%
%% and then include the figures with
%%   \import{<path to file>}{<filename>.pgf}
%%
%% Matplotlib used the following preamble
%%   
%%   \usepackage{fontspec}
%%   \setmainfont{DejaVuSerif.ttf}[Path=\detokenize{/opt/homebrew/lib/python3.10/site-packages/matplotlib/mpl-data/fonts/ttf/}]
%%   \setsansfont{DejaVuSans.ttf}[Path=\detokenize{/opt/homebrew/lib/python3.10/site-packages/matplotlib/mpl-data/fonts/ttf/}]
%%   \setmonofont{DejaVuSansMono.ttf}[Path=\detokenize{/opt/homebrew/lib/python3.10/site-packages/matplotlib/mpl-data/fonts/ttf/}]
%%   \makeatletter\@ifpackageloaded{underscore}{}{\usepackage[strings]{underscore}}\makeatother
%%
\begingroup%
\makeatletter%
\begin{pgfpicture}%
\pgfpathrectangle{\pgfpointorigin}{\pgfqpoint{5.712287in}{4.589413in}}%
\pgfusepath{use as bounding box, clip}%
\begin{pgfscope}%
\pgfsetbuttcap%
\pgfsetmiterjoin%
\definecolor{currentfill}{rgb}{1.000000,1.000000,1.000000}%
\pgfsetfillcolor{currentfill}%
\pgfsetlinewidth{0.000000pt}%
\definecolor{currentstroke}{rgb}{1.000000,1.000000,1.000000}%
\pgfsetstrokecolor{currentstroke}%
\pgfsetdash{}{0pt}%
\pgfpathmoveto{\pgfqpoint{0.000000in}{0.000000in}}%
\pgfpathlineto{\pgfqpoint{5.712287in}{0.000000in}}%
\pgfpathlineto{\pgfqpoint{5.712287in}{4.589413in}}%
\pgfpathlineto{\pgfqpoint{0.000000in}{4.589413in}}%
\pgfpathlineto{\pgfqpoint{0.000000in}{0.000000in}}%
\pgfpathclose%
\pgfusepath{fill}%
\end{pgfscope}%
\begin{pgfscope}%
\pgfsetbuttcap%
\pgfsetmiterjoin%
\definecolor{currentfill}{rgb}{1.000000,1.000000,1.000000}%
\pgfsetfillcolor{currentfill}%
\pgfsetlinewidth{0.000000pt}%
\definecolor{currentstroke}{rgb}{0.000000,0.000000,0.000000}%
\pgfsetstrokecolor{currentstroke}%
\pgfsetstrokeopacity{0.000000}%
\pgfsetdash{}{0pt}%
\pgfpathmoveto{\pgfqpoint{0.652287in}{0.740652in}}%
\pgfpathlineto{\pgfqpoint{5.612287in}{0.740652in}}%
\pgfpathlineto{\pgfqpoint{5.612287in}{4.436652in}}%
\pgfpathlineto{\pgfqpoint{0.652287in}{4.436652in}}%
\pgfpathlineto{\pgfqpoint{0.652287in}{0.740652in}}%
\pgfpathclose%
\pgfusepath{fill}%
\end{pgfscope}%
\begin{pgfscope}%
\pgfpathrectangle{\pgfqpoint{0.652287in}{0.740652in}}{\pgfqpoint{4.960000in}{3.696000in}}%
\pgfusepath{clip}%
\pgfsetbuttcap%
\pgfsetmiterjoin%
\definecolor{currentfill}{rgb}{0.916118,0.587424,0.640264}%
\pgfsetfillcolor{currentfill}%
\pgfsetlinewidth{1.505625pt}%
\definecolor{currentstroke}{rgb}{0.270588,0.270588,0.270588}%
\pgfsetstrokecolor{currentstroke}%
\pgfsetdash{}{0pt}%
\pgfpathmoveto{\pgfqpoint{0.668820in}{0.883059in}}%
\pgfpathlineto{\pgfqpoint{0.801087in}{0.883059in}}%
\pgfpathlineto{\pgfqpoint{0.801087in}{1.001737in}}%
\pgfpathlineto{\pgfqpoint{0.668820in}{1.001737in}}%
\pgfpathlineto{\pgfqpoint{0.668820in}{0.883059in}}%
\pgfpathlineto{\pgfqpoint{0.668820in}{0.883059in}}%
\pgfpathclose%
\pgfusepath{stroke,fill}%
\end{pgfscope}%
\begin{pgfscope}%
\pgfpathrectangle{\pgfqpoint{0.652287in}{0.740652in}}{\pgfqpoint{4.960000in}{3.696000in}}%
\pgfusepath{clip}%
\pgfsetbuttcap%
\pgfsetmiterjoin%
\definecolor{currentfill}{rgb}{0.914239,0.591537,0.567368}%
\pgfsetfillcolor{currentfill}%
\pgfsetlinewidth{1.505625pt}%
\definecolor{currentstroke}{rgb}{0.270588,0.270588,0.270588}%
\pgfsetstrokecolor{currentstroke}%
\pgfsetdash{}{0pt}%
\pgfpathmoveto{\pgfqpoint{0.834153in}{0.843068in}}%
\pgfpathlineto{\pgfqpoint{0.966420in}{0.843068in}}%
\pgfpathlineto{\pgfqpoint{0.966420in}{0.924269in}}%
\pgfpathlineto{\pgfqpoint{0.834153in}{0.924269in}}%
\pgfpathlineto{\pgfqpoint{0.834153in}{0.843068in}}%
\pgfpathlineto{\pgfqpoint{0.834153in}{0.843068in}}%
\pgfpathclose%
\pgfusepath{stroke,fill}%
\end{pgfscope}%
\begin{pgfscope}%
\pgfpathrectangle{\pgfqpoint{0.652287in}{0.740652in}}{\pgfqpoint{4.960000in}{3.696000in}}%
\pgfusepath{clip}%
\pgfsetbuttcap%
\pgfsetmiterjoin%
\definecolor{currentfill}{rgb}{0.897377,0.581331,0.441485}%
\pgfsetfillcolor{currentfill}%
\pgfsetlinewidth{1.505625pt}%
\definecolor{currentstroke}{rgb}{0.270588,0.270588,0.270588}%
\pgfsetstrokecolor{currentstroke}%
\pgfsetdash{}{0pt}%
\pgfpathmoveto{\pgfqpoint{0.999487in}{0.829689in}}%
\pgfpathlineto{\pgfqpoint{1.131753in}{0.829689in}}%
\pgfpathlineto{\pgfqpoint{1.131753in}{0.901650in}}%
\pgfpathlineto{\pgfqpoint{0.999487in}{0.901650in}}%
\pgfpathlineto{\pgfqpoint{0.999487in}{0.829689in}}%
\pgfpathlineto{\pgfqpoint{0.999487in}{0.829689in}}%
\pgfpathclose%
\pgfusepath{stroke,fill}%
\end{pgfscope}%
\begin{pgfscope}%
\pgfpathrectangle{\pgfqpoint{0.652287in}{0.740652in}}{\pgfqpoint{4.960000in}{3.696000in}}%
\pgfusepath{clip}%
\pgfsetbuttcap%
\pgfsetmiterjoin%
\definecolor{currentfill}{rgb}{0.843004,0.580250,0.304650}%
\pgfsetfillcolor{currentfill}%
\pgfsetlinewidth{1.505625pt}%
\definecolor{currentstroke}{rgb}{0.270588,0.270588,0.270588}%
\pgfsetstrokecolor{currentstroke}%
\pgfsetdash{}{0pt}%
\pgfpathmoveto{\pgfqpoint{1.164820in}{0.822851in}}%
\pgfpathlineto{\pgfqpoint{1.297087in}{0.822851in}}%
\pgfpathlineto{\pgfqpoint{1.297087in}{0.896475in}}%
\pgfpathlineto{\pgfqpoint{1.164820in}{0.896475in}}%
\pgfpathlineto{\pgfqpoint{1.164820in}{0.822851in}}%
\pgfpathlineto{\pgfqpoint{1.164820in}{0.822851in}}%
\pgfpathclose%
\pgfusepath{stroke,fill}%
\end{pgfscope}%
\begin{pgfscope}%
\pgfpathrectangle{\pgfqpoint{0.652287in}{0.740652in}}{\pgfqpoint{4.960000in}{3.696000in}}%
\pgfusepath{clip}%
\pgfsetbuttcap%
\pgfsetmiterjoin%
\definecolor{currentfill}{rgb}{0.778237,0.598661,0.294850}%
\pgfsetfillcolor{currentfill}%
\pgfsetlinewidth{1.505625pt}%
\definecolor{currentstroke}{rgb}{0.270588,0.270588,0.270588}%
\pgfsetstrokecolor{currentstroke}%
\pgfsetdash{}{0pt}%
\pgfpathmoveto{\pgfqpoint{1.330153in}{0.825771in}}%
\pgfpathlineto{\pgfqpoint{1.462420in}{0.825771in}}%
\pgfpathlineto{\pgfqpoint{1.462420in}{0.910853in}}%
\pgfpathlineto{\pgfqpoint{1.330153in}{0.910853in}}%
\pgfpathlineto{\pgfqpoint{1.330153in}{0.825771in}}%
\pgfpathlineto{\pgfqpoint{1.330153in}{0.825771in}}%
\pgfpathclose%
\pgfusepath{stroke,fill}%
\end{pgfscope}%
\begin{pgfscope}%
\pgfpathrectangle{\pgfqpoint{0.652287in}{0.740652in}}{\pgfqpoint{4.960000in}{3.696000in}}%
\pgfusepath{clip}%
\pgfsetbuttcap%
\pgfsetmiterjoin%
\definecolor{currentfill}{rgb}{0.723506,0.610560,0.286612}%
\pgfsetfillcolor{currentfill}%
\pgfsetlinewidth{1.505625pt}%
\definecolor{currentstroke}{rgb}{0.270588,0.270588,0.270588}%
\pgfsetstrokecolor{currentstroke}%
\pgfsetdash{}{0pt}%
\pgfpathmoveto{\pgfqpoint{1.495487in}{0.821964in}}%
\pgfpathlineto{\pgfqpoint{1.627753in}{0.821964in}}%
\pgfpathlineto{\pgfqpoint{1.627753in}{0.902204in}}%
\pgfpathlineto{\pgfqpoint{1.495487in}{0.902204in}}%
\pgfpathlineto{\pgfqpoint{1.495487in}{0.821964in}}%
\pgfpathlineto{\pgfqpoint{1.495487in}{0.821964in}}%
\pgfpathclose%
\pgfusepath{stroke,fill}%
\end{pgfscope}%
\begin{pgfscope}%
\pgfpathrectangle{\pgfqpoint{0.652287in}{0.740652in}}{\pgfqpoint{4.960000in}{3.696000in}}%
\pgfusepath{clip}%
\pgfsetbuttcap%
\pgfsetmiterjoin%
\definecolor{currentfill}{rgb}{0.671968,0.619218,0.278892}%
\pgfsetfillcolor{currentfill}%
\pgfsetlinewidth{1.505625pt}%
\definecolor{currentstroke}{rgb}{0.270588,0.270588,0.270588}%
\pgfsetstrokecolor{currentstroke}%
\pgfsetdash{}{0pt}%
\pgfpathmoveto{\pgfqpoint{1.660820in}{0.824403in}}%
\pgfpathlineto{\pgfqpoint{1.793087in}{0.824403in}}%
\pgfpathlineto{\pgfqpoint{1.793087in}{0.934138in}}%
\pgfpathlineto{\pgfqpoint{1.660820in}{0.934138in}}%
\pgfpathlineto{\pgfqpoint{1.660820in}{0.824403in}}%
\pgfpathlineto{\pgfqpoint{1.660820in}{0.824403in}}%
\pgfpathclose%
\pgfusepath{stroke,fill}%
\end{pgfscope}%
\begin{pgfscope}%
\pgfpathrectangle{\pgfqpoint{0.652287in}{0.740652in}}{\pgfqpoint{4.960000in}{3.696000in}}%
\pgfusepath{clip}%
\pgfsetbuttcap%
\pgfsetmiterjoin%
\definecolor{currentfill}{rgb}{0.619524,0.627270,0.272170}%
\pgfsetfillcolor{currentfill}%
\pgfsetlinewidth{1.505625pt}%
\definecolor{currentstroke}{rgb}{0.270588,0.270588,0.270588}%
\pgfsetstrokecolor{currentstroke}%
\pgfsetdash{}{0pt}%
\pgfpathmoveto{\pgfqpoint{1.826153in}{0.830465in}}%
\pgfpathlineto{\pgfqpoint{1.958420in}{0.830465in}}%
\pgfpathlineto{\pgfqpoint{1.958420in}{0.965997in}}%
\pgfpathlineto{\pgfqpoint{1.826153in}{0.965997in}}%
\pgfpathlineto{\pgfqpoint{1.826153in}{0.830465in}}%
\pgfpathlineto{\pgfqpoint{1.826153in}{0.830465in}}%
\pgfpathclose%
\pgfusepath{stroke,fill}%
\end{pgfscope}%
\begin{pgfscope}%
\pgfpathrectangle{\pgfqpoint{0.652287in}{0.740652in}}{\pgfqpoint{4.960000in}{3.696000in}}%
\pgfusepath{clip}%
\pgfsetbuttcap%
\pgfsetmiterjoin%
\definecolor{currentfill}{rgb}{0.568275,0.643917,0.274203}%
\pgfsetfillcolor{currentfill}%
\pgfsetlinewidth{1.505625pt}%
\definecolor{currentstroke}{rgb}{0.270588,0.270588,0.270588}%
\pgfsetstrokecolor{currentstroke}%
\pgfsetdash{}{0pt}%
\pgfpathmoveto{\pgfqpoint{1.991487in}{0.823442in}}%
\pgfpathlineto{\pgfqpoint{2.123753in}{0.823442in}}%
\pgfpathlineto{\pgfqpoint{2.123753in}{0.936023in}}%
\pgfpathlineto{\pgfqpoint{1.991487in}{0.936023in}}%
\pgfpathlineto{\pgfqpoint{1.991487in}{0.823442in}}%
\pgfpathlineto{\pgfqpoint{1.991487in}{0.823442in}}%
\pgfpathclose%
\pgfusepath{stroke,fill}%
\end{pgfscope}%
\begin{pgfscope}%
\pgfpathrectangle{\pgfqpoint{0.652287in}{0.740652in}}{\pgfqpoint{4.960000in}{3.696000in}}%
\pgfusepath{clip}%
\pgfsetbuttcap%
\pgfsetmiterjoin%
\definecolor{currentfill}{rgb}{0.498809,0.661940,0.276389}%
\pgfsetfillcolor{currentfill}%
\pgfsetlinewidth{1.505625pt}%
\definecolor{currentstroke}{rgb}{0.270588,0.270588,0.270588}%
\pgfsetstrokecolor{currentstroke}%
\pgfsetdash{}{0pt}%
\pgfpathmoveto{\pgfqpoint{2.156820in}{0.827508in}}%
\pgfpathlineto{\pgfqpoint{2.289087in}{0.827508in}}%
\pgfpathlineto{\pgfqpoint{2.289087in}{0.957164in}}%
\pgfpathlineto{\pgfqpoint{2.156820in}{0.957164in}}%
\pgfpathlineto{\pgfqpoint{2.156820in}{0.827508in}}%
\pgfpathlineto{\pgfqpoint{2.156820in}{0.827508in}}%
\pgfpathclose%
\pgfusepath{stroke,fill}%
\end{pgfscope}%
\begin{pgfscope}%
\pgfpathrectangle{\pgfqpoint{0.652287in}{0.740652in}}{\pgfqpoint{4.960000in}{3.696000in}}%
\pgfusepath{clip}%
\pgfsetbuttcap%
\pgfsetmiterjoin%
\definecolor{currentfill}{rgb}{0.376222,0.683486,0.278982}%
\pgfsetfillcolor{currentfill}%
\pgfsetlinewidth{1.505625pt}%
\definecolor{currentstroke}{rgb}{0.270588,0.270588,0.270588}%
\pgfsetstrokecolor{currentstroke}%
\pgfsetdash{}{0pt}%
\pgfpathmoveto{\pgfqpoint{2.322153in}{0.891966in}}%
\pgfpathlineto{\pgfqpoint{2.454420in}{0.891966in}}%
\pgfpathlineto{\pgfqpoint{2.454420in}{1.175126in}}%
\pgfpathlineto{\pgfqpoint{2.322153in}{1.175126in}}%
\pgfpathlineto{\pgfqpoint{2.322153in}{0.891966in}}%
\pgfpathlineto{\pgfqpoint{2.322153in}{0.891966in}}%
\pgfpathclose%
\pgfusepath{stroke,fill}%
\end{pgfscope}%
\begin{pgfscope}%
\pgfpathrectangle{\pgfqpoint{0.652287in}{0.740652in}}{\pgfqpoint{4.960000in}{3.696000in}}%
\pgfusepath{clip}%
\pgfsetbuttcap%
\pgfsetmiterjoin%
\definecolor{currentfill}{rgb}{0.283451,0.688333,0.416810}%
\pgfsetfillcolor{currentfill}%
\pgfsetlinewidth{1.505625pt}%
\definecolor{currentstroke}{rgb}{0.270588,0.270588,0.270588}%
\pgfsetstrokecolor{currentstroke}%
\pgfsetdash{}{0pt}%
\pgfpathmoveto{\pgfqpoint{2.487487in}{0.923604in}}%
\pgfpathlineto{\pgfqpoint{2.619753in}{0.923604in}}%
\pgfpathlineto{\pgfqpoint{2.619753in}{1.363742in}}%
\pgfpathlineto{\pgfqpoint{2.487487in}{1.363742in}}%
\pgfpathlineto{\pgfqpoint{2.487487in}{0.923604in}}%
\pgfpathlineto{\pgfqpoint{2.487487in}{0.923604in}}%
\pgfpathclose%
\pgfusepath{stroke,fill}%
\end{pgfscope}%
\begin{pgfscope}%
\pgfpathrectangle{\pgfqpoint{0.652287in}{0.740652in}}{\pgfqpoint{4.960000in}{3.696000in}}%
\pgfusepath{clip}%
\pgfsetbuttcap%
\pgfsetmiterjoin%
\definecolor{currentfill}{rgb}{0.287076,0.682718,0.512089}%
\pgfsetfillcolor{currentfill}%
\pgfsetlinewidth{1.505625pt}%
\definecolor{currentstroke}{rgb}{0.270588,0.270588,0.270588}%
\pgfsetstrokecolor{currentstroke}%
\pgfsetdash{}{0pt}%
\pgfpathmoveto{\pgfqpoint{2.652820in}{0.883170in}}%
\pgfpathlineto{\pgfqpoint{2.785087in}{0.883170in}}%
\pgfpathlineto{\pgfqpoint{2.785087in}{1.172955in}}%
\pgfpathlineto{\pgfqpoint{2.652820in}{1.172955in}}%
\pgfpathlineto{\pgfqpoint{2.652820in}{0.883170in}}%
\pgfpathlineto{\pgfqpoint{2.652820in}{0.883170in}}%
\pgfpathclose%
\pgfusepath{stroke,fill}%
\end{pgfscope}%
\begin{pgfscope}%
\pgfpathrectangle{\pgfqpoint{0.652287in}{0.740652in}}{\pgfqpoint{4.960000in}{3.696000in}}%
\pgfusepath{clip}%
\pgfsetbuttcap%
\pgfsetmiterjoin%
\definecolor{currentfill}{rgb}{0.289731,0.678413,0.568041}%
\pgfsetfillcolor{currentfill}%
\pgfsetlinewidth{1.505625pt}%
\definecolor{currentstroke}{rgb}{0.270588,0.270588,0.270588}%
\pgfsetstrokecolor{currentstroke}%
\pgfsetdash{}{0pt}%
\pgfpathmoveto{\pgfqpoint{2.818153in}{0.940910in}}%
\pgfpathlineto{\pgfqpoint{2.950420in}{0.940910in}}%
\pgfpathlineto{\pgfqpoint{2.950420in}{1.376068in}}%
\pgfpathlineto{\pgfqpoint{2.818153in}{1.376068in}}%
\pgfpathlineto{\pgfqpoint{2.818153in}{0.940910in}}%
\pgfpathlineto{\pgfqpoint{2.818153in}{0.940910in}}%
\pgfpathclose%
\pgfusepath{stroke,fill}%
\end{pgfscope}%
\begin{pgfscope}%
\pgfpathrectangle{\pgfqpoint{0.652287in}{0.740652in}}{\pgfqpoint{4.960000in}{3.696000in}}%
\pgfusepath{clip}%
\pgfsetbuttcap%
\pgfsetmiterjoin%
\definecolor{currentfill}{rgb}{0.291926,0.674723,0.609182}%
\pgfsetfillcolor{currentfill}%
\pgfsetlinewidth{1.505625pt}%
\definecolor{currentstroke}{rgb}{0.270588,0.270588,0.270588}%
\pgfsetstrokecolor{currentstroke}%
\pgfsetdash{}{0pt}%
\pgfpathmoveto{\pgfqpoint{2.983487in}{0.943562in}}%
\pgfpathlineto{\pgfqpoint{3.115753in}{0.943562in}}%
\pgfpathlineto{\pgfqpoint{3.115753in}{1.358512in}}%
\pgfpathlineto{\pgfqpoint{2.983487in}{1.358512in}}%
\pgfpathlineto{\pgfqpoint{2.983487in}{0.943562in}}%
\pgfpathlineto{\pgfqpoint{2.983487in}{0.943562in}}%
\pgfpathclose%
\pgfusepath{stroke,fill}%
\end{pgfscope}%
\begin{pgfscope}%
\pgfpathrectangle{\pgfqpoint{0.652287in}{0.740652in}}{\pgfqpoint{4.960000in}{3.696000in}}%
\pgfusepath{clip}%
\pgfsetbuttcap%
\pgfsetmiterjoin%
\definecolor{currentfill}{rgb}{0.293930,0.671245,0.643834}%
\pgfsetfillcolor{currentfill}%
\pgfsetlinewidth{1.505625pt}%
\definecolor{currentstroke}{rgb}{0.270588,0.270588,0.270588}%
\pgfsetstrokecolor{currentstroke}%
\pgfsetdash{}{0pt}%
\pgfpathmoveto{\pgfqpoint{3.148820in}{0.968991in}}%
\pgfpathlineto{\pgfqpoint{3.281087in}{0.968991in}}%
\pgfpathlineto{\pgfqpoint{3.281087in}{1.433818in}}%
\pgfpathlineto{\pgfqpoint{3.148820in}{1.433818in}}%
\pgfpathlineto{\pgfqpoint{3.148820in}{0.968991in}}%
\pgfpathlineto{\pgfqpoint{3.148820in}{0.968991in}}%
\pgfpathclose%
\pgfusepath{stroke,fill}%
\end{pgfscope}%
\begin{pgfscope}%
\pgfpathrectangle{\pgfqpoint{0.652287in}{0.740652in}}{\pgfqpoint{4.960000in}{3.696000in}}%
\pgfusepath{clip}%
\pgfsetbuttcap%
\pgfsetmiterjoin%
\definecolor{currentfill}{rgb}{0.297370,0.669103,0.677736}%
\pgfsetfillcolor{currentfill}%
\pgfsetlinewidth{1.505625pt}%
\definecolor{currentstroke}{rgb}{0.270588,0.270588,0.270588}%
\pgfsetstrokecolor{currentstroke}%
\pgfsetdash{}{0pt}%
\pgfpathmoveto{\pgfqpoint{3.314153in}{1.022019in}}%
\pgfpathlineto{\pgfqpoint{3.446420in}{1.022019in}}%
\pgfpathlineto{\pgfqpoint{3.446420in}{1.579071in}}%
\pgfpathlineto{\pgfqpoint{3.314153in}{1.579071in}}%
\pgfpathlineto{\pgfqpoint{3.314153in}{1.022019in}}%
\pgfpathlineto{\pgfqpoint{3.314153in}{1.022019in}}%
\pgfpathclose%
\pgfusepath{stroke,fill}%
\end{pgfscope}%
\begin{pgfscope}%
\pgfpathrectangle{\pgfqpoint{0.652287in}{0.740652in}}{\pgfqpoint{4.960000in}{3.696000in}}%
\pgfusepath{clip}%
\pgfsetbuttcap%
\pgfsetmiterjoin%
\definecolor{currentfill}{rgb}{0.305815,0.671318,0.717501}%
\pgfsetfillcolor{currentfill}%
\pgfsetlinewidth{1.505625pt}%
\definecolor{currentstroke}{rgb}{0.270588,0.270588,0.270588}%
\pgfsetstrokecolor{currentstroke}%
\pgfsetdash{}{0pt}%
\pgfpathmoveto{\pgfqpoint{3.479487in}{1.054331in}}%
\pgfpathlineto{\pgfqpoint{3.611753in}{1.054331in}}%
\pgfpathlineto{\pgfqpoint{3.611753in}{1.686070in}}%
\pgfpathlineto{\pgfqpoint{3.479487in}{1.686070in}}%
\pgfpathlineto{\pgfqpoint{3.479487in}{1.054331in}}%
\pgfpathlineto{\pgfqpoint{3.479487in}{1.054331in}}%
\pgfpathclose%
\pgfusepath{stroke,fill}%
\end{pgfscope}%
\begin{pgfscope}%
\pgfpathrectangle{\pgfqpoint{0.652287in}{0.740652in}}{\pgfqpoint{4.960000in}{3.696000in}}%
\pgfusepath{clip}%
\pgfsetbuttcap%
\pgfsetmiterjoin%
\definecolor{currentfill}{rgb}{0.315665,0.673465,0.762927}%
\pgfsetfillcolor{currentfill}%
\pgfsetlinewidth{1.505625pt}%
\definecolor{currentstroke}{rgb}{0.270588,0.270588,0.270588}%
\pgfsetstrokecolor{currentstroke}%
\pgfsetdash{}{0pt}%
\pgfpathmoveto{\pgfqpoint{3.644820in}{1.092437in}}%
\pgfpathlineto{\pgfqpoint{3.777087in}{1.092437in}}%
\pgfpathlineto{\pgfqpoint{3.777087in}{1.760434in}}%
\pgfpathlineto{\pgfqpoint{3.644820in}{1.760434in}}%
\pgfpathlineto{\pgfqpoint{3.644820in}{1.092437in}}%
\pgfpathlineto{\pgfqpoint{3.644820in}{1.092437in}}%
\pgfpathclose%
\pgfusepath{stroke,fill}%
\end{pgfscope}%
\begin{pgfscope}%
\pgfpathrectangle{\pgfqpoint{0.652287in}{0.740652in}}{\pgfqpoint{4.960000in}{3.696000in}}%
\pgfusepath{clip}%
\pgfsetbuttcap%
\pgfsetmiterjoin%
\definecolor{currentfill}{rgb}{0.328616,0.675585,0.821235}%
\pgfsetfillcolor{currentfill}%
\pgfsetlinewidth{1.505625pt}%
\definecolor{currentstroke}{rgb}{0.270588,0.270588,0.270588}%
\pgfsetstrokecolor{currentstroke}%
\pgfsetdash{}{0pt}%
\pgfpathmoveto{\pgfqpoint{3.810153in}{1.137048in}}%
\pgfpathlineto{\pgfqpoint{3.942420in}{1.137048in}}%
\pgfpathlineto{\pgfqpoint{3.942420in}{1.873245in}}%
\pgfpathlineto{\pgfqpoint{3.810153in}{1.873245in}}%
\pgfpathlineto{\pgfqpoint{3.810153in}{1.137048in}}%
\pgfpathlineto{\pgfqpoint{3.810153in}{1.137048in}}%
\pgfpathclose%
\pgfusepath{stroke,fill}%
\end{pgfscope}%
\begin{pgfscope}%
\pgfpathrectangle{\pgfqpoint{0.652287in}{0.740652in}}{\pgfqpoint{4.960000in}{3.696000in}}%
\pgfusepath{clip}%
\pgfsetbuttcap%
\pgfsetmiterjoin%
\definecolor{currentfill}{rgb}{0.426614,0.681401,0.887856}%
\pgfsetfillcolor{currentfill}%
\pgfsetlinewidth{1.505625pt}%
\definecolor{currentstroke}{rgb}{0.270588,0.270588,0.270588}%
\pgfsetstrokecolor{currentstroke}%
\pgfsetdash{}{0pt}%
\pgfpathmoveto{\pgfqpoint{3.975487in}{1.148783in}}%
\pgfpathlineto{\pgfqpoint{4.107753in}{1.148783in}}%
\pgfpathlineto{\pgfqpoint{4.107753in}{1.997902in}}%
\pgfpathlineto{\pgfqpoint{3.975487in}{1.997902in}}%
\pgfpathlineto{\pgfqpoint{3.975487in}{1.148783in}}%
\pgfpathlineto{\pgfqpoint{3.975487in}{1.148783in}}%
\pgfpathclose%
\pgfusepath{stroke,fill}%
\end{pgfscope}%
\begin{pgfscope}%
\pgfpathrectangle{\pgfqpoint{0.652287in}{0.740652in}}{\pgfqpoint{4.960000in}{3.696000in}}%
\pgfusepath{clip}%
\pgfsetbuttcap%
\pgfsetmiterjoin%
\definecolor{currentfill}{rgb}{0.580505,0.680677,0.909722}%
\pgfsetfillcolor{currentfill}%
\pgfsetlinewidth{1.505625pt}%
\definecolor{currentstroke}{rgb}{0.270588,0.270588,0.270588}%
\pgfsetstrokecolor{currentstroke}%
\pgfsetdash{}{0pt}%
\pgfpathmoveto{\pgfqpoint{4.140820in}{1.216734in}}%
\pgfpathlineto{\pgfqpoint{4.273087in}{1.216734in}}%
\pgfpathlineto{\pgfqpoint{4.273087in}{2.097306in}}%
\pgfpathlineto{\pgfqpoint{4.140820in}{2.097306in}}%
\pgfpathlineto{\pgfqpoint{4.140820in}{1.216734in}}%
\pgfpathlineto{\pgfqpoint{4.140820in}{1.216734in}}%
\pgfpathclose%
\pgfusepath{stroke,fill}%
\end{pgfscope}%
\begin{pgfscope}%
\pgfpathrectangle{\pgfqpoint{0.652287in}{0.740652in}}{\pgfqpoint{4.960000in}{3.696000in}}%
\pgfusepath{clip}%
\pgfsetbuttcap%
\pgfsetmiterjoin%
\definecolor{currentfill}{rgb}{0.679308,0.671172,0.922565}%
\pgfsetfillcolor{currentfill}%
\pgfsetlinewidth{1.505625pt}%
\definecolor{currentstroke}{rgb}{0.270588,0.270588,0.270588}%
\pgfsetstrokecolor{currentstroke}%
\pgfsetdash{}{0pt}%
\pgfpathmoveto{\pgfqpoint{4.306153in}{1.255625in}}%
\pgfpathlineto{\pgfqpoint{4.438420in}{1.255625in}}%
\pgfpathlineto{\pgfqpoint{4.438420in}{2.243593in}}%
\pgfpathlineto{\pgfqpoint{4.306153in}{2.243593in}}%
\pgfpathlineto{\pgfqpoint{4.306153in}{1.255625in}}%
\pgfpathlineto{\pgfqpoint{4.306153in}{1.255625in}}%
\pgfpathclose%
\pgfusepath{stroke,fill}%
\end{pgfscope}%
\begin{pgfscope}%
\pgfpathrectangle{\pgfqpoint{0.652287in}{0.740652in}}{\pgfqpoint{4.960000in}{3.696000in}}%
\pgfusepath{clip}%
\pgfsetbuttcap%
\pgfsetmiterjoin%
\definecolor{currentfill}{rgb}{0.746036,0.642703,0.918385}%
\pgfsetfillcolor{currentfill}%
\pgfsetlinewidth{1.505625pt}%
\definecolor{currentstroke}{rgb}{0.270588,0.270588,0.270588}%
\pgfsetstrokecolor{currentstroke}%
\pgfsetdash{}{0pt}%
\pgfpathmoveto{\pgfqpoint{4.471487in}{1.289305in}}%
\pgfpathlineto{\pgfqpoint{4.603753in}{1.289305in}}%
\pgfpathlineto{\pgfqpoint{4.603753in}{2.346259in}}%
\pgfpathlineto{\pgfqpoint{4.471487in}{2.346259in}}%
\pgfpathlineto{\pgfqpoint{4.471487in}{1.289305in}}%
\pgfpathlineto{\pgfqpoint{4.471487in}{1.289305in}}%
\pgfpathclose%
\pgfusepath{stroke,fill}%
\end{pgfscope}%
\begin{pgfscope}%
\pgfpathrectangle{\pgfqpoint{0.652287in}{0.740652in}}{\pgfqpoint{4.960000in}{3.696000in}}%
\pgfusepath{clip}%
\pgfsetbuttcap%
\pgfsetmiterjoin%
\definecolor{currentfill}{rgb}{0.808252,0.608370,0.913355}%
\pgfsetfillcolor{currentfill}%
\pgfsetlinewidth{1.505625pt}%
\definecolor{currentstroke}{rgb}{0.270588,0.270588,0.270588}%
\pgfsetstrokecolor{currentstroke}%
\pgfsetdash{}{0pt}%
\pgfpathmoveto{\pgfqpoint{4.636820in}{1.347517in}}%
\pgfpathlineto{\pgfqpoint{4.769087in}{1.347517in}}%
\pgfpathlineto{\pgfqpoint{4.769087in}{2.516321in}}%
\pgfpathlineto{\pgfqpoint{4.636820in}{2.516321in}}%
\pgfpathlineto{\pgfqpoint{4.636820in}{1.347517in}}%
\pgfpathlineto{\pgfqpoint{4.636820in}{1.347517in}}%
\pgfpathclose%
\pgfusepath{stroke,fill}%
\end{pgfscope}%
\begin{pgfscope}%
\pgfpathrectangle{\pgfqpoint{0.652287in}{0.740652in}}{\pgfqpoint{4.960000in}{3.696000in}}%
\pgfusepath{clip}%
\pgfsetbuttcap%
\pgfsetmiterjoin%
\definecolor{currentfill}{rgb}{0.873134,0.561119,0.906450}%
\pgfsetfillcolor{currentfill}%
\pgfsetlinewidth{1.505625pt}%
\definecolor{currentstroke}{rgb}{0.270588,0.270588,0.270588}%
\pgfsetstrokecolor{currentstroke}%
\pgfsetdash{}{0pt}%
\pgfpathmoveto{\pgfqpoint{4.802153in}{1.393865in}}%
\pgfpathlineto{\pgfqpoint{4.934420in}{1.393865in}}%
\pgfpathlineto{\pgfqpoint{4.934420in}{2.655901in}}%
\pgfpathlineto{\pgfqpoint{4.802153in}{2.655901in}}%
\pgfpathlineto{\pgfqpoint{4.802153in}{1.393865in}}%
\pgfpathlineto{\pgfqpoint{4.802153in}{1.393865in}}%
\pgfpathclose%
\pgfusepath{stroke,fill}%
\end{pgfscope}%
\begin{pgfscope}%
\pgfpathrectangle{\pgfqpoint{0.652287in}{0.740652in}}{\pgfqpoint{4.960000in}{3.696000in}}%
\pgfusepath{clip}%
\pgfsetbuttcap%
\pgfsetmiterjoin%
\definecolor{currentfill}{rgb}{0.905393,0.545626,0.860563}%
\pgfsetfillcolor{currentfill}%
\pgfsetlinewidth{1.505625pt}%
\definecolor{currentstroke}{rgb}{0.270588,0.270588,0.270588}%
\pgfsetstrokecolor{currentstroke}%
\pgfsetdash{}{0pt}%
\pgfpathmoveto{\pgfqpoint{4.967487in}{1.408297in}}%
\pgfpathlineto{\pgfqpoint{5.099753in}{1.408297in}}%
\pgfpathlineto{\pgfqpoint{5.099753in}{2.752551in}}%
\pgfpathlineto{\pgfqpoint{4.967487in}{2.752551in}}%
\pgfpathlineto{\pgfqpoint{4.967487in}{1.408297in}}%
\pgfpathlineto{\pgfqpoint{4.967487in}{1.408297in}}%
\pgfpathclose%
\pgfusepath{stroke,fill}%
\end{pgfscope}%
\begin{pgfscope}%
\pgfpathrectangle{\pgfqpoint{0.652287in}{0.740652in}}{\pgfqpoint{4.960000in}{3.696000in}}%
\pgfusepath{clip}%
\pgfsetbuttcap%
\pgfsetmiterjoin%
\definecolor{currentfill}{rgb}{0.908931,0.559636,0.801409}%
\pgfsetfillcolor{currentfill}%
\pgfsetlinewidth{1.505625pt}%
\definecolor{currentstroke}{rgb}{0.270588,0.270588,0.270588}%
\pgfsetstrokecolor{currentstroke}%
\pgfsetdash{}{0pt}%
\pgfpathmoveto{\pgfqpoint{5.132820in}{1.499108in}}%
\pgfpathlineto{\pgfqpoint{5.265087in}{1.499108in}}%
\pgfpathlineto{\pgfqpoint{5.265087in}{2.913124in}}%
\pgfpathlineto{\pgfqpoint{5.132820in}{2.913124in}}%
\pgfpathlineto{\pgfqpoint{5.132820in}{1.499108in}}%
\pgfpathlineto{\pgfqpoint{5.132820in}{1.499108in}}%
\pgfpathclose%
\pgfusepath{stroke,fill}%
\end{pgfscope}%
\begin{pgfscope}%
\pgfpathrectangle{\pgfqpoint{0.652287in}{0.740652in}}{\pgfqpoint{4.960000in}{3.696000in}}%
\pgfusepath{clip}%
\pgfsetbuttcap%
\pgfsetmiterjoin%
\definecolor{currentfill}{rgb}{0.911595,0.570037,0.749836}%
\pgfsetfillcolor{currentfill}%
\pgfsetlinewidth{1.505625pt}%
\definecolor{currentstroke}{rgb}{0.270588,0.270588,0.270588}%
\pgfsetstrokecolor{currentstroke}%
\pgfsetdash{}{0pt}%
\pgfpathmoveto{\pgfqpoint{5.298153in}{1.569794in}}%
\pgfpathlineto{\pgfqpoint{5.430420in}{1.569794in}}%
\pgfpathlineto{\pgfqpoint{5.430420in}{3.052408in}}%
\pgfpathlineto{\pgfqpoint{5.298153in}{3.052408in}}%
\pgfpathlineto{\pgfqpoint{5.298153in}{1.569794in}}%
\pgfpathlineto{\pgfqpoint{5.298153in}{1.569794in}}%
\pgfpathclose%
\pgfusepath{stroke,fill}%
\end{pgfscope}%
\begin{pgfscope}%
\pgfpathrectangle{\pgfqpoint{0.652287in}{0.740652in}}{\pgfqpoint{4.960000in}{3.696000in}}%
\pgfusepath{clip}%
\pgfsetbuttcap%
\pgfsetmiterjoin%
\definecolor{currentfill}{rgb}{0.913847,0.578735,0.699662}%
\pgfsetfillcolor{currentfill}%
\pgfsetlinewidth{1.505625pt}%
\definecolor{currentstroke}{rgb}{0.270588,0.270588,0.270588}%
\pgfsetstrokecolor{currentstroke}%
\pgfsetdash{}{0pt}%
\pgfpathmoveto{\pgfqpoint{5.463487in}{1.585742in}}%
\pgfpathlineto{\pgfqpoint{5.595753in}{1.585742in}}%
\pgfpathlineto{\pgfqpoint{5.595753in}{3.190712in}}%
\pgfpathlineto{\pgfqpoint{5.463487in}{3.190712in}}%
\pgfpathlineto{\pgfqpoint{5.463487in}{1.585742in}}%
\pgfpathlineto{\pgfqpoint{5.463487in}{1.585742in}}%
\pgfpathclose%
\pgfusepath{stroke,fill}%
\end{pgfscope}%
\begin{pgfscope}%
\pgfsetbuttcap%
\pgfsetroundjoin%
\definecolor{currentfill}{rgb}{0.000000,0.000000,0.000000}%
\pgfsetfillcolor{currentfill}%
\pgfsetlinewidth{0.803000pt}%
\definecolor{currentstroke}{rgb}{0.000000,0.000000,0.000000}%
\pgfsetstrokecolor{currentstroke}%
\pgfsetdash{}{0pt}%
\pgfsys@defobject{currentmarker}{\pgfqpoint{0.000000in}{-0.048611in}}{\pgfqpoint{0.000000in}{0.000000in}}{%
\pgfpathmoveto{\pgfqpoint{0.000000in}{0.000000in}}%
\pgfpathlineto{\pgfqpoint{0.000000in}{-0.048611in}}%
\pgfusepath{stroke,fill}%
}%
\begin{pgfscope}%
\pgfsys@transformshift{0.734953in}{0.740652in}%
\pgfsys@useobject{currentmarker}{}%
\end{pgfscope}%
\end{pgfscope}%
\begin{pgfscope}%
\definecolor{textcolor}{rgb}{0.000000,0.000000,0.000000}%
\pgfsetstrokecolor{textcolor}%
\pgfsetfillcolor{textcolor}%
\pgftext[x=0.773270in, y=0.378334in, left, base,rotate=90.000000]{\color{textcolor}\sffamily\fontsize{10.000000}{12.000000}\selectfont 200}%
\end{pgfscope}%
\begin{pgfscope}%
\pgfsetbuttcap%
\pgfsetroundjoin%
\definecolor{currentfill}{rgb}{0.000000,0.000000,0.000000}%
\pgfsetfillcolor{currentfill}%
\pgfsetlinewidth{0.803000pt}%
\definecolor{currentstroke}{rgb}{0.000000,0.000000,0.000000}%
\pgfsetstrokecolor{currentstroke}%
\pgfsetdash{}{0pt}%
\pgfsys@defobject{currentmarker}{\pgfqpoint{0.000000in}{-0.048611in}}{\pgfqpoint{0.000000in}{0.000000in}}{%
\pgfpathmoveto{\pgfqpoint{0.000000in}{0.000000in}}%
\pgfpathlineto{\pgfqpoint{0.000000in}{-0.048611in}}%
\pgfusepath{stroke,fill}%
}%
\begin{pgfscope}%
\pgfsys@transformshift{0.900287in}{0.740652in}%
\pgfsys@useobject{currentmarker}{}%
\end{pgfscope}%
\end{pgfscope}%
\begin{pgfscope}%
\pgfsetbuttcap%
\pgfsetroundjoin%
\definecolor{currentfill}{rgb}{0.000000,0.000000,0.000000}%
\pgfsetfillcolor{currentfill}%
\pgfsetlinewidth{0.803000pt}%
\definecolor{currentstroke}{rgb}{0.000000,0.000000,0.000000}%
\pgfsetstrokecolor{currentstroke}%
\pgfsetdash{}{0pt}%
\pgfsys@defobject{currentmarker}{\pgfqpoint{0.000000in}{-0.048611in}}{\pgfqpoint{0.000000in}{0.000000in}}{%
\pgfpathmoveto{\pgfqpoint{0.000000in}{0.000000in}}%
\pgfpathlineto{\pgfqpoint{0.000000in}{-0.048611in}}%
\pgfusepath{stroke,fill}%
}%
\begin{pgfscope}%
\pgfsys@transformshift{1.065620in}{0.740652in}%
\pgfsys@useobject{currentmarker}{}%
\end{pgfscope}%
\end{pgfscope}%
\begin{pgfscope}%
\pgfsetbuttcap%
\pgfsetroundjoin%
\definecolor{currentfill}{rgb}{0.000000,0.000000,0.000000}%
\pgfsetfillcolor{currentfill}%
\pgfsetlinewidth{0.803000pt}%
\definecolor{currentstroke}{rgb}{0.000000,0.000000,0.000000}%
\pgfsetstrokecolor{currentstroke}%
\pgfsetdash{}{0pt}%
\pgfsys@defobject{currentmarker}{\pgfqpoint{0.000000in}{-0.048611in}}{\pgfqpoint{0.000000in}{0.000000in}}{%
\pgfpathmoveto{\pgfqpoint{0.000000in}{0.000000in}}%
\pgfpathlineto{\pgfqpoint{0.000000in}{-0.048611in}}%
\pgfusepath{stroke,fill}%
}%
\begin{pgfscope}%
\pgfsys@transformshift{1.230953in}{0.740652in}%
\pgfsys@useobject{currentmarker}{}%
\end{pgfscope}%
\end{pgfscope}%
\begin{pgfscope}%
\pgfsetbuttcap%
\pgfsetroundjoin%
\definecolor{currentfill}{rgb}{0.000000,0.000000,0.000000}%
\pgfsetfillcolor{currentfill}%
\pgfsetlinewidth{0.803000pt}%
\definecolor{currentstroke}{rgb}{0.000000,0.000000,0.000000}%
\pgfsetstrokecolor{currentstroke}%
\pgfsetdash{}{0pt}%
\pgfsys@defobject{currentmarker}{\pgfqpoint{0.000000in}{-0.048611in}}{\pgfqpoint{0.000000in}{0.000000in}}{%
\pgfpathmoveto{\pgfqpoint{0.000000in}{0.000000in}}%
\pgfpathlineto{\pgfqpoint{0.000000in}{-0.048611in}}%
\pgfusepath{stroke,fill}%
}%
\begin{pgfscope}%
\pgfsys@transformshift{1.396287in}{0.740652in}%
\pgfsys@useobject{currentmarker}{}%
\end{pgfscope}%
\end{pgfscope}%
\begin{pgfscope}%
\definecolor{textcolor}{rgb}{0.000000,0.000000,0.000000}%
\pgfsetstrokecolor{textcolor}%
\pgfsetfillcolor{textcolor}%
\pgftext[x=1.434603in, y=0.289968in, left, base,rotate=90.000000]{\color{textcolor}\sffamily\fontsize{10.000000}{12.000000}\selectfont 1000}%
\end{pgfscope}%
\begin{pgfscope}%
\pgfsetbuttcap%
\pgfsetroundjoin%
\definecolor{currentfill}{rgb}{0.000000,0.000000,0.000000}%
\pgfsetfillcolor{currentfill}%
\pgfsetlinewidth{0.803000pt}%
\definecolor{currentstroke}{rgb}{0.000000,0.000000,0.000000}%
\pgfsetstrokecolor{currentstroke}%
\pgfsetdash{}{0pt}%
\pgfsys@defobject{currentmarker}{\pgfqpoint{0.000000in}{-0.048611in}}{\pgfqpoint{0.000000in}{0.000000in}}{%
\pgfpathmoveto{\pgfqpoint{0.000000in}{0.000000in}}%
\pgfpathlineto{\pgfqpoint{0.000000in}{-0.048611in}}%
\pgfusepath{stroke,fill}%
}%
\begin{pgfscope}%
\pgfsys@transformshift{1.561620in}{0.740652in}%
\pgfsys@useobject{currentmarker}{}%
\end{pgfscope}%
\end{pgfscope}%
\begin{pgfscope}%
\pgfsetbuttcap%
\pgfsetroundjoin%
\definecolor{currentfill}{rgb}{0.000000,0.000000,0.000000}%
\pgfsetfillcolor{currentfill}%
\pgfsetlinewidth{0.803000pt}%
\definecolor{currentstroke}{rgb}{0.000000,0.000000,0.000000}%
\pgfsetstrokecolor{currentstroke}%
\pgfsetdash{}{0pt}%
\pgfsys@defobject{currentmarker}{\pgfqpoint{0.000000in}{-0.048611in}}{\pgfqpoint{0.000000in}{0.000000in}}{%
\pgfpathmoveto{\pgfqpoint{0.000000in}{0.000000in}}%
\pgfpathlineto{\pgfqpoint{0.000000in}{-0.048611in}}%
\pgfusepath{stroke,fill}%
}%
\begin{pgfscope}%
\pgfsys@transformshift{1.726953in}{0.740652in}%
\pgfsys@useobject{currentmarker}{}%
\end{pgfscope}%
\end{pgfscope}%
\begin{pgfscope}%
\pgfsetbuttcap%
\pgfsetroundjoin%
\definecolor{currentfill}{rgb}{0.000000,0.000000,0.000000}%
\pgfsetfillcolor{currentfill}%
\pgfsetlinewidth{0.803000pt}%
\definecolor{currentstroke}{rgb}{0.000000,0.000000,0.000000}%
\pgfsetstrokecolor{currentstroke}%
\pgfsetdash{}{0pt}%
\pgfsys@defobject{currentmarker}{\pgfqpoint{0.000000in}{-0.048611in}}{\pgfqpoint{0.000000in}{0.000000in}}{%
\pgfpathmoveto{\pgfqpoint{0.000000in}{0.000000in}}%
\pgfpathlineto{\pgfqpoint{0.000000in}{-0.048611in}}%
\pgfusepath{stroke,fill}%
}%
\begin{pgfscope}%
\pgfsys@transformshift{1.892287in}{0.740652in}%
\pgfsys@useobject{currentmarker}{}%
\end{pgfscope}%
\end{pgfscope}%
\begin{pgfscope}%
\pgfsetbuttcap%
\pgfsetroundjoin%
\definecolor{currentfill}{rgb}{0.000000,0.000000,0.000000}%
\pgfsetfillcolor{currentfill}%
\pgfsetlinewidth{0.803000pt}%
\definecolor{currentstroke}{rgb}{0.000000,0.000000,0.000000}%
\pgfsetstrokecolor{currentstroke}%
\pgfsetdash{}{0pt}%
\pgfsys@defobject{currentmarker}{\pgfqpoint{0.000000in}{-0.048611in}}{\pgfqpoint{0.000000in}{0.000000in}}{%
\pgfpathmoveto{\pgfqpoint{0.000000in}{0.000000in}}%
\pgfpathlineto{\pgfqpoint{0.000000in}{-0.048611in}}%
\pgfusepath{stroke,fill}%
}%
\begin{pgfscope}%
\pgfsys@transformshift{2.057620in}{0.740652in}%
\pgfsys@useobject{currentmarker}{}%
\end{pgfscope}%
\end{pgfscope}%
\begin{pgfscope}%
\definecolor{textcolor}{rgb}{0.000000,0.000000,0.000000}%
\pgfsetstrokecolor{textcolor}%
\pgfsetfillcolor{textcolor}%
\pgftext[x=2.095937in, y=0.289968in, left, base,rotate=90.000000]{\color{textcolor}\sffamily\fontsize{10.000000}{12.000000}\selectfont 1800}%
\end{pgfscope}%
\begin{pgfscope}%
\pgfsetbuttcap%
\pgfsetroundjoin%
\definecolor{currentfill}{rgb}{0.000000,0.000000,0.000000}%
\pgfsetfillcolor{currentfill}%
\pgfsetlinewidth{0.803000pt}%
\definecolor{currentstroke}{rgb}{0.000000,0.000000,0.000000}%
\pgfsetstrokecolor{currentstroke}%
\pgfsetdash{}{0pt}%
\pgfsys@defobject{currentmarker}{\pgfqpoint{0.000000in}{-0.048611in}}{\pgfqpoint{0.000000in}{0.000000in}}{%
\pgfpathmoveto{\pgfqpoint{0.000000in}{0.000000in}}%
\pgfpathlineto{\pgfqpoint{0.000000in}{-0.048611in}}%
\pgfusepath{stroke,fill}%
}%
\begin{pgfscope}%
\pgfsys@transformshift{2.222953in}{0.740652in}%
\pgfsys@useobject{currentmarker}{}%
\end{pgfscope}%
\end{pgfscope}%
\begin{pgfscope}%
\pgfsetbuttcap%
\pgfsetroundjoin%
\definecolor{currentfill}{rgb}{0.000000,0.000000,0.000000}%
\pgfsetfillcolor{currentfill}%
\pgfsetlinewidth{0.803000pt}%
\definecolor{currentstroke}{rgb}{0.000000,0.000000,0.000000}%
\pgfsetstrokecolor{currentstroke}%
\pgfsetdash{}{0pt}%
\pgfsys@defobject{currentmarker}{\pgfqpoint{0.000000in}{-0.048611in}}{\pgfqpoint{0.000000in}{0.000000in}}{%
\pgfpathmoveto{\pgfqpoint{0.000000in}{0.000000in}}%
\pgfpathlineto{\pgfqpoint{0.000000in}{-0.048611in}}%
\pgfusepath{stroke,fill}%
}%
\begin{pgfscope}%
\pgfsys@transformshift{2.388287in}{0.740652in}%
\pgfsys@useobject{currentmarker}{}%
\end{pgfscope}%
\end{pgfscope}%
\begin{pgfscope}%
\pgfsetbuttcap%
\pgfsetroundjoin%
\definecolor{currentfill}{rgb}{0.000000,0.000000,0.000000}%
\pgfsetfillcolor{currentfill}%
\pgfsetlinewidth{0.803000pt}%
\definecolor{currentstroke}{rgb}{0.000000,0.000000,0.000000}%
\pgfsetstrokecolor{currentstroke}%
\pgfsetdash{}{0pt}%
\pgfsys@defobject{currentmarker}{\pgfqpoint{0.000000in}{-0.048611in}}{\pgfqpoint{0.000000in}{0.000000in}}{%
\pgfpathmoveto{\pgfqpoint{0.000000in}{0.000000in}}%
\pgfpathlineto{\pgfqpoint{0.000000in}{-0.048611in}}%
\pgfusepath{stroke,fill}%
}%
\begin{pgfscope}%
\pgfsys@transformshift{2.553620in}{0.740652in}%
\pgfsys@useobject{currentmarker}{}%
\end{pgfscope}%
\end{pgfscope}%
\begin{pgfscope}%
\pgfsetbuttcap%
\pgfsetroundjoin%
\definecolor{currentfill}{rgb}{0.000000,0.000000,0.000000}%
\pgfsetfillcolor{currentfill}%
\pgfsetlinewidth{0.803000pt}%
\definecolor{currentstroke}{rgb}{0.000000,0.000000,0.000000}%
\pgfsetstrokecolor{currentstroke}%
\pgfsetdash{}{0pt}%
\pgfsys@defobject{currentmarker}{\pgfqpoint{0.000000in}{-0.048611in}}{\pgfqpoint{0.000000in}{0.000000in}}{%
\pgfpathmoveto{\pgfqpoint{0.000000in}{0.000000in}}%
\pgfpathlineto{\pgfqpoint{0.000000in}{-0.048611in}}%
\pgfusepath{stroke,fill}%
}%
\begin{pgfscope}%
\pgfsys@transformshift{2.718953in}{0.740652in}%
\pgfsys@useobject{currentmarker}{}%
\end{pgfscope}%
\end{pgfscope}%
\begin{pgfscope}%
\definecolor{textcolor}{rgb}{0.000000,0.000000,0.000000}%
\pgfsetstrokecolor{textcolor}%
\pgfsetfillcolor{textcolor}%
\pgftext[x=2.757270in, y=0.289968in, left, base,rotate=90.000000]{\color{textcolor}\sffamily\fontsize{10.000000}{12.000000}\selectfont 2600}%
\end{pgfscope}%
\begin{pgfscope}%
\pgfsetbuttcap%
\pgfsetroundjoin%
\definecolor{currentfill}{rgb}{0.000000,0.000000,0.000000}%
\pgfsetfillcolor{currentfill}%
\pgfsetlinewidth{0.803000pt}%
\definecolor{currentstroke}{rgb}{0.000000,0.000000,0.000000}%
\pgfsetstrokecolor{currentstroke}%
\pgfsetdash{}{0pt}%
\pgfsys@defobject{currentmarker}{\pgfqpoint{0.000000in}{-0.048611in}}{\pgfqpoint{0.000000in}{0.000000in}}{%
\pgfpathmoveto{\pgfqpoint{0.000000in}{0.000000in}}%
\pgfpathlineto{\pgfqpoint{0.000000in}{-0.048611in}}%
\pgfusepath{stroke,fill}%
}%
\begin{pgfscope}%
\pgfsys@transformshift{2.884287in}{0.740652in}%
\pgfsys@useobject{currentmarker}{}%
\end{pgfscope}%
\end{pgfscope}%
\begin{pgfscope}%
\pgfsetbuttcap%
\pgfsetroundjoin%
\definecolor{currentfill}{rgb}{0.000000,0.000000,0.000000}%
\pgfsetfillcolor{currentfill}%
\pgfsetlinewidth{0.803000pt}%
\definecolor{currentstroke}{rgb}{0.000000,0.000000,0.000000}%
\pgfsetstrokecolor{currentstroke}%
\pgfsetdash{}{0pt}%
\pgfsys@defobject{currentmarker}{\pgfqpoint{0.000000in}{-0.048611in}}{\pgfqpoint{0.000000in}{0.000000in}}{%
\pgfpathmoveto{\pgfqpoint{0.000000in}{0.000000in}}%
\pgfpathlineto{\pgfqpoint{0.000000in}{-0.048611in}}%
\pgfusepath{stroke,fill}%
}%
\begin{pgfscope}%
\pgfsys@transformshift{3.049620in}{0.740652in}%
\pgfsys@useobject{currentmarker}{}%
\end{pgfscope}%
\end{pgfscope}%
\begin{pgfscope}%
\pgfsetbuttcap%
\pgfsetroundjoin%
\definecolor{currentfill}{rgb}{0.000000,0.000000,0.000000}%
\pgfsetfillcolor{currentfill}%
\pgfsetlinewidth{0.803000pt}%
\definecolor{currentstroke}{rgb}{0.000000,0.000000,0.000000}%
\pgfsetstrokecolor{currentstroke}%
\pgfsetdash{}{0pt}%
\pgfsys@defobject{currentmarker}{\pgfqpoint{0.000000in}{-0.048611in}}{\pgfqpoint{0.000000in}{0.000000in}}{%
\pgfpathmoveto{\pgfqpoint{0.000000in}{0.000000in}}%
\pgfpathlineto{\pgfqpoint{0.000000in}{-0.048611in}}%
\pgfusepath{stroke,fill}%
}%
\begin{pgfscope}%
\pgfsys@transformshift{3.214953in}{0.740652in}%
\pgfsys@useobject{currentmarker}{}%
\end{pgfscope}%
\end{pgfscope}%
\begin{pgfscope}%
\pgfsetbuttcap%
\pgfsetroundjoin%
\definecolor{currentfill}{rgb}{0.000000,0.000000,0.000000}%
\pgfsetfillcolor{currentfill}%
\pgfsetlinewidth{0.803000pt}%
\definecolor{currentstroke}{rgb}{0.000000,0.000000,0.000000}%
\pgfsetstrokecolor{currentstroke}%
\pgfsetdash{}{0pt}%
\pgfsys@defobject{currentmarker}{\pgfqpoint{0.000000in}{-0.048611in}}{\pgfqpoint{0.000000in}{0.000000in}}{%
\pgfpathmoveto{\pgfqpoint{0.000000in}{0.000000in}}%
\pgfpathlineto{\pgfqpoint{0.000000in}{-0.048611in}}%
\pgfusepath{stroke,fill}%
}%
\begin{pgfscope}%
\pgfsys@transformshift{3.380287in}{0.740652in}%
\pgfsys@useobject{currentmarker}{}%
\end{pgfscope}%
\end{pgfscope}%
\begin{pgfscope}%
\definecolor{textcolor}{rgb}{0.000000,0.000000,0.000000}%
\pgfsetstrokecolor{textcolor}%
\pgfsetfillcolor{textcolor}%
\pgftext[x=3.418603in, y=0.289968in, left, base,rotate=90.000000]{\color{textcolor}\sffamily\fontsize{10.000000}{12.000000}\selectfont 3400}%
\end{pgfscope}%
\begin{pgfscope}%
\pgfsetbuttcap%
\pgfsetroundjoin%
\definecolor{currentfill}{rgb}{0.000000,0.000000,0.000000}%
\pgfsetfillcolor{currentfill}%
\pgfsetlinewidth{0.803000pt}%
\definecolor{currentstroke}{rgb}{0.000000,0.000000,0.000000}%
\pgfsetstrokecolor{currentstroke}%
\pgfsetdash{}{0pt}%
\pgfsys@defobject{currentmarker}{\pgfqpoint{0.000000in}{-0.048611in}}{\pgfqpoint{0.000000in}{0.000000in}}{%
\pgfpathmoveto{\pgfqpoint{0.000000in}{0.000000in}}%
\pgfpathlineto{\pgfqpoint{0.000000in}{-0.048611in}}%
\pgfusepath{stroke,fill}%
}%
\begin{pgfscope}%
\pgfsys@transformshift{3.545620in}{0.740652in}%
\pgfsys@useobject{currentmarker}{}%
\end{pgfscope}%
\end{pgfscope}%
\begin{pgfscope}%
\pgfsetbuttcap%
\pgfsetroundjoin%
\definecolor{currentfill}{rgb}{0.000000,0.000000,0.000000}%
\pgfsetfillcolor{currentfill}%
\pgfsetlinewidth{0.803000pt}%
\definecolor{currentstroke}{rgb}{0.000000,0.000000,0.000000}%
\pgfsetstrokecolor{currentstroke}%
\pgfsetdash{}{0pt}%
\pgfsys@defobject{currentmarker}{\pgfqpoint{0.000000in}{-0.048611in}}{\pgfqpoint{0.000000in}{0.000000in}}{%
\pgfpathmoveto{\pgfqpoint{0.000000in}{0.000000in}}%
\pgfpathlineto{\pgfqpoint{0.000000in}{-0.048611in}}%
\pgfusepath{stroke,fill}%
}%
\begin{pgfscope}%
\pgfsys@transformshift{3.710953in}{0.740652in}%
\pgfsys@useobject{currentmarker}{}%
\end{pgfscope}%
\end{pgfscope}%
\begin{pgfscope}%
\pgfsetbuttcap%
\pgfsetroundjoin%
\definecolor{currentfill}{rgb}{0.000000,0.000000,0.000000}%
\pgfsetfillcolor{currentfill}%
\pgfsetlinewidth{0.803000pt}%
\definecolor{currentstroke}{rgb}{0.000000,0.000000,0.000000}%
\pgfsetstrokecolor{currentstroke}%
\pgfsetdash{}{0pt}%
\pgfsys@defobject{currentmarker}{\pgfqpoint{0.000000in}{-0.048611in}}{\pgfqpoint{0.000000in}{0.000000in}}{%
\pgfpathmoveto{\pgfqpoint{0.000000in}{0.000000in}}%
\pgfpathlineto{\pgfqpoint{0.000000in}{-0.048611in}}%
\pgfusepath{stroke,fill}%
}%
\begin{pgfscope}%
\pgfsys@transformshift{3.876287in}{0.740652in}%
\pgfsys@useobject{currentmarker}{}%
\end{pgfscope}%
\end{pgfscope}%
\begin{pgfscope}%
\pgfsetbuttcap%
\pgfsetroundjoin%
\definecolor{currentfill}{rgb}{0.000000,0.000000,0.000000}%
\pgfsetfillcolor{currentfill}%
\pgfsetlinewidth{0.803000pt}%
\definecolor{currentstroke}{rgb}{0.000000,0.000000,0.000000}%
\pgfsetstrokecolor{currentstroke}%
\pgfsetdash{}{0pt}%
\pgfsys@defobject{currentmarker}{\pgfqpoint{0.000000in}{-0.048611in}}{\pgfqpoint{0.000000in}{0.000000in}}{%
\pgfpathmoveto{\pgfqpoint{0.000000in}{0.000000in}}%
\pgfpathlineto{\pgfqpoint{0.000000in}{-0.048611in}}%
\pgfusepath{stroke,fill}%
}%
\begin{pgfscope}%
\pgfsys@transformshift{4.041620in}{0.740652in}%
\pgfsys@useobject{currentmarker}{}%
\end{pgfscope}%
\end{pgfscope}%
\begin{pgfscope}%
\definecolor{textcolor}{rgb}{0.000000,0.000000,0.000000}%
\pgfsetstrokecolor{textcolor}%
\pgfsetfillcolor{textcolor}%
\pgftext[x=4.079937in, y=0.289968in, left, base,rotate=90.000000]{\color{textcolor}\sffamily\fontsize{10.000000}{12.000000}\selectfont 4200}%
\end{pgfscope}%
\begin{pgfscope}%
\pgfsetbuttcap%
\pgfsetroundjoin%
\definecolor{currentfill}{rgb}{0.000000,0.000000,0.000000}%
\pgfsetfillcolor{currentfill}%
\pgfsetlinewidth{0.803000pt}%
\definecolor{currentstroke}{rgb}{0.000000,0.000000,0.000000}%
\pgfsetstrokecolor{currentstroke}%
\pgfsetdash{}{0pt}%
\pgfsys@defobject{currentmarker}{\pgfqpoint{0.000000in}{-0.048611in}}{\pgfqpoint{0.000000in}{0.000000in}}{%
\pgfpathmoveto{\pgfqpoint{0.000000in}{0.000000in}}%
\pgfpathlineto{\pgfqpoint{0.000000in}{-0.048611in}}%
\pgfusepath{stroke,fill}%
}%
\begin{pgfscope}%
\pgfsys@transformshift{4.206953in}{0.740652in}%
\pgfsys@useobject{currentmarker}{}%
\end{pgfscope}%
\end{pgfscope}%
\begin{pgfscope}%
\pgfsetbuttcap%
\pgfsetroundjoin%
\definecolor{currentfill}{rgb}{0.000000,0.000000,0.000000}%
\pgfsetfillcolor{currentfill}%
\pgfsetlinewidth{0.803000pt}%
\definecolor{currentstroke}{rgb}{0.000000,0.000000,0.000000}%
\pgfsetstrokecolor{currentstroke}%
\pgfsetdash{}{0pt}%
\pgfsys@defobject{currentmarker}{\pgfqpoint{0.000000in}{-0.048611in}}{\pgfqpoint{0.000000in}{0.000000in}}{%
\pgfpathmoveto{\pgfqpoint{0.000000in}{0.000000in}}%
\pgfpathlineto{\pgfqpoint{0.000000in}{-0.048611in}}%
\pgfusepath{stroke,fill}%
}%
\begin{pgfscope}%
\pgfsys@transformshift{4.372287in}{0.740652in}%
\pgfsys@useobject{currentmarker}{}%
\end{pgfscope}%
\end{pgfscope}%
\begin{pgfscope}%
\pgfsetbuttcap%
\pgfsetroundjoin%
\definecolor{currentfill}{rgb}{0.000000,0.000000,0.000000}%
\pgfsetfillcolor{currentfill}%
\pgfsetlinewidth{0.803000pt}%
\definecolor{currentstroke}{rgb}{0.000000,0.000000,0.000000}%
\pgfsetstrokecolor{currentstroke}%
\pgfsetdash{}{0pt}%
\pgfsys@defobject{currentmarker}{\pgfqpoint{0.000000in}{-0.048611in}}{\pgfqpoint{0.000000in}{0.000000in}}{%
\pgfpathmoveto{\pgfqpoint{0.000000in}{0.000000in}}%
\pgfpathlineto{\pgfqpoint{0.000000in}{-0.048611in}}%
\pgfusepath{stroke,fill}%
}%
\begin{pgfscope}%
\pgfsys@transformshift{4.537620in}{0.740652in}%
\pgfsys@useobject{currentmarker}{}%
\end{pgfscope}%
\end{pgfscope}%
\begin{pgfscope}%
\pgfsetbuttcap%
\pgfsetroundjoin%
\definecolor{currentfill}{rgb}{0.000000,0.000000,0.000000}%
\pgfsetfillcolor{currentfill}%
\pgfsetlinewidth{0.803000pt}%
\definecolor{currentstroke}{rgb}{0.000000,0.000000,0.000000}%
\pgfsetstrokecolor{currentstroke}%
\pgfsetdash{}{0pt}%
\pgfsys@defobject{currentmarker}{\pgfqpoint{0.000000in}{-0.048611in}}{\pgfqpoint{0.000000in}{0.000000in}}{%
\pgfpathmoveto{\pgfqpoint{0.000000in}{0.000000in}}%
\pgfpathlineto{\pgfqpoint{0.000000in}{-0.048611in}}%
\pgfusepath{stroke,fill}%
}%
\begin{pgfscope}%
\pgfsys@transformshift{4.702953in}{0.740652in}%
\pgfsys@useobject{currentmarker}{}%
\end{pgfscope}%
\end{pgfscope}%
\begin{pgfscope}%
\definecolor{textcolor}{rgb}{0.000000,0.000000,0.000000}%
\pgfsetstrokecolor{textcolor}%
\pgfsetfillcolor{textcolor}%
\pgftext[x=4.741270in, y=0.289968in, left, base,rotate=90.000000]{\color{textcolor}\sffamily\fontsize{10.000000}{12.000000}\selectfont 5000}%
\end{pgfscope}%
\begin{pgfscope}%
\pgfsetbuttcap%
\pgfsetroundjoin%
\definecolor{currentfill}{rgb}{0.000000,0.000000,0.000000}%
\pgfsetfillcolor{currentfill}%
\pgfsetlinewidth{0.803000pt}%
\definecolor{currentstroke}{rgb}{0.000000,0.000000,0.000000}%
\pgfsetstrokecolor{currentstroke}%
\pgfsetdash{}{0pt}%
\pgfsys@defobject{currentmarker}{\pgfqpoint{0.000000in}{-0.048611in}}{\pgfqpoint{0.000000in}{0.000000in}}{%
\pgfpathmoveto{\pgfqpoint{0.000000in}{0.000000in}}%
\pgfpathlineto{\pgfqpoint{0.000000in}{-0.048611in}}%
\pgfusepath{stroke,fill}%
}%
\begin{pgfscope}%
\pgfsys@transformshift{4.868287in}{0.740652in}%
\pgfsys@useobject{currentmarker}{}%
\end{pgfscope}%
\end{pgfscope}%
\begin{pgfscope}%
\pgfsetbuttcap%
\pgfsetroundjoin%
\definecolor{currentfill}{rgb}{0.000000,0.000000,0.000000}%
\pgfsetfillcolor{currentfill}%
\pgfsetlinewidth{0.803000pt}%
\definecolor{currentstroke}{rgb}{0.000000,0.000000,0.000000}%
\pgfsetstrokecolor{currentstroke}%
\pgfsetdash{}{0pt}%
\pgfsys@defobject{currentmarker}{\pgfqpoint{0.000000in}{-0.048611in}}{\pgfqpoint{0.000000in}{0.000000in}}{%
\pgfpathmoveto{\pgfqpoint{0.000000in}{0.000000in}}%
\pgfpathlineto{\pgfqpoint{0.000000in}{-0.048611in}}%
\pgfusepath{stroke,fill}%
}%
\begin{pgfscope}%
\pgfsys@transformshift{5.033620in}{0.740652in}%
\pgfsys@useobject{currentmarker}{}%
\end{pgfscope}%
\end{pgfscope}%
\begin{pgfscope}%
\pgfsetbuttcap%
\pgfsetroundjoin%
\definecolor{currentfill}{rgb}{0.000000,0.000000,0.000000}%
\pgfsetfillcolor{currentfill}%
\pgfsetlinewidth{0.803000pt}%
\definecolor{currentstroke}{rgb}{0.000000,0.000000,0.000000}%
\pgfsetstrokecolor{currentstroke}%
\pgfsetdash{}{0pt}%
\pgfsys@defobject{currentmarker}{\pgfqpoint{0.000000in}{-0.048611in}}{\pgfqpoint{0.000000in}{0.000000in}}{%
\pgfpathmoveto{\pgfqpoint{0.000000in}{0.000000in}}%
\pgfpathlineto{\pgfqpoint{0.000000in}{-0.048611in}}%
\pgfusepath{stroke,fill}%
}%
\begin{pgfscope}%
\pgfsys@transformshift{5.198953in}{0.740652in}%
\pgfsys@useobject{currentmarker}{}%
\end{pgfscope}%
\end{pgfscope}%
\begin{pgfscope}%
\pgfsetbuttcap%
\pgfsetroundjoin%
\definecolor{currentfill}{rgb}{0.000000,0.000000,0.000000}%
\pgfsetfillcolor{currentfill}%
\pgfsetlinewidth{0.803000pt}%
\definecolor{currentstroke}{rgb}{0.000000,0.000000,0.000000}%
\pgfsetstrokecolor{currentstroke}%
\pgfsetdash{}{0pt}%
\pgfsys@defobject{currentmarker}{\pgfqpoint{0.000000in}{-0.048611in}}{\pgfqpoint{0.000000in}{0.000000in}}{%
\pgfpathmoveto{\pgfqpoint{0.000000in}{0.000000in}}%
\pgfpathlineto{\pgfqpoint{0.000000in}{-0.048611in}}%
\pgfusepath{stroke,fill}%
}%
\begin{pgfscope}%
\pgfsys@transformshift{5.364287in}{0.740652in}%
\pgfsys@useobject{currentmarker}{}%
\end{pgfscope}%
\end{pgfscope}%
\begin{pgfscope}%
\definecolor{textcolor}{rgb}{0.000000,0.000000,0.000000}%
\pgfsetstrokecolor{textcolor}%
\pgfsetfillcolor{textcolor}%
\pgftext[x=5.402603in, y=0.289968in, left, base,rotate=90.000000]{\color{textcolor}\sffamily\fontsize{10.000000}{12.000000}\selectfont 5800}%
\end{pgfscope}%
\begin{pgfscope}%
\pgfsetbuttcap%
\pgfsetroundjoin%
\definecolor{currentfill}{rgb}{0.000000,0.000000,0.000000}%
\pgfsetfillcolor{currentfill}%
\pgfsetlinewidth{0.803000pt}%
\definecolor{currentstroke}{rgb}{0.000000,0.000000,0.000000}%
\pgfsetstrokecolor{currentstroke}%
\pgfsetdash{}{0pt}%
\pgfsys@defobject{currentmarker}{\pgfqpoint{0.000000in}{-0.048611in}}{\pgfqpoint{0.000000in}{0.000000in}}{%
\pgfpathmoveto{\pgfqpoint{0.000000in}{0.000000in}}%
\pgfpathlineto{\pgfqpoint{0.000000in}{-0.048611in}}%
\pgfusepath{stroke,fill}%
}%
\begin{pgfscope}%
\pgfsys@transformshift{5.529620in}{0.740652in}%
\pgfsys@useobject{currentmarker}{}%
\end{pgfscope}%
\end{pgfscope}%
\begin{pgfscope}%
\definecolor{textcolor}{rgb}{0.000000,0.000000,0.000000}%
\pgfsetstrokecolor{textcolor}%
\pgfsetfillcolor{textcolor}%
\pgftext[x=3.132287in,y=0.234413in,,top]{\color{textcolor}\sffamily\fontsize{10.000000}{12.000000}\selectfont msg size (bytes)}%
\end{pgfscope}%
\begin{pgfscope}%
\pgfsetbuttcap%
\pgfsetroundjoin%
\definecolor{currentfill}{rgb}{0.000000,0.000000,0.000000}%
\pgfsetfillcolor{currentfill}%
\pgfsetlinewidth{0.803000pt}%
\definecolor{currentstroke}{rgb}{0.000000,0.000000,0.000000}%
\pgfsetstrokecolor{currentstroke}%
\pgfsetdash{}{0pt}%
\pgfsys@defobject{currentmarker}{\pgfqpoint{-0.048611in}{0.000000in}}{\pgfqpoint{-0.000000in}{0.000000in}}{%
\pgfpathmoveto{\pgfqpoint{-0.000000in}{0.000000in}}%
\pgfpathlineto{\pgfqpoint{-0.048611in}{0.000000in}}%
\pgfusepath{stroke,fill}%
}%
\begin{pgfscope}%
\pgfsys@transformshift{0.652287in}{0.740652in}%
\pgfsys@useobject{currentmarker}{}%
\end{pgfscope}%
\end{pgfscope}%
\begin{pgfscope}%
\definecolor{textcolor}{rgb}{0.000000,0.000000,0.000000}%
\pgfsetstrokecolor{textcolor}%
\pgfsetfillcolor{textcolor}%
\pgftext[x=0.466699in, y=0.687890in, left, base]{\color{textcolor}\sffamily\fontsize{10.000000}{12.000000}\selectfont 0}%
\end{pgfscope}%
\begin{pgfscope}%
\pgfsetbuttcap%
\pgfsetroundjoin%
\definecolor{currentfill}{rgb}{0.000000,0.000000,0.000000}%
\pgfsetfillcolor{currentfill}%
\pgfsetlinewidth{0.803000pt}%
\definecolor{currentstroke}{rgb}{0.000000,0.000000,0.000000}%
\pgfsetstrokecolor{currentstroke}%
\pgfsetdash{}{0pt}%
\pgfsys@defobject{currentmarker}{\pgfqpoint{-0.048611in}{0.000000in}}{\pgfqpoint{-0.000000in}{0.000000in}}{%
\pgfpathmoveto{\pgfqpoint{-0.000000in}{0.000000in}}%
\pgfpathlineto{\pgfqpoint{-0.048611in}{0.000000in}}%
\pgfusepath{stroke,fill}%
}%
\begin{pgfscope}%
\pgfsys@transformshift{0.652287in}{1.479852in}%
\pgfsys@useobject{currentmarker}{}%
\end{pgfscope}%
\end{pgfscope}%
\begin{pgfscope}%
\definecolor{textcolor}{rgb}{0.000000,0.000000,0.000000}%
\pgfsetstrokecolor{textcolor}%
\pgfsetfillcolor{textcolor}%
\pgftext[x=0.378334in, y=1.427090in, left, base]{\color{textcolor}\sffamily\fontsize{10.000000}{12.000000}\selectfont 20}%
\end{pgfscope}%
\begin{pgfscope}%
\pgfsetbuttcap%
\pgfsetroundjoin%
\definecolor{currentfill}{rgb}{0.000000,0.000000,0.000000}%
\pgfsetfillcolor{currentfill}%
\pgfsetlinewidth{0.803000pt}%
\definecolor{currentstroke}{rgb}{0.000000,0.000000,0.000000}%
\pgfsetstrokecolor{currentstroke}%
\pgfsetdash{}{0pt}%
\pgfsys@defobject{currentmarker}{\pgfqpoint{-0.048611in}{0.000000in}}{\pgfqpoint{-0.000000in}{0.000000in}}{%
\pgfpathmoveto{\pgfqpoint{-0.000000in}{0.000000in}}%
\pgfpathlineto{\pgfqpoint{-0.048611in}{0.000000in}}%
\pgfusepath{stroke,fill}%
}%
\begin{pgfscope}%
\pgfsys@transformshift{0.652287in}{2.219052in}%
\pgfsys@useobject{currentmarker}{}%
\end{pgfscope}%
\end{pgfscope}%
\begin{pgfscope}%
\definecolor{textcolor}{rgb}{0.000000,0.000000,0.000000}%
\pgfsetstrokecolor{textcolor}%
\pgfsetfillcolor{textcolor}%
\pgftext[x=0.378334in, y=2.166290in, left, base]{\color{textcolor}\sffamily\fontsize{10.000000}{12.000000}\selectfont 40}%
\end{pgfscope}%
\begin{pgfscope}%
\pgfsetbuttcap%
\pgfsetroundjoin%
\definecolor{currentfill}{rgb}{0.000000,0.000000,0.000000}%
\pgfsetfillcolor{currentfill}%
\pgfsetlinewidth{0.803000pt}%
\definecolor{currentstroke}{rgb}{0.000000,0.000000,0.000000}%
\pgfsetstrokecolor{currentstroke}%
\pgfsetdash{}{0pt}%
\pgfsys@defobject{currentmarker}{\pgfqpoint{-0.048611in}{0.000000in}}{\pgfqpoint{-0.000000in}{0.000000in}}{%
\pgfpathmoveto{\pgfqpoint{-0.000000in}{0.000000in}}%
\pgfpathlineto{\pgfqpoint{-0.048611in}{0.000000in}}%
\pgfusepath{stroke,fill}%
}%
\begin{pgfscope}%
\pgfsys@transformshift{0.652287in}{2.958252in}%
\pgfsys@useobject{currentmarker}{}%
\end{pgfscope}%
\end{pgfscope}%
\begin{pgfscope}%
\definecolor{textcolor}{rgb}{0.000000,0.000000,0.000000}%
\pgfsetstrokecolor{textcolor}%
\pgfsetfillcolor{textcolor}%
\pgftext[x=0.378334in, y=2.905490in, left, base]{\color{textcolor}\sffamily\fontsize{10.000000}{12.000000}\selectfont 60}%
\end{pgfscope}%
\begin{pgfscope}%
\pgfsetbuttcap%
\pgfsetroundjoin%
\definecolor{currentfill}{rgb}{0.000000,0.000000,0.000000}%
\pgfsetfillcolor{currentfill}%
\pgfsetlinewidth{0.803000pt}%
\definecolor{currentstroke}{rgb}{0.000000,0.000000,0.000000}%
\pgfsetstrokecolor{currentstroke}%
\pgfsetdash{}{0pt}%
\pgfsys@defobject{currentmarker}{\pgfqpoint{-0.048611in}{0.000000in}}{\pgfqpoint{-0.000000in}{0.000000in}}{%
\pgfpathmoveto{\pgfqpoint{-0.000000in}{0.000000in}}%
\pgfpathlineto{\pgfqpoint{-0.048611in}{0.000000in}}%
\pgfusepath{stroke,fill}%
}%
\begin{pgfscope}%
\pgfsys@transformshift{0.652287in}{3.697452in}%
\pgfsys@useobject{currentmarker}{}%
\end{pgfscope}%
\end{pgfscope}%
\begin{pgfscope}%
\definecolor{textcolor}{rgb}{0.000000,0.000000,0.000000}%
\pgfsetstrokecolor{textcolor}%
\pgfsetfillcolor{textcolor}%
\pgftext[x=0.378334in, y=3.644690in, left, base]{\color{textcolor}\sffamily\fontsize{10.000000}{12.000000}\selectfont 80}%
\end{pgfscope}%
\begin{pgfscope}%
\pgfsetbuttcap%
\pgfsetroundjoin%
\definecolor{currentfill}{rgb}{0.000000,0.000000,0.000000}%
\pgfsetfillcolor{currentfill}%
\pgfsetlinewidth{0.803000pt}%
\definecolor{currentstroke}{rgb}{0.000000,0.000000,0.000000}%
\pgfsetstrokecolor{currentstroke}%
\pgfsetdash{}{0pt}%
\pgfsys@defobject{currentmarker}{\pgfqpoint{-0.048611in}{0.000000in}}{\pgfqpoint{-0.000000in}{0.000000in}}{%
\pgfpathmoveto{\pgfqpoint{-0.000000in}{0.000000in}}%
\pgfpathlineto{\pgfqpoint{-0.048611in}{0.000000in}}%
\pgfusepath{stroke,fill}%
}%
\begin{pgfscope}%
\pgfsys@transformshift{0.652287in}{4.436652in}%
\pgfsys@useobject{currentmarker}{}%
\end{pgfscope}%
\end{pgfscope}%
\begin{pgfscope}%
\definecolor{textcolor}{rgb}{0.000000,0.000000,0.000000}%
\pgfsetstrokecolor{textcolor}%
\pgfsetfillcolor{textcolor}%
\pgftext[x=0.289968in, y=4.383890in, left, base]{\color{textcolor}\sffamily\fontsize{10.000000}{12.000000}\selectfont 100}%
\end{pgfscope}%
\begin{pgfscope}%
\definecolor{textcolor}{rgb}{0.000000,0.000000,0.000000}%
\pgfsetstrokecolor{textcolor}%
\pgfsetfillcolor{textcolor}%
\pgftext[x=0.234413in,y=2.588652in,,bottom,rotate=90.000000]{\color{textcolor}\sffamily\fontsize{10.000000}{12.000000}\selectfont send time (ms)}%
\end{pgfscope}%
\begin{pgfscope}%
\pgfpathrectangle{\pgfqpoint{0.652287in}{0.740652in}}{\pgfqpoint{4.960000in}{3.696000in}}%
\pgfusepath{clip}%
\pgfsetrectcap%
\pgfsetroundjoin%
\pgfsetlinewidth{1.505625pt}%
\definecolor{currentstroke}{rgb}{0.270588,0.270588,0.270588}%
\pgfsetstrokecolor{currentstroke}%
\pgfsetdash{}{0pt}%
\pgfpathmoveto{\pgfqpoint{0.734953in}{0.883059in}}%
\pgfpathlineto{\pgfqpoint{0.734953in}{0.791398in}}%
\pgfusepath{stroke}%
\end{pgfscope}%
\begin{pgfscope}%
\pgfpathrectangle{\pgfqpoint{0.652287in}{0.740652in}}{\pgfqpoint{4.960000in}{3.696000in}}%
\pgfusepath{clip}%
\pgfsetrectcap%
\pgfsetroundjoin%
\pgfsetlinewidth{1.505625pt}%
\definecolor{currentstroke}{rgb}{0.270588,0.270588,0.270588}%
\pgfsetstrokecolor{currentstroke}%
\pgfsetdash{}{0pt}%
\pgfpathmoveto{\pgfqpoint{0.734953in}{1.001737in}}%
\pgfpathlineto{\pgfqpoint{0.734953in}{1.139672in}}%
\pgfusepath{stroke}%
\end{pgfscope}%
\begin{pgfscope}%
\pgfpathrectangle{\pgfqpoint{0.652287in}{0.740652in}}{\pgfqpoint{4.960000in}{3.696000in}}%
\pgfusepath{clip}%
\pgfsetrectcap%
\pgfsetroundjoin%
\pgfsetlinewidth{1.505625pt}%
\definecolor{currentstroke}{rgb}{0.270588,0.270588,0.270588}%
\pgfsetstrokecolor{currentstroke}%
\pgfsetdash{}{0pt}%
\pgfpathmoveto{\pgfqpoint{0.701887in}{0.791398in}}%
\pgfpathlineto{\pgfqpoint{0.768020in}{0.791398in}}%
\pgfusepath{stroke}%
\end{pgfscope}%
\begin{pgfscope}%
\pgfpathrectangle{\pgfqpoint{0.652287in}{0.740652in}}{\pgfqpoint{4.960000in}{3.696000in}}%
\pgfusepath{clip}%
\pgfsetrectcap%
\pgfsetroundjoin%
\pgfsetlinewidth{1.505625pt}%
\definecolor{currentstroke}{rgb}{0.270588,0.270588,0.270588}%
\pgfsetstrokecolor{currentstroke}%
\pgfsetdash{}{0pt}%
\pgfpathmoveto{\pgfqpoint{0.701887in}{1.139672in}}%
\pgfpathlineto{\pgfqpoint{0.768020in}{1.139672in}}%
\pgfusepath{stroke}%
\end{pgfscope}%
\begin{pgfscope}%
\pgfpathrectangle{\pgfqpoint{0.652287in}{0.740652in}}{\pgfqpoint{4.960000in}{3.696000in}}%
\pgfusepath{clip}%
\pgfsetrectcap%
\pgfsetroundjoin%
\pgfsetlinewidth{1.505625pt}%
\definecolor{currentstroke}{rgb}{0.270588,0.270588,0.270588}%
\pgfsetstrokecolor{currentstroke}%
\pgfsetdash{}{0pt}%
\pgfpathmoveto{\pgfqpoint{0.900287in}{0.843068in}}%
\pgfpathlineto{\pgfqpoint{0.900287in}{0.775247in}}%
\pgfusepath{stroke}%
\end{pgfscope}%
\begin{pgfscope}%
\pgfpathrectangle{\pgfqpoint{0.652287in}{0.740652in}}{\pgfqpoint{4.960000in}{3.696000in}}%
\pgfusepath{clip}%
\pgfsetrectcap%
\pgfsetroundjoin%
\pgfsetlinewidth{1.505625pt}%
\definecolor{currentstroke}{rgb}{0.270588,0.270588,0.270588}%
\pgfsetstrokecolor{currentstroke}%
\pgfsetdash{}{0pt}%
\pgfpathmoveto{\pgfqpoint{0.900287in}{0.924269in}}%
\pgfpathlineto{\pgfqpoint{0.900287in}{1.028681in}}%
\pgfusepath{stroke}%
\end{pgfscope}%
\begin{pgfscope}%
\pgfpathrectangle{\pgfqpoint{0.652287in}{0.740652in}}{\pgfqpoint{4.960000in}{3.696000in}}%
\pgfusepath{clip}%
\pgfsetrectcap%
\pgfsetroundjoin%
\pgfsetlinewidth{1.505625pt}%
\definecolor{currentstroke}{rgb}{0.270588,0.270588,0.270588}%
\pgfsetstrokecolor{currentstroke}%
\pgfsetdash{}{0pt}%
\pgfpathmoveto{\pgfqpoint{0.867220in}{0.775247in}}%
\pgfpathlineto{\pgfqpoint{0.933353in}{0.775247in}}%
\pgfusepath{stroke}%
\end{pgfscope}%
\begin{pgfscope}%
\pgfpathrectangle{\pgfqpoint{0.652287in}{0.740652in}}{\pgfqpoint{4.960000in}{3.696000in}}%
\pgfusepath{clip}%
\pgfsetrectcap%
\pgfsetroundjoin%
\pgfsetlinewidth{1.505625pt}%
\definecolor{currentstroke}{rgb}{0.270588,0.270588,0.270588}%
\pgfsetstrokecolor{currentstroke}%
\pgfsetdash{}{0pt}%
\pgfpathmoveto{\pgfqpoint{0.867220in}{1.028681in}}%
\pgfpathlineto{\pgfqpoint{0.933353in}{1.028681in}}%
\pgfusepath{stroke}%
\end{pgfscope}%
\begin{pgfscope}%
\pgfpathrectangle{\pgfqpoint{0.652287in}{0.740652in}}{\pgfqpoint{4.960000in}{3.696000in}}%
\pgfusepath{clip}%
\pgfsetrectcap%
\pgfsetroundjoin%
\pgfsetlinewidth{1.505625pt}%
\definecolor{currentstroke}{rgb}{0.270588,0.270588,0.270588}%
\pgfsetstrokecolor{currentstroke}%
\pgfsetdash{}{0pt}%
\pgfpathmoveto{\pgfqpoint{1.065620in}{0.829689in}}%
\pgfpathlineto{\pgfqpoint{1.065620in}{0.770737in}}%
\pgfusepath{stroke}%
\end{pgfscope}%
\begin{pgfscope}%
\pgfpathrectangle{\pgfqpoint{0.652287in}{0.740652in}}{\pgfqpoint{4.960000in}{3.696000in}}%
\pgfusepath{clip}%
\pgfsetrectcap%
\pgfsetroundjoin%
\pgfsetlinewidth{1.505625pt}%
\definecolor{currentstroke}{rgb}{0.270588,0.270588,0.270588}%
\pgfsetstrokecolor{currentstroke}%
\pgfsetdash{}{0pt}%
\pgfpathmoveto{\pgfqpoint{1.065620in}{0.901650in}}%
\pgfpathlineto{\pgfqpoint{1.065620in}{0.982777in}}%
\pgfusepath{stroke}%
\end{pgfscope}%
\begin{pgfscope}%
\pgfpathrectangle{\pgfqpoint{0.652287in}{0.740652in}}{\pgfqpoint{4.960000in}{3.696000in}}%
\pgfusepath{clip}%
\pgfsetrectcap%
\pgfsetroundjoin%
\pgfsetlinewidth{1.505625pt}%
\definecolor{currentstroke}{rgb}{0.270588,0.270588,0.270588}%
\pgfsetstrokecolor{currentstroke}%
\pgfsetdash{}{0pt}%
\pgfpathmoveto{\pgfqpoint{1.032553in}{0.770737in}}%
\pgfpathlineto{\pgfqpoint{1.098687in}{0.770737in}}%
\pgfusepath{stroke}%
\end{pgfscope}%
\begin{pgfscope}%
\pgfpathrectangle{\pgfqpoint{0.652287in}{0.740652in}}{\pgfqpoint{4.960000in}{3.696000in}}%
\pgfusepath{clip}%
\pgfsetrectcap%
\pgfsetroundjoin%
\pgfsetlinewidth{1.505625pt}%
\definecolor{currentstroke}{rgb}{0.270588,0.270588,0.270588}%
\pgfsetstrokecolor{currentstroke}%
\pgfsetdash{}{0pt}%
\pgfpathmoveto{\pgfqpoint{1.032553in}{0.982777in}}%
\pgfpathlineto{\pgfqpoint{1.098687in}{0.982777in}}%
\pgfusepath{stroke}%
\end{pgfscope}%
\begin{pgfscope}%
\pgfpathrectangle{\pgfqpoint{0.652287in}{0.740652in}}{\pgfqpoint{4.960000in}{3.696000in}}%
\pgfusepath{clip}%
\pgfsetrectcap%
\pgfsetroundjoin%
\pgfsetlinewidth{1.505625pt}%
\definecolor{currentstroke}{rgb}{0.270588,0.270588,0.270588}%
\pgfsetstrokecolor{currentstroke}%
\pgfsetdash{}{0pt}%
\pgfpathmoveto{\pgfqpoint{1.230953in}{0.822851in}}%
\pgfpathlineto{\pgfqpoint{1.230953in}{0.767891in}}%
\pgfusepath{stroke}%
\end{pgfscope}%
\begin{pgfscope}%
\pgfpathrectangle{\pgfqpoint{0.652287in}{0.740652in}}{\pgfqpoint{4.960000in}{3.696000in}}%
\pgfusepath{clip}%
\pgfsetrectcap%
\pgfsetroundjoin%
\pgfsetlinewidth{1.505625pt}%
\definecolor{currentstroke}{rgb}{0.270588,0.270588,0.270588}%
\pgfsetstrokecolor{currentstroke}%
\pgfsetdash{}{0pt}%
\pgfpathmoveto{\pgfqpoint{1.230953in}{0.896475in}}%
\pgfpathlineto{\pgfqpoint{1.230953in}{0.972317in}}%
\pgfusepath{stroke}%
\end{pgfscope}%
\begin{pgfscope}%
\pgfpathrectangle{\pgfqpoint{0.652287in}{0.740652in}}{\pgfqpoint{4.960000in}{3.696000in}}%
\pgfusepath{clip}%
\pgfsetrectcap%
\pgfsetroundjoin%
\pgfsetlinewidth{1.505625pt}%
\definecolor{currentstroke}{rgb}{0.270588,0.270588,0.270588}%
\pgfsetstrokecolor{currentstroke}%
\pgfsetdash{}{0pt}%
\pgfpathmoveto{\pgfqpoint{1.197887in}{0.767891in}}%
\pgfpathlineto{\pgfqpoint{1.264020in}{0.767891in}}%
\pgfusepath{stroke}%
\end{pgfscope}%
\begin{pgfscope}%
\pgfpathrectangle{\pgfqpoint{0.652287in}{0.740652in}}{\pgfqpoint{4.960000in}{3.696000in}}%
\pgfusepath{clip}%
\pgfsetrectcap%
\pgfsetroundjoin%
\pgfsetlinewidth{1.505625pt}%
\definecolor{currentstroke}{rgb}{0.270588,0.270588,0.270588}%
\pgfsetstrokecolor{currentstroke}%
\pgfsetdash{}{0pt}%
\pgfpathmoveto{\pgfqpoint{1.197887in}{0.972317in}}%
\pgfpathlineto{\pgfqpoint{1.264020in}{0.972317in}}%
\pgfusepath{stroke}%
\end{pgfscope}%
\begin{pgfscope}%
\pgfpathrectangle{\pgfqpoint{0.652287in}{0.740652in}}{\pgfqpoint{4.960000in}{3.696000in}}%
\pgfusepath{clip}%
\pgfsetrectcap%
\pgfsetroundjoin%
\pgfsetlinewidth{1.505625pt}%
\definecolor{currentstroke}{rgb}{0.270588,0.270588,0.270588}%
\pgfsetstrokecolor{currentstroke}%
\pgfsetdash{}{0pt}%
\pgfpathmoveto{\pgfqpoint{1.396287in}{0.825771in}}%
\pgfpathlineto{\pgfqpoint{1.396287in}{0.769924in}}%
\pgfusepath{stroke}%
\end{pgfscope}%
\begin{pgfscope}%
\pgfpathrectangle{\pgfqpoint{0.652287in}{0.740652in}}{\pgfqpoint{4.960000in}{3.696000in}}%
\pgfusepath{clip}%
\pgfsetrectcap%
\pgfsetroundjoin%
\pgfsetlinewidth{1.505625pt}%
\definecolor{currentstroke}{rgb}{0.270588,0.270588,0.270588}%
\pgfsetstrokecolor{currentstroke}%
\pgfsetdash{}{0pt}%
\pgfpathmoveto{\pgfqpoint{1.396287in}{0.910853in}}%
\pgfpathlineto{\pgfqpoint{1.396287in}{0.986066in}}%
\pgfusepath{stroke}%
\end{pgfscope}%
\begin{pgfscope}%
\pgfpathrectangle{\pgfqpoint{0.652287in}{0.740652in}}{\pgfqpoint{4.960000in}{3.696000in}}%
\pgfusepath{clip}%
\pgfsetrectcap%
\pgfsetroundjoin%
\pgfsetlinewidth{1.505625pt}%
\definecolor{currentstroke}{rgb}{0.270588,0.270588,0.270588}%
\pgfsetstrokecolor{currentstroke}%
\pgfsetdash{}{0pt}%
\pgfpathmoveto{\pgfqpoint{1.363220in}{0.769924in}}%
\pgfpathlineto{\pgfqpoint{1.429353in}{0.769924in}}%
\pgfusepath{stroke}%
\end{pgfscope}%
\begin{pgfscope}%
\pgfpathrectangle{\pgfqpoint{0.652287in}{0.740652in}}{\pgfqpoint{4.960000in}{3.696000in}}%
\pgfusepath{clip}%
\pgfsetrectcap%
\pgfsetroundjoin%
\pgfsetlinewidth{1.505625pt}%
\definecolor{currentstroke}{rgb}{0.270588,0.270588,0.270588}%
\pgfsetstrokecolor{currentstroke}%
\pgfsetdash{}{0pt}%
\pgfpathmoveto{\pgfqpoint{1.363220in}{0.986066in}}%
\pgfpathlineto{\pgfqpoint{1.429353in}{0.986066in}}%
\pgfusepath{stroke}%
\end{pgfscope}%
\begin{pgfscope}%
\pgfpathrectangle{\pgfqpoint{0.652287in}{0.740652in}}{\pgfqpoint{4.960000in}{3.696000in}}%
\pgfusepath{clip}%
\pgfsetrectcap%
\pgfsetroundjoin%
\pgfsetlinewidth{1.505625pt}%
\definecolor{currentstroke}{rgb}{0.270588,0.270588,0.270588}%
\pgfsetstrokecolor{currentstroke}%
\pgfsetdash{}{0pt}%
\pgfpathmoveto{\pgfqpoint{1.561620in}{0.821964in}}%
\pgfpathlineto{\pgfqpoint{1.561620in}{0.769037in}}%
\pgfusepath{stroke}%
\end{pgfscope}%
\begin{pgfscope}%
\pgfpathrectangle{\pgfqpoint{0.652287in}{0.740652in}}{\pgfqpoint{4.960000in}{3.696000in}}%
\pgfusepath{clip}%
\pgfsetrectcap%
\pgfsetroundjoin%
\pgfsetlinewidth{1.505625pt}%
\definecolor{currentstroke}{rgb}{0.270588,0.270588,0.270588}%
\pgfsetstrokecolor{currentstroke}%
\pgfsetdash{}{0pt}%
\pgfpathmoveto{\pgfqpoint{1.561620in}{0.902204in}}%
\pgfpathlineto{\pgfqpoint{1.561620in}{0.973537in}}%
\pgfusepath{stroke}%
\end{pgfscope}%
\begin{pgfscope}%
\pgfpathrectangle{\pgfqpoint{0.652287in}{0.740652in}}{\pgfqpoint{4.960000in}{3.696000in}}%
\pgfusepath{clip}%
\pgfsetrectcap%
\pgfsetroundjoin%
\pgfsetlinewidth{1.505625pt}%
\definecolor{currentstroke}{rgb}{0.270588,0.270588,0.270588}%
\pgfsetstrokecolor{currentstroke}%
\pgfsetdash{}{0pt}%
\pgfpathmoveto{\pgfqpoint{1.528553in}{0.769037in}}%
\pgfpathlineto{\pgfqpoint{1.594687in}{0.769037in}}%
\pgfusepath{stroke}%
\end{pgfscope}%
\begin{pgfscope}%
\pgfpathrectangle{\pgfqpoint{0.652287in}{0.740652in}}{\pgfqpoint{4.960000in}{3.696000in}}%
\pgfusepath{clip}%
\pgfsetrectcap%
\pgfsetroundjoin%
\pgfsetlinewidth{1.505625pt}%
\definecolor{currentstroke}{rgb}{0.270588,0.270588,0.270588}%
\pgfsetstrokecolor{currentstroke}%
\pgfsetdash{}{0pt}%
\pgfpathmoveto{\pgfqpoint{1.528553in}{0.973537in}}%
\pgfpathlineto{\pgfqpoint{1.594687in}{0.973537in}}%
\pgfusepath{stroke}%
\end{pgfscope}%
\begin{pgfscope}%
\pgfpathrectangle{\pgfqpoint{0.652287in}{0.740652in}}{\pgfqpoint{4.960000in}{3.696000in}}%
\pgfusepath{clip}%
\pgfsetrectcap%
\pgfsetroundjoin%
\pgfsetlinewidth{1.505625pt}%
\definecolor{currentstroke}{rgb}{0.270588,0.270588,0.270588}%
\pgfsetstrokecolor{currentstroke}%
\pgfsetdash{}{0pt}%
\pgfpathmoveto{\pgfqpoint{1.726953in}{0.824403in}}%
\pgfpathlineto{\pgfqpoint{1.726953in}{0.770109in}}%
\pgfusepath{stroke}%
\end{pgfscope}%
\begin{pgfscope}%
\pgfpathrectangle{\pgfqpoint{0.652287in}{0.740652in}}{\pgfqpoint{4.960000in}{3.696000in}}%
\pgfusepath{clip}%
\pgfsetrectcap%
\pgfsetroundjoin%
\pgfsetlinewidth{1.505625pt}%
\definecolor{currentstroke}{rgb}{0.270588,0.270588,0.270588}%
\pgfsetstrokecolor{currentstroke}%
\pgfsetdash{}{0pt}%
\pgfpathmoveto{\pgfqpoint{1.726953in}{0.934138in}}%
\pgfpathlineto{\pgfqpoint{1.726953in}{1.011532in}}%
\pgfusepath{stroke}%
\end{pgfscope}%
\begin{pgfscope}%
\pgfpathrectangle{\pgfqpoint{0.652287in}{0.740652in}}{\pgfqpoint{4.960000in}{3.696000in}}%
\pgfusepath{clip}%
\pgfsetrectcap%
\pgfsetroundjoin%
\pgfsetlinewidth{1.505625pt}%
\definecolor{currentstroke}{rgb}{0.270588,0.270588,0.270588}%
\pgfsetstrokecolor{currentstroke}%
\pgfsetdash{}{0pt}%
\pgfpathmoveto{\pgfqpoint{1.693887in}{0.770109in}}%
\pgfpathlineto{\pgfqpoint{1.760020in}{0.770109in}}%
\pgfusepath{stroke}%
\end{pgfscope}%
\begin{pgfscope}%
\pgfpathrectangle{\pgfqpoint{0.652287in}{0.740652in}}{\pgfqpoint{4.960000in}{3.696000in}}%
\pgfusepath{clip}%
\pgfsetrectcap%
\pgfsetroundjoin%
\pgfsetlinewidth{1.505625pt}%
\definecolor{currentstroke}{rgb}{0.270588,0.270588,0.270588}%
\pgfsetstrokecolor{currentstroke}%
\pgfsetdash{}{0pt}%
\pgfpathmoveto{\pgfqpoint{1.693887in}{1.011532in}}%
\pgfpathlineto{\pgfqpoint{1.760020in}{1.011532in}}%
\pgfusepath{stroke}%
\end{pgfscope}%
\begin{pgfscope}%
\pgfpathrectangle{\pgfqpoint{0.652287in}{0.740652in}}{\pgfqpoint{4.960000in}{3.696000in}}%
\pgfusepath{clip}%
\pgfsetrectcap%
\pgfsetroundjoin%
\pgfsetlinewidth{1.505625pt}%
\definecolor{currentstroke}{rgb}{0.270588,0.270588,0.270588}%
\pgfsetstrokecolor{currentstroke}%
\pgfsetdash{}{0pt}%
\pgfpathmoveto{\pgfqpoint{1.892287in}{0.830465in}}%
\pgfpathlineto{\pgfqpoint{1.892287in}{0.770700in}}%
\pgfusepath{stroke}%
\end{pgfscope}%
\begin{pgfscope}%
\pgfpathrectangle{\pgfqpoint{0.652287in}{0.740652in}}{\pgfqpoint{4.960000in}{3.696000in}}%
\pgfusepath{clip}%
\pgfsetrectcap%
\pgfsetroundjoin%
\pgfsetlinewidth{1.505625pt}%
\definecolor{currentstroke}{rgb}{0.270588,0.270588,0.270588}%
\pgfsetstrokecolor{currentstroke}%
\pgfsetdash{}{0pt}%
\pgfpathmoveto{\pgfqpoint{1.892287in}{0.965997in}}%
\pgfpathlineto{\pgfqpoint{1.892287in}{1.028755in}}%
\pgfusepath{stroke}%
\end{pgfscope}%
\begin{pgfscope}%
\pgfpathrectangle{\pgfqpoint{0.652287in}{0.740652in}}{\pgfqpoint{4.960000in}{3.696000in}}%
\pgfusepath{clip}%
\pgfsetrectcap%
\pgfsetroundjoin%
\pgfsetlinewidth{1.505625pt}%
\definecolor{currentstroke}{rgb}{0.270588,0.270588,0.270588}%
\pgfsetstrokecolor{currentstroke}%
\pgfsetdash{}{0pt}%
\pgfpathmoveto{\pgfqpoint{1.859220in}{0.770700in}}%
\pgfpathlineto{\pgfqpoint{1.925353in}{0.770700in}}%
\pgfusepath{stroke}%
\end{pgfscope}%
\begin{pgfscope}%
\pgfpathrectangle{\pgfqpoint{0.652287in}{0.740652in}}{\pgfqpoint{4.960000in}{3.696000in}}%
\pgfusepath{clip}%
\pgfsetrectcap%
\pgfsetroundjoin%
\pgfsetlinewidth{1.505625pt}%
\definecolor{currentstroke}{rgb}{0.270588,0.270588,0.270588}%
\pgfsetstrokecolor{currentstroke}%
\pgfsetdash{}{0pt}%
\pgfpathmoveto{\pgfqpoint{1.859220in}{1.028755in}}%
\pgfpathlineto{\pgfqpoint{1.925353in}{1.028755in}}%
\pgfusepath{stroke}%
\end{pgfscope}%
\begin{pgfscope}%
\pgfpathrectangle{\pgfqpoint{0.652287in}{0.740652in}}{\pgfqpoint{4.960000in}{3.696000in}}%
\pgfusepath{clip}%
\pgfsetrectcap%
\pgfsetroundjoin%
\pgfsetlinewidth{1.505625pt}%
\definecolor{currentstroke}{rgb}{0.270588,0.270588,0.270588}%
\pgfsetstrokecolor{currentstroke}%
\pgfsetdash{}{0pt}%
\pgfpathmoveto{\pgfqpoint{2.057620in}{0.823442in}}%
\pgfpathlineto{\pgfqpoint{2.057620in}{0.766635in}}%
\pgfusepath{stroke}%
\end{pgfscope}%
\begin{pgfscope}%
\pgfpathrectangle{\pgfqpoint{0.652287in}{0.740652in}}{\pgfqpoint{4.960000in}{3.696000in}}%
\pgfusepath{clip}%
\pgfsetrectcap%
\pgfsetroundjoin%
\pgfsetlinewidth{1.505625pt}%
\definecolor{currentstroke}{rgb}{0.270588,0.270588,0.270588}%
\pgfsetstrokecolor{currentstroke}%
\pgfsetdash{}{0pt}%
\pgfpathmoveto{\pgfqpoint{2.057620in}{0.936023in}}%
\pgfpathlineto{\pgfqpoint{2.057620in}{1.027831in}}%
\pgfusepath{stroke}%
\end{pgfscope}%
\begin{pgfscope}%
\pgfpathrectangle{\pgfqpoint{0.652287in}{0.740652in}}{\pgfqpoint{4.960000in}{3.696000in}}%
\pgfusepath{clip}%
\pgfsetrectcap%
\pgfsetroundjoin%
\pgfsetlinewidth{1.505625pt}%
\definecolor{currentstroke}{rgb}{0.270588,0.270588,0.270588}%
\pgfsetstrokecolor{currentstroke}%
\pgfsetdash{}{0pt}%
\pgfpathmoveto{\pgfqpoint{2.024553in}{0.766635in}}%
\pgfpathlineto{\pgfqpoint{2.090687in}{0.766635in}}%
\pgfusepath{stroke}%
\end{pgfscope}%
\begin{pgfscope}%
\pgfpathrectangle{\pgfqpoint{0.652287in}{0.740652in}}{\pgfqpoint{4.960000in}{3.696000in}}%
\pgfusepath{clip}%
\pgfsetrectcap%
\pgfsetroundjoin%
\pgfsetlinewidth{1.505625pt}%
\definecolor{currentstroke}{rgb}{0.270588,0.270588,0.270588}%
\pgfsetstrokecolor{currentstroke}%
\pgfsetdash{}{0pt}%
\pgfpathmoveto{\pgfqpoint{2.024553in}{1.027831in}}%
\pgfpathlineto{\pgfqpoint{2.090687in}{1.027831in}}%
\pgfusepath{stroke}%
\end{pgfscope}%
\begin{pgfscope}%
\pgfpathrectangle{\pgfqpoint{0.652287in}{0.740652in}}{\pgfqpoint{4.960000in}{3.696000in}}%
\pgfusepath{clip}%
\pgfsetrectcap%
\pgfsetroundjoin%
\pgfsetlinewidth{1.505625pt}%
\definecolor{currentstroke}{rgb}{0.270588,0.270588,0.270588}%
\pgfsetstrokecolor{currentstroke}%
\pgfsetdash{}{0pt}%
\pgfpathmoveto{\pgfqpoint{2.222953in}{0.827508in}}%
\pgfpathlineto{\pgfqpoint{2.222953in}{0.765674in}}%
\pgfusepath{stroke}%
\end{pgfscope}%
\begin{pgfscope}%
\pgfpathrectangle{\pgfqpoint{0.652287in}{0.740652in}}{\pgfqpoint{4.960000in}{3.696000in}}%
\pgfusepath{clip}%
\pgfsetrectcap%
\pgfsetroundjoin%
\pgfsetlinewidth{1.505625pt}%
\definecolor{currentstroke}{rgb}{0.270588,0.270588,0.270588}%
\pgfsetstrokecolor{currentstroke}%
\pgfsetdash{}{0pt}%
\pgfpathmoveto{\pgfqpoint{2.222953in}{0.957164in}}%
\pgfpathlineto{\pgfqpoint{2.222953in}{1.034632in}}%
\pgfusepath{stroke}%
\end{pgfscope}%
\begin{pgfscope}%
\pgfpathrectangle{\pgfqpoint{0.652287in}{0.740652in}}{\pgfqpoint{4.960000in}{3.696000in}}%
\pgfusepath{clip}%
\pgfsetrectcap%
\pgfsetroundjoin%
\pgfsetlinewidth{1.505625pt}%
\definecolor{currentstroke}{rgb}{0.270588,0.270588,0.270588}%
\pgfsetstrokecolor{currentstroke}%
\pgfsetdash{}{0pt}%
\pgfpathmoveto{\pgfqpoint{2.189887in}{0.765674in}}%
\pgfpathlineto{\pgfqpoint{2.256020in}{0.765674in}}%
\pgfusepath{stroke}%
\end{pgfscope}%
\begin{pgfscope}%
\pgfpathrectangle{\pgfqpoint{0.652287in}{0.740652in}}{\pgfqpoint{4.960000in}{3.696000in}}%
\pgfusepath{clip}%
\pgfsetrectcap%
\pgfsetroundjoin%
\pgfsetlinewidth{1.505625pt}%
\definecolor{currentstroke}{rgb}{0.270588,0.270588,0.270588}%
\pgfsetstrokecolor{currentstroke}%
\pgfsetdash{}{0pt}%
\pgfpathmoveto{\pgfqpoint{2.189887in}{1.034632in}}%
\pgfpathlineto{\pgfqpoint{2.256020in}{1.034632in}}%
\pgfusepath{stroke}%
\end{pgfscope}%
\begin{pgfscope}%
\pgfpathrectangle{\pgfqpoint{0.652287in}{0.740652in}}{\pgfqpoint{4.960000in}{3.696000in}}%
\pgfusepath{clip}%
\pgfsetrectcap%
\pgfsetroundjoin%
\pgfsetlinewidth{1.505625pt}%
\definecolor{currentstroke}{rgb}{0.270588,0.270588,0.270588}%
\pgfsetstrokecolor{currentstroke}%
\pgfsetdash{}{0pt}%
\pgfpathmoveto{\pgfqpoint{2.388287in}{0.891966in}}%
\pgfpathlineto{\pgfqpoint{2.388287in}{0.774692in}}%
\pgfusepath{stroke}%
\end{pgfscope}%
\begin{pgfscope}%
\pgfpathrectangle{\pgfqpoint{0.652287in}{0.740652in}}{\pgfqpoint{4.960000in}{3.696000in}}%
\pgfusepath{clip}%
\pgfsetrectcap%
\pgfsetroundjoin%
\pgfsetlinewidth{1.505625pt}%
\definecolor{currentstroke}{rgb}{0.270588,0.270588,0.270588}%
\pgfsetstrokecolor{currentstroke}%
\pgfsetdash{}{0pt}%
\pgfpathmoveto{\pgfqpoint{2.388287in}{1.175126in}}%
\pgfpathlineto{\pgfqpoint{2.388287in}{1.325840in}}%
\pgfusepath{stroke}%
\end{pgfscope}%
\begin{pgfscope}%
\pgfpathrectangle{\pgfqpoint{0.652287in}{0.740652in}}{\pgfqpoint{4.960000in}{3.696000in}}%
\pgfusepath{clip}%
\pgfsetrectcap%
\pgfsetroundjoin%
\pgfsetlinewidth{1.505625pt}%
\definecolor{currentstroke}{rgb}{0.270588,0.270588,0.270588}%
\pgfsetstrokecolor{currentstroke}%
\pgfsetdash{}{0pt}%
\pgfpathmoveto{\pgfqpoint{2.355220in}{0.774692in}}%
\pgfpathlineto{\pgfqpoint{2.421353in}{0.774692in}}%
\pgfusepath{stroke}%
\end{pgfscope}%
\begin{pgfscope}%
\pgfpathrectangle{\pgfqpoint{0.652287in}{0.740652in}}{\pgfqpoint{4.960000in}{3.696000in}}%
\pgfusepath{clip}%
\pgfsetrectcap%
\pgfsetroundjoin%
\pgfsetlinewidth{1.505625pt}%
\definecolor{currentstroke}{rgb}{0.270588,0.270588,0.270588}%
\pgfsetstrokecolor{currentstroke}%
\pgfsetdash{}{0pt}%
\pgfpathmoveto{\pgfqpoint{2.355220in}{1.325840in}}%
\pgfpathlineto{\pgfqpoint{2.421353in}{1.325840in}}%
\pgfusepath{stroke}%
\end{pgfscope}%
\begin{pgfscope}%
\pgfpathrectangle{\pgfqpoint{0.652287in}{0.740652in}}{\pgfqpoint{4.960000in}{3.696000in}}%
\pgfusepath{clip}%
\pgfsetrectcap%
\pgfsetroundjoin%
\pgfsetlinewidth{1.505625pt}%
\definecolor{currentstroke}{rgb}{0.270588,0.270588,0.270588}%
\pgfsetstrokecolor{currentstroke}%
\pgfsetdash{}{0pt}%
\pgfpathmoveto{\pgfqpoint{2.553620in}{0.923604in}}%
\pgfpathlineto{\pgfqpoint{2.553620in}{0.776873in}}%
\pgfusepath{stroke}%
\end{pgfscope}%
\begin{pgfscope}%
\pgfpathrectangle{\pgfqpoint{0.652287in}{0.740652in}}{\pgfqpoint{4.960000in}{3.696000in}}%
\pgfusepath{clip}%
\pgfsetrectcap%
\pgfsetroundjoin%
\pgfsetlinewidth{1.505625pt}%
\definecolor{currentstroke}{rgb}{0.270588,0.270588,0.270588}%
\pgfsetstrokecolor{currentstroke}%
\pgfsetdash{}{0pt}%
\pgfpathmoveto{\pgfqpoint{2.553620in}{1.363742in}}%
\pgfpathlineto{\pgfqpoint{2.553620in}{1.619265in}}%
\pgfusepath{stroke}%
\end{pgfscope}%
\begin{pgfscope}%
\pgfpathrectangle{\pgfqpoint{0.652287in}{0.740652in}}{\pgfqpoint{4.960000in}{3.696000in}}%
\pgfusepath{clip}%
\pgfsetrectcap%
\pgfsetroundjoin%
\pgfsetlinewidth{1.505625pt}%
\definecolor{currentstroke}{rgb}{0.270588,0.270588,0.270588}%
\pgfsetstrokecolor{currentstroke}%
\pgfsetdash{}{0pt}%
\pgfpathmoveto{\pgfqpoint{2.520553in}{0.776873in}}%
\pgfpathlineto{\pgfqpoint{2.586687in}{0.776873in}}%
\pgfusepath{stroke}%
\end{pgfscope}%
\begin{pgfscope}%
\pgfpathrectangle{\pgfqpoint{0.652287in}{0.740652in}}{\pgfqpoint{4.960000in}{3.696000in}}%
\pgfusepath{clip}%
\pgfsetrectcap%
\pgfsetroundjoin%
\pgfsetlinewidth{1.505625pt}%
\definecolor{currentstroke}{rgb}{0.270588,0.270588,0.270588}%
\pgfsetstrokecolor{currentstroke}%
\pgfsetdash{}{0pt}%
\pgfpathmoveto{\pgfqpoint{2.520553in}{1.619265in}}%
\pgfpathlineto{\pgfqpoint{2.586687in}{1.619265in}}%
\pgfusepath{stroke}%
\end{pgfscope}%
\begin{pgfscope}%
\pgfpathrectangle{\pgfqpoint{0.652287in}{0.740652in}}{\pgfqpoint{4.960000in}{3.696000in}}%
\pgfusepath{clip}%
\pgfsetrectcap%
\pgfsetroundjoin%
\pgfsetlinewidth{1.505625pt}%
\definecolor{currentstroke}{rgb}{0.270588,0.270588,0.270588}%
\pgfsetstrokecolor{currentstroke}%
\pgfsetdash{}{0pt}%
\pgfpathmoveto{\pgfqpoint{2.718953in}{0.883170in}}%
\pgfpathlineto{\pgfqpoint{2.718953in}{0.764787in}}%
\pgfusepath{stroke}%
\end{pgfscope}%
\begin{pgfscope}%
\pgfpathrectangle{\pgfqpoint{0.652287in}{0.740652in}}{\pgfqpoint{4.960000in}{3.696000in}}%
\pgfusepath{clip}%
\pgfsetrectcap%
\pgfsetroundjoin%
\pgfsetlinewidth{1.505625pt}%
\definecolor{currentstroke}{rgb}{0.270588,0.270588,0.270588}%
\pgfsetstrokecolor{currentstroke}%
\pgfsetdash{}{0pt}%
\pgfpathmoveto{\pgfqpoint{2.718953in}{1.172955in}}%
\pgfpathlineto{\pgfqpoint{2.718953in}{1.346057in}}%
\pgfusepath{stroke}%
\end{pgfscope}%
\begin{pgfscope}%
\pgfpathrectangle{\pgfqpoint{0.652287in}{0.740652in}}{\pgfqpoint{4.960000in}{3.696000in}}%
\pgfusepath{clip}%
\pgfsetrectcap%
\pgfsetroundjoin%
\pgfsetlinewidth{1.505625pt}%
\definecolor{currentstroke}{rgb}{0.270588,0.270588,0.270588}%
\pgfsetstrokecolor{currentstroke}%
\pgfsetdash{}{0pt}%
\pgfpathmoveto{\pgfqpoint{2.685887in}{0.764787in}}%
\pgfpathlineto{\pgfqpoint{2.752020in}{0.764787in}}%
\pgfusepath{stroke}%
\end{pgfscope}%
\begin{pgfscope}%
\pgfpathrectangle{\pgfqpoint{0.652287in}{0.740652in}}{\pgfqpoint{4.960000in}{3.696000in}}%
\pgfusepath{clip}%
\pgfsetrectcap%
\pgfsetroundjoin%
\pgfsetlinewidth{1.505625pt}%
\definecolor{currentstroke}{rgb}{0.270588,0.270588,0.270588}%
\pgfsetstrokecolor{currentstroke}%
\pgfsetdash{}{0pt}%
\pgfpathmoveto{\pgfqpoint{2.685887in}{1.346057in}}%
\pgfpathlineto{\pgfqpoint{2.752020in}{1.346057in}}%
\pgfusepath{stroke}%
\end{pgfscope}%
\begin{pgfscope}%
\pgfpathrectangle{\pgfqpoint{0.652287in}{0.740652in}}{\pgfqpoint{4.960000in}{3.696000in}}%
\pgfusepath{clip}%
\pgfsetrectcap%
\pgfsetroundjoin%
\pgfsetlinewidth{1.505625pt}%
\definecolor{currentstroke}{rgb}{0.270588,0.270588,0.270588}%
\pgfsetstrokecolor{currentstroke}%
\pgfsetdash{}{0pt}%
\pgfpathmoveto{\pgfqpoint{2.884287in}{0.940910in}}%
\pgfpathlineto{\pgfqpoint{2.884287in}{0.772585in}}%
\pgfusepath{stroke}%
\end{pgfscope}%
\begin{pgfscope}%
\pgfpathrectangle{\pgfqpoint{0.652287in}{0.740652in}}{\pgfqpoint{4.960000in}{3.696000in}}%
\pgfusepath{clip}%
\pgfsetrectcap%
\pgfsetroundjoin%
\pgfsetlinewidth{1.505625pt}%
\definecolor{currentstroke}{rgb}{0.270588,0.270588,0.270588}%
\pgfsetstrokecolor{currentstroke}%
\pgfsetdash{}{0pt}%
\pgfpathmoveto{\pgfqpoint{2.884287in}{1.376068in}}%
\pgfpathlineto{\pgfqpoint{2.884287in}{1.948542in}}%
\pgfusepath{stroke}%
\end{pgfscope}%
\begin{pgfscope}%
\pgfpathrectangle{\pgfqpoint{0.652287in}{0.740652in}}{\pgfqpoint{4.960000in}{3.696000in}}%
\pgfusepath{clip}%
\pgfsetrectcap%
\pgfsetroundjoin%
\pgfsetlinewidth{1.505625pt}%
\definecolor{currentstroke}{rgb}{0.270588,0.270588,0.270588}%
\pgfsetstrokecolor{currentstroke}%
\pgfsetdash{}{0pt}%
\pgfpathmoveto{\pgfqpoint{2.851220in}{0.772585in}}%
\pgfpathlineto{\pgfqpoint{2.917353in}{0.772585in}}%
\pgfusepath{stroke}%
\end{pgfscope}%
\begin{pgfscope}%
\pgfpathrectangle{\pgfqpoint{0.652287in}{0.740652in}}{\pgfqpoint{4.960000in}{3.696000in}}%
\pgfusepath{clip}%
\pgfsetrectcap%
\pgfsetroundjoin%
\pgfsetlinewidth{1.505625pt}%
\definecolor{currentstroke}{rgb}{0.270588,0.270588,0.270588}%
\pgfsetstrokecolor{currentstroke}%
\pgfsetdash{}{0pt}%
\pgfpathmoveto{\pgfqpoint{2.851220in}{1.948542in}}%
\pgfpathlineto{\pgfqpoint{2.917353in}{1.948542in}}%
\pgfusepath{stroke}%
\end{pgfscope}%
\begin{pgfscope}%
\pgfpathrectangle{\pgfqpoint{0.652287in}{0.740652in}}{\pgfqpoint{4.960000in}{3.696000in}}%
\pgfusepath{clip}%
\pgfsetrectcap%
\pgfsetroundjoin%
\pgfsetlinewidth{1.505625pt}%
\definecolor{currentstroke}{rgb}{0.270588,0.270588,0.270588}%
\pgfsetstrokecolor{currentstroke}%
\pgfsetdash{}{0pt}%
\pgfpathmoveto{\pgfqpoint{3.049620in}{0.943562in}}%
\pgfpathlineto{\pgfqpoint{3.049620in}{0.776355in}}%
\pgfusepath{stroke}%
\end{pgfscope}%
\begin{pgfscope}%
\pgfpathrectangle{\pgfqpoint{0.652287in}{0.740652in}}{\pgfqpoint{4.960000in}{3.696000in}}%
\pgfusepath{clip}%
\pgfsetrectcap%
\pgfsetroundjoin%
\pgfsetlinewidth{1.505625pt}%
\definecolor{currentstroke}{rgb}{0.270588,0.270588,0.270588}%
\pgfsetstrokecolor{currentstroke}%
\pgfsetdash{}{0pt}%
\pgfpathmoveto{\pgfqpoint{3.049620in}{1.358512in}}%
\pgfpathlineto{\pgfqpoint{3.049620in}{1.597828in}}%
\pgfusepath{stroke}%
\end{pgfscope}%
\begin{pgfscope}%
\pgfpathrectangle{\pgfqpoint{0.652287in}{0.740652in}}{\pgfqpoint{4.960000in}{3.696000in}}%
\pgfusepath{clip}%
\pgfsetrectcap%
\pgfsetroundjoin%
\pgfsetlinewidth{1.505625pt}%
\definecolor{currentstroke}{rgb}{0.270588,0.270588,0.270588}%
\pgfsetstrokecolor{currentstroke}%
\pgfsetdash{}{0pt}%
\pgfpathmoveto{\pgfqpoint{3.016553in}{0.776355in}}%
\pgfpathlineto{\pgfqpoint{3.082687in}{0.776355in}}%
\pgfusepath{stroke}%
\end{pgfscope}%
\begin{pgfscope}%
\pgfpathrectangle{\pgfqpoint{0.652287in}{0.740652in}}{\pgfqpoint{4.960000in}{3.696000in}}%
\pgfusepath{clip}%
\pgfsetrectcap%
\pgfsetroundjoin%
\pgfsetlinewidth{1.505625pt}%
\definecolor{currentstroke}{rgb}{0.270588,0.270588,0.270588}%
\pgfsetstrokecolor{currentstroke}%
\pgfsetdash{}{0pt}%
\pgfpathmoveto{\pgfqpoint{3.016553in}{1.597828in}}%
\pgfpathlineto{\pgfqpoint{3.082687in}{1.597828in}}%
\pgfusepath{stroke}%
\end{pgfscope}%
\begin{pgfscope}%
\pgfpathrectangle{\pgfqpoint{0.652287in}{0.740652in}}{\pgfqpoint{4.960000in}{3.696000in}}%
\pgfusepath{clip}%
\pgfsetrectcap%
\pgfsetroundjoin%
\pgfsetlinewidth{1.505625pt}%
\definecolor{currentstroke}{rgb}{0.270588,0.270588,0.270588}%
\pgfsetstrokecolor{currentstroke}%
\pgfsetdash{}{0pt}%
\pgfpathmoveto{\pgfqpoint{3.214953in}{0.968991in}}%
\pgfpathlineto{\pgfqpoint{3.214953in}{0.783563in}}%
\pgfusepath{stroke}%
\end{pgfscope}%
\begin{pgfscope}%
\pgfpathrectangle{\pgfqpoint{0.652287in}{0.740652in}}{\pgfqpoint{4.960000in}{3.696000in}}%
\pgfusepath{clip}%
\pgfsetrectcap%
\pgfsetroundjoin%
\pgfsetlinewidth{1.505625pt}%
\definecolor{currentstroke}{rgb}{0.270588,0.270588,0.270588}%
\pgfsetstrokecolor{currentstroke}%
\pgfsetdash{}{0pt}%
\pgfpathmoveto{\pgfqpoint{3.214953in}{1.433818in}}%
\pgfpathlineto{\pgfqpoint{3.214953in}{1.689046in}}%
\pgfusepath{stroke}%
\end{pgfscope}%
\begin{pgfscope}%
\pgfpathrectangle{\pgfqpoint{0.652287in}{0.740652in}}{\pgfqpoint{4.960000in}{3.696000in}}%
\pgfusepath{clip}%
\pgfsetrectcap%
\pgfsetroundjoin%
\pgfsetlinewidth{1.505625pt}%
\definecolor{currentstroke}{rgb}{0.270588,0.270588,0.270588}%
\pgfsetstrokecolor{currentstroke}%
\pgfsetdash{}{0pt}%
\pgfpathmoveto{\pgfqpoint{3.181887in}{0.783563in}}%
\pgfpathlineto{\pgfqpoint{3.248020in}{0.783563in}}%
\pgfusepath{stroke}%
\end{pgfscope}%
\begin{pgfscope}%
\pgfpathrectangle{\pgfqpoint{0.652287in}{0.740652in}}{\pgfqpoint{4.960000in}{3.696000in}}%
\pgfusepath{clip}%
\pgfsetrectcap%
\pgfsetroundjoin%
\pgfsetlinewidth{1.505625pt}%
\definecolor{currentstroke}{rgb}{0.270588,0.270588,0.270588}%
\pgfsetstrokecolor{currentstroke}%
\pgfsetdash{}{0pt}%
\pgfpathmoveto{\pgfqpoint{3.181887in}{1.689046in}}%
\pgfpathlineto{\pgfqpoint{3.248020in}{1.689046in}}%
\pgfusepath{stroke}%
\end{pgfscope}%
\begin{pgfscope}%
\pgfpathrectangle{\pgfqpoint{0.652287in}{0.740652in}}{\pgfqpoint{4.960000in}{3.696000in}}%
\pgfusepath{clip}%
\pgfsetrectcap%
\pgfsetroundjoin%
\pgfsetlinewidth{1.505625pt}%
\definecolor{currentstroke}{rgb}{0.270588,0.270588,0.270588}%
\pgfsetstrokecolor{currentstroke}%
\pgfsetdash{}{0pt}%
\pgfpathmoveto{\pgfqpoint{3.380287in}{1.022019in}}%
\pgfpathlineto{\pgfqpoint{3.380287in}{0.798014in}}%
\pgfusepath{stroke}%
\end{pgfscope}%
\begin{pgfscope}%
\pgfpathrectangle{\pgfqpoint{0.652287in}{0.740652in}}{\pgfqpoint{4.960000in}{3.696000in}}%
\pgfusepath{clip}%
\pgfsetrectcap%
\pgfsetroundjoin%
\pgfsetlinewidth{1.505625pt}%
\definecolor{currentstroke}{rgb}{0.270588,0.270588,0.270588}%
\pgfsetstrokecolor{currentstroke}%
\pgfsetdash{}{0pt}%
\pgfpathmoveto{\pgfqpoint{3.380287in}{1.579071in}}%
\pgfpathlineto{\pgfqpoint{3.380287in}{1.886523in}}%
\pgfusepath{stroke}%
\end{pgfscope}%
\begin{pgfscope}%
\pgfpathrectangle{\pgfqpoint{0.652287in}{0.740652in}}{\pgfqpoint{4.960000in}{3.696000in}}%
\pgfusepath{clip}%
\pgfsetrectcap%
\pgfsetroundjoin%
\pgfsetlinewidth{1.505625pt}%
\definecolor{currentstroke}{rgb}{0.270588,0.270588,0.270588}%
\pgfsetstrokecolor{currentstroke}%
\pgfsetdash{}{0pt}%
\pgfpathmoveto{\pgfqpoint{3.347220in}{0.798014in}}%
\pgfpathlineto{\pgfqpoint{3.413353in}{0.798014in}}%
\pgfusepath{stroke}%
\end{pgfscope}%
\begin{pgfscope}%
\pgfpathrectangle{\pgfqpoint{0.652287in}{0.740652in}}{\pgfqpoint{4.960000in}{3.696000in}}%
\pgfusepath{clip}%
\pgfsetrectcap%
\pgfsetroundjoin%
\pgfsetlinewidth{1.505625pt}%
\definecolor{currentstroke}{rgb}{0.270588,0.270588,0.270588}%
\pgfsetstrokecolor{currentstroke}%
\pgfsetdash{}{0pt}%
\pgfpathmoveto{\pgfqpoint{3.347220in}{1.886523in}}%
\pgfpathlineto{\pgfqpoint{3.413353in}{1.886523in}}%
\pgfusepath{stroke}%
\end{pgfscope}%
\begin{pgfscope}%
\pgfpathrectangle{\pgfqpoint{0.652287in}{0.740652in}}{\pgfqpoint{4.960000in}{3.696000in}}%
\pgfusepath{clip}%
\pgfsetrectcap%
\pgfsetroundjoin%
\pgfsetlinewidth{1.505625pt}%
\definecolor{currentstroke}{rgb}{0.270588,0.270588,0.270588}%
\pgfsetstrokecolor{currentstroke}%
\pgfsetdash{}{0pt}%
\pgfpathmoveto{\pgfqpoint{3.545620in}{1.054331in}}%
\pgfpathlineto{\pgfqpoint{3.545620in}{0.802153in}}%
\pgfusepath{stroke}%
\end{pgfscope}%
\begin{pgfscope}%
\pgfpathrectangle{\pgfqpoint{0.652287in}{0.740652in}}{\pgfqpoint{4.960000in}{3.696000in}}%
\pgfusepath{clip}%
\pgfsetrectcap%
\pgfsetroundjoin%
\pgfsetlinewidth{1.505625pt}%
\definecolor{currentstroke}{rgb}{0.270588,0.270588,0.270588}%
\pgfsetstrokecolor{currentstroke}%
\pgfsetdash{}{0pt}%
\pgfpathmoveto{\pgfqpoint{3.545620in}{1.686070in}}%
\pgfpathlineto{\pgfqpoint{3.545620in}{2.035472in}}%
\pgfusepath{stroke}%
\end{pgfscope}%
\begin{pgfscope}%
\pgfpathrectangle{\pgfqpoint{0.652287in}{0.740652in}}{\pgfqpoint{4.960000in}{3.696000in}}%
\pgfusepath{clip}%
\pgfsetrectcap%
\pgfsetroundjoin%
\pgfsetlinewidth{1.505625pt}%
\definecolor{currentstroke}{rgb}{0.270588,0.270588,0.270588}%
\pgfsetstrokecolor{currentstroke}%
\pgfsetdash{}{0pt}%
\pgfpathmoveto{\pgfqpoint{3.512553in}{0.802153in}}%
\pgfpathlineto{\pgfqpoint{3.578687in}{0.802153in}}%
\pgfusepath{stroke}%
\end{pgfscope}%
\begin{pgfscope}%
\pgfpathrectangle{\pgfqpoint{0.652287in}{0.740652in}}{\pgfqpoint{4.960000in}{3.696000in}}%
\pgfusepath{clip}%
\pgfsetrectcap%
\pgfsetroundjoin%
\pgfsetlinewidth{1.505625pt}%
\definecolor{currentstroke}{rgb}{0.270588,0.270588,0.270588}%
\pgfsetstrokecolor{currentstroke}%
\pgfsetdash{}{0pt}%
\pgfpathmoveto{\pgfqpoint{3.512553in}{2.035472in}}%
\pgfpathlineto{\pgfqpoint{3.578687in}{2.035472in}}%
\pgfusepath{stroke}%
\end{pgfscope}%
\begin{pgfscope}%
\pgfpathrectangle{\pgfqpoint{0.652287in}{0.740652in}}{\pgfqpoint{4.960000in}{3.696000in}}%
\pgfusepath{clip}%
\pgfsetrectcap%
\pgfsetroundjoin%
\pgfsetlinewidth{1.505625pt}%
\definecolor{currentstroke}{rgb}{0.270588,0.270588,0.270588}%
\pgfsetstrokecolor{currentstroke}%
\pgfsetdash{}{0pt}%
\pgfpathmoveto{\pgfqpoint{3.710953in}{1.092437in}}%
\pgfpathlineto{\pgfqpoint{3.710953in}{0.820892in}}%
\pgfusepath{stroke}%
\end{pgfscope}%
\begin{pgfscope}%
\pgfpathrectangle{\pgfqpoint{0.652287in}{0.740652in}}{\pgfqpoint{4.960000in}{3.696000in}}%
\pgfusepath{clip}%
\pgfsetrectcap%
\pgfsetroundjoin%
\pgfsetlinewidth{1.505625pt}%
\definecolor{currentstroke}{rgb}{0.270588,0.270588,0.270588}%
\pgfsetstrokecolor{currentstroke}%
\pgfsetdash{}{0pt}%
\pgfpathmoveto{\pgfqpoint{3.710953in}{1.760434in}}%
\pgfpathlineto{\pgfqpoint{3.710953in}{2.130791in}}%
\pgfusepath{stroke}%
\end{pgfscope}%
\begin{pgfscope}%
\pgfpathrectangle{\pgfqpoint{0.652287in}{0.740652in}}{\pgfqpoint{4.960000in}{3.696000in}}%
\pgfusepath{clip}%
\pgfsetrectcap%
\pgfsetroundjoin%
\pgfsetlinewidth{1.505625pt}%
\definecolor{currentstroke}{rgb}{0.270588,0.270588,0.270588}%
\pgfsetstrokecolor{currentstroke}%
\pgfsetdash{}{0pt}%
\pgfpathmoveto{\pgfqpoint{3.677887in}{0.820892in}}%
\pgfpathlineto{\pgfqpoint{3.744020in}{0.820892in}}%
\pgfusepath{stroke}%
\end{pgfscope}%
\begin{pgfscope}%
\pgfpathrectangle{\pgfqpoint{0.652287in}{0.740652in}}{\pgfqpoint{4.960000in}{3.696000in}}%
\pgfusepath{clip}%
\pgfsetrectcap%
\pgfsetroundjoin%
\pgfsetlinewidth{1.505625pt}%
\definecolor{currentstroke}{rgb}{0.270588,0.270588,0.270588}%
\pgfsetstrokecolor{currentstroke}%
\pgfsetdash{}{0pt}%
\pgfpathmoveto{\pgfqpoint{3.677887in}{2.130791in}}%
\pgfpathlineto{\pgfqpoint{3.744020in}{2.130791in}}%
\pgfusepath{stroke}%
\end{pgfscope}%
\begin{pgfscope}%
\pgfpathrectangle{\pgfqpoint{0.652287in}{0.740652in}}{\pgfqpoint{4.960000in}{3.696000in}}%
\pgfusepath{clip}%
\pgfsetrectcap%
\pgfsetroundjoin%
\pgfsetlinewidth{1.505625pt}%
\definecolor{currentstroke}{rgb}{0.270588,0.270588,0.270588}%
\pgfsetstrokecolor{currentstroke}%
\pgfsetdash{}{0pt}%
\pgfpathmoveto{\pgfqpoint{3.876287in}{1.137048in}}%
\pgfpathlineto{\pgfqpoint{3.876287in}{0.836304in}}%
\pgfusepath{stroke}%
\end{pgfscope}%
\begin{pgfscope}%
\pgfpathrectangle{\pgfqpoint{0.652287in}{0.740652in}}{\pgfqpoint{4.960000in}{3.696000in}}%
\pgfusepath{clip}%
\pgfsetrectcap%
\pgfsetroundjoin%
\pgfsetlinewidth{1.505625pt}%
\definecolor{currentstroke}{rgb}{0.270588,0.270588,0.270588}%
\pgfsetstrokecolor{currentstroke}%
\pgfsetdash{}{0pt}%
\pgfpathmoveto{\pgfqpoint{3.876287in}{1.873245in}}%
\pgfpathlineto{\pgfqpoint{3.876287in}{2.260336in}}%
\pgfusepath{stroke}%
\end{pgfscope}%
\begin{pgfscope}%
\pgfpathrectangle{\pgfqpoint{0.652287in}{0.740652in}}{\pgfqpoint{4.960000in}{3.696000in}}%
\pgfusepath{clip}%
\pgfsetrectcap%
\pgfsetroundjoin%
\pgfsetlinewidth{1.505625pt}%
\definecolor{currentstroke}{rgb}{0.270588,0.270588,0.270588}%
\pgfsetstrokecolor{currentstroke}%
\pgfsetdash{}{0pt}%
\pgfpathmoveto{\pgfqpoint{3.843220in}{0.836304in}}%
\pgfpathlineto{\pgfqpoint{3.909353in}{0.836304in}}%
\pgfusepath{stroke}%
\end{pgfscope}%
\begin{pgfscope}%
\pgfpathrectangle{\pgfqpoint{0.652287in}{0.740652in}}{\pgfqpoint{4.960000in}{3.696000in}}%
\pgfusepath{clip}%
\pgfsetrectcap%
\pgfsetroundjoin%
\pgfsetlinewidth{1.505625pt}%
\definecolor{currentstroke}{rgb}{0.270588,0.270588,0.270588}%
\pgfsetstrokecolor{currentstroke}%
\pgfsetdash{}{0pt}%
\pgfpathmoveto{\pgfqpoint{3.843220in}{2.260336in}}%
\pgfpathlineto{\pgfqpoint{3.909353in}{2.260336in}}%
\pgfusepath{stroke}%
\end{pgfscope}%
\begin{pgfscope}%
\pgfpathrectangle{\pgfqpoint{0.652287in}{0.740652in}}{\pgfqpoint{4.960000in}{3.696000in}}%
\pgfusepath{clip}%
\pgfsetrectcap%
\pgfsetroundjoin%
\pgfsetlinewidth{1.505625pt}%
\definecolor{currentstroke}{rgb}{0.270588,0.270588,0.270588}%
\pgfsetstrokecolor{currentstroke}%
\pgfsetdash{}{0pt}%
\pgfpathmoveto{\pgfqpoint{4.041620in}{1.148783in}}%
\pgfpathlineto{\pgfqpoint{4.041620in}{0.813833in}}%
\pgfusepath{stroke}%
\end{pgfscope}%
\begin{pgfscope}%
\pgfpathrectangle{\pgfqpoint{0.652287in}{0.740652in}}{\pgfqpoint{4.960000in}{3.696000in}}%
\pgfusepath{clip}%
\pgfsetrectcap%
\pgfsetroundjoin%
\pgfsetlinewidth{1.505625pt}%
\definecolor{currentstroke}{rgb}{0.270588,0.270588,0.270588}%
\pgfsetstrokecolor{currentstroke}%
\pgfsetdash{}{0pt}%
\pgfpathmoveto{\pgfqpoint{4.041620in}{1.997902in}}%
\pgfpathlineto{\pgfqpoint{4.041620in}{2.435896in}}%
\pgfusepath{stroke}%
\end{pgfscope}%
\begin{pgfscope}%
\pgfpathrectangle{\pgfqpoint{0.652287in}{0.740652in}}{\pgfqpoint{4.960000in}{3.696000in}}%
\pgfusepath{clip}%
\pgfsetrectcap%
\pgfsetroundjoin%
\pgfsetlinewidth{1.505625pt}%
\definecolor{currentstroke}{rgb}{0.270588,0.270588,0.270588}%
\pgfsetstrokecolor{currentstroke}%
\pgfsetdash{}{0pt}%
\pgfpathmoveto{\pgfqpoint{4.008553in}{0.813833in}}%
\pgfpathlineto{\pgfqpoint{4.074687in}{0.813833in}}%
\pgfusepath{stroke}%
\end{pgfscope}%
\begin{pgfscope}%
\pgfpathrectangle{\pgfqpoint{0.652287in}{0.740652in}}{\pgfqpoint{4.960000in}{3.696000in}}%
\pgfusepath{clip}%
\pgfsetrectcap%
\pgfsetroundjoin%
\pgfsetlinewidth{1.505625pt}%
\definecolor{currentstroke}{rgb}{0.270588,0.270588,0.270588}%
\pgfsetstrokecolor{currentstroke}%
\pgfsetdash{}{0pt}%
\pgfpathmoveto{\pgfqpoint{4.008553in}{2.435896in}}%
\pgfpathlineto{\pgfqpoint{4.074687in}{2.435896in}}%
\pgfusepath{stroke}%
\end{pgfscope}%
\begin{pgfscope}%
\pgfpathrectangle{\pgfqpoint{0.652287in}{0.740652in}}{\pgfqpoint{4.960000in}{3.696000in}}%
\pgfusepath{clip}%
\pgfsetrectcap%
\pgfsetroundjoin%
\pgfsetlinewidth{1.505625pt}%
\definecolor{currentstroke}{rgb}{0.270588,0.270588,0.270588}%
\pgfsetstrokecolor{currentstroke}%
\pgfsetdash{}{0pt}%
\pgfpathmoveto{\pgfqpoint{4.206953in}{1.216734in}}%
\pgfpathlineto{\pgfqpoint{4.206953in}{0.853491in}}%
\pgfusepath{stroke}%
\end{pgfscope}%
\begin{pgfscope}%
\pgfpathrectangle{\pgfqpoint{0.652287in}{0.740652in}}{\pgfqpoint{4.960000in}{3.696000in}}%
\pgfusepath{clip}%
\pgfsetrectcap%
\pgfsetroundjoin%
\pgfsetlinewidth{1.505625pt}%
\definecolor{currentstroke}{rgb}{0.270588,0.270588,0.270588}%
\pgfsetstrokecolor{currentstroke}%
\pgfsetdash{}{0pt}%
\pgfpathmoveto{\pgfqpoint{4.206953in}{2.097306in}}%
\pgfpathlineto{\pgfqpoint{4.206953in}{2.566772in}}%
\pgfusepath{stroke}%
\end{pgfscope}%
\begin{pgfscope}%
\pgfpathrectangle{\pgfqpoint{0.652287in}{0.740652in}}{\pgfqpoint{4.960000in}{3.696000in}}%
\pgfusepath{clip}%
\pgfsetrectcap%
\pgfsetroundjoin%
\pgfsetlinewidth{1.505625pt}%
\definecolor{currentstroke}{rgb}{0.270588,0.270588,0.270588}%
\pgfsetstrokecolor{currentstroke}%
\pgfsetdash{}{0pt}%
\pgfpathmoveto{\pgfqpoint{4.173887in}{0.853491in}}%
\pgfpathlineto{\pgfqpoint{4.240020in}{0.853491in}}%
\pgfusepath{stroke}%
\end{pgfscope}%
\begin{pgfscope}%
\pgfpathrectangle{\pgfqpoint{0.652287in}{0.740652in}}{\pgfqpoint{4.960000in}{3.696000in}}%
\pgfusepath{clip}%
\pgfsetrectcap%
\pgfsetroundjoin%
\pgfsetlinewidth{1.505625pt}%
\definecolor{currentstroke}{rgb}{0.270588,0.270588,0.270588}%
\pgfsetstrokecolor{currentstroke}%
\pgfsetdash{}{0pt}%
\pgfpathmoveto{\pgfqpoint{4.173887in}{2.566772in}}%
\pgfpathlineto{\pgfqpoint{4.240020in}{2.566772in}}%
\pgfusepath{stroke}%
\end{pgfscope}%
\begin{pgfscope}%
\pgfpathrectangle{\pgfqpoint{0.652287in}{0.740652in}}{\pgfqpoint{4.960000in}{3.696000in}}%
\pgfusepath{clip}%
\pgfsetrectcap%
\pgfsetroundjoin%
\pgfsetlinewidth{1.505625pt}%
\definecolor{currentstroke}{rgb}{0.270588,0.270588,0.270588}%
\pgfsetstrokecolor{currentstroke}%
\pgfsetdash{}{0pt}%
\pgfpathmoveto{\pgfqpoint{4.372287in}{1.255625in}}%
\pgfpathlineto{\pgfqpoint{4.372287in}{0.849832in}}%
\pgfusepath{stroke}%
\end{pgfscope}%
\begin{pgfscope}%
\pgfpathrectangle{\pgfqpoint{0.652287in}{0.740652in}}{\pgfqpoint{4.960000in}{3.696000in}}%
\pgfusepath{clip}%
\pgfsetrectcap%
\pgfsetroundjoin%
\pgfsetlinewidth{1.505625pt}%
\definecolor{currentstroke}{rgb}{0.270588,0.270588,0.270588}%
\pgfsetstrokecolor{currentstroke}%
\pgfsetdash{}{0pt}%
\pgfpathmoveto{\pgfqpoint{4.372287in}{2.243593in}}%
\pgfpathlineto{\pgfqpoint{4.372287in}{2.786425in}}%
\pgfusepath{stroke}%
\end{pgfscope}%
\begin{pgfscope}%
\pgfpathrectangle{\pgfqpoint{0.652287in}{0.740652in}}{\pgfqpoint{4.960000in}{3.696000in}}%
\pgfusepath{clip}%
\pgfsetrectcap%
\pgfsetroundjoin%
\pgfsetlinewidth{1.505625pt}%
\definecolor{currentstroke}{rgb}{0.270588,0.270588,0.270588}%
\pgfsetstrokecolor{currentstroke}%
\pgfsetdash{}{0pt}%
\pgfpathmoveto{\pgfqpoint{4.339220in}{0.849832in}}%
\pgfpathlineto{\pgfqpoint{4.405353in}{0.849832in}}%
\pgfusepath{stroke}%
\end{pgfscope}%
\begin{pgfscope}%
\pgfpathrectangle{\pgfqpoint{0.652287in}{0.740652in}}{\pgfqpoint{4.960000in}{3.696000in}}%
\pgfusepath{clip}%
\pgfsetrectcap%
\pgfsetroundjoin%
\pgfsetlinewidth{1.505625pt}%
\definecolor{currentstroke}{rgb}{0.270588,0.270588,0.270588}%
\pgfsetstrokecolor{currentstroke}%
\pgfsetdash{}{0pt}%
\pgfpathmoveto{\pgfqpoint{4.339220in}{2.786425in}}%
\pgfpathlineto{\pgfqpoint{4.405353in}{2.786425in}}%
\pgfusepath{stroke}%
\end{pgfscope}%
\begin{pgfscope}%
\pgfpathrectangle{\pgfqpoint{0.652287in}{0.740652in}}{\pgfqpoint{4.960000in}{3.696000in}}%
\pgfusepath{clip}%
\pgfsetrectcap%
\pgfsetroundjoin%
\pgfsetlinewidth{1.505625pt}%
\definecolor{currentstroke}{rgb}{0.270588,0.270588,0.270588}%
\pgfsetstrokecolor{currentstroke}%
\pgfsetdash{}{0pt}%
\pgfpathmoveto{\pgfqpoint{4.537620in}{1.289305in}}%
\pgfpathlineto{\pgfqpoint{4.537620in}{0.859220in}}%
\pgfusepath{stroke}%
\end{pgfscope}%
\begin{pgfscope}%
\pgfpathrectangle{\pgfqpoint{0.652287in}{0.740652in}}{\pgfqpoint{4.960000in}{3.696000in}}%
\pgfusepath{clip}%
\pgfsetrectcap%
\pgfsetroundjoin%
\pgfsetlinewidth{1.505625pt}%
\definecolor{currentstroke}{rgb}{0.270588,0.270588,0.270588}%
\pgfsetstrokecolor{currentstroke}%
\pgfsetdash{}{0pt}%
\pgfpathmoveto{\pgfqpoint{4.537620in}{2.346259in}}%
\pgfpathlineto{\pgfqpoint{4.537620in}{2.940289in}}%
\pgfusepath{stroke}%
\end{pgfscope}%
\begin{pgfscope}%
\pgfpathrectangle{\pgfqpoint{0.652287in}{0.740652in}}{\pgfqpoint{4.960000in}{3.696000in}}%
\pgfusepath{clip}%
\pgfsetrectcap%
\pgfsetroundjoin%
\pgfsetlinewidth{1.505625pt}%
\definecolor{currentstroke}{rgb}{0.270588,0.270588,0.270588}%
\pgfsetstrokecolor{currentstroke}%
\pgfsetdash{}{0pt}%
\pgfpathmoveto{\pgfqpoint{4.504553in}{0.859220in}}%
\pgfpathlineto{\pgfqpoint{4.570687in}{0.859220in}}%
\pgfusepath{stroke}%
\end{pgfscope}%
\begin{pgfscope}%
\pgfpathrectangle{\pgfqpoint{0.652287in}{0.740652in}}{\pgfqpoint{4.960000in}{3.696000in}}%
\pgfusepath{clip}%
\pgfsetrectcap%
\pgfsetroundjoin%
\pgfsetlinewidth{1.505625pt}%
\definecolor{currentstroke}{rgb}{0.270588,0.270588,0.270588}%
\pgfsetstrokecolor{currentstroke}%
\pgfsetdash{}{0pt}%
\pgfpathmoveto{\pgfqpoint{4.504553in}{2.940289in}}%
\pgfpathlineto{\pgfqpoint{4.570687in}{2.940289in}}%
\pgfusepath{stroke}%
\end{pgfscope}%
\begin{pgfscope}%
\pgfpathrectangle{\pgfqpoint{0.652287in}{0.740652in}}{\pgfqpoint{4.960000in}{3.696000in}}%
\pgfusepath{clip}%
\pgfsetrectcap%
\pgfsetroundjoin%
\pgfsetlinewidth{1.505625pt}%
\definecolor{currentstroke}{rgb}{0.270588,0.270588,0.270588}%
\pgfsetstrokecolor{currentstroke}%
\pgfsetdash{}{0pt}%
\pgfpathmoveto{\pgfqpoint{4.702953in}{1.347517in}}%
\pgfpathlineto{\pgfqpoint{4.702953in}{0.869273in}}%
\pgfusepath{stroke}%
\end{pgfscope}%
\begin{pgfscope}%
\pgfpathrectangle{\pgfqpoint{0.652287in}{0.740652in}}{\pgfqpoint{4.960000in}{3.696000in}}%
\pgfusepath{clip}%
\pgfsetrectcap%
\pgfsetroundjoin%
\pgfsetlinewidth{1.505625pt}%
\definecolor{currentstroke}{rgb}{0.270588,0.270588,0.270588}%
\pgfsetstrokecolor{currentstroke}%
\pgfsetdash{}{0pt}%
\pgfpathmoveto{\pgfqpoint{4.702953in}{2.516321in}}%
\pgfpathlineto{\pgfqpoint{4.702953in}{3.139171in}}%
\pgfusepath{stroke}%
\end{pgfscope}%
\begin{pgfscope}%
\pgfpathrectangle{\pgfqpoint{0.652287in}{0.740652in}}{\pgfqpoint{4.960000in}{3.696000in}}%
\pgfusepath{clip}%
\pgfsetrectcap%
\pgfsetroundjoin%
\pgfsetlinewidth{1.505625pt}%
\definecolor{currentstroke}{rgb}{0.270588,0.270588,0.270588}%
\pgfsetstrokecolor{currentstroke}%
\pgfsetdash{}{0pt}%
\pgfpathmoveto{\pgfqpoint{4.669887in}{0.869273in}}%
\pgfpathlineto{\pgfqpoint{4.736020in}{0.869273in}}%
\pgfusepath{stroke}%
\end{pgfscope}%
\begin{pgfscope}%
\pgfpathrectangle{\pgfqpoint{0.652287in}{0.740652in}}{\pgfqpoint{4.960000in}{3.696000in}}%
\pgfusepath{clip}%
\pgfsetrectcap%
\pgfsetroundjoin%
\pgfsetlinewidth{1.505625pt}%
\definecolor{currentstroke}{rgb}{0.270588,0.270588,0.270588}%
\pgfsetstrokecolor{currentstroke}%
\pgfsetdash{}{0pt}%
\pgfpathmoveto{\pgfqpoint{4.669887in}{3.139171in}}%
\pgfpathlineto{\pgfqpoint{4.736020in}{3.139171in}}%
\pgfusepath{stroke}%
\end{pgfscope}%
\begin{pgfscope}%
\pgfpathrectangle{\pgfqpoint{0.652287in}{0.740652in}}{\pgfqpoint{4.960000in}{3.696000in}}%
\pgfusepath{clip}%
\pgfsetrectcap%
\pgfsetroundjoin%
\pgfsetlinewidth{1.505625pt}%
\definecolor{currentstroke}{rgb}{0.270588,0.270588,0.270588}%
\pgfsetstrokecolor{currentstroke}%
\pgfsetdash{}{0pt}%
\pgfpathmoveto{\pgfqpoint{4.868287in}{1.393865in}}%
\pgfpathlineto{\pgfqpoint{4.868287in}{0.877552in}}%
\pgfusepath{stroke}%
\end{pgfscope}%
\begin{pgfscope}%
\pgfpathrectangle{\pgfqpoint{0.652287in}{0.740652in}}{\pgfqpoint{4.960000in}{3.696000in}}%
\pgfusepath{clip}%
\pgfsetrectcap%
\pgfsetroundjoin%
\pgfsetlinewidth{1.505625pt}%
\definecolor{currentstroke}{rgb}{0.270588,0.270588,0.270588}%
\pgfsetstrokecolor{currentstroke}%
\pgfsetdash{}{0pt}%
\pgfpathmoveto{\pgfqpoint{4.868287in}{2.655901in}}%
\pgfpathlineto{\pgfqpoint{4.868287in}{3.328074in}}%
\pgfusepath{stroke}%
\end{pgfscope}%
\begin{pgfscope}%
\pgfpathrectangle{\pgfqpoint{0.652287in}{0.740652in}}{\pgfqpoint{4.960000in}{3.696000in}}%
\pgfusepath{clip}%
\pgfsetrectcap%
\pgfsetroundjoin%
\pgfsetlinewidth{1.505625pt}%
\definecolor{currentstroke}{rgb}{0.270588,0.270588,0.270588}%
\pgfsetstrokecolor{currentstroke}%
\pgfsetdash{}{0pt}%
\pgfpathmoveto{\pgfqpoint{4.835220in}{0.877552in}}%
\pgfpathlineto{\pgfqpoint{4.901353in}{0.877552in}}%
\pgfusepath{stroke}%
\end{pgfscope}%
\begin{pgfscope}%
\pgfpathrectangle{\pgfqpoint{0.652287in}{0.740652in}}{\pgfqpoint{4.960000in}{3.696000in}}%
\pgfusepath{clip}%
\pgfsetrectcap%
\pgfsetroundjoin%
\pgfsetlinewidth{1.505625pt}%
\definecolor{currentstroke}{rgb}{0.270588,0.270588,0.270588}%
\pgfsetstrokecolor{currentstroke}%
\pgfsetdash{}{0pt}%
\pgfpathmoveto{\pgfqpoint{4.835220in}{3.328074in}}%
\pgfpathlineto{\pgfqpoint{4.901353in}{3.328074in}}%
\pgfusepath{stroke}%
\end{pgfscope}%
\begin{pgfscope}%
\pgfpathrectangle{\pgfqpoint{0.652287in}{0.740652in}}{\pgfqpoint{4.960000in}{3.696000in}}%
\pgfusepath{clip}%
\pgfsetrectcap%
\pgfsetroundjoin%
\pgfsetlinewidth{1.505625pt}%
\definecolor{currentstroke}{rgb}{0.270588,0.270588,0.270588}%
\pgfsetstrokecolor{currentstroke}%
\pgfsetdash{}{0pt}%
\pgfpathmoveto{\pgfqpoint{5.033620in}{1.408297in}}%
\pgfpathlineto{\pgfqpoint{5.033620in}{0.871971in}}%
\pgfusepath{stroke}%
\end{pgfscope}%
\begin{pgfscope}%
\pgfpathrectangle{\pgfqpoint{0.652287in}{0.740652in}}{\pgfqpoint{4.960000in}{3.696000in}}%
\pgfusepath{clip}%
\pgfsetrectcap%
\pgfsetroundjoin%
\pgfsetlinewidth{1.505625pt}%
\definecolor{currentstroke}{rgb}{0.270588,0.270588,0.270588}%
\pgfsetstrokecolor{currentstroke}%
\pgfsetdash{}{0pt}%
\pgfpathmoveto{\pgfqpoint{5.033620in}{2.752551in}}%
\pgfpathlineto{\pgfqpoint{5.033620in}{3.490846in}}%
\pgfusepath{stroke}%
\end{pgfscope}%
\begin{pgfscope}%
\pgfpathrectangle{\pgfqpoint{0.652287in}{0.740652in}}{\pgfqpoint{4.960000in}{3.696000in}}%
\pgfusepath{clip}%
\pgfsetrectcap%
\pgfsetroundjoin%
\pgfsetlinewidth{1.505625pt}%
\definecolor{currentstroke}{rgb}{0.270588,0.270588,0.270588}%
\pgfsetstrokecolor{currentstroke}%
\pgfsetdash{}{0pt}%
\pgfpathmoveto{\pgfqpoint{5.000553in}{0.871971in}}%
\pgfpathlineto{\pgfqpoint{5.066687in}{0.871971in}}%
\pgfusepath{stroke}%
\end{pgfscope}%
\begin{pgfscope}%
\pgfpathrectangle{\pgfqpoint{0.652287in}{0.740652in}}{\pgfqpoint{4.960000in}{3.696000in}}%
\pgfusepath{clip}%
\pgfsetrectcap%
\pgfsetroundjoin%
\pgfsetlinewidth{1.505625pt}%
\definecolor{currentstroke}{rgb}{0.270588,0.270588,0.270588}%
\pgfsetstrokecolor{currentstroke}%
\pgfsetdash{}{0pt}%
\pgfpathmoveto{\pgfqpoint{5.000553in}{3.490846in}}%
\pgfpathlineto{\pgfqpoint{5.066687in}{3.490846in}}%
\pgfusepath{stroke}%
\end{pgfscope}%
\begin{pgfscope}%
\pgfpathrectangle{\pgfqpoint{0.652287in}{0.740652in}}{\pgfqpoint{4.960000in}{3.696000in}}%
\pgfusepath{clip}%
\pgfsetrectcap%
\pgfsetroundjoin%
\pgfsetlinewidth{1.505625pt}%
\definecolor{currentstroke}{rgb}{0.270588,0.270588,0.270588}%
\pgfsetstrokecolor{currentstroke}%
\pgfsetdash{}{0pt}%
\pgfpathmoveto{\pgfqpoint{5.198953in}{1.499108in}}%
\pgfpathlineto{\pgfqpoint{5.198953in}{0.906639in}}%
\pgfusepath{stroke}%
\end{pgfscope}%
\begin{pgfscope}%
\pgfpathrectangle{\pgfqpoint{0.652287in}{0.740652in}}{\pgfqpoint{4.960000in}{3.696000in}}%
\pgfusepath{clip}%
\pgfsetrectcap%
\pgfsetroundjoin%
\pgfsetlinewidth{1.505625pt}%
\definecolor{currentstroke}{rgb}{0.270588,0.270588,0.270588}%
\pgfsetstrokecolor{currentstroke}%
\pgfsetdash{}{0pt}%
\pgfpathmoveto{\pgfqpoint{5.198953in}{2.913124in}}%
\pgfpathlineto{\pgfqpoint{5.198953in}{3.720515in}}%
\pgfusepath{stroke}%
\end{pgfscope}%
\begin{pgfscope}%
\pgfpathrectangle{\pgfqpoint{0.652287in}{0.740652in}}{\pgfqpoint{4.960000in}{3.696000in}}%
\pgfusepath{clip}%
\pgfsetrectcap%
\pgfsetroundjoin%
\pgfsetlinewidth{1.505625pt}%
\definecolor{currentstroke}{rgb}{0.270588,0.270588,0.270588}%
\pgfsetstrokecolor{currentstroke}%
\pgfsetdash{}{0pt}%
\pgfpathmoveto{\pgfqpoint{5.165887in}{0.906639in}}%
\pgfpathlineto{\pgfqpoint{5.232020in}{0.906639in}}%
\pgfusepath{stroke}%
\end{pgfscope}%
\begin{pgfscope}%
\pgfpathrectangle{\pgfqpoint{0.652287in}{0.740652in}}{\pgfqpoint{4.960000in}{3.696000in}}%
\pgfusepath{clip}%
\pgfsetrectcap%
\pgfsetroundjoin%
\pgfsetlinewidth{1.505625pt}%
\definecolor{currentstroke}{rgb}{0.270588,0.270588,0.270588}%
\pgfsetstrokecolor{currentstroke}%
\pgfsetdash{}{0pt}%
\pgfpathmoveto{\pgfqpoint{5.165887in}{3.720515in}}%
\pgfpathlineto{\pgfqpoint{5.232020in}{3.720515in}}%
\pgfusepath{stroke}%
\end{pgfscope}%
\begin{pgfscope}%
\pgfpathrectangle{\pgfqpoint{0.652287in}{0.740652in}}{\pgfqpoint{4.960000in}{3.696000in}}%
\pgfusepath{clip}%
\pgfsetrectcap%
\pgfsetroundjoin%
\pgfsetlinewidth{1.505625pt}%
\definecolor{currentstroke}{rgb}{0.270588,0.270588,0.270588}%
\pgfsetstrokecolor{currentstroke}%
\pgfsetdash{}{0pt}%
\pgfpathmoveto{\pgfqpoint{5.364287in}{1.569794in}}%
\pgfpathlineto{\pgfqpoint{5.364287in}{0.980411in}}%
\pgfusepath{stroke}%
\end{pgfscope}%
\begin{pgfscope}%
\pgfpathrectangle{\pgfqpoint{0.652287in}{0.740652in}}{\pgfqpoint{4.960000in}{3.696000in}}%
\pgfusepath{clip}%
\pgfsetrectcap%
\pgfsetroundjoin%
\pgfsetlinewidth{1.505625pt}%
\definecolor{currentstroke}{rgb}{0.270588,0.270588,0.270588}%
\pgfsetstrokecolor{currentstroke}%
\pgfsetdash{}{0pt}%
\pgfpathmoveto{\pgfqpoint{5.364287in}{3.052408in}}%
\pgfpathlineto{\pgfqpoint{5.364287in}{3.936472in}}%
\pgfusepath{stroke}%
\end{pgfscope}%
\begin{pgfscope}%
\pgfpathrectangle{\pgfqpoint{0.652287in}{0.740652in}}{\pgfqpoint{4.960000in}{3.696000in}}%
\pgfusepath{clip}%
\pgfsetrectcap%
\pgfsetroundjoin%
\pgfsetlinewidth{1.505625pt}%
\definecolor{currentstroke}{rgb}{0.270588,0.270588,0.270588}%
\pgfsetstrokecolor{currentstroke}%
\pgfsetdash{}{0pt}%
\pgfpathmoveto{\pgfqpoint{5.331220in}{0.980411in}}%
\pgfpathlineto{\pgfqpoint{5.397353in}{0.980411in}}%
\pgfusepath{stroke}%
\end{pgfscope}%
\begin{pgfscope}%
\pgfpathrectangle{\pgfqpoint{0.652287in}{0.740652in}}{\pgfqpoint{4.960000in}{3.696000in}}%
\pgfusepath{clip}%
\pgfsetrectcap%
\pgfsetroundjoin%
\pgfsetlinewidth{1.505625pt}%
\definecolor{currentstroke}{rgb}{0.270588,0.270588,0.270588}%
\pgfsetstrokecolor{currentstroke}%
\pgfsetdash{}{0pt}%
\pgfpathmoveto{\pgfqpoint{5.331220in}{3.936472in}}%
\pgfpathlineto{\pgfqpoint{5.397353in}{3.936472in}}%
\pgfusepath{stroke}%
\end{pgfscope}%
\begin{pgfscope}%
\pgfpathrectangle{\pgfqpoint{0.652287in}{0.740652in}}{\pgfqpoint{4.960000in}{3.696000in}}%
\pgfusepath{clip}%
\pgfsetrectcap%
\pgfsetroundjoin%
\pgfsetlinewidth{1.505625pt}%
\definecolor{currentstroke}{rgb}{0.270588,0.270588,0.270588}%
\pgfsetstrokecolor{currentstroke}%
\pgfsetdash{}{0pt}%
\pgfpathmoveto{\pgfqpoint{5.529620in}{1.585742in}}%
\pgfpathlineto{\pgfqpoint{5.529620in}{0.922126in}}%
\pgfusepath{stroke}%
\end{pgfscope}%
\begin{pgfscope}%
\pgfpathrectangle{\pgfqpoint{0.652287in}{0.740652in}}{\pgfqpoint{4.960000in}{3.696000in}}%
\pgfusepath{clip}%
\pgfsetrectcap%
\pgfsetroundjoin%
\pgfsetlinewidth{1.505625pt}%
\definecolor{currentstroke}{rgb}{0.270588,0.270588,0.270588}%
\pgfsetstrokecolor{currentstroke}%
\pgfsetdash{}{0pt}%
\pgfpathmoveto{\pgfqpoint{5.529620in}{3.190712in}}%
\pgfpathlineto{\pgfqpoint{5.529620in}{4.210309in}}%
\pgfusepath{stroke}%
\end{pgfscope}%
\begin{pgfscope}%
\pgfpathrectangle{\pgfqpoint{0.652287in}{0.740652in}}{\pgfqpoint{4.960000in}{3.696000in}}%
\pgfusepath{clip}%
\pgfsetrectcap%
\pgfsetroundjoin%
\pgfsetlinewidth{1.505625pt}%
\definecolor{currentstroke}{rgb}{0.270588,0.270588,0.270588}%
\pgfsetstrokecolor{currentstroke}%
\pgfsetdash{}{0pt}%
\pgfpathmoveto{\pgfqpoint{5.496553in}{0.922126in}}%
\pgfpathlineto{\pgfqpoint{5.562687in}{0.922126in}}%
\pgfusepath{stroke}%
\end{pgfscope}%
\begin{pgfscope}%
\pgfpathrectangle{\pgfqpoint{0.652287in}{0.740652in}}{\pgfqpoint{4.960000in}{3.696000in}}%
\pgfusepath{clip}%
\pgfsetrectcap%
\pgfsetroundjoin%
\pgfsetlinewidth{1.505625pt}%
\definecolor{currentstroke}{rgb}{0.270588,0.270588,0.270588}%
\pgfsetstrokecolor{currentstroke}%
\pgfsetdash{}{0pt}%
\pgfpathmoveto{\pgfqpoint{5.496553in}{4.210309in}}%
\pgfpathlineto{\pgfqpoint{5.562687in}{4.210309in}}%
\pgfusepath{stroke}%
\end{pgfscope}%
\begin{pgfscope}%
\pgfpathrectangle{\pgfqpoint{0.652287in}{0.740652in}}{\pgfqpoint{4.960000in}{3.696000in}}%
\pgfusepath{clip}%
\pgfsetrectcap%
\pgfsetroundjoin%
\pgfsetlinewidth{1.505625pt}%
\definecolor{currentstroke}{rgb}{0.270588,0.270588,0.270588}%
\pgfsetstrokecolor{currentstroke}%
\pgfsetdash{}{0pt}%
\pgfpathmoveto{\pgfqpoint{0.668820in}{0.932622in}}%
\pgfpathlineto{\pgfqpoint{0.801087in}{0.932622in}}%
\pgfusepath{stroke}%
\end{pgfscope}%
\begin{pgfscope}%
\pgfpathrectangle{\pgfqpoint{0.652287in}{0.740652in}}{\pgfqpoint{4.960000in}{3.696000in}}%
\pgfusepath{clip}%
\pgfsetrectcap%
\pgfsetroundjoin%
\pgfsetlinewidth{1.505625pt}%
\definecolor{currentstroke}{rgb}{0.270588,0.270588,0.270588}%
\pgfsetstrokecolor{currentstroke}%
\pgfsetdash{}{0pt}%
\pgfpathmoveto{\pgfqpoint{0.834153in}{0.884833in}}%
\pgfpathlineto{\pgfqpoint{0.966420in}{0.884833in}}%
\pgfusepath{stroke}%
\end{pgfscope}%
\begin{pgfscope}%
\pgfpathrectangle{\pgfqpoint{0.652287in}{0.740652in}}{\pgfqpoint{4.960000in}{3.696000in}}%
\pgfusepath{clip}%
\pgfsetrectcap%
\pgfsetroundjoin%
\pgfsetlinewidth{1.505625pt}%
\definecolor{currentstroke}{rgb}{0.270588,0.270588,0.270588}%
\pgfsetstrokecolor{currentstroke}%
\pgfsetdash{}{0pt}%
\pgfpathmoveto{\pgfqpoint{0.999487in}{0.868460in}}%
\pgfpathlineto{\pgfqpoint{1.131753in}{0.868460in}}%
\pgfusepath{stroke}%
\end{pgfscope}%
\begin{pgfscope}%
\pgfpathrectangle{\pgfqpoint{0.652287in}{0.740652in}}{\pgfqpoint{4.960000in}{3.696000in}}%
\pgfusepath{clip}%
\pgfsetrectcap%
\pgfsetroundjoin%
\pgfsetlinewidth{1.505625pt}%
\definecolor{currentstroke}{rgb}{0.270588,0.270588,0.270588}%
\pgfsetstrokecolor{currentstroke}%
\pgfsetdash{}{0pt}%
\pgfpathmoveto{\pgfqpoint{1.164820in}{0.863174in}}%
\pgfpathlineto{\pgfqpoint{1.297087in}{0.863174in}}%
\pgfusepath{stroke}%
\end{pgfscope}%
\begin{pgfscope}%
\pgfpathrectangle{\pgfqpoint{0.652287in}{0.740652in}}{\pgfqpoint{4.960000in}{3.696000in}}%
\pgfusepath{clip}%
\pgfsetrectcap%
\pgfsetroundjoin%
\pgfsetlinewidth{1.505625pt}%
\definecolor{currentstroke}{rgb}{0.270588,0.270588,0.270588}%
\pgfsetstrokecolor{currentstroke}%
\pgfsetdash{}{0pt}%
\pgfpathmoveto{\pgfqpoint{1.330153in}{0.869199in}}%
\pgfpathlineto{\pgfqpoint{1.462420in}{0.869199in}}%
\pgfusepath{stroke}%
\end{pgfscope}%
\begin{pgfscope}%
\pgfpathrectangle{\pgfqpoint{0.652287in}{0.740652in}}{\pgfqpoint{4.960000in}{3.696000in}}%
\pgfusepath{clip}%
\pgfsetrectcap%
\pgfsetroundjoin%
\pgfsetlinewidth{1.505625pt}%
\definecolor{currentstroke}{rgb}{0.270588,0.270588,0.270588}%
\pgfsetstrokecolor{currentstroke}%
\pgfsetdash{}{0pt}%
\pgfpathmoveto{\pgfqpoint{1.495487in}{0.864024in}}%
\pgfpathlineto{\pgfqpoint{1.627753in}{0.864024in}}%
\pgfusepath{stroke}%
\end{pgfscope}%
\begin{pgfscope}%
\pgfpathrectangle{\pgfqpoint{0.652287in}{0.740652in}}{\pgfqpoint{4.960000in}{3.696000in}}%
\pgfusepath{clip}%
\pgfsetrectcap%
\pgfsetroundjoin%
\pgfsetlinewidth{1.505625pt}%
\definecolor{currentstroke}{rgb}{0.270588,0.270588,0.270588}%
\pgfsetstrokecolor{currentstroke}%
\pgfsetdash{}{0pt}%
\pgfpathmoveto{\pgfqpoint{1.660820in}{0.882320in}}%
\pgfpathlineto{\pgfqpoint{1.793087in}{0.882320in}}%
\pgfusepath{stroke}%
\end{pgfscope}%
\begin{pgfscope}%
\pgfpathrectangle{\pgfqpoint{0.652287in}{0.740652in}}{\pgfqpoint{4.960000in}{3.696000in}}%
\pgfusepath{clip}%
\pgfsetrectcap%
\pgfsetroundjoin%
\pgfsetlinewidth{1.505625pt}%
\definecolor{currentstroke}{rgb}{0.270588,0.270588,0.270588}%
\pgfsetstrokecolor{currentstroke}%
\pgfsetdash{}{0pt}%
\pgfpathmoveto{\pgfqpoint{1.826153in}{0.900541in}}%
\pgfpathlineto{\pgfqpoint{1.958420in}{0.900541in}}%
\pgfusepath{stroke}%
\end{pgfscope}%
\begin{pgfscope}%
\pgfpathrectangle{\pgfqpoint{0.652287in}{0.740652in}}{\pgfqpoint{4.960000in}{3.696000in}}%
\pgfusepath{clip}%
\pgfsetrectcap%
\pgfsetroundjoin%
\pgfsetlinewidth{1.505625pt}%
\definecolor{currentstroke}{rgb}{0.270588,0.270588,0.270588}%
\pgfsetstrokecolor{currentstroke}%
\pgfsetdash{}{0pt}%
\pgfpathmoveto{\pgfqpoint{1.991487in}{0.882541in}}%
\pgfpathlineto{\pgfqpoint{2.123753in}{0.882541in}}%
\pgfusepath{stroke}%
\end{pgfscope}%
\begin{pgfscope}%
\pgfpathrectangle{\pgfqpoint{0.652287in}{0.740652in}}{\pgfqpoint{4.960000in}{3.696000in}}%
\pgfusepath{clip}%
\pgfsetrectcap%
\pgfsetroundjoin%
\pgfsetlinewidth{1.505625pt}%
\definecolor{currentstroke}{rgb}{0.270588,0.270588,0.270588}%
\pgfsetstrokecolor{currentstroke}%
\pgfsetdash{}{0pt}%
\pgfpathmoveto{\pgfqpoint{2.156820in}{0.894553in}}%
\pgfpathlineto{\pgfqpoint{2.289087in}{0.894553in}}%
\pgfusepath{stroke}%
\end{pgfscope}%
\begin{pgfscope}%
\pgfpathrectangle{\pgfqpoint{0.652287in}{0.740652in}}{\pgfqpoint{4.960000in}{3.696000in}}%
\pgfusepath{clip}%
\pgfsetrectcap%
\pgfsetroundjoin%
\pgfsetlinewidth{1.505625pt}%
\definecolor{currentstroke}{rgb}{0.270588,0.270588,0.270588}%
\pgfsetstrokecolor{currentstroke}%
\pgfsetdash{}{0pt}%
\pgfpathmoveto{\pgfqpoint{2.322153in}{1.033135in}}%
\pgfpathlineto{\pgfqpoint{2.454420in}{1.033135in}}%
\pgfusepath{stroke}%
\end{pgfscope}%
\begin{pgfscope}%
\pgfpathrectangle{\pgfqpoint{0.652287in}{0.740652in}}{\pgfqpoint{4.960000in}{3.696000in}}%
\pgfusepath{clip}%
\pgfsetrectcap%
\pgfsetroundjoin%
\pgfsetlinewidth{1.505625pt}%
\definecolor{currentstroke}{rgb}{0.270588,0.270588,0.270588}%
\pgfsetstrokecolor{currentstroke}%
\pgfsetdash{}{0pt}%
\pgfpathmoveto{\pgfqpoint{2.487487in}{1.103655in}}%
\pgfpathlineto{\pgfqpoint{2.619753in}{1.103655in}}%
\pgfusepath{stroke}%
\end{pgfscope}%
\begin{pgfscope}%
\pgfpathrectangle{\pgfqpoint{0.652287in}{0.740652in}}{\pgfqpoint{4.960000in}{3.696000in}}%
\pgfusepath{clip}%
\pgfsetrectcap%
\pgfsetroundjoin%
\pgfsetlinewidth{1.505625pt}%
\definecolor{currentstroke}{rgb}{0.270588,0.270588,0.270588}%
\pgfsetstrokecolor{currentstroke}%
\pgfsetdash{}{0pt}%
\pgfpathmoveto{\pgfqpoint{2.652820in}{1.028256in}}%
\pgfpathlineto{\pgfqpoint{2.785087in}{1.028256in}}%
\pgfusepath{stroke}%
\end{pgfscope}%
\begin{pgfscope}%
\pgfpathrectangle{\pgfqpoint{0.652287in}{0.740652in}}{\pgfqpoint{4.960000in}{3.696000in}}%
\pgfusepath{clip}%
\pgfsetrectcap%
\pgfsetroundjoin%
\pgfsetlinewidth{1.505625pt}%
\definecolor{currentstroke}{rgb}{0.270588,0.270588,0.270588}%
\pgfsetstrokecolor{currentstroke}%
\pgfsetdash{}{0pt}%
\pgfpathmoveto{\pgfqpoint{2.818153in}{1.151388in}}%
\pgfpathlineto{\pgfqpoint{2.950420in}{1.151388in}}%
\pgfusepath{stroke}%
\end{pgfscope}%
\begin{pgfscope}%
\pgfpathrectangle{\pgfqpoint{0.652287in}{0.740652in}}{\pgfqpoint{4.960000in}{3.696000in}}%
\pgfusepath{clip}%
\pgfsetrectcap%
\pgfsetroundjoin%
\pgfsetlinewidth{1.505625pt}%
\definecolor{currentstroke}{rgb}{0.270588,0.270588,0.270588}%
\pgfsetstrokecolor{currentstroke}%
\pgfsetdash{}{0pt}%
\pgfpathmoveto{\pgfqpoint{2.983487in}{1.148506in}}%
\pgfpathlineto{\pgfqpoint{3.115753in}{1.148506in}}%
\pgfusepath{stroke}%
\end{pgfscope}%
\begin{pgfscope}%
\pgfpathrectangle{\pgfqpoint{0.652287in}{0.740652in}}{\pgfqpoint{4.960000in}{3.696000in}}%
\pgfusepath{clip}%
\pgfsetrectcap%
\pgfsetroundjoin%
\pgfsetlinewidth{1.505625pt}%
\definecolor{currentstroke}{rgb}{0.270588,0.270588,0.270588}%
\pgfsetstrokecolor{currentstroke}%
\pgfsetdash{}{0pt}%
\pgfpathmoveto{\pgfqpoint{3.148820in}{1.199436in}}%
\pgfpathlineto{\pgfqpoint{3.281087in}{1.199436in}}%
\pgfusepath{stroke}%
\end{pgfscope}%
\begin{pgfscope}%
\pgfpathrectangle{\pgfqpoint{0.652287in}{0.740652in}}{\pgfqpoint{4.960000in}{3.696000in}}%
\pgfusepath{clip}%
\pgfsetrectcap%
\pgfsetroundjoin%
\pgfsetlinewidth{1.505625pt}%
\definecolor{currentstroke}{rgb}{0.270588,0.270588,0.270588}%
\pgfsetstrokecolor{currentstroke}%
\pgfsetdash{}{0pt}%
\pgfpathmoveto{\pgfqpoint{3.314153in}{1.300448in}}%
\pgfpathlineto{\pgfqpoint{3.446420in}{1.300448in}}%
\pgfusepath{stroke}%
\end{pgfscope}%
\begin{pgfscope}%
\pgfpathrectangle{\pgfqpoint{0.652287in}{0.740652in}}{\pgfqpoint{4.960000in}{3.696000in}}%
\pgfusepath{clip}%
\pgfsetrectcap%
\pgfsetroundjoin%
\pgfsetlinewidth{1.505625pt}%
\definecolor{currentstroke}{rgb}{0.270588,0.270588,0.270588}%
\pgfsetstrokecolor{currentstroke}%
\pgfsetdash{}{0pt}%
\pgfpathmoveto{\pgfqpoint{3.479487in}{1.368307in}}%
\pgfpathlineto{\pgfqpoint{3.611753in}{1.368307in}}%
\pgfusepath{stroke}%
\end{pgfscope}%
\begin{pgfscope}%
\pgfpathrectangle{\pgfqpoint{0.652287in}{0.740652in}}{\pgfqpoint{4.960000in}{3.696000in}}%
\pgfusepath{clip}%
\pgfsetrectcap%
\pgfsetroundjoin%
\pgfsetlinewidth{1.505625pt}%
\definecolor{currentstroke}{rgb}{0.270588,0.270588,0.270588}%
\pgfsetstrokecolor{currentstroke}%
\pgfsetdash{}{0pt}%
\pgfpathmoveto{\pgfqpoint{3.644820in}{1.425539in}}%
\pgfpathlineto{\pgfqpoint{3.777087in}{1.425539in}}%
\pgfusepath{stroke}%
\end{pgfscope}%
\begin{pgfscope}%
\pgfpathrectangle{\pgfqpoint{0.652287in}{0.740652in}}{\pgfqpoint{4.960000in}{3.696000in}}%
\pgfusepath{clip}%
\pgfsetrectcap%
\pgfsetroundjoin%
\pgfsetlinewidth{1.505625pt}%
\definecolor{currentstroke}{rgb}{0.270588,0.270588,0.270588}%
\pgfsetstrokecolor{currentstroke}%
\pgfsetdash{}{0pt}%
\pgfpathmoveto{\pgfqpoint{3.810153in}{1.504726in}}%
\pgfpathlineto{\pgfqpoint{3.942420in}{1.504726in}}%
\pgfusepath{stroke}%
\end{pgfscope}%
\begin{pgfscope}%
\pgfpathrectangle{\pgfqpoint{0.652287in}{0.740652in}}{\pgfqpoint{4.960000in}{3.696000in}}%
\pgfusepath{clip}%
\pgfsetrectcap%
\pgfsetroundjoin%
\pgfsetlinewidth{1.505625pt}%
\definecolor{currentstroke}{rgb}{0.270588,0.270588,0.270588}%
\pgfsetstrokecolor{currentstroke}%
\pgfsetdash{}{0pt}%
\pgfpathmoveto{\pgfqpoint{3.975487in}{1.574913in}}%
\pgfpathlineto{\pgfqpoint{4.107753in}{1.574913in}}%
\pgfusepath{stroke}%
\end{pgfscope}%
\begin{pgfscope}%
\pgfpathrectangle{\pgfqpoint{0.652287in}{0.740652in}}{\pgfqpoint{4.960000in}{3.696000in}}%
\pgfusepath{clip}%
\pgfsetrectcap%
\pgfsetroundjoin%
\pgfsetlinewidth{1.505625pt}%
\definecolor{currentstroke}{rgb}{0.270588,0.270588,0.270588}%
\pgfsetstrokecolor{currentstroke}%
\pgfsetdash{}{0pt}%
\pgfpathmoveto{\pgfqpoint{4.140820in}{1.656373in}}%
\pgfpathlineto{\pgfqpoint{4.273087in}{1.656373in}}%
\pgfusepath{stroke}%
\end{pgfscope}%
\begin{pgfscope}%
\pgfpathrectangle{\pgfqpoint{0.652287in}{0.740652in}}{\pgfqpoint{4.960000in}{3.696000in}}%
\pgfusepath{clip}%
\pgfsetrectcap%
\pgfsetroundjoin%
\pgfsetlinewidth{1.505625pt}%
\definecolor{currentstroke}{rgb}{0.270588,0.270588,0.270588}%
\pgfsetstrokecolor{currentstroke}%
\pgfsetdash{}{0pt}%
\pgfpathmoveto{\pgfqpoint{4.306153in}{1.750547in}}%
\pgfpathlineto{\pgfqpoint{4.438420in}{1.750547in}}%
\pgfusepath{stroke}%
\end{pgfscope}%
\begin{pgfscope}%
\pgfpathrectangle{\pgfqpoint{0.652287in}{0.740652in}}{\pgfqpoint{4.960000in}{3.696000in}}%
\pgfusepath{clip}%
\pgfsetrectcap%
\pgfsetroundjoin%
\pgfsetlinewidth{1.505625pt}%
\definecolor{currentstroke}{rgb}{0.270588,0.270588,0.270588}%
\pgfsetstrokecolor{currentstroke}%
\pgfsetdash{}{0pt}%
\pgfpathmoveto{\pgfqpoint{4.471487in}{1.817241in}}%
\pgfpathlineto{\pgfqpoint{4.603753in}{1.817241in}}%
\pgfusepath{stroke}%
\end{pgfscope}%
\begin{pgfscope}%
\pgfpathrectangle{\pgfqpoint{0.652287in}{0.740652in}}{\pgfqpoint{4.960000in}{3.696000in}}%
\pgfusepath{clip}%
\pgfsetrectcap%
\pgfsetroundjoin%
\pgfsetlinewidth{1.505625pt}%
\definecolor{currentstroke}{rgb}{0.270588,0.270588,0.270588}%
\pgfsetstrokecolor{currentstroke}%
\pgfsetdash{}{0pt}%
\pgfpathmoveto{\pgfqpoint{4.636820in}{1.928010in}}%
\pgfpathlineto{\pgfqpoint{4.769087in}{1.928010in}}%
\pgfusepath{stroke}%
\end{pgfscope}%
\begin{pgfscope}%
\pgfpathrectangle{\pgfqpoint{0.652287in}{0.740652in}}{\pgfqpoint{4.960000in}{3.696000in}}%
\pgfusepath{clip}%
\pgfsetrectcap%
\pgfsetroundjoin%
\pgfsetlinewidth{1.505625pt}%
\definecolor{currentstroke}{rgb}{0.270588,0.270588,0.270588}%
\pgfsetstrokecolor{currentstroke}%
\pgfsetdash{}{0pt}%
\pgfpathmoveto{\pgfqpoint{4.802153in}{2.025345in}}%
\pgfpathlineto{\pgfqpoint{4.934420in}{2.025345in}}%
\pgfusepath{stroke}%
\end{pgfscope}%
\begin{pgfscope}%
\pgfpathrectangle{\pgfqpoint{0.652287in}{0.740652in}}{\pgfqpoint{4.960000in}{3.696000in}}%
\pgfusepath{clip}%
\pgfsetrectcap%
\pgfsetroundjoin%
\pgfsetlinewidth{1.505625pt}%
\definecolor{currentstroke}{rgb}{0.270588,0.270588,0.270588}%
\pgfsetstrokecolor{currentstroke}%
\pgfsetdash{}{0pt}%
\pgfpathmoveto{\pgfqpoint{4.967487in}{2.078382in}}%
\pgfpathlineto{\pgfqpoint{5.099753in}{2.078382in}}%
\pgfusepath{stroke}%
\end{pgfscope}%
\begin{pgfscope}%
\pgfpathrectangle{\pgfqpoint{0.652287in}{0.740652in}}{\pgfqpoint{4.960000in}{3.696000in}}%
\pgfusepath{clip}%
\pgfsetrectcap%
\pgfsetroundjoin%
\pgfsetlinewidth{1.505625pt}%
\definecolor{currentstroke}{rgb}{0.270588,0.270588,0.270588}%
\pgfsetstrokecolor{currentstroke}%
\pgfsetdash{}{0pt}%
\pgfpathmoveto{\pgfqpoint{5.132820in}{2.206596in}}%
\pgfpathlineto{\pgfqpoint{5.265087in}{2.206596in}}%
\pgfusepath{stroke}%
\end{pgfscope}%
\begin{pgfscope}%
\pgfpathrectangle{\pgfqpoint{0.652287in}{0.740652in}}{\pgfqpoint{4.960000in}{3.696000in}}%
\pgfusepath{clip}%
\pgfsetrectcap%
\pgfsetroundjoin%
\pgfsetlinewidth{1.505625pt}%
\definecolor{currentstroke}{rgb}{0.270588,0.270588,0.270588}%
\pgfsetstrokecolor{currentstroke}%
\pgfsetdash{}{0pt}%
\pgfpathmoveto{\pgfqpoint{5.298153in}{2.316479in}}%
\pgfpathlineto{\pgfqpoint{5.430420in}{2.316479in}}%
\pgfusepath{stroke}%
\end{pgfscope}%
\begin{pgfscope}%
\pgfpathrectangle{\pgfqpoint{0.652287in}{0.740652in}}{\pgfqpoint{4.960000in}{3.696000in}}%
\pgfusepath{clip}%
\pgfsetrectcap%
\pgfsetroundjoin%
\pgfsetlinewidth{1.505625pt}%
\definecolor{currentstroke}{rgb}{0.270588,0.270588,0.270588}%
\pgfsetstrokecolor{currentstroke}%
\pgfsetdash{}{0pt}%
\pgfpathmoveto{\pgfqpoint{5.463487in}{2.393669in}}%
\pgfpathlineto{\pgfqpoint{5.595753in}{2.393669in}}%
\pgfusepath{stroke}%
\end{pgfscope}%
\begin{pgfscope}%
\pgfsetrectcap%
\pgfsetmiterjoin%
\pgfsetlinewidth{0.803000pt}%
\definecolor{currentstroke}{rgb}{0.000000,0.000000,0.000000}%
\pgfsetstrokecolor{currentstroke}%
\pgfsetdash{}{0pt}%
\pgfpathmoveto{\pgfqpoint{0.652287in}{0.740652in}}%
\pgfpathlineto{\pgfqpoint{0.652287in}{4.436652in}}%
\pgfusepath{stroke}%
\end{pgfscope}%
\begin{pgfscope}%
\pgfsetrectcap%
\pgfsetmiterjoin%
\pgfsetlinewidth{0.803000pt}%
\definecolor{currentstroke}{rgb}{0.000000,0.000000,0.000000}%
\pgfsetstrokecolor{currentstroke}%
\pgfsetdash{}{0pt}%
\pgfpathmoveto{\pgfqpoint{5.612287in}{0.740652in}}%
\pgfpathlineto{\pgfqpoint{5.612287in}{4.436652in}}%
\pgfusepath{stroke}%
\end{pgfscope}%
\begin{pgfscope}%
\pgfsetrectcap%
\pgfsetmiterjoin%
\pgfsetlinewidth{0.803000pt}%
\definecolor{currentstroke}{rgb}{0.000000,0.000000,0.000000}%
\pgfsetstrokecolor{currentstroke}%
\pgfsetdash{}{0pt}%
\pgfpathmoveto{\pgfqpoint{0.652287in}{0.740652in}}%
\pgfpathlineto{\pgfqpoint{5.612287in}{0.740652in}}%
\pgfusepath{stroke}%
\end{pgfscope}%
\begin{pgfscope}%
\pgfsetrectcap%
\pgfsetmiterjoin%
\pgfsetlinewidth{0.803000pt}%
\definecolor{currentstroke}{rgb}{0.000000,0.000000,0.000000}%
\pgfsetstrokecolor{currentstroke}%
\pgfsetdash{}{0pt}%
\pgfpathmoveto{\pgfqpoint{0.652287in}{4.436652in}}%
\pgfpathlineto{\pgfqpoint{5.612287in}{4.436652in}}%
\pgfusepath{stroke}%
\end{pgfscope}%
\end{pgfpicture}%
\makeatother%
\endgroup%
}
\caption{Benchmarking of Cap'n Proto send time for varying message sizes.}
\label{sizesendtime}
\end{figure}

I benchmarked the performance of Cap'n Proto \cite{capnp} for varying message sizes, measuring the maximum goodput at which messages could be sent (Figure~\ref{sizegoodput}), and the time taken to send (Figure~\ref{sizesendtime}). Figure~\ref{sizesendtime} is a box plot with whiskers plotted at the 5\%ile and 95\%ile with outliers excluded.

Cap'n Proto has an optimum message size of around 1000 bytes; at this point, data can be sent at the highest possible goodput (Figure~\ref{sizegoodput}). As the size of messages increases beyond this point, message serialisation costs cause the time taken to send messages to increase and the maximum goodput that can be reached to decrease; there is a rapid increase in send time as messages increase beyond 2000 bytes (Figures \ref{sizegoodput} and \ref{sizesendtime})

\subsection{Tezos Cryptography} \label{tezosbenchmark}

\begin{table}[h]
	\centering
	\begin{tabular}{|l|r|}
	\hline
	Function                 & Time (µs) \\ \hline
	Sign                     & 427.87   \\
	Check                    & 1,171.77 \\
	Aggregate (4 sigs)       & 302.90   \\
	Aggregate check (4 sigs) & 1,179.25 \\
	Aggregate (8 sigs)       & 605.38   \\
	Aggregate check (8 sigs) & 1,180.61 \\ \hline
	\end{tabular}
	\caption{Benchmarking of key functions of the Tezos Cryptography library}
	\label{tezostable}
\end{table}

I benchmarked key functions of the Tezos Cryptography library \cite{tezosCrypto} with Jane Street's Core\_bench module \cite{janestreetCoreBench2023}. Core\_bench is a micro-benchmarking library used to estimate the cost of operations in OCaml, it runs the operation many times and uses linear regression to try to reduce the effect of high variance between runs.

Cryptographic functions can take on the order of milliseconds to complete, with checking signatures demonstrated to be a particularly expensive operation (Table~\ref{tezostable}).

% ┌─────────────┬────────────┬─────────┬──────────┬──────────┬────────────┐
% │ Name        │   Time/Run │ mWd/Run │ mjWd/Run │ Prom/Run │ Percentage │
% ├─────────────┼────────────┼─────────┼──────────┼──────────┼────────────┤
% │ sign        │   427.87us │ 144.00w │          │          │     36.24  │
% │ check       │ 1_171.77us │  75.00w │          │          │     99.25  │
% │ agg_4       │   302.90us │ 484.00w │    0.15w │    0.15w │     25.66  │
% │ agg_check_4 │ 1_179.25us │  75.00w │          │          │     99.88  │
% │ agg_8       │   605.38us │ 944.00w │    0.35w │    0.35w │     51.28  │
% │ agg_check_8 │ 1_180.61us │  75.00w │          │          │    100.00  │
% └─────────────┴────────────┴─────────┴──────────┴──────────┴────────────┘

\section{HotStuff implementation benchmarks} \label{hotstuffbenchmarks}

\begin{figure}[h]
\centering
\resizebox{.8\textwidth}{!}{\input{images/heatmaps/timelatencyheatmap.pgf}}
\caption{Heatmaps showing the behaviour of the system under varying conditions.}
\label{heatmaps}
\end{figure}

\begin{figure}[h]
\centering
\resizebox{.6\textwidth}{!}{%% Creator: Matplotlib, PGF backend
%%
%% To include the figure in your LaTeX document, write
%%   \input{<filename>.pgf}
%%
%% Make sure the required packages are loaded in your preamble
%%   \usepackage{pgf}
%%
%% Also ensure that all the required font packages are loaded; for instance,
%% the lmodern package is sometimes necessary when using math font.
%%   \usepackage{lmodern}
%%
%% Figures using additional raster images can only be included by \input if
%% they are in the same directory as the main LaTeX file. For loading figures
%% from other directories you can use the `import` package
%%   \usepackage{import}
%%
%% and then include the figures with
%%   \import{<path to file>}{<filename>.pgf}
%%
%% Matplotlib used the following preamble
%%   
%%   \usepackage{fontspec}
%%   \setmainfont{DejaVuSerif.ttf}[Path=\detokenize{/opt/homebrew/lib/python3.10/site-packages/matplotlib/mpl-data/fonts/ttf/}]
%%   \setsansfont{DejaVuSans.ttf}[Path=\detokenize{/opt/homebrew/lib/python3.10/site-packages/matplotlib/mpl-data/fonts/ttf/}]
%%   \setmonofont{DejaVuSansMono.ttf}[Path=\detokenize{/opt/homebrew/lib/python3.10/site-packages/matplotlib/mpl-data/fonts/ttf/}]
%%   \makeatletter\@ifpackageloaded{underscore}{}{\usepackage[strings]{underscore}}\makeatother
%%
\begingroup%
\makeatletter%
\begin{pgfpicture}%
\pgfpathrectangle{\pgfpointorigin}{\pgfqpoint{6.400000in}{4.800000in}}%
\pgfusepath{use as bounding box, clip}%
\begin{pgfscope}%
\pgfsetbuttcap%
\pgfsetmiterjoin%
\definecolor{currentfill}{rgb}{1.000000,1.000000,1.000000}%
\pgfsetfillcolor{currentfill}%
\pgfsetlinewidth{0.000000pt}%
\definecolor{currentstroke}{rgb}{1.000000,1.000000,1.000000}%
\pgfsetstrokecolor{currentstroke}%
\pgfsetdash{}{0pt}%
\pgfpathmoveto{\pgfqpoint{0.000000in}{0.000000in}}%
\pgfpathlineto{\pgfqpoint{6.400000in}{0.000000in}}%
\pgfpathlineto{\pgfqpoint{6.400000in}{4.800000in}}%
\pgfpathlineto{\pgfqpoint{0.000000in}{4.800000in}}%
\pgfpathlineto{\pgfqpoint{0.000000in}{0.000000in}}%
\pgfpathclose%
\pgfusepath{fill}%
\end{pgfscope}%
\begin{pgfscope}%
\pgfsetbuttcap%
\pgfsetmiterjoin%
\definecolor{currentfill}{rgb}{1.000000,1.000000,1.000000}%
\pgfsetfillcolor{currentfill}%
\pgfsetlinewidth{0.000000pt}%
\definecolor{currentstroke}{rgb}{0.000000,0.000000,0.000000}%
\pgfsetstrokecolor{currentstroke}%
\pgfsetstrokeopacity{0.000000}%
\pgfsetdash{}{0pt}%
\pgfpathmoveto{\pgfqpoint{0.800000in}{0.528000in}}%
\pgfpathlineto{\pgfqpoint{5.760000in}{0.528000in}}%
\pgfpathlineto{\pgfqpoint{5.760000in}{4.224000in}}%
\pgfpathlineto{\pgfqpoint{0.800000in}{4.224000in}}%
\pgfpathlineto{\pgfqpoint{0.800000in}{0.528000in}}%
\pgfpathclose%
\pgfusepath{fill}%
\end{pgfscope}%
\begin{pgfscope}%
\pgfsetbuttcap%
\pgfsetroundjoin%
\definecolor{currentfill}{rgb}{0.000000,0.000000,0.000000}%
\pgfsetfillcolor{currentfill}%
\pgfsetlinewidth{0.803000pt}%
\definecolor{currentstroke}{rgb}{0.000000,0.000000,0.000000}%
\pgfsetstrokecolor{currentstroke}%
\pgfsetdash{}{0pt}%
\pgfsys@defobject{currentmarker}{\pgfqpoint{0.000000in}{-0.048611in}}{\pgfqpoint{0.000000in}{0.000000in}}{%
\pgfpathmoveto{\pgfqpoint{0.000000in}{0.000000in}}%
\pgfpathlineto{\pgfqpoint{0.000000in}{-0.048611in}}%
\pgfusepath{stroke,fill}%
}%
\begin{pgfscope}%
\pgfsys@transformshift{0.904991in}{0.528000in}%
\pgfsys@useobject{currentmarker}{}%
\end{pgfscope}%
\end{pgfscope}%
\begin{pgfscope}%
\definecolor{textcolor}{rgb}{0.000000,0.000000,0.000000}%
\pgfsetstrokecolor{textcolor}%
\pgfsetfillcolor{textcolor}%
\pgftext[x=0.904991in,y=0.430778in,,top]{\color{textcolor}\sffamily\fontsize{10.000000}{12.000000}\selectfont 50}%
\end{pgfscope}%
\begin{pgfscope}%
\pgfsetbuttcap%
\pgfsetroundjoin%
\definecolor{currentfill}{rgb}{0.000000,0.000000,0.000000}%
\pgfsetfillcolor{currentfill}%
\pgfsetlinewidth{0.803000pt}%
\definecolor{currentstroke}{rgb}{0.000000,0.000000,0.000000}%
\pgfsetstrokecolor{currentstroke}%
\pgfsetdash{}{0pt}%
\pgfsys@defobject{currentmarker}{\pgfqpoint{0.000000in}{-0.048611in}}{\pgfqpoint{0.000000in}{0.000000in}}{%
\pgfpathmoveto{\pgfqpoint{0.000000in}{0.000000in}}%
\pgfpathlineto{\pgfqpoint{0.000000in}{-0.048611in}}%
\pgfusepath{stroke,fill}%
}%
\begin{pgfscope}%
\pgfsys@transformshift{1.556726in}{0.528000in}%
\pgfsys@useobject{currentmarker}{}%
\end{pgfscope}%
\end{pgfscope}%
\begin{pgfscope}%
\definecolor{textcolor}{rgb}{0.000000,0.000000,0.000000}%
\pgfsetstrokecolor{textcolor}%
\pgfsetfillcolor{textcolor}%
\pgftext[x=1.556726in,y=0.430778in,,top]{\color{textcolor}\sffamily\fontsize{10.000000}{12.000000}\selectfont 100}%
\end{pgfscope}%
\begin{pgfscope}%
\pgfsetbuttcap%
\pgfsetroundjoin%
\definecolor{currentfill}{rgb}{0.000000,0.000000,0.000000}%
\pgfsetfillcolor{currentfill}%
\pgfsetlinewidth{0.803000pt}%
\definecolor{currentstroke}{rgb}{0.000000,0.000000,0.000000}%
\pgfsetstrokecolor{currentstroke}%
\pgfsetdash{}{0pt}%
\pgfsys@defobject{currentmarker}{\pgfqpoint{0.000000in}{-0.048611in}}{\pgfqpoint{0.000000in}{0.000000in}}{%
\pgfpathmoveto{\pgfqpoint{0.000000in}{0.000000in}}%
\pgfpathlineto{\pgfqpoint{0.000000in}{-0.048611in}}%
\pgfusepath{stroke,fill}%
}%
\begin{pgfscope}%
\pgfsys@transformshift{2.208460in}{0.528000in}%
\pgfsys@useobject{currentmarker}{}%
\end{pgfscope}%
\end{pgfscope}%
\begin{pgfscope}%
\definecolor{textcolor}{rgb}{0.000000,0.000000,0.000000}%
\pgfsetstrokecolor{textcolor}%
\pgfsetfillcolor{textcolor}%
\pgftext[x=2.208460in,y=0.430778in,,top]{\color{textcolor}\sffamily\fontsize{10.000000}{12.000000}\selectfont 150}%
\end{pgfscope}%
\begin{pgfscope}%
\pgfsetbuttcap%
\pgfsetroundjoin%
\definecolor{currentfill}{rgb}{0.000000,0.000000,0.000000}%
\pgfsetfillcolor{currentfill}%
\pgfsetlinewidth{0.803000pt}%
\definecolor{currentstroke}{rgb}{0.000000,0.000000,0.000000}%
\pgfsetstrokecolor{currentstroke}%
\pgfsetdash{}{0pt}%
\pgfsys@defobject{currentmarker}{\pgfqpoint{0.000000in}{-0.048611in}}{\pgfqpoint{0.000000in}{0.000000in}}{%
\pgfpathmoveto{\pgfqpoint{0.000000in}{0.000000in}}%
\pgfpathlineto{\pgfqpoint{0.000000in}{-0.048611in}}%
\pgfusepath{stroke,fill}%
}%
\begin{pgfscope}%
\pgfsys@transformshift{2.860195in}{0.528000in}%
\pgfsys@useobject{currentmarker}{}%
\end{pgfscope}%
\end{pgfscope}%
\begin{pgfscope}%
\definecolor{textcolor}{rgb}{0.000000,0.000000,0.000000}%
\pgfsetstrokecolor{textcolor}%
\pgfsetfillcolor{textcolor}%
\pgftext[x=2.860195in,y=0.430778in,,top]{\color{textcolor}\sffamily\fontsize{10.000000}{12.000000}\selectfont 200}%
\end{pgfscope}%
\begin{pgfscope}%
\pgfsetbuttcap%
\pgfsetroundjoin%
\definecolor{currentfill}{rgb}{0.000000,0.000000,0.000000}%
\pgfsetfillcolor{currentfill}%
\pgfsetlinewidth{0.803000pt}%
\definecolor{currentstroke}{rgb}{0.000000,0.000000,0.000000}%
\pgfsetstrokecolor{currentstroke}%
\pgfsetdash{}{0pt}%
\pgfsys@defobject{currentmarker}{\pgfqpoint{0.000000in}{-0.048611in}}{\pgfqpoint{0.000000in}{0.000000in}}{%
\pgfpathmoveto{\pgfqpoint{0.000000in}{0.000000in}}%
\pgfpathlineto{\pgfqpoint{0.000000in}{-0.048611in}}%
\pgfusepath{stroke,fill}%
}%
\begin{pgfscope}%
\pgfsys@transformshift{3.511930in}{0.528000in}%
\pgfsys@useobject{currentmarker}{}%
\end{pgfscope}%
\end{pgfscope}%
\begin{pgfscope}%
\definecolor{textcolor}{rgb}{0.000000,0.000000,0.000000}%
\pgfsetstrokecolor{textcolor}%
\pgfsetfillcolor{textcolor}%
\pgftext[x=3.511930in,y=0.430778in,,top]{\color{textcolor}\sffamily\fontsize{10.000000}{12.000000}\selectfont 250}%
\end{pgfscope}%
\begin{pgfscope}%
\pgfsetbuttcap%
\pgfsetroundjoin%
\definecolor{currentfill}{rgb}{0.000000,0.000000,0.000000}%
\pgfsetfillcolor{currentfill}%
\pgfsetlinewidth{0.803000pt}%
\definecolor{currentstroke}{rgb}{0.000000,0.000000,0.000000}%
\pgfsetstrokecolor{currentstroke}%
\pgfsetdash{}{0pt}%
\pgfsys@defobject{currentmarker}{\pgfqpoint{0.000000in}{-0.048611in}}{\pgfqpoint{0.000000in}{0.000000in}}{%
\pgfpathmoveto{\pgfqpoint{0.000000in}{0.000000in}}%
\pgfpathlineto{\pgfqpoint{0.000000in}{-0.048611in}}%
\pgfusepath{stroke,fill}%
}%
\begin{pgfscope}%
\pgfsys@transformshift{4.163665in}{0.528000in}%
\pgfsys@useobject{currentmarker}{}%
\end{pgfscope}%
\end{pgfscope}%
\begin{pgfscope}%
\definecolor{textcolor}{rgb}{0.000000,0.000000,0.000000}%
\pgfsetstrokecolor{textcolor}%
\pgfsetfillcolor{textcolor}%
\pgftext[x=4.163665in,y=0.430778in,,top]{\color{textcolor}\sffamily\fontsize{10.000000}{12.000000}\selectfont 300}%
\end{pgfscope}%
\begin{pgfscope}%
\pgfsetbuttcap%
\pgfsetroundjoin%
\definecolor{currentfill}{rgb}{0.000000,0.000000,0.000000}%
\pgfsetfillcolor{currentfill}%
\pgfsetlinewidth{0.803000pt}%
\definecolor{currentstroke}{rgb}{0.000000,0.000000,0.000000}%
\pgfsetstrokecolor{currentstroke}%
\pgfsetdash{}{0pt}%
\pgfsys@defobject{currentmarker}{\pgfqpoint{0.000000in}{-0.048611in}}{\pgfqpoint{0.000000in}{0.000000in}}{%
\pgfpathmoveto{\pgfqpoint{0.000000in}{0.000000in}}%
\pgfpathlineto{\pgfqpoint{0.000000in}{-0.048611in}}%
\pgfusepath{stroke,fill}%
}%
\begin{pgfscope}%
\pgfsys@transformshift{4.815400in}{0.528000in}%
\pgfsys@useobject{currentmarker}{}%
\end{pgfscope}%
\end{pgfscope}%
\begin{pgfscope}%
\definecolor{textcolor}{rgb}{0.000000,0.000000,0.000000}%
\pgfsetstrokecolor{textcolor}%
\pgfsetfillcolor{textcolor}%
\pgftext[x=4.815400in,y=0.430778in,,top]{\color{textcolor}\sffamily\fontsize{10.000000}{12.000000}\selectfont 350}%
\end{pgfscope}%
\begin{pgfscope}%
\pgfsetbuttcap%
\pgfsetroundjoin%
\definecolor{currentfill}{rgb}{0.000000,0.000000,0.000000}%
\pgfsetfillcolor{currentfill}%
\pgfsetlinewidth{0.803000pt}%
\definecolor{currentstroke}{rgb}{0.000000,0.000000,0.000000}%
\pgfsetstrokecolor{currentstroke}%
\pgfsetdash{}{0pt}%
\pgfsys@defobject{currentmarker}{\pgfqpoint{0.000000in}{-0.048611in}}{\pgfqpoint{0.000000in}{0.000000in}}{%
\pgfpathmoveto{\pgfqpoint{0.000000in}{0.000000in}}%
\pgfpathlineto{\pgfqpoint{0.000000in}{-0.048611in}}%
\pgfusepath{stroke,fill}%
}%
\begin{pgfscope}%
\pgfsys@transformshift{5.467135in}{0.528000in}%
\pgfsys@useobject{currentmarker}{}%
\end{pgfscope}%
\end{pgfscope}%
\begin{pgfscope}%
\definecolor{textcolor}{rgb}{0.000000,0.000000,0.000000}%
\pgfsetstrokecolor{textcolor}%
\pgfsetfillcolor{textcolor}%
\pgftext[x=5.467135in,y=0.430778in,,top]{\color{textcolor}\sffamily\fontsize{10.000000}{12.000000}\selectfont 400}%
\end{pgfscope}%
\begin{pgfscope}%
\definecolor{textcolor}{rgb}{0.000000,0.000000,0.000000}%
\pgfsetstrokecolor{textcolor}%
\pgfsetfillcolor{textcolor}%
\pgftext[x=3.280000in,y=0.240809in,,top]{\color{textcolor}\sffamily\fontsize{10.000000}{12.000000}\selectfont latency (ms)}%
\end{pgfscope}%
\begin{pgfscope}%
\pgfsetbuttcap%
\pgfsetroundjoin%
\definecolor{currentfill}{rgb}{0.000000,0.000000,0.000000}%
\pgfsetfillcolor{currentfill}%
\pgfsetlinewidth{0.803000pt}%
\definecolor{currentstroke}{rgb}{0.000000,0.000000,0.000000}%
\pgfsetstrokecolor{currentstroke}%
\pgfsetdash{}{0pt}%
\pgfsys@defobject{currentmarker}{\pgfqpoint{-0.048611in}{0.000000in}}{\pgfqpoint{-0.000000in}{0.000000in}}{%
\pgfpathmoveto{\pgfqpoint{-0.000000in}{0.000000in}}%
\pgfpathlineto{\pgfqpoint{-0.048611in}{0.000000in}}%
\pgfusepath{stroke,fill}%
}%
\begin{pgfscope}%
\pgfsys@transformshift{0.800000in}{0.528000in}%
\pgfsys@useobject{currentmarker}{}%
\end{pgfscope}%
\end{pgfscope}%
\begin{pgfscope}%
\definecolor{textcolor}{rgb}{0.000000,0.000000,0.000000}%
\pgfsetstrokecolor{textcolor}%
\pgfsetfillcolor{textcolor}%
\pgftext[x=0.481898in, y=0.475238in, left, base]{\color{textcolor}\sffamily\fontsize{10.000000}{12.000000}\selectfont 0.0}%
\end{pgfscope}%
\begin{pgfscope}%
\pgfsetbuttcap%
\pgfsetroundjoin%
\definecolor{currentfill}{rgb}{0.000000,0.000000,0.000000}%
\pgfsetfillcolor{currentfill}%
\pgfsetlinewidth{0.803000pt}%
\definecolor{currentstroke}{rgb}{0.000000,0.000000,0.000000}%
\pgfsetstrokecolor{currentstroke}%
\pgfsetdash{}{0pt}%
\pgfsys@defobject{currentmarker}{\pgfqpoint{-0.048611in}{0.000000in}}{\pgfqpoint{-0.000000in}{0.000000in}}{%
\pgfpathmoveto{\pgfqpoint{-0.000000in}{0.000000in}}%
\pgfpathlineto{\pgfqpoint{-0.048611in}{0.000000in}}%
\pgfusepath{stroke,fill}%
}%
\begin{pgfscope}%
\pgfsys@transformshift{0.800000in}{1.267200in}%
\pgfsys@useobject{currentmarker}{}%
\end{pgfscope}%
\end{pgfscope}%
\begin{pgfscope}%
\definecolor{textcolor}{rgb}{0.000000,0.000000,0.000000}%
\pgfsetstrokecolor{textcolor}%
\pgfsetfillcolor{textcolor}%
\pgftext[x=0.481898in, y=1.214438in, left, base]{\color{textcolor}\sffamily\fontsize{10.000000}{12.000000}\selectfont 0.2}%
\end{pgfscope}%
\begin{pgfscope}%
\pgfsetbuttcap%
\pgfsetroundjoin%
\definecolor{currentfill}{rgb}{0.000000,0.000000,0.000000}%
\pgfsetfillcolor{currentfill}%
\pgfsetlinewidth{0.803000pt}%
\definecolor{currentstroke}{rgb}{0.000000,0.000000,0.000000}%
\pgfsetstrokecolor{currentstroke}%
\pgfsetdash{}{0pt}%
\pgfsys@defobject{currentmarker}{\pgfqpoint{-0.048611in}{0.000000in}}{\pgfqpoint{-0.000000in}{0.000000in}}{%
\pgfpathmoveto{\pgfqpoint{-0.000000in}{0.000000in}}%
\pgfpathlineto{\pgfqpoint{-0.048611in}{0.000000in}}%
\pgfusepath{stroke,fill}%
}%
\begin{pgfscope}%
\pgfsys@transformshift{0.800000in}{2.006400in}%
\pgfsys@useobject{currentmarker}{}%
\end{pgfscope}%
\end{pgfscope}%
\begin{pgfscope}%
\definecolor{textcolor}{rgb}{0.000000,0.000000,0.000000}%
\pgfsetstrokecolor{textcolor}%
\pgfsetfillcolor{textcolor}%
\pgftext[x=0.481898in, y=1.953638in, left, base]{\color{textcolor}\sffamily\fontsize{10.000000}{12.000000}\selectfont 0.4}%
\end{pgfscope}%
\begin{pgfscope}%
\pgfsetbuttcap%
\pgfsetroundjoin%
\definecolor{currentfill}{rgb}{0.000000,0.000000,0.000000}%
\pgfsetfillcolor{currentfill}%
\pgfsetlinewidth{0.803000pt}%
\definecolor{currentstroke}{rgb}{0.000000,0.000000,0.000000}%
\pgfsetstrokecolor{currentstroke}%
\pgfsetdash{}{0pt}%
\pgfsys@defobject{currentmarker}{\pgfqpoint{-0.048611in}{0.000000in}}{\pgfqpoint{-0.000000in}{0.000000in}}{%
\pgfpathmoveto{\pgfqpoint{-0.000000in}{0.000000in}}%
\pgfpathlineto{\pgfqpoint{-0.048611in}{0.000000in}}%
\pgfusepath{stroke,fill}%
}%
\begin{pgfscope}%
\pgfsys@transformshift{0.800000in}{2.745600in}%
\pgfsys@useobject{currentmarker}{}%
\end{pgfscope}%
\end{pgfscope}%
\begin{pgfscope}%
\definecolor{textcolor}{rgb}{0.000000,0.000000,0.000000}%
\pgfsetstrokecolor{textcolor}%
\pgfsetfillcolor{textcolor}%
\pgftext[x=0.481898in, y=2.692838in, left, base]{\color{textcolor}\sffamily\fontsize{10.000000}{12.000000}\selectfont 0.6}%
\end{pgfscope}%
\begin{pgfscope}%
\pgfsetbuttcap%
\pgfsetroundjoin%
\definecolor{currentfill}{rgb}{0.000000,0.000000,0.000000}%
\pgfsetfillcolor{currentfill}%
\pgfsetlinewidth{0.803000pt}%
\definecolor{currentstroke}{rgb}{0.000000,0.000000,0.000000}%
\pgfsetstrokecolor{currentstroke}%
\pgfsetdash{}{0pt}%
\pgfsys@defobject{currentmarker}{\pgfqpoint{-0.048611in}{0.000000in}}{\pgfqpoint{-0.000000in}{0.000000in}}{%
\pgfpathmoveto{\pgfqpoint{-0.000000in}{0.000000in}}%
\pgfpathlineto{\pgfqpoint{-0.048611in}{0.000000in}}%
\pgfusepath{stroke,fill}%
}%
\begin{pgfscope}%
\pgfsys@transformshift{0.800000in}{3.484800in}%
\pgfsys@useobject{currentmarker}{}%
\end{pgfscope}%
\end{pgfscope}%
\begin{pgfscope}%
\definecolor{textcolor}{rgb}{0.000000,0.000000,0.000000}%
\pgfsetstrokecolor{textcolor}%
\pgfsetfillcolor{textcolor}%
\pgftext[x=0.481898in, y=3.432038in, left, base]{\color{textcolor}\sffamily\fontsize{10.000000}{12.000000}\selectfont 0.8}%
\end{pgfscope}%
\begin{pgfscope}%
\pgfsetbuttcap%
\pgfsetroundjoin%
\definecolor{currentfill}{rgb}{0.000000,0.000000,0.000000}%
\pgfsetfillcolor{currentfill}%
\pgfsetlinewidth{0.803000pt}%
\definecolor{currentstroke}{rgb}{0.000000,0.000000,0.000000}%
\pgfsetstrokecolor{currentstroke}%
\pgfsetdash{}{0pt}%
\pgfsys@defobject{currentmarker}{\pgfqpoint{-0.048611in}{0.000000in}}{\pgfqpoint{-0.000000in}{0.000000in}}{%
\pgfpathmoveto{\pgfqpoint{-0.000000in}{0.000000in}}%
\pgfpathlineto{\pgfqpoint{-0.048611in}{0.000000in}}%
\pgfusepath{stroke,fill}%
}%
\begin{pgfscope}%
\pgfsys@transformshift{0.800000in}{4.224000in}%
\pgfsys@useobject{currentmarker}{}%
\end{pgfscope}%
\end{pgfscope}%
\begin{pgfscope}%
\definecolor{textcolor}{rgb}{0.000000,0.000000,0.000000}%
\pgfsetstrokecolor{textcolor}%
\pgfsetfillcolor{textcolor}%
\pgftext[x=0.481898in, y=4.171238in, left, base]{\color{textcolor}\sffamily\fontsize{10.000000}{12.000000}\selectfont 1.0}%
\end{pgfscope}%
\begin{pgfscope}%
\definecolor{textcolor}{rgb}{0.000000,0.000000,0.000000}%
\pgfsetstrokecolor{textcolor}%
\pgfsetfillcolor{textcolor}%
\pgftext[x=0.426343in,y=2.376000in,,bottom,rotate=90.000000]{\color{textcolor}\sffamily\fontsize{10.000000}{12.000000}\selectfont proportion of requests}%
\end{pgfscope}%
\begin{pgfscope}%
\pgfpathrectangle{\pgfqpoint{0.800000in}{0.528000in}}{\pgfqpoint{4.960000in}{3.696000in}}%
\pgfusepath{clip}%
\pgfsetrectcap%
\pgfsetroundjoin%
\pgfsetlinewidth{1.505625pt}%
\definecolor{currentstroke}{rgb}{0.121569,0.466667,0.705882}%
\pgfsetstrokecolor{currentstroke}%
\pgfsetdash{}{0pt}%
\pgfpathmoveto{\pgfqpoint{1.025455in}{0.528000in}}%
\pgfpathlineto{\pgfqpoint{1.025455in}{0.528924in}}%
\pgfpathlineto{\pgfqpoint{1.114221in}{0.529848in}}%
\pgfpathlineto{\pgfqpoint{1.114221in}{0.530772in}}%
\pgfpathlineto{\pgfqpoint{1.135024in}{0.531696in}}%
\pgfpathlineto{\pgfqpoint{1.135024in}{0.532620in}}%
\pgfpathlineto{\pgfqpoint{1.171652in}{0.533544in}}%
\pgfpathlineto{\pgfqpoint{1.171652in}{0.534468in}}%
\pgfpathlineto{\pgfqpoint{1.202903in}{0.535392in}}%
\pgfpathlineto{\pgfqpoint{1.202903in}{0.536316in}}%
\pgfpathlineto{\pgfqpoint{1.221519in}{0.537240in}}%
\pgfpathlineto{\pgfqpoint{1.221519in}{0.538164in}}%
\pgfpathlineto{\pgfqpoint{1.230429in}{0.539088in}}%
\pgfpathlineto{\pgfqpoint{1.230429in}{0.540012in}}%
\pgfpathlineto{\pgfqpoint{1.236865in}{0.540936in}}%
\pgfpathlineto{\pgfqpoint{1.236865in}{0.541860in}}%
\pgfpathlineto{\pgfqpoint{1.253381in}{0.542784in}}%
\pgfpathlineto{\pgfqpoint{1.253381in}{0.543708in}}%
\pgfpathlineto{\pgfqpoint{1.262605in}{0.544632in}}%
\pgfpathlineto{\pgfqpoint{1.262605in}{0.545556in}}%
\pgfpathlineto{\pgfqpoint{1.273253in}{0.546480in}}%
\pgfpathlineto{\pgfqpoint{1.273253in}{0.547404in}}%
\pgfpathlineto{\pgfqpoint{1.285754in}{0.548328in}}%
\pgfpathlineto{\pgfqpoint{1.285754in}{0.549252in}}%
\pgfpathlineto{\pgfqpoint{1.293865in}{0.550176in}}%
\pgfpathlineto{\pgfqpoint{1.294620in}{0.552024in}}%
\pgfpathlineto{\pgfqpoint{1.307640in}{0.552948in}}%
\pgfpathlineto{\pgfqpoint{1.307640in}{0.553872in}}%
\pgfpathlineto{\pgfqpoint{1.312988in}{0.554796in}}%
\pgfpathlineto{\pgfqpoint{1.312988in}{0.555720in}}%
\pgfpathlineto{\pgfqpoint{1.318718in}{0.556644in}}%
\pgfpathlineto{\pgfqpoint{1.318718in}{0.557568in}}%
\pgfpathlineto{\pgfqpoint{1.322365in}{0.558492in}}%
\pgfpathlineto{\pgfqpoint{1.322365in}{0.559416in}}%
\pgfpathlineto{\pgfqpoint{1.342432in}{0.560340in}}%
\pgfpathlineto{\pgfqpoint{1.343499in}{0.562188in}}%
\pgfpathlineto{\pgfqpoint{1.353280in}{0.563112in}}%
\pgfpathlineto{\pgfqpoint{1.353280in}{0.564036in}}%
\pgfpathlineto{\pgfqpoint{1.356085in}{0.564960in}}%
\pgfpathlineto{\pgfqpoint{1.356085in}{0.565884in}}%
\pgfpathlineto{\pgfqpoint{1.360489in}{0.566808in}}%
\pgfpathlineto{\pgfqpoint{1.361423in}{0.568656in}}%
\pgfpathlineto{\pgfqpoint{1.363926in}{0.569580in}}%
\pgfpathlineto{\pgfqpoint{1.364035in}{0.571428in}}%
\pgfpathlineto{\pgfqpoint{1.368887in}{0.572352in}}%
\pgfpathlineto{\pgfqpoint{1.368887in}{0.573276in}}%
\pgfpathlineto{\pgfqpoint{1.376083in}{0.574200in}}%
\pgfpathlineto{\pgfqpoint{1.377100in}{0.576972in}}%
\pgfpathlineto{\pgfqpoint{1.380830in}{0.577896in}}%
\pgfpathlineto{\pgfqpoint{1.381443in}{0.579744in}}%
\pgfpathlineto{\pgfqpoint{1.383264in}{0.580668in}}%
\pgfpathlineto{\pgfqpoint{1.383264in}{0.581592in}}%
\pgfpathlineto{\pgfqpoint{1.387455in}{0.582516in}}%
\pgfpathlineto{\pgfqpoint{1.387455in}{0.583440in}}%
\pgfpathlineto{\pgfqpoint{1.393968in}{0.584364in}}%
\pgfpathlineto{\pgfqpoint{1.393968in}{0.585288in}}%
\pgfpathlineto{\pgfqpoint{1.396561in}{0.586212in}}%
\pgfpathlineto{\pgfqpoint{1.396561in}{0.587136in}}%
\pgfpathlineto{\pgfqpoint{1.400493in}{0.588060in}}%
\pgfpathlineto{\pgfqpoint{1.400493in}{0.588984in}}%
\pgfpathlineto{\pgfqpoint{1.406651in}{0.589908in}}%
\pgfpathlineto{\pgfqpoint{1.407058in}{0.591756in}}%
\pgfpathlineto{\pgfqpoint{1.414764in}{0.592680in}}%
\pgfpathlineto{\pgfqpoint{1.414764in}{0.593604in}}%
\pgfpathlineto{\pgfqpoint{1.420370in}{0.594528in}}%
\pgfpathlineto{\pgfqpoint{1.420801in}{0.597300in}}%
\pgfpathlineto{\pgfqpoint{1.426127in}{0.598224in}}%
\pgfpathlineto{\pgfqpoint{1.426127in}{0.599148in}}%
\pgfpathlineto{\pgfqpoint{1.433364in}{0.600072in}}%
\pgfpathlineto{\pgfqpoint{1.434044in}{0.606540in}}%
\pgfpathlineto{\pgfqpoint{1.437410in}{0.607464in}}%
\pgfpathlineto{\pgfqpoint{1.438064in}{0.609312in}}%
\pgfpathlineto{\pgfqpoint{1.439340in}{0.610236in}}%
\pgfpathlineto{\pgfqpoint{1.440050in}{0.613008in}}%
\pgfpathlineto{\pgfqpoint{1.441849in}{0.613932in}}%
\pgfpathlineto{\pgfqpoint{1.442693in}{0.615780in}}%
\pgfpathlineto{\pgfqpoint{1.444569in}{0.616704in}}%
\pgfpathlineto{\pgfqpoint{1.445170in}{0.618552in}}%
\pgfpathlineto{\pgfqpoint{1.449190in}{0.619476in}}%
\pgfpathlineto{\pgfqpoint{1.449892in}{0.621324in}}%
\pgfpathlineto{\pgfqpoint{1.454997in}{0.622248in}}%
\pgfpathlineto{\pgfqpoint{1.454997in}{0.623172in}}%
\pgfpathlineto{\pgfqpoint{1.457766in}{0.624096in}}%
\pgfpathlineto{\pgfqpoint{1.458347in}{0.626868in}}%
\pgfpathlineto{\pgfqpoint{1.462118in}{0.627792in}}%
\pgfpathlineto{\pgfqpoint{1.462640in}{0.629640in}}%
\pgfpathlineto{\pgfqpoint{1.465567in}{0.630564in}}%
\pgfpathlineto{\pgfqpoint{1.466610in}{0.632412in}}%
\pgfpathlineto{\pgfqpoint{1.468369in}{0.633336in}}%
\pgfpathlineto{\pgfqpoint{1.468369in}{0.634260in}}%
\pgfpathlineto{\pgfqpoint{1.470371in}{0.635184in}}%
\pgfpathlineto{\pgfqpoint{1.470371in}{0.636108in}}%
\pgfpathlineto{\pgfqpoint{1.474144in}{0.637032in}}%
\pgfpathlineto{\pgfqpoint{1.474323in}{0.638880in}}%
\pgfpathlineto{\pgfqpoint{1.476471in}{0.639804in}}%
\pgfpathlineto{\pgfqpoint{1.477295in}{0.643500in}}%
\pgfpathlineto{\pgfqpoint{1.478157in}{0.644424in}}%
\pgfpathlineto{\pgfqpoint{1.478577in}{0.646272in}}%
\pgfpathlineto{\pgfqpoint{1.482039in}{0.647196in}}%
\pgfpathlineto{\pgfqpoint{1.482039in}{0.648120in}}%
\pgfpathlineto{\pgfqpoint{1.485179in}{0.649044in}}%
\pgfpathlineto{\pgfqpoint{1.486183in}{0.650892in}}%
\pgfpathlineto{\pgfqpoint{1.489317in}{0.651816in}}%
\pgfpathlineto{\pgfqpoint{1.489727in}{0.654588in}}%
\pgfpathlineto{\pgfqpoint{1.492921in}{0.655512in}}%
\pgfpathlineto{\pgfqpoint{1.493159in}{0.657360in}}%
\pgfpathlineto{\pgfqpoint{1.495608in}{0.658284in}}%
\pgfpathlineto{\pgfqpoint{1.495608in}{0.659208in}}%
\pgfpathlineto{\pgfqpoint{1.497488in}{0.660132in}}%
\pgfpathlineto{\pgfqpoint{1.497488in}{0.661056in}}%
\pgfpathlineto{\pgfqpoint{1.499228in}{0.661980in}}%
\pgfpathlineto{\pgfqpoint{1.499743in}{0.664752in}}%
\pgfpathlineto{\pgfqpoint{1.501102in}{0.665676in}}%
\pgfpathlineto{\pgfqpoint{1.501681in}{0.669372in}}%
\pgfpathlineto{\pgfqpoint{1.503266in}{0.670296in}}%
\pgfpathlineto{\pgfqpoint{1.504080in}{0.673992in}}%
\pgfpathlineto{\pgfqpoint{1.504606in}{0.674916in}}%
\pgfpathlineto{\pgfqpoint{1.505160in}{0.676764in}}%
\pgfpathlineto{\pgfqpoint{1.508056in}{0.677688in}}%
\pgfpathlineto{\pgfqpoint{1.509044in}{0.679536in}}%
\pgfpathlineto{\pgfqpoint{1.510124in}{0.680460in}}%
\pgfpathlineto{\pgfqpoint{1.510433in}{0.682308in}}%
\pgfpathlineto{\pgfqpoint{1.511695in}{0.683232in}}%
\pgfpathlineto{\pgfqpoint{1.512685in}{0.686004in}}%
\pgfpathlineto{\pgfqpoint{1.515681in}{0.686928in}}%
\pgfpathlineto{\pgfqpoint{1.516541in}{0.689700in}}%
\pgfpathlineto{\pgfqpoint{1.517357in}{0.690624in}}%
\pgfpathlineto{\pgfqpoint{1.518130in}{0.692472in}}%
\pgfpathlineto{\pgfqpoint{1.520004in}{0.693396in}}%
\pgfpathlineto{\pgfqpoint{1.520613in}{0.695244in}}%
\pgfpathlineto{\pgfqpoint{1.524107in}{0.696168in}}%
\pgfpathlineto{\pgfqpoint{1.524166in}{0.698016in}}%
\pgfpathlineto{\pgfqpoint{1.526730in}{0.698940in}}%
\pgfpathlineto{\pgfqpoint{1.526989in}{0.700788in}}%
\pgfpathlineto{\pgfqpoint{1.528579in}{0.701712in}}%
\pgfpathlineto{\pgfqpoint{1.529305in}{0.703560in}}%
\pgfpathlineto{\pgfqpoint{1.530137in}{0.704484in}}%
\pgfpathlineto{\pgfqpoint{1.530687in}{0.707256in}}%
\pgfpathlineto{\pgfqpoint{1.531902in}{0.708180in}}%
\pgfpathlineto{\pgfqpoint{1.531902in}{0.709104in}}%
\pgfpathlineto{\pgfqpoint{1.534677in}{0.710028in}}%
\pgfpathlineto{\pgfqpoint{1.535140in}{0.712800in}}%
\pgfpathlineto{\pgfqpoint{1.537144in}{0.713724in}}%
\pgfpathlineto{\pgfqpoint{1.538143in}{0.717420in}}%
\pgfpathlineto{\pgfqpoint{1.538590in}{0.718344in}}%
\pgfpathlineto{\pgfqpoint{1.538590in}{0.719268in}}%
\pgfpathlineto{\pgfqpoint{1.542607in}{0.720192in}}%
\pgfpathlineto{\pgfqpoint{1.542607in}{0.721116in}}%
\pgfpathlineto{\pgfqpoint{1.545855in}{0.722040in}}%
\pgfpathlineto{\pgfqpoint{1.546546in}{0.724812in}}%
\pgfpathlineto{\pgfqpoint{1.549319in}{0.725736in}}%
\pgfpathlineto{\pgfqpoint{1.550250in}{0.728508in}}%
\pgfpathlineto{\pgfqpoint{1.552905in}{0.729432in}}%
\pgfpathlineto{\pgfqpoint{1.553913in}{0.732204in}}%
\pgfpathlineto{\pgfqpoint{1.555773in}{0.733128in}}%
\pgfpathlineto{\pgfqpoint{1.556657in}{0.738672in}}%
\pgfpathlineto{\pgfqpoint{1.558389in}{0.739596in}}%
\pgfpathlineto{\pgfqpoint{1.558865in}{0.743292in}}%
\pgfpathlineto{\pgfqpoint{1.561037in}{0.744216in}}%
\pgfpathlineto{\pgfqpoint{1.561884in}{0.746988in}}%
\pgfpathlineto{\pgfqpoint{1.563458in}{0.747912in}}%
\pgfpathlineto{\pgfqpoint{1.563948in}{0.749760in}}%
\pgfpathlineto{\pgfqpoint{1.565122in}{0.750684in}}%
\pgfpathlineto{\pgfqpoint{1.566189in}{0.756228in}}%
\pgfpathlineto{\pgfqpoint{1.567700in}{0.757152in}}%
\pgfpathlineto{\pgfqpoint{1.568636in}{0.759000in}}%
\pgfpathlineto{\pgfqpoint{1.569342in}{0.759924in}}%
\pgfpathlineto{\pgfqpoint{1.570216in}{0.764544in}}%
\pgfpathlineto{\pgfqpoint{1.571087in}{0.765468in}}%
\pgfpathlineto{\pgfqpoint{1.572025in}{0.769164in}}%
\pgfpathlineto{\pgfqpoint{1.573149in}{0.770088in}}%
\pgfpathlineto{\pgfqpoint{1.573973in}{0.773784in}}%
\pgfpathlineto{\pgfqpoint{1.574502in}{0.774708in}}%
\pgfpathlineto{\pgfqpoint{1.575382in}{0.776556in}}%
\pgfpathlineto{\pgfqpoint{1.576249in}{0.777480in}}%
\pgfpathlineto{\pgfqpoint{1.577173in}{0.780252in}}%
\pgfpathlineto{\pgfqpoint{1.578185in}{0.781176in}}%
\pgfpathlineto{\pgfqpoint{1.578881in}{0.784872in}}%
\pgfpathlineto{\pgfqpoint{1.580290in}{0.785796in}}%
\pgfpathlineto{\pgfqpoint{1.581082in}{0.788568in}}%
\pgfpathlineto{\pgfqpoint{1.582161in}{0.789492in}}%
\pgfpathlineto{\pgfqpoint{1.583161in}{0.792264in}}%
\pgfpathlineto{\pgfqpoint{1.583534in}{0.793188in}}%
\pgfpathlineto{\pgfqpoint{1.584413in}{0.795036in}}%
\pgfpathlineto{\pgfqpoint{1.584686in}{0.795960in}}%
\pgfpathlineto{\pgfqpoint{1.584926in}{0.797808in}}%
\pgfpathlineto{\pgfqpoint{1.586408in}{0.798732in}}%
\pgfpathlineto{\pgfqpoint{1.586985in}{0.801504in}}%
\pgfpathlineto{\pgfqpoint{1.587795in}{0.802428in}}%
\pgfpathlineto{\pgfqpoint{1.587795in}{0.803352in}}%
\pgfpathlineto{\pgfqpoint{1.591216in}{0.804276in}}%
\pgfpathlineto{\pgfqpoint{1.591475in}{0.806124in}}%
\pgfpathlineto{\pgfqpoint{1.593367in}{0.807048in}}%
\pgfpathlineto{\pgfqpoint{1.594030in}{0.808896in}}%
\pgfpathlineto{\pgfqpoint{1.595558in}{0.809820in}}%
\pgfpathlineto{\pgfqpoint{1.595647in}{0.812592in}}%
\pgfpathlineto{\pgfqpoint{1.597588in}{0.813516in}}%
\pgfpathlineto{\pgfqpoint{1.598444in}{0.815364in}}%
\pgfpathlineto{\pgfqpoint{1.599598in}{0.816288in}}%
\pgfpathlineto{\pgfqpoint{1.600580in}{0.820908in}}%
\pgfpathlineto{\pgfqpoint{1.602453in}{0.821832in}}%
\pgfpathlineto{\pgfqpoint{1.603548in}{0.826452in}}%
\pgfpathlineto{\pgfqpoint{1.605997in}{0.827376in}}%
\pgfpathlineto{\pgfqpoint{1.606998in}{0.829224in}}%
\pgfpathlineto{\pgfqpoint{1.607947in}{0.830148in}}%
\pgfpathlineto{\pgfqpoint{1.608616in}{0.836616in}}%
\pgfpathlineto{\pgfqpoint{1.610787in}{0.837540in}}%
\pgfpathlineto{\pgfqpoint{1.611398in}{0.840312in}}%
\pgfpathlineto{\pgfqpoint{1.615561in}{0.841236in}}%
\pgfpathlineto{\pgfqpoint{1.616669in}{0.845856in}}%
\pgfpathlineto{\pgfqpoint{1.618138in}{0.846780in}}%
\pgfpathlineto{\pgfqpoint{1.619070in}{0.849552in}}%
\pgfpathlineto{\pgfqpoint{1.620537in}{0.850476in}}%
\pgfpathlineto{\pgfqpoint{1.621516in}{0.852324in}}%
\pgfpathlineto{\pgfqpoint{1.624807in}{0.853248in}}%
\pgfpathlineto{\pgfqpoint{1.625578in}{0.859716in}}%
\pgfpathlineto{\pgfqpoint{1.627085in}{0.860640in}}%
\pgfpathlineto{\pgfqpoint{1.628037in}{0.866184in}}%
\pgfpathlineto{\pgfqpoint{1.628830in}{0.867108in}}%
\pgfpathlineto{\pgfqpoint{1.629715in}{0.868956in}}%
\pgfpathlineto{\pgfqpoint{1.630398in}{0.869880in}}%
\pgfpathlineto{\pgfqpoint{1.631471in}{0.878196in}}%
\pgfpathlineto{\pgfqpoint{1.632416in}{0.879120in}}%
\pgfpathlineto{\pgfqpoint{1.633481in}{0.883740in}}%
\pgfpathlineto{\pgfqpoint{1.634344in}{0.884664in}}%
\pgfpathlineto{\pgfqpoint{1.635300in}{0.888360in}}%
\pgfpathlineto{\pgfqpoint{1.636834in}{0.889284in}}%
\pgfpathlineto{\pgfqpoint{1.637918in}{0.895752in}}%
\pgfpathlineto{\pgfqpoint{1.638743in}{0.896676in}}%
\pgfpathlineto{\pgfqpoint{1.639814in}{0.899448in}}%
\pgfpathlineto{\pgfqpoint{1.640479in}{0.900372in}}%
\pgfpathlineto{\pgfqpoint{1.641399in}{0.904068in}}%
\pgfpathlineto{\pgfqpoint{1.642517in}{0.904992in}}%
\pgfpathlineto{\pgfqpoint{1.643574in}{0.907764in}}%
\pgfpathlineto{\pgfqpoint{1.644344in}{0.908688in}}%
\pgfpathlineto{\pgfqpoint{1.645445in}{0.912384in}}%
\pgfpathlineto{\pgfqpoint{1.646658in}{0.913308in}}%
\pgfpathlineto{\pgfqpoint{1.646734in}{0.915156in}}%
\pgfpathlineto{\pgfqpoint{1.649803in}{0.916080in}}%
\pgfpathlineto{\pgfqpoint{1.650908in}{0.921624in}}%
\pgfpathlineto{\pgfqpoint{1.651463in}{0.922548in}}%
\pgfpathlineto{\pgfqpoint{1.651731in}{0.924396in}}%
\pgfpathlineto{\pgfqpoint{1.653312in}{0.925320in}}%
\pgfpathlineto{\pgfqpoint{1.653901in}{0.928092in}}%
\pgfpathlineto{\pgfqpoint{1.654642in}{0.929016in}}%
\pgfpathlineto{\pgfqpoint{1.655485in}{0.931788in}}%
\pgfpathlineto{\pgfqpoint{1.656012in}{0.932712in}}%
\pgfpathlineto{\pgfqpoint{1.657036in}{0.935484in}}%
\pgfpathlineto{\pgfqpoint{1.658616in}{0.936408in}}%
\pgfpathlineto{\pgfqpoint{1.659081in}{0.938256in}}%
\pgfpathlineto{\pgfqpoint{1.661114in}{0.939180in}}%
\pgfpathlineto{\pgfqpoint{1.661782in}{0.941952in}}%
\pgfpathlineto{\pgfqpoint{1.662639in}{0.942876in}}%
\pgfpathlineto{\pgfqpoint{1.663740in}{0.948420in}}%
\pgfpathlineto{\pgfqpoint{1.664618in}{0.949344in}}%
\pgfpathlineto{\pgfqpoint{1.664647in}{0.951192in}}%
\pgfpathlineto{\pgfqpoint{1.666044in}{0.952116in}}%
\pgfpathlineto{\pgfqpoint{1.666945in}{0.955812in}}%
\pgfpathlineto{\pgfqpoint{1.668813in}{0.956736in}}%
\pgfpathlineto{\pgfqpoint{1.669344in}{0.958584in}}%
\pgfpathlineto{\pgfqpoint{1.670235in}{0.959508in}}%
\pgfpathlineto{\pgfqpoint{1.670672in}{0.961356in}}%
\pgfpathlineto{\pgfqpoint{1.672574in}{0.962280in}}%
\pgfpathlineto{\pgfqpoint{1.673148in}{0.965976in}}%
\pgfpathlineto{\pgfqpoint{1.674842in}{0.966900in}}%
\pgfpathlineto{\pgfqpoint{1.674842in}{0.967824in}}%
\pgfpathlineto{\pgfqpoint{1.677923in}{0.968748in}}%
\pgfpathlineto{\pgfqpoint{1.678952in}{0.973368in}}%
\pgfpathlineto{\pgfqpoint{1.679813in}{0.974292in}}%
\pgfpathlineto{\pgfqpoint{1.680920in}{0.977064in}}%
\pgfpathlineto{\pgfqpoint{1.681329in}{0.977988in}}%
\pgfpathlineto{\pgfqpoint{1.681619in}{0.979836in}}%
\pgfpathlineto{\pgfqpoint{1.683541in}{0.980760in}}%
\pgfpathlineto{\pgfqpoint{1.684379in}{0.984456in}}%
\pgfpathlineto{\pgfqpoint{1.685495in}{0.985380in}}%
\pgfpathlineto{\pgfqpoint{1.686401in}{0.987228in}}%
\pgfpathlineto{\pgfqpoint{1.686794in}{0.988152in}}%
\pgfpathlineto{\pgfqpoint{1.687823in}{0.991848in}}%
\pgfpathlineto{\pgfqpoint{1.688389in}{0.992772in}}%
\pgfpathlineto{\pgfqpoint{1.689145in}{0.998316in}}%
\pgfpathlineto{\pgfqpoint{1.690482in}{0.999240in}}%
\pgfpathlineto{\pgfqpoint{1.690619in}{1.001088in}}%
\pgfpathlineto{\pgfqpoint{1.693435in}{1.002012in}}%
\pgfpathlineto{\pgfqpoint{1.694536in}{1.004784in}}%
\pgfpathlineto{\pgfqpoint{1.695184in}{1.005708in}}%
\pgfpathlineto{\pgfqpoint{1.696228in}{1.010328in}}%
\pgfpathlineto{\pgfqpoint{1.696415in}{1.011252in}}%
\pgfpathlineto{\pgfqpoint{1.697354in}{1.016796in}}%
\pgfpathlineto{\pgfqpoint{1.698147in}{1.017720in}}%
\pgfpathlineto{\pgfqpoint{1.698953in}{1.021416in}}%
\pgfpathlineto{\pgfqpoint{1.699505in}{1.022340in}}%
\pgfpathlineto{\pgfqpoint{1.700395in}{1.029732in}}%
\pgfpathlineto{\pgfqpoint{1.700850in}{1.030656in}}%
\pgfpathlineto{\pgfqpoint{1.701280in}{1.033428in}}%
\pgfpathlineto{\pgfqpoint{1.702976in}{1.034352in}}%
\pgfpathlineto{\pgfqpoint{1.703605in}{1.038048in}}%
\pgfpathlineto{\pgfqpoint{1.704880in}{1.038972in}}%
\pgfpathlineto{\pgfqpoint{1.705745in}{1.043592in}}%
\pgfpathlineto{\pgfqpoint{1.706284in}{1.044516in}}%
\pgfpathlineto{\pgfqpoint{1.707117in}{1.051908in}}%
\pgfpathlineto{\pgfqpoint{1.709164in}{1.052832in}}%
\pgfpathlineto{\pgfqpoint{1.709480in}{1.055604in}}%
\pgfpathlineto{\pgfqpoint{1.710511in}{1.056528in}}%
\pgfpathlineto{\pgfqpoint{1.711412in}{1.062072in}}%
\pgfpathlineto{\pgfqpoint{1.712139in}{1.062996in}}%
\pgfpathlineto{\pgfqpoint{1.712358in}{1.065768in}}%
\pgfpathlineto{\pgfqpoint{1.714069in}{1.066692in}}%
\pgfpathlineto{\pgfqpoint{1.714991in}{1.069464in}}%
\pgfpathlineto{\pgfqpoint{1.716069in}{1.070388in}}%
\pgfpathlineto{\pgfqpoint{1.716938in}{1.073160in}}%
\pgfpathlineto{\pgfqpoint{1.717588in}{1.074084in}}%
\pgfpathlineto{\pgfqpoint{1.718070in}{1.076856in}}%
\pgfpathlineto{\pgfqpoint{1.719409in}{1.077780in}}%
\pgfpathlineto{\pgfqpoint{1.719849in}{1.081476in}}%
\pgfpathlineto{\pgfqpoint{1.720726in}{1.082400in}}%
\pgfpathlineto{\pgfqpoint{1.720959in}{1.085172in}}%
\pgfpathlineto{\pgfqpoint{1.722974in}{1.086096in}}%
\pgfpathlineto{\pgfqpoint{1.724070in}{1.088868in}}%
\pgfpathlineto{\pgfqpoint{1.724600in}{1.089792in}}%
\pgfpathlineto{\pgfqpoint{1.725582in}{1.093488in}}%
\pgfpathlineto{\pgfqpoint{1.726220in}{1.094412in}}%
\pgfpathlineto{\pgfqpoint{1.726546in}{1.098108in}}%
\pgfpathlineto{\pgfqpoint{1.727910in}{1.099032in}}%
\pgfpathlineto{\pgfqpoint{1.728493in}{1.103652in}}%
\pgfpathlineto{\pgfqpoint{1.729179in}{1.104576in}}%
\pgfpathlineto{\pgfqpoint{1.729179in}{1.105500in}}%
\pgfpathlineto{\pgfqpoint{1.730785in}{1.106424in}}%
\pgfpathlineto{\pgfqpoint{1.731711in}{1.109196in}}%
\pgfpathlineto{\pgfqpoint{1.732930in}{1.110120in}}%
\pgfpathlineto{\pgfqpoint{1.733721in}{1.112892in}}%
\pgfpathlineto{\pgfqpoint{1.734789in}{1.113816in}}%
\pgfpathlineto{\pgfqpoint{1.735448in}{1.117512in}}%
\pgfpathlineto{\pgfqpoint{1.738777in}{1.118436in}}%
\pgfpathlineto{\pgfqpoint{1.739821in}{1.121208in}}%
\pgfpathlineto{\pgfqpoint{1.740197in}{1.122132in}}%
\pgfpathlineto{\pgfqpoint{1.740197in}{1.123056in}}%
\pgfpathlineto{\pgfqpoint{1.741935in}{1.123980in}}%
\pgfpathlineto{\pgfqpoint{1.742381in}{1.125828in}}%
\pgfpathlineto{\pgfqpoint{1.745243in}{1.126752in}}%
\pgfpathlineto{\pgfqpoint{1.745644in}{1.128600in}}%
\pgfpathlineto{\pgfqpoint{1.746706in}{1.129524in}}%
\pgfpathlineto{\pgfqpoint{1.746706in}{1.130448in}}%
\pgfpathlineto{\pgfqpoint{1.748471in}{1.131372in}}%
\pgfpathlineto{\pgfqpoint{1.749574in}{1.134144in}}%
\pgfpathlineto{\pgfqpoint{1.749912in}{1.135068in}}%
\pgfpathlineto{\pgfqpoint{1.750929in}{1.142460in}}%
\pgfpathlineto{\pgfqpoint{1.751191in}{1.143384in}}%
\pgfpathlineto{\pgfqpoint{1.752145in}{1.147080in}}%
\pgfpathlineto{\pgfqpoint{1.753627in}{1.148004in}}%
\pgfpathlineto{\pgfqpoint{1.753840in}{1.150776in}}%
\pgfpathlineto{\pgfqpoint{1.754997in}{1.151700in}}%
\pgfpathlineto{\pgfqpoint{1.756041in}{1.159092in}}%
\pgfpathlineto{\pgfqpoint{1.756700in}{1.160016in}}%
\pgfpathlineto{\pgfqpoint{1.757668in}{1.164636in}}%
\pgfpathlineto{\pgfqpoint{1.758179in}{1.165560in}}%
\pgfpathlineto{\pgfqpoint{1.759009in}{1.169256in}}%
\pgfpathlineto{\pgfqpoint{1.759863in}{1.170180in}}%
\pgfpathlineto{\pgfqpoint{1.760651in}{1.172028in}}%
\pgfpathlineto{\pgfqpoint{1.762399in}{1.172952in}}%
\pgfpathlineto{\pgfqpoint{1.763371in}{1.176648in}}%
\pgfpathlineto{\pgfqpoint{1.764880in}{1.177572in}}%
\pgfpathlineto{\pgfqpoint{1.765841in}{1.183116in}}%
\pgfpathlineto{\pgfqpoint{1.766078in}{1.184040in}}%
\pgfpathlineto{\pgfqpoint{1.767072in}{1.190508in}}%
\pgfpathlineto{\pgfqpoint{1.768140in}{1.191432in}}%
\pgfpathlineto{\pgfqpoint{1.769018in}{1.195128in}}%
\pgfpathlineto{\pgfqpoint{1.769350in}{1.196052in}}%
\pgfpathlineto{\pgfqpoint{1.770423in}{1.201596in}}%
\pgfpathlineto{\pgfqpoint{1.771225in}{1.202520in}}%
\pgfpathlineto{\pgfqpoint{1.772257in}{1.207140in}}%
\pgfpathlineto{\pgfqpoint{1.772892in}{1.208064in}}%
\pgfpathlineto{\pgfqpoint{1.773895in}{1.210836in}}%
\pgfpathlineto{\pgfqpoint{1.774755in}{1.211760in}}%
\pgfpathlineto{\pgfqpoint{1.775862in}{1.214532in}}%
\pgfpathlineto{\pgfqpoint{1.776174in}{1.215456in}}%
\pgfpathlineto{\pgfqpoint{1.776713in}{1.218228in}}%
\pgfpathlineto{\pgfqpoint{1.777798in}{1.219152in}}%
\pgfpathlineto{\pgfqpoint{1.778761in}{1.222848in}}%
\pgfpathlineto{\pgfqpoint{1.779515in}{1.223772in}}%
\pgfpathlineto{\pgfqpoint{1.779999in}{1.227468in}}%
\pgfpathlineto{\pgfqpoint{1.780751in}{1.228392in}}%
\pgfpathlineto{\pgfqpoint{1.781795in}{1.232088in}}%
\pgfpathlineto{\pgfqpoint{1.782438in}{1.233012in}}%
\pgfpathlineto{\pgfqpoint{1.783117in}{1.234860in}}%
\pgfpathlineto{\pgfqpoint{1.783925in}{1.235784in}}%
\pgfpathlineto{\pgfqpoint{1.784580in}{1.238556in}}%
\pgfpathlineto{\pgfqpoint{1.786858in}{1.239480in}}%
\pgfpathlineto{\pgfqpoint{1.787599in}{1.243176in}}%
\pgfpathlineto{\pgfqpoint{1.788637in}{1.244100in}}%
\pgfpathlineto{\pgfqpoint{1.789684in}{1.246872in}}%
\pgfpathlineto{\pgfqpoint{1.791215in}{1.247796in}}%
\pgfpathlineto{\pgfqpoint{1.792129in}{1.251492in}}%
\pgfpathlineto{\pgfqpoint{1.793685in}{1.252416in}}%
\pgfpathlineto{\pgfqpoint{1.794595in}{1.257036in}}%
\pgfpathlineto{\pgfqpoint{1.795767in}{1.257960in}}%
\pgfpathlineto{\pgfqpoint{1.796261in}{1.261656in}}%
\pgfpathlineto{\pgfqpoint{1.797709in}{1.262580in}}%
\pgfpathlineto{\pgfqpoint{1.798612in}{1.267200in}}%
\pgfpathlineto{\pgfqpoint{1.799739in}{1.268124in}}%
\pgfpathlineto{\pgfqpoint{1.799739in}{1.269048in}}%
\pgfpathlineto{\pgfqpoint{1.801039in}{1.269972in}}%
\pgfpathlineto{\pgfqpoint{1.801828in}{1.271820in}}%
\pgfpathlineto{\pgfqpoint{1.803479in}{1.272744in}}%
\pgfpathlineto{\pgfqpoint{1.804581in}{1.276440in}}%
\pgfpathlineto{\pgfqpoint{1.805087in}{1.277364in}}%
\pgfpathlineto{\pgfqpoint{1.805357in}{1.280136in}}%
\pgfpathlineto{\pgfqpoint{1.806528in}{1.281060in}}%
\pgfpathlineto{\pgfqpoint{1.807569in}{1.284756in}}%
\pgfpathlineto{\pgfqpoint{1.808921in}{1.285680in}}%
\pgfpathlineto{\pgfqpoint{1.809913in}{1.288452in}}%
\pgfpathlineto{\pgfqpoint{1.810942in}{1.289376in}}%
\pgfpathlineto{\pgfqpoint{1.811603in}{1.292148in}}%
\pgfpathlineto{\pgfqpoint{1.812292in}{1.293072in}}%
\pgfpathlineto{\pgfqpoint{1.812469in}{1.294920in}}%
\pgfpathlineto{\pgfqpoint{1.814025in}{1.295844in}}%
\pgfpathlineto{\pgfqpoint{1.815019in}{1.301388in}}%
\pgfpathlineto{\pgfqpoint{1.815757in}{1.302312in}}%
\pgfpathlineto{\pgfqpoint{1.816824in}{1.304160in}}%
\pgfpathlineto{\pgfqpoint{1.817383in}{1.305084in}}%
\pgfpathlineto{\pgfqpoint{1.817579in}{1.307856in}}%
\pgfpathlineto{\pgfqpoint{1.818757in}{1.308780in}}%
\pgfpathlineto{\pgfqpoint{1.819569in}{1.314324in}}%
\pgfpathlineto{\pgfqpoint{1.820384in}{1.315248in}}%
\pgfpathlineto{\pgfqpoint{1.821396in}{1.320792in}}%
\pgfpathlineto{\pgfqpoint{1.822108in}{1.321716in}}%
\pgfpathlineto{\pgfqpoint{1.822467in}{1.325412in}}%
\pgfpathlineto{\pgfqpoint{1.823908in}{1.326336in}}%
\pgfpathlineto{\pgfqpoint{1.824643in}{1.329108in}}%
\pgfpathlineto{\pgfqpoint{1.825536in}{1.330032in}}%
\pgfpathlineto{\pgfqpoint{1.826376in}{1.332804in}}%
\pgfpathlineto{\pgfqpoint{1.826801in}{1.333728in}}%
\pgfpathlineto{\pgfqpoint{1.827757in}{1.338348in}}%
\pgfpathlineto{\pgfqpoint{1.828249in}{1.339272in}}%
\pgfpathlineto{\pgfqpoint{1.829219in}{1.342968in}}%
\pgfpathlineto{\pgfqpoint{1.829746in}{1.343892in}}%
\pgfpathlineto{\pgfqpoint{1.829746in}{1.344816in}}%
\pgfpathlineto{\pgfqpoint{1.831501in}{1.345740in}}%
\pgfpathlineto{\pgfqpoint{1.832505in}{1.350360in}}%
\pgfpathlineto{\pgfqpoint{1.832806in}{1.351284in}}%
\pgfpathlineto{\pgfqpoint{1.833546in}{1.354980in}}%
\pgfpathlineto{\pgfqpoint{1.834278in}{1.355904in}}%
\pgfpathlineto{\pgfqpoint{1.835088in}{1.361448in}}%
\pgfpathlineto{\pgfqpoint{1.835757in}{1.362372in}}%
\pgfpathlineto{\pgfqpoint{1.836849in}{1.371612in}}%
\pgfpathlineto{\pgfqpoint{1.837044in}{1.372536in}}%
\pgfpathlineto{\pgfqpoint{1.837208in}{1.374384in}}%
\pgfpathlineto{\pgfqpoint{1.838629in}{1.375308in}}%
\pgfpathlineto{\pgfqpoint{1.839693in}{1.379928in}}%
\pgfpathlineto{\pgfqpoint{1.840353in}{1.380852in}}%
\pgfpathlineto{\pgfqpoint{1.841252in}{1.383624in}}%
\pgfpathlineto{\pgfqpoint{1.842231in}{1.384548in}}%
\pgfpathlineto{\pgfqpoint{1.843293in}{1.391016in}}%
\pgfpathlineto{\pgfqpoint{1.844458in}{1.391940in}}%
\pgfpathlineto{\pgfqpoint{1.845415in}{1.395636in}}%
\pgfpathlineto{\pgfqpoint{1.846216in}{1.396560in}}%
\pgfpathlineto{\pgfqpoint{1.846877in}{1.398408in}}%
\pgfpathlineto{\pgfqpoint{1.847630in}{1.399332in}}%
\pgfpathlineto{\pgfqpoint{1.847630in}{1.400256in}}%
\pgfpathlineto{\pgfqpoint{1.850178in}{1.401180in}}%
\pgfpathlineto{\pgfqpoint{1.850685in}{1.404876in}}%
\pgfpathlineto{\pgfqpoint{1.851432in}{1.405800in}}%
\pgfpathlineto{\pgfqpoint{1.851432in}{1.406724in}}%
\pgfpathlineto{\pgfqpoint{1.853061in}{1.407648in}}%
\pgfpathlineto{\pgfqpoint{1.853696in}{1.413192in}}%
\pgfpathlineto{\pgfqpoint{1.855398in}{1.414116in}}%
\pgfpathlineto{\pgfqpoint{1.856478in}{1.417812in}}%
\pgfpathlineto{\pgfqpoint{1.857478in}{1.418736in}}%
\pgfpathlineto{\pgfqpoint{1.858512in}{1.424280in}}%
\pgfpathlineto{\pgfqpoint{1.860548in}{1.425204in}}%
\pgfpathlineto{\pgfqpoint{1.861505in}{1.427976in}}%
\pgfpathlineto{\pgfqpoint{1.861879in}{1.428900in}}%
\pgfpathlineto{\pgfqpoint{1.862957in}{1.432596in}}%
\pgfpathlineto{\pgfqpoint{1.863259in}{1.433520in}}%
\pgfpathlineto{\pgfqpoint{1.864322in}{1.438140in}}%
\pgfpathlineto{\pgfqpoint{1.865387in}{1.439064in}}%
\pgfpathlineto{\pgfqpoint{1.865991in}{1.441836in}}%
\pgfpathlineto{\pgfqpoint{1.868196in}{1.442760in}}%
\pgfpathlineto{\pgfqpoint{1.869055in}{1.447380in}}%
\pgfpathlineto{\pgfqpoint{1.870552in}{1.448304in}}%
\pgfpathlineto{\pgfqpoint{1.871621in}{1.452000in}}%
\pgfpathlineto{\pgfqpoint{1.872792in}{1.452924in}}%
\pgfpathlineto{\pgfqpoint{1.873770in}{1.456620in}}%
\pgfpathlineto{\pgfqpoint{1.874474in}{1.457544in}}%
\pgfpathlineto{\pgfqpoint{1.875256in}{1.461240in}}%
\pgfpathlineto{\pgfqpoint{1.876711in}{1.462164in}}%
\pgfpathlineto{\pgfqpoint{1.877268in}{1.464936in}}%
\pgfpathlineto{\pgfqpoint{1.878834in}{1.465860in}}%
\pgfpathlineto{\pgfqpoint{1.879223in}{1.468632in}}%
\pgfpathlineto{\pgfqpoint{1.880407in}{1.469556in}}%
\pgfpathlineto{\pgfqpoint{1.881104in}{1.471404in}}%
\pgfpathlineto{\pgfqpoint{1.882106in}{1.472328in}}%
\pgfpathlineto{\pgfqpoint{1.883182in}{1.474176in}}%
\pgfpathlineto{\pgfqpoint{1.883689in}{1.475100in}}%
\pgfpathlineto{\pgfqpoint{1.884594in}{1.482492in}}%
\pgfpathlineto{\pgfqpoint{1.885919in}{1.483416in}}%
\pgfpathlineto{\pgfqpoint{1.886931in}{1.488960in}}%
\pgfpathlineto{\pgfqpoint{1.888440in}{1.489884in}}%
\pgfpathlineto{\pgfqpoint{1.888887in}{1.493580in}}%
\pgfpathlineto{\pgfqpoint{1.889934in}{1.494504in}}%
\pgfpathlineto{\pgfqpoint{1.891014in}{1.500972in}}%
\pgfpathlineto{\pgfqpoint{1.892380in}{1.501896in}}%
\pgfpathlineto{\pgfqpoint{1.893424in}{1.504668in}}%
\pgfpathlineto{\pgfqpoint{1.893859in}{1.505592in}}%
\pgfpathlineto{\pgfqpoint{1.894877in}{1.510212in}}%
\pgfpathlineto{\pgfqpoint{1.895318in}{1.511136in}}%
\pgfpathlineto{\pgfqpoint{1.895847in}{1.514832in}}%
\pgfpathlineto{\pgfqpoint{1.896757in}{1.515756in}}%
\pgfpathlineto{\pgfqpoint{1.897094in}{1.519452in}}%
\pgfpathlineto{\pgfqpoint{1.898820in}{1.520376in}}%
\pgfpathlineto{\pgfqpoint{1.899555in}{1.527768in}}%
\pgfpathlineto{\pgfqpoint{1.900384in}{1.528692in}}%
\pgfpathlineto{\pgfqpoint{1.901314in}{1.532388in}}%
\pgfpathlineto{\pgfqpoint{1.901924in}{1.533312in}}%
\pgfpathlineto{\pgfqpoint{1.902946in}{1.538856in}}%
\pgfpathlineto{\pgfqpoint{1.903476in}{1.539780in}}%
\pgfpathlineto{\pgfqpoint{1.904470in}{1.542552in}}%
\pgfpathlineto{\pgfqpoint{1.905522in}{1.543476in}}%
\pgfpathlineto{\pgfqpoint{1.906399in}{1.549944in}}%
\pgfpathlineto{\pgfqpoint{1.907176in}{1.550868in}}%
\pgfpathlineto{\pgfqpoint{1.908235in}{1.556412in}}%
\pgfpathlineto{\pgfqpoint{1.908862in}{1.557336in}}%
\pgfpathlineto{\pgfqpoint{1.909708in}{1.559184in}}%
\pgfpathlineto{\pgfqpoint{1.910516in}{1.560108in}}%
\pgfpathlineto{\pgfqpoint{1.911434in}{1.566576in}}%
\pgfpathlineto{\pgfqpoint{1.911881in}{1.567500in}}%
\pgfpathlineto{\pgfqpoint{1.912928in}{1.569348in}}%
\pgfpathlineto{\pgfqpoint{1.914158in}{1.570272in}}%
\pgfpathlineto{\pgfqpoint{1.915188in}{1.574892in}}%
\pgfpathlineto{\pgfqpoint{1.915455in}{1.575816in}}%
\pgfpathlineto{\pgfqpoint{1.916284in}{1.579512in}}%
\pgfpathlineto{\pgfqpoint{1.917374in}{1.580436in}}%
\pgfpathlineto{\pgfqpoint{1.918280in}{1.583208in}}%
\pgfpathlineto{\pgfqpoint{1.919195in}{1.584132in}}%
\pgfpathlineto{\pgfqpoint{1.919734in}{1.587828in}}%
\pgfpathlineto{\pgfqpoint{1.920408in}{1.588752in}}%
\pgfpathlineto{\pgfqpoint{1.921442in}{1.591524in}}%
\pgfpathlineto{\pgfqpoint{1.922296in}{1.592448in}}%
\pgfpathlineto{\pgfqpoint{1.923223in}{1.595220in}}%
\pgfpathlineto{\pgfqpoint{1.924408in}{1.596144in}}%
\pgfpathlineto{\pgfqpoint{1.925189in}{1.598916in}}%
\pgfpathlineto{\pgfqpoint{1.925738in}{1.599840in}}%
\pgfpathlineto{\pgfqpoint{1.926818in}{1.603536in}}%
\pgfpathlineto{\pgfqpoint{1.927728in}{1.604460in}}%
\pgfpathlineto{\pgfqpoint{1.928788in}{1.609080in}}%
\pgfpathlineto{\pgfqpoint{1.929107in}{1.610004in}}%
\pgfpathlineto{\pgfqpoint{1.929107in}{1.610928in}}%
\pgfpathlineto{\pgfqpoint{1.931455in}{1.611852in}}%
\pgfpathlineto{\pgfqpoint{1.931953in}{1.614624in}}%
\pgfpathlineto{\pgfqpoint{1.932638in}{1.615548in}}%
\pgfpathlineto{\pgfqpoint{1.933725in}{1.620168in}}%
\pgfpathlineto{\pgfqpoint{1.934131in}{1.621092in}}%
\pgfpathlineto{\pgfqpoint{1.934131in}{1.622016in}}%
\pgfpathlineto{\pgfqpoint{1.936205in}{1.622940in}}%
\pgfpathlineto{\pgfqpoint{1.936921in}{1.627560in}}%
\pgfpathlineto{\pgfqpoint{1.938251in}{1.628484in}}%
\pgfpathlineto{\pgfqpoint{1.938830in}{1.632180in}}%
\pgfpathlineto{\pgfqpoint{1.941344in}{1.633104in}}%
\pgfpathlineto{\pgfqpoint{1.942407in}{1.635876in}}%
\pgfpathlineto{\pgfqpoint{1.942807in}{1.636800in}}%
\pgfpathlineto{\pgfqpoint{1.942934in}{1.638648in}}%
\pgfpathlineto{\pgfqpoint{1.944328in}{1.639572in}}%
\pgfpathlineto{\pgfqpoint{1.945316in}{1.646040in}}%
\pgfpathlineto{\pgfqpoint{1.945507in}{1.646964in}}%
\pgfpathlineto{\pgfqpoint{1.946380in}{1.648812in}}%
\pgfpathlineto{\pgfqpoint{1.947131in}{1.649736in}}%
\pgfpathlineto{\pgfqpoint{1.947889in}{1.652508in}}%
\pgfpathlineto{\pgfqpoint{1.949653in}{1.653432in}}%
\pgfpathlineto{\pgfqpoint{1.950489in}{1.657128in}}%
\pgfpathlineto{\pgfqpoint{1.950969in}{1.658052in}}%
\pgfpathlineto{\pgfqpoint{1.952045in}{1.662672in}}%
\pgfpathlineto{\pgfqpoint{1.952816in}{1.663596in}}%
\pgfpathlineto{\pgfqpoint{1.953592in}{1.669140in}}%
\pgfpathlineto{\pgfqpoint{1.954238in}{1.670064in}}%
\pgfpathlineto{\pgfqpoint{1.955196in}{1.674684in}}%
\pgfpathlineto{\pgfqpoint{1.955905in}{1.675608in}}%
\pgfpathlineto{\pgfqpoint{1.957010in}{1.683000in}}%
\pgfpathlineto{\pgfqpoint{1.958353in}{1.683924in}}%
\pgfpathlineto{\pgfqpoint{1.959198in}{1.686696in}}%
\pgfpathlineto{\pgfqpoint{1.961707in}{1.687620in}}%
\pgfpathlineto{\pgfqpoint{1.962658in}{1.689468in}}%
\pgfpathlineto{\pgfqpoint{1.963442in}{1.690392in}}%
\pgfpathlineto{\pgfqpoint{1.963940in}{1.694088in}}%
\pgfpathlineto{\pgfqpoint{1.964981in}{1.695012in}}%
\pgfpathlineto{\pgfqpoint{1.965906in}{1.699632in}}%
\pgfpathlineto{\pgfqpoint{1.966700in}{1.700556in}}%
\pgfpathlineto{\pgfqpoint{1.967618in}{1.702404in}}%
\pgfpathlineto{\pgfqpoint{1.968186in}{1.703328in}}%
\pgfpathlineto{\pgfqpoint{1.969132in}{1.708872in}}%
\pgfpathlineto{\pgfqpoint{1.970142in}{1.709796in}}%
\pgfpathlineto{\pgfqpoint{1.970676in}{1.715340in}}%
\pgfpathlineto{\pgfqpoint{1.972154in}{1.716264in}}%
\pgfpathlineto{\pgfqpoint{1.973167in}{1.721808in}}%
\pgfpathlineto{\pgfqpoint{1.973543in}{1.722732in}}%
\pgfpathlineto{\pgfqpoint{1.974193in}{1.726428in}}%
\pgfpathlineto{\pgfqpoint{1.974882in}{1.727352in}}%
\pgfpathlineto{\pgfqpoint{1.975828in}{1.735668in}}%
\pgfpathlineto{\pgfqpoint{1.976077in}{1.736592in}}%
\pgfpathlineto{\pgfqpoint{1.977140in}{1.741212in}}%
\pgfpathlineto{\pgfqpoint{1.978550in}{1.742136in}}%
\pgfpathlineto{\pgfqpoint{1.979369in}{1.744908in}}%
\pgfpathlineto{\pgfqpoint{1.980149in}{1.745832in}}%
\pgfpathlineto{\pgfqpoint{1.980488in}{1.747680in}}%
\pgfpathlineto{\pgfqpoint{1.981333in}{1.748604in}}%
\pgfpathlineto{\pgfqpoint{1.982412in}{1.752300in}}%
\pgfpathlineto{\pgfqpoint{1.982975in}{1.753224in}}%
\pgfpathlineto{\pgfqpoint{1.984038in}{1.755072in}}%
\pgfpathlineto{\pgfqpoint{1.984572in}{1.755996in}}%
\pgfpathlineto{\pgfqpoint{1.985418in}{1.759692in}}%
\pgfpathlineto{\pgfqpoint{1.986391in}{1.760616in}}%
\pgfpathlineto{\pgfqpoint{1.986897in}{1.762464in}}%
\pgfpathlineto{\pgfqpoint{1.987676in}{1.763388in}}%
\pgfpathlineto{\pgfqpoint{1.988658in}{1.767084in}}%
\pgfpathlineto{\pgfqpoint{1.989712in}{1.768008in}}%
\pgfpathlineto{\pgfqpoint{1.990536in}{1.769856in}}%
\pgfpathlineto{\pgfqpoint{1.991330in}{1.770780in}}%
\pgfpathlineto{\pgfqpoint{1.992337in}{1.774476in}}%
\pgfpathlineto{\pgfqpoint{1.993575in}{1.775400in}}%
\pgfpathlineto{\pgfqpoint{1.994437in}{1.780944in}}%
\pgfpathlineto{\pgfqpoint{1.995025in}{1.781868in}}%
\pgfpathlineto{\pgfqpoint{1.995025in}{1.782792in}}%
\pgfpathlineto{\pgfqpoint{1.996943in}{1.783716in}}%
\pgfpathlineto{\pgfqpoint{1.997146in}{1.785564in}}%
\pgfpathlineto{\pgfqpoint{1.998246in}{1.786488in}}%
\pgfpathlineto{\pgfqpoint{1.998776in}{1.789260in}}%
\pgfpathlineto{\pgfqpoint{1.999426in}{1.790184in}}%
\pgfpathlineto{\pgfqpoint{2.000328in}{1.794804in}}%
\pgfpathlineto{\pgfqpoint{2.001124in}{1.795728in}}%
\pgfpathlineto{\pgfqpoint{2.001873in}{1.799424in}}%
\pgfpathlineto{\pgfqpoint{2.002491in}{1.800348in}}%
\pgfpathlineto{\pgfqpoint{2.002887in}{1.802196in}}%
\pgfpathlineto{\pgfqpoint{2.004359in}{1.803120in}}%
\pgfpathlineto{\pgfqpoint{2.005032in}{1.805892in}}%
\pgfpathlineto{\pgfqpoint{2.007288in}{1.806816in}}%
\pgfpathlineto{\pgfqpoint{2.008209in}{1.811436in}}%
\pgfpathlineto{\pgfqpoint{2.009306in}{1.812360in}}%
\pgfpathlineto{\pgfqpoint{2.010364in}{1.817904in}}%
\pgfpathlineto{\pgfqpoint{2.011154in}{1.818828in}}%
\pgfpathlineto{\pgfqpoint{2.012190in}{1.820676in}}%
\pgfpathlineto{\pgfqpoint{2.012694in}{1.821600in}}%
\pgfpathlineto{\pgfqpoint{2.013489in}{1.828068in}}%
\pgfpathlineto{\pgfqpoint{2.015346in}{1.828992in}}%
\pgfpathlineto{\pgfqpoint{2.016245in}{1.835460in}}%
\pgfpathlineto{\pgfqpoint{2.016880in}{1.836384in}}%
\pgfpathlineto{\pgfqpoint{2.017793in}{1.838232in}}%
\pgfpathlineto{\pgfqpoint{2.018510in}{1.839156in}}%
\pgfpathlineto{\pgfqpoint{2.019597in}{1.841928in}}%
\pgfpathlineto{\pgfqpoint{2.019794in}{1.842852in}}%
\pgfpathlineto{\pgfqpoint{2.020312in}{1.849320in}}%
\pgfpathlineto{\pgfqpoint{2.021480in}{1.850244in}}%
\pgfpathlineto{\pgfqpoint{2.021919in}{1.853016in}}%
\pgfpathlineto{\pgfqpoint{2.023327in}{1.853940in}}%
\pgfpathlineto{\pgfqpoint{2.024124in}{1.858560in}}%
\pgfpathlineto{\pgfqpoint{2.024665in}{1.859484in}}%
\pgfpathlineto{\pgfqpoint{2.024913in}{1.862256in}}%
\pgfpathlineto{\pgfqpoint{2.027273in}{1.863180in}}%
\pgfpathlineto{\pgfqpoint{2.028118in}{1.867800in}}%
\pgfpathlineto{\pgfqpoint{2.029011in}{1.868724in}}%
\pgfpathlineto{\pgfqpoint{2.029179in}{1.870572in}}%
\pgfpathlineto{\pgfqpoint{2.030492in}{1.871496in}}%
\pgfpathlineto{\pgfqpoint{2.031554in}{1.875192in}}%
\pgfpathlineto{\pgfqpoint{2.032134in}{1.876116in}}%
\pgfpathlineto{\pgfqpoint{2.033219in}{1.885356in}}%
\pgfpathlineto{\pgfqpoint{2.034302in}{1.886280in}}%
\pgfpathlineto{\pgfqpoint{2.035180in}{1.891824in}}%
\pgfpathlineto{\pgfqpoint{2.035842in}{1.892748in}}%
\pgfpathlineto{\pgfqpoint{2.036063in}{1.895520in}}%
\pgfpathlineto{\pgfqpoint{2.037494in}{1.896444in}}%
\pgfpathlineto{\pgfqpoint{2.038511in}{1.901064in}}%
\pgfpathlineto{\pgfqpoint{2.040430in}{1.901988in}}%
\pgfpathlineto{\pgfqpoint{2.041482in}{1.905684in}}%
\pgfpathlineto{\pgfqpoint{2.041719in}{1.906608in}}%
\pgfpathlineto{\pgfqpoint{2.042763in}{1.911228in}}%
\pgfpathlineto{\pgfqpoint{2.043453in}{1.912152in}}%
\pgfpathlineto{\pgfqpoint{2.043996in}{1.915848in}}%
\pgfpathlineto{\pgfqpoint{2.045795in}{1.916772in}}%
\pgfpathlineto{\pgfqpoint{2.046311in}{1.922316in}}%
\pgfpathlineto{\pgfqpoint{2.047288in}{1.923240in}}%
\pgfpathlineto{\pgfqpoint{2.048119in}{1.927860in}}%
\pgfpathlineto{\pgfqpoint{2.048610in}{1.928784in}}%
\pgfpathlineto{\pgfqpoint{2.049314in}{1.933404in}}%
\pgfpathlineto{\pgfqpoint{2.049884in}{1.934328in}}%
\pgfpathlineto{\pgfqpoint{2.050538in}{1.937100in}}%
\pgfpathlineto{\pgfqpoint{2.052326in}{1.938024in}}%
\pgfpathlineto{\pgfqpoint{2.053260in}{1.942644in}}%
\pgfpathlineto{\pgfqpoint{2.054203in}{1.943568in}}%
\pgfpathlineto{\pgfqpoint{2.054494in}{1.946340in}}%
\pgfpathlineto{\pgfqpoint{2.055527in}{1.947264in}}%
\pgfpathlineto{\pgfqpoint{2.056549in}{1.950960in}}%
\pgfpathlineto{\pgfqpoint{2.057678in}{1.951884in}}%
\pgfpathlineto{\pgfqpoint{2.058473in}{1.953732in}}%
\pgfpathlineto{\pgfqpoint{2.059460in}{1.954656in}}%
\pgfpathlineto{\pgfqpoint{2.060516in}{1.959276in}}%
\pgfpathlineto{\pgfqpoint{2.061381in}{1.960200in}}%
\pgfpathlineto{\pgfqpoint{2.062224in}{1.964820in}}%
\pgfpathlineto{\pgfqpoint{2.063034in}{1.965744in}}%
\pgfpathlineto{\pgfqpoint{2.063842in}{1.967592in}}%
\pgfpathlineto{\pgfqpoint{2.064534in}{1.968516in}}%
\pgfpathlineto{\pgfqpoint{2.065584in}{1.972212in}}%
\pgfpathlineto{\pgfqpoint{2.066237in}{1.973136in}}%
\pgfpathlineto{\pgfqpoint{2.067333in}{1.976832in}}%
\pgfpathlineto{\pgfqpoint{2.067885in}{1.977756in}}%
\pgfpathlineto{\pgfqpoint{2.068901in}{1.981452in}}%
\pgfpathlineto{\pgfqpoint{2.069335in}{1.982376in}}%
\pgfpathlineto{\pgfqpoint{2.069930in}{1.986072in}}%
\pgfpathlineto{\pgfqpoint{2.070523in}{1.986996in}}%
\pgfpathlineto{\pgfqpoint{2.071614in}{1.991616in}}%
\pgfpathlineto{\pgfqpoint{2.071774in}{1.992540in}}%
\pgfpathlineto{\pgfqpoint{2.072434in}{1.995312in}}%
\pgfpathlineto{\pgfqpoint{2.073407in}{1.996236in}}%
\pgfpathlineto{\pgfqpoint{2.074022in}{1.999008in}}%
\pgfpathlineto{\pgfqpoint{2.074902in}{1.999932in}}%
\pgfpathlineto{\pgfqpoint{2.075672in}{2.002704in}}%
\pgfpathlineto{\pgfqpoint{2.076490in}{2.003628in}}%
\pgfpathlineto{\pgfqpoint{2.077373in}{2.007324in}}%
\pgfpathlineto{\pgfqpoint{2.077784in}{2.008248in}}%
\pgfpathlineto{\pgfqpoint{2.078093in}{2.010096in}}%
\pgfpathlineto{\pgfqpoint{2.079401in}{2.011020in}}%
\pgfpathlineto{\pgfqpoint{2.080323in}{2.013792in}}%
\pgfpathlineto{\pgfqpoint{2.081515in}{2.014716in}}%
\pgfpathlineto{\pgfqpoint{2.082558in}{2.021184in}}%
\pgfpathlineto{\pgfqpoint{2.082895in}{2.022108in}}%
\pgfpathlineto{\pgfqpoint{2.083440in}{2.026728in}}%
\pgfpathlineto{\pgfqpoint{2.084292in}{2.027652in}}%
\pgfpathlineto{\pgfqpoint{2.084844in}{2.030424in}}%
\pgfpathlineto{\pgfqpoint{2.085618in}{2.031348in}}%
\pgfpathlineto{\pgfqpoint{2.086717in}{2.035044in}}%
\pgfpathlineto{\pgfqpoint{2.086994in}{2.035968in}}%
\pgfpathlineto{\pgfqpoint{2.087870in}{2.038740in}}%
\pgfpathlineto{\pgfqpoint{2.089174in}{2.039664in}}%
\pgfpathlineto{\pgfqpoint{2.089550in}{2.041512in}}%
\pgfpathlineto{\pgfqpoint{2.090808in}{2.042436in}}%
\pgfpathlineto{\pgfqpoint{2.091435in}{2.047980in}}%
\pgfpathlineto{\pgfqpoint{2.093471in}{2.048904in}}%
\pgfpathlineto{\pgfqpoint{2.094525in}{2.056296in}}%
\pgfpathlineto{\pgfqpoint{2.094777in}{2.057220in}}%
\pgfpathlineto{\pgfqpoint{2.094966in}{2.059068in}}%
\pgfpathlineto{\pgfqpoint{2.096611in}{2.059992in}}%
\pgfpathlineto{\pgfqpoint{2.096818in}{2.061840in}}%
\pgfpathlineto{\pgfqpoint{2.098041in}{2.062764in}}%
\pgfpathlineto{\pgfqpoint{2.099001in}{2.064612in}}%
\pgfpathlineto{\pgfqpoint{2.099934in}{2.065536in}}%
\pgfpathlineto{\pgfqpoint{2.100425in}{2.068308in}}%
\pgfpathlineto{\pgfqpoint{2.101420in}{2.069232in}}%
\pgfpathlineto{\pgfqpoint{2.102523in}{2.072928in}}%
\pgfpathlineto{\pgfqpoint{2.103378in}{2.073852in}}%
\pgfpathlineto{\pgfqpoint{2.103972in}{2.078472in}}%
\pgfpathlineto{\pgfqpoint{2.104846in}{2.079396in}}%
\pgfpathlineto{\pgfqpoint{2.105932in}{2.084016in}}%
\pgfpathlineto{\pgfqpoint{2.106242in}{2.084940in}}%
\pgfpathlineto{\pgfqpoint{2.107223in}{2.090484in}}%
\pgfpathlineto{\pgfqpoint{2.108884in}{2.091408in}}%
\pgfpathlineto{\pgfqpoint{2.109208in}{2.093256in}}%
\pgfpathlineto{\pgfqpoint{2.110592in}{2.094180in}}%
\pgfpathlineto{\pgfqpoint{2.110594in}{2.096028in}}%
\pgfpathlineto{\pgfqpoint{2.111803in}{2.096952in}}%
\pgfpathlineto{\pgfqpoint{2.112725in}{2.101572in}}%
\pgfpathlineto{\pgfqpoint{2.113008in}{2.102496in}}%
\pgfpathlineto{\pgfqpoint{2.113287in}{2.104344in}}%
\pgfpathlineto{\pgfqpoint{2.115278in}{2.105268in}}%
\pgfpathlineto{\pgfqpoint{2.116246in}{2.109888in}}%
\pgfpathlineto{\pgfqpoint{2.116749in}{2.110812in}}%
\pgfpathlineto{\pgfqpoint{2.117508in}{2.115432in}}%
\pgfpathlineto{\pgfqpoint{2.118227in}{2.116356in}}%
\pgfpathlineto{\pgfqpoint{2.119048in}{2.119128in}}%
\pgfpathlineto{\pgfqpoint{2.120932in}{2.120052in}}%
\pgfpathlineto{\pgfqpoint{2.121886in}{2.122824in}}%
\pgfpathlineto{\pgfqpoint{2.122599in}{2.123748in}}%
\pgfpathlineto{\pgfqpoint{2.123221in}{2.128368in}}%
\pgfpathlineto{\pgfqpoint{2.124169in}{2.129292in}}%
\pgfpathlineto{\pgfqpoint{2.125174in}{2.132064in}}%
\pgfpathlineto{\pgfqpoint{2.125802in}{2.132988in}}%
\pgfpathlineto{\pgfqpoint{2.126815in}{2.137608in}}%
\pgfpathlineto{\pgfqpoint{2.127689in}{2.138532in}}%
\pgfpathlineto{\pgfqpoint{2.128618in}{2.141304in}}%
\pgfpathlineto{\pgfqpoint{2.129079in}{2.142228in}}%
\pgfpathlineto{\pgfqpoint{2.129975in}{2.145000in}}%
\pgfpathlineto{\pgfqpoint{2.131108in}{2.145924in}}%
\pgfpathlineto{\pgfqpoint{2.131893in}{2.148696in}}%
\pgfpathlineto{\pgfqpoint{2.132537in}{2.149620in}}%
\pgfpathlineto{\pgfqpoint{2.132537in}{2.150544in}}%
\pgfpathlineto{\pgfqpoint{2.133861in}{2.151468in}}%
\pgfpathlineto{\pgfqpoint{2.134737in}{2.154240in}}%
\pgfpathlineto{\pgfqpoint{2.136770in}{2.155164in}}%
\pgfpathlineto{\pgfqpoint{2.137023in}{2.157012in}}%
\pgfpathlineto{\pgfqpoint{2.139865in}{2.157936in}}%
\pgfpathlineto{\pgfqpoint{2.140579in}{2.164404in}}%
\pgfpathlineto{\pgfqpoint{2.141080in}{2.165328in}}%
\pgfpathlineto{\pgfqpoint{2.141346in}{2.167176in}}%
\pgfpathlineto{\pgfqpoint{2.142840in}{2.168100in}}%
\pgfpathlineto{\pgfqpoint{2.142954in}{2.169948in}}%
\pgfpathlineto{\pgfqpoint{2.144937in}{2.170872in}}%
\pgfpathlineto{\pgfqpoint{2.145779in}{2.177340in}}%
\pgfpathlineto{\pgfqpoint{2.146718in}{2.178264in}}%
\pgfpathlineto{\pgfqpoint{2.147619in}{2.182884in}}%
\pgfpathlineto{\pgfqpoint{2.148567in}{2.183808in}}%
\pgfpathlineto{\pgfqpoint{2.149477in}{2.187504in}}%
\pgfpathlineto{\pgfqpoint{2.150450in}{2.188428in}}%
\pgfpathlineto{\pgfqpoint{2.151455in}{2.191200in}}%
\pgfpathlineto{\pgfqpoint{2.152273in}{2.192124in}}%
\pgfpathlineto{\pgfqpoint{2.153369in}{2.198592in}}%
\pgfpathlineto{\pgfqpoint{2.154502in}{2.199516in}}%
\pgfpathlineto{\pgfqpoint{2.155532in}{2.202288in}}%
\pgfpathlineto{\pgfqpoint{2.157200in}{2.203212in}}%
\pgfpathlineto{\pgfqpoint{2.158078in}{2.207832in}}%
\pgfpathlineto{\pgfqpoint{2.158532in}{2.208756in}}%
\pgfpathlineto{\pgfqpoint{2.159460in}{2.212452in}}%
\pgfpathlineto{\pgfqpoint{2.160557in}{2.213376in}}%
\pgfpathlineto{\pgfqpoint{2.161654in}{2.216148in}}%
\pgfpathlineto{\pgfqpoint{2.162340in}{2.217072in}}%
\pgfpathlineto{\pgfqpoint{2.162835in}{2.222616in}}%
\pgfpathlineto{\pgfqpoint{2.163850in}{2.223540in}}%
\pgfpathlineto{\pgfqpoint{2.164183in}{2.229084in}}%
\pgfpathlineto{\pgfqpoint{2.165090in}{2.230008in}}%
\pgfpathlineto{\pgfqpoint{2.166013in}{2.236476in}}%
\pgfpathlineto{\pgfqpoint{2.166869in}{2.237400in}}%
\pgfpathlineto{\pgfqpoint{2.167948in}{2.241096in}}%
\pgfpathlineto{\pgfqpoint{2.169570in}{2.242020in}}%
\pgfpathlineto{\pgfqpoint{2.170624in}{2.245716in}}%
\pgfpathlineto{\pgfqpoint{2.171766in}{2.246640in}}%
\pgfpathlineto{\pgfqpoint{2.172571in}{2.249412in}}%
\pgfpathlineto{\pgfqpoint{2.173912in}{2.250336in}}%
\pgfpathlineto{\pgfqpoint{2.174596in}{2.253108in}}%
\pgfpathlineto{\pgfqpoint{2.175211in}{2.254032in}}%
\pgfpathlineto{\pgfqpoint{2.176322in}{2.257728in}}%
\pgfpathlineto{\pgfqpoint{2.176464in}{2.258652in}}%
\pgfpathlineto{\pgfqpoint{2.177462in}{2.263272in}}%
\pgfpathlineto{\pgfqpoint{2.178037in}{2.264196in}}%
\pgfpathlineto{\pgfqpoint{2.178331in}{2.266968in}}%
\pgfpathlineto{\pgfqpoint{2.179792in}{2.267892in}}%
\pgfpathlineto{\pgfqpoint{2.180734in}{2.270664in}}%
\pgfpathlineto{\pgfqpoint{2.181073in}{2.271588in}}%
\pgfpathlineto{\pgfqpoint{2.181786in}{2.274360in}}%
\pgfpathlineto{\pgfqpoint{2.183702in}{2.275284in}}%
\pgfpathlineto{\pgfqpoint{2.184790in}{2.282676in}}%
\pgfpathlineto{\pgfqpoint{2.185609in}{2.283600in}}%
\pgfpathlineto{\pgfqpoint{2.185609in}{2.284524in}}%
\pgfpathlineto{\pgfqpoint{2.187642in}{2.285448in}}%
\pgfpathlineto{\pgfqpoint{2.188698in}{2.288220in}}%
\pgfpathlineto{\pgfqpoint{2.190683in}{2.289144in}}%
\pgfpathlineto{\pgfqpoint{2.191183in}{2.293764in}}%
\pgfpathlineto{\pgfqpoint{2.191983in}{2.294688in}}%
\pgfpathlineto{\pgfqpoint{2.193026in}{2.299308in}}%
\pgfpathlineto{\pgfqpoint{2.194016in}{2.300232in}}%
\pgfpathlineto{\pgfqpoint{2.194321in}{2.303004in}}%
\pgfpathlineto{\pgfqpoint{2.195601in}{2.303928in}}%
\pgfpathlineto{\pgfqpoint{2.196209in}{2.307624in}}%
\pgfpathlineto{\pgfqpoint{2.197542in}{2.308548in}}%
\pgfpathlineto{\pgfqpoint{2.198485in}{2.312244in}}%
\pgfpathlineto{\pgfqpoint{2.198864in}{2.313168in}}%
\pgfpathlineto{\pgfqpoint{2.199816in}{2.316864in}}%
\pgfpathlineto{\pgfqpoint{2.200623in}{2.317788in}}%
\pgfpathlineto{\pgfqpoint{2.201039in}{2.319636in}}%
\pgfpathlineto{\pgfqpoint{2.202524in}{2.320560in}}%
\pgfpathlineto{\pgfqpoint{2.202902in}{2.323332in}}%
\pgfpathlineto{\pgfqpoint{2.204758in}{2.324256in}}%
\pgfpathlineto{\pgfqpoint{2.205358in}{2.327952in}}%
\pgfpathlineto{\pgfqpoint{2.207936in}{2.328876in}}%
\pgfpathlineto{\pgfqpoint{2.208114in}{2.330724in}}%
\pgfpathlineto{\pgfqpoint{2.209361in}{2.331648in}}%
\pgfpathlineto{\pgfqpoint{2.210320in}{2.335344in}}%
\pgfpathlineto{\pgfqpoint{2.210755in}{2.336268in}}%
\pgfpathlineto{\pgfqpoint{2.211820in}{2.340888in}}%
\pgfpathlineto{\pgfqpoint{2.212847in}{2.341812in}}%
\pgfpathlineto{\pgfqpoint{2.213916in}{2.346432in}}%
\pgfpathlineto{\pgfqpoint{2.214724in}{2.347356in}}%
\pgfpathlineto{\pgfqpoint{2.215828in}{2.351052in}}%
\pgfpathlineto{\pgfqpoint{2.216725in}{2.351976in}}%
\pgfpathlineto{\pgfqpoint{2.217831in}{2.355672in}}%
\pgfpathlineto{\pgfqpoint{2.218516in}{2.356596in}}%
\pgfpathlineto{\pgfqpoint{2.219094in}{2.359368in}}%
\pgfpathlineto{\pgfqpoint{2.220132in}{2.360292in}}%
\pgfpathlineto{\pgfqpoint{2.220386in}{2.362140in}}%
\pgfpathlineto{\pgfqpoint{2.221942in}{2.363064in}}%
\pgfpathlineto{\pgfqpoint{2.222740in}{2.366760in}}%
\pgfpathlineto{\pgfqpoint{2.223121in}{2.367684in}}%
\pgfpathlineto{\pgfqpoint{2.223121in}{2.368608in}}%
\pgfpathlineto{\pgfqpoint{2.225028in}{2.369532in}}%
\pgfpathlineto{\pgfqpoint{2.226101in}{2.374152in}}%
\pgfpathlineto{\pgfqpoint{2.226328in}{2.375076in}}%
\pgfpathlineto{\pgfqpoint{2.227377in}{2.377848in}}%
\pgfpathlineto{\pgfqpoint{2.228433in}{2.378772in}}%
\pgfpathlineto{\pgfqpoint{2.228784in}{2.380620in}}%
\pgfpathlineto{\pgfqpoint{2.230322in}{2.381544in}}%
\pgfpathlineto{\pgfqpoint{2.230916in}{2.386164in}}%
\pgfpathlineto{\pgfqpoint{2.232216in}{2.387088in}}%
\pgfpathlineto{\pgfqpoint{2.232654in}{2.391708in}}%
\pgfpathlineto{\pgfqpoint{2.234643in}{2.392632in}}%
\pgfpathlineto{\pgfqpoint{2.235298in}{2.396328in}}%
\pgfpathlineto{\pgfqpoint{2.236350in}{2.397252in}}%
\pgfpathlineto{\pgfqpoint{2.237207in}{2.402796in}}%
\pgfpathlineto{\pgfqpoint{2.238698in}{2.403720in}}%
\pgfpathlineto{\pgfqpoint{2.239700in}{2.407416in}}%
\pgfpathlineto{\pgfqpoint{2.241773in}{2.408340in}}%
\pgfpathlineto{\pgfqpoint{2.242352in}{2.413884in}}%
\pgfpathlineto{\pgfqpoint{2.243347in}{2.414808in}}%
\pgfpathlineto{\pgfqpoint{2.244169in}{2.420352in}}%
\pgfpathlineto{\pgfqpoint{2.244730in}{2.421276in}}%
\pgfpathlineto{\pgfqpoint{2.245785in}{2.424972in}}%
\pgfpathlineto{\pgfqpoint{2.247445in}{2.425896in}}%
\pgfpathlineto{\pgfqpoint{2.248051in}{2.428668in}}%
\pgfpathlineto{\pgfqpoint{2.249395in}{2.429592in}}%
\pgfpathlineto{\pgfqpoint{2.250491in}{2.435136in}}%
\pgfpathlineto{\pgfqpoint{2.251530in}{2.436060in}}%
\pgfpathlineto{\pgfqpoint{2.252342in}{2.439756in}}%
\pgfpathlineto{\pgfqpoint{2.253194in}{2.440680in}}%
\pgfpathlineto{\pgfqpoint{2.254290in}{2.444376in}}%
\pgfpathlineto{\pgfqpoint{2.255518in}{2.445300in}}%
\pgfpathlineto{\pgfqpoint{2.256447in}{2.450844in}}%
\pgfpathlineto{\pgfqpoint{2.258260in}{2.451768in}}%
\pgfpathlineto{\pgfqpoint{2.259210in}{2.453616in}}%
\pgfpathlineto{\pgfqpoint{2.261522in}{2.454540in}}%
\pgfpathlineto{\pgfqpoint{2.262363in}{2.458236in}}%
\pgfpathlineto{\pgfqpoint{2.263417in}{2.459160in}}%
\pgfpathlineto{\pgfqpoint{2.264321in}{2.465628in}}%
\pgfpathlineto{\pgfqpoint{2.266057in}{2.466552in}}%
\pgfpathlineto{\pgfqpoint{2.266701in}{2.470248in}}%
\pgfpathlineto{\pgfqpoint{2.267346in}{2.471172in}}%
\pgfpathlineto{\pgfqpoint{2.267601in}{2.474868in}}%
\pgfpathlineto{\pgfqpoint{2.268894in}{2.475792in}}%
\pgfpathlineto{\pgfqpoint{2.269069in}{2.477640in}}%
\pgfpathlineto{\pgfqpoint{2.270236in}{2.478564in}}%
\pgfpathlineto{\pgfqpoint{2.270626in}{2.481336in}}%
\pgfpathlineto{\pgfqpoint{2.272026in}{2.482260in}}%
\pgfpathlineto{\pgfqpoint{2.272911in}{2.485956in}}%
\pgfpathlineto{\pgfqpoint{2.274005in}{2.486880in}}%
\pgfpathlineto{\pgfqpoint{2.275051in}{2.488728in}}%
\pgfpathlineto{\pgfqpoint{2.275483in}{2.489652in}}%
\pgfpathlineto{\pgfqpoint{2.275483in}{2.490576in}}%
\pgfpathlineto{\pgfqpoint{2.277176in}{2.491500in}}%
\pgfpathlineto{\pgfqpoint{2.278274in}{2.497968in}}%
\pgfpathlineto{\pgfqpoint{2.279465in}{2.498892in}}%
\pgfpathlineto{\pgfqpoint{2.280008in}{2.503512in}}%
\pgfpathlineto{\pgfqpoint{2.280684in}{2.504436in}}%
\pgfpathlineto{\pgfqpoint{2.281347in}{2.507208in}}%
\pgfpathlineto{\pgfqpoint{2.282256in}{2.508132in}}%
\pgfpathlineto{\pgfqpoint{2.283121in}{2.511828in}}%
\pgfpathlineto{\pgfqpoint{2.284030in}{2.512752in}}%
\pgfpathlineto{\pgfqpoint{2.284064in}{2.514600in}}%
\pgfpathlineto{\pgfqpoint{2.286603in}{2.515524in}}%
\pgfpathlineto{\pgfqpoint{2.287053in}{2.517372in}}%
\pgfpathlineto{\pgfqpoint{2.288089in}{2.518296in}}%
\pgfpathlineto{\pgfqpoint{2.288863in}{2.523840in}}%
\pgfpathlineto{\pgfqpoint{2.289877in}{2.524764in}}%
\pgfpathlineto{\pgfqpoint{2.290954in}{2.529384in}}%
\pgfpathlineto{\pgfqpoint{2.291334in}{2.530308in}}%
\pgfpathlineto{\pgfqpoint{2.292109in}{2.534004in}}%
\pgfpathlineto{\pgfqpoint{2.293853in}{2.534928in}}%
\pgfpathlineto{\pgfqpoint{2.294943in}{2.538624in}}%
\pgfpathlineto{\pgfqpoint{2.295518in}{2.539548in}}%
\pgfpathlineto{\pgfqpoint{2.296475in}{2.543244in}}%
\pgfpathlineto{\pgfqpoint{2.297651in}{2.544168in}}%
\pgfpathlineto{\pgfqpoint{2.298118in}{2.546940in}}%
\pgfpathlineto{\pgfqpoint{2.300270in}{2.547864in}}%
\pgfpathlineto{\pgfqpoint{2.300751in}{2.550636in}}%
\pgfpathlineto{\pgfqpoint{2.301715in}{2.551560in}}%
\pgfpathlineto{\pgfqpoint{2.302541in}{2.556180in}}%
\pgfpathlineto{\pgfqpoint{2.303900in}{2.557104in}}%
\pgfpathlineto{\pgfqpoint{2.304351in}{2.559876in}}%
\pgfpathlineto{\pgfqpoint{2.305211in}{2.560800in}}%
\pgfpathlineto{\pgfqpoint{2.305211in}{2.561724in}}%
\pgfpathlineto{\pgfqpoint{2.306614in}{2.562648in}}%
\pgfpathlineto{\pgfqpoint{2.306904in}{2.565420in}}%
\pgfpathlineto{\pgfqpoint{2.308171in}{2.566344in}}%
\pgfpathlineto{\pgfqpoint{2.308729in}{2.569116in}}%
\pgfpathlineto{\pgfqpoint{2.309995in}{2.570040in}}%
\pgfpathlineto{\pgfqpoint{2.310079in}{2.572812in}}%
\pgfpathlineto{\pgfqpoint{2.311431in}{2.573736in}}%
\pgfpathlineto{\pgfqpoint{2.312167in}{2.577432in}}%
\pgfpathlineto{\pgfqpoint{2.313847in}{2.578356in}}%
\pgfpathlineto{\pgfqpoint{2.314480in}{2.580204in}}%
\pgfpathlineto{\pgfqpoint{2.315726in}{2.581128in}}%
\pgfpathlineto{\pgfqpoint{2.316636in}{2.583900in}}%
\pgfpathlineto{\pgfqpoint{2.317543in}{2.584824in}}%
\pgfpathlineto{\pgfqpoint{2.318194in}{2.587596in}}%
\pgfpathlineto{\pgfqpoint{2.319258in}{2.588520in}}%
\pgfpathlineto{\pgfqpoint{2.319889in}{2.591292in}}%
\pgfpathlineto{\pgfqpoint{2.321201in}{2.592216in}}%
\pgfpathlineto{\pgfqpoint{2.322149in}{2.595912in}}%
\pgfpathlineto{\pgfqpoint{2.322465in}{2.596836in}}%
\pgfpathlineto{\pgfqpoint{2.322590in}{2.599608in}}%
\pgfpathlineto{\pgfqpoint{2.324202in}{2.600532in}}%
\pgfpathlineto{\pgfqpoint{2.325187in}{2.604228in}}%
\pgfpathlineto{\pgfqpoint{2.326394in}{2.605152in}}%
\pgfpathlineto{\pgfqpoint{2.327382in}{2.607924in}}%
\pgfpathlineto{\pgfqpoint{2.329244in}{2.608848in}}%
\pgfpathlineto{\pgfqpoint{2.330109in}{2.610696in}}%
\pgfpathlineto{\pgfqpoint{2.331339in}{2.611620in}}%
\pgfpathlineto{\pgfqpoint{2.331339in}{2.612544in}}%
\pgfpathlineto{\pgfqpoint{2.333748in}{2.613468in}}%
\pgfpathlineto{\pgfqpoint{2.334545in}{2.615316in}}%
\pgfpathlineto{\pgfqpoint{2.335376in}{2.616240in}}%
\pgfpathlineto{\pgfqpoint{2.336363in}{2.619936in}}%
\pgfpathlineto{\pgfqpoint{2.336961in}{2.620860in}}%
\pgfpathlineto{\pgfqpoint{2.337500in}{2.624556in}}%
\pgfpathlineto{\pgfqpoint{2.339991in}{2.625480in}}%
\pgfpathlineto{\pgfqpoint{2.340628in}{2.628252in}}%
\pgfpathlineto{\pgfqpoint{2.341816in}{2.629176in}}%
\pgfpathlineto{\pgfqpoint{2.341816in}{2.630100in}}%
\pgfpathlineto{\pgfqpoint{2.343984in}{2.631024in}}%
\pgfpathlineto{\pgfqpoint{2.345063in}{2.632872in}}%
\pgfpathlineto{\pgfqpoint{2.345911in}{2.633796in}}%
\pgfpathlineto{\pgfqpoint{2.346795in}{2.640264in}}%
\pgfpathlineto{\pgfqpoint{2.347152in}{2.641188in}}%
\pgfpathlineto{\pgfqpoint{2.347668in}{2.644884in}}%
\pgfpathlineto{\pgfqpoint{2.348550in}{2.645808in}}%
\pgfpathlineto{\pgfqpoint{2.348787in}{2.648580in}}%
\pgfpathlineto{\pgfqpoint{2.349882in}{2.649504in}}%
\pgfpathlineto{\pgfqpoint{2.350937in}{2.653200in}}%
\pgfpathlineto{\pgfqpoint{2.351739in}{2.654124in}}%
\pgfpathlineto{\pgfqpoint{2.352654in}{2.658744in}}%
\pgfpathlineto{\pgfqpoint{2.353013in}{2.659668in}}%
\pgfpathlineto{\pgfqpoint{2.353379in}{2.662440in}}%
\pgfpathlineto{\pgfqpoint{2.354512in}{2.663364in}}%
\pgfpathlineto{\pgfqpoint{2.355369in}{2.665212in}}%
\pgfpathlineto{\pgfqpoint{2.356379in}{2.666136in}}%
\pgfpathlineto{\pgfqpoint{2.357389in}{2.668908in}}%
\pgfpathlineto{\pgfqpoint{2.358039in}{2.669832in}}%
\pgfpathlineto{\pgfqpoint{2.359133in}{2.673528in}}%
\pgfpathlineto{\pgfqpoint{2.359813in}{2.674452in}}%
\pgfpathlineto{\pgfqpoint{2.359813in}{2.675376in}}%
\pgfpathlineto{\pgfqpoint{2.361062in}{2.676300in}}%
\pgfpathlineto{\pgfqpoint{2.362016in}{2.679996in}}%
\pgfpathlineto{\pgfqpoint{2.362636in}{2.680920in}}%
\pgfpathlineto{\pgfqpoint{2.363607in}{2.683692in}}%
\pgfpathlineto{\pgfqpoint{2.365581in}{2.684616in}}%
\pgfpathlineto{\pgfqpoint{2.366264in}{2.687388in}}%
\pgfpathlineto{\pgfqpoint{2.367805in}{2.688312in}}%
\pgfpathlineto{\pgfqpoint{2.368753in}{2.691084in}}%
\pgfpathlineto{\pgfqpoint{2.370688in}{2.692008in}}%
\pgfpathlineto{\pgfqpoint{2.371761in}{2.694780in}}%
\pgfpathlineto{\pgfqpoint{2.372472in}{2.695704in}}%
\pgfpathlineto{\pgfqpoint{2.372849in}{2.698476in}}%
\pgfpathlineto{\pgfqpoint{2.374281in}{2.699400in}}%
\pgfpathlineto{\pgfqpoint{2.374342in}{2.701248in}}%
\pgfpathlineto{\pgfqpoint{2.375712in}{2.702172in}}%
\pgfpathlineto{\pgfqpoint{2.376660in}{2.706792in}}%
\pgfpathlineto{\pgfqpoint{2.376992in}{2.707716in}}%
\pgfpathlineto{\pgfqpoint{2.377505in}{2.711412in}}%
\pgfpathlineto{\pgfqpoint{2.379257in}{2.712336in}}%
\pgfpathlineto{\pgfqpoint{2.379359in}{2.716956in}}%
\pgfpathlineto{\pgfqpoint{2.381714in}{2.717880in}}%
\pgfpathlineto{\pgfqpoint{2.382151in}{2.719728in}}%
\pgfpathlineto{\pgfqpoint{2.383486in}{2.720652in}}%
\pgfpathlineto{\pgfqpoint{2.384109in}{2.724348in}}%
\pgfpathlineto{\pgfqpoint{2.386308in}{2.725272in}}%
\pgfpathlineto{\pgfqpoint{2.387192in}{2.728968in}}%
\pgfpathlineto{\pgfqpoint{2.388432in}{2.729892in}}%
\pgfpathlineto{\pgfqpoint{2.389462in}{2.733588in}}%
\pgfpathlineto{\pgfqpoint{2.389849in}{2.734512in}}%
\pgfpathlineto{\pgfqpoint{2.390886in}{2.737284in}}%
\pgfpathlineto{\pgfqpoint{2.391920in}{2.738208in}}%
\pgfpathlineto{\pgfqpoint{2.392662in}{2.741904in}}%
\pgfpathlineto{\pgfqpoint{2.394047in}{2.742828in}}%
\pgfpathlineto{\pgfqpoint{2.394047in}{2.743752in}}%
\pgfpathlineto{\pgfqpoint{2.395635in}{2.744676in}}%
\pgfpathlineto{\pgfqpoint{2.395635in}{2.745600in}}%
\pgfpathlineto{\pgfqpoint{2.397737in}{2.746524in}}%
\pgfpathlineto{\pgfqpoint{2.398789in}{2.748372in}}%
\pgfpathlineto{\pgfqpoint{2.401084in}{2.749296in}}%
\pgfpathlineto{\pgfqpoint{2.401889in}{2.751144in}}%
\pgfpathlineto{\pgfqpoint{2.404034in}{2.752068in}}%
\pgfpathlineto{\pgfqpoint{2.405086in}{2.754840in}}%
\pgfpathlineto{\pgfqpoint{2.406616in}{2.755764in}}%
\pgfpathlineto{\pgfqpoint{2.406629in}{2.757612in}}%
\pgfpathlineto{\pgfqpoint{2.408018in}{2.758536in}}%
\pgfpathlineto{\pgfqpoint{2.408421in}{2.761308in}}%
\pgfpathlineto{\pgfqpoint{2.409486in}{2.762232in}}%
\pgfpathlineto{\pgfqpoint{2.410239in}{2.766852in}}%
\pgfpathlineto{\pgfqpoint{2.411444in}{2.767776in}}%
\pgfpathlineto{\pgfqpoint{2.411807in}{2.769624in}}%
\pgfpathlineto{\pgfqpoint{2.413221in}{2.770548in}}%
\pgfpathlineto{\pgfqpoint{2.414269in}{2.774244in}}%
\pgfpathlineto{\pgfqpoint{2.414964in}{2.775168in}}%
\pgfpathlineto{\pgfqpoint{2.415679in}{2.777940in}}%
\pgfpathlineto{\pgfqpoint{2.416585in}{2.778864in}}%
\pgfpathlineto{\pgfqpoint{2.417020in}{2.783484in}}%
\pgfpathlineto{\pgfqpoint{2.418090in}{2.784408in}}%
\pgfpathlineto{\pgfqpoint{2.418915in}{2.789952in}}%
\pgfpathlineto{\pgfqpoint{2.419755in}{2.790876in}}%
\pgfpathlineto{\pgfqpoint{2.419755in}{2.791800in}}%
\pgfpathlineto{\pgfqpoint{2.421788in}{2.792724in}}%
\pgfpathlineto{\pgfqpoint{2.422520in}{2.794572in}}%
\pgfpathlineto{\pgfqpoint{2.423039in}{2.795496in}}%
\pgfpathlineto{\pgfqpoint{2.423828in}{2.799192in}}%
\pgfpathlineto{\pgfqpoint{2.424481in}{2.800116in}}%
\pgfpathlineto{\pgfqpoint{2.424888in}{2.801964in}}%
\pgfpathlineto{\pgfqpoint{2.425740in}{2.802888in}}%
\pgfpathlineto{\pgfqpoint{2.426519in}{2.805660in}}%
\pgfpathlineto{\pgfqpoint{2.428195in}{2.806584in}}%
\pgfpathlineto{\pgfqpoint{2.428900in}{2.808432in}}%
\pgfpathlineto{\pgfqpoint{2.430045in}{2.809356in}}%
\pgfpathlineto{\pgfqpoint{2.430286in}{2.811204in}}%
\pgfpathlineto{\pgfqpoint{2.431276in}{2.812128in}}%
\pgfpathlineto{\pgfqpoint{2.431954in}{2.814900in}}%
\pgfpathlineto{\pgfqpoint{2.432740in}{2.815824in}}%
\pgfpathlineto{\pgfqpoint{2.433051in}{2.817672in}}%
\pgfpathlineto{\pgfqpoint{2.434195in}{2.818596in}}%
\pgfpathlineto{\pgfqpoint{2.434793in}{2.821368in}}%
\pgfpathlineto{\pgfqpoint{2.435802in}{2.822292in}}%
\pgfpathlineto{\pgfqpoint{2.436058in}{2.824140in}}%
\pgfpathlineto{\pgfqpoint{2.436983in}{2.825064in}}%
\pgfpathlineto{\pgfqpoint{2.437837in}{2.826912in}}%
\pgfpathlineto{\pgfqpoint{2.438626in}{2.827836in}}%
\pgfpathlineto{\pgfqpoint{2.439274in}{2.830608in}}%
\pgfpathlineto{\pgfqpoint{2.439843in}{2.831532in}}%
\pgfpathlineto{\pgfqpoint{2.440914in}{2.836152in}}%
\pgfpathlineto{\pgfqpoint{2.441555in}{2.837076in}}%
\pgfpathlineto{\pgfqpoint{2.442378in}{2.839848in}}%
\pgfpathlineto{\pgfqpoint{2.443622in}{2.840772in}}%
\pgfpathlineto{\pgfqpoint{2.444231in}{2.843544in}}%
\pgfpathlineto{\pgfqpoint{2.445348in}{2.844468in}}%
\pgfpathlineto{\pgfqpoint{2.446382in}{2.848164in}}%
\pgfpathlineto{\pgfqpoint{2.447446in}{2.849088in}}%
\pgfpathlineto{\pgfqpoint{2.447446in}{2.850012in}}%
\pgfpathlineto{\pgfqpoint{2.448921in}{2.850936in}}%
\pgfpathlineto{\pgfqpoint{2.449611in}{2.854632in}}%
\pgfpathlineto{\pgfqpoint{2.452076in}{2.855556in}}%
\pgfpathlineto{\pgfqpoint{2.453115in}{2.858328in}}%
\pgfpathlineto{\pgfqpoint{2.455894in}{2.859252in}}%
\pgfpathlineto{\pgfqpoint{2.456412in}{2.862948in}}%
\pgfpathlineto{\pgfqpoint{2.457566in}{2.863872in}}%
\pgfpathlineto{\pgfqpoint{2.457566in}{2.864796in}}%
\pgfpathlineto{\pgfqpoint{2.458950in}{2.865720in}}%
\pgfpathlineto{\pgfqpoint{2.459532in}{2.868492in}}%
\pgfpathlineto{\pgfqpoint{2.460976in}{2.869416in}}%
\pgfpathlineto{\pgfqpoint{2.461530in}{2.871264in}}%
\pgfpathlineto{\pgfqpoint{2.462410in}{2.872188in}}%
\pgfpathlineto{\pgfqpoint{2.463037in}{2.874036in}}%
\pgfpathlineto{\pgfqpoint{2.465699in}{2.874960in}}%
\pgfpathlineto{\pgfqpoint{2.466003in}{2.876808in}}%
\pgfpathlineto{\pgfqpoint{2.467649in}{2.877732in}}%
\pgfpathlineto{\pgfqpoint{2.468121in}{2.881428in}}%
\pgfpathlineto{\pgfqpoint{2.469139in}{2.882352in}}%
\pgfpathlineto{\pgfqpoint{2.469139in}{2.883276in}}%
\pgfpathlineto{\pgfqpoint{2.472690in}{2.884200in}}%
\pgfpathlineto{\pgfqpoint{2.473672in}{2.888820in}}%
\pgfpathlineto{\pgfqpoint{2.474161in}{2.889744in}}%
\pgfpathlineto{\pgfqpoint{2.474961in}{2.893440in}}%
\pgfpathlineto{\pgfqpoint{2.476026in}{2.894364in}}%
\pgfpathlineto{\pgfqpoint{2.476161in}{2.897136in}}%
\pgfpathlineto{\pgfqpoint{2.477770in}{2.898060in}}%
\pgfpathlineto{\pgfqpoint{2.477954in}{2.899908in}}%
\pgfpathlineto{\pgfqpoint{2.480361in}{2.900832in}}%
\pgfpathlineto{\pgfqpoint{2.481091in}{2.902680in}}%
\pgfpathlineto{\pgfqpoint{2.481761in}{2.903604in}}%
\pgfpathlineto{\pgfqpoint{2.482719in}{2.908224in}}%
\pgfpathlineto{\pgfqpoint{2.484182in}{2.909148in}}%
\pgfpathlineto{\pgfqpoint{2.485020in}{2.913768in}}%
\pgfpathlineto{\pgfqpoint{2.486262in}{2.914692in}}%
\pgfpathlineto{\pgfqpoint{2.487341in}{2.917464in}}%
\pgfpathlineto{\pgfqpoint{2.488623in}{2.918388in}}%
\pgfpathlineto{\pgfqpoint{2.489572in}{2.921160in}}%
\pgfpathlineto{\pgfqpoint{2.490011in}{2.922084in}}%
\pgfpathlineto{\pgfqpoint{2.490211in}{2.923932in}}%
\pgfpathlineto{\pgfqpoint{2.491207in}{2.924856in}}%
\pgfpathlineto{\pgfqpoint{2.491694in}{2.927628in}}%
\pgfpathlineto{\pgfqpoint{2.492623in}{2.928552in}}%
\pgfpathlineto{\pgfqpoint{2.493544in}{2.931324in}}%
\pgfpathlineto{\pgfqpoint{2.494799in}{2.932248in}}%
\pgfpathlineto{\pgfqpoint{2.495055in}{2.935020in}}%
\pgfpathlineto{\pgfqpoint{2.496238in}{2.935944in}}%
\pgfpathlineto{\pgfqpoint{2.496238in}{2.936868in}}%
\pgfpathlineto{\pgfqpoint{2.497490in}{2.937792in}}%
\pgfpathlineto{\pgfqpoint{2.498367in}{2.943336in}}%
\pgfpathlineto{\pgfqpoint{2.499459in}{2.944260in}}%
\pgfpathlineto{\pgfqpoint{2.499629in}{2.947032in}}%
\pgfpathlineto{\pgfqpoint{2.501606in}{2.947956in}}%
\pgfpathlineto{\pgfqpoint{2.502555in}{2.949804in}}%
\pgfpathlineto{\pgfqpoint{2.504260in}{2.950728in}}%
\pgfpathlineto{\pgfqpoint{2.505285in}{2.956272in}}%
\pgfpathlineto{\pgfqpoint{2.505529in}{2.957196in}}%
\pgfpathlineto{\pgfqpoint{2.506106in}{2.964588in}}%
\pgfpathlineto{\pgfqpoint{2.508671in}{2.965512in}}%
\pgfpathlineto{\pgfqpoint{2.509754in}{2.968284in}}%
\pgfpathlineto{\pgfqpoint{2.510913in}{2.969208in}}%
\pgfpathlineto{\pgfqpoint{2.511853in}{2.971056in}}%
\pgfpathlineto{\pgfqpoint{2.512528in}{2.971980in}}%
\pgfpathlineto{\pgfqpoint{2.513570in}{2.979372in}}%
\pgfpathlineto{\pgfqpoint{2.515383in}{2.980296in}}%
\pgfpathlineto{\pgfqpoint{2.516408in}{2.983068in}}%
\pgfpathlineto{\pgfqpoint{2.518028in}{2.983992in}}%
\pgfpathlineto{\pgfqpoint{2.519091in}{2.987688in}}%
\pgfpathlineto{\pgfqpoint{2.521482in}{2.988612in}}%
\pgfpathlineto{\pgfqpoint{2.521482in}{2.989536in}}%
\pgfpathlineto{\pgfqpoint{2.523398in}{2.990460in}}%
\pgfpathlineto{\pgfqpoint{2.524075in}{2.993232in}}%
\pgfpathlineto{\pgfqpoint{2.525653in}{2.994156in}}%
\pgfpathlineto{\pgfqpoint{2.526661in}{2.998776in}}%
\pgfpathlineto{\pgfqpoint{2.528351in}{2.999700in}}%
\pgfpathlineto{\pgfqpoint{2.529136in}{3.002472in}}%
\pgfpathlineto{\pgfqpoint{2.530553in}{3.003396in}}%
\pgfpathlineto{\pgfqpoint{2.531271in}{3.006168in}}%
\pgfpathlineto{\pgfqpoint{2.533011in}{3.007092in}}%
\pgfpathlineto{\pgfqpoint{2.533011in}{3.008016in}}%
\pgfpathlineto{\pgfqpoint{2.535917in}{3.008940in}}%
\pgfpathlineto{\pgfqpoint{2.536614in}{3.010788in}}%
\pgfpathlineto{\pgfqpoint{2.537798in}{3.011712in}}%
\pgfpathlineto{\pgfqpoint{2.538774in}{3.014484in}}%
\pgfpathlineto{\pgfqpoint{2.541592in}{3.015408in}}%
\pgfpathlineto{\pgfqpoint{2.541592in}{3.016332in}}%
\pgfpathlineto{\pgfqpoint{2.543364in}{3.017256in}}%
\pgfpathlineto{\pgfqpoint{2.544228in}{3.021876in}}%
\pgfpathlineto{\pgfqpoint{2.544771in}{3.022800in}}%
\pgfpathlineto{\pgfqpoint{2.545864in}{3.031116in}}%
\pgfpathlineto{\pgfqpoint{2.547163in}{3.032040in}}%
\pgfpathlineto{\pgfqpoint{2.548013in}{3.035736in}}%
\pgfpathlineto{\pgfqpoint{2.549377in}{3.036660in}}%
\pgfpathlineto{\pgfqpoint{2.550303in}{3.039432in}}%
\pgfpathlineto{\pgfqpoint{2.550760in}{3.040356in}}%
\pgfpathlineto{\pgfqpoint{2.551615in}{3.044976in}}%
\pgfpathlineto{\pgfqpoint{2.552054in}{3.045900in}}%
\pgfpathlineto{\pgfqpoint{2.552336in}{3.047748in}}%
\pgfpathlineto{\pgfqpoint{2.555043in}{3.048672in}}%
\pgfpathlineto{\pgfqpoint{2.555043in}{3.049596in}}%
\pgfpathlineto{\pgfqpoint{2.556607in}{3.050520in}}%
\pgfpathlineto{\pgfqpoint{2.557559in}{3.053292in}}%
\pgfpathlineto{\pgfqpoint{2.558830in}{3.054216in}}%
\pgfpathlineto{\pgfqpoint{2.559201in}{3.057912in}}%
\pgfpathlineto{\pgfqpoint{2.560553in}{3.058836in}}%
\pgfpathlineto{\pgfqpoint{2.560553in}{3.059760in}}%
\pgfpathlineto{\pgfqpoint{2.562454in}{3.060684in}}%
\pgfpathlineto{\pgfqpoint{2.562454in}{3.061608in}}%
\pgfpathlineto{\pgfqpoint{2.567008in}{3.062532in}}%
\pgfpathlineto{\pgfqpoint{2.567838in}{3.065304in}}%
\pgfpathlineto{\pgfqpoint{2.569282in}{3.066228in}}%
\pgfpathlineto{\pgfqpoint{2.570252in}{3.069924in}}%
\pgfpathlineto{\pgfqpoint{2.570936in}{3.070848in}}%
\pgfpathlineto{\pgfqpoint{2.571334in}{3.073620in}}%
\pgfpathlineto{\pgfqpoint{2.572115in}{3.074544in}}%
\pgfpathlineto{\pgfqpoint{2.572896in}{3.076392in}}%
\pgfpathlineto{\pgfqpoint{2.573683in}{3.077316in}}%
\pgfpathlineto{\pgfqpoint{2.574724in}{3.082860in}}%
\pgfpathlineto{\pgfqpoint{2.575170in}{3.083784in}}%
\pgfpathlineto{\pgfqpoint{2.576211in}{3.088404in}}%
\pgfpathlineto{\pgfqpoint{2.578638in}{3.089328in}}%
\pgfpathlineto{\pgfqpoint{2.579359in}{3.091176in}}%
\pgfpathlineto{\pgfqpoint{2.581007in}{3.092100in}}%
\pgfpathlineto{\pgfqpoint{2.581368in}{3.094872in}}%
\pgfpathlineto{\pgfqpoint{2.582347in}{3.095796in}}%
\pgfpathlineto{\pgfqpoint{2.582840in}{3.098568in}}%
\pgfpathlineto{\pgfqpoint{2.583864in}{3.099492in}}%
\pgfpathlineto{\pgfqpoint{2.584791in}{3.103188in}}%
\pgfpathlineto{\pgfqpoint{2.585647in}{3.104112in}}%
\pgfpathlineto{\pgfqpoint{2.586357in}{3.106884in}}%
\pgfpathlineto{\pgfqpoint{2.588410in}{3.107808in}}%
\pgfpathlineto{\pgfqpoint{2.589218in}{3.109656in}}%
\pgfpathlineto{\pgfqpoint{2.592453in}{3.110580in}}%
\pgfpathlineto{\pgfqpoint{2.592922in}{3.113352in}}%
\pgfpathlineto{\pgfqpoint{2.594544in}{3.114276in}}%
\pgfpathlineto{\pgfqpoint{2.594662in}{3.116124in}}%
\pgfpathlineto{\pgfqpoint{2.598066in}{3.117048in}}%
\pgfpathlineto{\pgfqpoint{2.598130in}{3.118896in}}%
\pgfpathlineto{\pgfqpoint{2.599647in}{3.119820in}}%
\pgfpathlineto{\pgfqpoint{2.600699in}{3.122592in}}%
\pgfpathlineto{\pgfqpoint{2.600951in}{3.123516in}}%
\pgfpathlineto{\pgfqpoint{2.601615in}{3.125364in}}%
\pgfpathlineto{\pgfqpoint{2.602969in}{3.126288in}}%
\pgfpathlineto{\pgfqpoint{2.604043in}{3.129060in}}%
\pgfpathlineto{\pgfqpoint{2.605072in}{3.129984in}}%
\pgfpathlineto{\pgfqpoint{2.606054in}{3.131832in}}%
\pgfpathlineto{\pgfqpoint{2.607790in}{3.132756in}}%
\pgfpathlineto{\pgfqpoint{2.607952in}{3.134604in}}%
\pgfpathlineto{\pgfqpoint{2.609014in}{3.135528in}}%
\pgfpathlineto{\pgfqpoint{2.609959in}{3.139224in}}%
\pgfpathlineto{\pgfqpoint{2.610354in}{3.140148in}}%
\pgfpathlineto{\pgfqpoint{2.611426in}{3.141996in}}%
\pgfpathlineto{\pgfqpoint{2.612867in}{3.142920in}}%
\pgfpathlineto{\pgfqpoint{2.613701in}{3.145692in}}%
\pgfpathlineto{\pgfqpoint{2.614041in}{3.146616in}}%
\pgfpathlineto{\pgfqpoint{2.614918in}{3.151236in}}%
\pgfpathlineto{\pgfqpoint{2.615878in}{3.152160in}}%
\pgfpathlineto{\pgfqpoint{2.616829in}{3.156780in}}%
\pgfpathlineto{\pgfqpoint{2.618779in}{3.157704in}}%
\pgfpathlineto{\pgfqpoint{2.618779in}{3.158628in}}%
\pgfpathlineto{\pgfqpoint{2.620373in}{3.159552in}}%
\pgfpathlineto{\pgfqpoint{2.621248in}{3.163248in}}%
\pgfpathlineto{\pgfqpoint{2.622165in}{3.164172in}}%
\pgfpathlineto{\pgfqpoint{2.622564in}{3.166020in}}%
\pgfpathlineto{\pgfqpoint{2.623680in}{3.166944in}}%
\pgfpathlineto{\pgfqpoint{2.624481in}{3.168792in}}%
\pgfpathlineto{\pgfqpoint{2.626123in}{3.169716in}}%
\pgfpathlineto{\pgfqpoint{2.626724in}{3.172488in}}%
\pgfpathlineto{\pgfqpoint{2.627532in}{3.173412in}}%
\pgfpathlineto{\pgfqpoint{2.628516in}{3.178956in}}%
\pgfpathlineto{\pgfqpoint{2.629138in}{3.179880in}}%
\pgfpathlineto{\pgfqpoint{2.629958in}{3.181728in}}%
\pgfpathlineto{\pgfqpoint{2.631169in}{3.182652in}}%
\pgfpathlineto{\pgfqpoint{2.631336in}{3.185424in}}%
\pgfpathlineto{\pgfqpoint{2.633054in}{3.186348in}}%
\pgfpathlineto{\pgfqpoint{2.633717in}{3.188196in}}%
\pgfpathlineto{\pgfqpoint{2.636374in}{3.189120in}}%
\pgfpathlineto{\pgfqpoint{2.637159in}{3.190968in}}%
\pgfpathlineto{\pgfqpoint{2.639483in}{3.191892in}}%
\pgfpathlineto{\pgfqpoint{2.640481in}{3.194664in}}%
\pgfpathlineto{\pgfqpoint{2.641665in}{3.195588in}}%
\pgfpathlineto{\pgfqpoint{2.642526in}{3.197436in}}%
\pgfpathlineto{\pgfqpoint{2.643569in}{3.198360in}}%
\pgfpathlineto{\pgfqpoint{2.644635in}{3.202056in}}%
\pgfpathlineto{\pgfqpoint{2.645109in}{3.202980in}}%
\pgfpathlineto{\pgfqpoint{2.646054in}{3.205752in}}%
\pgfpathlineto{\pgfqpoint{2.647454in}{3.206676in}}%
\pgfpathlineto{\pgfqpoint{2.648020in}{3.208524in}}%
\pgfpathlineto{\pgfqpoint{2.649589in}{3.209448in}}%
\pgfpathlineto{\pgfqpoint{2.650699in}{3.213144in}}%
\pgfpathlineto{\pgfqpoint{2.650918in}{3.214068in}}%
\pgfpathlineto{\pgfqpoint{2.651527in}{3.218688in}}%
\pgfpathlineto{\pgfqpoint{2.653546in}{3.219612in}}%
\pgfpathlineto{\pgfqpoint{2.653546in}{3.220536in}}%
\pgfpathlineto{\pgfqpoint{2.656007in}{3.221460in}}%
\pgfpathlineto{\pgfqpoint{2.656533in}{3.223308in}}%
\pgfpathlineto{\pgfqpoint{2.658579in}{3.224232in}}%
\pgfpathlineto{\pgfqpoint{2.658811in}{3.226080in}}%
\pgfpathlineto{\pgfqpoint{2.661585in}{3.227004in}}%
\pgfpathlineto{\pgfqpoint{2.661952in}{3.228852in}}%
\pgfpathlineto{\pgfqpoint{2.664694in}{3.229776in}}%
\pgfpathlineto{\pgfqpoint{2.664694in}{3.230700in}}%
\pgfpathlineto{\pgfqpoint{2.667997in}{3.231624in}}%
\pgfpathlineto{\pgfqpoint{2.668651in}{3.234396in}}%
\pgfpathlineto{\pgfqpoint{2.670246in}{3.235320in}}%
\pgfpathlineto{\pgfqpoint{2.671110in}{3.239940in}}%
\pgfpathlineto{\pgfqpoint{2.672092in}{3.240864in}}%
\pgfpathlineto{\pgfqpoint{2.672846in}{3.243636in}}%
\pgfpathlineto{\pgfqpoint{2.675078in}{3.244560in}}%
\pgfpathlineto{\pgfqpoint{2.676082in}{3.247332in}}%
\pgfpathlineto{\pgfqpoint{2.676826in}{3.248256in}}%
\pgfpathlineto{\pgfqpoint{2.677394in}{3.252876in}}%
\pgfpathlineto{\pgfqpoint{2.679391in}{3.253800in}}%
\pgfpathlineto{\pgfqpoint{2.680152in}{3.256572in}}%
\pgfpathlineto{\pgfqpoint{2.681861in}{3.257496in}}%
\pgfpathlineto{\pgfqpoint{2.682733in}{3.260268in}}%
\pgfpathlineto{\pgfqpoint{2.683668in}{3.261192in}}%
\pgfpathlineto{\pgfqpoint{2.684712in}{3.263964in}}%
\pgfpathlineto{\pgfqpoint{2.685220in}{3.264888in}}%
\pgfpathlineto{\pgfqpoint{2.686130in}{3.268584in}}%
\pgfpathlineto{\pgfqpoint{2.688383in}{3.269508in}}%
\pgfpathlineto{\pgfqpoint{2.689167in}{3.272280in}}%
\pgfpathlineto{\pgfqpoint{2.690064in}{3.273204in}}%
\pgfpathlineto{\pgfqpoint{2.690098in}{3.275976in}}%
\pgfpathlineto{\pgfqpoint{2.693515in}{3.276900in}}%
\pgfpathlineto{\pgfqpoint{2.694187in}{3.280596in}}%
\pgfpathlineto{\pgfqpoint{2.697649in}{3.281520in}}%
\pgfpathlineto{\pgfqpoint{2.698342in}{3.288912in}}%
\pgfpathlineto{\pgfqpoint{2.699062in}{3.289836in}}%
\pgfpathlineto{\pgfqpoint{2.699648in}{3.291684in}}%
\pgfpathlineto{\pgfqpoint{2.700350in}{3.292608in}}%
\pgfpathlineto{\pgfqpoint{2.700350in}{3.293532in}}%
\pgfpathlineto{\pgfqpoint{2.702393in}{3.294456in}}%
\pgfpathlineto{\pgfqpoint{2.703050in}{3.296304in}}%
\pgfpathlineto{\pgfqpoint{2.703668in}{3.297228in}}%
\pgfpathlineto{\pgfqpoint{2.704682in}{3.300000in}}%
\pgfpathlineto{\pgfqpoint{2.705063in}{3.300924in}}%
\pgfpathlineto{\pgfqpoint{2.706081in}{3.303696in}}%
\pgfpathlineto{\pgfqpoint{2.706866in}{3.304620in}}%
\pgfpathlineto{\pgfqpoint{2.707918in}{3.307392in}}%
\pgfpathlineto{\pgfqpoint{2.710016in}{3.308316in}}%
\pgfpathlineto{\pgfqpoint{2.710449in}{3.311088in}}%
\pgfpathlineto{\pgfqpoint{2.712982in}{3.312012in}}%
\pgfpathlineto{\pgfqpoint{2.713499in}{3.315708in}}%
\pgfpathlineto{\pgfqpoint{2.715576in}{3.316632in}}%
\pgfpathlineto{\pgfqpoint{2.716071in}{3.318480in}}%
\pgfpathlineto{\pgfqpoint{2.719133in}{3.319404in}}%
\pgfpathlineto{\pgfqpoint{2.719658in}{3.322176in}}%
\pgfpathlineto{\pgfqpoint{2.720909in}{3.323100in}}%
\pgfpathlineto{\pgfqpoint{2.721357in}{3.326796in}}%
\pgfpathlineto{\pgfqpoint{2.723058in}{3.327720in}}%
\pgfpathlineto{\pgfqpoint{2.723794in}{3.329568in}}%
\pgfpathlineto{\pgfqpoint{2.725792in}{3.330492in}}%
\pgfpathlineto{\pgfqpoint{2.726519in}{3.332340in}}%
\pgfpathlineto{\pgfqpoint{2.728379in}{3.333264in}}%
\pgfpathlineto{\pgfqpoint{2.729169in}{3.336036in}}%
\pgfpathlineto{\pgfqpoint{2.730695in}{3.336960in}}%
\pgfpathlineto{\pgfqpoint{2.731386in}{3.338808in}}%
\pgfpathlineto{\pgfqpoint{2.732313in}{3.339732in}}%
\pgfpathlineto{\pgfqpoint{2.733300in}{3.341580in}}%
\pgfpathlineto{\pgfqpoint{2.734698in}{3.342504in}}%
\pgfpathlineto{\pgfqpoint{2.734698in}{3.343428in}}%
\pgfpathlineto{\pgfqpoint{2.737500in}{3.344352in}}%
\pgfpathlineto{\pgfqpoint{2.737925in}{3.347124in}}%
\pgfpathlineto{\pgfqpoint{2.740267in}{3.348048in}}%
\pgfpathlineto{\pgfqpoint{2.740270in}{3.349896in}}%
\pgfpathlineto{\pgfqpoint{2.741710in}{3.350820in}}%
\pgfpathlineto{\pgfqpoint{2.742707in}{3.354516in}}%
\pgfpathlineto{\pgfqpoint{2.743961in}{3.355440in}}%
\pgfpathlineto{\pgfqpoint{2.744495in}{3.358212in}}%
\pgfpathlineto{\pgfqpoint{2.745707in}{3.359136in}}%
\pgfpathlineto{\pgfqpoint{2.746502in}{3.363756in}}%
\pgfpathlineto{\pgfqpoint{2.750130in}{3.364680in}}%
\pgfpathlineto{\pgfqpoint{2.751131in}{3.367452in}}%
\pgfpathlineto{\pgfqpoint{2.751876in}{3.368376in}}%
\pgfpathlineto{\pgfqpoint{2.752633in}{3.370224in}}%
\pgfpathlineto{\pgfqpoint{2.753338in}{3.371148in}}%
\pgfpathlineto{\pgfqpoint{2.754233in}{3.373920in}}%
\pgfpathlineto{\pgfqpoint{2.754759in}{3.374844in}}%
\pgfpathlineto{\pgfqpoint{2.755803in}{3.376692in}}%
\pgfpathlineto{\pgfqpoint{2.757523in}{3.377616in}}%
\pgfpathlineto{\pgfqpoint{2.757596in}{3.379464in}}%
\pgfpathlineto{\pgfqpoint{2.760294in}{3.380388in}}%
\pgfpathlineto{\pgfqpoint{2.760294in}{3.381312in}}%
\pgfpathlineto{\pgfqpoint{2.763279in}{3.382236in}}%
\pgfpathlineto{\pgfqpoint{2.763864in}{3.384084in}}%
\pgfpathlineto{\pgfqpoint{2.766773in}{3.385008in}}%
\pgfpathlineto{\pgfqpoint{2.767624in}{3.389628in}}%
\pgfpathlineto{\pgfqpoint{2.768093in}{3.390552in}}%
\pgfpathlineto{\pgfqpoint{2.768665in}{3.393324in}}%
\pgfpathlineto{\pgfqpoint{2.769432in}{3.394248in}}%
\pgfpathlineto{\pgfqpoint{2.769949in}{3.396096in}}%
\pgfpathlineto{\pgfqpoint{2.772432in}{3.397020in}}%
\pgfpathlineto{\pgfqpoint{2.773399in}{3.399792in}}%
\pgfpathlineto{\pgfqpoint{2.774589in}{3.400716in}}%
\pgfpathlineto{\pgfqpoint{2.775652in}{3.404412in}}%
\pgfpathlineto{\pgfqpoint{2.776044in}{3.405336in}}%
\pgfpathlineto{\pgfqpoint{2.776669in}{3.408108in}}%
\pgfpathlineto{\pgfqpoint{2.777617in}{3.409032in}}%
\pgfpathlineto{\pgfqpoint{2.778530in}{3.411804in}}%
\pgfpathlineto{\pgfqpoint{2.779425in}{3.412728in}}%
\pgfpathlineto{\pgfqpoint{2.780375in}{3.414576in}}%
\pgfpathlineto{\pgfqpoint{2.783855in}{3.415500in}}%
\pgfpathlineto{\pgfqpoint{2.784530in}{3.417348in}}%
\pgfpathlineto{\pgfqpoint{2.786020in}{3.418272in}}%
\pgfpathlineto{\pgfqpoint{2.786606in}{3.420120in}}%
\pgfpathlineto{\pgfqpoint{2.787560in}{3.421044in}}%
\pgfpathlineto{\pgfqpoint{2.787560in}{3.421968in}}%
\pgfpathlineto{\pgfqpoint{2.793160in}{3.422892in}}%
\pgfpathlineto{\pgfqpoint{2.793727in}{3.425664in}}%
\pgfpathlineto{\pgfqpoint{2.795211in}{3.426588in}}%
\pgfpathlineto{\pgfqpoint{2.796054in}{3.429360in}}%
\pgfpathlineto{\pgfqpoint{2.797913in}{3.430284in}}%
\pgfpathlineto{\pgfqpoint{2.798409in}{3.433056in}}%
\pgfpathlineto{\pgfqpoint{2.799686in}{3.433980in}}%
\pgfpathlineto{\pgfqpoint{2.799983in}{3.436752in}}%
\pgfpathlineto{\pgfqpoint{2.801728in}{3.437676in}}%
\pgfpathlineto{\pgfqpoint{2.801728in}{3.438600in}}%
\pgfpathlineto{\pgfqpoint{2.803999in}{3.439524in}}%
\pgfpathlineto{\pgfqpoint{2.805038in}{3.441372in}}%
\pgfpathlineto{\pgfqpoint{2.806796in}{3.442296in}}%
\pgfpathlineto{\pgfqpoint{2.807584in}{3.445068in}}%
\pgfpathlineto{\pgfqpoint{2.810565in}{3.445992in}}%
\pgfpathlineto{\pgfqpoint{2.811047in}{3.448764in}}%
\pgfpathlineto{\pgfqpoint{2.812160in}{3.449688in}}%
\pgfpathlineto{\pgfqpoint{2.812737in}{3.453384in}}%
\pgfpathlineto{\pgfqpoint{2.814032in}{3.454308in}}%
\pgfpathlineto{\pgfqpoint{2.815099in}{3.458004in}}%
\pgfpathlineto{\pgfqpoint{2.815368in}{3.458928in}}%
\pgfpathlineto{\pgfqpoint{2.816178in}{3.460776in}}%
\pgfpathlineto{\pgfqpoint{2.819904in}{3.461700in}}%
\pgfpathlineto{\pgfqpoint{2.820624in}{3.465396in}}%
\pgfpathlineto{\pgfqpoint{2.823641in}{3.466320in}}%
\pgfpathlineto{\pgfqpoint{2.824342in}{3.470016in}}%
\pgfpathlineto{\pgfqpoint{2.826881in}{3.470940in}}%
\pgfpathlineto{\pgfqpoint{2.827134in}{3.472788in}}%
\pgfpathlineto{\pgfqpoint{2.830901in}{3.473712in}}%
\pgfpathlineto{\pgfqpoint{2.831856in}{3.477408in}}%
\pgfpathlineto{\pgfqpoint{2.833795in}{3.478332in}}%
\pgfpathlineto{\pgfqpoint{2.834755in}{3.482952in}}%
\pgfpathlineto{\pgfqpoint{2.835085in}{3.483876in}}%
\pgfpathlineto{\pgfqpoint{2.836173in}{3.489420in}}%
\pgfpathlineto{\pgfqpoint{2.837588in}{3.490344in}}%
\pgfpathlineto{\pgfqpoint{2.837936in}{3.492192in}}%
\pgfpathlineto{\pgfqpoint{2.839792in}{3.493116in}}%
\pgfpathlineto{\pgfqpoint{2.840814in}{3.495888in}}%
\pgfpathlineto{\pgfqpoint{2.841939in}{3.496812in}}%
\pgfpathlineto{\pgfqpoint{2.841939in}{3.497736in}}%
\pgfpathlineto{\pgfqpoint{2.844643in}{3.498660in}}%
\pgfpathlineto{\pgfqpoint{2.845635in}{3.500508in}}%
\pgfpathlineto{\pgfqpoint{2.846350in}{3.501432in}}%
\pgfpathlineto{\pgfqpoint{2.847388in}{3.503280in}}%
\pgfpathlineto{\pgfqpoint{2.848194in}{3.504204in}}%
\pgfpathlineto{\pgfqpoint{2.848693in}{3.506052in}}%
\pgfpathlineto{\pgfqpoint{2.851575in}{3.506976in}}%
\pgfpathlineto{\pgfqpoint{2.851965in}{3.508824in}}%
\pgfpathlineto{\pgfqpoint{2.853980in}{3.509748in}}%
\pgfpathlineto{\pgfqpoint{2.853980in}{3.510672in}}%
\pgfpathlineto{\pgfqpoint{2.855917in}{3.511596in}}%
\pgfpathlineto{\pgfqpoint{2.856731in}{3.516216in}}%
\pgfpathlineto{\pgfqpoint{2.860432in}{3.517140in}}%
\pgfpathlineto{\pgfqpoint{2.860432in}{3.518064in}}%
\pgfpathlineto{\pgfqpoint{2.864775in}{3.518988in}}%
\pgfpathlineto{\pgfqpoint{2.865559in}{3.521760in}}%
\pgfpathlineto{\pgfqpoint{2.867650in}{3.522684in}}%
\pgfpathlineto{\pgfqpoint{2.868641in}{3.526380in}}%
\pgfpathlineto{\pgfqpoint{2.869275in}{3.527304in}}%
\pgfpathlineto{\pgfqpoint{2.870034in}{3.529152in}}%
\pgfpathlineto{\pgfqpoint{2.872839in}{3.530076in}}%
\pgfpathlineto{\pgfqpoint{2.873491in}{3.533772in}}%
\pgfpathlineto{\pgfqpoint{2.875054in}{3.534696in}}%
\pgfpathlineto{\pgfqpoint{2.875128in}{3.536544in}}%
\pgfpathlineto{\pgfqpoint{2.877047in}{3.537468in}}%
\pgfpathlineto{\pgfqpoint{2.877377in}{3.540240in}}%
\pgfpathlineto{\pgfqpoint{2.879059in}{3.541164in}}%
\pgfpathlineto{\pgfqpoint{2.879632in}{3.543936in}}%
\pgfpathlineto{\pgfqpoint{2.882815in}{3.544860in}}%
\pgfpathlineto{\pgfqpoint{2.883277in}{3.547632in}}%
\pgfpathlineto{\pgfqpoint{2.884828in}{3.548556in}}%
\pgfpathlineto{\pgfqpoint{2.885274in}{3.551328in}}%
\pgfpathlineto{\pgfqpoint{2.888124in}{3.552252in}}%
\pgfpathlineto{\pgfqpoint{2.888989in}{3.555948in}}%
\pgfpathlineto{\pgfqpoint{2.892267in}{3.556872in}}%
\pgfpathlineto{\pgfqpoint{2.892637in}{3.558720in}}%
\pgfpathlineto{\pgfqpoint{2.894155in}{3.559644in}}%
\pgfpathlineto{\pgfqpoint{2.895183in}{3.561492in}}%
\pgfpathlineto{\pgfqpoint{2.896719in}{3.562416in}}%
\pgfpathlineto{\pgfqpoint{2.897105in}{3.565188in}}%
\pgfpathlineto{\pgfqpoint{2.899030in}{3.566112in}}%
\pgfpathlineto{\pgfqpoint{2.899030in}{3.567036in}}%
\pgfpathlineto{\pgfqpoint{2.901611in}{3.567960in}}%
\pgfpathlineto{\pgfqpoint{2.902423in}{3.572580in}}%
\pgfpathlineto{\pgfqpoint{2.903820in}{3.573504in}}%
\pgfpathlineto{\pgfqpoint{2.904845in}{3.576276in}}%
\pgfpathlineto{\pgfqpoint{2.905571in}{3.577200in}}%
\pgfpathlineto{\pgfqpoint{2.905571in}{3.578124in}}%
\pgfpathlineto{\pgfqpoint{2.907072in}{3.579048in}}%
\pgfpathlineto{\pgfqpoint{2.907072in}{3.579972in}}%
\pgfpathlineto{\pgfqpoint{2.910476in}{3.580896in}}%
\pgfpathlineto{\pgfqpoint{2.910906in}{3.583668in}}%
\pgfpathlineto{\pgfqpoint{2.914059in}{3.584592in}}%
\pgfpathlineto{\pgfqpoint{2.914596in}{3.586440in}}%
\pgfpathlineto{\pgfqpoint{2.919156in}{3.587364in}}%
\pgfpathlineto{\pgfqpoint{2.919988in}{3.591984in}}%
\pgfpathlineto{\pgfqpoint{2.922078in}{3.592908in}}%
\pgfpathlineto{\pgfqpoint{2.923008in}{3.594756in}}%
\pgfpathlineto{\pgfqpoint{2.925137in}{3.595680in}}%
\pgfpathlineto{\pgfqpoint{2.925679in}{3.598452in}}%
\pgfpathlineto{\pgfqpoint{2.927597in}{3.599376in}}%
\pgfpathlineto{\pgfqpoint{2.927597in}{3.600300in}}%
\pgfpathlineto{\pgfqpoint{2.930733in}{3.601224in}}%
\pgfpathlineto{\pgfqpoint{2.930733in}{3.602148in}}%
\pgfpathlineto{\pgfqpoint{2.933425in}{3.603072in}}%
\pgfpathlineto{\pgfqpoint{2.933765in}{3.604920in}}%
\pgfpathlineto{\pgfqpoint{2.935067in}{3.605844in}}%
\pgfpathlineto{\pgfqpoint{2.935067in}{3.606768in}}%
\pgfpathlineto{\pgfqpoint{2.936512in}{3.607692in}}%
\pgfpathlineto{\pgfqpoint{2.936622in}{3.609540in}}%
\pgfpathlineto{\pgfqpoint{2.938033in}{3.610464in}}%
\pgfpathlineto{\pgfqpoint{2.938074in}{3.612312in}}%
\pgfpathlineto{\pgfqpoint{2.939356in}{3.613236in}}%
\pgfpathlineto{\pgfqpoint{2.939508in}{3.616932in}}%
\pgfpathlineto{\pgfqpoint{2.941405in}{3.617856in}}%
\pgfpathlineto{\pgfqpoint{2.942212in}{3.622476in}}%
\pgfpathlineto{\pgfqpoint{2.942892in}{3.623400in}}%
\pgfpathlineto{\pgfqpoint{2.943864in}{3.625248in}}%
\pgfpathlineto{\pgfqpoint{2.946338in}{3.626172in}}%
\pgfpathlineto{\pgfqpoint{2.946615in}{3.628020in}}%
\pgfpathlineto{\pgfqpoint{2.948627in}{3.628944in}}%
\pgfpathlineto{\pgfqpoint{2.949547in}{3.633564in}}%
\pgfpathlineto{\pgfqpoint{2.951275in}{3.634488in}}%
\pgfpathlineto{\pgfqpoint{2.951979in}{3.637260in}}%
\pgfpathlineto{\pgfqpoint{2.953857in}{3.638184in}}%
\pgfpathlineto{\pgfqpoint{2.954155in}{3.640032in}}%
\pgfpathlineto{\pgfqpoint{2.958490in}{3.640956in}}%
\pgfpathlineto{\pgfqpoint{2.959394in}{3.645576in}}%
\pgfpathlineto{\pgfqpoint{2.962806in}{3.646500in}}%
\pgfpathlineto{\pgfqpoint{2.962904in}{3.648348in}}%
\pgfpathlineto{\pgfqpoint{2.965350in}{3.649272in}}%
\pgfpathlineto{\pgfqpoint{2.966353in}{3.652968in}}%
\pgfpathlineto{\pgfqpoint{2.968667in}{3.653892in}}%
\pgfpathlineto{\pgfqpoint{2.968667in}{3.654816in}}%
\pgfpathlineto{\pgfqpoint{2.970634in}{3.655740in}}%
\pgfpathlineto{\pgfqpoint{2.971566in}{3.658512in}}%
\pgfpathlineto{\pgfqpoint{2.972192in}{3.659436in}}%
\pgfpathlineto{\pgfqpoint{2.973275in}{3.662208in}}%
\pgfpathlineto{\pgfqpoint{2.978333in}{3.663132in}}%
\pgfpathlineto{\pgfqpoint{2.979056in}{3.666828in}}%
\pgfpathlineto{\pgfqpoint{2.980640in}{3.667752in}}%
\pgfpathlineto{\pgfqpoint{2.981228in}{3.671448in}}%
\pgfpathlineto{\pgfqpoint{2.984124in}{3.672372in}}%
\pgfpathlineto{\pgfqpoint{2.984124in}{3.673296in}}%
\pgfpathlineto{\pgfqpoint{2.989258in}{3.674220in}}%
\pgfpathlineto{\pgfqpoint{2.990220in}{3.676068in}}%
\pgfpathlineto{\pgfqpoint{2.991466in}{3.676992in}}%
\pgfpathlineto{\pgfqpoint{2.992166in}{3.679764in}}%
\pgfpathlineto{\pgfqpoint{2.994793in}{3.680688in}}%
\pgfpathlineto{\pgfqpoint{2.995463in}{3.683460in}}%
\pgfpathlineto{\pgfqpoint{2.998347in}{3.684384in}}%
\pgfpathlineto{\pgfqpoint{2.999032in}{3.686232in}}%
\pgfpathlineto{\pgfqpoint{3.001566in}{3.687156in}}%
\pgfpathlineto{\pgfqpoint{3.001999in}{3.689928in}}%
\pgfpathlineto{\pgfqpoint{3.004485in}{3.690852in}}%
\pgfpathlineto{\pgfqpoint{3.005427in}{3.693624in}}%
\pgfpathlineto{\pgfqpoint{3.007047in}{3.694548in}}%
\pgfpathlineto{\pgfqpoint{3.007561in}{3.697320in}}%
\pgfpathlineto{\pgfqpoint{3.009968in}{3.698244in}}%
\pgfpathlineto{\pgfqpoint{3.010802in}{3.701940in}}%
\pgfpathlineto{\pgfqpoint{3.012096in}{3.702864in}}%
\pgfpathlineto{\pgfqpoint{3.013120in}{3.705636in}}%
\pgfpathlineto{\pgfqpoint{3.016478in}{3.706560in}}%
\pgfpathlineto{\pgfqpoint{3.016478in}{3.707484in}}%
\pgfpathlineto{\pgfqpoint{3.018741in}{3.708408in}}%
\pgfpathlineto{\pgfqpoint{3.019404in}{3.712104in}}%
\pgfpathlineto{\pgfqpoint{3.021994in}{3.713028in}}%
\pgfpathlineto{\pgfqpoint{3.021994in}{3.713952in}}%
\pgfpathlineto{\pgfqpoint{3.024731in}{3.714876in}}%
\pgfpathlineto{\pgfqpoint{3.024731in}{3.715800in}}%
\pgfpathlineto{\pgfqpoint{3.026845in}{3.716724in}}%
\pgfpathlineto{\pgfqpoint{3.027893in}{3.721344in}}%
\pgfpathlineto{\pgfqpoint{3.029251in}{3.722268in}}%
\pgfpathlineto{\pgfqpoint{3.029586in}{3.725040in}}%
\pgfpathlineto{\pgfqpoint{3.030675in}{3.725964in}}%
\pgfpathlineto{\pgfqpoint{3.031481in}{3.728736in}}%
\pgfpathlineto{\pgfqpoint{3.033406in}{3.729660in}}%
\pgfpathlineto{\pgfqpoint{3.034478in}{3.731508in}}%
\pgfpathlineto{\pgfqpoint{3.038582in}{3.732432in}}%
\pgfpathlineto{\pgfqpoint{3.038582in}{3.733356in}}%
\pgfpathlineto{\pgfqpoint{3.041546in}{3.734280in}}%
\pgfpathlineto{\pgfqpoint{3.042008in}{3.736128in}}%
\pgfpathlineto{\pgfqpoint{3.045709in}{3.737052in}}%
\pgfpathlineto{\pgfqpoint{3.046144in}{3.738900in}}%
\pgfpathlineto{\pgfqpoint{3.048710in}{3.739824in}}%
\pgfpathlineto{\pgfqpoint{3.048710in}{3.740748in}}%
\pgfpathlineto{\pgfqpoint{3.050408in}{3.741672in}}%
\pgfpathlineto{\pgfqpoint{3.051365in}{3.743520in}}%
\pgfpathlineto{\pgfqpoint{3.055106in}{3.744444in}}%
\pgfpathlineto{\pgfqpoint{3.055571in}{3.747216in}}%
\pgfpathlineto{\pgfqpoint{3.060267in}{3.748140in}}%
\pgfpathlineto{\pgfqpoint{3.060267in}{3.749064in}}%
\pgfpathlineto{\pgfqpoint{3.062095in}{3.749988in}}%
\pgfpathlineto{\pgfqpoint{3.063126in}{3.754608in}}%
\pgfpathlineto{\pgfqpoint{3.063680in}{3.755532in}}%
\pgfpathlineto{\pgfqpoint{3.063936in}{3.758304in}}%
\pgfpathlineto{\pgfqpoint{3.066988in}{3.759228in}}%
\pgfpathlineto{\pgfqpoint{3.068059in}{3.762924in}}%
\pgfpathlineto{\pgfqpoint{3.069737in}{3.763848in}}%
\pgfpathlineto{\pgfqpoint{3.070206in}{3.765696in}}%
\pgfpathlineto{\pgfqpoint{3.071866in}{3.766620in}}%
\pgfpathlineto{\pgfqpoint{3.072773in}{3.769392in}}%
\pgfpathlineto{\pgfqpoint{3.074617in}{3.770316in}}%
\pgfpathlineto{\pgfqpoint{3.075576in}{3.774012in}}%
\pgfpathlineto{\pgfqpoint{3.078024in}{3.774936in}}%
\pgfpathlineto{\pgfqpoint{3.078864in}{3.776784in}}%
\pgfpathlineto{\pgfqpoint{3.079704in}{3.777708in}}%
\pgfpathlineto{\pgfqpoint{3.080382in}{3.781404in}}%
\pgfpathlineto{\pgfqpoint{3.084113in}{3.782328in}}%
\pgfpathlineto{\pgfqpoint{3.085178in}{3.784176in}}%
\pgfpathlineto{\pgfqpoint{3.088392in}{3.785100in}}%
\pgfpathlineto{\pgfqpoint{3.089060in}{3.786948in}}%
\pgfpathlineto{\pgfqpoint{3.090309in}{3.787872in}}%
\pgfpathlineto{\pgfqpoint{3.090309in}{3.788796in}}%
\pgfpathlineto{\pgfqpoint{3.095953in}{3.789720in}}%
\pgfpathlineto{\pgfqpoint{3.097043in}{3.792492in}}%
\pgfpathlineto{\pgfqpoint{3.097989in}{3.793416in}}%
\pgfpathlineto{\pgfqpoint{3.099043in}{3.795264in}}%
\pgfpathlineto{\pgfqpoint{3.099886in}{3.796188in}}%
\pgfpathlineto{\pgfqpoint{3.100470in}{3.798960in}}%
\pgfpathlineto{\pgfqpoint{3.103600in}{3.799884in}}%
\pgfpathlineto{\pgfqpoint{3.104259in}{3.803580in}}%
\pgfpathlineto{\pgfqpoint{3.106192in}{3.804504in}}%
\pgfpathlineto{\pgfqpoint{3.107151in}{3.806352in}}%
\pgfpathlineto{\pgfqpoint{3.111291in}{3.807276in}}%
\pgfpathlineto{\pgfqpoint{3.112173in}{3.810972in}}%
\pgfpathlineto{\pgfqpoint{3.114015in}{3.811896in}}%
\pgfpathlineto{\pgfqpoint{3.114660in}{3.813744in}}%
\pgfpathlineto{\pgfqpoint{3.116245in}{3.814668in}}%
\pgfpathlineto{\pgfqpoint{3.117046in}{3.816516in}}%
\pgfpathlineto{\pgfqpoint{3.122087in}{3.817440in}}%
\pgfpathlineto{\pgfqpoint{3.122554in}{3.819288in}}%
\pgfpathlineto{\pgfqpoint{3.125305in}{3.820212in}}%
\pgfpathlineto{\pgfqpoint{3.126182in}{3.822060in}}%
\pgfpathlineto{\pgfqpoint{3.127365in}{3.822984in}}%
\pgfpathlineto{\pgfqpoint{3.127365in}{3.823908in}}%
\pgfpathlineto{\pgfqpoint{3.128945in}{3.824832in}}%
\pgfpathlineto{\pgfqpoint{3.129774in}{3.827604in}}%
\pgfpathlineto{\pgfqpoint{3.131233in}{3.828528in}}%
\pgfpathlineto{\pgfqpoint{3.131799in}{3.831300in}}%
\pgfpathlineto{\pgfqpoint{3.133289in}{3.832224in}}%
\pgfpathlineto{\pgfqpoint{3.133289in}{3.833148in}}%
\pgfpathlineto{\pgfqpoint{3.136232in}{3.834072in}}%
\pgfpathlineto{\pgfqpoint{3.136952in}{3.835920in}}%
\pgfpathlineto{\pgfqpoint{3.139097in}{3.836844in}}%
\pgfpathlineto{\pgfqpoint{3.139289in}{3.838692in}}%
\pgfpathlineto{\pgfqpoint{3.140886in}{3.839616in}}%
\pgfpathlineto{\pgfqpoint{3.141873in}{3.841464in}}%
\pgfpathlineto{\pgfqpoint{3.144338in}{3.842388in}}%
\pgfpathlineto{\pgfqpoint{3.144338in}{3.843312in}}%
\pgfpathlineto{\pgfqpoint{3.146304in}{3.844236in}}%
\pgfpathlineto{\pgfqpoint{3.146304in}{3.845160in}}%
\pgfpathlineto{\pgfqpoint{3.149794in}{3.846084in}}%
\pgfpathlineto{\pgfqpoint{3.149794in}{3.847008in}}%
\pgfpathlineto{\pgfqpoint{3.154441in}{3.847932in}}%
\pgfpathlineto{\pgfqpoint{3.155417in}{3.850704in}}%
\pgfpathlineto{\pgfqpoint{3.157693in}{3.851628in}}%
\pgfpathlineto{\pgfqpoint{3.158637in}{3.853476in}}%
\pgfpathlineto{\pgfqpoint{3.159450in}{3.854400in}}%
\pgfpathlineto{\pgfqpoint{3.160072in}{3.857172in}}%
\pgfpathlineto{\pgfqpoint{3.161874in}{3.858096in}}%
\pgfpathlineto{\pgfqpoint{3.161874in}{3.859020in}}%
\pgfpathlineto{\pgfqpoint{3.166567in}{3.859944in}}%
\pgfpathlineto{\pgfqpoint{3.166567in}{3.860868in}}%
\pgfpathlineto{\pgfqpoint{3.169498in}{3.861792in}}%
\pgfpathlineto{\pgfqpoint{3.169743in}{3.863640in}}%
\pgfpathlineto{\pgfqpoint{3.172022in}{3.864564in}}%
\pgfpathlineto{\pgfqpoint{3.172219in}{3.866412in}}%
\pgfpathlineto{\pgfqpoint{3.174024in}{3.867336in}}%
\pgfpathlineto{\pgfqpoint{3.174024in}{3.868260in}}%
\pgfpathlineto{\pgfqpoint{3.176832in}{3.869184in}}%
\pgfpathlineto{\pgfqpoint{3.176832in}{3.870108in}}%
\pgfpathlineto{\pgfqpoint{3.181566in}{3.871032in}}%
\pgfpathlineto{\pgfqpoint{3.182113in}{3.873804in}}%
\pgfpathlineto{\pgfqpoint{3.185499in}{3.874728in}}%
\pgfpathlineto{\pgfqpoint{3.186107in}{3.878424in}}%
\pgfpathlineto{\pgfqpoint{3.188746in}{3.879348in}}%
\pgfpathlineto{\pgfqpoint{3.188746in}{3.880272in}}%
\pgfpathlineto{\pgfqpoint{3.192671in}{3.881196in}}%
\pgfpathlineto{\pgfqpoint{3.193245in}{3.883044in}}%
\pgfpathlineto{\pgfqpoint{3.194486in}{3.883968in}}%
\pgfpathlineto{\pgfqpoint{3.194641in}{3.885816in}}%
\pgfpathlineto{\pgfqpoint{3.197663in}{3.886740in}}%
\pgfpathlineto{\pgfqpoint{3.198450in}{3.888588in}}%
\pgfpathlineto{\pgfqpoint{3.199319in}{3.889512in}}%
\pgfpathlineto{\pgfqpoint{3.199928in}{3.891360in}}%
\pgfpathlineto{\pgfqpoint{3.201582in}{3.892284in}}%
\pgfpathlineto{\pgfqpoint{3.202463in}{3.895056in}}%
\pgfpathlineto{\pgfqpoint{3.205824in}{3.895980in}}%
\pgfpathlineto{\pgfqpoint{3.205824in}{3.896904in}}%
\pgfpathlineto{\pgfqpoint{3.209524in}{3.897828in}}%
\pgfpathlineto{\pgfqpoint{3.209900in}{3.899676in}}%
\pgfpathlineto{\pgfqpoint{3.213014in}{3.900600in}}%
\pgfpathlineto{\pgfqpoint{3.213465in}{3.903372in}}%
\pgfpathlineto{\pgfqpoint{3.215701in}{3.904296in}}%
\pgfpathlineto{\pgfqpoint{3.215701in}{3.905220in}}%
\pgfpathlineto{\pgfqpoint{3.221936in}{3.906144in}}%
\pgfpathlineto{\pgfqpoint{3.222528in}{3.907992in}}%
\pgfpathlineto{\pgfqpoint{3.224705in}{3.908916in}}%
\pgfpathlineto{\pgfqpoint{3.225702in}{3.910764in}}%
\pgfpathlineto{\pgfqpoint{3.226687in}{3.911688in}}%
\pgfpathlineto{\pgfqpoint{3.226687in}{3.912612in}}%
\pgfpathlineto{\pgfqpoint{3.228525in}{3.913536in}}%
\pgfpathlineto{\pgfqpoint{3.229430in}{3.915384in}}%
\pgfpathlineto{\pgfqpoint{3.234343in}{3.916308in}}%
\pgfpathlineto{\pgfqpoint{3.235010in}{3.919080in}}%
\pgfpathlineto{\pgfqpoint{3.237450in}{3.920004in}}%
\pgfpathlineto{\pgfqpoint{3.237450in}{3.920928in}}%
\pgfpathlineto{\pgfqpoint{3.239954in}{3.921852in}}%
\pgfpathlineto{\pgfqpoint{3.239954in}{3.922776in}}%
\pgfpathlineto{\pgfqpoint{3.242950in}{3.923700in}}%
\pgfpathlineto{\pgfqpoint{3.242964in}{3.925548in}}%
\pgfpathlineto{\pgfqpoint{3.244455in}{3.926472in}}%
\pgfpathlineto{\pgfqpoint{3.244455in}{3.927396in}}%
\pgfpathlineto{\pgfqpoint{3.250187in}{3.928320in}}%
\pgfpathlineto{\pgfqpoint{3.251157in}{3.930168in}}%
\pgfpathlineto{\pgfqpoint{3.251854in}{3.931092in}}%
\pgfpathlineto{\pgfqpoint{3.252107in}{3.932940in}}%
\pgfpathlineto{\pgfqpoint{3.255204in}{3.933864in}}%
\pgfpathlineto{\pgfqpoint{3.256026in}{3.935712in}}%
\pgfpathlineto{\pgfqpoint{3.258473in}{3.936636in}}%
\pgfpathlineto{\pgfqpoint{3.258473in}{3.937560in}}%
\pgfpathlineto{\pgfqpoint{3.263558in}{3.938484in}}%
\pgfpathlineto{\pgfqpoint{3.264562in}{3.941256in}}%
\pgfpathlineto{\pgfqpoint{3.266376in}{3.942180in}}%
\pgfpathlineto{\pgfqpoint{3.267411in}{3.944028in}}%
\pgfpathlineto{\pgfqpoint{3.269150in}{3.944952in}}%
\pgfpathlineto{\pgfqpoint{3.269468in}{3.947724in}}%
\pgfpathlineto{\pgfqpoint{3.270608in}{3.948648in}}%
\pgfpathlineto{\pgfqpoint{3.270608in}{3.949572in}}%
\pgfpathlineto{\pgfqpoint{3.279429in}{3.950496in}}%
\pgfpathlineto{\pgfqpoint{3.279429in}{3.951420in}}%
\pgfpathlineto{\pgfqpoint{3.281358in}{3.952344in}}%
\pgfpathlineto{\pgfqpoint{3.281358in}{3.953268in}}%
\pgfpathlineto{\pgfqpoint{3.285990in}{3.954192in}}%
\pgfpathlineto{\pgfqpoint{3.286972in}{3.956040in}}%
\pgfpathlineto{\pgfqpoint{3.288573in}{3.956964in}}%
\pgfpathlineto{\pgfqpoint{3.288573in}{3.957888in}}%
\pgfpathlineto{\pgfqpoint{3.292998in}{3.958812in}}%
\pgfpathlineto{\pgfqpoint{3.293849in}{3.960660in}}%
\pgfpathlineto{\pgfqpoint{3.296235in}{3.961584in}}%
\pgfpathlineto{\pgfqpoint{3.296577in}{3.963432in}}%
\pgfpathlineto{\pgfqpoint{3.300326in}{3.964356in}}%
\pgfpathlineto{\pgfqpoint{3.301109in}{3.966204in}}%
\pgfpathlineto{\pgfqpoint{3.304592in}{3.967128in}}%
\pgfpathlineto{\pgfqpoint{3.304592in}{3.968052in}}%
\pgfpathlineto{\pgfqpoint{3.308295in}{3.968976in}}%
\pgfpathlineto{\pgfqpoint{3.308876in}{3.970824in}}%
\pgfpathlineto{\pgfqpoint{3.312827in}{3.971748in}}%
\pgfpathlineto{\pgfqpoint{3.313085in}{3.973596in}}%
\pgfpathlineto{\pgfqpoint{3.319584in}{3.974520in}}%
\pgfpathlineto{\pgfqpoint{3.320535in}{3.977292in}}%
\pgfpathlineto{\pgfqpoint{3.321664in}{3.978216in}}%
\pgfpathlineto{\pgfqpoint{3.321664in}{3.979140in}}%
\pgfpathlineto{\pgfqpoint{3.325136in}{3.980064in}}%
\pgfpathlineto{\pgfqpoint{3.325861in}{3.981912in}}%
\pgfpathlineto{\pgfqpoint{3.327345in}{3.982836in}}%
\pgfpathlineto{\pgfqpoint{3.327345in}{3.983760in}}%
\pgfpathlineto{\pgfqpoint{3.331543in}{3.984684in}}%
\pgfpathlineto{\pgfqpoint{3.331543in}{3.985608in}}%
\pgfpathlineto{\pgfqpoint{3.335482in}{3.986532in}}%
\pgfpathlineto{\pgfqpoint{3.336394in}{3.989304in}}%
\pgfpathlineto{\pgfqpoint{3.338436in}{3.990228in}}%
\pgfpathlineto{\pgfqpoint{3.339294in}{3.993000in}}%
\pgfpathlineto{\pgfqpoint{3.342884in}{3.993924in}}%
\pgfpathlineto{\pgfqpoint{3.343112in}{3.995772in}}%
\pgfpathlineto{\pgfqpoint{3.345871in}{3.996696in}}%
\pgfpathlineto{\pgfqpoint{3.345871in}{3.997620in}}%
\pgfpathlineto{\pgfqpoint{3.347810in}{3.998544in}}%
\pgfpathlineto{\pgfqpoint{3.348158in}{4.000392in}}%
\pgfpathlineto{\pgfqpoint{3.351209in}{4.001316in}}%
\pgfpathlineto{\pgfqpoint{3.351209in}{4.002240in}}%
\pgfpathlineto{\pgfqpoint{3.353189in}{4.003164in}}%
\pgfpathlineto{\pgfqpoint{3.353189in}{4.004088in}}%
\pgfpathlineto{\pgfqpoint{3.362054in}{4.005012in}}%
\pgfpathlineto{\pgfqpoint{3.362447in}{4.006860in}}%
\pgfpathlineto{\pgfqpoint{3.366725in}{4.007784in}}%
\pgfpathlineto{\pgfqpoint{3.366725in}{4.008708in}}%
\pgfpathlineto{\pgfqpoint{3.370824in}{4.009632in}}%
\pgfpathlineto{\pgfqpoint{3.370824in}{4.010556in}}%
\pgfpathlineto{\pgfqpoint{3.377367in}{4.011480in}}%
\pgfpathlineto{\pgfqpoint{3.377367in}{4.012404in}}%
\pgfpathlineto{\pgfqpoint{3.380985in}{4.013328in}}%
\pgfpathlineto{\pgfqpoint{3.382048in}{4.015176in}}%
\pgfpathlineto{\pgfqpoint{3.382741in}{4.016100in}}%
\pgfpathlineto{\pgfqpoint{3.382741in}{4.017024in}}%
\pgfpathlineto{\pgfqpoint{3.386758in}{4.017948in}}%
\pgfpathlineto{\pgfqpoint{3.387474in}{4.019796in}}%
\pgfpathlineto{\pgfqpoint{3.388916in}{4.020720in}}%
\pgfpathlineto{\pgfqpoint{3.389147in}{4.022568in}}%
\pgfpathlineto{\pgfqpoint{3.392072in}{4.023492in}}%
\pgfpathlineto{\pgfqpoint{3.392072in}{4.024416in}}%
\pgfpathlineto{\pgfqpoint{3.394573in}{4.025340in}}%
\pgfpathlineto{\pgfqpoint{3.395385in}{4.027188in}}%
\pgfpathlineto{\pgfqpoint{3.402719in}{4.028112in}}%
\pgfpathlineto{\pgfqpoint{3.403430in}{4.030884in}}%
\pgfpathlineto{\pgfqpoint{3.407534in}{4.031808in}}%
\pgfpathlineto{\pgfqpoint{3.407534in}{4.032732in}}%
\pgfpathlineto{\pgfqpoint{3.412072in}{4.033656in}}%
\pgfpathlineto{\pgfqpoint{3.412743in}{4.036428in}}%
\pgfpathlineto{\pgfqpoint{3.416740in}{4.037352in}}%
\pgfpathlineto{\pgfqpoint{3.417328in}{4.039200in}}%
\pgfpathlineto{\pgfqpoint{3.424702in}{4.040124in}}%
\pgfpathlineto{\pgfqpoint{3.424702in}{4.041048in}}%
\pgfpathlineto{\pgfqpoint{3.431491in}{4.041972in}}%
\pgfpathlineto{\pgfqpoint{3.431491in}{4.042896in}}%
\pgfpathlineto{\pgfqpoint{3.433757in}{4.043820in}}%
\pgfpathlineto{\pgfqpoint{3.434194in}{4.045668in}}%
\pgfpathlineto{\pgfqpoint{3.441131in}{4.046592in}}%
\pgfpathlineto{\pgfqpoint{3.441131in}{4.047516in}}%
\pgfpathlineto{\pgfqpoint{3.444836in}{4.048440in}}%
\pgfpathlineto{\pgfqpoint{3.445856in}{4.050288in}}%
\pgfpathlineto{\pgfqpoint{3.450130in}{4.051212in}}%
\pgfpathlineto{\pgfqpoint{3.450130in}{4.052136in}}%
\pgfpathlineto{\pgfqpoint{3.452431in}{4.053060in}}%
\pgfpathlineto{\pgfqpoint{3.453420in}{4.055832in}}%
\pgfpathlineto{\pgfqpoint{3.457917in}{4.056756in}}%
\pgfpathlineto{\pgfqpoint{3.457917in}{4.057680in}}%
\pgfpathlineto{\pgfqpoint{3.461357in}{4.058604in}}%
\pgfpathlineto{\pgfqpoint{3.461550in}{4.060452in}}%
\pgfpathlineto{\pgfqpoint{3.466968in}{4.061376in}}%
\pgfpathlineto{\pgfqpoint{3.466968in}{4.062300in}}%
\pgfpathlineto{\pgfqpoint{3.471723in}{4.063224in}}%
\pgfpathlineto{\pgfqpoint{3.472110in}{4.065072in}}%
\pgfpathlineto{\pgfqpoint{3.475682in}{4.065996in}}%
\pgfpathlineto{\pgfqpoint{3.476442in}{4.067844in}}%
\pgfpathlineto{\pgfqpoint{3.481518in}{4.068768in}}%
\pgfpathlineto{\pgfqpoint{3.481518in}{4.069692in}}%
\pgfpathlineto{\pgfqpoint{3.483105in}{4.070616in}}%
\pgfpathlineto{\pgfqpoint{3.483105in}{4.071540in}}%
\pgfpathlineto{\pgfqpoint{3.493861in}{4.072464in}}%
\pgfpathlineto{\pgfqpoint{3.493861in}{4.073388in}}%
\pgfpathlineto{\pgfqpoint{3.495677in}{4.074312in}}%
\pgfpathlineto{\pgfqpoint{3.495677in}{4.075236in}}%
\pgfpathlineto{\pgfqpoint{3.500699in}{4.076160in}}%
\pgfpathlineto{\pgfqpoint{3.500699in}{4.077084in}}%
\pgfpathlineto{\pgfqpoint{3.506409in}{4.078008in}}%
\pgfpathlineto{\pgfqpoint{3.506409in}{4.078932in}}%
\pgfpathlineto{\pgfqpoint{3.513256in}{4.079856in}}%
\pgfpathlineto{\pgfqpoint{3.514049in}{4.081704in}}%
\pgfpathlineto{\pgfqpoint{3.520187in}{4.082628in}}%
\pgfpathlineto{\pgfqpoint{3.520187in}{4.083552in}}%
\pgfpathlineto{\pgfqpoint{3.525838in}{4.084476in}}%
\pgfpathlineto{\pgfqpoint{3.525838in}{4.085400in}}%
\pgfpathlineto{\pgfqpoint{3.528227in}{4.086324in}}%
\pgfpathlineto{\pgfqpoint{3.528645in}{4.088172in}}%
\pgfpathlineto{\pgfqpoint{3.530898in}{4.089096in}}%
\pgfpathlineto{\pgfqpoint{3.531789in}{4.090944in}}%
\pgfpathlineto{\pgfqpoint{3.536481in}{4.091868in}}%
\pgfpathlineto{\pgfqpoint{3.536481in}{4.092792in}}%
\pgfpathlineto{\pgfqpoint{3.552541in}{4.093716in}}%
\pgfpathlineto{\pgfqpoint{3.552541in}{4.094640in}}%
\pgfpathlineto{\pgfqpoint{3.560307in}{4.095564in}}%
\pgfpathlineto{\pgfqpoint{3.561366in}{4.097412in}}%
\pgfpathlineto{\pgfqpoint{3.570623in}{4.098336in}}%
\pgfpathlineto{\pgfqpoint{3.570623in}{4.099260in}}%
\pgfpathlineto{\pgfqpoint{3.580581in}{4.100184in}}%
\pgfpathlineto{\pgfqpoint{3.580581in}{4.101108in}}%
\pgfpathlineto{\pgfqpoint{3.585696in}{4.102032in}}%
\pgfpathlineto{\pgfqpoint{3.585696in}{4.102956in}}%
\pgfpathlineto{\pgfqpoint{3.590246in}{4.103880in}}%
\pgfpathlineto{\pgfqpoint{3.590246in}{4.104804in}}%
\pgfpathlineto{\pgfqpoint{3.597269in}{4.105728in}}%
\pgfpathlineto{\pgfqpoint{3.597982in}{4.108500in}}%
\pgfpathlineto{\pgfqpoint{3.605259in}{4.109424in}}%
\pgfpathlineto{\pgfqpoint{3.605259in}{4.110348in}}%
\pgfpathlineto{\pgfqpoint{3.616047in}{4.111272in}}%
\pgfpathlineto{\pgfqpoint{3.616047in}{4.112196in}}%
\pgfpathlineto{\pgfqpoint{3.621316in}{4.113120in}}%
\pgfpathlineto{\pgfqpoint{3.621316in}{4.114044in}}%
\pgfpathlineto{\pgfqpoint{3.634008in}{4.114968in}}%
\pgfpathlineto{\pgfqpoint{3.634008in}{4.115892in}}%
\pgfpathlineto{\pgfqpoint{3.645750in}{4.116816in}}%
\pgfpathlineto{\pgfqpoint{3.645750in}{4.117740in}}%
\pgfpathlineto{\pgfqpoint{3.656621in}{4.118664in}}%
\pgfpathlineto{\pgfqpoint{3.657245in}{4.120512in}}%
\pgfpathlineto{\pgfqpoint{3.661965in}{4.121436in}}%
\pgfpathlineto{\pgfqpoint{3.661965in}{4.122360in}}%
\pgfpathlineto{\pgfqpoint{3.667606in}{4.123284in}}%
\pgfpathlineto{\pgfqpoint{3.667606in}{4.124208in}}%
\pgfpathlineto{\pgfqpoint{3.676911in}{4.125132in}}%
\pgfpathlineto{\pgfqpoint{3.676911in}{4.126056in}}%
\pgfpathlineto{\pgfqpoint{3.684088in}{4.126980in}}%
\pgfpathlineto{\pgfqpoint{3.684088in}{4.127904in}}%
\pgfpathlineto{\pgfqpoint{3.707738in}{4.128828in}}%
\pgfpathlineto{\pgfqpoint{3.707738in}{4.129752in}}%
\pgfpathlineto{\pgfqpoint{3.724224in}{4.130676in}}%
\pgfpathlineto{\pgfqpoint{3.724224in}{4.131600in}}%
\pgfpathlineto{\pgfqpoint{3.728573in}{4.132524in}}%
\pgfpathlineto{\pgfqpoint{3.728573in}{4.133448in}}%
\pgfpathlineto{\pgfqpoint{3.735993in}{4.134372in}}%
\pgfpathlineto{\pgfqpoint{3.735993in}{4.135296in}}%
\pgfpathlineto{\pgfqpoint{3.750517in}{4.136220in}}%
\pgfpathlineto{\pgfqpoint{3.750517in}{4.137144in}}%
\pgfpathlineto{\pgfqpoint{3.772819in}{4.138068in}}%
\pgfpathlineto{\pgfqpoint{3.772819in}{4.138992in}}%
\pgfpathlineto{\pgfqpoint{3.778118in}{4.139916in}}%
\pgfpathlineto{\pgfqpoint{3.778118in}{4.140840in}}%
\pgfpathlineto{\pgfqpoint{3.793183in}{4.141764in}}%
\pgfpathlineto{\pgfqpoint{3.793183in}{4.142688in}}%
\pgfpathlineto{\pgfqpoint{3.795642in}{4.143612in}}%
\pgfpathlineto{\pgfqpoint{3.795642in}{4.144536in}}%
\pgfpathlineto{\pgfqpoint{3.814241in}{4.145460in}}%
\pgfpathlineto{\pgfqpoint{3.814241in}{4.146384in}}%
\pgfpathlineto{\pgfqpoint{3.821042in}{4.147308in}}%
\pgfpathlineto{\pgfqpoint{3.821042in}{4.148232in}}%
\pgfpathlineto{\pgfqpoint{3.857451in}{4.149156in}}%
\pgfpathlineto{\pgfqpoint{3.857451in}{4.150080in}}%
\pgfpathlineto{\pgfqpoint{3.877524in}{4.151004in}}%
\pgfpathlineto{\pgfqpoint{3.877524in}{4.151928in}}%
\pgfpathlineto{\pgfqpoint{3.901370in}{4.152852in}}%
\pgfpathlineto{\pgfqpoint{3.901370in}{4.153776in}}%
\pgfpathlineto{\pgfqpoint{3.952685in}{4.154700in}}%
\pgfpathlineto{\pgfqpoint{3.953419in}{4.156548in}}%
\pgfpathlineto{\pgfqpoint{3.997798in}{4.157472in}}%
\pgfpathlineto{\pgfqpoint{3.997798in}{4.158396in}}%
\pgfpathlineto{\pgfqpoint{4.033404in}{4.159320in}}%
\pgfpathlineto{\pgfqpoint{4.033404in}{4.160244in}}%
\pgfpathlineto{\pgfqpoint{4.103907in}{4.161168in}}%
\pgfpathlineto{\pgfqpoint{4.103907in}{4.162092in}}%
\pgfpathlineto{\pgfqpoint{4.191389in}{4.163016in}}%
\pgfpathlineto{\pgfqpoint{4.191389in}{4.163940in}}%
\pgfpathlineto{\pgfqpoint{4.225037in}{4.164864in}}%
\pgfpathlineto{\pgfqpoint{4.225037in}{4.165788in}}%
\pgfpathlineto{\pgfqpoint{4.263058in}{4.166712in}}%
\pgfpathlineto{\pgfqpoint{4.263058in}{4.167636in}}%
\pgfpathlineto{\pgfqpoint{4.318284in}{4.168560in}}%
\pgfpathlineto{\pgfqpoint{4.318284in}{4.169484in}}%
\pgfpathlineto{\pgfqpoint{4.332599in}{4.170408in}}%
\pgfpathlineto{\pgfqpoint{4.332599in}{4.171332in}}%
\pgfpathlineto{\pgfqpoint{4.388415in}{4.172256in}}%
\pgfpathlineto{\pgfqpoint{4.388464in}{4.174104in}}%
\pgfpathlineto{\pgfqpoint{4.423076in}{4.175028in}}%
\pgfpathlineto{\pgfqpoint{4.423076in}{4.175952in}}%
\pgfpathlineto{\pgfqpoint{4.458052in}{4.176876in}}%
\pgfpathlineto{\pgfqpoint{4.458052in}{4.177800in}}%
\pgfpathlineto{\pgfqpoint{4.493161in}{4.178724in}}%
\pgfpathlineto{\pgfqpoint{4.493161in}{4.179648in}}%
\pgfpathlineto{\pgfqpoint{4.519632in}{4.180572in}}%
\pgfpathlineto{\pgfqpoint{4.519632in}{4.181496in}}%
\pgfpathlineto{\pgfqpoint{4.526786in}{4.182420in}}%
\pgfpathlineto{\pgfqpoint{4.526786in}{4.183344in}}%
\pgfpathlineto{\pgfqpoint{4.584781in}{4.184268in}}%
\pgfpathlineto{\pgfqpoint{4.584781in}{4.185192in}}%
\pgfpathlineto{\pgfqpoint{4.591346in}{4.186116in}}%
\pgfpathlineto{\pgfqpoint{4.591346in}{4.187040in}}%
\pgfpathlineto{\pgfqpoint{4.621205in}{4.187964in}}%
\pgfpathlineto{\pgfqpoint{4.621205in}{4.188888in}}%
\pgfpathlineto{\pgfqpoint{4.655358in}{4.189812in}}%
\pgfpathlineto{\pgfqpoint{4.655358in}{4.190736in}}%
\pgfpathlineto{\pgfqpoint{4.662457in}{4.191660in}}%
\pgfpathlineto{\pgfqpoint{4.662457in}{4.192584in}}%
\pgfpathlineto{\pgfqpoint{4.714275in}{4.193508in}}%
\pgfpathlineto{\pgfqpoint{4.714275in}{4.194432in}}%
\pgfpathlineto{\pgfqpoint{4.721915in}{4.195356in}}%
\pgfpathlineto{\pgfqpoint{4.721915in}{4.196280in}}%
\pgfpathlineto{\pgfqpoint{4.783636in}{4.197204in}}%
\pgfpathlineto{\pgfqpoint{4.783636in}{4.198128in}}%
\pgfpathlineto{\pgfqpoint{4.792412in}{4.199052in}}%
\pgfpathlineto{\pgfqpoint{4.792412in}{4.199976in}}%
\pgfpathlineto{\pgfqpoint{4.840252in}{4.200900in}}%
\pgfpathlineto{\pgfqpoint{4.840252in}{4.201824in}}%
\pgfpathlineto{\pgfqpoint{4.896010in}{4.202748in}}%
\pgfpathlineto{\pgfqpoint{4.896010in}{4.203672in}}%
\pgfpathlineto{\pgfqpoint{4.950254in}{4.204596in}}%
\pgfpathlineto{\pgfqpoint{4.950254in}{4.205520in}}%
\pgfpathlineto{\pgfqpoint{4.975981in}{4.206444in}}%
\pgfpathlineto{\pgfqpoint{4.975981in}{4.207368in}}%
\pgfpathlineto{\pgfqpoint{5.025878in}{4.208292in}}%
\pgfpathlineto{\pgfqpoint{5.025878in}{4.209216in}}%
\pgfpathlineto{\pgfqpoint{5.076978in}{4.210140in}}%
\pgfpathlineto{\pgfqpoint{5.076978in}{4.211064in}}%
\pgfpathlineto{\pgfqpoint{5.102424in}{4.211988in}}%
\pgfpathlineto{\pgfqpoint{5.102424in}{4.212912in}}%
\pgfpathlineto{\pgfqpoint{5.153960in}{4.213836in}}%
\pgfpathlineto{\pgfqpoint{5.153960in}{4.214760in}}%
\pgfpathlineto{\pgfqpoint{5.206942in}{4.215684in}}%
\pgfpathlineto{\pgfqpoint{5.206942in}{4.216608in}}%
\pgfpathlineto{\pgfqpoint{5.243629in}{4.217532in}}%
\pgfpathlineto{\pgfqpoint{5.243629in}{4.218456in}}%
\pgfpathlineto{\pgfqpoint{5.293772in}{4.219380in}}%
\pgfpathlineto{\pgfqpoint{5.293772in}{4.220304in}}%
\pgfpathlineto{\pgfqpoint{5.404929in}{4.221228in}}%
\pgfpathlineto{\pgfqpoint{5.404929in}{4.222152in}}%
\pgfpathlineto{\pgfqpoint{5.534545in}{4.223076in}}%
\pgfpathlineto{\pgfqpoint{5.534545in}{4.224000in}}%
\pgfpathlineto{\pgfqpoint{5.534545in}{4.224000in}}%
\pgfusepath{stroke}%
\end{pgfscope}%
\begin{pgfscope}%
\pgfsetrectcap%
\pgfsetmiterjoin%
\pgfsetlinewidth{0.803000pt}%
\definecolor{currentstroke}{rgb}{0.000000,0.000000,0.000000}%
\pgfsetstrokecolor{currentstroke}%
\pgfsetdash{}{0pt}%
\pgfpathmoveto{\pgfqpoint{0.800000in}{0.528000in}}%
\pgfpathlineto{\pgfqpoint{0.800000in}{4.224000in}}%
\pgfusepath{stroke}%
\end{pgfscope}%
\begin{pgfscope}%
\pgfsetrectcap%
\pgfsetmiterjoin%
\pgfsetlinewidth{0.803000pt}%
\definecolor{currentstroke}{rgb}{0.000000,0.000000,0.000000}%
\pgfsetstrokecolor{currentstroke}%
\pgfsetdash{}{0pt}%
\pgfpathmoveto{\pgfqpoint{5.760000in}{0.528000in}}%
\pgfpathlineto{\pgfqpoint{5.760000in}{4.224000in}}%
\pgfusepath{stroke}%
\end{pgfscope}%
\begin{pgfscope}%
\pgfsetrectcap%
\pgfsetmiterjoin%
\pgfsetlinewidth{0.803000pt}%
\definecolor{currentstroke}{rgb}{0.000000,0.000000,0.000000}%
\pgfsetstrokecolor{currentstroke}%
\pgfsetdash{}{0pt}%
\pgfpathmoveto{\pgfqpoint{0.800000in}{0.528000in}}%
\pgfpathlineto{\pgfqpoint{5.760000in}{0.528000in}}%
\pgfusepath{stroke}%
\end{pgfscope}%
\begin{pgfscope}%
\pgfsetrectcap%
\pgfsetmiterjoin%
\pgfsetlinewidth{0.803000pt}%
\definecolor{currentstroke}{rgb}{0.000000,0.000000,0.000000}%
\pgfsetstrokecolor{currentstroke}%
\pgfsetdash{}{0pt}%
\pgfpathmoveto{\pgfqpoint{0.800000in}{4.224000in}}%
\pgfpathlineto{\pgfqpoint{5.760000in}{4.224000in}}%
\pgfusepath{stroke}%
\end{pgfscope}%
\end{pgfpicture}%
\makeatother%
\endgroup%
}
\caption{Cumulative latency plot for the system when exhibiting stable latency.}
\label{ecdfstable}
\end{figure}

I now analyse the performance and behaviour of the system with different parameters and under different conditions. I argue that the optimisations described in Section~\ref{performance} were effective in improving system performance, but there are fundamental limitations caused by the latency costs of Cap'n Proto serialisation (Section~\ref{capnpbenchmark}) and (to a lesser extent) cryptography (Section~\ref{tezosbenchmark}).

To illustrate the performance of the system under different conditions, several heatmaps (Figure~\ref{heatmaps}) and a cumulative latency plot (Figure~\ref{ecdfstable}) are presented for tests run for 20s with 4 nodes. Figure~\ref{heatmaps}(a) and Figure~\ref{ecdfstable} show an experiment with a throughput of 200req/s and a batch size of 300. Figure~\ref{heatmaps}(b) shows an experiment with a throughout of 2000req/s and a batch size of 300. Figure~\ref{heatmaps}(c) shows an experiment with a throughout of 2000req/s and unlimited batch sizes.

In most cases, the system exhibits stable latency throughout an experiment while goodput is equal to throughput, meaning that the system is not overloaded (Figure~\ref{heatmaps}(a), Figure~\ref{ecdfstable}). When the throughput exceeds the amount the system can keep up with, there is rapid growth in latency as commands queue on the nodes (Figure~\ref{heatmaps}(b)). Since HotStuff is a partially synchronous protocol (Section~\ref{hotstufftheory}), an increase in latency means that view times increase, decreasing goodput. Once the system is overloaded, the goodput levels off at around its maximum value as throughput is increased.

The comparison of batch sizes in Section~\ref{batchsizeseval} indicates that the batching implementation described in Section~\ref{batching} is effective, as the system can achieve much greater goodput with batch sizes greater than 1 (equivalent to no batching). This section also provides evidence that serialisation latency is a bottleneck, as view times begin to increase exponentially as batch sizes increase (Figure~\ref{heatmaps}(c)), due to messages being larger and taking longer to serialise.

The study of node counts (Section~\ref{nodecountseval}) gives further evidence that message serialisation is a bottleneck; higher node counts mean more internal messages being sent, causing a decline in performance due to serialisation costs. This also supports the conclusion that cryptography is a bottleneck, as more nodes mean more messages must be signed and aggregated.

The ablation study (Section~\ref{ablation}) compares the performance of the system with different optimizations enabled, demonstrating their effectiveness in increasing goodput and lowering latency. It is also demonstrated that cryptography is a bottleneck, as there is an increase in latency with cryptography disabled.

In the wide area network (WAN) simulation study (Section~\ref{minineteval}), the performance of the system is evaluated in a simulated network (Section~\ref{testing}) with link latency similar to what may be observed in a wide area network.

In the view-change study (Section~\ref{viewchange}), it is shown that the view-change protocol (Section~\ref{viewchange}) effectively ensures the system progresses once a node has died, albeit with a significant performance penalty.

\subsection{Batch sizes} \label{batchsizeseval}

\begin{figure}[h!]
\centering
\resizebox{.6\textwidth}{!}{%% Creator: Matplotlib, PGF backend
%%
%% To include the figure in your LaTeX document, write
%%   \input{<filename>.pgf}
%%
%% Make sure the required packages are loaded in your preamble
%%   \usepackage{pgf}
%%
%% Also ensure that all the required font packages are loaded; for instance,
%% the lmodern package is sometimes necessary when using math font.
%%   \usepackage{lmodern}
%%
%% Figures using additional raster images can only be included by \input if
%% they are in the same directory as the main LaTeX file. For loading figures
%% from other directories you can use the `import` package
%%   \usepackage{import}
%%
%% and then include the figures with
%%   \import{<path to file>}{<filename>.pgf}
%%
%% Matplotlib used the following preamble
%%   
%%   \usepackage{fontspec}
%%   \setmainfont{DejaVuSerif.ttf}[Path=\detokenize{/opt/homebrew/lib/python3.10/site-packages/matplotlib/mpl-data/fonts/ttf/}]
%%   \setsansfont{DejaVuSans.ttf}[Path=\detokenize{/opt/homebrew/lib/python3.10/site-packages/matplotlib/mpl-data/fonts/ttf/}]
%%   \setmonofont{DejaVuSansMono.ttf}[Path=\detokenize{/opt/homebrew/lib/python3.10/site-packages/matplotlib/mpl-data/fonts/ttf/}]
%%   \makeatletter\@ifpackageloaded{underscore}{}{\usepackage[strings]{underscore}}\makeatother
%%
\begingroup%
\makeatletter%
\begin{pgfpicture}%
\pgfpathrectangle{\pgfpointorigin}{\pgfqpoint{6.400000in}{4.800000in}}%
\pgfusepath{use as bounding box, clip}%
\begin{pgfscope}%
\pgfsetbuttcap%
\pgfsetmiterjoin%
\definecolor{currentfill}{rgb}{1.000000,1.000000,1.000000}%
\pgfsetfillcolor{currentfill}%
\pgfsetlinewidth{0.000000pt}%
\definecolor{currentstroke}{rgb}{1.000000,1.000000,1.000000}%
\pgfsetstrokecolor{currentstroke}%
\pgfsetdash{}{0pt}%
\pgfpathmoveto{\pgfqpoint{0.000000in}{0.000000in}}%
\pgfpathlineto{\pgfqpoint{6.400000in}{0.000000in}}%
\pgfpathlineto{\pgfqpoint{6.400000in}{4.800000in}}%
\pgfpathlineto{\pgfqpoint{0.000000in}{4.800000in}}%
\pgfpathlineto{\pgfqpoint{0.000000in}{0.000000in}}%
\pgfpathclose%
\pgfusepath{fill}%
\end{pgfscope}%
\begin{pgfscope}%
\pgfsetbuttcap%
\pgfsetmiterjoin%
\definecolor{currentfill}{rgb}{1.000000,1.000000,1.000000}%
\pgfsetfillcolor{currentfill}%
\pgfsetlinewidth{0.000000pt}%
\definecolor{currentstroke}{rgb}{0.000000,0.000000,0.000000}%
\pgfsetstrokecolor{currentstroke}%
\pgfsetstrokeopacity{0.000000}%
\pgfsetdash{}{0pt}%
\pgfpathmoveto{\pgfqpoint{0.800000in}{0.528000in}}%
\pgfpathlineto{\pgfqpoint{5.760000in}{0.528000in}}%
\pgfpathlineto{\pgfqpoint{5.760000in}{4.224000in}}%
\pgfpathlineto{\pgfqpoint{0.800000in}{4.224000in}}%
\pgfpathlineto{\pgfqpoint{0.800000in}{0.528000in}}%
\pgfpathclose%
\pgfusepath{fill}%
\end{pgfscope}%
\begin{pgfscope}%
\pgfpathrectangle{\pgfqpoint{0.800000in}{0.528000in}}{\pgfqpoint{4.960000in}{3.696000in}}%
\pgfusepath{clip}%
\pgfsetbuttcap%
\pgfsetroundjoin%
\definecolor{currentfill}{rgb}{0.003922,0.450980,0.698039}%
\pgfsetfillcolor{currentfill}%
\pgfsetfillopacity{0.200000}%
\pgfsetlinewidth{1.003750pt}%
\definecolor{currentstroke}{rgb}{0.003922,0.450980,0.698039}%
\pgfsetstrokecolor{currentstroke}%
\pgfsetstrokeopacity{0.200000}%
\pgfsetdash{}{0pt}%
\pgfpathmoveto{\pgfqpoint{0.802480in}{0.531696in}}%
\pgfpathlineto{\pgfqpoint{0.802480in}{0.531696in}}%
\pgfpathlineto{\pgfqpoint{0.862000in}{0.620383in}}%
\pgfpathlineto{\pgfqpoint{0.924000in}{0.712800in}}%
\pgfpathlineto{\pgfqpoint{1.048000in}{0.897304in}}%
\pgfpathlineto{\pgfqpoint{1.296000in}{1.265931in}}%
\pgfpathlineto{\pgfqpoint{1.792000in}{1.953887in}}%
\pgfpathlineto{\pgfqpoint{2.288000in}{2.263846in}}%
\pgfpathlineto{\pgfqpoint{2.784000in}{2.404237in}}%
\pgfpathlineto{\pgfqpoint{3.280000in}{2.570015in}}%
\pgfpathlineto{\pgfqpoint{3.776000in}{2.634051in}}%
\pgfpathlineto{\pgfqpoint{4.272000in}{2.685133in}}%
\pgfpathlineto{\pgfqpoint{4.768000in}{2.798333in}}%
\pgfpathlineto{\pgfqpoint{5.264000in}{2.827639in}}%
\pgfpathlineto{\pgfqpoint{5.760000in}{2.880518in}}%
\pgfpathlineto{\pgfqpoint{6.256000in}{2.754683in}}%
\pgfpathlineto{\pgfqpoint{6.752000in}{2.892815in}}%
\pgfpathlineto{\pgfqpoint{7.248000in}{2.866140in}}%
\pgfpathlineto{\pgfqpoint{7.744000in}{2.903008in}}%
\pgfpathlineto{\pgfqpoint{8.240000in}{2.858901in}}%
\pgfpathlineto{\pgfqpoint{8.736000in}{2.877058in}}%
\pgfpathlineto{\pgfqpoint{9.232000in}{2.940106in}}%
\pgfpathlineto{\pgfqpoint{9.728000in}{2.838385in}}%
\pgfpathlineto{\pgfqpoint{10.224000in}{2.879732in}}%
\pgfpathlineto{\pgfqpoint{10.720000in}{2.884800in}}%
\pgfpathlineto{\pgfqpoint{10.720000in}{7.261723in}}%
\pgfpathlineto{\pgfqpoint{10.720000in}{7.261723in}}%
\pgfpathlineto{\pgfqpoint{10.224000in}{7.175690in}}%
\pgfpathlineto{\pgfqpoint{9.728000in}{6.853113in}}%
\pgfpathlineto{\pgfqpoint{9.232000in}{6.726760in}}%
\pgfpathlineto{\pgfqpoint{8.736000in}{6.332100in}}%
\pgfpathlineto{\pgfqpoint{8.240000in}{6.135435in}}%
\pgfpathlineto{\pgfqpoint{7.744000in}{5.949950in}}%
\pgfpathlineto{\pgfqpoint{7.248000in}{5.856517in}}%
\pgfpathlineto{\pgfqpoint{6.752000in}{5.480693in}}%
\pgfpathlineto{\pgfqpoint{6.256000in}{5.374108in}}%
\pgfpathlineto{\pgfqpoint{5.760000in}{5.235068in}}%
\pgfpathlineto{\pgfqpoint{5.264000in}{4.975342in}}%
\pgfpathlineto{\pgfqpoint{4.768000in}{4.603880in}}%
\pgfpathlineto{\pgfqpoint{4.272000in}{4.314526in}}%
\pgfpathlineto{\pgfqpoint{3.776000in}{3.948612in}}%
\pgfpathlineto{\pgfqpoint{3.280000in}{3.554929in}}%
\pgfpathlineto{\pgfqpoint{2.784000in}{3.107268in}}%
\pgfpathlineto{\pgfqpoint{2.288000in}{2.614297in}}%
\pgfpathlineto{\pgfqpoint{1.792000in}{1.995632in}}%
\pgfpathlineto{\pgfqpoint{1.296000in}{1.267039in}}%
\pgfpathlineto{\pgfqpoint{1.048000in}{0.897592in}}%
\pgfpathlineto{\pgfqpoint{0.924000in}{0.712800in}}%
\pgfpathlineto{\pgfqpoint{0.862000in}{0.620400in}}%
\pgfpathlineto{\pgfqpoint{0.802480in}{0.531696in}}%
\pgfpathlineto{\pgfqpoint{0.802480in}{0.531696in}}%
\pgfpathclose%
\pgfusepath{stroke,fill}%
\end{pgfscope}%
\begin{pgfscope}%
\pgfsetbuttcap%
\pgfsetroundjoin%
\definecolor{currentfill}{rgb}{0.000000,0.000000,0.000000}%
\pgfsetfillcolor{currentfill}%
\pgfsetlinewidth{0.803000pt}%
\definecolor{currentstroke}{rgb}{0.000000,0.000000,0.000000}%
\pgfsetstrokecolor{currentstroke}%
\pgfsetdash{}{0pt}%
\pgfsys@defobject{currentmarker}{\pgfqpoint{0.000000in}{-0.048611in}}{\pgfqpoint{0.000000in}{0.000000in}}{%
\pgfpathmoveto{\pgfqpoint{0.000000in}{0.000000in}}%
\pgfpathlineto{\pgfqpoint{0.000000in}{-0.048611in}}%
\pgfusepath{stroke,fill}%
}%
\begin{pgfscope}%
\pgfsys@transformshift{0.800000in}{0.528000in}%
\pgfsys@useobject{currentmarker}{}%
\end{pgfscope}%
\end{pgfscope}%
\begin{pgfscope}%
\definecolor{textcolor}{rgb}{0.000000,0.000000,0.000000}%
\pgfsetstrokecolor{textcolor}%
\pgfsetfillcolor{textcolor}%
\pgftext[x=0.800000in,y=0.430778in,,top]{\color{textcolor}\sffamily\fontsize{10.000000}{12.000000}\selectfont 0}%
\end{pgfscope}%
\begin{pgfscope}%
\pgfsetbuttcap%
\pgfsetroundjoin%
\definecolor{currentfill}{rgb}{0.000000,0.000000,0.000000}%
\pgfsetfillcolor{currentfill}%
\pgfsetlinewidth{0.803000pt}%
\definecolor{currentstroke}{rgb}{0.000000,0.000000,0.000000}%
\pgfsetstrokecolor{currentstroke}%
\pgfsetdash{}{0pt}%
\pgfsys@defobject{currentmarker}{\pgfqpoint{0.000000in}{-0.048611in}}{\pgfqpoint{0.000000in}{0.000000in}}{%
\pgfpathmoveto{\pgfqpoint{0.000000in}{0.000000in}}%
\pgfpathlineto{\pgfqpoint{0.000000in}{-0.048611in}}%
\pgfusepath{stroke,fill}%
}%
\begin{pgfscope}%
\pgfsys@transformshift{1.420000in}{0.528000in}%
\pgfsys@useobject{currentmarker}{}%
\end{pgfscope}%
\end{pgfscope}%
\begin{pgfscope}%
\definecolor{textcolor}{rgb}{0.000000,0.000000,0.000000}%
\pgfsetstrokecolor{textcolor}%
\pgfsetfillcolor{textcolor}%
\pgftext[x=1.420000in,y=0.430778in,,top]{\color{textcolor}\sffamily\fontsize{10.000000}{12.000000}\selectfont 250}%
\end{pgfscope}%
\begin{pgfscope}%
\pgfsetbuttcap%
\pgfsetroundjoin%
\definecolor{currentfill}{rgb}{0.000000,0.000000,0.000000}%
\pgfsetfillcolor{currentfill}%
\pgfsetlinewidth{0.803000pt}%
\definecolor{currentstroke}{rgb}{0.000000,0.000000,0.000000}%
\pgfsetstrokecolor{currentstroke}%
\pgfsetdash{}{0pt}%
\pgfsys@defobject{currentmarker}{\pgfqpoint{0.000000in}{-0.048611in}}{\pgfqpoint{0.000000in}{0.000000in}}{%
\pgfpathmoveto{\pgfqpoint{0.000000in}{0.000000in}}%
\pgfpathlineto{\pgfqpoint{0.000000in}{-0.048611in}}%
\pgfusepath{stroke,fill}%
}%
\begin{pgfscope}%
\pgfsys@transformshift{2.040000in}{0.528000in}%
\pgfsys@useobject{currentmarker}{}%
\end{pgfscope}%
\end{pgfscope}%
\begin{pgfscope}%
\definecolor{textcolor}{rgb}{0.000000,0.000000,0.000000}%
\pgfsetstrokecolor{textcolor}%
\pgfsetfillcolor{textcolor}%
\pgftext[x=2.040000in,y=0.430778in,,top]{\color{textcolor}\sffamily\fontsize{10.000000}{12.000000}\selectfont 500}%
\end{pgfscope}%
\begin{pgfscope}%
\pgfsetbuttcap%
\pgfsetroundjoin%
\definecolor{currentfill}{rgb}{0.000000,0.000000,0.000000}%
\pgfsetfillcolor{currentfill}%
\pgfsetlinewidth{0.803000pt}%
\definecolor{currentstroke}{rgb}{0.000000,0.000000,0.000000}%
\pgfsetstrokecolor{currentstroke}%
\pgfsetdash{}{0pt}%
\pgfsys@defobject{currentmarker}{\pgfqpoint{0.000000in}{-0.048611in}}{\pgfqpoint{0.000000in}{0.000000in}}{%
\pgfpathmoveto{\pgfqpoint{0.000000in}{0.000000in}}%
\pgfpathlineto{\pgfqpoint{0.000000in}{-0.048611in}}%
\pgfusepath{stroke,fill}%
}%
\begin{pgfscope}%
\pgfsys@transformshift{2.660000in}{0.528000in}%
\pgfsys@useobject{currentmarker}{}%
\end{pgfscope}%
\end{pgfscope}%
\begin{pgfscope}%
\definecolor{textcolor}{rgb}{0.000000,0.000000,0.000000}%
\pgfsetstrokecolor{textcolor}%
\pgfsetfillcolor{textcolor}%
\pgftext[x=2.660000in,y=0.430778in,,top]{\color{textcolor}\sffamily\fontsize{10.000000}{12.000000}\selectfont 750}%
\end{pgfscope}%
\begin{pgfscope}%
\pgfsetbuttcap%
\pgfsetroundjoin%
\definecolor{currentfill}{rgb}{0.000000,0.000000,0.000000}%
\pgfsetfillcolor{currentfill}%
\pgfsetlinewidth{0.803000pt}%
\definecolor{currentstroke}{rgb}{0.000000,0.000000,0.000000}%
\pgfsetstrokecolor{currentstroke}%
\pgfsetdash{}{0pt}%
\pgfsys@defobject{currentmarker}{\pgfqpoint{0.000000in}{-0.048611in}}{\pgfqpoint{0.000000in}{0.000000in}}{%
\pgfpathmoveto{\pgfqpoint{0.000000in}{0.000000in}}%
\pgfpathlineto{\pgfqpoint{0.000000in}{-0.048611in}}%
\pgfusepath{stroke,fill}%
}%
\begin{pgfscope}%
\pgfsys@transformshift{3.280000in}{0.528000in}%
\pgfsys@useobject{currentmarker}{}%
\end{pgfscope}%
\end{pgfscope}%
\begin{pgfscope}%
\definecolor{textcolor}{rgb}{0.000000,0.000000,0.000000}%
\pgfsetstrokecolor{textcolor}%
\pgfsetfillcolor{textcolor}%
\pgftext[x=3.280000in,y=0.430778in,,top]{\color{textcolor}\sffamily\fontsize{10.000000}{12.000000}\selectfont 1000}%
\end{pgfscope}%
\begin{pgfscope}%
\pgfsetbuttcap%
\pgfsetroundjoin%
\definecolor{currentfill}{rgb}{0.000000,0.000000,0.000000}%
\pgfsetfillcolor{currentfill}%
\pgfsetlinewidth{0.803000pt}%
\definecolor{currentstroke}{rgb}{0.000000,0.000000,0.000000}%
\pgfsetstrokecolor{currentstroke}%
\pgfsetdash{}{0pt}%
\pgfsys@defobject{currentmarker}{\pgfqpoint{0.000000in}{-0.048611in}}{\pgfqpoint{0.000000in}{0.000000in}}{%
\pgfpathmoveto{\pgfqpoint{0.000000in}{0.000000in}}%
\pgfpathlineto{\pgfqpoint{0.000000in}{-0.048611in}}%
\pgfusepath{stroke,fill}%
}%
\begin{pgfscope}%
\pgfsys@transformshift{3.900000in}{0.528000in}%
\pgfsys@useobject{currentmarker}{}%
\end{pgfscope}%
\end{pgfscope}%
\begin{pgfscope}%
\definecolor{textcolor}{rgb}{0.000000,0.000000,0.000000}%
\pgfsetstrokecolor{textcolor}%
\pgfsetfillcolor{textcolor}%
\pgftext[x=3.900000in,y=0.430778in,,top]{\color{textcolor}\sffamily\fontsize{10.000000}{12.000000}\selectfont 1250}%
\end{pgfscope}%
\begin{pgfscope}%
\pgfsetbuttcap%
\pgfsetroundjoin%
\definecolor{currentfill}{rgb}{0.000000,0.000000,0.000000}%
\pgfsetfillcolor{currentfill}%
\pgfsetlinewidth{0.803000pt}%
\definecolor{currentstroke}{rgb}{0.000000,0.000000,0.000000}%
\pgfsetstrokecolor{currentstroke}%
\pgfsetdash{}{0pt}%
\pgfsys@defobject{currentmarker}{\pgfqpoint{0.000000in}{-0.048611in}}{\pgfqpoint{0.000000in}{0.000000in}}{%
\pgfpathmoveto{\pgfqpoint{0.000000in}{0.000000in}}%
\pgfpathlineto{\pgfqpoint{0.000000in}{-0.048611in}}%
\pgfusepath{stroke,fill}%
}%
\begin{pgfscope}%
\pgfsys@transformshift{4.520000in}{0.528000in}%
\pgfsys@useobject{currentmarker}{}%
\end{pgfscope}%
\end{pgfscope}%
\begin{pgfscope}%
\definecolor{textcolor}{rgb}{0.000000,0.000000,0.000000}%
\pgfsetstrokecolor{textcolor}%
\pgfsetfillcolor{textcolor}%
\pgftext[x=4.520000in,y=0.430778in,,top]{\color{textcolor}\sffamily\fontsize{10.000000}{12.000000}\selectfont 1500}%
\end{pgfscope}%
\begin{pgfscope}%
\pgfsetbuttcap%
\pgfsetroundjoin%
\definecolor{currentfill}{rgb}{0.000000,0.000000,0.000000}%
\pgfsetfillcolor{currentfill}%
\pgfsetlinewidth{0.803000pt}%
\definecolor{currentstroke}{rgb}{0.000000,0.000000,0.000000}%
\pgfsetstrokecolor{currentstroke}%
\pgfsetdash{}{0pt}%
\pgfsys@defobject{currentmarker}{\pgfqpoint{0.000000in}{-0.048611in}}{\pgfqpoint{0.000000in}{0.000000in}}{%
\pgfpathmoveto{\pgfqpoint{0.000000in}{0.000000in}}%
\pgfpathlineto{\pgfqpoint{0.000000in}{-0.048611in}}%
\pgfusepath{stroke,fill}%
}%
\begin{pgfscope}%
\pgfsys@transformshift{5.140000in}{0.528000in}%
\pgfsys@useobject{currentmarker}{}%
\end{pgfscope}%
\end{pgfscope}%
\begin{pgfscope}%
\definecolor{textcolor}{rgb}{0.000000,0.000000,0.000000}%
\pgfsetstrokecolor{textcolor}%
\pgfsetfillcolor{textcolor}%
\pgftext[x=5.140000in,y=0.430778in,,top]{\color{textcolor}\sffamily\fontsize{10.000000}{12.000000}\selectfont 1750}%
\end{pgfscope}%
\begin{pgfscope}%
\pgfsetbuttcap%
\pgfsetroundjoin%
\definecolor{currentfill}{rgb}{0.000000,0.000000,0.000000}%
\pgfsetfillcolor{currentfill}%
\pgfsetlinewidth{0.803000pt}%
\definecolor{currentstroke}{rgb}{0.000000,0.000000,0.000000}%
\pgfsetstrokecolor{currentstroke}%
\pgfsetdash{}{0pt}%
\pgfsys@defobject{currentmarker}{\pgfqpoint{0.000000in}{-0.048611in}}{\pgfqpoint{0.000000in}{0.000000in}}{%
\pgfpathmoveto{\pgfqpoint{0.000000in}{0.000000in}}%
\pgfpathlineto{\pgfqpoint{0.000000in}{-0.048611in}}%
\pgfusepath{stroke,fill}%
}%
\begin{pgfscope}%
\pgfsys@transformshift{5.760000in}{0.528000in}%
\pgfsys@useobject{currentmarker}{}%
\end{pgfscope}%
\end{pgfscope}%
\begin{pgfscope}%
\definecolor{textcolor}{rgb}{0.000000,0.000000,0.000000}%
\pgfsetstrokecolor{textcolor}%
\pgfsetfillcolor{textcolor}%
\pgftext[x=5.760000in,y=0.430778in,,top]{\color{textcolor}\sffamily\fontsize{10.000000}{12.000000}\selectfont 2000}%
\end{pgfscope}%
\begin{pgfscope}%
\definecolor{textcolor}{rgb}{0.000000,0.000000,0.000000}%
\pgfsetstrokecolor{textcolor}%
\pgfsetfillcolor{textcolor}%
\pgftext[x=3.280000in,y=0.240809in,,top]{\color{textcolor}\sffamily\fontsize{10.000000}{12.000000}\selectfont throughput (req/s)}%
\end{pgfscope}%
\begin{pgfscope}%
\pgfsetbuttcap%
\pgfsetroundjoin%
\definecolor{currentfill}{rgb}{0.000000,0.000000,0.000000}%
\pgfsetfillcolor{currentfill}%
\pgfsetlinewidth{0.803000pt}%
\definecolor{currentstroke}{rgb}{0.000000,0.000000,0.000000}%
\pgfsetstrokecolor{currentstroke}%
\pgfsetdash{}{0pt}%
\pgfsys@defobject{currentmarker}{\pgfqpoint{-0.048611in}{0.000000in}}{\pgfqpoint{-0.000000in}{0.000000in}}{%
\pgfpathmoveto{\pgfqpoint{-0.000000in}{0.000000in}}%
\pgfpathlineto{\pgfqpoint{-0.048611in}{0.000000in}}%
\pgfusepath{stroke,fill}%
}%
\begin{pgfscope}%
\pgfsys@transformshift{0.800000in}{0.528000in}%
\pgfsys@useobject{currentmarker}{}%
\end{pgfscope}%
\end{pgfscope}%
\begin{pgfscope}%
\definecolor{textcolor}{rgb}{0.000000,0.000000,0.000000}%
\pgfsetstrokecolor{textcolor}%
\pgfsetfillcolor{textcolor}%
\pgftext[x=0.614412in, y=0.475238in, left, base]{\color{textcolor}\sffamily\fontsize{10.000000}{12.000000}\selectfont 0}%
\end{pgfscope}%
\begin{pgfscope}%
\pgfsetbuttcap%
\pgfsetroundjoin%
\definecolor{currentfill}{rgb}{0.000000,0.000000,0.000000}%
\pgfsetfillcolor{currentfill}%
\pgfsetlinewidth{0.803000pt}%
\definecolor{currentstroke}{rgb}{0.000000,0.000000,0.000000}%
\pgfsetstrokecolor{currentstroke}%
\pgfsetdash{}{0pt}%
\pgfsys@defobject{currentmarker}{\pgfqpoint{-0.048611in}{0.000000in}}{\pgfqpoint{-0.000000in}{0.000000in}}{%
\pgfpathmoveto{\pgfqpoint{-0.000000in}{0.000000in}}%
\pgfpathlineto{\pgfqpoint{-0.048611in}{0.000000in}}%
\pgfusepath{stroke,fill}%
}%
\begin{pgfscope}%
\pgfsys@transformshift{0.800000in}{1.267200in}%
\pgfsys@useobject{currentmarker}{}%
\end{pgfscope}%
\end{pgfscope}%
\begin{pgfscope}%
\definecolor{textcolor}{rgb}{0.000000,0.000000,0.000000}%
\pgfsetstrokecolor{textcolor}%
\pgfsetfillcolor{textcolor}%
\pgftext[x=0.437682in, y=1.214438in, left, base]{\color{textcolor}\sffamily\fontsize{10.000000}{12.000000}\selectfont 200}%
\end{pgfscope}%
\begin{pgfscope}%
\pgfsetbuttcap%
\pgfsetroundjoin%
\definecolor{currentfill}{rgb}{0.000000,0.000000,0.000000}%
\pgfsetfillcolor{currentfill}%
\pgfsetlinewidth{0.803000pt}%
\definecolor{currentstroke}{rgb}{0.000000,0.000000,0.000000}%
\pgfsetstrokecolor{currentstroke}%
\pgfsetdash{}{0pt}%
\pgfsys@defobject{currentmarker}{\pgfqpoint{-0.048611in}{0.000000in}}{\pgfqpoint{-0.000000in}{0.000000in}}{%
\pgfpathmoveto{\pgfqpoint{-0.000000in}{0.000000in}}%
\pgfpathlineto{\pgfqpoint{-0.048611in}{0.000000in}}%
\pgfusepath{stroke,fill}%
}%
\begin{pgfscope}%
\pgfsys@transformshift{0.800000in}{2.006400in}%
\pgfsys@useobject{currentmarker}{}%
\end{pgfscope}%
\end{pgfscope}%
\begin{pgfscope}%
\definecolor{textcolor}{rgb}{0.000000,0.000000,0.000000}%
\pgfsetstrokecolor{textcolor}%
\pgfsetfillcolor{textcolor}%
\pgftext[x=0.437682in, y=1.953638in, left, base]{\color{textcolor}\sffamily\fontsize{10.000000}{12.000000}\selectfont 400}%
\end{pgfscope}%
\begin{pgfscope}%
\pgfsetbuttcap%
\pgfsetroundjoin%
\definecolor{currentfill}{rgb}{0.000000,0.000000,0.000000}%
\pgfsetfillcolor{currentfill}%
\pgfsetlinewidth{0.803000pt}%
\definecolor{currentstroke}{rgb}{0.000000,0.000000,0.000000}%
\pgfsetstrokecolor{currentstroke}%
\pgfsetdash{}{0pt}%
\pgfsys@defobject{currentmarker}{\pgfqpoint{-0.048611in}{0.000000in}}{\pgfqpoint{-0.000000in}{0.000000in}}{%
\pgfpathmoveto{\pgfqpoint{-0.000000in}{0.000000in}}%
\pgfpathlineto{\pgfqpoint{-0.048611in}{0.000000in}}%
\pgfusepath{stroke,fill}%
}%
\begin{pgfscope}%
\pgfsys@transformshift{0.800000in}{2.745600in}%
\pgfsys@useobject{currentmarker}{}%
\end{pgfscope}%
\end{pgfscope}%
\begin{pgfscope}%
\definecolor{textcolor}{rgb}{0.000000,0.000000,0.000000}%
\pgfsetstrokecolor{textcolor}%
\pgfsetfillcolor{textcolor}%
\pgftext[x=0.437682in, y=2.692838in, left, base]{\color{textcolor}\sffamily\fontsize{10.000000}{12.000000}\selectfont 600}%
\end{pgfscope}%
\begin{pgfscope}%
\pgfsetbuttcap%
\pgfsetroundjoin%
\definecolor{currentfill}{rgb}{0.000000,0.000000,0.000000}%
\pgfsetfillcolor{currentfill}%
\pgfsetlinewidth{0.803000pt}%
\definecolor{currentstroke}{rgb}{0.000000,0.000000,0.000000}%
\pgfsetstrokecolor{currentstroke}%
\pgfsetdash{}{0pt}%
\pgfsys@defobject{currentmarker}{\pgfqpoint{-0.048611in}{0.000000in}}{\pgfqpoint{-0.000000in}{0.000000in}}{%
\pgfpathmoveto{\pgfqpoint{-0.000000in}{0.000000in}}%
\pgfpathlineto{\pgfqpoint{-0.048611in}{0.000000in}}%
\pgfusepath{stroke,fill}%
}%
\begin{pgfscope}%
\pgfsys@transformshift{0.800000in}{3.484800in}%
\pgfsys@useobject{currentmarker}{}%
\end{pgfscope}%
\end{pgfscope}%
\begin{pgfscope}%
\definecolor{textcolor}{rgb}{0.000000,0.000000,0.000000}%
\pgfsetstrokecolor{textcolor}%
\pgfsetfillcolor{textcolor}%
\pgftext[x=0.437682in, y=3.432038in, left, base]{\color{textcolor}\sffamily\fontsize{10.000000}{12.000000}\selectfont 800}%
\end{pgfscope}%
\begin{pgfscope}%
\pgfsetbuttcap%
\pgfsetroundjoin%
\definecolor{currentfill}{rgb}{0.000000,0.000000,0.000000}%
\pgfsetfillcolor{currentfill}%
\pgfsetlinewidth{0.803000pt}%
\definecolor{currentstroke}{rgb}{0.000000,0.000000,0.000000}%
\pgfsetstrokecolor{currentstroke}%
\pgfsetdash{}{0pt}%
\pgfsys@defobject{currentmarker}{\pgfqpoint{-0.048611in}{0.000000in}}{\pgfqpoint{-0.000000in}{0.000000in}}{%
\pgfpathmoveto{\pgfqpoint{-0.000000in}{0.000000in}}%
\pgfpathlineto{\pgfqpoint{-0.048611in}{0.000000in}}%
\pgfusepath{stroke,fill}%
}%
\begin{pgfscope}%
\pgfsys@transformshift{0.800000in}{4.224000in}%
\pgfsys@useobject{currentmarker}{}%
\end{pgfscope}%
\end{pgfscope}%
\begin{pgfscope}%
\definecolor{textcolor}{rgb}{0.000000,0.000000,0.000000}%
\pgfsetstrokecolor{textcolor}%
\pgfsetfillcolor{textcolor}%
\pgftext[x=0.349316in, y=4.171238in, left, base]{\color{textcolor}\sffamily\fontsize{10.000000}{12.000000}\selectfont 1000}%
\end{pgfscope}%
\begin{pgfscope}%
\definecolor{textcolor}{rgb}{0.000000,0.000000,0.000000}%
\pgfsetstrokecolor{textcolor}%
\pgfsetfillcolor{textcolor}%
\pgftext[x=0.293761in,y=2.376000in,,bottom,rotate=90.000000]{\color{textcolor}\sffamily\fontsize{10.000000}{12.000000}\selectfont goodput (req/s)}%
\end{pgfscope}%
\begin{pgfscope}%
\pgfpathrectangle{\pgfqpoint{0.800000in}{0.528000in}}{\pgfqpoint{4.960000in}{3.696000in}}%
\pgfusepath{clip}%
\pgfsetbuttcap%
\pgfsetroundjoin%
\pgfsetlinewidth{1.505625pt}%
\definecolor{currentstroke}{rgb}{0.003922,0.450980,0.698039}%
\pgfsetstrokecolor{currentstroke}%
\pgfsetdash{{5.550000pt}{2.400000pt}}{0.000000pt}%
\pgfpathmoveto{\pgfqpoint{0.802480in}{0.531696in}}%
\pgfpathlineto{\pgfqpoint{0.862000in}{0.620394in}}%
\pgfpathlineto{\pgfqpoint{0.924000in}{0.712800in}}%
\pgfpathlineto{\pgfqpoint{1.048000in}{0.897482in}}%
\pgfpathlineto{\pgfqpoint{1.296000in}{1.266553in}}%
\pgfpathlineto{\pgfqpoint{1.792000in}{1.977061in}}%
\pgfpathlineto{\pgfqpoint{2.288000in}{2.444062in}}%
\pgfpathlineto{\pgfqpoint{2.784000in}{2.764210in}}%
\pgfpathlineto{\pgfqpoint{3.280000in}{3.041182in}}%
\pgfpathlineto{\pgfqpoint{3.776000in}{3.292672in}}%
\pgfpathlineto{\pgfqpoint{4.272000in}{3.469831in}}%
\pgfpathlineto{\pgfqpoint{4.768000in}{3.678780in}}%
\pgfpathlineto{\pgfqpoint{5.264000in}{3.853168in}}%
\pgfpathlineto{\pgfqpoint{5.760000in}{4.010204in}}%
\pgfpathlineto{\pgfqpoint{5.770000in}{4.011028in}}%
\pgfusepath{stroke}%
\end{pgfscope}%
\begin{pgfscope}%
\pgfsetrectcap%
\pgfsetmiterjoin%
\pgfsetlinewidth{0.803000pt}%
\definecolor{currentstroke}{rgb}{0.000000,0.000000,0.000000}%
\pgfsetstrokecolor{currentstroke}%
\pgfsetdash{}{0pt}%
\pgfpathmoveto{\pgfqpoint{0.800000in}{0.528000in}}%
\pgfpathlineto{\pgfqpoint{0.800000in}{4.224000in}}%
\pgfusepath{stroke}%
\end{pgfscope}%
\begin{pgfscope}%
\pgfsetrectcap%
\pgfsetmiterjoin%
\pgfsetlinewidth{0.803000pt}%
\definecolor{currentstroke}{rgb}{0.000000,0.000000,0.000000}%
\pgfsetstrokecolor{currentstroke}%
\pgfsetdash{}{0pt}%
\pgfpathmoveto{\pgfqpoint{5.760000in}{0.528000in}}%
\pgfpathlineto{\pgfqpoint{5.760000in}{4.224000in}}%
\pgfusepath{stroke}%
\end{pgfscope}%
\begin{pgfscope}%
\pgfsetrectcap%
\pgfsetmiterjoin%
\pgfsetlinewidth{0.803000pt}%
\definecolor{currentstroke}{rgb}{0.000000,0.000000,0.000000}%
\pgfsetstrokecolor{currentstroke}%
\pgfsetdash{}{0pt}%
\pgfpathmoveto{\pgfqpoint{0.800000in}{0.528000in}}%
\pgfpathlineto{\pgfqpoint{5.760000in}{0.528000in}}%
\pgfusepath{stroke}%
\end{pgfscope}%
\begin{pgfscope}%
\pgfsetrectcap%
\pgfsetmiterjoin%
\pgfsetlinewidth{0.803000pt}%
\definecolor{currentstroke}{rgb}{0.000000,0.000000,0.000000}%
\pgfsetstrokecolor{currentstroke}%
\pgfsetdash{}{0pt}%
\pgfpathmoveto{\pgfqpoint{0.800000in}{4.224000in}}%
\pgfpathlineto{\pgfqpoint{5.760000in}{4.224000in}}%
\pgfusepath{stroke}%
\end{pgfscope}%
\begin{pgfscope}%
\pgfsetbuttcap%
\pgfsetmiterjoin%
\definecolor{currentfill}{rgb}{1.000000,1.000000,1.000000}%
\pgfsetfillcolor{currentfill}%
\pgfsetfillopacity{0.800000}%
\pgfsetlinewidth{1.003750pt}%
\definecolor{currentstroke}{rgb}{0.800000,0.800000,0.800000}%
\pgfsetstrokecolor{currentstroke}%
\pgfsetstrokeopacity{0.800000}%
\pgfsetdash{}{0pt}%
\pgfpathmoveto{\pgfqpoint{0.897222in}{3.705174in}}%
\pgfpathlineto{\pgfqpoint{1.658344in}{3.705174in}}%
\pgfpathquadraticcurveto{\pgfqpoint{1.686122in}{3.705174in}}{\pgfqpoint{1.686122in}{3.732952in}}%
\pgfpathlineto{\pgfqpoint{1.686122in}{4.126778in}}%
\pgfpathquadraticcurveto{\pgfqpoint{1.686122in}{4.154556in}}{\pgfqpoint{1.658344in}{4.154556in}}%
\pgfpathlineto{\pgfqpoint{0.897222in}{4.154556in}}%
\pgfpathquadraticcurveto{\pgfqpoint{0.869444in}{4.154556in}}{\pgfqpoint{0.869444in}{4.126778in}}%
\pgfpathlineto{\pgfqpoint{0.869444in}{3.732952in}}%
\pgfpathquadraticcurveto{\pgfqpoint{0.869444in}{3.705174in}}{\pgfqpoint{0.897222in}{3.705174in}}%
\pgfpathlineto{\pgfqpoint{0.897222in}{3.705174in}}%
\pgfpathclose%
\pgfusepath{stroke,fill}%
\end{pgfscope}%
\begin{pgfscope}%
\definecolor{textcolor}{rgb}{0.000000,0.000000,0.000000}%
\pgfsetstrokecolor{textcolor}%
\pgfsetfillcolor{textcolor}%
\pgftext[x=0.925000in,y=3.993477in,left,base]{\color{textcolor}\sffamily\fontsize{10.000000}{12.000000}\selectfont batch size}%
\end{pgfscope}%
\begin{pgfscope}%
\pgfsetrectcap%
\pgfsetroundjoin%
\pgfsetlinewidth{1.505625pt}%
\definecolor{currentstroke}{rgb}{0.003922,0.450980,0.698039}%
\pgfsetstrokecolor{currentstroke}%
\pgfsetdash{}{0pt}%
\pgfpathmoveto{\pgfqpoint{0.950791in}{3.838231in}}%
\pgfpathlineto{\pgfqpoint{1.089680in}{3.838231in}}%
\pgfpathlineto{\pgfqpoint{1.228569in}{3.838231in}}%
\pgfusepath{stroke}%
\end{pgfscope}%
\begin{pgfscope}%
\definecolor{textcolor}{rgb}{0.000000,0.000000,0.000000}%
\pgfsetstrokecolor{textcolor}%
\pgfsetfillcolor{textcolor}%
\pgftext[x=1.339680in,y=3.789620in,left,base]{\color{textcolor}\sffamily\fontsize{10.000000}{12.000000}\selectfont 300}%
\end{pgfscope}%
\end{pgfpicture}%
\makeatother%
\endgroup%
}
\caption{Benchmarking of goodput for varying throughputs and batch sizes.}
\label{throughputgoodputbatch}
\end{figure}

\begin{figure}[h!]
\centering
\resizebox{.6\textwidth}{!}{%% Creator: Matplotlib, PGF backend
%%
%% To include the figure in your LaTeX document, write
%%   \input{<filename>.pgf}
%%
%% Make sure the required packages are loaded in your preamble
%%   \usepackage{pgf}
%%
%% Also ensure that all the required font packages are loaded; for instance,
%% the lmodern package is sometimes necessary when using math font.
%%   \usepackage{lmodern}
%%
%% Figures using additional raster images can only be included by \input if
%% they are in the same directory as the main LaTeX file. For loading figures
%% from other directories you can use the `import` package
%%   \usepackage{import}
%%
%% and then include the figures with
%%   \import{<path to file>}{<filename>.pgf}
%%
%% Matplotlib used the following preamble
%%   
%%   \usepackage{fontspec}
%%   \setmainfont{DejaVuSerif.ttf}[Path=\detokenize{/opt/homebrew/lib/python3.10/site-packages/matplotlib/mpl-data/fonts/ttf/}]
%%   \setsansfont{DejaVuSans.ttf}[Path=\detokenize{/opt/homebrew/lib/python3.10/site-packages/matplotlib/mpl-data/fonts/ttf/}]
%%   \setmonofont{DejaVuSansMono.ttf}[Path=\detokenize{/opt/homebrew/lib/python3.10/site-packages/matplotlib/mpl-data/fonts/ttf/}]
%%   \makeatletter\@ifpackageloaded{underscore}{}{\usepackage[strings]{underscore}}\makeatother
%%
\begingroup%
\makeatletter%
\begin{pgfpicture}%
\pgfpathrectangle{\pgfpointorigin}{\pgfqpoint{5.840000in}{5.000000in}}%
\pgfusepath{use as bounding box, clip}%
\begin{pgfscope}%
\pgfsetbuttcap%
\pgfsetmiterjoin%
\definecolor{currentfill}{rgb}{1.000000,1.000000,1.000000}%
\pgfsetfillcolor{currentfill}%
\pgfsetlinewidth{0.000000pt}%
\definecolor{currentstroke}{rgb}{1.000000,1.000000,1.000000}%
\pgfsetstrokecolor{currentstroke}%
\pgfsetdash{}{0pt}%
\pgfpathmoveto{\pgfqpoint{0.000000in}{0.000000in}}%
\pgfpathlineto{\pgfqpoint{5.840000in}{0.000000in}}%
\pgfpathlineto{\pgfqpoint{5.840000in}{5.000000in}}%
\pgfpathlineto{\pgfqpoint{0.000000in}{5.000000in}}%
\pgfpathlineto{\pgfqpoint{0.000000in}{0.000000in}}%
\pgfpathclose%
\pgfusepath{fill}%
\end{pgfscope}%
\begin{pgfscope}%
\pgfsetbuttcap%
\pgfsetmiterjoin%
\definecolor{currentfill}{rgb}{1.000000,1.000000,1.000000}%
\pgfsetfillcolor{currentfill}%
\pgfsetlinewidth{0.000000pt}%
\definecolor{currentstroke}{rgb}{0.000000,0.000000,0.000000}%
\pgfsetstrokecolor{currentstroke}%
\pgfsetstrokeopacity{0.000000}%
\pgfsetdash{}{0pt}%
\pgfpathmoveto{\pgfqpoint{0.775584in}{0.582778in}}%
\pgfpathlineto{\pgfqpoint{4.939579in}{0.582778in}}%
\pgfpathlineto{\pgfqpoint{4.939579in}{4.850000in}}%
\pgfpathlineto{\pgfqpoint{0.775584in}{4.850000in}}%
\pgfpathlineto{\pgfqpoint{0.775584in}{0.582778in}}%
\pgfpathclose%
\pgfusepath{fill}%
\end{pgfscope}%
\begin{pgfscope}%
\pgfpathrectangle{\pgfqpoint{0.775584in}{0.582778in}}{\pgfqpoint{4.163995in}{4.267222in}}%
\pgfusepath{clip}%
\pgfsetbuttcap%
\pgfsetroundjoin%
\definecolor{currentfill}{rgb}{0.003922,0.450980,0.698039}%
\pgfsetfillcolor{currentfill}%
\pgfsetfillopacity{0.800000}%
\pgfsetlinewidth{1.003750pt}%
\definecolor{currentstroke}{rgb}{0.003922,0.450980,0.698039}%
\pgfsetstrokecolor{currentstroke}%
\pgfsetstrokeopacity{0.800000}%
\pgfsetdash{}{0pt}%
\pgfsys@defobject{currentmarker}{\pgfqpoint{-0.041667in}{-0.041667in}}{\pgfqpoint{0.041667in}{0.041667in}}{%
\pgfpathmoveto{\pgfqpoint{0.000000in}{-0.041667in}}%
\pgfpathcurveto{\pgfqpoint{0.011050in}{-0.041667in}}{\pgfqpoint{0.021649in}{-0.037276in}}{\pgfqpoint{0.029463in}{-0.029463in}}%
\pgfpathcurveto{\pgfqpoint{0.037276in}{-0.021649in}}{\pgfqpoint{0.041667in}{-0.011050in}}{\pgfqpoint{0.041667in}{0.000000in}}%
\pgfpathcurveto{\pgfqpoint{0.041667in}{0.011050in}}{\pgfqpoint{0.037276in}{0.021649in}}{\pgfqpoint{0.029463in}{0.029463in}}%
\pgfpathcurveto{\pgfqpoint{0.021649in}{0.037276in}}{\pgfqpoint{0.011050in}{0.041667in}}{\pgfqpoint{0.000000in}{0.041667in}}%
\pgfpathcurveto{\pgfqpoint{-0.011050in}{0.041667in}}{\pgfqpoint{-0.021649in}{0.037276in}}{\pgfqpoint{-0.029463in}{0.029463in}}%
\pgfpathcurveto{\pgfqpoint{-0.037276in}{0.021649in}}{\pgfqpoint{-0.041667in}{0.011050in}}{\pgfqpoint{-0.041667in}{0.000000in}}%
\pgfpathcurveto{\pgfqpoint{-0.041667in}{-0.011050in}}{\pgfqpoint{-0.037276in}{-0.021649in}}{\pgfqpoint{-0.029463in}{-0.029463in}}%
\pgfpathcurveto{\pgfqpoint{-0.021649in}{-0.037276in}}{\pgfqpoint{-0.011050in}{-0.041667in}}{\pgfqpoint{0.000000in}{-0.041667in}}%
\pgfpathlineto{\pgfqpoint{0.000000in}{-0.041667in}}%
\pgfpathclose%
\pgfusepath{stroke,fill}%
}%
\begin{pgfscope}%
\pgfsys@transformshift{0.879684in}{1.366021in}%
\pgfsys@useobject{currentmarker}{}%
\end{pgfscope}%
\begin{pgfscope}%
\pgfsys@transformshift{2.441182in}{1.162599in}%
\pgfsys@useobject{currentmarker}{}%
\end{pgfscope}%
\begin{pgfscope}%
\pgfsys@transformshift{1.191942in}{1.101537in}%
\pgfsys@useobject{currentmarker}{}%
\end{pgfscope}%
\begin{pgfscope}%
\pgfsys@transformshift{0.779748in}{1.158632in}%
\pgfsys@useobject{currentmarker}{}%
\end{pgfscope}%
\begin{pgfscope}%
\pgfsys@transformshift{3.231688in}{1.475453in}%
\pgfsys@useobject{currentmarker}{}%
\end{pgfscope}%
\begin{pgfscope}%
\pgfsys@transformshift{15.690565in}{1.312705in}%
\pgfsys@useobject{currentmarker}{}%
\end{pgfscope}%
\begin{pgfscope}%
\pgfsys@transformshift{10.754005in}{1.360062in}%
\pgfsys@useobject{currentmarker}{}%
\end{pgfscope}%
\begin{pgfscope}%
\pgfsys@transformshift{13.226384in}{1.401302in}%
\pgfsys@useobject{currentmarker}{}%
\end{pgfscope}%
\begin{pgfscope}%
\pgfsys@transformshift{14.897918in}{1.409979in}%
\pgfsys@useobject{currentmarker}{}%
\end{pgfscope}%
\begin{pgfscope}%
\pgfsys@transformshift{1.191984in}{1.070776in}%
\pgfsys@useobject{currentmarker}{}%
\end{pgfscope}%
\begin{pgfscope}%
\pgfsys@transformshift{0.983784in}{1.042237in}%
\pgfsys@useobject{currentmarker}{}%
\end{pgfscope}%
\begin{pgfscope}%
\pgfsys@transformshift{3.273981in}{1.176756in}%
\pgfsys@useobject{currentmarker}{}%
\end{pgfscope}%
\begin{pgfscope}%
\pgfsys@transformshift{4.103452in}{1.235595in}%
\pgfsys@useobject{currentmarker}{}%
\end{pgfscope}%
\begin{pgfscope}%
\pgfsys@transformshift{0.879684in}{1.055277in}%
\pgfsys@useobject{currentmarker}{}%
\end{pgfscope}%
\begin{pgfscope}%
\pgfsys@transformshift{2.434722in}{1.109604in}%
\pgfsys@useobject{currentmarker}{}%
\end{pgfscope}%
\begin{pgfscope}%
\pgfsys@transformshift{16.525289in}{1.313998in}%
\pgfsys@useobject{currentmarker}{}%
\end{pgfscope}%
\begin{pgfscope}%
\pgfsys@transformshift{17.420965in}{1.591464in}%
\pgfsys@useobject{currentmarker}{}%
\end{pgfscope}%
\begin{pgfscope}%
\pgfsys@transformshift{2.428544in}{1.257699in}%
\pgfsys@useobject{currentmarker}{}%
\end{pgfscope}%
\begin{pgfscope}%
\pgfsys@transformshift{10.742975in}{1.345668in}%
\pgfsys@useobject{currentmarker}{}%
\end{pgfscope}%
\begin{pgfscope}%
\pgfsys@transformshift{0.983784in}{1.697861in}%
\pgfsys@useobject{currentmarker}{}%
\end{pgfscope}%
\begin{pgfscope}%
\pgfsys@transformshift{6.605177in}{1.467102in}%
\pgfsys@useobject{currentmarker}{}%
\end{pgfscope}%
\begin{pgfscope}%
\pgfsys@transformshift{12.416163in}{1.367798in}%
\pgfsys@useobject{currentmarker}{}%
\end{pgfscope}%
\begin{pgfscope}%
\pgfsys@transformshift{0.779748in}{1.270344in}%
\pgfsys@useobject{currentmarker}{}%
\end{pgfscope}%
\begin{pgfscope}%
\pgfsys@transformshift{2.420999in}{2.417717in}%
\pgfsys@useobject{currentmarker}{}%
\end{pgfscope}%
\begin{pgfscope}%
\pgfsys@transformshift{1.191984in}{1.787472in}%
\pgfsys@useobject{currentmarker}{}%
\end{pgfscope}%
\begin{pgfscope}%
\pgfsys@transformshift{1.608383in}{1.462079in}%
\pgfsys@useobject{currentmarker}{}%
\end{pgfscope}%
\begin{pgfscope}%
\pgfsys@transformshift{0.879684in}{1.068042in}%
\pgfsys@useobject{currentmarker}{}%
\end{pgfscope}%
\begin{pgfscope}%
\pgfsys@transformshift{0.779748in}{1.099846in}%
\pgfsys@useobject{currentmarker}{}%
\end{pgfscope}%
\begin{pgfscope}%
\pgfsys@transformshift{2.333753in}{4.200735in}%
\pgfsys@useobject{currentmarker}{}%
\end{pgfscope}%
\begin{pgfscope}%
\pgfsys@transformshift{9.071022in}{1.300310in}%
\pgfsys@useobject{currentmarker}{}%
\end{pgfscope}%
\begin{pgfscope}%
\pgfsys@transformshift{7.410856in}{1.214691in}%
\pgfsys@useobject{currentmarker}{}%
\end{pgfscope}%
\begin{pgfscope}%
\pgfsys@transformshift{0.779748in}{2.185355in}%
\pgfsys@useobject{currentmarker}{}%
\end{pgfscope}%
\begin{pgfscope}%
\pgfsys@transformshift{8.569133in}{2.288404in}%
\pgfsys@useobject{currentmarker}{}%
\end{pgfscope}%
\begin{pgfscope}%
\pgfsys@transformshift{0.779748in}{1.124510in}%
\pgfsys@useobject{currentmarker}{}%
\end{pgfscope}%
\begin{pgfscope}%
\pgfsys@transformshift{0.879684in}{1.056201in}%
\pgfsys@useobject{currentmarker}{}%
\end{pgfscope}%
\begin{pgfscope}%
\pgfsys@transformshift{8.168948in}{2.978111in}%
\pgfsys@useobject{currentmarker}{}%
\end{pgfscope}%
\begin{pgfscope}%
\pgfsys@transformshift{0.779748in}{1.115711in}%
\pgfsys@useobject{currentmarker}{}%
\end{pgfscope}%
\begin{pgfscope}%
\pgfsys@transformshift{1.191984in}{1.108546in}%
\pgfsys@useobject{currentmarker}{}%
\end{pgfscope}%
\begin{pgfscope}%
\pgfsys@transformshift{3.265137in}{1.133321in}%
\pgfsys@useobject{currentmarker}{}%
\end{pgfscope}%
\begin{pgfscope}%
\pgfsys@transformshift{8.692771in}{1.876031in}%
\pgfsys@useobject{currentmarker}{}%
\end{pgfscope}%
\begin{pgfscope}%
\pgfsys@transformshift{0.983784in}{1.014434in}%
\pgfsys@useobject{currentmarker}{}%
\end{pgfscope}%
\begin{pgfscope}%
\pgfsys@transformshift{0.879684in}{1.076455in}%
\pgfsys@useobject{currentmarker}{}%
\end{pgfscope}%
\begin{pgfscope}%
\pgfsys@transformshift{0.983784in}{1.326593in}%
\pgfsys@useobject{currentmarker}{}%
\end{pgfscope}%
\begin{pgfscope}%
\pgfsys@transformshift{12.364185in}{1.404617in}%
\pgfsys@useobject{currentmarker}{}%
\end{pgfscope}%
\begin{pgfscope}%
\pgfsys@transformshift{4.106780in}{1.158822in}%
\pgfsys@useobject{currentmarker}{}%
\end{pgfscope}%
\begin{pgfscope}%
\pgfsys@transformshift{14.072650in}{1.456127in}%
\pgfsys@useobject{currentmarker}{}%
\end{pgfscope}%
\begin{pgfscope}%
\pgfsys@transformshift{0.879684in}{1.809467in}%
\pgfsys@useobject{currentmarker}{}%
\end{pgfscope}%
\begin{pgfscope}%
\pgfsys@transformshift{0.879684in}{0.991475in}%
\pgfsys@useobject{currentmarker}{}%
\end{pgfscope}%
\begin{pgfscope}%
\pgfsys@transformshift{4.106450in}{1.169290in}%
\pgfsys@useobject{currentmarker}{}%
\end{pgfscope}%
\begin{pgfscope}%
\pgfsys@transformshift{11.601971in}{1.352966in}%
\pgfsys@useobject{currentmarker}{}%
\end{pgfscope}%
\begin{pgfscope}%
\pgfsys@transformshift{1.191984in}{1.797212in}%
\pgfsys@useobject{currentmarker}{}%
\end{pgfscope}%
\begin{pgfscope}%
\pgfsys@transformshift{9.103574in}{1.324336in}%
\pgfsys@useobject{currentmarker}{}%
\end{pgfscope}%
\begin{pgfscope}%
\pgfsys@transformshift{4.813271in}{2.120940in}%
\pgfsys@useobject{currentmarker}{}%
\end{pgfscope}%
\begin{pgfscope}%
\pgfsys@transformshift{1.191306in}{1.069740in}%
\pgfsys@useobject{currentmarker}{}%
\end{pgfscope}%
\begin{pgfscope}%
\pgfsys@transformshift{4.915566in}{1.279457in}%
\pgfsys@useobject{currentmarker}{}%
\end{pgfscope}%
\begin{pgfscope}%
\pgfsys@transformshift{7.437976in}{1.772813in}%
\pgfsys@useobject{currentmarker}{}%
\end{pgfscope}%
\begin{pgfscope}%
\pgfsys@transformshift{1.608383in}{2.121694in}%
\pgfsys@useobject{currentmarker}{}%
\end{pgfscope}%
\begin{pgfscope}%
\pgfsys@transformshift{1.191944in}{1.106718in}%
\pgfsys@useobject{currentmarker}{}%
\end{pgfscope}%
\begin{pgfscope}%
\pgfsys@transformshift{6.549546in}{1.499619in}%
\pgfsys@useobject{currentmarker}{}%
\end{pgfscope}%
\begin{pgfscope}%
\pgfsys@transformshift{4.931316in}{1.159794in}%
\pgfsys@useobject{currentmarker}{}%
\end{pgfscope}%
\begin{pgfscope}%
\pgfsys@transformshift{17.387853in}{1.507688in}%
\pgfsys@useobject{currentmarker}{}%
\end{pgfscope}%
\begin{pgfscope}%
\pgfsys@transformshift{8.191165in}{1.851038in}%
\pgfsys@useobject{currentmarker}{}%
\end{pgfscope}%
\begin{pgfscope}%
\pgfsys@transformshift{4.931977in}{1.170923in}%
\pgfsys@useobject{currentmarker}{}%
\end{pgfscope}%
\begin{pgfscope}%
\pgfsys@transformshift{5.720201in}{1.335012in}%
\pgfsys@useobject{currentmarker}{}%
\end{pgfscope}%
\begin{pgfscope}%
\pgfsys@transformshift{4.106780in}{1.210327in}%
\pgfsys@useobject{currentmarker}{}%
\end{pgfscope}%
\begin{pgfscope}%
\pgfsys@transformshift{3.273981in}{1.169248in}%
\pgfsys@useobject{currentmarker}{}%
\end{pgfscope}%
\begin{pgfscope}%
\pgfsys@transformshift{9.059671in}{1.303627in}%
\pgfsys@useobject{currentmarker}{}%
\end{pgfscope}%
\begin{pgfscope}%
\pgfsys@transformshift{15.754825in}{1.489018in}%
\pgfsys@useobject{currentmarker}{}%
\end{pgfscope}%
\begin{pgfscope}%
\pgfsys@transformshift{16.598765in}{1.312941in}%
\pgfsys@useobject{currentmarker}{}%
\end{pgfscope}%
\begin{pgfscope}%
\pgfsys@transformshift{1.606172in}{1.202261in}%
\pgfsys@useobject{currentmarker}{}%
\end{pgfscope}%
\begin{pgfscope}%
\pgfsys@transformshift{0.983784in}{1.075965in}%
\pgfsys@useobject{currentmarker}{}%
\end{pgfscope}%
\begin{pgfscope}%
\pgfsys@transformshift{8.258711in}{4.779619in}%
\pgfsys@useobject{currentmarker}{}%
\end{pgfscope}%
\begin{pgfscope}%
\pgfsys@transformshift{0.983784in}{1.342422in}%
\pgfsys@useobject{currentmarker}{}%
\end{pgfscope}%
\begin{pgfscope}%
\pgfsys@transformshift{4.106780in}{1.198715in}%
\pgfsys@useobject{currentmarker}{}%
\end{pgfscope}%
\begin{pgfscope}%
\pgfsys@transformshift{4.695248in}{3.841688in}%
\pgfsys@useobject{currentmarker}{}%
\end{pgfscope}%
\begin{pgfscope}%
\pgfsys@transformshift{3.273981in}{1.137878in}%
\pgfsys@useobject{currentmarker}{}%
\end{pgfscope}%
\begin{pgfscope}%
\pgfsys@transformshift{4.711823in}{2.006035in}%
\pgfsys@useobject{currentmarker}{}%
\end{pgfscope}%
\begin{pgfscope}%
\pgfsys@transformshift{0.983784in}{1.076366in}%
\pgfsys@useobject{currentmarker}{}%
\end{pgfscope}%
\begin{pgfscope}%
\pgfsys@transformshift{0.879684in}{1.065496in}%
\pgfsys@useobject{currentmarker}{}%
\end{pgfscope}%
\begin{pgfscope}%
\pgfsys@transformshift{3.244041in}{1.503025in}%
\pgfsys@useobject{currentmarker}{}%
\end{pgfscope}%
\begin{pgfscope}%
\pgfsys@transformshift{2.285949in}{4.049219in}%
\pgfsys@useobject{currentmarker}{}%
\end{pgfscope}%
\begin{pgfscope}%
\pgfsys@transformshift{2.441182in}{1.282782in}%
\pgfsys@useobject{currentmarker}{}%
\end{pgfscope}%
\begin{pgfscope}%
\pgfsys@transformshift{1.191984in}{1.078876in}%
\pgfsys@useobject{currentmarker}{}%
\end{pgfscope}%
\begin{pgfscope}%
\pgfsys@transformshift{0.983784in}{1.643542in}%
\pgfsys@useobject{currentmarker}{}%
\end{pgfscope}%
\begin{pgfscope}%
\pgfsys@transformshift{5.764089in}{1.192776in}%
\pgfsys@useobject{currentmarker}{}%
\end{pgfscope}%
\begin{pgfscope}%
\pgfsys@transformshift{0.879684in}{1.072218in}%
\pgfsys@useobject{currentmarker}{}%
\end{pgfscope}%
\begin{pgfscope}%
\pgfsys@transformshift{9.889487in}{1.341326in}%
\pgfsys@useobject{currentmarker}{}%
\end{pgfscope}%
\begin{pgfscope}%
\pgfsys@transformshift{8.144285in}{1.495168in}%
\pgfsys@useobject{currentmarker}{}%
\end{pgfscope}%
\begin{pgfscope}%
\pgfsys@transformshift{5.717258in}{1.319399in}%
\pgfsys@useobject{currentmarker}{}%
\end{pgfscope}%
\begin{pgfscope}%
\pgfsys@transformshift{0.779748in}{1.093924in}%
\pgfsys@useobject{currentmarker}{}%
\end{pgfscope}%
\begin{pgfscope}%
\pgfsys@transformshift{1.604670in}{1.191602in}%
\pgfsys@useobject{currentmarker}{}%
\end{pgfscope}%
\begin{pgfscope}%
\pgfsys@transformshift{4.591963in}{1.969706in}%
\pgfsys@useobject{currentmarker}{}%
\end{pgfscope}%
\begin{pgfscope}%
\pgfsys@transformshift{7.437976in}{1.715056in}%
\pgfsys@useobject{currentmarker}{}%
\end{pgfscope}%
\begin{pgfscope}%
\pgfsys@transformshift{2.402387in}{2.301215in}%
\pgfsys@useobject{currentmarker}{}%
\end{pgfscope}%
\begin{pgfscope}%
\pgfsys@transformshift{2.422994in}{1.255773in}%
\pgfsys@useobject{currentmarker}{}%
\end{pgfscope}%
\begin{pgfscope}%
\pgfsys@transformshift{1.606747in}{1.131844in}%
\pgfsys@useobject{currentmarker}{}%
\end{pgfscope}%
\begin{pgfscope}%
\pgfsys@transformshift{0.779748in}{1.133821in}%
\pgfsys@useobject{currentmarker}{}%
\end{pgfscope}%
\begin{pgfscope}%
\pgfsys@transformshift{0.779748in}{1.214152in}%
\pgfsys@useobject{currentmarker}{}%
\end{pgfscope}%
\begin{pgfscope}%
\pgfsys@transformshift{0.779748in}{1.094910in}%
\pgfsys@useobject{currentmarker}{}%
\end{pgfscope}%
\begin{pgfscope}%
\pgfsys@transformshift{0.779748in}{1.104423in}%
\pgfsys@useobject{currentmarker}{}%
\end{pgfscope}%
\begin{pgfscope}%
\pgfsys@transformshift{0.983784in}{1.108957in}%
\pgfsys@useobject{currentmarker}{}%
\end{pgfscope}%
\begin{pgfscope}%
\pgfsys@transformshift{16.417380in}{1.599253in}%
\pgfsys@useobject{currentmarker}{}%
\end{pgfscope}%
\begin{pgfscope}%
\pgfsys@transformshift{0.879684in}{1.359926in}%
\pgfsys@useobject{currentmarker}{}%
\end{pgfscope}%
\begin{pgfscope}%
\pgfsys@transformshift{4.015546in}{1.579875in}%
\pgfsys@useobject{currentmarker}{}%
\end{pgfscope}%
\begin{pgfscope}%
\pgfsys@transformshift{1.191984in}{1.053112in}%
\pgfsys@useobject{currentmarker}{}%
\end{pgfscope}%
\begin{pgfscope}%
\pgfsys@transformshift{0.983784in}{1.055286in}%
\pgfsys@useobject{currentmarker}{}%
\end{pgfscope}%
\begin{pgfscope}%
\pgfsys@transformshift{0.779748in}{1.661005in}%
\pgfsys@useobject{currentmarker}{}%
\end{pgfscope}%
\begin{pgfscope}%
\pgfsys@transformshift{0.879684in}{1.777641in}%
\pgfsys@useobject{currentmarker}{}%
\end{pgfscope}%
\begin{pgfscope}%
\pgfsys@transformshift{1.608383in}{1.469096in}%
\pgfsys@useobject{currentmarker}{}%
\end{pgfscope}%
\begin{pgfscope}%
\pgfsys@transformshift{10.769172in}{1.364858in}%
\pgfsys@useobject{currentmarker}{}%
\end{pgfscope}%
\begin{pgfscope}%
\pgfsys@transformshift{1.191779in}{1.102995in}%
\pgfsys@useobject{currentmarker}{}%
\end{pgfscope}%
\begin{pgfscope}%
\pgfsys@transformshift{2.439110in}{1.447311in}%
\pgfsys@useobject{currentmarker}{}%
\end{pgfscope}%
\begin{pgfscope}%
\pgfsys@transformshift{0.879570in}{1.116171in}%
\pgfsys@useobject{currentmarker}{}%
\end{pgfscope}%
\begin{pgfscope}%
\pgfsys@transformshift{0.779748in}{2.216596in}%
\pgfsys@useobject{currentmarker}{}%
\end{pgfscope}%
\begin{pgfscope}%
\pgfsys@transformshift{1.191984in}{1.343780in}%
\pgfsys@useobject{currentmarker}{}%
\end{pgfscope}%
\begin{pgfscope}%
\pgfsys@transformshift{2.392568in}{2.233622in}%
\pgfsys@useobject{currentmarker}{}%
\end{pgfscope}%
\begin{pgfscope}%
\pgfsys@transformshift{4.927423in}{1.249716in}%
\pgfsys@useobject{currentmarker}{}%
\end{pgfscope}%
\begin{pgfscope}%
\pgfsys@transformshift{3.986215in}{1.812245in}%
\pgfsys@useobject{currentmarker}{}%
\end{pgfscope}%
\begin{pgfscope}%
\pgfsys@transformshift{1.191984in}{1.330350in}%
\pgfsys@useobject{currentmarker}{}%
\end{pgfscope}%
\begin{pgfscope}%
\pgfsys@transformshift{1.607907in}{1.077085in}%
\pgfsys@useobject{currentmarker}{}%
\end{pgfscope}%
\begin{pgfscope}%
\pgfsys@transformshift{0.983784in}{1.041079in}%
\pgfsys@useobject{currentmarker}{}%
\end{pgfscope}%
\begin{pgfscope}%
\pgfsys@transformshift{11.601971in}{1.342452in}%
\pgfsys@useobject{currentmarker}{}%
\end{pgfscope}%
\begin{pgfscope}%
\pgfsys@transformshift{1.608383in}{1.208029in}%
\pgfsys@useobject{currentmarker}{}%
\end{pgfscope}%
\begin{pgfscope}%
\pgfsys@transformshift{1.608383in}{1.129400in}%
\pgfsys@useobject{currentmarker}{}%
\end{pgfscope}%
\begin{pgfscope}%
\pgfsys@transformshift{5.724948in}{1.342417in}%
\pgfsys@useobject{currentmarker}{}%
\end{pgfscope}%
\begin{pgfscope}%
\pgfsys@transformshift{6.605177in}{1.380483in}%
\pgfsys@useobject{currentmarker}{}%
\end{pgfscope}%
\begin{pgfscope}%
\pgfsys@transformshift{1.603944in}{1.121866in}%
\pgfsys@useobject{currentmarker}{}%
\end{pgfscope}%
\begin{pgfscope}%
\pgfsys@transformshift{6.605177in}{1.189456in}%
\pgfsys@useobject{currentmarker}{}%
\end{pgfscope}%
\begin{pgfscope}%
\pgfsys@transformshift{4.103037in}{1.144184in}%
\pgfsys@useobject{currentmarker}{}%
\end{pgfscope}%
\begin{pgfscope}%
\pgfsys@transformshift{1.608383in}{1.079455in}%
\pgfsys@useobject{currentmarker}{}%
\end{pgfscope}%
\begin{pgfscope}%
\pgfsys@transformshift{0.983784in}{1.081237in}%
\pgfsys@useobject{currentmarker}{}%
\end{pgfscope}%
\begin{pgfscope}%
\pgfsys@transformshift{1.608383in}{1.470042in}%
\pgfsys@useobject{currentmarker}{}%
\end{pgfscope}%
\begin{pgfscope}%
\pgfsys@transformshift{0.879684in}{1.001898in}%
\pgfsys@useobject{currentmarker}{}%
\end{pgfscope}%
\begin{pgfscope}%
\pgfsys@transformshift{8.081222in}{2.054577in}%
\pgfsys@useobject{currentmarker}{}%
\end{pgfscope}%
\begin{pgfscope}%
\pgfsys@transformshift{2.441182in}{1.170795in}%
\pgfsys@useobject{currentmarker}{}%
\end{pgfscope}%
\begin{pgfscope}%
\pgfsys@transformshift{14.893761in}{1.393745in}%
\pgfsys@useobject{currentmarker}{}%
\end{pgfscope}%
\begin{pgfscope}%
\pgfsys@transformshift{11.564106in}{1.341757in}%
\pgfsys@useobject{currentmarker}{}%
\end{pgfscope}%
\begin{pgfscope}%
\pgfsys@transformshift{13.954777in}{1.362334in}%
\pgfsys@useobject{currentmarker}{}%
\end{pgfscope}%
\begin{pgfscope}%
\pgfsys@transformshift{2.439943in}{1.094930in}%
\pgfsys@useobject{currentmarker}{}%
\end{pgfscope}%
\begin{pgfscope}%
\pgfsys@transformshift{6.588979in}{1.178470in}%
\pgfsys@useobject{currentmarker}{}%
\end{pgfscope}%
\begin{pgfscope}%
\pgfsys@transformshift{2.275712in}{4.462995in}%
\pgfsys@useobject{currentmarker}{}%
\end{pgfscope}%
\begin{pgfscope}%
\pgfsys@transformshift{0.983784in}{1.085326in}%
\pgfsys@useobject{currentmarker}{}%
\end{pgfscope}%
\begin{pgfscope}%
\pgfsys@transformshift{0.879684in}{1.398183in}%
\pgfsys@useobject{currentmarker}{}%
\end{pgfscope}%
\begin{pgfscope}%
\pgfsys@transformshift{0.983784in}{1.060596in}%
\pgfsys@useobject{currentmarker}{}%
\end{pgfscope}%
\begin{pgfscope}%
\pgfsys@transformshift{0.879684in}{1.836274in}%
\pgfsys@useobject{currentmarker}{}%
\end{pgfscope}%
\begin{pgfscope}%
\pgfsys@transformshift{13.156906in}{1.351267in}%
\pgfsys@useobject{currentmarker}{}%
\end{pgfscope}%
\begin{pgfscope}%
\pgfsys@transformshift{1.608383in}{1.147483in}%
\pgfsys@useobject{currentmarker}{}%
\end{pgfscope}%
\begin{pgfscope}%
\pgfsys@transformshift{0.983784in}{1.342461in}%
\pgfsys@useobject{currentmarker}{}%
\end{pgfscope}%
\begin{pgfscope}%
\pgfsys@transformshift{0.879684in}{1.066132in}%
\pgfsys@useobject{currentmarker}{}%
\end{pgfscope}%
\begin{pgfscope}%
\pgfsys@transformshift{0.983784in}{1.025610in}%
\pgfsys@useobject{currentmarker}{}%
\end{pgfscope}%
\begin{pgfscope}%
\pgfsys@transformshift{8.268532in}{1.305850in}%
\pgfsys@useobject{currentmarker}{}%
\end{pgfscope}%
\begin{pgfscope}%
\pgfsys@transformshift{1.608383in}{2.149923in}%
\pgfsys@useobject{currentmarker}{}%
\end{pgfscope}%
\begin{pgfscope}%
\pgfsys@transformshift{1.608383in}{2.167686in}%
\pgfsys@useobject{currentmarker}{}%
\end{pgfscope}%
\begin{pgfscope}%
\pgfsys@transformshift{13.984393in}{1.441342in}%
\pgfsys@useobject{currentmarker}{}%
\end{pgfscope}%
\begin{pgfscope}%
\pgfsys@transformshift{7.403800in}{1.274834in}%
\pgfsys@useobject{currentmarker}{}%
\end{pgfscope}%
\begin{pgfscope}%
\pgfsys@transformshift{9.936373in}{1.292082in}%
\pgfsys@useobject{currentmarker}{}%
\end{pgfscope}%
\begin{pgfscope}%
\pgfsys@transformshift{3.186469in}{2.139271in}%
\pgfsys@useobject{currentmarker}{}%
\end{pgfscope}%
\begin{pgfscope}%
\pgfsys@transformshift{2.441182in}{1.157986in}%
\pgfsys@useobject{currentmarker}{}%
\end{pgfscope}%
\begin{pgfscope}%
\pgfsys@transformshift{3.261524in}{1.439083in}%
\pgfsys@useobject{currentmarker}{}%
\end{pgfscope}%
\begin{pgfscope}%
\pgfsys@transformshift{2.437792in}{1.456123in}%
\pgfsys@useobject{currentmarker}{}%
\end{pgfscope}%
\begin{pgfscope}%
\pgfsys@transformshift{1.190555in}{1.056418in}%
\pgfsys@useobject{currentmarker}{}%
\end{pgfscope}%
\begin{pgfscope}%
\pgfsys@transformshift{8.270775in}{1.302672in}%
\pgfsys@useobject{currentmarker}{}%
\end{pgfscope}%
\begin{pgfscope}%
\pgfsys@transformshift{0.779748in}{2.089780in}%
\pgfsys@useobject{currentmarker}{}%
\end{pgfscope}%
\begin{pgfscope}%
\pgfsys@transformshift{0.779748in}{1.532984in}%
\pgfsys@useobject{currentmarker}{}%
\end{pgfscope}%
\begin{pgfscope}%
\pgfsys@transformshift{0.779748in}{1.678429in}%
\pgfsys@useobject{currentmarker}{}%
\end{pgfscope}%
\begin{pgfscope}%
\pgfsys@transformshift{0.983784in}{1.075355in}%
\pgfsys@useobject{currentmarker}{}%
\end{pgfscope}%
\begin{pgfscope}%
\pgfsys@transformshift{2.425932in}{1.464212in}%
\pgfsys@useobject{currentmarker}{}%
\end{pgfscope}%
\begin{pgfscope}%
\pgfsys@transformshift{4.935623in}{1.259090in}%
\pgfsys@useobject{currentmarker}{}%
\end{pgfscope}%
\begin{pgfscope}%
\pgfsys@transformshift{7.362969in}{1.644827in}%
\pgfsys@useobject{currentmarker}{}%
\end{pgfscope}%
\begin{pgfscope}%
\pgfsys@transformshift{0.983784in}{1.797732in}%
\pgfsys@useobject{currentmarker}{}%
\end{pgfscope}%
\begin{pgfscope}%
\pgfsys@transformshift{5.764468in}{1.159341in}%
\pgfsys@useobject{currentmarker}{}%
\end{pgfscope}%
\begin{pgfscope}%
\pgfsys@transformshift{8.270775in}{1.250136in}%
\pgfsys@useobject{currentmarker}{}%
\end{pgfscope}%
\begin{pgfscope}%
\pgfsys@transformshift{15.685790in}{1.321693in}%
\pgfsys@useobject{currentmarker}{}%
\end{pgfscope}%
\begin{pgfscope}%
\pgfsys@transformshift{8.832165in}{2.361818in}%
\pgfsys@useobject{currentmarker}{}%
\end{pgfscope}%
\begin{pgfscope}%
\pgfsys@transformshift{1.608383in}{1.081581in}%
\pgfsys@useobject{currentmarker}{}%
\end{pgfscope}%
\begin{pgfscope}%
\pgfsys@transformshift{13.227595in}{1.344104in}%
\pgfsys@useobject{currentmarker}{}%
\end{pgfscope}%
\begin{pgfscope}%
\pgfsys@transformshift{5.766924in}{1.152168in}%
\pgfsys@useobject{currentmarker}{}%
\end{pgfscope}%
\begin{pgfscope}%
\pgfsys@transformshift{4.035966in}{1.603885in}%
\pgfsys@useobject{currentmarker}{}%
\end{pgfscope}%
\begin{pgfscope}%
\pgfsys@transformshift{6.605177in}{1.217187in}%
\pgfsys@useobject{currentmarker}{}%
\end{pgfscope}%
\begin{pgfscope}%
\pgfsys@transformshift{1.191984in}{1.112391in}%
\pgfsys@useobject{currentmarker}{}%
\end{pgfscope}%
\begin{pgfscope}%
\pgfsys@transformshift{0.779748in}{1.233461in}%
\pgfsys@useobject{currentmarker}{}%
\end{pgfscope}%
\begin{pgfscope}%
\pgfsys@transformshift{1.191984in}{1.062224in}%
\pgfsys@useobject{currentmarker}{}%
\end{pgfscope}%
\begin{pgfscope}%
\pgfsys@transformshift{12.365368in}{1.371727in}%
\pgfsys@useobject{currentmarker}{}%
\end{pgfscope}%
\begin{pgfscope}%
\pgfsys@transformshift{3.152693in}{2.293489in}%
\pgfsys@useobject{currentmarker}{}%
\end{pgfscope}%
\begin{pgfscope}%
\pgfsys@transformshift{0.879684in}{0.994399in}%
\pgfsys@useobject{currentmarker}{}%
\end{pgfscope}%
\begin{pgfscope}%
\pgfsys@transformshift{1.607743in}{1.135707in}%
\pgfsys@useobject{currentmarker}{}%
\end{pgfscope}%
\begin{pgfscope}%
\pgfsys@transformshift{3.270281in}{1.177616in}%
\pgfsys@useobject{currentmarker}{}%
\end{pgfscope}%
\begin{pgfscope}%
\pgfsys@transformshift{14.879801in}{1.350424in}%
\pgfsys@useobject{currentmarker}{}%
\end{pgfscope}%
\begin{pgfscope}%
\pgfsys@transformshift{7.431008in}{1.210387in}%
\pgfsys@useobject{currentmarker}{}%
\end{pgfscope}%
\begin{pgfscope}%
\pgfsys@transformshift{3.273943in}{1.118019in}%
\pgfsys@useobject{currentmarker}{}%
\end{pgfscope}%
\begin{pgfscope}%
\pgfsys@transformshift{17.319627in}{1.515969in}%
\pgfsys@useobject{currentmarker}{}%
\end{pgfscope}%
\begin{pgfscope}%
\pgfsys@transformshift{4.943739in}{2.707497in}%
\pgfsys@useobject{currentmarker}{}%
\end{pgfscope}%
\begin{pgfscope}%
\pgfsys@transformshift{9.936373in}{1.307208in}%
\pgfsys@useobject{currentmarker}{}%
\end{pgfscope}%
\begin{pgfscope}%
\pgfsys@transformshift{0.879684in}{1.057447in}%
\pgfsys@useobject{currentmarker}{}%
\end{pgfscope}%
\begin{pgfscope}%
\pgfsys@transformshift{0.779748in}{1.068025in}%
\pgfsys@useobject{currentmarker}{}%
\end{pgfscope}%
\begin{pgfscope}%
\pgfsys@transformshift{4.927631in}{1.194306in}%
\pgfsys@useobject{currentmarker}{}%
\end{pgfscope}%
\begin{pgfscope}%
\pgfsys@transformshift{4.911521in}{2.883437in}%
\pgfsys@useobject{currentmarker}{}%
\end{pgfscope}%
\begin{pgfscope}%
\pgfsys@transformshift{1.191984in}{1.743290in}%
\pgfsys@useobject{currentmarker}{}%
\end{pgfscope}%
\begin{pgfscope}%
\pgfsys@transformshift{1.608383in}{1.073342in}%
\pgfsys@useobject{currentmarker}{}%
\end{pgfscope}%
\begin{pgfscope}%
\pgfsys@transformshift{3.146282in}{2.387726in}%
\pgfsys@useobject{currentmarker}{}%
\end{pgfscope}%
\begin{pgfscope}%
\pgfsys@transformshift{1.191984in}{1.035362in}%
\pgfsys@useobject{currentmarker}{}%
\end{pgfscope}%
\begin{pgfscope}%
\pgfsys@transformshift{1.191984in}{1.320985in}%
\pgfsys@useobject{currentmarker}{}%
\end{pgfscope}%
\begin{pgfscope}%
\pgfsys@transformshift{2.441182in}{1.109649in}%
\pgfsys@useobject{currentmarker}{}%
\end{pgfscope}%
\end{pgfscope}%
\begin{pgfscope}%
\pgfsetbuttcap%
\pgfsetroundjoin%
\definecolor{currentfill}{rgb}{0.000000,0.000000,0.000000}%
\pgfsetfillcolor{currentfill}%
\pgfsetlinewidth{0.803000pt}%
\definecolor{currentstroke}{rgb}{0.000000,0.000000,0.000000}%
\pgfsetstrokecolor{currentstroke}%
\pgfsetdash{}{0pt}%
\pgfsys@defobject{currentmarker}{\pgfqpoint{0.000000in}{-0.048611in}}{\pgfqpoint{0.000000in}{0.000000in}}{%
\pgfpathmoveto{\pgfqpoint{0.000000in}{0.000000in}}%
\pgfpathlineto{\pgfqpoint{0.000000in}{-0.048611in}}%
\pgfusepath{stroke,fill}%
}%
\begin{pgfscope}%
\pgfsys@transformshift{0.775584in}{0.582778in}%
\pgfsys@useobject{currentmarker}{}%
\end{pgfscope}%
\end{pgfscope}%
\begin{pgfscope}%
\definecolor{textcolor}{rgb}{0.000000,0.000000,0.000000}%
\pgfsetstrokecolor{textcolor}%
\pgfsetfillcolor{textcolor}%
\pgftext[x=0.775584in,y=0.485556in,,top]{\color{textcolor}\sffamily\fontsize{10.000000}{12.000000}\selectfont 0}%
\end{pgfscope}%
\begin{pgfscope}%
\pgfsetbuttcap%
\pgfsetroundjoin%
\definecolor{currentfill}{rgb}{0.000000,0.000000,0.000000}%
\pgfsetfillcolor{currentfill}%
\pgfsetlinewidth{0.803000pt}%
\definecolor{currentstroke}{rgb}{0.000000,0.000000,0.000000}%
\pgfsetstrokecolor{currentstroke}%
\pgfsetdash{}{0pt}%
\pgfsys@defobject{currentmarker}{\pgfqpoint{0.000000in}{-0.048611in}}{\pgfqpoint{0.000000in}{0.000000in}}{%
\pgfpathmoveto{\pgfqpoint{0.000000in}{0.000000in}}%
\pgfpathlineto{\pgfqpoint{0.000000in}{-0.048611in}}%
\pgfusepath{stroke,fill}%
}%
\begin{pgfscope}%
\pgfsys@transformshift{1.608383in}{0.582778in}%
\pgfsys@useobject{currentmarker}{}%
\end{pgfscope}%
\end{pgfscope}%
\begin{pgfscope}%
\definecolor{textcolor}{rgb}{0.000000,0.000000,0.000000}%
\pgfsetstrokecolor{textcolor}%
\pgfsetfillcolor{textcolor}%
\pgftext[x=1.608383in,y=0.485556in,,top]{\color{textcolor}\sffamily\fontsize{10.000000}{12.000000}\selectfont 200}%
\end{pgfscope}%
\begin{pgfscope}%
\pgfsetbuttcap%
\pgfsetroundjoin%
\definecolor{currentfill}{rgb}{0.000000,0.000000,0.000000}%
\pgfsetfillcolor{currentfill}%
\pgfsetlinewidth{0.803000pt}%
\definecolor{currentstroke}{rgb}{0.000000,0.000000,0.000000}%
\pgfsetstrokecolor{currentstroke}%
\pgfsetdash{}{0pt}%
\pgfsys@defobject{currentmarker}{\pgfqpoint{0.000000in}{-0.048611in}}{\pgfqpoint{0.000000in}{0.000000in}}{%
\pgfpathmoveto{\pgfqpoint{0.000000in}{0.000000in}}%
\pgfpathlineto{\pgfqpoint{0.000000in}{-0.048611in}}%
\pgfusepath{stroke,fill}%
}%
\begin{pgfscope}%
\pgfsys@transformshift{2.441182in}{0.582778in}%
\pgfsys@useobject{currentmarker}{}%
\end{pgfscope}%
\end{pgfscope}%
\begin{pgfscope}%
\definecolor{textcolor}{rgb}{0.000000,0.000000,0.000000}%
\pgfsetstrokecolor{textcolor}%
\pgfsetfillcolor{textcolor}%
\pgftext[x=2.441182in,y=0.485556in,,top]{\color{textcolor}\sffamily\fontsize{10.000000}{12.000000}\selectfont 400}%
\end{pgfscope}%
\begin{pgfscope}%
\pgfsetbuttcap%
\pgfsetroundjoin%
\definecolor{currentfill}{rgb}{0.000000,0.000000,0.000000}%
\pgfsetfillcolor{currentfill}%
\pgfsetlinewidth{0.803000pt}%
\definecolor{currentstroke}{rgb}{0.000000,0.000000,0.000000}%
\pgfsetstrokecolor{currentstroke}%
\pgfsetdash{}{0pt}%
\pgfsys@defobject{currentmarker}{\pgfqpoint{0.000000in}{-0.048611in}}{\pgfqpoint{0.000000in}{0.000000in}}{%
\pgfpathmoveto{\pgfqpoint{0.000000in}{0.000000in}}%
\pgfpathlineto{\pgfqpoint{0.000000in}{-0.048611in}}%
\pgfusepath{stroke,fill}%
}%
\begin{pgfscope}%
\pgfsys@transformshift{3.273981in}{0.582778in}%
\pgfsys@useobject{currentmarker}{}%
\end{pgfscope}%
\end{pgfscope}%
\begin{pgfscope}%
\definecolor{textcolor}{rgb}{0.000000,0.000000,0.000000}%
\pgfsetstrokecolor{textcolor}%
\pgfsetfillcolor{textcolor}%
\pgftext[x=3.273981in,y=0.485556in,,top]{\color{textcolor}\sffamily\fontsize{10.000000}{12.000000}\selectfont 600}%
\end{pgfscope}%
\begin{pgfscope}%
\pgfsetbuttcap%
\pgfsetroundjoin%
\definecolor{currentfill}{rgb}{0.000000,0.000000,0.000000}%
\pgfsetfillcolor{currentfill}%
\pgfsetlinewidth{0.803000pt}%
\definecolor{currentstroke}{rgb}{0.000000,0.000000,0.000000}%
\pgfsetstrokecolor{currentstroke}%
\pgfsetdash{}{0pt}%
\pgfsys@defobject{currentmarker}{\pgfqpoint{0.000000in}{-0.048611in}}{\pgfqpoint{0.000000in}{0.000000in}}{%
\pgfpathmoveto{\pgfqpoint{0.000000in}{0.000000in}}%
\pgfpathlineto{\pgfqpoint{0.000000in}{-0.048611in}}%
\pgfusepath{stroke,fill}%
}%
\begin{pgfscope}%
\pgfsys@transformshift{4.106780in}{0.582778in}%
\pgfsys@useobject{currentmarker}{}%
\end{pgfscope}%
\end{pgfscope}%
\begin{pgfscope}%
\definecolor{textcolor}{rgb}{0.000000,0.000000,0.000000}%
\pgfsetstrokecolor{textcolor}%
\pgfsetfillcolor{textcolor}%
\pgftext[x=4.106780in,y=0.485556in,,top]{\color{textcolor}\sffamily\fontsize{10.000000}{12.000000}\selectfont 800}%
\end{pgfscope}%
\begin{pgfscope}%
\pgfsetbuttcap%
\pgfsetroundjoin%
\definecolor{currentfill}{rgb}{0.000000,0.000000,0.000000}%
\pgfsetfillcolor{currentfill}%
\pgfsetlinewidth{0.803000pt}%
\definecolor{currentstroke}{rgb}{0.000000,0.000000,0.000000}%
\pgfsetstrokecolor{currentstroke}%
\pgfsetdash{}{0pt}%
\pgfsys@defobject{currentmarker}{\pgfqpoint{0.000000in}{-0.048611in}}{\pgfqpoint{0.000000in}{0.000000in}}{%
\pgfpathmoveto{\pgfqpoint{0.000000in}{0.000000in}}%
\pgfpathlineto{\pgfqpoint{0.000000in}{-0.048611in}}%
\pgfusepath{stroke,fill}%
}%
\begin{pgfscope}%
\pgfsys@transformshift{4.939579in}{0.582778in}%
\pgfsys@useobject{currentmarker}{}%
\end{pgfscope}%
\end{pgfscope}%
\begin{pgfscope}%
\definecolor{textcolor}{rgb}{0.000000,0.000000,0.000000}%
\pgfsetstrokecolor{textcolor}%
\pgfsetfillcolor{textcolor}%
\pgftext[x=4.939579in,y=0.485556in,,top]{\color{textcolor}\sffamily\fontsize{10.000000}{12.000000}\selectfont 1000}%
\end{pgfscope}%
\begin{pgfscope}%
\definecolor{textcolor}{rgb}{0.000000,0.000000,0.000000}%
\pgfsetstrokecolor{textcolor}%
\pgfsetfillcolor{textcolor}%
\pgftext[x=2.857582in,y=0.295587in,,top]{\color{textcolor}\sffamily\fontsize{10.000000}{12.000000}\selectfont goodput (req/s)}%
\end{pgfscope}%
\begin{pgfscope}%
\pgfsetbuttcap%
\pgfsetroundjoin%
\definecolor{currentfill}{rgb}{0.000000,0.000000,0.000000}%
\pgfsetfillcolor{currentfill}%
\pgfsetlinewidth{0.803000pt}%
\definecolor{currentstroke}{rgb}{0.000000,0.000000,0.000000}%
\pgfsetstrokecolor{currentstroke}%
\pgfsetdash{}{0pt}%
\pgfsys@defobject{currentmarker}{\pgfqpoint{-0.048611in}{0.000000in}}{\pgfqpoint{-0.000000in}{0.000000in}}{%
\pgfpathmoveto{\pgfqpoint{-0.000000in}{0.000000in}}%
\pgfpathlineto{\pgfqpoint{-0.048611in}{0.000000in}}%
\pgfusepath{stroke,fill}%
}%
\begin{pgfscope}%
\pgfsys@transformshift{0.775584in}{0.582778in}%
\pgfsys@useobject{currentmarker}{}%
\end{pgfscope}%
\end{pgfscope}%
\begin{pgfscope}%
\definecolor{textcolor}{rgb}{0.000000,0.000000,0.000000}%
\pgfsetstrokecolor{textcolor}%
\pgfsetfillcolor{textcolor}%
\pgftext[x=0.457483in, y=0.530016in, left, base]{\color{textcolor}\sffamily\fontsize{10.000000}{12.000000}\selectfont 0.0}%
\end{pgfscope}%
\begin{pgfscope}%
\pgfsetbuttcap%
\pgfsetroundjoin%
\definecolor{currentfill}{rgb}{0.000000,0.000000,0.000000}%
\pgfsetfillcolor{currentfill}%
\pgfsetlinewidth{0.803000pt}%
\definecolor{currentstroke}{rgb}{0.000000,0.000000,0.000000}%
\pgfsetstrokecolor{currentstroke}%
\pgfsetdash{}{0pt}%
\pgfsys@defobject{currentmarker}{\pgfqpoint{-0.048611in}{0.000000in}}{\pgfqpoint{-0.000000in}{0.000000in}}{%
\pgfpathmoveto{\pgfqpoint{-0.000000in}{0.000000in}}%
\pgfpathlineto{\pgfqpoint{-0.048611in}{0.000000in}}%
\pgfusepath{stroke,fill}%
}%
\begin{pgfscope}%
\pgfsys@transformshift{0.775584in}{1.436222in}%
\pgfsys@useobject{currentmarker}{}%
\end{pgfscope}%
\end{pgfscope}%
\begin{pgfscope}%
\definecolor{textcolor}{rgb}{0.000000,0.000000,0.000000}%
\pgfsetstrokecolor{textcolor}%
\pgfsetfillcolor{textcolor}%
\pgftext[x=0.457483in, y=1.383461in, left, base]{\color{textcolor}\sffamily\fontsize{10.000000}{12.000000}\selectfont 0.2}%
\end{pgfscope}%
\begin{pgfscope}%
\pgfsetbuttcap%
\pgfsetroundjoin%
\definecolor{currentfill}{rgb}{0.000000,0.000000,0.000000}%
\pgfsetfillcolor{currentfill}%
\pgfsetlinewidth{0.803000pt}%
\definecolor{currentstroke}{rgb}{0.000000,0.000000,0.000000}%
\pgfsetstrokecolor{currentstroke}%
\pgfsetdash{}{0pt}%
\pgfsys@defobject{currentmarker}{\pgfqpoint{-0.048611in}{0.000000in}}{\pgfqpoint{-0.000000in}{0.000000in}}{%
\pgfpathmoveto{\pgfqpoint{-0.000000in}{0.000000in}}%
\pgfpathlineto{\pgfqpoint{-0.048611in}{0.000000in}}%
\pgfusepath{stroke,fill}%
}%
\begin{pgfscope}%
\pgfsys@transformshift{0.775584in}{2.289667in}%
\pgfsys@useobject{currentmarker}{}%
\end{pgfscope}%
\end{pgfscope}%
\begin{pgfscope}%
\definecolor{textcolor}{rgb}{0.000000,0.000000,0.000000}%
\pgfsetstrokecolor{textcolor}%
\pgfsetfillcolor{textcolor}%
\pgftext[x=0.457483in, y=2.236905in, left, base]{\color{textcolor}\sffamily\fontsize{10.000000}{12.000000}\selectfont 0.4}%
\end{pgfscope}%
\begin{pgfscope}%
\pgfsetbuttcap%
\pgfsetroundjoin%
\definecolor{currentfill}{rgb}{0.000000,0.000000,0.000000}%
\pgfsetfillcolor{currentfill}%
\pgfsetlinewidth{0.803000pt}%
\definecolor{currentstroke}{rgb}{0.000000,0.000000,0.000000}%
\pgfsetstrokecolor{currentstroke}%
\pgfsetdash{}{0pt}%
\pgfsys@defobject{currentmarker}{\pgfqpoint{-0.048611in}{0.000000in}}{\pgfqpoint{-0.000000in}{0.000000in}}{%
\pgfpathmoveto{\pgfqpoint{-0.000000in}{0.000000in}}%
\pgfpathlineto{\pgfqpoint{-0.048611in}{0.000000in}}%
\pgfusepath{stroke,fill}%
}%
\begin{pgfscope}%
\pgfsys@transformshift{0.775584in}{3.143111in}%
\pgfsys@useobject{currentmarker}{}%
\end{pgfscope}%
\end{pgfscope}%
\begin{pgfscope}%
\definecolor{textcolor}{rgb}{0.000000,0.000000,0.000000}%
\pgfsetstrokecolor{textcolor}%
\pgfsetfillcolor{textcolor}%
\pgftext[x=0.457483in, y=3.090350in, left, base]{\color{textcolor}\sffamily\fontsize{10.000000}{12.000000}\selectfont 0.6}%
\end{pgfscope}%
\begin{pgfscope}%
\pgfsetbuttcap%
\pgfsetroundjoin%
\definecolor{currentfill}{rgb}{0.000000,0.000000,0.000000}%
\pgfsetfillcolor{currentfill}%
\pgfsetlinewidth{0.803000pt}%
\definecolor{currentstroke}{rgb}{0.000000,0.000000,0.000000}%
\pgfsetstrokecolor{currentstroke}%
\pgfsetdash{}{0pt}%
\pgfsys@defobject{currentmarker}{\pgfqpoint{-0.048611in}{0.000000in}}{\pgfqpoint{-0.000000in}{0.000000in}}{%
\pgfpathmoveto{\pgfqpoint{-0.000000in}{0.000000in}}%
\pgfpathlineto{\pgfqpoint{-0.048611in}{0.000000in}}%
\pgfusepath{stroke,fill}%
}%
\begin{pgfscope}%
\pgfsys@transformshift{0.775584in}{3.996556in}%
\pgfsys@useobject{currentmarker}{}%
\end{pgfscope}%
\end{pgfscope}%
\begin{pgfscope}%
\definecolor{textcolor}{rgb}{0.000000,0.000000,0.000000}%
\pgfsetstrokecolor{textcolor}%
\pgfsetfillcolor{textcolor}%
\pgftext[x=0.457483in, y=3.943794in, left, base]{\color{textcolor}\sffamily\fontsize{10.000000}{12.000000}\selectfont 0.8}%
\end{pgfscope}%
\begin{pgfscope}%
\pgfsetbuttcap%
\pgfsetroundjoin%
\definecolor{currentfill}{rgb}{0.000000,0.000000,0.000000}%
\pgfsetfillcolor{currentfill}%
\pgfsetlinewidth{0.803000pt}%
\definecolor{currentstroke}{rgb}{0.000000,0.000000,0.000000}%
\pgfsetstrokecolor{currentstroke}%
\pgfsetdash{}{0pt}%
\pgfsys@defobject{currentmarker}{\pgfqpoint{-0.048611in}{0.000000in}}{\pgfqpoint{-0.000000in}{0.000000in}}{%
\pgfpathmoveto{\pgfqpoint{-0.000000in}{0.000000in}}%
\pgfpathlineto{\pgfqpoint{-0.048611in}{0.000000in}}%
\pgfusepath{stroke,fill}%
}%
\begin{pgfscope}%
\pgfsys@transformshift{0.775584in}{4.850000in}%
\pgfsys@useobject{currentmarker}{}%
\end{pgfscope}%
\end{pgfscope}%
\begin{pgfscope}%
\definecolor{textcolor}{rgb}{0.000000,0.000000,0.000000}%
\pgfsetstrokecolor{textcolor}%
\pgfsetfillcolor{textcolor}%
\pgftext[x=0.457483in, y=4.797238in, left, base]{\color{textcolor}\sffamily\fontsize{10.000000}{12.000000}\selectfont 1.0}%
\end{pgfscope}%
\begin{pgfscope}%
\definecolor{textcolor}{rgb}{0.000000,0.000000,0.000000}%
\pgfsetstrokecolor{textcolor}%
\pgfsetfillcolor{textcolor}%
\pgftext[x=0.401927in,y=2.716389in,,bottom,rotate=90.000000]{\color{textcolor}\sffamily\fontsize{10.000000}{12.000000}\selectfont median latency (s)}%
\end{pgfscope}%
\begin{pgfscope}%
\pgfpathrectangle{\pgfqpoint{0.775584in}{0.582778in}}{\pgfqpoint{4.163995in}{4.267222in}}%
\pgfusepath{clip}%
\pgfsetrectcap%
\pgfsetroundjoin%
\pgfsetlinewidth{2.258437pt}%
\definecolor{currentstroke}{rgb}{0.003922,0.450980,0.698039}%
\pgfsetstrokecolor{currentstroke}%
\pgfsetdash{}{0pt}%
\pgfpathmoveto{\pgfqpoint{0.779748in}{1.276083in}}%
\pgfpathlineto{\pgfqpoint{0.947842in}{1.308062in}}%
\pgfpathlineto{\pgfqpoint{1.115935in}{1.338516in}}%
\pgfpathlineto{\pgfqpoint{1.284028in}{1.367472in}}%
\pgfpathlineto{\pgfqpoint{1.452121in}{1.394958in}}%
\pgfpathlineto{\pgfqpoint{1.620214in}{1.421005in}}%
\pgfpathlineto{\pgfqpoint{1.788307in}{1.445639in}}%
\pgfpathlineto{\pgfqpoint{1.956400in}{1.468890in}}%
\pgfpathlineto{\pgfqpoint{2.124493in}{1.490786in}}%
\pgfpathlineto{\pgfqpoint{2.292586in}{1.511357in}}%
\pgfpathlineto{\pgfqpoint{2.460679in}{1.530629in}}%
\pgfpathlineto{\pgfqpoint{2.628773in}{1.548633in}}%
\pgfpathlineto{\pgfqpoint{2.796866in}{1.565397in}}%
\pgfpathlineto{\pgfqpoint{2.964959in}{1.580949in}}%
\pgfpathlineto{\pgfqpoint{3.133052in}{1.595318in}}%
\pgfpathlineto{\pgfqpoint{3.301145in}{1.608533in}}%
\pgfpathlineto{\pgfqpoint{3.469238in}{1.620622in}}%
\pgfpathlineto{\pgfqpoint{3.637331in}{1.631613in}}%
\pgfpathlineto{\pgfqpoint{3.805424in}{1.641537in}}%
\pgfpathlineto{\pgfqpoint{3.973517in}{1.650420in}}%
\pgfpathlineto{\pgfqpoint{4.141610in}{1.658291in}}%
\pgfpathlineto{\pgfqpoint{4.309703in}{1.665180in}}%
\pgfpathlineto{\pgfqpoint{4.477797in}{1.671115in}}%
\pgfpathlineto{\pgfqpoint{4.645890in}{1.676124in}}%
\pgfpathlineto{\pgfqpoint{4.813983in}{1.680237in}}%
\pgfpathlineto{\pgfqpoint{4.949579in}{1.682853in}}%
\pgfusepath{stroke}%
\end{pgfscope}%
\begin{pgfscope}%
\pgfsetrectcap%
\pgfsetmiterjoin%
\pgfsetlinewidth{0.803000pt}%
\definecolor{currentstroke}{rgb}{0.000000,0.000000,0.000000}%
\pgfsetstrokecolor{currentstroke}%
\pgfsetdash{}{0pt}%
\pgfpathmoveto{\pgfqpoint{0.775584in}{0.582778in}}%
\pgfpathlineto{\pgfqpoint{0.775584in}{4.850000in}}%
\pgfusepath{stroke}%
\end{pgfscope}%
\begin{pgfscope}%
\pgfsetrectcap%
\pgfsetmiterjoin%
\pgfsetlinewidth{0.803000pt}%
\definecolor{currentstroke}{rgb}{0.000000,0.000000,0.000000}%
\pgfsetstrokecolor{currentstroke}%
\pgfsetdash{}{0pt}%
\pgfpathmoveto{\pgfqpoint{0.775584in}{0.582778in}}%
\pgfpathlineto{\pgfqpoint{4.939579in}{0.582778in}}%
\pgfusepath{stroke}%
\end{pgfscope}%
\begin{pgfscope}%
\pgfsetbuttcap%
\pgfsetmiterjoin%
\definecolor{currentfill}{rgb}{1.000000,1.000000,1.000000}%
\pgfsetfillcolor{currentfill}%
\pgfsetfillopacity{0.800000}%
\pgfsetlinewidth{1.003750pt}%
\definecolor{currentstroke}{rgb}{0.800000,0.800000,0.800000}%
\pgfsetstrokecolor{currentstroke}%
\pgfsetstrokeopacity{0.800000}%
\pgfsetdash{}{0pt}%
\pgfpathmoveto{\pgfqpoint{4.081235in}{4.331174in}}%
\pgfpathlineto{\pgfqpoint{4.842357in}{4.331174in}}%
\pgfpathquadraticcurveto{\pgfqpoint{4.870135in}{4.331174in}}{\pgfqpoint{4.870135in}{4.358952in}}%
\pgfpathlineto{\pgfqpoint{4.870135in}{4.752778in}}%
\pgfpathquadraticcurveto{\pgfqpoint{4.870135in}{4.780556in}}{\pgfqpoint{4.842357in}{4.780556in}}%
\pgfpathlineto{\pgfqpoint{4.081235in}{4.780556in}}%
\pgfpathquadraticcurveto{\pgfqpoint{4.053457in}{4.780556in}}{\pgfqpoint{4.053457in}{4.752778in}}%
\pgfpathlineto{\pgfqpoint{4.053457in}{4.358952in}}%
\pgfpathquadraticcurveto{\pgfqpoint{4.053457in}{4.331174in}}{\pgfqpoint{4.081235in}{4.331174in}}%
\pgfpathlineto{\pgfqpoint{4.081235in}{4.331174in}}%
\pgfpathclose%
\pgfusepath{stroke,fill}%
\end{pgfscope}%
\begin{pgfscope}%
\definecolor{textcolor}{rgb}{0.000000,0.000000,0.000000}%
\pgfsetstrokecolor{textcolor}%
\pgfsetfillcolor{textcolor}%
\pgftext[x=4.109013in,y=4.619477in,left,base]{\color{textcolor}\sffamily\fontsize{10.000000}{12.000000}\selectfont batch size}%
\end{pgfscope}%
\begin{pgfscope}%
\pgfsetbuttcap%
\pgfsetroundjoin%
\definecolor{currentfill}{rgb}{0.003922,0.450980,0.698039}%
\pgfsetfillcolor{currentfill}%
\pgfsetfillopacity{0.800000}%
\pgfsetlinewidth{1.003750pt}%
\definecolor{currentstroke}{rgb}{0.003922,0.450980,0.698039}%
\pgfsetstrokecolor{currentstroke}%
\pgfsetstrokeopacity{0.800000}%
\pgfsetdash{}{0pt}%
\pgfsys@defobject{currentmarker}{\pgfqpoint{-0.041667in}{-0.041667in}}{\pgfqpoint{0.041667in}{0.041667in}}{%
\pgfpathmoveto{\pgfqpoint{0.000000in}{-0.041667in}}%
\pgfpathcurveto{\pgfqpoint{0.011050in}{-0.041667in}}{\pgfqpoint{0.021649in}{-0.037276in}}{\pgfqpoint{0.029463in}{-0.029463in}}%
\pgfpathcurveto{\pgfqpoint{0.037276in}{-0.021649in}}{\pgfqpoint{0.041667in}{-0.011050in}}{\pgfqpoint{0.041667in}{0.000000in}}%
\pgfpathcurveto{\pgfqpoint{0.041667in}{0.011050in}}{\pgfqpoint{0.037276in}{0.021649in}}{\pgfqpoint{0.029463in}{0.029463in}}%
\pgfpathcurveto{\pgfqpoint{0.021649in}{0.037276in}}{\pgfqpoint{0.011050in}{0.041667in}}{\pgfqpoint{0.000000in}{0.041667in}}%
\pgfpathcurveto{\pgfqpoint{-0.011050in}{0.041667in}}{\pgfqpoint{-0.021649in}{0.037276in}}{\pgfqpoint{-0.029463in}{0.029463in}}%
\pgfpathcurveto{\pgfqpoint{-0.037276in}{0.021649in}}{\pgfqpoint{-0.041667in}{0.011050in}}{\pgfqpoint{-0.041667in}{0.000000in}}%
\pgfpathcurveto{\pgfqpoint{-0.041667in}{-0.011050in}}{\pgfqpoint{-0.037276in}{-0.021649in}}{\pgfqpoint{-0.029463in}{-0.029463in}}%
\pgfpathcurveto{\pgfqpoint{-0.021649in}{-0.037276in}}{\pgfqpoint{-0.011050in}{-0.041667in}}{\pgfqpoint{0.000000in}{-0.041667in}}%
\pgfpathlineto{\pgfqpoint{0.000000in}{-0.041667in}}%
\pgfpathclose%
\pgfusepath{stroke,fill}%
}%
\begin{pgfscope}%
\pgfsys@transformshift{4.273693in}{4.452078in}%
\pgfsys@useobject{currentmarker}{}%
\end{pgfscope}%
\end{pgfscope}%
\begin{pgfscope}%
\definecolor{textcolor}{rgb}{0.000000,0.000000,0.000000}%
\pgfsetstrokecolor{textcolor}%
\pgfsetfillcolor{textcolor}%
\pgftext[x=4.523693in,y=4.415620in,left,base]{\color{textcolor}\sffamily\fontsize{10.000000}{12.000000}\selectfont 300}%
\end{pgfscope}%
\end{pgfpicture}%
\makeatother%
\endgroup%
}
\caption{Benchmarking of goodput and median latency while varying throughputs and batch sizes.}
\label{goodputlatencybatch}
\end{figure}

This study compares the performance of the system for varying limits on batch sizes (Section~\ref{batchsizes}). Experiments were run for 20 seconds on a network of 4 nodes. The experiment was run for longer to reduce the higher variance observed in tests where the system is overloaded. Figure~\ref{goodputlatencybatch} omits results with latency of above 1s, to show the performance of the system as it begins to be overloaded.

At lower throughputs the system is not overloaded; throughput grows linearly with goodput (Figure~\ref{throughputgoodputbatch}), as the system can respond to all incoming requests with roughly constant latency throughout an experiment (Figure~\ref{heatmaps}(a)). During this period batches are not filled, so larger throughputs result in larger messages and a slow increase in latency due to increasing serialisation latency (Figure~\ref{goodputlatencybatch}). The system can reach a higher goodput before being overloaded if it has a larger batch size, as each view results in more commands being committed; this supports the conclusion that batching is an effective optimisation.

Once throughput is increased enough, batches begin to be filled up and the system is overloaded. This results in the goodput flattening out (Figure~\ref{throughputgoodputbatch}), as the system cannot handle the volume of requests; commands begin to queue on the nodes and latency rapidly increases throughout an experiment (Figure~\ref{heatmaps}(b), Figure~\ref{goodputlatencybatch}). There is a slight decrease from the peak goodput due to the overheads of queueing.

For higher batch size limits (especially unlimited), the goodput declines more significantly once the system is overloaded. This is because the benefits of larger batches are offset by messages becoming larger, causing increased serialisation latency, which increases view times and lower goodput. For large batch sizes, view times increase exponentially, as shown by the growing vertical gaps between commands being committed in Figure~\ref{heatmaps}(c).

There is a clear trade-off between larger batch sizes that result in more commands being committed, and batches becoming too large and incurring exponential serialisation latency. The optimum for the system appears to be a batch size of around 600 commands, with a maximum goodput of around 900req/s (Figure~\ref{throughputgoodputbatch}).

% There is a trade-off between having larger batches to process more commands, and messages becoming too large and increasing serialisation latency, which is apparent from Figure~\ref{throughputgoodputbatch}. With small batch sizes each view only commits a small number of commands, leading to low goodputs; an extreme example of this is a batch size of 1 (equivalent to no batching). As batch sizes increase the goodput reached also increases, with goodput peaking at around 700req/s with a batch size of 300 commands. At this point increasing batch size further causes goodput to decrease due to the increased latency of serialising large messages; each view takes longer so less nodes are committed per second (even though each node contains more commands). When the batch size is unlimited messages grow very large as throughput increases, and increased serialisation latency causes goodput to decline.

% Figure~\ref{throughputlatencybatch} shows that latency scales linearly with throughput while the system is not overloaded, that is, when the goodput is within 5\% of the target throughput. This is because larger throughputs result in larger message sizes, and increased latency due to serialisation time. 

\subsection{Node counts} \label{nodecountseval}

\begin{figure}[h!]
\centering
\resizebox{.6\textwidth}{!}{%% Creator: Matplotlib, PGF backend
%%
%% To include the figure in your LaTeX document, write
%%   \input{<filename>.pgf}
%%
%% Make sure the required packages are loaded in your preamble
%%   \usepackage{pgf}
%%
%% Also ensure that all the required font packages are loaded; for instance,
%% the lmodern package is sometimes necessary when using math font.
%%   \usepackage{lmodern}
%%
%% Figures using additional raster images can only be included by \input if
%% they are in the same directory as the main LaTeX file. For loading figures
%% from other directories you can use the `import` package
%%   \usepackage{import}
%%
%% and then include the figures with
%%   \import{<path to file>}{<filename>.pgf}
%%
%% Matplotlib used the following preamble
%%   
%%   \usepackage{fontspec}
%%   \setmainfont{DejaVuSerif.ttf}[Path=\detokenize{/opt/homebrew/lib/python3.10/site-packages/matplotlib/mpl-data/fonts/ttf/}]
%%   \setsansfont{DejaVuSans.ttf}[Path=\detokenize{/opt/homebrew/lib/python3.10/site-packages/matplotlib/mpl-data/fonts/ttf/}]
%%   \setmonofont{DejaVuSansMono.ttf}[Path=\detokenize{/opt/homebrew/lib/python3.10/site-packages/matplotlib/mpl-data/fonts/ttf/}]
%%   \makeatletter\@ifpackageloaded{underscore}{}{\usepackage[strings]{underscore}}\makeatother
%%
\begingroup%
\makeatletter%
\begin{pgfpicture}%
\pgfpathrectangle{\pgfpointorigin}{\pgfqpoint{5.840000in}{5.000000in}}%
\pgfusepath{use as bounding box, clip}%
\begin{pgfscope}%
\pgfsetbuttcap%
\pgfsetmiterjoin%
\definecolor{currentfill}{rgb}{1.000000,1.000000,1.000000}%
\pgfsetfillcolor{currentfill}%
\pgfsetlinewidth{0.000000pt}%
\definecolor{currentstroke}{rgb}{1.000000,1.000000,1.000000}%
\pgfsetstrokecolor{currentstroke}%
\pgfsetdash{}{0pt}%
\pgfpathmoveto{\pgfqpoint{0.000000in}{0.000000in}}%
\pgfpathlineto{\pgfqpoint{5.840000in}{0.000000in}}%
\pgfpathlineto{\pgfqpoint{5.840000in}{5.000000in}}%
\pgfpathlineto{\pgfqpoint{0.000000in}{5.000000in}}%
\pgfpathlineto{\pgfqpoint{0.000000in}{0.000000in}}%
\pgfpathclose%
\pgfusepath{fill}%
\end{pgfscope}%
\begin{pgfscope}%
\pgfsetbuttcap%
\pgfsetmiterjoin%
\definecolor{currentfill}{rgb}{1.000000,1.000000,1.000000}%
\pgfsetfillcolor{currentfill}%
\pgfsetlinewidth{0.000000pt}%
\definecolor{currentstroke}{rgb}{0.000000,0.000000,0.000000}%
\pgfsetstrokecolor{currentstroke}%
\pgfsetstrokeopacity{0.000000}%
\pgfsetdash{}{0pt}%
\pgfpathmoveto{\pgfqpoint{0.776314in}{0.582778in}}%
\pgfpathlineto{\pgfqpoint{4.939579in}{0.582778in}}%
\pgfpathlineto{\pgfqpoint{4.939579in}{4.850000in}}%
\pgfpathlineto{\pgfqpoint{0.776314in}{4.850000in}}%
\pgfpathlineto{\pgfqpoint{0.776314in}{0.582778in}}%
\pgfpathclose%
\pgfusepath{fill}%
\end{pgfscope}%
\begin{pgfscope}%
\pgfpathrectangle{\pgfqpoint{0.776314in}{0.582778in}}{\pgfqpoint{4.163265in}{4.267222in}}%
\pgfusepath{clip}%
\pgfsetbuttcap%
\pgfsetroundjoin%
\definecolor{currentfill}{rgb}{0.003922,0.450980,0.698039}%
\pgfsetfillcolor{currentfill}%
\pgfsetfillopacity{0.200000}%
\pgfsetlinewidth{1.003750pt}%
\definecolor{currentstroke}{rgb}{0.003922,0.450980,0.698039}%
\pgfsetstrokecolor{currentstroke}%
\pgfsetstrokeopacity{0.200000}%
\pgfsetdash{}{0pt}%
\pgfsys@defobject{currentmarker}{\pgfqpoint{0.777355in}{0.583845in}}{\pgfqpoint{4.939579in}{4.847285in}}{%
\pgfpathmoveto{\pgfqpoint{0.777355in}{0.583845in}}%
\pgfpathlineto{\pgfqpoint{0.777355in}{0.583845in}}%
\pgfpathlineto{\pgfqpoint{0.802335in}{0.609448in}}%
\pgfpathlineto{\pgfqpoint{0.828355in}{0.636118in}}%
\pgfpathlineto{\pgfqpoint{0.880396in}{0.689285in}}%
\pgfpathlineto{\pgfqpoint{0.984478in}{0.796017in}}%
\pgfpathlineto{\pgfqpoint{1.192641in}{1.007845in}}%
\pgfpathlineto{\pgfqpoint{1.400804in}{1.220595in}}%
\pgfpathlineto{\pgfqpoint{1.608967in}{1.435263in}}%
\pgfpathlineto{\pgfqpoint{1.817131in}{1.646522in}}%
\pgfpathlineto{\pgfqpoint{2.025294in}{1.860821in}}%
\pgfpathlineto{\pgfqpoint{2.233457in}{2.072155in}}%
\pgfpathlineto{\pgfqpoint{2.441620in}{2.280911in}}%
\pgfpathlineto{\pgfqpoint{2.649784in}{2.502453in}}%
\pgfpathlineto{\pgfqpoint{2.857947in}{2.705141in}}%
\pgfpathlineto{\pgfqpoint{3.066110in}{2.917738in}}%
\pgfpathlineto{\pgfqpoint{3.274273in}{3.136399in}}%
\pgfpathlineto{\pgfqpoint{3.482437in}{3.346771in}}%
\pgfpathlineto{\pgfqpoint{3.690600in}{3.551750in}}%
\pgfpathlineto{\pgfqpoint{3.898763in}{3.754843in}}%
\pgfpathlineto{\pgfqpoint{4.106926in}{3.959256in}}%
\pgfpathlineto{\pgfqpoint{4.315090in}{4.196244in}}%
\pgfpathlineto{\pgfqpoint{4.523253in}{4.402737in}}%
\pgfpathlineto{\pgfqpoint{4.731416in}{4.599384in}}%
\pgfpathlineto{\pgfqpoint{4.939579in}{4.821322in}}%
\pgfpathlineto{\pgfqpoint{4.939579in}{4.847285in}}%
\pgfpathlineto{\pgfqpoint{4.939579in}{4.847285in}}%
\pgfpathlineto{\pgfqpoint{4.731416in}{4.636639in}}%
\pgfpathlineto{\pgfqpoint{4.523253in}{4.420423in}}%
\pgfpathlineto{\pgfqpoint{4.315090in}{4.200886in}}%
\pgfpathlineto{\pgfqpoint{4.106926in}{3.989454in}}%
\pgfpathlineto{\pgfqpoint{3.898763in}{3.772953in}}%
\pgfpathlineto{\pgfqpoint{3.690600in}{3.565066in}}%
\pgfpathlineto{\pgfqpoint{3.482437in}{3.356472in}}%
\pgfpathlineto{\pgfqpoint{3.274273in}{3.143111in}}%
\pgfpathlineto{\pgfqpoint{3.066110in}{2.929750in}}%
\pgfpathlineto{\pgfqpoint{2.857947in}{2.716389in}}%
\pgfpathlineto{\pgfqpoint{2.649784in}{2.503028in}}%
\pgfpathlineto{\pgfqpoint{2.441620in}{2.287882in}}%
\pgfpathlineto{\pgfqpoint{2.233457in}{2.076306in}}%
\pgfpathlineto{\pgfqpoint{2.025294in}{1.861547in}}%
\pgfpathlineto{\pgfqpoint{1.817131in}{1.647636in}}%
\pgfpathlineto{\pgfqpoint{1.608967in}{1.436222in}}%
\pgfpathlineto{\pgfqpoint{1.400804in}{1.222861in}}%
\pgfpathlineto{\pgfqpoint{1.192641in}{1.009500in}}%
\pgfpathlineto{\pgfqpoint{0.984478in}{0.796139in}}%
\pgfpathlineto{\pgfqpoint{0.880396in}{0.689458in}}%
\pgfpathlineto{\pgfqpoint{0.828355in}{0.636118in}}%
\pgfpathlineto{\pgfqpoint{0.802335in}{0.609448in}}%
\pgfpathlineto{\pgfqpoint{0.777355in}{0.583845in}}%
\pgfpathlineto{\pgfqpoint{0.777355in}{0.583845in}}%
\pgfpathclose%
\pgfusepath{stroke,fill}%
}%
\begin{pgfscope}%
\pgfsys@transformshift{0.000000in}{0.000000in}%
\pgfsys@useobject{currentmarker}{}%
\end{pgfscope}%
\end{pgfscope}%
\begin{pgfscope}%
\pgfpathrectangle{\pgfqpoint{0.776314in}{0.582778in}}{\pgfqpoint{4.163265in}{4.267222in}}%
\pgfusepath{clip}%
\pgfsetbuttcap%
\pgfsetroundjoin%
\definecolor{currentfill}{rgb}{0.870588,0.560784,0.019608}%
\pgfsetfillcolor{currentfill}%
\pgfsetfillopacity{0.200000}%
\pgfsetlinewidth{1.003750pt}%
\definecolor{currentstroke}{rgb}{0.870588,0.560784,0.019608}%
\pgfsetstrokecolor{currentstroke}%
\pgfsetstrokeopacity{0.200000}%
\pgfsetdash{}{0pt}%
\pgfsys@defobject{currentmarker}{\pgfqpoint{0.777355in}{0.583845in}}{\pgfqpoint{4.939579in}{2.646855in}}{%
\pgfpathmoveto{\pgfqpoint{0.777355in}{0.583845in}}%
\pgfpathlineto{\pgfqpoint{0.777355in}{0.583845in}}%
\pgfpathlineto{\pgfqpoint{0.802335in}{0.609419in}}%
\pgfpathlineto{\pgfqpoint{0.828355in}{0.636118in}}%
\pgfpathlineto{\pgfqpoint{0.880396in}{0.689406in}}%
\pgfpathlineto{\pgfqpoint{0.984478in}{0.796139in}}%
\pgfpathlineto{\pgfqpoint{1.192641in}{1.009500in}}%
\pgfpathlineto{\pgfqpoint{1.400804in}{1.221913in}}%
\pgfpathlineto{\pgfqpoint{1.608967in}{1.435370in}}%
\pgfpathlineto{\pgfqpoint{1.817131in}{1.643431in}}%
\pgfpathlineto{\pgfqpoint{2.025294in}{1.848823in}}%
\pgfpathlineto{\pgfqpoint{2.233457in}{2.062053in}}%
\pgfpathlineto{\pgfqpoint{2.441620in}{2.270450in}}%
\pgfpathlineto{\pgfqpoint{2.649784in}{2.454465in}}%
\pgfpathlineto{\pgfqpoint{2.857947in}{2.476940in}}%
\pgfpathlineto{\pgfqpoint{3.066110in}{2.192647in}}%
\pgfpathlineto{\pgfqpoint{3.274273in}{2.344683in}}%
\pgfpathlineto{\pgfqpoint{3.482437in}{2.310442in}}%
\pgfpathlineto{\pgfqpoint{3.690600in}{2.315469in}}%
\pgfpathlineto{\pgfqpoint{3.898763in}{2.144824in}}%
\pgfpathlineto{\pgfqpoint{4.106926in}{2.197374in}}%
\pgfpathlineto{\pgfqpoint{4.315090in}{2.203637in}}%
\pgfpathlineto{\pgfqpoint{4.523253in}{2.096988in}}%
\pgfpathlineto{\pgfqpoint{4.731416in}{2.196261in}}%
\pgfpathlineto{\pgfqpoint{4.939579in}{2.088148in}}%
\pgfpathlineto{\pgfqpoint{4.939579in}{2.308405in}}%
\pgfpathlineto{\pgfqpoint{4.939579in}{2.308405in}}%
\pgfpathlineto{\pgfqpoint{4.731416in}{2.259190in}}%
\pgfpathlineto{\pgfqpoint{4.523253in}{2.253129in}}%
\pgfpathlineto{\pgfqpoint{4.315090in}{2.272140in}}%
\pgfpathlineto{\pgfqpoint{4.106926in}{2.347828in}}%
\pgfpathlineto{\pgfqpoint{3.898763in}{2.347474in}}%
\pgfpathlineto{\pgfqpoint{3.690600in}{2.425968in}}%
\pgfpathlineto{\pgfqpoint{3.482437in}{2.470849in}}%
\pgfpathlineto{\pgfqpoint{3.274273in}{2.461408in}}%
\pgfpathlineto{\pgfqpoint{3.066110in}{2.646855in}}%
\pgfpathlineto{\pgfqpoint{2.857947in}{2.611142in}}%
\pgfpathlineto{\pgfqpoint{2.649784in}{2.482632in}}%
\pgfpathlineto{\pgfqpoint{2.441620in}{2.289667in}}%
\pgfpathlineto{\pgfqpoint{2.233457in}{2.076306in}}%
\pgfpathlineto{\pgfqpoint{2.025294in}{1.850793in}}%
\pgfpathlineto{\pgfqpoint{1.817131in}{1.648570in}}%
\pgfpathlineto{\pgfqpoint{1.608967in}{1.436222in}}%
\pgfpathlineto{\pgfqpoint{1.400804in}{1.222861in}}%
\pgfpathlineto{\pgfqpoint{1.192641in}{1.009500in}}%
\pgfpathlineto{\pgfqpoint{0.984478in}{0.796139in}}%
\pgfpathlineto{\pgfqpoint{0.880396in}{0.689458in}}%
\pgfpathlineto{\pgfqpoint{0.828355in}{0.636118in}}%
\pgfpathlineto{\pgfqpoint{0.802335in}{0.609448in}}%
\pgfpathlineto{\pgfqpoint{0.777355in}{0.583845in}}%
\pgfpathlineto{\pgfqpoint{0.777355in}{0.583845in}}%
\pgfpathclose%
\pgfusepath{stroke,fill}%
}%
\begin{pgfscope}%
\pgfsys@transformshift{0.000000in}{0.000000in}%
\pgfsys@useobject{currentmarker}{}%
\end{pgfscope}%
\end{pgfscope}%
\begin{pgfscope}%
\pgfpathrectangle{\pgfqpoint{0.776314in}{0.582778in}}{\pgfqpoint{4.163265in}{4.267222in}}%
\pgfusepath{clip}%
\pgfsetbuttcap%
\pgfsetroundjoin%
\definecolor{currentfill}{rgb}{0.007843,0.619608,0.450980}%
\pgfsetfillcolor{currentfill}%
\pgfsetfillopacity{0.200000}%
\pgfsetlinewidth{1.003750pt}%
\definecolor{currentstroke}{rgb}{0.007843,0.619608,0.450980}%
\pgfsetstrokecolor{currentstroke}%
\pgfsetstrokeopacity{0.200000}%
\pgfsetdash{}{0pt}%
\pgfsys@defobject{currentmarker}{\pgfqpoint{0.777355in}{0.583845in}}{\pgfqpoint{4.939579in}{1.650649in}}{%
\pgfpathmoveto{\pgfqpoint{0.777355in}{0.583845in}}%
\pgfpathlineto{\pgfqpoint{0.777355in}{0.583845in}}%
\pgfpathlineto{\pgfqpoint{0.802335in}{0.609448in}}%
\pgfpathlineto{\pgfqpoint{0.828355in}{0.636118in}}%
\pgfpathlineto{\pgfqpoint{0.880396in}{0.689092in}}%
\pgfpathlineto{\pgfqpoint{0.984478in}{0.795001in}}%
\pgfpathlineto{\pgfqpoint{1.192641in}{1.004840in}}%
\pgfpathlineto{\pgfqpoint{1.400804in}{1.212026in}}%
\pgfpathlineto{\pgfqpoint{1.608967in}{1.405334in}}%
\pgfpathlineto{\pgfqpoint{1.817131in}{1.560525in}}%
\pgfpathlineto{\pgfqpoint{2.025294in}{1.586986in}}%
\pgfpathlineto{\pgfqpoint{2.233457in}{1.518539in}}%
\pgfpathlineto{\pgfqpoint{2.441620in}{1.446476in}}%
\pgfpathlineto{\pgfqpoint{2.649784in}{1.384611in}}%
\pgfpathlineto{\pgfqpoint{2.857947in}{1.359731in}}%
\pgfpathlineto{\pgfqpoint{3.066110in}{1.340767in}}%
\pgfpathlineto{\pgfqpoint{3.274273in}{1.352190in}}%
\pgfpathlineto{\pgfqpoint{3.482437in}{1.338596in}}%
\pgfpathlineto{\pgfqpoint{3.690600in}{1.350507in}}%
\pgfpathlineto{\pgfqpoint{3.898763in}{1.306662in}}%
\pgfpathlineto{\pgfqpoint{4.106926in}{1.316498in}}%
\pgfpathlineto{\pgfqpoint{4.315090in}{1.330574in}}%
\pgfpathlineto{\pgfqpoint{4.523253in}{1.314688in}}%
\pgfpathlineto{\pgfqpoint{4.731416in}{1.317627in}}%
\pgfpathlineto{\pgfqpoint{4.939579in}{1.292674in}}%
\pgfpathlineto{\pgfqpoint{4.939579in}{1.329192in}}%
\pgfpathlineto{\pgfqpoint{4.939579in}{1.329192in}}%
\pgfpathlineto{\pgfqpoint{4.731416in}{1.348511in}}%
\pgfpathlineto{\pgfqpoint{4.523253in}{1.343435in}}%
\pgfpathlineto{\pgfqpoint{4.315090in}{1.350032in}}%
\pgfpathlineto{\pgfqpoint{4.106926in}{1.331857in}}%
\pgfpathlineto{\pgfqpoint{3.898763in}{1.333252in}}%
\pgfpathlineto{\pgfqpoint{3.690600in}{1.355789in}}%
\pgfpathlineto{\pgfqpoint{3.482437in}{1.345611in}}%
\pgfpathlineto{\pgfqpoint{3.274273in}{1.363224in}}%
\pgfpathlineto{\pgfqpoint{3.066110in}{1.368426in}}%
\pgfpathlineto{\pgfqpoint{2.857947in}{1.390592in}}%
\pgfpathlineto{\pgfqpoint{2.649784in}{1.404295in}}%
\pgfpathlineto{\pgfqpoint{2.441620in}{1.509113in}}%
\pgfpathlineto{\pgfqpoint{2.233457in}{1.541141in}}%
\pgfpathlineto{\pgfqpoint{2.025294in}{1.650649in}}%
\pgfpathlineto{\pgfqpoint{1.817131in}{1.617223in}}%
\pgfpathlineto{\pgfqpoint{1.608967in}{1.418080in}}%
\pgfpathlineto{\pgfqpoint{1.400804in}{1.219670in}}%
\pgfpathlineto{\pgfqpoint{1.192641in}{1.009500in}}%
\pgfpathlineto{\pgfqpoint{0.984478in}{0.795975in}}%
\pgfpathlineto{\pgfqpoint{0.880396in}{0.689458in}}%
\pgfpathlineto{\pgfqpoint{0.828355in}{0.636118in}}%
\pgfpathlineto{\pgfqpoint{0.802335in}{0.609448in}}%
\pgfpathlineto{\pgfqpoint{0.777355in}{0.583845in}}%
\pgfpathlineto{\pgfqpoint{0.777355in}{0.583845in}}%
\pgfpathclose%
\pgfusepath{stroke,fill}%
}%
\begin{pgfscope}%
\pgfsys@transformshift{0.000000in}{0.000000in}%
\pgfsys@useobject{currentmarker}{}%
\end{pgfscope}%
\end{pgfscope}%
\begin{pgfscope}%
\pgfpathrectangle{\pgfqpoint{0.776314in}{0.582778in}}{\pgfqpoint{4.163265in}{4.267222in}}%
\pgfusepath{clip}%
\pgfsetbuttcap%
\pgfsetroundjoin%
\definecolor{currentfill}{rgb}{0.835294,0.368627,0.000000}%
\pgfsetfillcolor{currentfill}%
\pgfsetfillopacity{0.200000}%
\pgfsetlinewidth{1.003750pt}%
\definecolor{currentstroke}{rgb}{0.835294,0.368627,0.000000}%
\pgfsetstrokecolor{currentstroke}%
\pgfsetstrokeopacity{0.200000}%
\pgfsetdash{}{0pt}%
\pgfsys@defobject{currentmarker}{\pgfqpoint{0.777355in}{0.583845in}}{\pgfqpoint{4.939579in}{1.239424in}}{%
\pgfpathmoveto{\pgfqpoint{0.777355in}{0.583845in}}%
\pgfpathlineto{\pgfqpoint{0.777355in}{0.583845in}}%
\pgfpathlineto{\pgfqpoint{0.802335in}{0.609448in}}%
\pgfpathlineto{\pgfqpoint{0.828355in}{0.636118in}}%
\pgfpathlineto{\pgfqpoint{0.880396in}{0.689448in}}%
\pgfpathlineto{\pgfqpoint{0.984478in}{0.795187in}}%
\pgfpathlineto{\pgfqpoint{1.192641in}{1.005593in}}%
\pgfpathlineto{\pgfqpoint{1.400804in}{1.190145in}}%
\pgfpathlineto{\pgfqpoint{1.608967in}{1.188582in}}%
\pgfpathlineto{\pgfqpoint{1.817131in}{1.101217in}}%
\pgfpathlineto{\pgfqpoint{2.025294in}{1.081993in}}%
\pgfpathlineto{\pgfqpoint{2.233457in}{1.059514in}}%
\pgfpathlineto{\pgfqpoint{2.441620in}{1.048807in}}%
\pgfpathlineto{\pgfqpoint{2.649784in}{1.046861in}}%
\pgfpathlineto{\pgfqpoint{2.857947in}{1.054474in}}%
\pgfpathlineto{\pgfqpoint{3.066110in}{1.027462in}}%
\pgfpathlineto{\pgfqpoint{3.274273in}{1.055961in}}%
\pgfpathlineto{\pgfqpoint{3.482437in}{1.038027in}}%
\pgfpathlineto{\pgfqpoint{3.690600in}{1.023031in}}%
\pgfpathlineto{\pgfqpoint{3.898763in}{1.041517in}}%
\pgfpathlineto{\pgfqpoint{4.106926in}{1.016707in}}%
\pgfpathlineto{\pgfqpoint{4.315090in}{1.036980in}}%
\pgfpathlineto{\pgfqpoint{4.523253in}{1.020235in}}%
\pgfpathlineto{\pgfqpoint{4.731416in}{1.008330in}}%
\pgfpathlineto{\pgfqpoint{4.939579in}{1.028414in}}%
\pgfpathlineto{\pgfqpoint{4.939579in}{1.037330in}}%
\pgfpathlineto{\pgfqpoint{4.939579in}{1.037330in}}%
\pgfpathlineto{\pgfqpoint{4.731416in}{1.061607in}}%
\pgfpathlineto{\pgfqpoint{4.523253in}{1.051976in}}%
\pgfpathlineto{\pgfqpoint{4.315090in}{1.044672in}}%
\pgfpathlineto{\pgfqpoint{4.106926in}{1.071809in}}%
\pgfpathlineto{\pgfqpoint{3.898763in}{1.045533in}}%
\pgfpathlineto{\pgfqpoint{3.690600in}{1.044251in}}%
\pgfpathlineto{\pgfqpoint{3.482437in}{1.079413in}}%
\pgfpathlineto{\pgfqpoint{3.274273in}{1.082359in}}%
\pgfpathlineto{\pgfqpoint{3.066110in}{1.065800in}}%
\pgfpathlineto{\pgfqpoint{2.857947in}{1.075132in}}%
\pgfpathlineto{\pgfqpoint{2.649784in}{1.064514in}}%
\pgfpathlineto{\pgfqpoint{2.441620in}{1.084806in}}%
\pgfpathlineto{\pgfqpoint{2.233457in}{1.093603in}}%
\pgfpathlineto{\pgfqpoint{2.025294in}{1.096119in}}%
\pgfpathlineto{\pgfqpoint{1.817131in}{1.126273in}}%
\pgfpathlineto{\pgfqpoint{1.608967in}{1.239424in}}%
\pgfpathlineto{\pgfqpoint{1.400804in}{1.200441in}}%
\pgfpathlineto{\pgfqpoint{1.192641in}{1.008969in}}%
\pgfpathlineto{\pgfqpoint{0.984478in}{0.796139in}}%
\pgfpathlineto{\pgfqpoint{0.880396in}{0.689458in}}%
\pgfpathlineto{\pgfqpoint{0.828355in}{0.636118in}}%
\pgfpathlineto{\pgfqpoint{0.802335in}{0.609448in}}%
\pgfpathlineto{\pgfqpoint{0.777355in}{0.583845in}}%
\pgfpathlineto{\pgfqpoint{0.777355in}{0.583845in}}%
\pgfpathclose%
\pgfusepath{stroke,fill}%
}%
\begin{pgfscope}%
\pgfsys@transformshift{0.000000in}{0.000000in}%
\pgfsys@useobject{currentmarker}{}%
\end{pgfscope}%
\end{pgfscope}%
\begin{pgfscope}%
\pgfpathrectangle{\pgfqpoint{0.776314in}{0.582778in}}{\pgfqpoint{4.163265in}{4.267222in}}%
\pgfusepath{clip}%
\pgfsetbuttcap%
\pgfsetroundjoin%
\definecolor{currentfill}{rgb}{0.800000,0.470588,0.737255}%
\pgfsetfillcolor{currentfill}%
\pgfsetfillopacity{0.200000}%
\pgfsetlinewidth{1.003750pt}%
\definecolor{currentstroke}{rgb}{0.800000,0.470588,0.737255}%
\pgfsetstrokecolor{currentstroke}%
\pgfsetstrokeopacity{0.200000}%
\pgfsetdash{}{0pt}%
\pgfsys@defobject{currentmarker}{\pgfqpoint{0.777355in}{0.583845in}}{\pgfqpoint{4.939579in}{1.062390in}}{%
\pgfpathmoveto{\pgfqpoint{0.777355in}{0.583845in}}%
\pgfpathlineto{\pgfqpoint{0.777355in}{0.583845in}}%
\pgfpathlineto{\pgfqpoint{0.802335in}{0.609448in}}%
\pgfpathlineto{\pgfqpoint{0.828355in}{0.636118in}}%
\pgfpathlineto{\pgfqpoint{0.880396in}{0.689458in}}%
\pgfpathlineto{\pgfqpoint{0.984478in}{0.796139in}}%
\pgfpathlineto{\pgfqpoint{1.192641in}{0.997045in}}%
\pgfpathlineto{\pgfqpoint{1.400804in}{1.038514in}}%
\pgfpathlineto{\pgfqpoint{1.608967in}{0.981191in}}%
\pgfpathlineto{\pgfqpoint{1.817131in}{0.959308in}}%
\pgfpathlineto{\pgfqpoint{2.025294in}{0.969021in}}%
\pgfpathlineto{\pgfqpoint{2.233457in}{0.937423in}}%
\pgfpathlineto{\pgfqpoint{2.441620in}{0.956981in}}%
\pgfpathlineto{\pgfqpoint{2.649784in}{0.954590in}}%
\pgfpathlineto{\pgfqpoint{2.857947in}{0.922885in}}%
\pgfpathlineto{\pgfqpoint{3.066110in}{0.946566in}}%
\pgfpathlineto{\pgfqpoint{3.274273in}{0.940952in}}%
\pgfpathlineto{\pgfqpoint{3.482437in}{0.938011in}}%
\pgfpathlineto{\pgfqpoint{3.690600in}{0.917226in}}%
\pgfpathlineto{\pgfqpoint{3.898763in}{0.924350in}}%
\pgfpathlineto{\pgfqpoint{4.106926in}{0.930172in}}%
\pgfpathlineto{\pgfqpoint{4.315090in}{0.912600in}}%
\pgfpathlineto{\pgfqpoint{4.523253in}{0.926650in}}%
\pgfpathlineto{\pgfqpoint{4.731416in}{0.918982in}}%
\pgfpathlineto{\pgfqpoint{4.939579in}{0.926503in}}%
\pgfpathlineto{\pgfqpoint{4.939579in}{0.938031in}}%
\pgfpathlineto{\pgfqpoint{4.939579in}{0.938031in}}%
\pgfpathlineto{\pgfqpoint{4.731416in}{0.936190in}}%
\pgfpathlineto{\pgfqpoint{4.523253in}{0.932038in}}%
\pgfpathlineto{\pgfqpoint{4.315090in}{0.944383in}}%
\pgfpathlineto{\pgfqpoint{4.106926in}{0.942104in}}%
\pgfpathlineto{\pgfqpoint{3.898763in}{0.955241in}}%
\pgfpathlineto{\pgfqpoint{3.690600in}{0.942451in}}%
\pgfpathlineto{\pgfqpoint{3.482437in}{0.952655in}}%
\pgfpathlineto{\pgfqpoint{3.274273in}{0.953879in}}%
\pgfpathlineto{\pgfqpoint{3.066110in}{0.966292in}}%
\pgfpathlineto{\pgfqpoint{2.857947in}{0.969547in}}%
\pgfpathlineto{\pgfqpoint{2.649784in}{0.958659in}}%
\pgfpathlineto{\pgfqpoint{2.441620in}{0.969073in}}%
\pgfpathlineto{\pgfqpoint{2.233457in}{0.976483in}}%
\pgfpathlineto{\pgfqpoint{2.025294in}{0.975626in}}%
\pgfpathlineto{\pgfqpoint{1.817131in}{0.978438in}}%
\pgfpathlineto{\pgfqpoint{1.608967in}{1.005816in}}%
\pgfpathlineto{\pgfqpoint{1.400804in}{1.062390in}}%
\pgfpathlineto{\pgfqpoint{1.192641in}{1.004329in}}%
\pgfpathlineto{\pgfqpoint{0.984478in}{0.796139in}}%
\pgfpathlineto{\pgfqpoint{0.880396in}{0.689458in}}%
\pgfpathlineto{\pgfqpoint{0.828355in}{0.636118in}}%
\pgfpathlineto{\pgfqpoint{0.802335in}{0.609448in}}%
\pgfpathlineto{\pgfqpoint{0.777355in}{0.583845in}}%
\pgfpathlineto{\pgfqpoint{0.777355in}{0.583845in}}%
\pgfpathclose%
\pgfusepath{stroke,fill}%
}%
\begin{pgfscope}%
\pgfsys@transformshift{0.000000in}{0.000000in}%
\pgfsys@useobject{currentmarker}{}%
\end{pgfscope}%
\end{pgfscope}%
\begin{pgfscope}%
\pgfpathrectangle{\pgfqpoint{0.776314in}{0.582778in}}{\pgfqpoint{4.163265in}{4.267222in}}%
\pgfusepath{clip}%
\pgfsetbuttcap%
\pgfsetroundjoin%
\definecolor{currentfill}{rgb}{0.792157,0.568627,0.380392}%
\pgfsetfillcolor{currentfill}%
\pgfsetfillopacity{0.200000}%
\pgfsetlinewidth{1.003750pt}%
\definecolor{currentstroke}{rgb}{0.792157,0.568627,0.380392}%
\pgfsetstrokecolor{currentstroke}%
\pgfsetstrokeopacity{0.200000}%
\pgfsetdash{}{0pt}%
\pgfsys@defobject{currentmarker}{\pgfqpoint{0.777355in}{0.583845in}}{\pgfqpoint{4.939579in}{0.981977in}}{%
\pgfpathmoveto{\pgfqpoint{0.777355in}{0.583845in}}%
\pgfpathlineto{\pgfqpoint{0.777355in}{0.583845in}}%
\pgfpathlineto{\pgfqpoint{0.802335in}{0.609448in}}%
\pgfpathlineto{\pgfqpoint{0.828355in}{0.636118in}}%
\pgfpathlineto{\pgfqpoint{0.880396in}{0.689458in}}%
\pgfpathlineto{\pgfqpoint{0.984478in}{0.796139in}}%
\pgfpathlineto{\pgfqpoint{1.192641in}{0.967107in}}%
\pgfpathlineto{\pgfqpoint{1.400804in}{0.907165in}}%
\pgfpathlineto{\pgfqpoint{1.608967in}{0.865923in}}%
\pgfpathlineto{\pgfqpoint{1.817131in}{0.878169in}}%
\pgfpathlineto{\pgfqpoint{2.025294in}{0.873357in}}%
\pgfpathlineto{\pgfqpoint{2.233457in}{0.878627in}}%
\pgfpathlineto{\pgfqpoint{2.441620in}{0.869986in}}%
\pgfpathlineto{\pgfqpoint{2.649784in}{0.865009in}}%
\pgfpathlineto{\pgfqpoint{2.857947in}{0.860830in}}%
\pgfpathlineto{\pgfqpoint{3.066110in}{0.861438in}}%
\pgfpathlineto{\pgfqpoint{3.274273in}{0.857353in}}%
\pgfpathlineto{\pgfqpoint{3.482437in}{0.861800in}}%
\pgfpathlineto{\pgfqpoint{3.690600in}{0.861963in}}%
\pgfpathlineto{\pgfqpoint{3.898763in}{0.871036in}}%
\pgfpathlineto{\pgfqpoint{4.106926in}{0.859813in}}%
\pgfpathlineto{\pgfqpoint{4.315090in}{0.836024in}}%
\pgfpathlineto{\pgfqpoint{4.523253in}{0.868875in}}%
\pgfpathlineto{\pgfqpoint{4.731416in}{0.850404in}}%
\pgfpathlineto{\pgfqpoint{4.939579in}{0.847171in}}%
\pgfpathlineto{\pgfqpoint{4.939579in}{0.866197in}}%
\pgfpathlineto{\pgfqpoint{4.939579in}{0.866197in}}%
\pgfpathlineto{\pgfqpoint{4.731416in}{0.863733in}}%
\pgfpathlineto{\pgfqpoint{4.523253in}{0.870987in}}%
\pgfpathlineto{\pgfqpoint{4.315090in}{0.858624in}}%
\pgfpathlineto{\pgfqpoint{4.106926in}{0.873470in}}%
\pgfpathlineto{\pgfqpoint{3.898763in}{0.881068in}}%
\pgfpathlineto{\pgfqpoint{3.690600in}{0.866542in}}%
\pgfpathlineto{\pgfqpoint{3.482437in}{0.872361in}}%
\pgfpathlineto{\pgfqpoint{3.274273in}{0.876018in}}%
\pgfpathlineto{\pgfqpoint{3.066110in}{0.873015in}}%
\pgfpathlineto{\pgfqpoint{2.857947in}{0.881247in}}%
\pgfpathlineto{\pgfqpoint{2.649784in}{0.876564in}}%
\pgfpathlineto{\pgfqpoint{2.441620in}{0.882934in}}%
\pgfpathlineto{\pgfqpoint{2.233457in}{0.887404in}}%
\pgfpathlineto{\pgfqpoint{2.025294in}{0.891092in}}%
\pgfpathlineto{\pgfqpoint{1.817131in}{0.885017in}}%
\pgfpathlineto{\pgfqpoint{1.608967in}{0.899987in}}%
\pgfpathlineto{\pgfqpoint{1.400804in}{0.920507in}}%
\pgfpathlineto{\pgfqpoint{1.192641in}{0.981977in}}%
\pgfpathlineto{\pgfqpoint{0.984478in}{0.796139in}}%
\pgfpathlineto{\pgfqpoint{0.880396in}{0.689458in}}%
\pgfpathlineto{\pgfqpoint{0.828355in}{0.636118in}}%
\pgfpathlineto{\pgfqpoint{0.802335in}{0.609448in}}%
\pgfpathlineto{\pgfqpoint{0.777355in}{0.583845in}}%
\pgfpathlineto{\pgfqpoint{0.777355in}{0.583845in}}%
\pgfpathclose%
\pgfusepath{stroke,fill}%
}%
\begin{pgfscope}%
\pgfsys@transformshift{0.000000in}{0.000000in}%
\pgfsys@useobject{currentmarker}{}%
\end{pgfscope}%
\end{pgfscope}%
\begin{pgfscope}%
\pgfsetbuttcap%
\pgfsetroundjoin%
\definecolor{currentfill}{rgb}{0.000000,0.000000,0.000000}%
\pgfsetfillcolor{currentfill}%
\pgfsetlinewidth{0.803000pt}%
\definecolor{currentstroke}{rgb}{0.000000,0.000000,0.000000}%
\pgfsetstrokecolor{currentstroke}%
\pgfsetdash{}{0pt}%
\pgfsys@defobject{currentmarker}{\pgfqpoint{0.000000in}{-0.048611in}}{\pgfqpoint{0.000000in}{0.000000in}}{%
\pgfpathmoveto{\pgfqpoint{0.000000in}{0.000000in}}%
\pgfpathlineto{\pgfqpoint{0.000000in}{-0.048611in}}%
\pgfusepath{stroke,fill}%
}%
\begin{pgfscope}%
\pgfsys@transformshift{0.776314in}{0.582778in}%
\pgfsys@useobject{currentmarker}{}%
\end{pgfscope}%
\end{pgfscope}%
\begin{pgfscope}%
\definecolor{textcolor}{rgb}{0.000000,0.000000,0.000000}%
\pgfsetstrokecolor{textcolor}%
\pgfsetfillcolor{textcolor}%
\pgftext[x=0.776314in,y=0.485556in,,top]{\color{textcolor}\sffamily\fontsize{10.000000}{12.000000}\selectfont 0}%
\end{pgfscope}%
\begin{pgfscope}%
\pgfsetbuttcap%
\pgfsetroundjoin%
\definecolor{currentfill}{rgb}{0.000000,0.000000,0.000000}%
\pgfsetfillcolor{currentfill}%
\pgfsetlinewidth{0.803000pt}%
\definecolor{currentstroke}{rgb}{0.000000,0.000000,0.000000}%
\pgfsetstrokecolor{currentstroke}%
\pgfsetdash{}{0pt}%
\pgfsys@defobject{currentmarker}{\pgfqpoint{0.000000in}{-0.048611in}}{\pgfqpoint{0.000000in}{0.000000in}}{%
\pgfpathmoveto{\pgfqpoint{0.000000in}{0.000000in}}%
\pgfpathlineto{\pgfqpoint{0.000000in}{-0.048611in}}%
\pgfusepath{stroke,fill}%
}%
\begin{pgfscope}%
\pgfsys@transformshift{1.296723in}{0.582778in}%
\pgfsys@useobject{currentmarker}{}%
\end{pgfscope}%
\end{pgfscope}%
\begin{pgfscope}%
\definecolor{textcolor}{rgb}{0.000000,0.000000,0.000000}%
\pgfsetstrokecolor{textcolor}%
\pgfsetfillcolor{textcolor}%
\pgftext[x=1.296723in,y=0.485556in,,top]{\color{textcolor}\sffamily\fontsize{10.000000}{12.000000}\selectfont 500}%
\end{pgfscope}%
\begin{pgfscope}%
\pgfsetbuttcap%
\pgfsetroundjoin%
\definecolor{currentfill}{rgb}{0.000000,0.000000,0.000000}%
\pgfsetfillcolor{currentfill}%
\pgfsetlinewidth{0.803000pt}%
\definecolor{currentstroke}{rgb}{0.000000,0.000000,0.000000}%
\pgfsetstrokecolor{currentstroke}%
\pgfsetdash{}{0pt}%
\pgfsys@defobject{currentmarker}{\pgfqpoint{0.000000in}{-0.048611in}}{\pgfqpoint{0.000000in}{0.000000in}}{%
\pgfpathmoveto{\pgfqpoint{0.000000in}{0.000000in}}%
\pgfpathlineto{\pgfqpoint{0.000000in}{-0.048611in}}%
\pgfusepath{stroke,fill}%
}%
\begin{pgfscope}%
\pgfsys@transformshift{1.817131in}{0.582778in}%
\pgfsys@useobject{currentmarker}{}%
\end{pgfscope}%
\end{pgfscope}%
\begin{pgfscope}%
\definecolor{textcolor}{rgb}{0.000000,0.000000,0.000000}%
\pgfsetstrokecolor{textcolor}%
\pgfsetfillcolor{textcolor}%
\pgftext[x=1.817131in,y=0.485556in,,top]{\color{textcolor}\sffamily\fontsize{10.000000}{12.000000}\selectfont 1000}%
\end{pgfscope}%
\begin{pgfscope}%
\pgfsetbuttcap%
\pgfsetroundjoin%
\definecolor{currentfill}{rgb}{0.000000,0.000000,0.000000}%
\pgfsetfillcolor{currentfill}%
\pgfsetlinewidth{0.803000pt}%
\definecolor{currentstroke}{rgb}{0.000000,0.000000,0.000000}%
\pgfsetstrokecolor{currentstroke}%
\pgfsetdash{}{0pt}%
\pgfsys@defobject{currentmarker}{\pgfqpoint{0.000000in}{-0.048611in}}{\pgfqpoint{0.000000in}{0.000000in}}{%
\pgfpathmoveto{\pgfqpoint{0.000000in}{0.000000in}}%
\pgfpathlineto{\pgfqpoint{0.000000in}{-0.048611in}}%
\pgfusepath{stroke,fill}%
}%
\begin{pgfscope}%
\pgfsys@transformshift{2.337539in}{0.582778in}%
\pgfsys@useobject{currentmarker}{}%
\end{pgfscope}%
\end{pgfscope}%
\begin{pgfscope}%
\definecolor{textcolor}{rgb}{0.000000,0.000000,0.000000}%
\pgfsetstrokecolor{textcolor}%
\pgfsetfillcolor{textcolor}%
\pgftext[x=2.337539in,y=0.485556in,,top]{\color{textcolor}\sffamily\fontsize{10.000000}{12.000000}\selectfont 1500}%
\end{pgfscope}%
\begin{pgfscope}%
\pgfsetbuttcap%
\pgfsetroundjoin%
\definecolor{currentfill}{rgb}{0.000000,0.000000,0.000000}%
\pgfsetfillcolor{currentfill}%
\pgfsetlinewidth{0.803000pt}%
\definecolor{currentstroke}{rgb}{0.000000,0.000000,0.000000}%
\pgfsetstrokecolor{currentstroke}%
\pgfsetdash{}{0pt}%
\pgfsys@defobject{currentmarker}{\pgfqpoint{0.000000in}{-0.048611in}}{\pgfqpoint{0.000000in}{0.000000in}}{%
\pgfpathmoveto{\pgfqpoint{0.000000in}{0.000000in}}%
\pgfpathlineto{\pgfqpoint{0.000000in}{-0.048611in}}%
\pgfusepath{stroke,fill}%
}%
\begin{pgfscope}%
\pgfsys@transformshift{2.857947in}{0.582778in}%
\pgfsys@useobject{currentmarker}{}%
\end{pgfscope}%
\end{pgfscope}%
\begin{pgfscope}%
\definecolor{textcolor}{rgb}{0.000000,0.000000,0.000000}%
\pgfsetstrokecolor{textcolor}%
\pgfsetfillcolor{textcolor}%
\pgftext[x=2.857947in,y=0.485556in,,top]{\color{textcolor}\sffamily\fontsize{10.000000}{12.000000}\selectfont 2000}%
\end{pgfscope}%
\begin{pgfscope}%
\pgfsetbuttcap%
\pgfsetroundjoin%
\definecolor{currentfill}{rgb}{0.000000,0.000000,0.000000}%
\pgfsetfillcolor{currentfill}%
\pgfsetlinewidth{0.803000pt}%
\definecolor{currentstroke}{rgb}{0.000000,0.000000,0.000000}%
\pgfsetstrokecolor{currentstroke}%
\pgfsetdash{}{0pt}%
\pgfsys@defobject{currentmarker}{\pgfqpoint{0.000000in}{-0.048611in}}{\pgfqpoint{0.000000in}{0.000000in}}{%
\pgfpathmoveto{\pgfqpoint{0.000000in}{0.000000in}}%
\pgfpathlineto{\pgfqpoint{0.000000in}{-0.048611in}}%
\pgfusepath{stroke,fill}%
}%
\begin{pgfscope}%
\pgfsys@transformshift{3.378355in}{0.582778in}%
\pgfsys@useobject{currentmarker}{}%
\end{pgfscope}%
\end{pgfscope}%
\begin{pgfscope}%
\definecolor{textcolor}{rgb}{0.000000,0.000000,0.000000}%
\pgfsetstrokecolor{textcolor}%
\pgfsetfillcolor{textcolor}%
\pgftext[x=3.378355in,y=0.485556in,,top]{\color{textcolor}\sffamily\fontsize{10.000000}{12.000000}\selectfont 2500}%
\end{pgfscope}%
\begin{pgfscope}%
\pgfsetbuttcap%
\pgfsetroundjoin%
\definecolor{currentfill}{rgb}{0.000000,0.000000,0.000000}%
\pgfsetfillcolor{currentfill}%
\pgfsetlinewidth{0.803000pt}%
\definecolor{currentstroke}{rgb}{0.000000,0.000000,0.000000}%
\pgfsetstrokecolor{currentstroke}%
\pgfsetdash{}{0pt}%
\pgfsys@defobject{currentmarker}{\pgfqpoint{0.000000in}{-0.048611in}}{\pgfqpoint{0.000000in}{0.000000in}}{%
\pgfpathmoveto{\pgfqpoint{0.000000in}{0.000000in}}%
\pgfpathlineto{\pgfqpoint{0.000000in}{-0.048611in}}%
\pgfusepath{stroke,fill}%
}%
\begin{pgfscope}%
\pgfsys@transformshift{3.898763in}{0.582778in}%
\pgfsys@useobject{currentmarker}{}%
\end{pgfscope}%
\end{pgfscope}%
\begin{pgfscope}%
\definecolor{textcolor}{rgb}{0.000000,0.000000,0.000000}%
\pgfsetstrokecolor{textcolor}%
\pgfsetfillcolor{textcolor}%
\pgftext[x=3.898763in,y=0.485556in,,top]{\color{textcolor}\sffamily\fontsize{10.000000}{12.000000}\selectfont 3000}%
\end{pgfscope}%
\begin{pgfscope}%
\pgfsetbuttcap%
\pgfsetroundjoin%
\definecolor{currentfill}{rgb}{0.000000,0.000000,0.000000}%
\pgfsetfillcolor{currentfill}%
\pgfsetlinewidth{0.803000pt}%
\definecolor{currentstroke}{rgb}{0.000000,0.000000,0.000000}%
\pgfsetstrokecolor{currentstroke}%
\pgfsetdash{}{0pt}%
\pgfsys@defobject{currentmarker}{\pgfqpoint{0.000000in}{-0.048611in}}{\pgfqpoint{0.000000in}{0.000000in}}{%
\pgfpathmoveto{\pgfqpoint{0.000000in}{0.000000in}}%
\pgfpathlineto{\pgfqpoint{0.000000in}{-0.048611in}}%
\pgfusepath{stroke,fill}%
}%
\begin{pgfscope}%
\pgfsys@transformshift{4.419171in}{0.582778in}%
\pgfsys@useobject{currentmarker}{}%
\end{pgfscope}%
\end{pgfscope}%
\begin{pgfscope}%
\definecolor{textcolor}{rgb}{0.000000,0.000000,0.000000}%
\pgfsetstrokecolor{textcolor}%
\pgfsetfillcolor{textcolor}%
\pgftext[x=4.419171in,y=0.485556in,,top]{\color{textcolor}\sffamily\fontsize{10.000000}{12.000000}\selectfont 3500}%
\end{pgfscope}%
\begin{pgfscope}%
\pgfsetbuttcap%
\pgfsetroundjoin%
\definecolor{currentfill}{rgb}{0.000000,0.000000,0.000000}%
\pgfsetfillcolor{currentfill}%
\pgfsetlinewidth{0.803000pt}%
\definecolor{currentstroke}{rgb}{0.000000,0.000000,0.000000}%
\pgfsetstrokecolor{currentstroke}%
\pgfsetdash{}{0pt}%
\pgfsys@defobject{currentmarker}{\pgfqpoint{0.000000in}{-0.048611in}}{\pgfqpoint{0.000000in}{0.000000in}}{%
\pgfpathmoveto{\pgfqpoint{0.000000in}{0.000000in}}%
\pgfpathlineto{\pgfqpoint{0.000000in}{-0.048611in}}%
\pgfusepath{stroke,fill}%
}%
\begin{pgfscope}%
\pgfsys@transformshift{4.939579in}{0.582778in}%
\pgfsys@useobject{currentmarker}{}%
\end{pgfscope}%
\end{pgfscope}%
\begin{pgfscope}%
\definecolor{textcolor}{rgb}{0.000000,0.000000,0.000000}%
\pgfsetstrokecolor{textcolor}%
\pgfsetfillcolor{textcolor}%
\pgftext[x=4.939579in,y=0.485556in,,top]{\color{textcolor}\sffamily\fontsize{10.000000}{12.000000}\selectfont 4000}%
\end{pgfscope}%
\begin{pgfscope}%
\definecolor{textcolor}{rgb}{0.000000,0.000000,0.000000}%
\pgfsetstrokecolor{textcolor}%
\pgfsetfillcolor{textcolor}%
\pgftext[x=2.857947in,y=0.295587in,,top]{\color{textcolor}\sffamily\fontsize{10.000000}{12.000000}\selectfont throughput (req/s)}%
\end{pgfscope}%
\begin{pgfscope}%
\pgfsetbuttcap%
\pgfsetroundjoin%
\definecolor{currentfill}{rgb}{0.000000,0.000000,0.000000}%
\pgfsetfillcolor{currentfill}%
\pgfsetlinewidth{0.803000pt}%
\definecolor{currentstroke}{rgb}{0.000000,0.000000,0.000000}%
\pgfsetstrokecolor{currentstroke}%
\pgfsetdash{}{0pt}%
\pgfsys@defobject{currentmarker}{\pgfqpoint{-0.048611in}{0.000000in}}{\pgfqpoint{-0.000000in}{0.000000in}}{%
\pgfpathmoveto{\pgfqpoint{-0.000000in}{0.000000in}}%
\pgfpathlineto{\pgfqpoint{-0.048611in}{0.000000in}}%
\pgfusepath{stroke,fill}%
}%
\begin{pgfscope}%
\pgfsys@transformshift{0.776314in}{0.582778in}%
\pgfsys@useobject{currentmarker}{}%
\end{pgfscope}%
\end{pgfscope}%
\begin{pgfscope}%
\definecolor{textcolor}{rgb}{0.000000,0.000000,0.000000}%
\pgfsetstrokecolor{textcolor}%
\pgfsetfillcolor{textcolor}%
\pgftext[x=0.590727in, y=0.530016in, left, base]{\color{textcolor}\sffamily\fontsize{10.000000}{12.000000}\selectfont 0}%
\end{pgfscope}%
\begin{pgfscope}%
\pgfsetbuttcap%
\pgfsetroundjoin%
\definecolor{currentfill}{rgb}{0.000000,0.000000,0.000000}%
\pgfsetfillcolor{currentfill}%
\pgfsetlinewidth{0.803000pt}%
\definecolor{currentstroke}{rgb}{0.000000,0.000000,0.000000}%
\pgfsetstrokecolor{currentstroke}%
\pgfsetdash{}{0pt}%
\pgfsys@defobject{currentmarker}{\pgfqpoint{-0.048611in}{0.000000in}}{\pgfqpoint{-0.000000in}{0.000000in}}{%
\pgfpathmoveto{\pgfqpoint{-0.000000in}{0.000000in}}%
\pgfpathlineto{\pgfqpoint{-0.048611in}{0.000000in}}%
\pgfusepath{stroke,fill}%
}%
\begin{pgfscope}%
\pgfsys@transformshift{0.776314in}{1.116181in}%
\pgfsys@useobject{currentmarker}{}%
\end{pgfscope}%
\end{pgfscope}%
\begin{pgfscope}%
\definecolor{textcolor}{rgb}{0.000000,0.000000,0.000000}%
\pgfsetstrokecolor{textcolor}%
\pgfsetfillcolor{textcolor}%
\pgftext[x=0.413996in, y=1.063419in, left, base]{\color{textcolor}\sffamily\fontsize{10.000000}{12.000000}\selectfont 500}%
\end{pgfscope}%
\begin{pgfscope}%
\pgfsetbuttcap%
\pgfsetroundjoin%
\definecolor{currentfill}{rgb}{0.000000,0.000000,0.000000}%
\pgfsetfillcolor{currentfill}%
\pgfsetlinewidth{0.803000pt}%
\definecolor{currentstroke}{rgb}{0.000000,0.000000,0.000000}%
\pgfsetstrokecolor{currentstroke}%
\pgfsetdash{}{0pt}%
\pgfsys@defobject{currentmarker}{\pgfqpoint{-0.048611in}{0.000000in}}{\pgfqpoint{-0.000000in}{0.000000in}}{%
\pgfpathmoveto{\pgfqpoint{-0.000000in}{0.000000in}}%
\pgfpathlineto{\pgfqpoint{-0.048611in}{0.000000in}}%
\pgfusepath{stroke,fill}%
}%
\begin{pgfscope}%
\pgfsys@transformshift{0.776314in}{1.649583in}%
\pgfsys@useobject{currentmarker}{}%
\end{pgfscope}%
\end{pgfscope}%
\begin{pgfscope}%
\definecolor{textcolor}{rgb}{0.000000,0.000000,0.000000}%
\pgfsetstrokecolor{textcolor}%
\pgfsetfillcolor{textcolor}%
\pgftext[x=0.325631in, y=1.596822in, left, base]{\color{textcolor}\sffamily\fontsize{10.000000}{12.000000}\selectfont 1000}%
\end{pgfscope}%
\begin{pgfscope}%
\pgfsetbuttcap%
\pgfsetroundjoin%
\definecolor{currentfill}{rgb}{0.000000,0.000000,0.000000}%
\pgfsetfillcolor{currentfill}%
\pgfsetlinewidth{0.803000pt}%
\definecolor{currentstroke}{rgb}{0.000000,0.000000,0.000000}%
\pgfsetstrokecolor{currentstroke}%
\pgfsetdash{}{0pt}%
\pgfsys@defobject{currentmarker}{\pgfqpoint{-0.048611in}{0.000000in}}{\pgfqpoint{-0.000000in}{0.000000in}}{%
\pgfpathmoveto{\pgfqpoint{-0.000000in}{0.000000in}}%
\pgfpathlineto{\pgfqpoint{-0.048611in}{0.000000in}}%
\pgfusepath{stroke,fill}%
}%
\begin{pgfscope}%
\pgfsys@transformshift{0.776314in}{2.182986in}%
\pgfsys@useobject{currentmarker}{}%
\end{pgfscope}%
\end{pgfscope}%
\begin{pgfscope}%
\definecolor{textcolor}{rgb}{0.000000,0.000000,0.000000}%
\pgfsetstrokecolor{textcolor}%
\pgfsetfillcolor{textcolor}%
\pgftext[x=0.325631in, y=2.130225in, left, base]{\color{textcolor}\sffamily\fontsize{10.000000}{12.000000}\selectfont 1500}%
\end{pgfscope}%
\begin{pgfscope}%
\pgfsetbuttcap%
\pgfsetroundjoin%
\definecolor{currentfill}{rgb}{0.000000,0.000000,0.000000}%
\pgfsetfillcolor{currentfill}%
\pgfsetlinewidth{0.803000pt}%
\definecolor{currentstroke}{rgb}{0.000000,0.000000,0.000000}%
\pgfsetstrokecolor{currentstroke}%
\pgfsetdash{}{0pt}%
\pgfsys@defobject{currentmarker}{\pgfqpoint{-0.048611in}{0.000000in}}{\pgfqpoint{-0.000000in}{0.000000in}}{%
\pgfpathmoveto{\pgfqpoint{-0.000000in}{0.000000in}}%
\pgfpathlineto{\pgfqpoint{-0.048611in}{0.000000in}}%
\pgfusepath{stroke,fill}%
}%
\begin{pgfscope}%
\pgfsys@transformshift{0.776314in}{2.716389in}%
\pgfsys@useobject{currentmarker}{}%
\end{pgfscope}%
\end{pgfscope}%
\begin{pgfscope}%
\definecolor{textcolor}{rgb}{0.000000,0.000000,0.000000}%
\pgfsetstrokecolor{textcolor}%
\pgfsetfillcolor{textcolor}%
\pgftext[x=0.325631in, y=2.663627in, left, base]{\color{textcolor}\sffamily\fontsize{10.000000}{12.000000}\selectfont 2000}%
\end{pgfscope}%
\begin{pgfscope}%
\pgfsetbuttcap%
\pgfsetroundjoin%
\definecolor{currentfill}{rgb}{0.000000,0.000000,0.000000}%
\pgfsetfillcolor{currentfill}%
\pgfsetlinewidth{0.803000pt}%
\definecolor{currentstroke}{rgb}{0.000000,0.000000,0.000000}%
\pgfsetstrokecolor{currentstroke}%
\pgfsetdash{}{0pt}%
\pgfsys@defobject{currentmarker}{\pgfqpoint{-0.048611in}{0.000000in}}{\pgfqpoint{-0.000000in}{0.000000in}}{%
\pgfpathmoveto{\pgfqpoint{-0.000000in}{0.000000in}}%
\pgfpathlineto{\pgfqpoint{-0.048611in}{0.000000in}}%
\pgfusepath{stroke,fill}%
}%
\begin{pgfscope}%
\pgfsys@transformshift{0.776314in}{3.249792in}%
\pgfsys@useobject{currentmarker}{}%
\end{pgfscope}%
\end{pgfscope}%
\begin{pgfscope}%
\definecolor{textcolor}{rgb}{0.000000,0.000000,0.000000}%
\pgfsetstrokecolor{textcolor}%
\pgfsetfillcolor{textcolor}%
\pgftext[x=0.325631in, y=3.197030in, left, base]{\color{textcolor}\sffamily\fontsize{10.000000}{12.000000}\selectfont 2500}%
\end{pgfscope}%
\begin{pgfscope}%
\pgfsetbuttcap%
\pgfsetroundjoin%
\definecolor{currentfill}{rgb}{0.000000,0.000000,0.000000}%
\pgfsetfillcolor{currentfill}%
\pgfsetlinewidth{0.803000pt}%
\definecolor{currentstroke}{rgb}{0.000000,0.000000,0.000000}%
\pgfsetstrokecolor{currentstroke}%
\pgfsetdash{}{0pt}%
\pgfsys@defobject{currentmarker}{\pgfqpoint{-0.048611in}{0.000000in}}{\pgfqpoint{-0.000000in}{0.000000in}}{%
\pgfpathmoveto{\pgfqpoint{-0.000000in}{0.000000in}}%
\pgfpathlineto{\pgfqpoint{-0.048611in}{0.000000in}}%
\pgfusepath{stroke,fill}%
}%
\begin{pgfscope}%
\pgfsys@transformshift{0.776314in}{3.783194in}%
\pgfsys@useobject{currentmarker}{}%
\end{pgfscope}%
\end{pgfscope}%
\begin{pgfscope}%
\definecolor{textcolor}{rgb}{0.000000,0.000000,0.000000}%
\pgfsetstrokecolor{textcolor}%
\pgfsetfillcolor{textcolor}%
\pgftext[x=0.325631in, y=3.730433in, left, base]{\color{textcolor}\sffamily\fontsize{10.000000}{12.000000}\selectfont 3000}%
\end{pgfscope}%
\begin{pgfscope}%
\pgfsetbuttcap%
\pgfsetroundjoin%
\definecolor{currentfill}{rgb}{0.000000,0.000000,0.000000}%
\pgfsetfillcolor{currentfill}%
\pgfsetlinewidth{0.803000pt}%
\definecolor{currentstroke}{rgb}{0.000000,0.000000,0.000000}%
\pgfsetstrokecolor{currentstroke}%
\pgfsetdash{}{0pt}%
\pgfsys@defobject{currentmarker}{\pgfqpoint{-0.048611in}{0.000000in}}{\pgfqpoint{-0.000000in}{0.000000in}}{%
\pgfpathmoveto{\pgfqpoint{-0.000000in}{0.000000in}}%
\pgfpathlineto{\pgfqpoint{-0.048611in}{0.000000in}}%
\pgfusepath{stroke,fill}%
}%
\begin{pgfscope}%
\pgfsys@transformshift{0.776314in}{4.316597in}%
\pgfsys@useobject{currentmarker}{}%
\end{pgfscope}%
\end{pgfscope}%
\begin{pgfscope}%
\definecolor{textcolor}{rgb}{0.000000,0.000000,0.000000}%
\pgfsetstrokecolor{textcolor}%
\pgfsetfillcolor{textcolor}%
\pgftext[x=0.325631in, y=4.263836in, left, base]{\color{textcolor}\sffamily\fontsize{10.000000}{12.000000}\selectfont 3500}%
\end{pgfscope}%
\begin{pgfscope}%
\pgfsetbuttcap%
\pgfsetroundjoin%
\definecolor{currentfill}{rgb}{0.000000,0.000000,0.000000}%
\pgfsetfillcolor{currentfill}%
\pgfsetlinewidth{0.803000pt}%
\definecolor{currentstroke}{rgb}{0.000000,0.000000,0.000000}%
\pgfsetstrokecolor{currentstroke}%
\pgfsetdash{}{0pt}%
\pgfsys@defobject{currentmarker}{\pgfqpoint{-0.048611in}{0.000000in}}{\pgfqpoint{-0.000000in}{0.000000in}}{%
\pgfpathmoveto{\pgfqpoint{-0.000000in}{0.000000in}}%
\pgfpathlineto{\pgfqpoint{-0.048611in}{0.000000in}}%
\pgfusepath{stroke,fill}%
}%
\begin{pgfscope}%
\pgfsys@transformshift{0.776314in}{4.850000in}%
\pgfsys@useobject{currentmarker}{}%
\end{pgfscope}%
\end{pgfscope}%
\begin{pgfscope}%
\definecolor{textcolor}{rgb}{0.000000,0.000000,0.000000}%
\pgfsetstrokecolor{textcolor}%
\pgfsetfillcolor{textcolor}%
\pgftext[x=0.325631in, y=4.797238in, left, base]{\color{textcolor}\sffamily\fontsize{10.000000}{12.000000}\selectfont 4000}%
\end{pgfscope}%
\begin{pgfscope}%
\definecolor{textcolor}{rgb}{0.000000,0.000000,0.000000}%
\pgfsetstrokecolor{textcolor}%
\pgfsetfillcolor{textcolor}%
\pgftext[x=0.270075in,y=2.716389in,,bottom,rotate=90.000000]{\color{textcolor}\sffamily\fontsize{10.000000}{12.000000}\selectfont goodput (req/s)}%
\end{pgfscope}%
\begin{pgfscope}%
\pgfpathrectangle{\pgfqpoint{0.776314in}{0.582778in}}{\pgfqpoint{4.163265in}{4.267222in}}%
\pgfusepath{clip}%
\pgfsetbuttcap%
\pgfsetroundjoin%
\pgfsetlinewidth{1.505625pt}%
\definecolor{currentstroke}{rgb}{0.003922,0.450980,0.698039}%
\pgfsetstrokecolor{currentstroke}%
\pgfsetdash{{5.550000pt}{2.400000pt}}{0.000000pt}%
\pgfpathmoveto{\pgfqpoint{0.777355in}{0.583845in}}%
\pgfpathlineto{\pgfqpoint{0.802335in}{0.609448in}}%
\pgfpathlineto{\pgfqpoint{0.828355in}{0.636118in}}%
\pgfpathlineto{\pgfqpoint{0.880396in}{0.689400in}}%
\pgfpathlineto{\pgfqpoint{0.984478in}{0.796098in}}%
\pgfpathlineto{\pgfqpoint{1.192641in}{1.008842in}}%
\pgfpathlineto{\pgfqpoint{1.400804in}{1.222103in}}%
\pgfpathlineto{\pgfqpoint{1.608967in}{1.435874in}}%
\pgfpathlineto{\pgfqpoint{1.817131in}{1.647208in}}%
\pgfpathlineto{\pgfqpoint{2.025294in}{1.861095in}}%
\pgfpathlineto{\pgfqpoint{2.233457in}{2.074922in}}%
\pgfpathlineto{\pgfqpoint{2.441620in}{2.283837in}}%
\pgfpathlineto{\pgfqpoint{2.649784in}{2.502836in}}%
\pgfpathlineto{\pgfqpoint{2.857947in}{2.709860in}}%
\pgfpathlineto{\pgfqpoint{3.066110in}{2.925746in}}%
\pgfpathlineto{\pgfqpoint{3.274273in}{3.139579in}}%
\pgfpathlineto{\pgfqpoint{3.482437in}{3.353239in}}%
\pgfpathlineto{\pgfqpoint{3.690600in}{3.556290in}}%
\pgfpathlineto{\pgfqpoint{3.898763in}{3.766813in}}%
\pgfpathlineto{\pgfqpoint{4.106926in}{3.971851in}}%
\pgfpathlineto{\pgfqpoint{4.315090in}{4.198984in}}%
\pgfpathlineto{\pgfqpoint{4.523253in}{4.409040in}}%
\pgfpathlineto{\pgfqpoint{4.731416in}{4.614874in}}%
\pgfpathlineto{\pgfqpoint{4.939579in}{4.835803in}}%
\pgfusepath{stroke}%
\end{pgfscope}%
\begin{pgfscope}%
\pgfpathrectangle{\pgfqpoint{0.776314in}{0.582778in}}{\pgfqpoint{4.163265in}{4.267222in}}%
\pgfusepath{clip}%
\pgfsetbuttcap%
\pgfsetroundjoin%
\pgfsetlinewidth{1.505625pt}%
\definecolor{currentstroke}{rgb}{0.870588,0.560784,0.019608}%
\pgfsetstrokecolor{currentstroke}%
\pgfsetdash{{5.550000pt}{2.400000pt}}{0.000000pt}%
\pgfpathmoveto{\pgfqpoint{0.777355in}{0.583845in}}%
\pgfpathlineto{\pgfqpoint{0.802335in}{0.609438in}}%
\pgfpathlineto{\pgfqpoint{0.828355in}{0.636118in}}%
\pgfpathlineto{\pgfqpoint{0.880396in}{0.689441in}}%
\pgfpathlineto{\pgfqpoint{0.984478in}{0.796139in}}%
\pgfpathlineto{\pgfqpoint{1.192641in}{1.009500in}}%
\pgfpathlineto{\pgfqpoint{1.400804in}{1.222545in}}%
\pgfpathlineto{\pgfqpoint{1.608967in}{1.435938in}}%
\pgfpathlineto{\pgfqpoint{1.817131in}{1.646157in}}%
\pgfpathlineto{\pgfqpoint{2.025294in}{1.849731in}}%
\pgfpathlineto{\pgfqpoint{2.233457in}{2.071555in}}%
\pgfpathlineto{\pgfqpoint{2.441620in}{2.283261in}}%
\pgfpathlineto{\pgfqpoint{2.649784in}{2.469239in}}%
\pgfpathlineto{\pgfqpoint{2.857947in}{2.555850in}}%
\pgfpathlineto{\pgfqpoint{3.066110in}{2.446480in}}%
\pgfpathlineto{\pgfqpoint{3.274273in}{2.416742in}}%
\pgfpathlineto{\pgfqpoint{3.482437in}{2.415431in}}%
\pgfpathlineto{\pgfqpoint{3.690600in}{2.361188in}}%
\pgfpathlineto{\pgfqpoint{3.898763in}{2.252278in}}%
\pgfpathlineto{\pgfqpoint{4.106926in}{2.279349in}}%
\pgfpathlineto{\pgfqpoint{4.315090in}{2.244094in}}%
\pgfpathlineto{\pgfqpoint{4.523253in}{2.195316in}}%
\pgfpathlineto{\pgfqpoint{4.731416in}{2.225309in}}%
\pgfpathlineto{\pgfqpoint{4.939579in}{2.212133in}}%
\pgfusepath{stroke}%
\end{pgfscope}%
\begin{pgfscope}%
\pgfpathrectangle{\pgfqpoint{0.776314in}{0.582778in}}{\pgfqpoint{4.163265in}{4.267222in}}%
\pgfusepath{clip}%
\pgfsetbuttcap%
\pgfsetroundjoin%
\pgfsetlinewidth{1.505625pt}%
\definecolor{currentstroke}{rgb}{0.007843,0.619608,0.450980}%
\pgfsetstrokecolor{currentstroke}%
\pgfsetdash{{5.550000pt}{2.400000pt}}{0.000000pt}%
\pgfpathmoveto{\pgfqpoint{0.777355in}{0.583845in}}%
\pgfpathlineto{\pgfqpoint{0.802335in}{0.609448in}}%
\pgfpathlineto{\pgfqpoint{0.828355in}{0.636118in}}%
\pgfpathlineto{\pgfqpoint{0.880396in}{0.689336in}}%
\pgfpathlineto{\pgfqpoint{0.984478in}{0.795565in}}%
\pgfpathlineto{\pgfqpoint{1.192641in}{1.006867in}}%
\pgfpathlineto{\pgfqpoint{1.400804in}{1.215629in}}%
\pgfpathlineto{\pgfqpoint{1.608967in}{1.412087in}}%
\pgfpathlineto{\pgfqpoint{1.817131in}{1.589660in}}%
\pgfpathlineto{\pgfqpoint{2.025294in}{1.626677in}}%
\pgfpathlineto{\pgfqpoint{2.233457in}{1.531332in}}%
\pgfpathlineto{\pgfqpoint{2.441620in}{1.485732in}}%
\pgfpathlineto{\pgfqpoint{2.649784in}{1.397649in}}%
\pgfpathlineto{\pgfqpoint{2.857947in}{1.378783in}}%
\pgfpathlineto{\pgfqpoint{3.066110in}{1.354740in}}%
\pgfpathlineto{\pgfqpoint{3.274273in}{1.359355in}}%
\pgfpathlineto{\pgfqpoint{3.482437in}{1.342398in}}%
\pgfpathlineto{\pgfqpoint{3.690600in}{1.352589in}}%
\pgfpathlineto{\pgfqpoint{3.898763in}{1.323185in}}%
\pgfpathlineto{\pgfqpoint{4.106926in}{1.323277in}}%
\pgfpathlineto{\pgfqpoint{4.315090in}{1.337371in}}%
\pgfpathlineto{\pgfqpoint{4.523253in}{1.324961in}}%
\pgfpathlineto{\pgfqpoint{4.731416in}{1.332031in}}%
\pgfpathlineto{\pgfqpoint{4.939579in}{1.313740in}}%
\pgfusepath{stroke}%
\end{pgfscope}%
\begin{pgfscope}%
\pgfpathrectangle{\pgfqpoint{0.776314in}{0.582778in}}{\pgfqpoint{4.163265in}{4.267222in}}%
\pgfusepath{clip}%
\pgfsetbuttcap%
\pgfsetroundjoin%
\pgfsetlinewidth{1.505625pt}%
\definecolor{currentstroke}{rgb}{0.835294,0.368627,0.000000}%
\pgfsetstrokecolor{currentstroke}%
\pgfsetdash{{5.550000pt}{2.400000pt}}{0.000000pt}%
\pgfpathmoveto{\pgfqpoint{0.777355in}{0.583845in}}%
\pgfpathlineto{\pgfqpoint{0.802335in}{0.609448in}}%
\pgfpathlineto{\pgfqpoint{0.828355in}{0.636118in}}%
\pgfpathlineto{\pgfqpoint{0.880396in}{0.689451in}}%
\pgfpathlineto{\pgfqpoint{0.984478in}{0.795633in}}%
\pgfpathlineto{\pgfqpoint{1.192641in}{1.007731in}}%
\pgfpathlineto{\pgfqpoint{1.400804in}{1.194124in}}%
\pgfpathlineto{\pgfqpoint{1.608967in}{1.208427in}}%
\pgfpathlineto{\pgfqpoint{1.817131in}{1.113612in}}%
\pgfpathlineto{\pgfqpoint{2.025294in}{1.091274in}}%
\pgfpathlineto{\pgfqpoint{2.233457in}{1.073575in}}%
\pgfpathlineto{\pgfqpoint{2.441620in}{1.062446in}}%
\pgfpathlineto{\pgfqpoint{2.649784in}{1.056993in}}%
\pgfpathlineto{\pgfqpoint{2.857947in}{1.064027in}}%
\pgfpathlineto{\pgfqpoint{3.066110in}{1.048692in}}%
\pgfpathlineto{\pgfqpoint{3.274273in}{1.071000in}}%
\pgfpathlineto{\pgfqpoint{3.482437in}{1.064606in}}%
\pgfpathlineto{\pgfqpoint{3.690600in}{1.032307in}}%
\pgfpathlineto{\pgfqpoint{3.898763in}{1.043030in}}%
\pgfpathlineto{\pgfqpoint{4.106926in}{1.038748in}}%
\pgfpathlineto{\pgfqpoint{4.315090in}{1.041690in}}%
\pgfpathlineto{\pgfqpoint{4.523253in}{1.040577in}}%
\pgfpathlineto{\pgfqpoint{4.731416in}{1.030158in}}%
\pgfpathlineto{\pgfqpoint{4.939579in}{1.031785in}}%
\pgfusepath{stroke}%
\end{pgfscope}%
\begin{pgfscope}%
\pgfpathrectangle{\pgfqpoint{0.776314in}{0.582778in}}{\pgfqpoint{4.163265in}{4.267222in}}%
\pgfusepath{clip}%
\pgfsetbuttcap%
\pgfsetroundjoin%
\pgfsetlinewidth{1.505625pt}%
\definecolor{currentstroke}{rgb}{0.800000,0.470588,0.737255}%
\pgfsetstrokecolor{currentstroke}%
\pgfsetdash{{5.550000pt}{2.400000pt}}{0.000000pt}%
\pgfpathmoveto{\pgfqpoint{0.777355in}{0.583845in}}%
\pgfpathlineto{\pgfqpoint{0.802335in}{0.609448in}}%
\pgfpathlineto{\pgfqpoint{0.828355in}{0.636118in}}%
\pgfpathlineto{\pgfqpoint{0.880396in}{0.689458in}}%
\pgfpathlineto{\pgfqpoint{0.984478in}{0.796139in}}%
\pgfpathlineto{\pgfqpoint{1.192641in}{1.000312in}}%
\pgfpathlineto{\pgfqpoint{1.400804in}{1.047452in}}%
\pgfpathlineto{\pgfqpoint{1.608967in}{0.991495in}}%
\pgfpathlineto{\pgfqpoint{1.817131in}{0.971455in}}%
\pgfpathlineto{\pgfqpoint{2.025294in}{0.972879in}}%
\pgfpathlineto{\pgfqpoint{2.233457in}{0.956255in}}%
\pgfpathlineto{\pgfqpoint{2.441620in}{0.962537in}}%
\pgfpathlineto{\pgfqpoint{2.649784in}{0.956158in}}%
\pgfpathlineto{\pgfqpoint{2.857947in}{0.949440in}}%
\pgfpathlineto{\pgfqpoint{3.066110in}{0.953286in}}%
\pgfpathlineto{\pgfqpoint{3.274273in}{0.949445in}}%
\pgfpathlineto{\pgfqpoint{3.482437in}{0.944549in}}%
\pgfpathlineto{\pgfqpoint{3.690600in}{0.931889in}}%
\pgfpathlineto{\pgfqpoint{3.898763in}{0.939087in}}%
\pgfpathlineto{\pgfqpoint{4.106926in}{0.935954in}}%
\pgfpathlineto{\pgfqpoint{4.315090in}{0.929174in}}%
\pgfpathlineto{\pgfqpoint{4.523253in}{0.929605in}}%
\pgfpathlineto{\pgfqpoint{4.731416in}{0.928976in}}%
\pgfpathlineto{\pgfqpoint{4.939579in}{0.932662in}}%
\pgfusepath{stroke}%
\end{pgfscope}%
\begin{pgfscope}%
\pgfpathrectangle{\pgfqpoint{0.776314in}{0.582778in}}{\pgfqpoint{4.163265in}{4.267222in}}%
\pgfusepath{clip}%
\pgfsetbuttcap%
\pgfsetroundjoin%
\pgfsetlinewidth{1.505625pt}%
\definecolor{currentstroke}{rgb}{0.792157,0.568627,0.380392}%
\pgfsetstrokecolor{currentstroke}%
\pgfsetdash{{5.550000pt}{2.400000pt}}{0.000000pt}%
\pgfpathmoveto{\pgfqpoint{0.777355in}{0.583845in}}%
\pgfpathlineto{\pgfqpoint{0.802335in}{0.609448in}}%
\pgfpathlineto{\pgfqpoint{0.828355in}{0.636118in}}%
\pgfpathlineto{\pgfqpoint{0.880396in}{0.689458in}}%
\pgfpathlineto{\pgfqpoint{0.984478in}{0.796139in}}%
\pgfpathlineto{\pgfqpoint{1.192641in}{0.972938in}}%
\pgfpathlineto{\pgfqpoint{1.400804in}{0.913102in}}%
\pgfpathlineto{\pgfqpoint{1.608967in}{0.885575in}}%
\pgfpathlineto{\pgfqpoint{1.817131in}{0.880971in}}%
\pgfpathlineto{\pgfqpoint{2.025294in}{0.882946in}}%
\pgfpathlineto{\pgfqpoint{2.233457in}{0.883771in}}%
\pgfpathlineto{\pgfqpoint{2.441620in}{0.875462in}}%
\pgfpathlineto{\pgfqpoint{2.649784in}{0.872409in}}%
\pgfpathlineto{\pgfqpoint{2.857947in}{0.869284in}}%
\pgfpathlineto{\pgfqpoint{3.066110in}{0.869003in}}%
\pgfpathlineto{\pgfqpoint{3.274273in}{0.865002in}}%
\pgfpathlineto{\pgfqpoint{3.482437in}{0.868686in}}%
\pgfpathlineto{\pgfqpoint{3.690600in}{0.864592in}}%
\pgfpathlineto{\pgfqpoint{3.898763in}{0.875245in}}%
\pgfpathlineto{\pgfqpoint{4.106926in}{0.866623in}}%
\pgfpathlineto{\pgfqpoint{4.315090in}{0.850326in}}%
\pgfpathlineto{\pgfqpoint{4.523253in}{0.869836in}}%
\pgfpathlineto{\pgfqpoint{4.731416in}{0.858207in}}%
\pgfpathlineto{\pgfqpoint{4.939579in}{0.856297in}}%
\pgfusepath{stroke}%
\end{pgfscope}%
\begin{pgfscope}%
\pgfsetrectcap%
\pgfsetmiterjoin%
\pgfsetlinewidth{0.803000pt}%
\definecolor{currentstroke}{rgb}{0.000000,0.000000,0.000000}%
\pgfsetstrokecolor{currentstroke}%
\pgfsetdash{}{0pt}%
\pgfpathmoveto{\pgfqpoint{0.776314in}{0.582778in}}%
\pgfpathlineto{\pgfqpoint{0.776314in}{4.850000in}}%
\pgfusepath{stroke}%
\end{pgfscope}%
\begin{pgfscope}%
\pgfsetrectcap%
\pgfsetmiterjoin%
\pgfsetlinewidth{0.803000pt}%
\definecolor{currentstroke}{rgb}{0.000000,0.000000,0.000000}%
\pgfsetstrokecolor{currentstroke}%
\pgfsetdash{}{0pt}%
\pgfpathmoveto{\pgfqpoint{4.939579in}{0.582778in}}%
\pgfpathlineto{\pgfqpoint{4.939579in}{4.850000in}}%
\pgfusepath{stroke}%
\end{pgfscope}%
\begin{pgfscope}%
\pgfsetrectcap%
\pgfsetmiterjoin%
\pgfsetlinewidth{0.803000pt}%
\definecolor{currentstroke}{rgb}{0.000000,0.000000,0.000000}%
\pgfsetstrokecolor{currentstroke}%
\pgfsetdash{}{0pt}%
\pgfpathmoveto{\pgfqpoint{0.776314in}{0.582778in}}%
\pgfpathlineto{\pgfqpoint{4.939579in}{0.582778in}}%
\pgfusepath{stroke}%
\end{pgfscope}%
\begin{pgfscope}%
\pgfsetrectcap%
\pgfsetmiterjoin%
\pgfsetlinewidth{0.803000pt}%
\definecolor{currentstroke}{rgb}{0.000000,0.000000,0.000000}%
\pgfsetstrokecolor{currentstroke}%
\pgfsetdash{}{0pt}%
\pgfpathmoveto{\pgfqpoint{0.776314in}{4.850000in}}%
\pgfpathlineto{\pgfqpoint{4.939579in}{4.850000in}}%
\pgfusepath{stroke}%
\end{pgfscope}%
\begin{pgfscope}%
\pgfsetbuttcap%
\pgfsetmiterjoin%
\definecolor{currentfill}{rgb}{1.000000,1.000000,1.000000}%
\pgfsetfillcolor{currentfill}%
\pgfsetfillopacity{0.800000}%
\pgfsetlinewidth{1.003750pt}%
\definecolor{currentstroke}{rgb}{0.800000,0.800000,0.800000}%
\pgfsetstrokecolor{currentstroke}%
\pgfsetstrokeopacity{0.800000}%
\pgfsetdash{}{0pt}%
\pgfpathmoveto{\pgfqpoint{0.873537in}{3.311888in}}%
\pgfpathlineto{\pgfqpoint{1.711699in}{3.311888in}}%
\pgfpathquadraticcurveto{\pgfqpoint{1.739476in}{3.311888in}}{\pgfqpoint{1.739476in}{3.339666in}}%
\pgfpathlineto{\pgfqpoint{1.739476in}{4.752778in}}%
\pgfpathquadraticcurveto{\pgfqpoint{1.739476in}{4.780556in}}{\pgfqpoint{1.711699in}{4.780556in}}%
\pgfpathlineto{\pgfqpoint{0.873537in}{4.780556in}}%
\pgfpathquadraticcurveto{\pgfqpoint{0.845759in}{4.780556in}}{\pgfqpoint{0.845759in}{4.752778in}}%
\pgfpathlineto{\pgfqpoint{0.845759in}{3.339666in}}%
\pgfpathquadraticcurveto{\pgfqpoint{0.845759in}{3.311888in}}{\pgfqpoint{0.873537in}{3.311888in}}%
\pgfpathlineto{\pgfqpoint{0.873537in}{3.311888in}}%
\pgfpathclose%
\pgfusepath{stroke,fill}%
\end{pgfscope}%
\begin{pgfscope}%
\definecolor{textcolor}{rgb}{0.000000,0.000000,0.000000}%
\pgfsetstrokecolor{textcolor}%
\pgfsetfillcolor{textcolor}%
\pgftext[x=0.901314in,y=4.619477in,left,base]{\color{textcolor}\sffamily\fontsize{10.000000}{12.000000}\selectfont node count}%
\end{pgfscope}%
\begin{pgfscope}%
\pgfsetrectcap%
\pgfsetroundjoin%
\pgfsetlinewidth{1.505625pt}%
\definecolor{currentstroke}{rgb}{0.003922,0.450980,0.698039}%
\pgfsetstrokecolor{currentstroke}%
\pgfsetdash{}{0pt}%
\pgfpathmoveto{\pgfqpoint{1.009808in}{4.464231in}}%
\pgfpathlineto{\pgfqpoint{1.148697in}{4.464231in}}%
\pgfpathlineto{\pgfqpoint{1.287586in}{4.464231in}}%
\pgfusepath{stroke}%
\end{pgfscope}%
\begin{pgfscope}%
\definecolor{textcolor}{rgb}{0.000000,0.000000,0.000000}%
\pgfsetstrokecolor{textcolor}%
\pgfsetfillcolor{textcolor}%
\pgftext[x=1.398697in,y=4.415620in,left,base]{\color{textcolor}\sffamily\fontsize{10.000000}{12.000000}\selectfont 1}%
\end{pgfscope}%
\begin{pgfscope}%
\pgfsetrectcap%
\pgfsetroundjoin%
\pgfsetlinewidth{1.505625pt}%
\definecolor{currentstroke}{rgb}{0.870588,0.560784,0.019608}%
\pgfsetstrokecolor{currentstroke}%
\pgfsetdash{}{0pt}%
\pgfpathmoveto{\pgfqpoint{1.009808in}{4.260374in}}%
\pgfpathlineto{\pgfqpoint{1.148697in}{4.260374in}}%
\pgfpathlineto{\pgfqpoint{1.287586in}{4.260374in}}%
\pgfusepath{stroke}%
\end{pgfscope}%
\begin{pgfscope}%
\definecolor{textcolor}{rgb}{0.000000,0.000000,0.000000}%
\pgfsetstrokecolor{textcolor}%
\pgfsetfillcolor{textcolor}%
\pgftext[x=1.398697in,y=4.211762in,left,base]{\color{textcolor}\sffamily\fontsize{10.000000}{12.000000}\selectfont 2}%
\end{pgfscope}%
\begin{pgfscope}%
\pgfsetrectcap%
\pgfsetroundjoin%
\pgfsetlinewidth{1.505625pt}%
\definecolor{currentstroke}{rgb}{0.007843,0.619608,0.450980}%
\pgfsetstrokecolor{currentstroke}%
\pgfsetdash{}{0pt}%
\pgfpathmoveto{\pgfqpoint{1.009808in}{4.056516in}}%
\pgfpathlineto{\pgfqpoint{1.148697in}{4.056516in}}%
\pgfpathlineto{\pgfqpoint{1.287586in}{4.056516in}}%
\pgfusepath{stroke}%
\end{pgfscope}%
\begin{pgfscope}%
\definecolor{textcolor}{rgb}{0.000000,0.000000,0.000000}%
\pgfsetstrokecolor{textcolor}%
\pgfsetfillcolor{textcolor}%
\pgftext[x=1.398697in,y=4.007905in,left,base]{\color{textcolor}\sffamily\fontsize{10.000000}{12.000000}\selectfont 4}%
\end{pgfscope}%
\begin{pgfscope}%
\pgfsetrectcap%
\pgfsetroundjoin%
\pgfsetlinewidth{1.505625pt}%
\definecolor{currentstroke}{rgb}{0.835294,0.368627,0.000000}%
\pgfsetstrokecolor{currentstroke}%
\pgfsetdash{}{0pt}%
\pgfpathmoveto{\pgfqpoint{1.009808in}{3.852659in}}%
\pgfpathlineto{\pgfqpoint{1.148697in}{3.852659in}}%
\pgfpathlineto{\pgfqpoint{1.287586in}{3.852659in}}%
\pgfusepath{stroke}%
\end{pgfscope}%
\begin{pgfscope}%
\definecolor{textcolor}{rgb}{0.000000,0.000000,0.000000}%
\pgfsetstrokecolor{textcolor}%
\pgfsetfillcolor{textcolor}%
\pgftext[x=1.398697in,y=3.804048in,left,base]{\color{textcolor}\sffamily\fontsize{10.000000}{12.000000}\selectfont 7}%
\end{pgfscope}%
\begin{pgfscope}%
\pgfsetrectcap%
\pgfsetroundjoin%
\pgfsetlinewidth{1.505625pt}%
\definecolor{currentstroke}{rgb}{0.800000,0.470588,0.737255}%
\pgfsetstrokecolor{currentstroke}%
\pgfsetdash{}{0pt}%
\pgfpathmoveto{\pgfqpoint{1.009808in}{3.648802in}}%
\pgfpathlineto{\pgfqpoint{1.148697in}{3.648802in}}%
\pgfpathlineto{\pgfqpoint{1.287586in}{3.648802in}}%
\pgfusepath{stroke}%
\end{pgfscope}%
\begin{pgfscope}%
\definecolor{textcolor}{rgb}{0.000000,0.000000,0.000000}%
\pgfsetstrokecolor{textcolor}%
\pgfsetfillcolor{textcolor}%
\pgftext[x=1.398697in,y=3.600191in,left,base]{\color{textcolor}\sffamily\fontsize{10.000000}{12.000000}\selectfont 10}%
\end{pgfscope}%
\begin{pgfscope}%
\pgfsetrectcap%
\pgfsetroundjoin%
\pgfsetlinewidth{1.505625pt}%
\definecolor{currentstroke}{rgb}{0.792157,0.568627,0.380392}%
\pgfsetstrokecolor{currentstroke}%
\pgfsetdash{}{0pt}%
\pgfpathmoveto{\pgfqpoint{1.009808in}{3.444945in}}%
\pgfpathlineto{\pgfqpoint{1.148697in}{3.444945in}}%
\pgfpathlineto{\pgfqpoint{1.287586in}{3.444945in}}%
\pgfusepath{stroke}%
\end{pgfscope}%
\begin{pgfscope}%
\definecolor{textcolor}{rgb}{0.000000,0.000000,0.000000}%
\pgfsetstrokecolor{textcolor}%
\pgfsetfillcolor{textcolor}%
\pgftext[x=1.398697in,y=3.396334in,left,base]{\color{textcolor}\sffamily\fontsize{10.000000}{12.000000}\selectfont 13}%
\end{pgfscope}%
\end{pgfpicture}%
\makeatother%
\endgroup%
}
\caption{Benchmarking of goodput for varying throughputs and node counts.}
\label{throughputgoodputnodes}
\end{figure}

\begin{figure}[h!]
\centering
\resizebox{.6\textwidth}{!}{%% Creator: Matplotlib, PGF backend
%%
%% To include the figure in your LaTeX document, write
%%   \input{<filename>.pgf}
%%
%% Make sure the required packages are loaded in your preamble
%%   \usepackage{pgf}
%%
%% Also ensure that all the required font packages are loaded; for instance,
%% the lmodern package is sometimes necessary when using math font.
%%   \usepackage{lmodern}
%%
%% Figures using additional raster images can only be included by \input if
%% they are in the same directory as the main LaTeX file. For loading figures
%% from other directories you can use the `import` package
%%   \usepackage{import}
%%
%% and then include the figures with
%%   \import{<path to file>}{<filename>.pgf}
%%
%% Matplotlib used the following preamble
%%   
%%   \usepackage{fontspec}
%%   \setmainfont{DejaVuSerif.ttf}[Path=\detokenize{/opt/homebrew/lib/python3.10/site-packages/matplotlib/mpl-data/fonts/ttf/}]
%%   \setsansfont{DejaVuSans.ttf}[Path=\detokenize{/opt/homebrew/lib/python3.10/site-packages/matplotlib/mpl-data/fonts/ttf/}]
%%   \setmonofont{DejaVuSansMono.ttf}[Path=\detokenize{/opt/homebrew/lib/python3.10/site-packages/matplotlib/mpl-data/fonts/ttf/}]
%%   \makeatletter\@ifpackageloaded{underscore}{}{\usepackage[strings]{underscore}}\makeatother
%%
\begingroup%
\makeatletter%
\begin{pgfpicture}%
\pgfpathrectangle{\pgfpointorigin}{\pgfqpoint{5.580000in}{5.000000in}}%
\pgfusepath{use as bounding box, clip}%
\begin{pgfscope}%
\pgfsetbuttcap%
\pgfsetmiterjoin%
\definecolor{currentfill}{rgb}{1.000000,1.000000,1.000000}%
\pgfsetfillcolor{currentfill}%
\pgfsetlinewidth{0.000000pt}%
\definecolor{currentstroke}{rgb}{1.000000,1.000000,1.000000}%
\pgfsetstrokecolor{currentstroke}%
\pgfsetdash{}{0pt}%
\pgfpathmoveto{\pgfqpoint{0.000000in}{0.000000in}}%
\pgfpathlineto{\pgfqpoint{5.580000in}{0.000000in}}%
\pgfpathlineto{\pgfqpoint{5.580000in}{5.000000in}}%
\pgfpathlineto{\pgfqpoint{0.000000in}{5.000000in}}%
\pgfpathlineto{\pgfqpoint{0.000000in}{0.000000in}}%
\pgfpathclose%
\pgfusepath{fill}%
\end{pgfscope}%
\begin{pgfscope}%
\pgfsetbuttcap%
\pgfsetmiterjoin%
\definecolor{currentfill}{rgb}{1.000000,1.000000,1.000000}%
\pgfsetfillcolor{currentfill}%
\pgfsetlinewidth{0.000000pt}%
\definecolor{currentstroke}{rgb}{0.000000,0.000000,0.000000}%
\pgfsetstrokecolor{currentstroke}%
\pgfsetstrokeopacity{0.000000}%
\pgfsetdash{}{0pt}%
\pgfpathmoveto{\pgfqpoint{0.741055in}{0.582778in}}%
\pgfpathlineto{\pgfqpoint{4.936930in}{0.582778in}}%
\pgfpathlineto{\pgfqpoint{4.936930in}{4.850000in}}%
\pgfpathlineto{\pgfqpoint{0.741055in}{4.850000in}}%
\pgfpathlineto{\pgfqpoint{0.741055in}{0.582778in}}%
\pgfpathclose%
\pgfusepath{fill}%
\end{pgfscope}%
\begin{pgfscope}%
\pgfpathrectangle{\pgfqpoint{0.741055in}{0.582778in}}{\pgfqpoint{4.195875in}{4.267222in}}%
\pgfusepath{clip}%
\pgfsetbuttcap%
\pgfsetroundjoin%
\definecolor{currentfill}{rgb}{0.003922,0.450980,0.698039}%
\pgfsetfillcolor{currentfill}%
\pgfsetfillopacity{0.800000}%
\pgfsetlinewidth{1.003750pt}%
\definecolor{currentstroke}{rgb}{0.003922,0.450980,0.698039}%
\pgfsetstrokecolor{currentstroke}%
\pgfsetstrokeopacity{0.800000}%
\pgfsetdash{}{0pt}%
\pgfsys@defobject{currentmarker}{\pgfqpoint{-0.041667in}{-0.041667in}}{\pgfqpoint{0.041667in}{0.041667in}}{%
\pgfpathmoveto{\pgfqpoint{0.000000in}{-0.041667in}}%
\pgfpathcurveto{\pgfqpoint{0.011050in}{-0.041667in}}{\pgfqpoint{0.021649in}{-0.037276in}}{\pgfqpoint{0.029463in}{-0.029463in}}%
\pgfpathcurveto{\pgfqpoint{0.037276in}{-0.021649in}}{\pgfqpoint{0.041667in}{-0.011050in}}{\pgfqpoint{0.041667in}{0.000000in}}%
\pgfpathcurveto{\pgfqpoint{0.041667in}{0.011050in}}{\pgfqpoint{0.037276in}{0.021649in}}{\pgfqpoint{0.029463in}{0.029463in}}%
\pgfpathcurveto{\pgfqpoint{0.021649in}{0.037276in}}{\pgfqpoint{0.011050in}{0.041667in}}{\pgfqpoint{0.000000in}{0.041667in}}%
\pgfpathcurveto{\pgfqpoint{-0.011050in}{0.041667in}}{\pgfqpoint{-0.021649in}{0.037276in}}{\pgfqpoint{-0.029463in}{0.029463in}}%
\pgfpathcurveto{\pgfqpoint{-0.037276in}{0.021649in}}{\pgfqpoint{-0.041667in}{0.011050in}}{\pgfqpoint{-0.041667in}{0.000000in}}%
\pgfpathcurveto{\pgfqpoint{-0.041667in}{-0.011050in}}{\pgfqpoint{-0.037276in}{-0.021649in}}{\pgfqpoint{-0.029463in}{-0.029463in}}%
\pgfpathcurveto{\pgfqpoint{-0.021649in}{-0.037276in}}{\pgfqpoint{-0.011050in}{-0.041667in}}{\pgfqpoint{0.000000in}{-0.041667in}}%
\pgfpathlineto{\pgfqpoint{0.000000in}{-0.041667in}}%
\pgfpathclose%
\pgfusepath{stroke,fill}%
}%
\begin{pgfscope}%
\pgfsys@transformshift{3.560153in}{2.328415in}%
\pgfsys@useobject{currentmarker}{}%
\end{pgfscope}%
\begin{pgfscope}%
\pgfsys@transformshift{3.595760in}{2.198275in}%
\pgfsys@useobject{currentmarker}{}%
\end{pgfscope}%
\begin{pgfscope}%
\pgfsys@transformshift{1.376591in}{1.444694in}%
\pgfsys@useobject{currentmarker}{}%
\end{pgfscope}%
\begin{pgfscope}%
\pgfsys@transformshift{2.935179in}{1.554520in}%
\pgfsys@useobject{currentmarker}{}%
\end{pgfscope}%
\begin{pgfscope}%
\pgfsys@transformshift{3.257509in}{1.724029in}%
\pgfsys@useobject{currentmarker}{}%
\end{pgfscope}%
\begin{pgfscope}%
\pgfsys@transformshift{3.266033in}{1.687371in}%
\pgfsys@useobject{currentmarker}{}%
\end{pgfscope}%
\begin{pgfscope}%
\pgfsys@transformshift{3.629310in}{3.690711in}%
\pgfsys@useobject{currentmarker}{}%
\end{pgfscope}%
\begin{pgfscope}%
\pgfsys@transformshift{0.931777in}{1.492764in}%
\pgfsys@useobject{currentmarker}{}%
\end{pgfscope}%
\begin{pgfscope}%
\pgfsys@transformshift{1.376591in}{1.353257in}%
\pgfsys@useobject{currentmarker}{}%
\end{pgfscope}%
\begin{pgfscope}%
\pgfsys@transformshift{2.939197in}{1.520495in}%
\pgfsys@useobject{currentmarker}{}%
\end{pgfscope}%
\begin{pgfscope}%
\pgfsys@transformshift{3.578862in}{4.010160in}%
\pgfsys@useobject{currentmarker}{}%
\end{pgfscope}%
\begin{pgfscope}%
\pgfsys@transformshift{3.592988in}{2.168711in}%
\pgfsys@useobject{currentmarker}{}%
\end{pgfscope}%
\begin{pgfscope}%
\pgfsys@transformshift{3.259899in}{1.712451in}%
\pgfsys@useobject{currentmarker}{}%
\end{pgfscope}%
\begin{pgfscope}%
\pgfsys@transformshift{3.614807in}{3.727122in}%
\pgfsys@useobject{currentmarker}{}%
\end{pgfscope}%
\begin{pgfscope}%
\pgfsys@transformshift{2.924366in}{1.662792in}%
\pgfsys@useobject{currentmarker}{}%
\end{pgfscope}%
\begin{pgfscope}%
\pgfsys@transformshift{3.251687in}{2.098035in}%
\pgfsys@useobject{currentmarker}{}%
\end{pgfscope}%
\begin{pgfscope}%
\pgfsys@transformshift{2.930333in}{1.474366in}%
\pgfsys@useobject{currentmarker}{}%
\end{pgfscope}%
\begin{pgfscope}%
\pgfsys@transformshift{3.561820in}{2.933556in}%
\pgfsys@useobject{currentmarker}{}%
\end{pgfscope}%
\begin{pgfscope}%
\pgfsys@transformshift{3.830302in}{4.395757in}%
\pgfsys@useobject{currentmarker}{}%
\end{pgfscope}%
\begin{pgfscope}%
\pgfsys@transformshift{3.246156in}{1.765337in}%
\pgfsys@useobject{currentmarker}{}%
\end{pgfscope}%
\begin{pgfscope}%
\pgfsys@transformshift{3.573838in}{2.876416in}%
\pgfsys@useobject{currentmarker}{}%
\end{pgfscope}%
\begin{pgfscope}%
\pgfsys@transformshift{3.262017in}{1.847874in}%
\pgfsys@useobject{currentmarker}{}%
\end{pgfscope}%
\begin{pgfscope}%
\pgfsys@transformshift{1.376591in}{1.915993in}%
\pgfsys@useobject{currentmarker}{}%
\end{pgfscope}%
\begin{pgfscope}%
\pgfsys@transformshift{3.850351in}{4.291015in}%
\pgfsys@useobject{currentmarker}{}%
\end{pgfscope}%
\begin{pgfscope}%
\pgfsys@transformshift{1.376591in}{1.803431in}%
\pgfsys@useobject{currentmarker}{}%
\end{pgfscope}%
\begin{pgfscope}%
\pgfsys@transformshift{3.564519in}{2.711331in}%
\pgfsys@useobject{currentmarker}{}%
\end{pgfscope}%
\begin{pgfscope}%
\pgfsys@transformshift{2.939197in}{1.667073in}%
\pgfsys@useobject{currentmarker}{}%
\end{pgfscope}%
\begin{pgfscope}%
\pgfsys@transformshift{1.376591in}{1.487225in}%
\pgfsys@useobject{currentmarker}{}%
\end{pgfscope}%
\begin{pgfscope}%
\pgfsys@transformshift{3.842597in}{4.610884in}%
\pgfsys@useobject{currentmarker}{}%
\end{pgfscope}%
\begin{pgfscope}%
\pgfsys@transformshift{1.376591in}{1.826511in}%
\pgfsys@useobject{currentmarker}{}%
\end{pgfscope}%
\begin{pgfscope}%
\pgfsys@transformshift{3.263447in}{1.681454in}%
\pgfsys@useobject{currentmarker}{}%
\end{pgfscope}%
\begin{pgfscope}%
\pgfsys@transformshift{3.582457in}{2.086147in}%
\pgfsys@useobject{currentmarker}{}%
\end{pgfscope}%
\begin{pgfscope}%
\pgfsys@transformshift{3.890849in}{2.958194in}%
\pgfsys@useobject{currentmarker}{}%
\end{pgfscope}%
\begin{pgfscope}%
\pgfsys@transformshift{3.582385in}{2.235769in}%
\pgfsys@useobject{currentmarker}{}%
\end{pgfscope}%
\begin{pgfscope}%
\pgfsys@transformshift{3.905614in}{2.752897in}%
\pgfsys@useobject{currentmarker}{}%
\end{pgfscope}%
\begin{pgfscope}%
\pgfsys@transformshift{1.376591in}{1.668745in}%
\pgfsys@useobject{currentmarker}{}%
\end{pgfscope}%
\begin{pgfscope}%
\pgfsys@transformshift{3.263591in}{1.649884in}%
\pgfsys@useobject{currentmarker}{}%
\end{pgfscope}%
\begin{pgfscope}%
\pgfsys@transformshift{2.939197in}{1.606684in}%
\pgfsys@useobject{currentmarker}{}%
\end{pgfscope}%
\begin{pgfscope}%
\pgfsys@transformshift{4.174805in}{4.556089in}%
\pgfsys@useobject{currentmarker}{}%
\end{pgfscope}%
\begin{pgfscope}%
\pgfsys@transformshift{3.610209in}{2.384926in}%
\pgfsys@useobject{currentmarker}{}%
\end{pgfscope}%
\begin{pgfscope}%
\pgfsys@transformshift{3.888210in}{3.239910in}%
\pgfsys@useobject{currentmarker}{}%
\end{pgfscope}%
\begin{pgfscope}%
\pgfsys@transformshift{1.376591in}{1.618215in}%
\pgfsys@useobject{currentmarker}{}%
\end{pgfscope}%
\begin{pgfscope}%
\pgfsys@transformshift{2.929060in}{1.504304in}%
\pgfsys@useobject{currentmarker}{}%
\end{pgfscope}%
\begin{pgfscope}%
\pgfsys@transformshift{3.256990in}{1.654880in}%
\pgfsys@useobject{currentmarker}{}%
\end{pgfscope}%
\begin{pgfscope}%
\pgfsys@transformshift{2.931352in}{1.472700in}%
\pgfsys@useobject{currentmarker}{}%
\end{pgfscope}%
\begin{pgfscope}%
\pgfsys@transformshift{3.612175in}{1.130530in}%
\pgfsys@useobject{currentmarker}{}%
\end{pgfscope}%
\begin{pgfscope}%
\pgfsys@transformshift{3.610772in}{1.122087in}%
\pgfsys@useobject{currentmarker}{}%
\end{pgfscope}%
\begin{pgfscope}%
\pgfsys@transformshift{4.273688in}{1.580557in}%
\pgfsys@useobject{currentmarker}{}%
\end{pgfscope}%
\begin{pgfscope}%
\pgfsys@transformshift{4.420505in}{2.586079in}%
\pgfsys@useobject{currentmarker}{}%
\end{pgfscope}%
\begin{pgfscope}%
\pgfsys@transformshift{3.275304in}{1.056013in}%
\pgfsys@useobject{currentmarker}{}%
\end{pgfscope}%
\begin{pgfscope}%
\pgfsys@transformshift{3.945846in}{1.235651in}%
\pgfsys@useobject{currentmarker}{}%
\end{pgfscope}%
\begin{pgfscope}%
\pgfsys@transformshift{4.416610in}{2.578698in}%
\pgfsys@useobject{currentmarker}{}%
\end{pgfscope}%
\begin{pgfscope}%
\pgfsys@transformshift{2.939197in}{1.017465in}%
\pgfsys@useobject{currentmarker}{}%
\end{pgfscope}%
\begin{pgfscope}%
\pgfsys@transformshift{1.376591in}{1.082571in}%
\pgfsys@useobject{currentmarker}{}%
\end{pgfscope}%
\begin{pgfscope}%
\pgfsys@transformshift{3.610129in}{1.130999in}%
\pgfsys@useobject{currentmarker}{}%
\end{pgfscope}%
\begin{pgfscope}%
\pgfsys@transformshift{3.274965in}{1.048586in}%
\pgfsys@useobject{currentmarker}{}%
\end{pgfscope}%
\begin{pgfscope}%
\pgfsys@transformshift{2.939197in}{1.021243in}%
\pgfsys@useobject{currentmarker}{}%
\end{pgfscope}%
\begin{pgfscope}%
\pgfsys@transformshift{3.275686in}{1.052494in}%
\pgfsys@useobject{currentmarker}{}%
\end{pgfscope}%
\begin{pgfscope}%
\pgfsys@transformshift{4.272844in}{1.705030in}%
\pgfsys@useobject{currentmarker}{}%
\end{pgfscope}%
\begin{pgfscope}%
\pgfsys@transformshift{3.943818in}{1.256339in}%
\pgfsys@useobject{currentmarker}{}%
\end{pgfscope}%
\begin{pgfscope}%
\pgfsys@transformshift{4.265555in}{1.611226in}%
\pgfsys@useobject{currentmarker}{}%
\end{pgfscope}%
\begin{pgfscope}%
\pgfsys@transformshift{4.420302in}{2.422570in}%
\pgfsys@useobject{currentmarker}{}%
\end{pgfscope}%
\begin{pgfscope}%
\pgfsys@transformshift{1.376591in}{1.066231in}%
\pgfsys@useobject{currentmarker}{}%
\end{pgfscope}%
\begin{pgfscope}%
\pgfsys@transformshift{1.376591in}{1.131929in}%
\pgfsys@useobject{currentmarker}{}%
\end{pgfscope}%
\begin{pgfscope}%
\pgfsys@transformshift{3.941253in}{1.287928in}%
\pgfsys@useobject{currentmarker}{}%
\end{pgfscope}%
\begin{pgfscope}%
\pgfsys@transformshift{2.939197in}{1.016863in}%
\pgfsys@useobject{currentmarker}{}%
\end{pgfscope}%
\begin{pgfscope}%
\pgfsys@transformshift{4.708366in}{2.368314in}%
\pgfsys@useobject{currentmarker}{}%
\end{pgfscope}%
\begin{pgfscope}%
\pgfsys@transformshift{3.275152in}{1.032169in}%
\pgfsys@useobject{currentmarker}{}%
\end{pgfscope}%
\begin{pgfscope}%
\pgfsys@transformshift{4.607117in}{1.642004in}%
\pgfsys@useobject{currentmarker}{}%
\end{pgfscope}%
\begin{pgfscope}%
\pgfsys@transformshift{3.946036in}{1.108067in}%
\pgfsys@useobject{currentmarker}{}%
\end{pgfscope}%
\begin{pgfscope}%
\pgfsys@transformshift{2.939197in}{0.984527in}%
\pgfsys@useobject{currentmarker}{}%
\end{pgfscope}%
\begin{pgfscope}%
\pgfsys@transformshift{1.376591in}{1.110345in}%
\pgfsys@useobject{currentmarker}{}%
\end{pgfscope}%
\begin{pgfscope}%
\pgfsys@transformshift{3.608990in}{1.050171in}%
\pgfsys@useobject{currentmarker}{}%
\end{pgfscope}%
\begin{pgfscope}%
\pgfsys@transformshift{4.468764in}{1.485579in}%
\pgfsys@useobject{currentmarker}{}%
\end{pgfscope}%
\begin{pgfscope}%
\pgfsys@transformshift{2.939197in}{1.004110in}%
\pgfsys@useobject{currentmarker}{}%
\end{pgfscope}%
\begin{pgfscope}%
\pgfsys@transformshift{4.709062in}{3.546529in}%
\pgfsys@useobject{currentmarker}{}%
\end{pgfscope}%
\begin{pgfscope}%
\pgfsys@transformshift{4.703281in}{2.246796in}%
\pgfsys@useobject{currentmarker}{}%
\end{pgfscope}%
\begin{pgfscope}%
\pgfsys@transformshift{3.275686in}{1.014933in}%
\pgfsys@useobject{currentmarker}{}%
\end{pgfscope}%
\begin{pgfscope}%
\pgfsys@transformshift{4.712803in}{3.272199in}%
\pgfsys@useobject{currentmarker}{}%
\end{pgfscope}%
\begin{pgfscope}%
\pgfsys@transformshift{4.704308in}{3.563534in}%
\pgfsys@useobject{currentmarker}{}%
\end{pgfscope}%
\begin{pgfscope}%
\pgfsys@transformshift{4.608899in}{1.766523in}%
\pgfsys@useobject{currentmarker}{}%
\end{pgfscope}%
\begin{pgfscope}%
\pgfsys@transformshift{3.609172in}{1.060944in}%
\pgfsys@useobject{currentmarker}{}%
\end{pgfscope}%
\begin{pgfscope}%
\pgfsys@transformshift{4.708398in}{2.272344in}%
\pgfsys@useobject{currentmarker}{}%
\end{pgfscope}%
\begin{pgfscope}%
\pgfsys@transformshift{1.376591in}{1.128480in}%
\pgfsys@useobject{currentmarker}{}%
\end{pgfscope}%
\begin{pgfscope}%
\pgfsys@transformshift{4.280600in}{1.292696in}%
\pgfsys@useobject{currentmarker}{}%
\end{pgfscope}%
\begin{pgfscope}%
\pgfsys@transformshift{4.280712in}{1.294701in}%
\pgfsys@useobject{currentmarker}{}%
\end{pgfscope}%
\begin{pgfscope}%
\pgfsys@transformshift{3.275525in}{1.020657in}%
\pgfsys@useobject{currentmarker}{}%
\end{pgfscope}%
\begin{pgfscope}%
\pgfsys@transformshift{1.376591in}{1.057956in}%
\pgfsys@useobject{currentmarker}{}%
\end{pgfscope}%
\begin{pgfscope}%
\pgfsys@transformshift{4.472644in}{1.500057in}%
\pgfsys@useobject{currentmarker}{}%
\end{pgfscope}%
\begin{pgfscope}%
\pgfsys@transformshift{4.604403in}{1.891251in}%
\pgfsys@useobject{currentmarker}{}%
\end{pgfscope}%
\begin{pgfscope}%
\pgfsys@transformshift{4.277356in}{1.289091in}%
\pgfsys@useobject{currentmarker}{}%
\end{pgfscope}%
\begin{pgfscope}%
\pgfsys@transformshift{3.947250in}{1.116803in}%
\pgfsys@useobject{currentmarker}{}%
\end{pgfscope}%
\begin{pgfscope}%
\pgfsys@transformshift{3.944944in}{1.137212in}%
\pgfsys@useobject{currentmarker}{}%
\end{pgfscope}%
\begin{pgfscope}%
\pgfsys@transformshift{4.469358in}{1.505824in}%
\pgfsys@useobject{currentmarker}{}%
\end{pgfscope}%
\begin{pgfscope}%
\pgfsys@transformshift{3.612175in}{1.058100in}%
\pgfsys@useobject{currentmarker}{}%
\end{pgfscope}%
\begin{pgfscope}%
\pgfsys@transformshift{2.939197in}{1.006297in}%
\pgfsys@useobject{currentmarker}{}%
\end{pgfscope}%
\begin{pgfscope}%
\pgfsys@transformshift{2.939197in}{0.854557in}%
\pgfsys@useobject{currentmarker}{}%
\end{pgfscope}%
\begin{pgfscope}%
\pgfsys@transformshift{4.469443in}{1.413128in}%
\pgfsys@useobject{currentmarker}{}%
\end{pgfscope}%
\begin{pgfscope}%
\pgfsys@transformshift{4.276999in}{1.178991in}%
\pgfsys@useobject{currentmarker}{}%
\end{pgfscope}%
\begin{pgfscope}%
\pgfsys@transformshift{1.376591in}{0.967704in}%
\pgfsys@useobject{currentmarker}{}%
\end{pgfscope}%
\begin{pgfscope}%
\pgfsys@transformshift{4.740817in}{2.977762in}%
\pgfsys@useobject{currentmarker}{}%
\end{pgfscope}%
\begin{pgfscope}%
\pgfsys@transformshift{4.720404in}{1.767725in}%
\pgfsys@useobject{currentmarker}{}%
\end{pgfscope}%
\begin{pgfscope}%
\pgfsys@transformshift{3.946753in}{0.989206in}%
\pgfsys@useobject{currentmarker}{}%
\end{pgfscope}%
\begin{pgfscope}%
\pgfsys@transformshift{1.376591in}{0.902836in}%
\pgfsys@useobject{currentmarker}{}%
\end{pgfscope}%
\begin{pgfscope}%
\pgfsys@transformshift{3.612175in}{0.909008in}%
\pgfsys@useobject{currentmarker}{}%
\end{pgfscope}%
\begin{pgfscope}%
\pgfsys@transformshift{4.278908in}{1.146766in}%
\pgfsys@useobject{currentmarker}{}%
\end{pgfscope}%
\begin{pgfscope}%
\pgfsys@transformshift{4.473348in}{1.418824in}%
\pgfsys@useobject{currentmarker}{}%
\end{pgfscope}%
\begin{pgfscope}%
\pgfsys@transformshift{2.939197in}{0.869043in}%
\pgfsys@useobject{currentmarker}{}%
\end{pgfscope}%
\begin{pgfscope}%
\pgfsys@transformshift{4.746209in}{2.467162in}%
\pgfsys@useobject{currentmarker}{}%
\end{pgfscope}%
\begin{pgfscope}%
\pgfsys@transformshift{4.610908in}{1.666672in}%
\pgfsys@useobject{currentmarker}{}%
\end{pgfscope}%
\begin{pgfscope}%
\pgfsys@transformshift{3.274864in}{0.892089in}%
\pgfsys@useobject{currentmarker}{}%
\end{pgfscope}%
\begin{pgfscope}%
\pgfsys@transformshift{4.700925in}{2.722549in}%
\pgfsys@useobject{currentmarker}{}%
\end{pgfscope}%
\begin{pgfscope}%
\pgfsys@transformshift{3.275686in}{0.867862in}%
\pgfsys@useobject{currentmarker}{}%
\end{pgfscope}%
\begin{pgfscope}%
\pgfsys@transformshift{3.273330in}{0.902839in}%
\pgfsys@useobject{currentmarker}{}%
\end{pgfscope}%
\begin{pgfscope}%
\pgfsys@transformshift{3.948043in}{1.001493in}%
\pgfsys@useobject{currentmarker}{}%
\end{pgfscope}%
\begin{pgfscope}%
\pgfsys@transformshift{1.376591in}{1.037736in}%
\pgfsys@useobject{currentmarker}{}%
\end{pgfscope}%
\begin{pgfscope}%
\pgfsys@transformshift{4.741753in}{2.547478in}%
\pgfsys@useobject{currentmarker}{}%
\end{pgfscope}%
\begin{pgfscope}%
\pgfsys@transformshift{3.611776in}{0.913335in}%
\pgfsys@useobject{currentmarker}{}%
\end{pgfscope}%
\begin{pgfscope}%
\pgfsys@transformshift{3.612175in}{0.924878in}%
\pgfsys@useobject{currentmarker}{}%
\end{pgfscope}%
\begin{pgfscope}%
\pgfsys@transformshift{4.285153in}{1.139040in}%
\pgfsys@useobject{currentmarker}{}%
\end{pgfscope}%
\begin{pgfscope}%
\pgfsys@transformshift{3.947334in}{0.969392in}%
\pgfsys@useobject{currentmarker}{}%
\end{pgfscope}%
\begin{pgfscope}%
\pgfsys@transformshift{4.606262in}{1.653269in}%
\pgfsys@useobject{currentmarker}{}%
\end{pgfscope}%
\begin{pgfscope}%
\pgfsys@transformshift{4.474127in}{1.320092in}%
\pgfsys@useobject{currentmarker}{}%
\end{pgfscope}%
\begin{pgfscope}%
\pgfsys@transformshift{2.939197in}{0.841337in}%
\pgfsys@useobject{currentmarker}{}%
\end{pgfscope}%
\begin{pgfscope}%
\pgfsys@transformshift{4.605307in}{1.693418in}%
\pgfsys@useobject{currentmarker}{}%
\end{pgfscope}%
\begin{pgfscope}%
\pgfsys@transformshift{4.700507in}{1.970496in}%
\pgfsys@useobject{currentmarker}{}%
\end{pgfscope}%
\end{pgfscope}%
\begin{pgfscope}%
\pgfsetbuttcap%
\pgfsetroundjoin%
\definecolor{currentfill}{rgb}{0.000000,0.000000,0.000000}%
\pgfsetfillcolor{currentfill}%
\pgfsetlinewidth{0.803000pt}%
\definecolor{currentstroke}{rgb}{0.000000,0.000000,0.000000}%
\pgfsetstrokecolor{currentstroke}%
\pgfsetdash{}{0pt}%
\pgfsys@defobject{currentmarker}{\pgfqpoint{0.000000in}{-0.048611in}}{\pgfqpoint{0.000000in}{0.000000in}}{%
\pgfpathmoveto{\pgfqpoint{0.000000in}{0.000000in}}%
\pgfpathlineto{\pgfqpoint{0.000000in}{-0.048611in}}%
\pgfusepath{stroke,fill}%
}%
\begin{pgfscope}%
\pgfsys@transformshift{1.376591in}{0.582778in}%
\pgfsys@useobject{currentmarker}{}%
\end{pgfscope}%
\end{pgfscope}%
\begin{pgfscope}%
\definecolor{textcolor}{rgb}{0.000000,0.000000,0.000000}%
\pgfsetstrokecolor{textcolor}%
\pgfsetfillcolor{textcolor}%
\pgftext[x=1.376591in,y=0.485556in,,top]{\color{textcolor}\sffamily\fontsize{10.000000}{12.000000}\selectfont \(\displaystyle {10^{0}}\)}%
\end{pgfscope}%
\begin{pgfscope}%
\pgfsetbuttcap%
\pgfsetroundjoin%
\definecolor{currentfill}{rgb}{0.000000,0.000000,0.000000}%
\pgfsetfillcolor{currentfill}%
\pgfsetlinewidth{0.803000pt}%
\definecolor{currentstroke}{rgb}{0.000000,0.000000,0.000000}%
\pgfsetstrokecolor{currentstroke}%
\pgfsetdash{}{0pt}%
\pgfsys@defobject{currentmarker}{\pgfqpoint{0.000000in}{-0.048611in}}{\pgfqpoint{0.000000in}{0.000000in}}{%
\pgfpathmoveto{\pgfqpoint{0.000000in}{0.000000in}}%
\pgfpathlineto{\pgfqpoint{0.000000in}{-0.048611in}}%
\pgfusepath{stroke,fill}%
}%
\begin{pgfscope}%
\pgfsys@transformshift{2.494383in}{0.582778in}%
\pgfsys@useobject{currentmarker}{}%
\end{pgfscope}%
\end{pgfscope}%
\begin{pgfscope}%
\definecolor{textcolor}{rgb}{0.000000,0.000000,0.000000}%
\pgfsetstrokecolor{textcolor}%
\pgfsetfillcolor{textcolor}%
\pgftext[x=2.494383in,y=0.485556in,,top]{\color{textcolor}\sffamily\fontsize{10.000000}{12.000000}\selectfont \(\displaystyle {10^{1}}\)}%
\end{pgfscope}%
\begin{pgfscope}%
\pgfsetbuttcap%
\pgfsetroundjoin%
\definecolor{currentfill}{rgb}{0.000000,0.000000,0.000000}%
\pgfsetfillcolor{currentfill}%
\pgfsetlinewidth{0.803000pt}%
\definecolor{currentstroke}{rgb}{0.000000,0.000000,0.000000}%
\pgfsetstrokecolor{currentstroke}%
\pgfsetdash{}{0pt}%
\pgfsys@defobject{currentmarker}{\pgfqpoint{0.000000in}{-0.048611in}}{\pgfqpoint{0.000000in}{0.000000in}}{%
\pgfpathmoveto{\pgfqpoint{0.000000in}{0.000000in}}%
\pgfpathlineto{\pgfqpoint{0.000000in}{-0.048611in}}%
\pgfusepath{stroke,fill}%
}%
\begin{pgfscope}%
\pgfsys@transformshift{3.612175in}{0.582778in}%
\pgfsys@useobject{currentmarker}{}%
\end{pgfscope}%
\end{pgfscope}%
\begin{pgfscope}%
\definecolor{textcolor}{rgb}{0.000000,0.000000,0.000000}%
\pgfsetstrokecolor{textcolor}%
\pgfsetfillcolor{textcolor}%
\pgftext[x=3.612175in,y=0.485556in,,top]{\color{textcolor}\sffamily\fontsize{10.000000}{12.000000}\selectfont \(\displaystyle {10^{2}}\)}%
\end{pgfscope}%
\begin{pgfscope}%
\pgfsetbuttcap%
\pgfsetroundjoin%
\definecolor{currentfill}{rgb}{0.000000,0.000000,0.000000}%
\pgfsetfillcolor{currentfill}%
\pgfsetlinewidth{0.803000pt}%
\definecolor{currentstroke}{rgb}{0.000000,0.000000,0.000000}%
\pgfsetstrokecolor{currentstroke}%
\pgfsetdash{}{0pt}%
\pgfsys@defobject{currentmarker}{\pgfqpoint{0.000000in}{-0.048611in}}{\pgfqpoint{0.000000in}{0.000000in}}{%
\pgfpathmoveto{\pgfqpoint{0.000000in}{0.000000in}}%
\pgfpathlineto{\pgfqpoint{0.000000in}{-0.048611in}}%
\pgfusepath{stroke,fill}%
}%
\begin{pgfscope}%
\pgfsys@transformshift{4.729967in}{0.582778in}%
\pgfsys@useobject{currentmarker}{}%
\end{pgfscope}%
\end{pgfscope}%
\begin{pgfscope}%
\definecolor{textcolor}{rgb}{0.000000,0.000000,0.000000}%
\pgfsetstrokecolor{textcolor}%
\pgfsetfillcolor{textcolor}%
\pgftext[x=4.729967in,y=0.485556in,,top]{\color{textcolor}\sffamily\fontsize{10.000000}{12.000000}\selectfont \(\displaystyle {10^{3}}\)}%
\end{pgfscope}%
\begin{pgfscope}%
\pgfsetbuttcap%
\pgfsetroundjoin%
\definecolor{currentfill}{rgb}{0.000000,0.000000,0.000000}%
\pgfsetfillcolor{currentfill}%
\pgfsetlinewidth{0.602250pt}%
\definecolor{currentstroke}{rgb}{0.000000,0.000000,0.000000}%
\pgfsetstrokecolor{currentstroke}%
\pgfsetdash{}{0pt}%
\pgfsys@defobject{currentmarker}{\pgfqpoint{0.000000in}{-0.027778in}}{\pgfqpoint{0.000000in}{0.000000in}}{%
\pgfpathmoveto{\pgfqpoint{0.000000in}{0.000000in}}%
\pgfpathlineto{\pgfqpoint{0.000000in}{-0.027778in}}%
\pgfusepath{stroke,fill}%
}%
\begin{pgfscope}%
\pgfsys@transformshift{0.792121in}{0.582778in}%
\pgfsys@useobject{currentmarker}{}%
\end{pgfscope}%
\end{pgfscope}%
\begin{pgfscope}%
\pgfsetbuttcap%
\pgfsetroundjoin%
\definecolor{currentfill}{rgb}{0.000000,0.000000,0.000000}%
\pgfsetfillcolor{currentfill}%
\pgfsetlinewidth{0.602250pt}%
\definecolor{currentstroke}{rgb}{0.000000,0.000000,0.000000}%
\pgfsetstrokecolor{currentstroke}%
\pgfsetdash{}{0pt}%
\pgfsys@defobject{currentmarker}{\pgfqpoint{0.000000in}{-0.027778in}}{\pgfqpoint{0.000000in}{0.000000in}}{%
\pgfpathmoveto{\pgfqpoint{0.000000in}{0.000000in}}%
\pgfpathlineto{\pgfqpoint{0.000000in}{-0.027778in}}%
\pgfusepath{stroke,fill}%
}%
\begin{pgfscope}%
\pgfsys@transformshift{0.931777in}{0.582778in}%
\pgfsys@useobject{currentmarker}{}%
\end{pgfscope}%
\end{pgfscope}%
\begin{pgfscope}%
\pgfsetbuttcap%
\pgfsetroundjoin%
\definecolor{currentfill}{rgb}{0.000000,0.000000,0.000000}%
\pgfsetfillcolor{currentfill}%
\pgfsetlinewidth{0.602250pt}%
\definecolor{currentstroke}{rgb}{0.000000,0.000000,0.000000}%
\pgfsetstrokecolor{currentstroke}%
\pgfsetdash{}{0pt}%
\pgfsys@defobject{currentmarker}{\pgfqpoint{0.000000in}{-0.027778in}}{\pgfqpoint{0.000000in}{0.000000in}}{%
\pgfpathmoveto{\pgfqpoint{0.000000in}{0.000000in}}%
\pgfpathlineto{\pgfqpoint{0.000000in}{-0.027778in}}%
\pgfusepath{stroke,fill}%
}%
\begin{pgfscope}%
\pgfsys@transformshift{1.040102in}{0.582778in}%
\pgfsys@useobject{currentmarker}{}%
\end{pgfscope}%
\end{pgfscope}%
\begin{pgfscope}%
\pgfsetbuttcap%
\pgfsetroundjoin%
\definecolor{currentfill}{rgb}{0.000000,0.000000,0.000000}%
\pgfsetfillcolor{currentfill}%
\pgfsetlinewidth{0.602250pt}%
\definecolor{currentstroke}{rgb}{0.000000,0.000000,0.000000}%
\pgfsetstrokecolor{currentstroke}%
\pgfsetdash{}{0pt}%
\pgfsys@defobject{currentmarker}{\pgfqpoint{0.000000in}{-0.027778in}}{\pgfqpoint{0.000000in}{0.000000in}}{%
\pgfpathmoveto{\pgfqpoint{0.000000in}{0.000000in}}%
\pgfpathlineto{\pgfqpoint{0.000000in}{-0.027778in}}%
\pgfusepath{stroke,fill}%
}%
\begin{pgfscope}%
\pgfsys@transformshift{1.128610in}{0.582778in}%
\pgfsys@useobject{currentmarker}{}%
\end{pgfscope}%
\end{pgfscope}%
\begin{pgfscope}%
\pgfsetbuttcap%
\pgfsetroundjoin%
\definecolor{currentfill}{rgb}{0.000000,0.000000,0.000000}%
\pgfsetfillcolor{currentfill}%
\pgfsetlinewidth{0.602250pt}%
\definecolor{currentstroke}{rgb}{0.000000,0.000000,0.000000}%
\pgfsetstrokecolor{currentstroke}%
\pgfsetdash{}{0pt}%
\pgfsys@defobject{currentmarker}{\pgfqpoint{0.000000in}{-0.027778in}}{\pgfqpoint{0.000000in}{0.000000in}}{%
\pgfpathmoveto{\pgfqpoint{0.000000in}{0.000000in}}%
\pgfpathlineto{\pgfqpoint{0.000000in}{-0.027778in}}%
\pgfusepath{stroke,fill}%
}%
\begin{pgfscope}%
\pgfsys@transformshift{1.203443in}{0.582778in}%
\pgfsys@useobject{currentmarker}{}%
\end{pgfscope}%
\end{pgfscope}%
\begin{pgfscope}%
\pgfsetbuttcap%
\pgfsetroundjoin%
\definecolor{currentfill}{rgb}{0.000000,0.000000,0.000000}%
\pgfsetfillcolor{currentfill}%
\pgfsetlinewidth{0.602250pt}%
\definecolor{currentstroke}{rgb}{0.000000,0.000000,0.000000}%
\pgfsetstrokecolor{currentstroke}%
\pgfsetdash{}{0pt}%
\pgfsys@defobject{currentmarker}{\pgfqpoint{0.000000in}{-0.027778in}}{\pgfqpoint{0.000000in}{0.000000in}}{%
\pgfpathmoveto{\pgfqpoint{0.000000in}{0.000000in}}%
\pgfpathlineto{\pgfqpoint{0.000000in}{-0.027778in}}%
\pgfusepath{stroke,fill}%
}%
\begin{pgfscope}%
\pgfsys@transformshift{1.268266in}{0.582778in}%
\pgfsys@useobject{currentmarker}{}%
\end{pgfscope}%
\end{pgfscope}%
\begin{pgfscope}%
\pgfsetbuttcap%
\pgfsetroundjoin%
\definecolor{currentfill}{rgb}{0.000000,0.000000,0.000000}%
\pgfsetfillcolor{currentfill}%
\pgfsetlinewidth{0.602250pt}%
\definecolor{currentstroke}{rgb}{0.000000,0.000000,0.000000}%
\pgfsetstrokecolor{currentstroke}%
\pgfsetdash{}{0pt}%
\pgfsys@defobject{currentmarker}{\pgfqpoint{0.000000in}{-0.027778in}}{\pgfqpoint{0.000000in}{0.000000in}}{%
\pgfpathmoveto{\pgfqpoint{0.000000in}{0.000000in}}%
\pgfpathlineto{\pgfqpoint{0.000000in}{-0.027778in}}%
\pgfusepath{stroke,fill}%
}%
\begin{pgfscope}%
\pgfsys@transformshift{1.325443in}{0.582778in}%
\pgfsys@useobject{currentmarker}{}%
\end{pgfscope}%
\end{pgfscope}%
\begin{pgfscope}%
\pgfsetbuttcap%
\pgfsetroundjoin%
\definecolor{currentfill}{rgb}{0.000000,0.000000,0.000000}%
\pgfsetfillcolor{currentfill}%
\pgfsetlinewidth{0.602250pt}%
\definecolor{currentstroke}{rgb}{0.000000,0.000000,0.000000}%
\pgfsetstrokecolor{currentstroke}%
\pgfsetdash{}{0pt}%
\pgfsys@defobject{currentmarker}{\pgfqpoint{0.000000in}{-0.027778in}}{\pgfqpoint{0.000000in}{0.000000in}}{%
\pgfpathmoveto{\pgfqpoint{0.000000in}{0.000000in}}%
\pgfpathlineto{\pgfqpoint{0.000000in}{-0.027778in}}%
\pgfusepath{stroke,fill}%
}%
\begin{pgfscope}%
\pgfsys@transformshift{1.713080in}{0.582778in}%
\pgfsys@useobject{currentmarker}{}%
\end{pgfscope}%
\end{pgfscope}%
\begin{pgfscope}%
\pgfsetbuttcap%
\pgfsetroundjoin%
\definecolor{currentfill}{rgb}{0.000000,0.000000,0.000000}%
\pgfsetfillcolor{currentfill}%
\pgfsetlinewidth{0.602250pt}%
\definecolor{currentstroke}{rgb}{0.000000,0.000000,0.000000}%
\pgfsetstrokecolor{currentstroke}%
\pgfsetdash{}{0pt}%
\pgfsys@defobject{currentmarker}{\pgfqpoint{0.000000in}{-0.027778in}}{\pgfqpoint{0.000000in}{0.000000in}}{%
\pgfpathmoveto{\pgfqpoint{0.000000in}{0.000000in}}%
\pgfpathlineto{\pgfqpoint{0.000000in}{-0.027778in}}%
\pgfusepath{stroke,fill}%
}%
\begin{pgfscope}%
\pgfsys@transformshift{1.909913in}{0.582778in}%
\pgfsys@useobject{currentmarker}{}%
\end{pgfscope}%
\end{pgfscope}%
\begin{pgfscope}%
\pgfsetbuttcap%
\pgfsetroundjoin%
\definecolor{currentfill}{rgb}{0.000000,0.000000,0.000000}%
\pgfsetfillcolor{currentfill}%
\pgfsetlinewidth{0.602250pt}%
\definecolor{currentstroke}{rgb}{0.000000,0.000000,0.000000}%
\pgfsetstrokecolor{currentstroke}%
\pgfsetdash{}{0pt}%
\pgfsys@defobject{currentmarker}{\pgfqpoint{0.000000in}{-0.027778in}}{\pgfqpoint{0.000000in}{0.000000in}}{%
\pgfpathmoveto{\pgfqpoint{0.000000in}{0.000000in}}%
\pgfpathlineto{\pgfqpoint{0.000000in}{-0.027778in}}%
\pgfusepath{stroke,fill}%
}%
\begin{pgfscope}%
\pgfsys@transformshift{2.049569in}{0.582778in}%
\pgfsys@useobject{currentmarker}{}%
\end{pgfscope}%
\end{pgfscope}%
\begin{pgfscope}%
\pgfsetbuttcap%
\pgfsetroundjoin%
\definecolor{currentfill}{rgb}{0.000000,0.000000,0.000000}%
\pgfsetfillcolor{currentfill}%
\pgfsetlinewidth{0.602250pt}%
\definecolor{currentstroke}{rgb}{0.000000,0.000000,0.000000}%
\pgfsetstrokecolor{currentstroke}%
\pgfsetdash{}{0pt}%
\pgfsys@defobject{currentmarker}{\pgfqpoint{0.000000in}{-0.027778in}}{\pgfqpoint{0.000000in}{0.000000in}}{%
\pgfpathmoveto{\pgfqpoint{0.000000in}{0.000000in}}%
\pgfpathlineto{\pgfqpoint{0.000000in}{-0.027778in}}%
\pgfusepath{stroke,fill}%
}%
\begin{pgfscope}%
\pgfsys@transformshift{2.157894in}{0.582778in}%
\pgfsys@useobject{currentmarker}{}%
\end{pgfscope}%
\end{pgfscope}%
\begin{pgfscope}%
\pgfsetbuttcap%
\pgfsetroundjoin%
\definecolor{currentfill}{rgb}{0.000000,0.000000,0.000000}%
\pgfsetfillcolor{currentfill}%
\pgfsetlinewidth{0.602250pt}%
\definecolor{currentstroke}{rgb}{0.000000,0.000000,0.000000}%
\pgfsetstrokecolor{currentstroke}%
\pgfsetdash{}{0pt}%
\pgfsys@defobject{currentmarker}{\pgfqpoint{0.000000in}{-0.027778in}}{\pgfqpoint{0.000000in}{0.000000in}}{%
\pgfpathmoveto{\pgfqpoint{0.000000in}{0.000000in}}%
\pgfpathlineto{\pgfqpoint{0.000000in}{-0.027778in}}%
\pgfusepath{stroke,fill}%
}%
\begin{pgfscope}%
\pgfsys@transformshift{2.246402in}{0.582778in}%
\pgfsys@useobject{currentmarker}{}%
\end{pgfscope}%
\end{pgfscope}%
\begin{pgfscope}%
\pgfsetbuttcap%
\pgfsetroundjoin%
\definecolor{currentfill}{rgb}{0.000000,0.000000,0.000000}%
\pgfsetfillcolor{currentfill}%
\pgfsetlinewidth{0.602250pt}%
\definecolor{currentstroke}{rgb}{0.000000,0.000000,0.000000}%
\pgfsetstrokecolor{currentstroke}%
\pgfsetdash{}{0pt}%
\pgfsys@defobject{currentmarker}{\pgfqpoint{0.000000in}{-0.027778in}}{\pgfqpoint{0.000000in}{0.000000in}}{%
\pgfpathmoveto{\pgfqpoint{0.000000in}{0.000000in}}%
\pgfpathlineto{\pgfqpoint{0.000000in}{-0.027778in}}%
\pgfusepath{stroke,fill}%
}%
\begin{pgfscope}%
\pgfsys@transformshift{2.321235in}{0.582778in}%
\pgfsys@useobject{currentmarker}{}%
\end{pgfscope}%
\end{pgfscope}%
\begin{pgfscope}%
\pgfsetbuttcap%
\pgfsetroundjoin%
\definecolor{currentfill}{rgb}{0.000000,0.000000,0.000000}%
\pgfsetfillcolor{currentfill}%
\pgfsetlinewidth{0.602250pt}%
\definecolor{currentstroke}{rgb}{0.000000,0.000000,0.000000}%
\pgfsetstrokecolor{currentstroke}%
\pgfsetdash{}{0pt}%
\pgfsys@defobject{currentmarker}{\pgfqpoint{0.000000in}{-0.027778in}}{\pgfqpoint{0.000000in}{0.000000in}}{%
\pgfpathmoveto{\pgfqpoint{0.000000in}{0.000000in}}%
\pgfpathlineto{\pgfqpoint{0.000000in}{-0.027778in}}%
\pgfusepath{stroke,fill}%
}%
\begin{pgfscope}%
\pgfsys@transformshift{2.386058in}{0.582778in}%
\pgfsys@useobject{currentmarker}{}%
\end{pgfscope}%
\end{pgfscope}%
\begin{pgfscope}%
\pgfsetbuttcap%
\pgfsetroundjoin%
\definecolor{currentfill}{rgb}{0.000000,0.000000,0.000000}%
\pgfsetfillcolor{currentfill}%
\pgfsetlinewidth{0.602250pt}%
\definecolor{currentstroke}{rgb}{0.000000,0.000000,0.000000}%
\pgfsetstrokecolor{currentstroke}%
\pgfsetdash{}{0pt}%
\pgfsys@defobject{currentmarker}{\pgfqpoint{0.000000in}{-0.027778in}}{\pgfqpoint{0.000000in}{0.000000in}}{%
\pgfpathmoveto{\pgfqpoint{0.000000in}{0.000000in}}%
\pgfpathlineto{\pgfqpoint{0.000000in}{-0.027778in}}%
\pgfusepath{stroke,fill}%
}%
\begin{pgfscope}%
\pgfsys@transformshift{2.443236in}{0.582778in}%
\pgfsys@useobject{currentmarker}{}%
\end{pgfscope}%
\end{pgfscope}%
\begin{pgfscope}%
\pgfsetbuttcap%
\pgfsetroundjoin%
\definecolor{currentfill}{rgb}{0.000000,0.000000,0.000000}%
\pgfsetfillcolor{currentfill}%
\pgfsetlinewidth{0.602250pt}%
\definecolor{currentstroke}{rgb}{0.000000,0.000000,0.000000}%
\pgfsetstrokecolor{currentstroke}%
\pgfsetdash{}{0pt}%
\pgfsys@defobject{currentmarker}{\pgfqpoint{0.000000in}{-0.027778in}}{\pgfqpoint{0.000000in}{0.000000in}}{%
\pgfpathmoveto{\pgfqpoint{0.000000in}{0.000000in}}%
\pgfpathlineto{\pgfqpoint{0.000000in}{-0.027778in}}%
\pgfusepath{stroke,fill}%
}%
\begin{pgfscope}%
\pgfsys@transformshift{2.830872in}{0.582778in}%
\pgfsys@useobject{currentmarker}{}%
\end{pgfscope}%
\end{pgfscope}%
\begin{pgfscope}%
\pgfsetbuttcap%
\pgfsetroundjoin%
\definecolor{currentfill}{rgb}{0.000000,0.000000,0.000000}%
\pgfsetfillcolor{currentfill}%
\pgfsetlinewidth{0.602250pt}%
\definecolor{currentstroke}{rgb}{0.000000,0.000000,0.000000}%
\pgfsetstrokecolor{currentstroke}%
\pgfsetdash{}{0pt}%
\pgfsys@defobject{currentmarker}{\pgfqpoint{0.000000in}{-0.027778in}}{\pgfqpoint{0.000000in}{0.000000in}}{%
\pgfpathmoveto{\pgfqpoint{0.000000in}{0.000000in}}%
\pgfpathlineto{\pgfqpoint{0.000000in}{-0.027778in}}%
\pgfusepath{stroke,fill}%
}%
\begin{pgfscope}%
\pgfsys@transformshift{3.027705in}{0.582778in}%
\pgfsys@useobject{currentmarker}{}%
\end{pgfscope}%
\end{pgfscope}%
\begin{pgfscope}%
\pgfsetbuttcap%
\pgfsetroundjoin%
\definecolor{currentfill}{rgb}{0.000000,0.000000,0.000000}%
\pgfsetfillcolor{currentfill}%
\pgfsetlinewidth{0.602250pt}%
\definecolor{currentstroke}{rgb}{0.000000,0.000000,0.000000}%
\pgfsetstrokecolor{currentstroke}%
\pgfsetdash{}{0pt}%
\pgfsys@defobject{currentmarker}{\pgfqpoint{0.000000in}{-0.027778in}}{\pgfqpoint{0.000000in}{0.000000in}}{%
\pgfpathmoveto{\pgfqpoint{0.000000in}{0.000000in}}%
\pgfpathlineto{\pgfqpoint{0.000000in}{-0.027778in}}%
\pgfusepath{stroke,fill}%
}%
\begin{pgfscope}%
\pgfsys@transformshift{3.167361in}{0.582778in}%
\pgfsys@useobject{currentmarker}{}%
\end{pgfscope}%
\end{pgfscope}%
\begin{pgfscope}%
\pgfsetbuttcap%
\pgfsetroundjoin%
\definecolor{currentfill}{rgb}{0.000000,0.000000,0.000000}%
\pgfsetfillcolor{currentfill}%
\pgfsetlinewidth{0.602250pt}%
\definecolor{currentstroke}{rgb}{0.000000,0.000000,0.000000}%
\pgfsetstrokecolor{currentstroke}%
\pgfsetdash{}{0pt}%
\pgfsys@defobject{currentmarker}{\pgfqpoint{0.000000in}{-0.027778in}}{\pgfqpoint{0.000000in}{0.000000in}}{%
\pgfpathmoveto{\pgfqpoint{0.000000in}{0.000000in}}%
\pgfpathlineto{\pgfqpoint{0.000000in}{-0.027778in}}%
\pgfusepath{stroke,fill}%
}%
\begin{pgfscope}%
\pgfsys@transformshift{3.275686in}{0.582778in}%
\pgfsys@useobject{currentmarker}{}%
\end{pgfscope}%
\end{pgfscope}%
\begin{pgfscope}%
\pgfsetbuttcap%
\pgfsetroundjoin%
\definecolor{currentfill}{rgb}{0.000000,0.000000,0.000000}%
\pgfsetfillcolor{currentfill}%
\pgfsetlinewidth{0.602250pt}%
\definecolor{currentstroke}{rgb}{0.000000,0.000000,0.000000}%
\pgfsetstrokecolor{currentstroke}%
\pgfsetdash{}{0pt}%
\pgfsys@defobject{currentmarker}{\pgfqpoint{0.000000in}{-0.027778in}}{\pgfqpoint{0.000000in}{0.000000in}}{%
\pgfpathmoveto{\pgfqpoint{0.000000in}{0.000000in}}%
\pgfpathlineto{\pgfqpoint{0.000000in}{-0.027778in}}%
\pgfusepath{stroke,fill}%
}%
\begin{pgfscope}%
\pgfsys@transformshift{3.364194in}{0.582778in}%
\pgfsys@useobject{currentmarker}{}%
\end{pgfscope}%
\end{pgfscope}%
\begin{pgfscope}%
\pgfsetbuttcap%
\pgfsetroundjoin%
\definecolor{currentfill}{rgb}{0.000000,0.000000,0.000000}%
\pgfsetfillcolor{currentfill}%
\pgfsetlinewidth{0.602250pt}%
\definecolor{currentstroke}{rgb}{0.000000,0.000000,0.000000}%
\pgfsetstrokecolor{currentstroke}%
\pgfsetdash{}{0pt}%
\pgfsys@defobject{currentmarker}{\pgfqpoint{0.000000in}{-0.027778in}}{\pgfqpoint{0.000000in}{0.000000in}}{%
\pgfpathmoveto{\pgfqpoint{0.000000in}{0.000000in}}%
\pgfpathlineto{\pgfqpoint{0.000000in}{-0.027778in}}%
\pgfusepath{stroke,fill}%
}%
\begin{pgfscope}%
\pgfsys@transformshift{3.439027in}{0.582778in}%
\pgfsys@useobject{currentmarker}{}%
\end{pgfscope}%
\end{pgfscope}%
\begin{pgfscope}%
\pgfsetbuttcap%
\pgfsetroundjoin%
\definecolor{currentfill}{rgb}{0.000000,0.000000,0.000000}%
\pgfsetfillcolor{currentfill}%
\pgfsetlinewidth{0.602250pt}%
\definecolor{currentstroke}{rgb}{0.000000,0.000000,0.000000}%
\pgfsetstrokecolor{currentstroke}%
\pgfsetdash{}{0pt}%
\pgfsys@defobject{currentmarker}{\pgfqpoint{0.000000in}{-0.027778in}}{\pgfqpoint{0.000000in}{0.000000in}}{%
\pgfpathmoveto{\pgfqpoint{0.000000in}{0.000000in}}%
\pgfpathlineto{\pgfqpoint{0.000000in}{-0.027778in}}%
\pgfusepath{stroke,fill}%
}%
\begin{pgfscope}%
\pgfsys@transformshift{3.503850in}{0.582778in}%
\pgfsys@useobject{currentmarker}{}%
\end{pgfscope}%
\end{pgfscope}%
\begin{pgfscope}%
\pgfsetbuttcap%
\pgfsetroundjoin%
\definecolor{currentfill}{rgb}{0.000000,0.000000,0.000000}%
\pgfsetfillcolor{currentfill}%
\pgfsetlinewidth{0.602250pt}%
\definecolor{currentstroke}{rgb}{0.000000,0.000000,0.000000}%
\pgfsetstrokecolor{currentstroke}%
\pgfsetdash{}{0pt}%
\pgfsys@defobject{currentmarker}{\pgfqpoint{0.000000in}{-0.027778in}}{\pgfqpoint{0.000000in}{0.000000in}}{%
\pgfpathmoveto{\pgfqpoint{0.000000in}{0.000000in}}%
\pgfpathlineto{\pgfqpoint{0.000000in}{-0.027778in}}%
\pgfusepath{stroke,fill}%
}%
\begin{pgfscope}%
\pgfsys@transformshift{3.561028in}{0.582778in}%
\pgfsys@useobject{currentmarker}{}%
\end{pgfscope}%
\end{pgfscope}%
\begin{pgfscope}%
\pgfsetbuttcap%
\pgfsetroundjoin%
\definecolor{currentfill}{rgb}{0.000000,0.000000,0.000000}%
\pgfsetfillcolor{currentfill}%
\pgfsetlinewidth{0.602250pt}%
\definecolor{currentstroke}{rgb}{0.000000,0.000000,0.000000}%
\pgfsetstrokecolor{currentstroke}%
\pgfsetdash{}{0pt}%
\pgfsys@defobject{currentmarker}{\pgfqpoint{0.000000in}{-0.027778in}}{\pgfqpoint{0.000000in}{0.000000in}}{%
\pgfpathmoveto{\pgfqpoint{0.000000in}{0.000000in}}%
\pgfpathlineto{\pgfqpoint{0.000000in}{-0.027778in}}%
\pgfusepath{stroke,fill}%
}%
\begin{pgfscope}%
\pgfsys@transformshift{3.948664in}{0.582778in}%
\pgfsys@useobject{currentmarker}{}%
\end{pgfscope}%
\end{pgfscope}%
\begin{pgfscope}%
\pgfsetbuttcap%
\pgfsetroundjoin%
\definecolor{currentfill}{rgb}{0.000000,0.000000,0.000000}%
\pgfsetfillcolor{currentfill}%
\pgfsetlinewidth{0.602250pt}%
\definecolor{currentstroke}{rgb}{0.000000,0.000000,0.000000}%
\pgfsetstrokecolor{currentstroke}%
\pgfsetdash{}{0pt}%
\pgfsys@defobject{currentmarker}{\pgfqpoint{0.000000in}{-0.027778in}}{\pgfqpoint{0.000000in}{0.000000in}}{%
\pgfpathmoveto{\pgfqpoint{0.000000in}{0.000000in}}%
\pgfpathlineto{\pgfqpoint{0.000000in}{-0.027778in}}%
\pgfusepath{stroke,fill}%
}%
\begin{pgfscope}%
\pgfsys@transformshift{4.145498in}{0.582778in}%
\pgfsys@useobject{currentmarker}{}%
\end{pgfscope}%
\end{pgfscope}%
\begin{pgfscope}%
\pgfsetbuttcap%
\pgfsetroundjoin%
\definecolor{currentfill}{rgb}{0.000000,0.000000,0.000000}%
\pgfsetfillcolor{currentfill}%
\pgfsetlinewidth{0.602250pt}%
\definecolor{currentstroke}{rgb}{0.000000,0.000000,0.000000}%
\pgfsetstrokecolor{currentstroke}%
\pgfsetdash{}{0pt}%
\pgfsys@defobject{currentmarker}{\pgfqpoint{0.000000in}{-0.027778in}}{\pgfqpoint{0.000000in}{0.000000in}}{%
\pgfpathmoveto{\pgfqpoint{0.000000in}{0.000000in}}%
\pgfpathlineto{\pgfqpoint{0.000000in}{-0.027778in}}%
\pgfusepath{stroke,fill}%
}%
\begin{pgfscope}%
\pgfsys@transformshift{4.285153in}{0.582778in}%
\pgfsys@useobject{currentmarker}{}%
\end{pgfscope}%
\end{pgfscope}%
\begin{pgfscope}%
\pgfsetbuttcap%
\pgfsetroundjoin%
\definecolor{currentfill}{rgb}{0.000000,0.000000,0.000000}%
\pgfsetfillcolor{currentfill}%
\pgfsetlinewidth{0.602250pt}%
\definecolor{currentstroke}{rgb}{0.000000,0.000000,0.000000}%
\pgfsetstrokecolor{currentstroke}%
\pgfsetdash{}{0pt}%
\pgfsys@defobject{currentmarker}{\pgfqpoint{0.000000in}{-0.027778in}}{\pgfqpoint{0.000000in}{0.000000in}}{%
\pgfpathmoveto{\pgfqpoint{0.000000in}{0.000000in}}%
\pgfpathlineto{\pgfqpoint{0.000000in}{-0.027778in}}%
\pgfusepath{stroke,fill}%
}%
\begin{pgfscope}%
\pgfsys@transformshift{4.393478in}{0.582778in}%
\pgfsys@useobject{currentmarker}{}%
\end{pgfscope}%
\end{pgfscope}%
\begin{pgfscope}%
\pgfsetbuttcap%
\pgfsetroundjoin%
\definecolor{currentfill}{rgb}{0.000000,0.000000,0.000000}%
\pgfsetfillcolor{currentfill}%
\pgfsetlinewidth{0.602250pt}%
\definecolor{currentstroke}{rgb}{0.000000,0.000000,0.000000}%
\pgfsetstrokecolor{currentstroke}%
\pgfsetdash{}{0pt}%
\pgfsys@defobject{currentmarker}{\pgfqpoint{0.000000in}{-0.027778in}}{\pgfqpoint{0.000000in}{0.000000in}}{%
\pgfpathmoveto{\pgfqpoint{0.000000in}{0.000000in}}%
\pgfpathlineto{\pgfqpoint{0.000000in}{-0.027778in}}%
\pgfusepath{stroke,fill}%
}%
\begin{pgfscope}%
\pgfsys@transformshift{4.481987in}{0.582778in}%
\pgfsys@useobject{currentmarker}{}%
\end{pgfscope}%
\end{pgfscope}%
\begin{pgfscope}%
\pgfsetbuttcap%
\pgfsetroundjoin%
\definecolor{currentfill}{rgb}{0.000000,0.000000,0.000000}%
\pgfsetfillcolor{currentfill}%
\pgfsetlinewidth{0.602250pt}%
\definecolor{currentstroke}{rgb}{0.000000,0.000000,0.000000}%
\pgfsetstrokecolor{currentstroke}%
\pgfsetdash{}{0pt}%
\pgfsys@defobject{currentmarker}{\pgfqpoint{0.000000in}{-0.027778in}}{\pgfqpoint{0.000000in}{0.000000in}}{%
\pgfpathmoveto{\pgfqpoint{0.000000in}{0.000000in}}%
\pgfpathlineto{\pgfqpoint{0.000000in}{-0.027778in}}%
\pgfusepath{stroke,fill}%
}%
\begin{pgfscope}%
\pgfsys@transformshift{4.556819in}{0.582778in}%
\pgfsys@useobject{currentmarker}{}%
\end{pgfscope}%
\end{pgfscope}%
\begin{pgfscope}%
\pgfsetbuttcap%
\pgfsetroundjoin%
\definecolor{currentfill}{rgb}{0.000000,0.000000,0.000000}%
\pgfsetfillcolor{currentfill}%
\pgfsetlinewidth{0.602250pt}%
\definecolor{currentstroke}{rgb}{0.000000,0.000000,0.000000}%
\pgfsetstrokecolor{currentstroke}%
\pgfsetdash{}{0pt}%
\pgfsys@defobject{currentmarker}{\pgfqpoint{0.000000in}{-0.027778in}}{\pgfqpoint{0.000000in}{0.000000in}}{%
\pgfpathmoveto{\pgfqpoint{0.000000in}{0.000000in}}%
\pgfpathlineto{\pgfqpoint{0.000000in}{-0.027778in}}%
\pgfusepath{stroke,fill}%
}%
\begin{pgfscope}%
\pgfsys@transformshift{4.621642in}{0.582778in}%
\pgfsys@useobject{currentmarker}{}%
\end{pgfscope}%
\end{pgfscope}%
\begin{pgfscope}%
\pgfsetbuttcap%
\pgfsetroundjoin%
\definecolor{currentfill}{rgb}{0.000000,0.000000,0.000000}%
\pgfsetfillcolor{currentfill}%
\pgfsetlinewidth{0.602250pt}%
\definecolor{currentstroke}{rgb}{0.000000,0.000000,0.000000}%
\pgfsetstrokecolor{currentstroke}%
\pgfsetdash{}{0pt}%
\pgfsys@defobject{currentmarker}{\pgfqpoint{0.000000in}{-0.027778in}}{\pgfqpoint{0.000000in}{0.000000in}}{%
\pgfpathmoveto{\pgfqpoint{0.000000in}{0.000000in}}%
\pgfpathlineto{\pgfqpoint{0.000000in}{-0.027778in}}%
\pgfusepath{stroke,fill}%
}%
\begin{pgfscope}%
\pgfsys@transformshift{4.678820in}{0.582778in}%
\pgfsys@useobject{currentmarker}{}%
\end{pgfscope}%
\end{pgfscope}%
\begin{pgfscope}%
\definecolor{textcolor}{rgb}{0.000000,0.000000,0.000000}%
\pgfsetstrokecolor{textcolor}%
\pgfsetfillcolor{textcolor}%
\pgftext[x=2.838993in,y=0.295587in,,top]{\color{textcolor}\sffamily\fontsize{10.000000}{12.000000}\selectfont goodput (req/s)}%
\end{pgfscope}%
\begin{pgfscope}%
\pgfsetbuttcap%
\pgfsetroundjoin%
\definecolor{currentfill}{rgb}{0.000000,0.000000,0.000000}%
\pgfsetfillcolor{currentfill}%
\pgfsetlinewidth{0.803000pt}%
\definecolor{currentstroke}{rgb}{0.000000,0.000000,0.000000}%
\pgfsetstrokecolor{currentstroke}%
\pgfsetdash{}{0pt}%
\pgfsys@defobject{currentmarker}{\pgfqpoint{-0.048611in}{0.000000in}}{\pgfqpoint{-0.000000in}{0.000000in}}{%
\pgfpathmoveto{\pgfqpoint{-0.000000in}{0.000000in}}%
\pgfpathlineto{\pgfqpoint{-0.048611in}{0.000000in}}%
\pgfusepath{stroke,fill}%
}%
\begin{pgfscope}%
\pgfsys@transformshift{0.741055in}{0.582778in}%
\pgfsys@useobject{currentmarker}{}%
\end{pgfscope}%
\end{pgfscope}%
\begin{pgfscope}%
\definecolor{textcolor}{rgb}{0.000000,0.000000,0.000000}%
\pgfsetstrokecolor{textcolor}%
\pgfsetfillcolor{textcolor}%
\pgftext[x=0.422953in, y=0.530016in, left, base]{\color{textcolor}\sffamily\fontsize{10.000000}{12.000000}\selectfont 0.0}%
\end{pgfscope}%
\begin{pgfscope}%
\pgfsetbuttcap%
\pgfsetroundjoin%
\definecolor{currentfill}{rgb}{0.000000,0.000000,0.000000}%
\pgfsetfillcolor{currentfill}%
\pgfsetlinewidth{0.803000pt}%
\definecolor{currentstroke}{rgb}{0.000000,0.000000,0.000000}%
\pgfsetstrokecolor{currentstroke}%
\pgfsetdash{}{0pt}%
\pgfsys@defobject{currentmarker}{\pgfqpoint{-0.048611in}{0.000000in}}{\pgfqpoint{-0.000000in}{0.000000in}}{%
\pgfpathmoveto{\pgfqpoint{-0.000000in}{0.000000in}}%
\pgfpathlineto{\pgfqpoint{-0.048611in}{0.000000in}}%
\pgfusepath{stroke,fill}%
}%
\begin{pgfscope}%
\pgfsys@transformshift{0.741055in}{1.436222in}%
\pgfsys@useobject{currentmarker}{}%
\end{pgfscope}%
\end{pgfscope}%
\begin{pgfscope}%
\definecolor{textcolor}{rgb}{0.000000,0.000000,0.000000}%
\pgfsetstrokecolor{textcolor}%
\pgfsetfillcolor{textcolor}%
\pgftext[x=0.422953in, y=1.383461in, left, base]{\color{textcolor}\sffamily\fontsize{10.000000}{12.000000}\selectfont 0.2}%
\end{pgfscope}%
\begin{pgfscope}%
\pgfsetbuttcap%
\pgfsetroundjoin%
\definecolor{currentfill}{rgb}{0.000000,0.000000,0.000000}%
\pgfsetfillcolor{currentfill}%
\pgfsetlinewidth{0.803000pt}%
\definecolor{currentstroke}{rgb}{0.000000,0.000000,0.000000}%
\pgfsetstrokecolor{currentstroke}%
\pgfsetdash{}{0pt}%
\pgfsys@defobject{currentmarker}{\pgfqpoint{-0.048611in}{0.000000in}}{\pgfqpoint{-0.000000in}{0.000000in}}{%
\pgfpathmoveto{\pgfqpoint{-0.000000in}{0.000000in}}%
\pgfpathlineto{\pgfqpoint{-0.048611in}{0.000000in}}%
\pgfusepath{stroke,fill}%
}%
\begin{pgfscope}%
\pgfsys@transformshift{0.741055in}{2.289667in}%
\pgfsys@useobject{currentmarker}{}%
\end{pgfscope}%
\end{pgfscope}%
\begin{pgfscope}%
\definecolor{textcolor}{rgb}{0.000000,0.000000,0.000000}%
\pgfsetstrokecolor{textcolor}%
\pgfsetfillcolor{textcolor}%
\pgftext[x=0.422953in, y=2.236905in, left, base]{\color{textcolor}\sffamily\fontsize{10.000000}{12.000000}\selectfont 0.4}%
\end{pgfscope}%
\begin{pgfscope}%
\pgfsetbuttcap%
\pgfsetroundjoin%
\definecolor{currentfill}{rgb}{0.000000,0.000000,0.000000}%
\pgfsetfillcolor{currentfill}%
\pgfsetlinewidth{0.803000pt}%
\definecolor{currentstroke}{rgb}{0.000000,0.000000,0.000000}%
\pgfsetstrokecolor{currentstroke}%
\pgfsetdash{}{0pt}%
\pgfsys@defobject{currentmarker}{\pgfqpoint{-0.048611in}{0.000000in}}{\pgfqpoint{-0.000000in}{0.000000in}}{%
\pgfpathmoveto{\pgfqpoint{-0.000000in}{0.000000in}}%
\pgfpathlineto{\pgfqpoint{-0.048611in}{0.000000in}}%
\pgfusepath{stroke,fill}%
}%
\begin{pgfscope}%
\pgfsys@transformshift{0.741055in}{3.143111in}%
\pgfsys@useobject{currentmarker}{}%
\end{pgfscope}%
\end{pgfscope}%
\begin{pgfscope}%
\definecolor{textcolor}{rgb}{0.000000,0.000000,0.000000}%
\pgfsetstrokecolor{textcolor}%
\pgfsetfillcolor{textcolor}%
\pgftext[x=0.422953in, y=3.090350in, left, base]{\color{textcolor}\sffamily\fontsize{10.000000}{12.000000}\selectfont 0.6}%
\end{pgfscope}%
\begin{pgfscope}%
\pgfsetbuttcap%
\pgfsetroundjoin%
\definecolor{currentfill}{rgb}{0.000000,0.000000,0.000000}%
\pgfsetfillcolor{currentfill}%
\pgfsetlinewidth{0.803000pt}%
\definecolor{currentstroke}{rgb}{0.000000,0.000000,0.000000}%
\pgfsetstrokecolor{currentstroke}%
\pgfsetdash{}{0pt}%
\pgfsys@defobject{currentmarker}{\pgfqpoint{-0.048611in}{0.000000in}}{\pgfqpoint{-0.000000in}{0.000000in}}{%
\pgfpathmoveto{\pgfqpoint{-0.000000in}{0.000000in}}%
\pgfpathlineto{\pgfqpoint{-0.048611in}{0.000000in}}%
\pgfusepath{stroke,fill}%
}%
\begin{pgfscope}%
\pgfsys@transformshift{0.741055in}{3.996556in}%
\pgfsys@useobject{currentmarker}{}%
\end{pgfscope}%
\end{pgfscope}%
\begin{pgfscope}%
\definecolor{textcolor}{rgb}{0.000000,0.000000,0.000000}%
\pgfsetstrokecolor{textcolor}%
\pgfsetfillcolor{textcolor}%
\pgftext[x=0.422953in, y=3.943794in, left, base]{\color{textcolor}\sffamily\fontsize{10.000000}{12.000000}\selectfont 0.8}%
\end{pgfscope}%
\begin{pgfscope}%
\pgfsetbuttcap%
\pgfsetroundjoin%
\definecolor{currentfill}{rgb}{0.000000,0.000000,0.000000}%
\pgfsetfillcolor{currentfill}%
\pgfsetlinewidth{0.803000pt}%
\definecolor{currentstroke}{rgb}{0.000000,0.000000,0.000000}%
\pgfsetstrokecolor{currentstroke}%
\pgfsetdash{}{0pt}%
\pgfsys@defobject{currentmarker}{\pgfqpoint{-0.048611in}{0.000000in}}{\pgfqpoint{-0.000000in}{0.000000in}}{%
\pgfpathmoveto{\pgfqpoint{-0.000000in}{0.000000in}}%
\pgfpathlineto{\pgfqpoint{-0.048611in}{0.000000in}}%
\pgfusepath{stroke,fill}%
}%
\begin{pgfscope}%
\pgfsys@transformshift{0.741055in}{4.850000in}%
\pgfsys@useobject{currentmarker}{}%
\end{pgfscope}%
\end{pgfscope}%
\begin{pgfscope}%
\definecolor{textcolor}{rgb}{0.000000,0.000000,0.000000}%
\pgfsetstrokecolor{textcolor}%
\pgfsetfillcolor{textcolor}%
\pgftext[x=0.422953in, y=4.797238in, left, base]{\color{textcolor}\sffamily\fontsize{10.000000}{12.000000}\selectfont 1.0}%
\end{pgfscope}%
\begin{pgfscope}%
\definecolor{textcolor}{rgb}{0.000000,0.000000,0.000000}%
\pgfsetstrokecolor{textcolor}%
\pgfsetfillcolor{textcolor}%
\pgftext[x=0.367398in,y=2.716389in,,bottom,rotate=90.000000]{\color{textcolor}\sffamily\fontsize{10.000000}{12.000000}\selectfont median latency (s)}%
\end{pgfscope}%
\begin{pgfscope}%
\pgfpathrectangle{\pgfqpoint{0.741055in}{0.582778in}}{\pgfqpoint{4.195875in}{4.267222in}}%
\pgfusepath{clip}%
\pgfsetrectcap%
\pgfsetroundjoin%
\pgfsetlinewidth{2.258437pt}%
\definecolor{currentstroke}{rgb}{0.003922,0.450980,0.698039}%
\pgfsetstrokecolor{currentstroke}%
\pgfsetdash{}{0pt}%
\pgfpathmoveto{\pgfqpoint{0.931777in}{1.533301in}}%
\pgfpathlineto{\pgfqpoint{2.533566in}{1.541414in}}%
\pgfpathlineto{\pgfqpoint{2.861016in}{1.549554in}}%
\pgfpathlineto{\pgfqpoint{3.054798in}{1.557719in}}%
\pgfpathlineto{\pgfqpoint{3.192921in}{1.565909in}}%
\pgfpathlineto{\pgfqpoint{3.300324in}{1.574126in}}%
\pgfpathlineto{\pgfqpoint{3.388217in}{1.582368in}}%
\pgfpathlineto{\pgfqpoint{3.462609in}{1.590636in}}%
\pgfpathlineto{\pgfqpoint{3.527101in}{1.598930in}}%
\pgfpathlineto{\pgfqpoint{3.584022in}{1.607249in}}%
\pgfpathlineto{\pgfqpoint{3.634964in}{1.615594in}}%
\pgfpathlineto{\pgfqpoint{3.681064in}{1.623965in}}%
\pgfpathlineto{\pgfqpoint{3.723163in}{1.632362in}}%
\pgfpathlineto{\pgfqpoint{3.761901in}{1.640784in}}%
\pgfpathlineto{\pgfqpoint{3.797775in}{1.649232in}}%
\pgfpathlineto{\pgfqpoint{3.831179in}{1.657706in}}%
\pgfpathlineto{\pgfqpoint{3.862432in}{1.666206in}}%
\pgfpathlineto{\pgfqpoint{3.891794in}{1.674731in}}%
\pgfpathlineto{\pgfqpoint{3.919481in}{1.683282in}}%
\pgfpathlineto{\pgfqpoint{3.945674in}{1.691858in}}%
\pgfpathlineto{\pgfqpoint{3.970525in}{1.700461in}}%
\pgfpathlineto{\pgfqpoint{3.994166in}{1.709089in}}%
\pgfpathlineto{\pgfqpoint{4.016710in}{1.717743in}}%
\pgfpathlineto{\pgfqpoint{4.038252in}{1.726422in}}%
\pgfpathlineto{\pgfqpoint{4.058879in}{1.735128in}}%
\pgfpathlineto{\pgfqpoint{4.078665in}{1.743859in}}%
\pgfpathlineto{\pgfqpoint{4.097676in}{1.752616in}}%
\pgfpathlineto{\pgfqpoint{4.115971in}{1.761398in}}%
\pgfpathlineto{\pgfqpoint{4.133601in}{1.770206in}}%
\pgfpathlineto{\pgfqpoint{4.150613in}{1.779040in}}%
\pgfpathlineto{\pgfqpoint{4.167049in}{1.787900in}}%
\pgfpathlineto{\pgfqpoint{4.182947in}{1.796785in}}%
\pgfpathlineto{\pgfqpoint{4.198341in}{1.805697in}}%
\pgfpathlineto{\pgfqpoint{4.213262in}{1.814633in}}%
\pgfpathlineto{\pgfqpoint{4.227737in}{1.823596in}}%
\pgfpathlineto{\pgfqpoint{4.241794in}{1.832584in}}%
\pgfpathlineto{\pgfqpoint{4.255454in}{1.841598in}}%
\pgfpathlineto{\pgfqpoint{4.268741in}{1.850638in}}%
\pgfpathlineto{\pgfqpoint{4.281674in}{1.859704in}}%
\pgfpathlineto{\pgfqpoint{4.294272in}{1.868795in}}%
\pgfpathlineto{\pgfqpoint{4.306550in}{1.877912in}}%
\pgfpathlineto{\pgfqpoint{4.318526in}{1.887054in}}%
\pgfpathlineto{\pgfqpoint{4.330213in}{1.896223in}}%
\pgfpathlineto{\pgfqpoint{4.341626in}{1.905417in}}%
\pgfpathlineto{\pgfqpoint{4.352776in}{1.914637in}}%
\pgfpathlineto{\pgfqpoint{4.363676in}{1.923882in}}%
\pgfpathlineto{\pgfqpoint{4.374337in}{1.933154in}}%
\pgfpathlineto{\pgfqpoint{4.384769in}{1.942451in}}%
\pgfpathlineto{\pgfqpoint{4.394981in}{1.951773in}}%
\pgfpathlineto{\pgfqpoint{4.404983in}{1.961122in}}%
\pgfpathlineto{\pgfqpoint{4.414782in}{1.970496in}}%
\pgfpathlineto{\pgfqpoint{4.424388in}{1.979896in}}%
\pgfpathlineto{\pgfqpoint{4.433808in}{1.989322in}}%
\pgfpathlineto{\pgfqpoint{4.443048in}{1.998773in}}%
\pgfpathlineto{\pgfqpoint{4.452116in}{2.008250in}}%
\pgfpathlineto{\pgfqpoint{4.461017in}{2.017753in}}%
\pgfpathlineto{\pgfqpoint{4.469758in}{2.027282in}}%
\pgfpathlineto{\pgfqpoint{4.478345in}{2.036836in}}%
\pgfpathlineto{\pgfqpoint{4.486782in}{2.046416in}}%
\pgfpathlineto{\pgfqpoint{4.495075in}{2.056022in}}%
\pgfpathlineto{\pgfqpoint{4.503229in}{2.065653in}}%
\pgfpathlineto{\pgfqpoint{4.511248in}{2.075310in}}%
\pgfpathlineto{\pgfqpoint{4.519137in}{2.084993in}}%
\pgfpathlineto{\pgfqpoint{4.526899in}{2.094702in}}%
\pgfpathlineto{\pgfqpoint{4.534540in}{2.104436in}}%
\pgfpathlineto{\pgfqpoint{4.542062in}{2.114196in}}%
\pgfpathlineto{\pgfqpoint{4.549469in}{2.123982in}}%
\pgfpathlineto{\pgfqpoint{4.556765in}{2.133794in}}%
\pgfpathlineto{\pgfqpoint{4.563953in}{2.143631in}}%
\pgfpathlineto{\pgfqpoint{4.571036in}{2.153494in}}%
\pgfpathlineto{\pgfqpoint{4.578017in}{2.163383in}}%
\pgfpathlineto{\pgfqpoint{4.584899in}{2.173297in}}%
\pgfpathlineto{\pgfqpoint{4.591685in}{2.183238in}}%
\pgfpathlineto{\pgfqpoint{4.598378in}{2.193203in}}%
\pgfpathlineto{\pgfqpoint{4.604979in}{2.203195in}}%
\pgfpathlineto{\pgfqpoint{4.611492in}{2.213213in}}%
\pgfpathlineto{\pgfqpoint{4.617919in}{2.223256in}}%
\pgfpathlineto{\pgfqpoint{4.624261in}{2.233324in}}%
\pgfpathlineto{\pgfqpoint{4.630522in}{2.243419in}}%
\pgfpathlineto{\pgfqpoint{4.636703in}{2.253539in}}%
\pgfpathlineto{\pgfqpoint{4.642807in}{2.263685in}}%
\pgfpathlineto{\pgfqpoint{4.648834in}{2.273857in}}%
\pgfpathlineto{\pgfqpoint{4.654788in}{2.284054in}}%
\pgfpathlineto{\pgfqpoint{4.660670in}{2.294278in}}%
\pgfpathlineto{\pgfqpoint{4.666481in}{2.304527in}}%
\pgfpathlineto{\pgfqpoint{4.672223in}{2.314801in}}%
\pgfpathlineto{\pgfqpoint{4.677899in}{2.325102in}}%
\pgfpathlineto{\pgfqpoint{4.683508in}{2.335428in}}%
\pgfpathlineto{\pgfqpoint{4.689054in}{2.345779in}}%
\pgfpathlineto{\pgfqpoint{4.694537in}{2.356157in}}%
\pgfpathlineto{\pgfqpoint{4.699959in}{2.366560in}}%
\pgfpathlineto{\pgfqpoint{4.705321in}{2.376989in}}%
\pgfpathlineto{\pgfqpoint{4.710624in}{2.387444in}}%
\pgfpathlineto{\pgfqpoint{4.715870in}{2.397924in}}%
\pgfpathlineto{\pgfqpoint{4.721060in}{2.408431in}}%
\pgfpathlineto{\pgfqpoint{4.726195in}{2.418962in}}%
\pgfpathlineto{\pgfqpoint{4.731276in}{2.429520in}}%
\pgfpathlineto{\pgfqpoint{4.736305in}{2.440103in}}%
\pgfpathlineto{\pgfqpoint{4.741282in}{2.450713in}}%
\pgfpathlineto{\pgfqpoint{4.746209in}{2.461347in}}%
\pgfusepath{stroke}%
\end{pgfscope}%
\begin{pgfscope}%
\pgfsetrectcap%
\pgfsetmiterjoin%
\pgfsetlinewidth{0.803000pt}%
\definecolor{currentstroke}{rgb}{0.000000,0.000000,0.000000}%
\pgfsetstrokecolor{currentstroke}%
\pgfsetdash{}{0pt}%
\pgfpathmoveto{\pgfqpoint{0.741055in}{0.582778in}}%
\pgfpathlineto{\pgfqpoint{0.741055in}{4.850000in}}%
\pgfusepath{stroke}%
\end{pgfscope}%
\begin{pgfscope}%
\pgfsetrectcap%
\pgfsetmiterjoin%
\pgfsetlinewidth{0.803000pt}%
\definecolor{currentstroke}{rgb}{0.000000,0.000000,0.000000}%
\pgfsetstrokecolor{currentstroke}%
\pgfsetdash{}{0pt}%
\pgfpathmoveto{\pgfqpoint{0.741055in}{0.582778in}}%
\pgfpathlineto{\pgfqpoint{4.936930in}{0.582778in}}%
\pgfusepath{stroke}%
\end{pgfscope}%
\begin{pgfscope}%
\pgfsetbuttcap%
\pgfsetmiterjoin%
\definecolor{currentfill}{rgb}{1.000000,1.000000,1.000000}%
\pgfsetfillcolor{currentfill}%
\pgfsetfillopacity{0.800000}%
\pgfsetlinewidth{1.003750pt}%
\definecolor{currentstroke}{rgb}{0.800000,0.800000,0.800000}%
\pgfsetstrokecolor{currentstroke}%
\pgfsetstrokeopacity{0.800000}%
\pgfsetdash{}{0pt}%
\pgfpathmoveto{\pgfqpoint{0.838277in}{4.331174in}}%
\pgfpathlineto{\pgfqpoint{1.676439in}{4.331174in}}%
\pgfpathquadraticcurveto{\pgfqpoint{1.704217in}{4.331174in}}{\pgfqpoint{1.704217in}{4.358952in}}%
\pgfpathlineto{\pgfqpoint{1.704217in}{4.752778in}}%
\pgfpathquadraticcurveto{\pgfqpoint{1.704217in}{4.780556in}}{\pgfqpoint{1.676439in}{4.780556in}}%
\pgfpathlineto{\pgfqpoint{0.838277in}{4.780556in}}%
\pgfpathquadraticcurveto{\pgfqpoint{0.810499in}{4.780556in}}{\pgfqpoint{0.810499in}{4.752778in}}%
\pgfpathlineto{\pgfqpoint{0.810499in}{4.358952in}}%
\pgfpathquadraticcurveto{\pgfqpoint{0.810499in}{4.331174in}}{\pgfqpoint{0.838277in}{4.331174in}}%
\pgfpathlineto{\pgfqpoint{0.838277in}{4.331174in}}%
\pgfpathclose%
\pgfusepath{stroke,fill}%
\end{pgfscope}%
\begin{pgfscope}%
\definecolor{textcolor}{rgb}{0.000000,0.000000,0.000000}%
\pgfsetstrokecolor{textcolor}%
\pgfsetfillcolor{textcolor}%
\pgftext[x=0.866055in,y=4.619477in,left,base]{\color{textcolor}\sffamily\fontsize{10.000000}{12.000000}\selectfont node count}%
\end{pgfscope}%
\begin{pgfscope}%
\pgfsetbuttcap%
\pgfsetroundjoin%
\definecolor{currentfill}{rgb}{0.003922,0.450980,0.698039}%
\pgfsetfillcolor{currentfill}%
\pgfsetfillopacity{0.800000}%
\pgfsetlinewidth{1.003750pt}%
\definecolor{currentstroke}{rgb}{0.003922,0.450980,0.698039}%
\pgfsetstrokecolor{currentstroke}%
\pgfsetstrokeopacity{0.800000}%
\pgfsetdash{}{0pt}%
\pgfsys@defobject{currentmarker}{\pgfqpoint{-0.041667in}{-0.041667in}}{\pgfqpoint{0.041667in}{0.041667in}}{%
\pgfpathmoveto{\pgfqpoint{0.000000in}{-0.041667in}}%
\pgfpathcurveto{\pgfqpoint{0.011050in}{-0.041667in}}{\pgfqpoint{0.021649in}{-0.037276in}}{\pgfqpoint{0.029463in}{-0.029463in}}%
\pgfpathcurveto{\pgfqpoint{0.037276in}{-0.021649in}}{\pgfqpoint{0.041667in}{-0.011050in}}{\pgfqpoint{0.041667in}{0.000000in}}%
\pgfpathcurveto{\pgfqpoint{0.041667in}{0.011050in}}{\pgfqpoint{0.037276in}{0.021649in}}{\pgfqpoint{0.029463in}{0.029463in}}%
\pgfpathcurveto{\pgfqpoint{0.021649in}{0.037276in}}{\pgfqpoint{0.011050in}{0.041667in}}{\pgfqpoint{0.000000in}{0.041667in}}%
\pgfpathcurveto{\pgfqpoint{-0.011050in}{0.041667in}}{\pgfqpoint{-0.021649in}{0.037276in}}{\pgfqpoint{-0.029463in}{0.029463in}}%
\pgfpathcurveto{\pgfqpoint{-0.037276in}{0.021649in}}{\pgfqpoint{-0.041667in}{0.011050in}}{\pgfqpoint{-0.041667in}{0.000000in}}%
\pgfpathcurveto{\pgfqpoint{-0.041667in}{-0.011050in}}{\pgfqpoint{-0.037276in}{-0.021649in}}{\pgfqpoint{-0.029463in}{-0.029463in}}%
\pgfpathcurveto{\pgfqpoint{-0.021649in}{-0.037276in}}{\pgfqpoint{-0.011050in}{-0.041667in}}{\pgfqpoint{0.000000in}{-0.041667in}}%
\pgfpathlineto{\pgfqpoint{0.000000in}{-0.041667in}}%
\pgfpathclose%
\pgfusepath{stroke,fill}%
}%
\begin{pgfscope}%
\pgfsys@transformshift{1.157620in}{4.452078in}%
\pgfsys@useobject{currentmarker}{}%
\end{pgfscope}%
\end{pgfscope}%
\begin{pgfscope}%
\definecolor{textcolor}{rgb}{0.000000,0.000000,0.000000}%
\pgfsetstrokecolor{textcolor}%
\pgfsetfillcolor{textcolor}%
\pgftext[x=1.407620in,y=4.415620in,left,base]{\color{textcolor}\sffamily\fontsize{10.000000}{12.000000}\selectfont 4}%
\end{pgfscope}%
\end{pgfpicture}%
\makeatother%
\endgroup%
}
\caption{Benchmarking of goodput and median latency while varying throughputs and node counts.}
\label{goodputlatencynodes}
\end{figure}

% \begin{figure}[h!]
% \centering
% %% Creator: Matplotlib, PGF backend
%%
%% To include the figure in your LaTeX document, write
%%   \input{<filename>.pgf}
%%
%% Make sure the required packages are loaded in your preamble
%%   \usepackage{pgf}
%%
%% Also ensure that all the required font packages are loaded; for instance,
%% the lmodern package is sometimes necessary when using math font.
%%   \usepackage{lmodern}
%%
%% Figures using additional raster images can only be included by \input if
%% they are in the same directory as the main LaTeX file. For loading figures
%% from other directories you can use the `import` package
%%   \usepackage{import}
%%
%% and then include the figures with
%%   \import{<path to file>}{<filename>.pgf}
%%
%% Matplotlib used the following preamble
%%   
%%   \usepackage{fontspec}
%%   \setmainfont{DejaVuSerif.ttf}[Path=\detokenize{/opt/homebrew/lib/python3.10/site-packages/matplotlib/mpl-data/fonts/ttf/}]
%%   \setsansfont{DejaVuSans.ttf}[Path=\detokenize{/opt/homebrew/lib/python3.10/site-packages/matplotlib/mpl-data/fonts/ttf/}]
%%   \setmonofont{DejaVuSansMono.ttf}[Path=\detokenize{/opt/homebrew/lib/python3.10/site-packages/matplotlib/mpl-data/fonts/ttf/}]
%%   \makeatletter\@ifpackageloaded{underscore}{}{\usepackage[strings]{underscore}}\makeatother
%%
\begingroup%
\makeatletter%
\begin{pgfpicture}%
\pgfpathrectangle{\pgfpointorigin}{\pgfqpoint{5.580000in}{5.000000in}}%
\pgfusepath{use as bounding box, clip}%
\begin{pgfscope}%
\pgfsetbuttcap%
\pgfsetmiterjoin%
\definecolor{currentfill}{rgb}{1.000000,1.000000,1.000000}%
\pgfsetfillcolor{currentfill}%
\pgfsetlinewidth{0.000000pt}%
\definecolor{currentstroke}{rgb}{1.000000,1.000000,1.000000}%
\pgfsetstrokecolor{currentstroke}%
\pgfsetdash{}{0pt}%
\pgfpathmoveto{\pgfqpoint{0.000000in}{0.000000in}}%
\pgfpathlineto{\pgfqpoint{5.580000in}{0.000000in}}%
\pgfpathlineto{\pgfqpoint{5.580000in}{5.000000in}}%
\pgfpathlineto{\pgfqpoint{0.000000in}{5.000000in}}%
\pgfpathlineto{\pgfqpoint{0.000000in}{0.000000in}}%
\pgfpathclose%
\pgfusepath{fill}%
\end{pgfscope}%
\begin{pgfscope}%
\pgfsetbuttcap%
\pgfsetmiterjoin%
\definecolor{currentfill}{rgb}{1.000000,1.000000,1.000000}%
\pgfsetfillcolor{currentfill}%
\pgfsetlinewidth{0.000000pt}%
\definecolor{currentstroke}{rgb}{0.000000,0.000000,0.000000}%
\pgfsetstrokecolor{currentstroke}%
\pgfsetstrokeopacity{0.000000}%
\pgfsetdash{}{0pt}%
\pgfpathmoveto{\pgfqpoint{1.038190in}{0.582778in}}%
\pgfpathlineto{\pgfqpoint{4.936930in}{0.582778in}}%
\pgfpathlineto{\pgfqpoint{4.936930in}{4.850000in}}%
\pgfpathlineto{\pgfqpoint{1.038190in}{4.850000in}}%
\pgfpathlineto{\pgfqpoint{1.038190in}{0.582778in}}%
\pgfpathclose%
\pgfusepath{fill}%
\end{pgfscope}%
\begin{pgfscope}%
\pgfpathrectangle{\pgfqpoint{1.038190in}{0.582778in}}{\pgfqpoint{3.898740in}{4.267222in}}%
\pgfusepath{clip}%
\pgfsetbuttcap%
\pgfsetroundjoin%
\definecolor{currentfill}{rgb}{0.003922,0.450980,0.698039}%
\pgfsetfillcolor{currentfill}%
\pgfsetfillopacity{0.800000}%
\pgfsetlinewidth{1.003750pt}%
\definecolor{currentstroke}{rgb}{0.003922,0.450980,0.698039}%
\pgfsetstrokecolor{currentstroke}%
\pgfsetstrokeopacity{0.800000}%
\pgfsetdash{}{0pt}%
\pgfsys@defobject{currentmarker}{\pgfqpoint{-0.041667in}{-0.041667in}}{\pgfqpoint{0.041667in}{0.041667in}}{%
\pgfpathmoveto{\pgfqpoint{0.000000in}{-0.041667in}}%
\pgfpathcurveto{\pgfqpoint{0.011050in}{-0.041667in}}{\pgfqpoint{0.021649in}{-0.037276in}}{\pgfqpoint{0.029463in}{-0.029463in}}%
\pgfpathcurveto{\pgfqpoint{0.037276in}{-0.021649in}}{\pgfqpoint{0.041667in}{-0.011050in}}{\pgfqpoint{0.041667in}{0.000000in}}%
\pgfpathcurveto{\pgfqpoint{0.041667in}{0.011050in}}{\pgfqpoint{0.037276in}{0.021649in}}{\pgfqpoint{0.029463in}{0.029463in}}%
\pgfpathcurveto{\pgfqpoint{0.021649in}{0.037276in}}{\pgfqpoint{0.011050in}{0.041667in}}{\pgfqpoint{0.000000in}{0.041667in}}%
\pgfpathcurveto{\pgfqpoint{-0.011050in}{0.041667in}}{\pgfqpoint{-0.021649in}{0.037276in}}{\pgfqpoint{-0.029463in}{0.029463in}}%
\pgfpathcurveto{\pgfqpoint{-0.037276in}{0.021649in}}{\pgfqpoint{-0.041667in}{0.011050in}}{\pgfqpoint{-0.041667in}{0.000000in}}%
\pgfpathcurveto{\pgfqpoint{-0.041667in}{-0.011050in}}{\pgfqpoint{-0.037276in}{-0.021649in}}{\pgfqpoint{-0.029463in}{-0.029463in}}%
\pgfpathcurveto{\pgfqpoint{-0.021649in}{-0.037276in}}{\pgfqpoint{-0.011050in}{-0.041667in}}{\pgfqpoint{0.000000in}{-0.041667in}}%
\pgfpathlineto{\pgfqpoint{0.000000in}{-0.041667in}}%
\pgfpathclose%
\pgfusepath{stroke,fill}%
}%
\begin{pgfscope}%
\pgfsys@transformshift{1.215405in}{4.841279in}%
\pgfsys@useobject{currentmarker}{}%
\end{pgfscope}%
\begin{pgfscope}%
\pgfsys@transformshift{4.759715in}{4.832483in}%
\pgfsys@useobject{currentmarker}{}%
\end{pgfscope}%
\end{pgfscope}%
\begin{pgfscope}%
\pgfsetbuttcap%
\pgfsetroundjoin%
\definecolor{currentfill}{rgb}{0.000000,0.000000,0.000000}%
\pgfsetfillcolor{currentfill}%
\pgfsetlinewidth{0.602250pt}%
\definecolor{currentstroke}{rgb}{0.000000,0.000000,0.000000}%
\pgfsetstrokecolor{currentstroke}%
\pgfsetdash{}{0pt}%
\pgfsys@defobject{currentmarker}{\pgfqpoint{0.000000in}{-0.027778in}}{\pgfqpoint{0.000000in}{0.000000in}}{%
\pgfpathmoveto{\pgfqpoint{0.000000in}{0.000000in}}%
\pgfpathlineto{\pgfqpoint{0.000000in}{-0.027778in}}%
\pgfusepath{stroke,fill}%
}%
\begin{pgfscope}%
\pgfsys@transformshift{1.685645in}{0.582778in}%
\pgfsys@useobject{currentmarker}{}%
\end{pgfscope}%
\end{pgfscope}%
\begin{pgfscope}%
\definecolor{textcolor}{rgb}{0.000000,0.000000,0.000000}%
\pgfsetstrokecolor{textcolor}%
\pgfsetfillcolor{textcolor}%
\pgftext[x=1.685645in,y=0.507778in,,top]{\color{textcolor}\sffamily\fontsize{10.000000}{12.000000}\selectfont \(\displaystyle {4.58\times10^{1}}\)}%
\end{pgfscope}%
\begin{pgfscope}%
\pgfsetbuttcap%
\pgfsetroundjoin%
\definecolor{currentfill}{rgb}{0.000000,0.000000,0.000000}%
\pgfsetfillcolor{currentfill}%
\pgfsetlinewidth{0.602250pt}%
\definecolor{currentstroke}{rgb}{0.000000,0.000000,0.000000}%
\pgfsetstrokecolor{currentstroke}%
\pgfsetdash{}{0pt}%
\pgfsys@defobject{currentmarker}{\pgfqpoint{0.000000in}{-0.027778in}}{\pgfqpoint{0.000000in}{0.000000in}}{%
\pgfpathmoveto{\pgfqpoint{0.000000in}{0.000000in}}%
\pgfpathlineto{\pgfqpoint{0.000000in}{-0.027778in}}%
\pgfusepath{stroke,fill}%
}%
\begin{pgfscope}%
\pgfsys@transformshift{2.481308in}{0.582778in}%
\pgfsys@useobject{currentmarker}{}%
\end{pgfscope}%
\end{pgfscope}%
\begin{pgfscope}%
\definecolor{textcolor}{rgb}{0.000000,0.000000,0.000000}%
\pgfsetstrokecolor{textcolor}%
\pgfsetfillcolor{textcolor}%
\pgftext[x=2.481308in,y=0.507778in,,top]{\color{textcolor}\sffamily\fontsize{10.000000}{12.000000}\selectfont \(\displaystyle {4.59\times10^{1}}\)}%
\end{pgfscope}%
\begin{pgfscope}%
\pgfsetbuttcap%
\pgfsetroundjoin%
\definecolor{currentfill}{rgb}{0.000000,0.000000,0.000000}%
\pgfsetfillcolor{currentfill}%
\pgfsetlinewidth{0.602250pt}%
\definecolor{currentstroke}{rgb}{0.000000,0.000000,0.000000}%
\pgfsetstrokecolor{currentstroke}%
\pgfsetdash{}{0pt}%
\pgfsys@defobject{currentmarker}{\pgfqpoint{0.000000in}{-0.027778in}}{\pgfqpoint{0.000000in}{0.000000in}}{%
\pgfpathmoveto{\pgfqpoint{0.000000in}{0.000000in}}%
\pgfpathlineto{\pgfqpoint{0.000000in}{-0.027778in}}%
\pgfusepath{stroke,fill}%
}%
\begin{pgfscope}%
\pgfsys@transformshift{3.275238in}{0.582778in}%
\pgfsys@useobject{currentmarker}{}%
\end{pgfscope}%
\end{pgfscope}%
\begin{pgfscope}%
\definecolor{textcolor}{rgb}{0.000000,0.000000,0.000000}%
\pgfsetstrokecolor{textcolor}%
\pgfsetfillcolor{textcolor}%
\pgftext[x=3.275238in,y=0.507778in,,top]{\color{textcolor}\sffamily\fontsize{10.000000}{12.000000}\selectfont \(\displaystyle {4.6\times10^{1}}\)}%
\end{pgfscope}%
\begin{pgfscope}%
\pgfsetbuttcap%
\pgfsetroundjoin%
\definecolor{currentfill}{rgb}{0.000000,0.000000,0.000000}%
\pgfsetfillcolor{currentfill}%
\pgfsetlinewidth{0.602250pt}%
\definecolor{currentstroke}{rgb}{0.000000,0.000000,0.000000}%
\pgfsetstrokecolor{currentstroke}%
\pgfsetdash{}{0pt}%
\pgfsys@defobject{currentmarker}{\pgfqpoint{0.000000in}{-0.027778in}}{\pgfqpoint{0.000000in}{0.000000in}}{%
\pgfpathmoveto{\pgfqpoint{0.000000in}{0.000000in}}%
\pgfpathlineto{\pgfqpoint{0.000000in}{-0.027778in}}%
\pgfusepath{stroke,fill}%
}%
\begin{pgfscope}%
\pgfsys@transformshift{4.067445in}{0.582778in}%
\pgfsys@useobject{currentmarker}{}%
\end{pgfscope}%
\end{pgfscope}%
\begin{pgfscope}%
\definecolor{textcolor}{rgb}{0.000000,0.000000,0.000000}%
\pgfsetstrokecolor{textcolor}%
\pgfsetfillcolor{textcolor}%
\pgftext[x=4.067445in,y=0.507778in,,top]{\color{textcolor}\sffamily\fontsize{10.000000}{12.000000}\selectfont \(\displaystyle {4.61\times10^{1}}\)}%
\end{pgfscope}%
\begin{pgfscope}%
\pgfsetbuttcap%
\pgfsetroundjoin%
\definecolor{currentfill}{rgb}{0.000000,0.000000,0.000000}%
\pgfsetfillcolor{currentfill}%
\pgfsetlinewidth{0.602250pt}%
\definecolor{currentstroke}{rgb}{0.000000,0.000000,0.000000}%
\pgfsetstrokecolor{currentstroke}%
\pgfsetdash{}{0pt}%
\pgfsys@defobject{currentmarker}{\pgfqpoint{0.000000in}{-0.027778in}}{\pgfqpoint{0.000000in}{0.000000in}}{%
\pgfpathmoveto{\pgfqpoint{0.000000in}{0.000000in}}%
\pgfpathlineto{\pgfqpoint{0.000000in}{-0.027778in}}%
\pgfusepath{stroke,fill}%
}%
\begin{pgfscope}%
\pgfsys@transformshift{4.857935in}{0.582778in}%
\pgfsys@useobject{currentmarker}{}%
\end{pgfscope}%
\end{pgfscope}%
\begin{pgfscope}%
\definecolor{textcolor}{rgb}{0.000000,0.000000,0.000000}%
\pgfsetstrokecolor{textcolor}%
\pgfsetfillcolor{textcolor}%
\pgftext[x=4.857935in,y=0.507778in,,top]{\color{textcolor}\sffamily\fontsize{10.000000}{12.000000}\selectfont \(\displaystyle {4.62\times10^{1}}\)}%
\end{pgfscope}%
\begin{pgfscope}%
\definecolor{textcolor}{rgb}{0.000000,0.000000,0.000000}%
\pgfsetstrokecolor{textcolor}%
\pgfsetfillcolor{textcolor}%
\pgftext[x=2.987560in,y=0.317809in,,top]{\color{textcolor}\sffamily\fontsize{10.000000}{12.000000}\selectfont goodput (req/s)}%
\end{pgfscope}%
\begin{pgfscope}%
\pgfsetbuttcap%
\pgfsetroundjoin%
\definecolor{currentfill}{rgb}{0.000000,0.000000,0.000000}%
\pgfsetfillcolor{currentfill}%
\pgfsetlinewidth{0.803000pt}%
\definecolor{currentstroke}{rgb}{0.000000,0.000000,0.000000}%
\pgfsetstrokecolor{currentstroke}%
\pgfsetdash{}{0pt}%
\pgfsys@defobject{currentmarker}{\pgfqpoint{-0.048611in}{0.000000in}}{\pgfqpoint{-0.000000in}{0.000000in}}{%
\pgfpathmoveto{\pgfqpoint{-0.000000in}{0.000000in}}%
\pgfpathlineto{\pgfqpoint{-0.048611in}{0.000000in}}%
\pgfusepath{stroke,fill}%
}%
\begin{pgfscope}%
\pgfsys@transformshift{1.038190in}{0.582778in}%
\pgfsys@useobject{currentmarker}{}%
\end{pgfscope}%
\end{pgfscope}%
\begin{pgfscope}%
\definecolor{textcolor}{rgb}{0.000000,0.000000,0.000000}%
\pgfsetstrokecolor{textcolor}%
\pgfsetfillcolor{textcolor}%
\pgftext[x=0.720088in, y=0.530016in, left, base]{\color{textcolor}\sffamily\fontsize{10.000000}{12.000000}\selectfont 0.0}%
\end{pgfscope}%
\begin{pgfscope}%
\pgfsetbuttcap%
\pgfsetroundjoin%
\definecolor{currentfill}{rgb}{0.000000,0.000000,0.000000}%
\pgfsetfillcolor{currentfill}%
\pgfsetlinewidth{0.803000pt}%
\definecolor{currentstroke}{rgb}{0.000000,0.000000,0.000000}%
\pgfsetstrokecolor{currentstroke}%
\pgfsetdash{}{0pt}%
\pgfsys@defobject{currentmarker}{\pgfqpoint{-0.048611in}{0.000000in}}{\pgfqpoint{-0.000000in}{0.000000in}}{%
\pgfpathmoveto{\pgfqpoint{-0.000000in}{0.000000in}}%
\pgfpathlineto{\pgfqpoint{-0.048611in}{0.000000in}}%
\pgfusepath{stroke,fill}%
}%
\begin{pgfscope}%
\pgfsys@transformshift{1.038190in}{1.436222in}%
\pgfsys@useobject{currentmarker}{}%
\end{pgfscope}%
\end{pgfscope}%
\begin{pgfscope}%
\definecolor{textcolor}{rgb}{0.000000,0.000000,0.000000}%
\pgfsetstrokecolor{textcolor}%
\pgfsetfillcolor{textcolor}%
\pgftext[x=0.720088in, y=1.383461in, left, base]{\color{textcolor}\sffamily\fontsize{10.000000}{12.000000}\selectfont 0.2}%
\end{pgfscope}%
\begin{pgfscope}%
\pgfsetbuttcap%
\pgfsetroundjoin%
\definecolor{currentfill}{rgb}{0.000000,0.000000,0.000000}%
\pgfsetfillcolor{currentfill}%
\pgfsetlinewidth{0.803000pt}%
\definecolor{currentstroke}{rgb}{0.000000,0.000000,0.000000}%
\pgfsetstrokecolor{currentstroke}%
\pgfsetdash{}{0pt}%
\pgfsys@defobject{currentmarker}{\pgfqpoint{-0.048611in}{0.000000in}}{\pgfqpoint{-0.000000in}{0.000000in}}{%
\pgfpathmoveto{\pgfqpoint{-0.000000in}{0.000000in}}%
\pgfpathlineto{\pgfqpoint{-0.048611in}{0.000000in}}%
\pgfusepath{stroke,fill}%
}%
\begin{pgfscope}%
\pgfsys@transformshift{1.038190in}{2.289667in}%
\pgfsys@useobject{currentmarker}{}%
\end{pgfscope}%
\end{pgfscope}%
\begin{pgfscope}%
\definecolor{textcolor}{rgb}{0.000000,0.000000,0.000000}%
\pgfsetstrokecolor{textcolor}%
\pgfsetfillcolor{textcolor}%
\pgftext[x=0.720088in, y=2.236905in, left, base]{\color{textcolor}\sffamily\fontsize{10.000000}{12.000000}\selectfont 0.4}%
\end{pgfscope}%
\begin{pgfscope}%
\pgfsetbuttcap%
\pgfsetroundjoin%
\definecolor{currentfill}{rgb}{0.000000,0.000000,0.000000}%
\pgfsetfillcolor{currentfill}%
\pgfsetlinewidth{0.803000pt}%
\definecolor{currentstroke}{rgb}{0.000000,0.000000,0.000000}%
\pgfsetstrokecolor{currentstroke}%
\pgfsetdash{}{0pt}%
\pgfsys@defobject{currentmarker}{\pgfqpoint{-0.048611in}{0.000000in}}{\pgfqpoint{-0.000000in}{0.000000in}}{%
\pgfpathmoveto{\pgfqpoint{-0.000000in}{0.000000in}}%
\pgfpathlineto{\pgfqpoint{-0.048611in}{0.000000in}}%
\pgfusepath{stroke,fill}%
}%
\begin{pgfscope}%
\pgfsys@transformshift{1.038190in}{3.143111in}%
\pgfsys@useobject{currentmarker}{}%
\end{pgfscope}%
\end{pgfscope}%
\begin{pgfscope}%
\definecolor{textcolor}{rgb}{0.000000,0.000000,0.000000}%
\pgfsetstrokecolor{textcolor}%
\pgfsetfillcolor{textcolor}%
\pgftext[x=0.720088in, y=3.090350in, left, base]{\color{textcolor}\sffamily\fontsize{10.000000}{12.000000}\selectfont 0.6}%
\end{pgfscope}%
\begin{pgfscope}%
\pgfsetbuttcap%
\pgfsetroundjoin%
\definecolor{currentfill}{rgb}{0.000000,0.000000,0.000000}%
\pgfsetfillcolor{currentfill}%
\pgfsetlinewidth{0.803000pt}%
\definecolor{currentstroke}{rgb}{0.000000,0.000000,0.000000}%
\pgfsetstrokecolor{currentstroke}%
\pgfsetdash{}{0pt}%
\pgfsys@defobject{currentmarker}{\pgfqpoint{-0.048611in}{0.000000in}}{\pgfqpoint{-0.000000in}{0.000000in}}{%
\pgfpathmoveto{\pgfqpoint{-0.000000in}{0.000000in}}%
\pgfpathlineto{\pgfqpoint{-0.048611in}{0.000000in}}%
\pgfusepath{stroke,fill}%
}%
\begin{pgfscope}%
\pgfsys@transformshift{1.038190in}{3.996556in}%
\pgfsys@useobject{currentmarker}{}%
\end{pgfscope}%
\end{pgfscope}%
\begin{pgfscope}%
\definecolor{textcolor}{rgb}{0.000000,0.000000,0.000000}%
\pgfsetstrokecolor{textcolor}%
\pgfsetfillcolor{textcolor}%
\pgftext[x=0.720088in, y=3.943794in, left, base]{\color{textcolor}\sffamily\fontsize{10.000000}{12.000000}\selectfont 0.8}%
\end{pgfscope}%
\begin{pgfscope}%
\pgfsetbuttcap%
\pgfsetroundjoin%
\definecolor{currentfill}{rgb}{0.000000,0.000000,0.000000}%
\pgfsetfillcolor{currentfill}%
\pgfsetlinewidth{0.803000pt}%
\definecolor{currentstroke}{rgb}{0.000000,0.000000,0.000000}%
\pgfsetstrokecolor{currentstroke}%
\pgfsetdash{}{0pt}%
\pgfsys@defobject{currentmarker}{\pgfqpoint{-0.048611in}{0.000000in}}{\pgfqpoint{-0.000000in}{0.000000in}}{%
\pgfpathmoveto{\pgfqpoint{-0.000000in}{0.000000in}}%
\pgfpathlineto{\pgfqpoint{-0.048611in}{0.000000in}}%
\pgfusepath{stroke,fill}%
}%
\begin{pgfscope}%
\pgfsys@transformshift{1.038190in}{4.850000in}%
\pgfsys@useobject{currentmarker}{}%
\end{pgfscope}%
\end{pgfscope}%
\begin{pgfscope}%
\definecolor{textcolor}{rgb}{0.000000,0.000000,0.000000}%
\pgfsetstrokecolor{textcolor}%
\pgfsetfillcolor{textcolor}%
\pgftext[x=0.720088in, y=4.797238in, left, base]{\color{textcolor}\sffamily\fontsize{10.000000}{12.000000}\selectfont 1.0}%
\end{pgfscope}%
\begin{pgfscope}%
\definecolor{textcolor}{rgb}{0.000000,0.000000,0.000000}%
\pgfsetstrokecolor{textcolor}%
\pgfsetfillcolor{textcolor}%
\pgftext[x=0.664533in,y=2.716389in,,bottom,rotate=90.000000]{\color{textcolor}\sffamily\fontsize{10.000000}{12.000000}\selectfont 95\%ile latency (s)}%
\end{pgfscope}%
\begin{pgfscope}%
\pgfpathrectangle{\pgfqpoint{1.038190in}{0.582778in}}{\pgfqpoint{3.898740in}{4.267222in}}%
\pgfusepath{clip}%
\pgfsetrectcap%
\pgfsetroundjoin%
\pgfsetlinewidth{2.258437pt}%
\definecolor{currentstroke}{rgb}{0.003922,0.450980,0.698039}%
\pgfsetstrokecolor{currentstroke}%
\pgfsetdash{}{0pt}%
\pgfpathmoveto{\pgfqpoint{1.215405in}{4.841279in}}%
\pgfpathlineto{\pgfqpoint{1.251379in}{4.841192in}}%
\pgfpathlineto{\pgfqpoint{1.287350in}{4.841104in}}%
\pgfpathlineto{\pgfqpoint{1.323316in}{4.841016in}}%
\pgfpathlineto{\pgfqpoint{1.359279in}{4.840928in}}%
\pgfpathlineto{\pgfqpoint{1.395239in}{4.840840in}}%
\pgfpathlineto{\pgfqpoint{1.431195in}{4.840752in}}%
\pgfpathlineto{\pgfqpoint{1.467148in}{4.840665in}}%
\pgfpathlineto{\pgfqpoint{1.503097in}{4.840577in}}%
\pgfpathlineto{\pgfqpoint{1.539042in}{4.840489in}}%
\pgfpathlineto{\pgfqpoint{1.574984in}{4.840401in}}%
\pgfpathlineto{\pgfqpoint{1.610922in}{4.840313in}}%
\pgfpathlineto{\pgfqpoint{1.646857in}{4.840225in}}%
\pgfpathlineto{\pgfqpoint{1.682788in}{4.840137in}}%
\pgfpathlineto{\pgfqpoint{1.718716in}{4.840049in}}%
\pgfpathlineto{\pgfqpoint{1.754640in}{4.839961in}}%
\pgfpathlineto{\pgfqpoint{1.790561in}{4.839872in}}%
\pgfpathlineto{\pgfqpoint{1.826478in}{4.839784in}}%
\pgfpathlineto{\pgfqpoint{1.862392in}{4.839696in}}%
\pgfpathlineto{\pgfqpoint{1.898302in}{4.839608in}}%
\pgfpathlineto{\pgfqpoint{1.934208in}{4.839520in}}%
\pgfpathlineto{\pgfqpoint{1.970111in}{4.839432in}}%
\pgfpathlineto{\pgfqpoint{2.006011in}{4.839343in}}%
\pgfpathlineto{\pgfqpoint{2.041907in}{4.839255in}}%
\pgfpathlineto{\pgfqpoint{2.077799in}{4.839167in}}%
\pgfpathlineto{\pgfqpoint{2.113688in}{4.839079in}}%
\pgfpathlineto{\pgfqpoint{2.149573in}{4.838990in}}%
\pgfpathlineto{\pgfqpoint{2.185455in}{4.838902in}}%
\pgfpathlineto{\pgfqpoint{2.221334in}{4.838814in}}%
\pgfpathlineto{\pgfqpoint{2.257208in}{4.838725in}}%
\pgfpathlineto{\pgfqpoint{2.293080in}{4.838637in}}%
\pgfpathlineto{\pgfqpoint{2.328947in}{4.838548in}}%
\pgfpathlineto{\pgfqpoint{2.364811in}{4.838460in}}%
\pgfpathlineto{\pgfqpoint{2.400672in}{4.838371in}}%
\pgfpathlineto{\pgfqpoint{2.436529in}{4.838283in}}%
\pgfpathlineto{\pgfqpoint{2.472383in}{4.838194in}}%
\pgfpathlineto{\pgfqpoint{2.508233in}{4.838106in}}%
\pgfpathlineto{\pgfqpoint{2.544079in}{4.838017in}}%
\pgfpathlineto{\pgfqpoint{2.579922in}{4.837929in}}%
\pgfpathlineto{\pgfqpoint{2.615762in}{4.837840in}}%
\pgfpathlineto{\pgfqpoint{2.651598in}{4.837752in}}%
\pgfpathlineto{\pgfqpoint{2.687430in}{4.837663in}}%
\pgfpathlineto{\pgfqpoint{2.723259in}{4.837574in}}%
\pgfpathlineto{\pgfqpoint{2.759085in}{4.837486in}}%
\pgfpathlineto{\pgfqpoint{2.794907in}{4.837397in}}%
\pgfpathlineto{\pgfqpoint{2.830725in}{4.837308in}}%
\pgfpathlineto{\pgfqpoint{2.866540in}{4.837219in}}%
\pgfpathlineto{\pgfqpoint{2.902351in}{4.837131in}}%
\pgfpathlineto{\pgfqpoint{2.938159in}{4.837042in}}%
\pgfpathlineto{\pgfqpoint{2.973964in}{4.836953in}}%
\pgfpathlineto{\pgfqpoint{3.009764in}{4.836864in}}%
\pgfpathlineto{\pgfqpoint{3.045562in}{4.836775in}}%
\pgfpathlineto{\pgfqpoint{3.081355in}{4.836686in}}%
\pgfpathlineto{\pgfqpoint{3.117146in}{4.836597in}}%
\pgfpathlineto{\pgfqpoint{3.152933in}{4.836508in}}%
\pgfpathlineto{\pgfqpoint{3.188716in}{4.836419in}}%
\pgfpathlineto{\pgfqpoint{3.224496in}{4.836330in}}%
\pgfpathlineto{\pgfqpoint{3.260272in}{4.836241in}}%
\pgfpathlineto{\pgfqpoint{3.296045in}{4.836152in}}%
\pgfpathlineto{\pgfqpoint{3.331814in}{4.836063in}}%
\pgfpathlineto{\pgfqpoint{3.367579in}{4.835974in}}%
\pgfpathlineto{\pgfqpoint{3.403342in}{4.835885in}}%
\pgfpathlineto{\pgfqpoint{3.439100in}{4.835796in}}%
\pgfpathlineto{\pgfqpoint{3.474856in}{4.835707in}}%
\pgfpathlineto{\pgfqpoint{3.510607in}{4.835618in}}%
\pgfpathlineto{\pgfqpoint{3.546355in}{4.835529in}}%
\pgfpathlineto{\pgfqpoint{3.582100in}{4.835439in}}%
\pgfpathlineto{\pgfqpoint{3.617841in}{4.835350in}}%
\pgfpathlineto{\pgfqpoint{3.653579in}{4.835261in}}%
\pgfpathlineto{\pgfqpoint{3.689313in}{4.835172in}}%
\pgfpathlineto{\pgfqpoint{3.725044in}{4.835082in}}%
\pgfpathlineto{\pgfqpoint{3.760771in}{4.834993in}}%
\pgfpathlineto{\pgfqpoint{3.796495in}{4.834904in}}%
\pgfpathlineto{\pgfqpoint{3.832215in}{4.834814in}}%
\pgfpathlineto{\pgfqpoint{3.867932in}{4.834725in}}%
\pgfpathlineto{\pgfqpoint{3.903645in}{4.834635in}}%
\pgfpathlineto{\pgfqpoint{3.939355in}{4.834546in}}%
\pgfpathlineto{\pgfqpoint{3.975061in}{4.834457in}}%
\pgfpathlineto{\pgfqpoint{4.010764in}{4.834367in}}%
\pgfpathlineto{\pgfqpoint{4.046463in}{4.834278in}}%
\pgfpathlineto{\pgfqpoint{4.082159in}{4.834188in}}%
\pgfpathlineto{\pgfqpoint{4.117851in}{4.834098in}}%
\pgfpathlineto{\pgfqpoint{4.153540in}{4.834009in}}%
\pgfpathlineto{\pgfqpoint{4.189225in}{4.833919in}}%
\pgfpathlineto{\pgfqpoint{4.224907in}{4.833830in}}%
\pgfpathlineto{\pgfqpoint{4.260585in}{4.833740in}}%
\pgfpathlineto{\pgfqpoint{4.296260in}{4.833650in}}%
\pgfpathlineto{\pgfqpoint{4.331931in}{4.833561in}}%
\pgfpathlineto{\pgfqpoint{4.367599in}{4.833471in}}%
\pgfpathlineto{\pgfqpoint{4.403263in}{4.833381in}}%
\pgfpathlineto{\pgfqpoint{4.438924in}{4.833292in}}%
\pgfpathlineto{\pgfqpoint{4.474581in}{4.833202in}}%
\pgfpathlineto{\pgfqpoint{4.510235in}{4.833112in}}%
\pgfpathlineto{\pgfqpoint{4.545885in}{4.833022in}}%
\pgfpathlineto{\pgfqpoint{4.581532in}{4.832932in}}%
\pgfpathlineto{\pgfqpoint{4.617176in}{4.832842in}}%
\pgfpathlineto{\pgfqpoint{4.652816in}{4.832753in}}%
\pgfpathlineto{\pgfqpoint{4.688452in}{4.832663in}}%
\pgfpathlineto{\pgfqpoint{4.724085in}{4.832573in}}%
\pgfpathlineto{\pgfqpoint{4.759715in}{4.832483in}}%
\pgfusepath{stroke}%
\end{pgfscope}%
\begin{pgfscope}%
\pgfsetrectcap%
\pgfsetmiterjoin%
\pgfsetlinewidth{0.803000pt}%
\definecolor{currentstroke}{rgb}{0.000000,0.000000,0.000000}%
\pgfsetstrokecolor{currentstroke}%
\pgfsetdash{}{0pt}%
\pgfpathmoveto{\pgfqpoint{1.038190in}{0.582778in}}%
\pgfpathlineto{\pgfqpoint{1.038190in}{4.850000in}}%
\pgfusepath{stroke}%
\end{pgfscope}%
\begin{pgfscope}%
\pgfsetrectcap%
\pgfsetmiterjoin%
\pgfsetlinewidth{0.803000pt}%
\definecolor{currentstroke}{rgb}{0.000000,0.000000,0.000000}%
\pgfsetstrokecolor{currentstroke}%
\pgfsetdash{}{0pt}%
\pgfpathmoveto{\pgfqpoint{1.038190in}{0.582778in}}%
\pgfpathlineto{\pgfqpoint{4.936930in}{0.582778in}}%
\pgfusepath{stroke}%
\end{pgfscope}%
\begin{pgfscope}%
\pgfsetbuttcap%
\pgfsetmiterjoin%
\definecolor{currentfill}{rgb}{1.000000,1.000000,1.000000}%
\pgfsetfillcolor{currentfill}%
\pgfsetfillopacity{0.800000}%
\pgfsetlinewidth{1.003750pt}%
\definecolor{currentstroke}{rgb}{0.800000,0.800000,0.800000}%
\pgfsetstrokecolor{currentstroke}%
\pgfsetstrokeopacity{0.800000}%
\pgfsetdash{}{0pt}%
\pgfpathmoveto{\pgfqpoint{4.001546in}{4.331174in}}%
\pgfpathlineto{\pgfqpoint{4.839708in}{4.331174in}}%
\pgfpathquadraticcurveto{\pgfqpoint{4.867486in}{4.331174in}}{\pgfqpoint{4.867486in}{4.358952in}}%
\pgfpathlineto{\pgfqpoint{4.867486in}{4.752778in}}%
\pgfpathquadraticcurveto{\pgfqpoint{4.867486in}{4.780556in}}{\pgfqpoint{4.839708in}{4.780556in}}%
\pgfpathlineto{\pgfqpoint{4.001546in}{4.780556in}}%
\pgfpathquadraticcurveto{\pgfqpoint{3.973768in}{4.780556in}}{\pgfqpoint{3.973768in}{4.752778in}}%
\pgfpathlineto{\pgfqpoint{3.973768in}{4.358952in}}%
\pgfpathquadraticcurveto{\pgfqpoint{3.973768in}{4.331174in}}{\pgfqpoint{4.001546in}{4.331174in}}%
\pgfpathlineto{\pgfqpoint{4.001546in}{4.331174in}}%
\pgfpathclose%
\pgfusepath{stroke,fill}%
\end{pgfscope}%
\begin{pgfscope}%
\definecolor{textcolor}{rgb}{0.000000,0.000000,0.000000}%
\pgfsetstrokecolor{textcolor}%
\pgfsetfillcolor{textcolor}%
\pgftext[x=4.029324in,y=4.619477in,left,base]{\color{textcolor}\sffamily\fontsize{10.000000}{12.000000}\selectfont node count}%
\end{pgfscope}%
\begin{pgfscope}%
\pgfsetbuttcap%
\pgfsetroundjoin%
\definecolor{currentfill}{rgb}{0.003922,0.450980,0.698039}%
\pgfsetfillcolor{currentfill}%
\pgfsetfillopacity{0.800000}%
\pgfsetlinewidth{1.003750pt}%
\definecolor{currentstroke}{rgb}{0.003922,0.450980,0.698039}%
\pgfsetstrokecolor{currentstroke}%
\pgfsetstrokeopacity{0.800000}%
\pgfsetdash{}{0pt}%
\pgfsys@defobject{currentmarker}{\pgfqpoint{-0.041667in}{-0.041667in}}{\pgfqpoint{0.041667in}{0.041667in}}{%
\pgfpathmoveto{\pgfqpoint{0.000000in}{-0.041667in}}%
\pgfpathcurveto{\pgfqpoint{0.011050in}{-0.041667in}}{\pgfqpoint{0.021649in}{-0.037276in}}{\pgfqpoint{0.029463in}{-0.029463in}}%
\pgfpathcurveto{\pgfqpoint{0.037276in}{-0.021649in}}{\pgfqpoint{0.041667in}{-0.011050in}}{\pgfqpoint{0.041667in}{0.000000in}}%
\pgfpathcurveto{\pgfqpoint{0.041667in}{0.011050in}}{\pgfqpoint{0.037276in}{0.021649in}}{\pgfqpoint{0.029463in}{0.029463in}}%
\pgfpathcurveto{\pgfqpoint{0.021649in}{0.037276in}}{\pgfqpoint{0.011050in}{0.041667in}}{\pgfqpoint{0.000000in}{0.041667in}}%
\pgfpathcurveto{\pgfqpoint{-0.011050in}{0.041667in}}{\pgfqpoint{-0.021649in}{0.037276in}}{\pgfqpoint{-0.029463in}{0.029463in}}%
\pgfpathcurveto{\pgfqpoint{-0.037276in}{0.021649in}}{\pgfqpoint{-0.041667in}{0.011050in}}{\pgfqpoint{-0.041667in}{0.000000in}}%
\pgfpathcurveto{\pgfqpoint{-0.041667in}{-0.011050in}}{\pgfqpoint{-0.037276in}{-0.021649in}}{\pgfqpoint{-0.029463in}{-0.029463in}}%
\pgfpathcurveto{\pgfqpoint{-0.021649in}{-0.037276in}}{\pgfqpoint{-0.011050in}{-0.041667in}}{\pgfqpoint{0.000000in}{-0.041667in}}%
\pgfpathlineto{\pgfqpoint{0.000000in}{-0.041667in}}%
\pgfpathclose%
\pgfusepath{stroke,fill}%
}%
\begin{pgfscope}%
\pgfsys@transformshift{4.320889in}{4.452078in}%
\pgfsys@useobject{currentmarker}{}%
\end{pgfscope}%
\end{pgfscope}%
\begin{pgfscope}%
\definecolor{textcolor}{rgb}{0.000000,0.000000,0.000000}%
\pgfsetstrokecolor{textcolor}%
\pgfsetfillcolor{textcolor}%
\pgftext[x=4.570889in,y=4.415620in,left,base]{\color{textcolor}\sffamily\fontsize{10.000000}{12.000000}\selectfont 7}%
\end{pgfscope}%
\end{pgfpicture}%
\makeatother%
\endgroup%

% \caption{Benchmarking of goodput and 95\%ile latency while varying throughputs and node counts.}
% \label{goodput95latencynodes}
% \end{figure}

This study compares the performance of the system for varying node counts. Node counts were chosen such that $n = 3f + 1$ for some $f$, as choosing another value would decrease performance without any benefit of increased fault-tolerance\footnote{A node count of 2 was also tested as it is the smallest node count where internal messages are exchanged.}. All experiments were run for 10s with a batch size of 300. Figure~\ref{goodputlatencynodes} omits results with latency of above 1s, to show the performance of the system as it begins to be overloaded.

As node count increases the latency increases (Figure~\ref{goodputlatencynodes}). This is because larger node counts mean that each view requires more internal messages to be sent to progress. Sending internal messages is expensive due to the latency of serialisation and cryptography, so this results in increased overall latency. Additionally, increasing the node count increases the number of messages that must be signed, and makes aggregating signatures slower (Section~\ref{tezosbenchmark}).

Latency increases slowly with throughput while the system is not overloaded, then begins to rapidly increase once the system is overloaded and requests start to queue (Figure~\ref{goodputlatencynodes}). Notably, the latency for a system of 1 node increases slowly, as there are no internal messages, just client requests and responses.

The larger the node count, the lower the maximum goodput that can be reached (Figure~\ref{throughputgoodputnodes}). This is again due to larger node counts resulting in more internal messages, causing more latency since this is a bottleneck. Increased latency causes each view to take longer, reducing the number of requests that can be responded to each second.

\subsection{Ablation study} \label{ablation}

\begin{table}[h!]
\centering
\begin{tabular}{|c|c|c|c|c|}
\hline
Version & Chaining & Truncation & Filtering & Crypto \\ \hline
BASIC & \xmark & \xmark & \xmark & \cmark \\ \hline
CHAIN & \cmark & \xmark & \xmark & \cmark \\ \hline
FILT & \cmark & \xmark & \cmark & \cmark \\ \hline
TRUNC & \cmark & \cmark & \xmark & \cmark \\ \hline
ALL & \cmark & \cmark & \cmark & \cmark \\ \hline
NOCRY & \cmark & \cmark & \cmark & \xmark \\ \hline
\end{tabular}
\caption{Features enabled in different versions.}
\label{versiontable}
\end{table}

\begin{figure}[h!]
\centering
\resizebox{.6\textwidth}{!}{%% Creator: Matplotlib, PGF backend
%%
%% To include the figure in your LaTeX document, write
%%   \input{<filename>.pgf}
%%
%% Make sure the required packages are loaded in your preamble
%%   \usepackage{pgf}
%%
%% Also ensure that all the required font packages are loaded; for instance,
%% the lmodern package is sometimes necessary when using math font.
%%   \usepackage{lmodern}
%%
%% Figures using additional raster images can only be included by \input if
%% they are in the same directory as the main LaTeX file. For loading figures
%% from other directories you can use the `import` package
%%   \usepackage{import}
%%
%% and then include the figures with
%%   \import{<path to file>}{<filename>.pgf}
%%
%% Matplotlib used the following preamble
%%   
%%   \usepackage{fontspec}
%%   \setmainfont{DejaVuSerif.ttf}[Path=\detokenize{/opt/homebrew/lib/python3.10/site-packages/matplotlib/mpl-data/fonts/ttf/}]
%%   \setsansfont{DejaVuSans.ttf}[Path=\detokenize{/opt/homebrew/lib/python3.10/site-packages/matplotlib/mpl-data/fonts/ttf/}]
%%   \setmonofont{DejaVuSansMono.ttf}[Path=\detokenize{/opt/homebrew/lib/python3.10/site-packages/matplotlib/mpl-data/fonts/ttf/}]
%%   \makeatletter\@ifpackageloaded{underscore}{}{\usepackage[strings]{underscore}}\makeatother
%%
\begingroup%
\makeatletter%
\begin{pgfpicture}%
\pgfpathrectangle{\pgfpointorigin}{\pgfqpoint{6.400000in}{4.800000in}}%
\pgfusepath{use as bounding box, clip}%
\begin{pgfscope}%
\pgfsetbuttcap%
\pgfsetmiterjoin%
\definecolor{currentfill}{rgb}{1.000000,1.000000,1.000000}%
\pgfsetfillcolor{currentfill}%
\pgfsetlinewidth{0.000000pt}%
\definecolor{currentstroke}{rgb}{1.000000,1.000000,1.000000}%
\pgfsetstrokecolor{currentstroke}%
\pgfsetdash{}{0pt}%
\pgfpathmoveto{\pgfqpoint{0.000000in}{0.000000in}}%
\pgfpathlineto{\pgfqpoint{6.400000in}{0.000000in}}%
\pgfpathlineto{\pgfqpoint{6.400000in}{4.800000in}}%
\pgfpathlineto{\pgfqpoint{0.000000in}{4.800000in}}%
\pgfpathlineto{\pgfqpoint{0.000000in}{0.000000in}}%
\pgfpathclose%
\pgfusepath{fill}%
\end{pgfscope}%
\begin{pgfscope}%
\pgfsetbuttcap%
\pgfsetmiterjoin%
\definecolor{currentfill}{rgb}{1.000000,1.000000,1.000000}%
\pgfsetfillcolor{currentfill}%
\pgfsetlinewidth{0.000000pt}%
\definecolor{currentstroke}{rgb}{0.000000,0.000000,0.000000}%
\pgfsetstrokecolor{currentstroke}%
\pgfsetstrokeopacity{0.000000}%
\pgfsetdash{}{0pt}%
\pgfpathmoveto{\pgfqpoint{0.800000in}{0.528000in}}%
\pgfpathlineto{\pgfqpoint{5.760000in}{0.528000in}}%
\pgfpathlineto{\pgfqpoint{5.760000in}{4.224000in}}%
\pgfpathlineto{\pgfqpoint{0.800000in}{4.224000in}}%
\pgfpathlineto{\pgfqpoint{0.800000in}{0.528000in}}%
\pgfpathclose%
\pgfusepath{fill}%
\end{pgfscope}%
\begin{pgfscope}%
\pgfpathrectangle{\pgfqpoint{0.800000in}{0.528000in}}{\pgfqpoint{4.960000in}{3.696000in}}%
\pgfusepath{clip}%
\pgfsetbuttcap%
\pgfsetroundjoin%
\definecolor{currentfill}{rgb}{0.003922,0.450980,0.698039}%
\pgfsetfillcolor{currentfill}%
\pgfsetfillopacity{0.200000}%
\pgfsetlinewidth{1.003750pt}%
\definecolor{currentstroke}{rgb}{0.003922,0.450980,0.698039}%
\pgfsetstrokecolor{currentstroke}%
\pgfsetstrokeopacity{0.200000}%
\pgfsetdash{}{0pt}%
\pgfpathmoveto{\pgfqpoint{0.802480in}{0.532620in}}%
\pgfpathlineto{\pgfqpoint{0.802480in}{0.532620in}}%
\pgfpathlineto{\pgfqpoint{0.862000in}{0.643479in}}%
\pgfpathlineto{\pgfqpoint{0.924000in}{0.759000in}}%
\pgfpathlineto{\pgfqpoint{1.048000in}{0.989635in}}%
\pgfpathlineto{\pgfqpoint{1.296000in}{1.450467in}}%
\pgfpathlineto{\pgfqpoint{1.792000in}{2.312143in}}%
\pgfpathlineto{\pgfqpoint{2.288000in}{2.687817in}}%
\pgfpathlineto{\pgfqpoint{2.784000in}{2.863113in}}%
\pgfpathlineto{\pgfqpoint{3.280000in}{2.992811in}}%
\pgfpathlineto{\pgfqpoint{3.776000in}{3.204785in}}%
\pgfpathlineto{\pgfqpoint{4.272000in}{3.281259in}}%
\pgfpathlineto{\pgfqpoint{4.768000in}{3.261633in}}%
\pgfpathlineto{\pgfqpoint{5.264000in}{3.415920in}}%
\pgfpathlineto{\pgfqpoint{5.760000in}{3.370991in}}%
\pgfpathlineto{\pgfqpoint{6.256000in}{3.453592in}}%
\pgfpathlineto{\pgfqpoint{6.752000in}{3.437845in}}%
\pgfpathlineto{\pgfqpoint{7.248000in}{3.560410in}}%
\pgfpathlineto{\pgfqpoint{7.744000in}{3.609055in}}%
\pgfpathlineto{\pgfqpoint{8.240000in}{3.404028in}}%
\pgfpathlineto{\pgfqpoint{8.736000in}{3.554864in}}%
\pgfpathlineto{\pgfqpoint{9.232000in}{3.560972in}}%
\pgfpathlineto{\pgfqpoint{9.728000in}{3.415431in}}%
\pgfpathlineto{\pgfqpoint{10.224000in}{3.425071in}}%
\pgfpathlineto{\pgfqpoint{10.720000in}{3.518612in}}%
\pgfpathlineto{\pgfqpoint{10.720000in}{8.999004in}}%
\pgfpathlineto{\pgfqpoint{10.720000in}{8.999004in}}%
\pgfpathlineto{\pgfqpoint{10.224000in}{8.692190in}}%
\pgfpathlineto{\pgfqpoint{9.728000in}{8.542813in}}%
\pgfpathlineto{\pgfqpoint{9.232000in}{8.176902in}}%
\pgfpathlineto{\pgfqpoint{8.736000in}{7.888196in}}%
\pgfpathlineto{\pgfqpoint{8.240000in}{7.373015in}}%
\pgfpathlineto{\pgfqpoint{7.744000in}{7.389232in}}%
\pgfpathlineto{\pgfqpoint{7.248000in}{7.257582in}}%
\pgfpathlineto{\pgfqpoint{6.752000in}{6.730127in}}%
\pgfpathlineto{\pgfqpoint{6.256000in}{6.376806in}}%
\pgfpathlineto{\pgfqpoint{5.760000in}{6.442465in}}%
\pgfpathlineto{\pgfqpoint{5.264000in}{6.160533in}}%
\pgfpathlineto{\pgfqpoint{4.768000in}{5.661752in}}%
\pgfpathlineto{\pgfqpoint{4.272000in}{5.204204in}}%
\pgfpathlineto{\pgfqpoint{3.776000in}{4.848196in}}%
\pgfpathlineto{\pgfqpoint{3.280000in}{4.313938in}}%
\pgfpathlineto{\pgfqpoint{2.784000in}{3.758322in}}%
\pgfpathlineto{\pgfqpoint{2.288000in}{3.141141in}}%
\pgfpathlineto{\pgfqpoint{1.792000in}{2.363859in}}%
\pgfpathlineto{\pgfqpoint{1.296000in}{1.451801in}}%
\pgfpathlineto{\pgfqpoint{1.048000in}{0.989990in}}%
\pgfpathlineto{\pgfqpoint{0.924000in}{0.759000in}}%
\pgfpathlineto{\pgfqpoint{0.862000in}{0.643500in}}%
\pgfpathlineto{\pgfqpoint{0.802480in}{0.532620in}}%
\pgfpathlineto{\pgfqpoint{0.802480in}{0.532620in}}%
\pgfpathclose%
\pgfusepath{stroke,fill}%
\end{pgfscope}%
\begin{pgfscope}%
\pgfsetbuttcap%
\pgfsetroundjoin%
\definecolor{currentfill}{rgb}{0.000000,0.000000,0.000000}%
\pgfsetfillcolor{currentfill}%
\pgfsetlinewidth{0.803000pt}%
\definecolor{currentstroke}{rgb}{0.000000,0.000000,0.000000}%
\pgfsetstrokecolor{currentstroke}%
\pgfsetdash{}{0pt}%
\pgfsys@defobject{currentmarker}{\pgfqpoint{0.000000in}{-0.048611in}}{\pgfqpoint{0.000000in}{0.000000in}}{%
\pgfpathmoveto{\pgfqpoint{0.000000in}{0.000000in}}%
\pgfpathlineto{\pgfqpoint{0.000000in}{-0.048611in}}%
\pgfusepath{stroke,fill}%
}%
\begin{pgfscope}%
\pgfsys@transformshift{0.800000in}{0.528000in}%
\pgfsys@useobject{currentmarker}{}%
\end{pgfscope}%
\end{pgfscope}%
\begin{pgfscope}%
\definecolor{textcolor}{rgb}{0.000000,0.000000,0.000000}%
\pgfsetstrokecolor{textcolor}%
\pgfsetfillcolor{textcolor}%
\pgftext[x=0.800000in,y=0.430778in,,top]{\color{textcolor}\sffamily\fontsize{10.000000}{12.000000}\selectfont 0}%
\end{pgfscope}%
\begin{pgfscope}%
\pgfsetbuttcap%
\pgfsetroundjoin%
\definecolor{currentfill}{rgb}{0.000000,0.000000,0.000000}%
\pgfsetfillcolor{currentfill}%
\pgfsetlinewidth{0.803000pt}%
\definecolor{currentstroke}{rgb}{0.000000,0.000000,0.000000}%
\pgfsetstrokecolor{currentstroke}%
\pgfsetdash{}{0pt}%
\pgfsys@defobject{currentmarker}{\pgfqpoint{0.000000in}{-0.048611in}}{\pgfqpoint{0.000000in}{0.000000in}}{%
\pgfpathmoveto{\pgfqpoint{0.000000in}{0.000000in}}%
\pgfpathlineto{\pgfqpoint{0.000000in}{-0.048611in}}%
\pgfusepath{stroke,fill}%
}%
\begin{pgfscope}%
\pgfsys@transformshift{1.420000in}{0.528000in}%
\pgfsys@useobject{currentmarker}{}%
\end{pgfscope}%
\end{pgfscope}%
\begin{pgfscope}%
\definecolor{textcolor}{rgb}{0.000000,0.000000,0.000000}%
\pgfsetstrokecolor{textcolor}%
\pgfsetfillcolor{textcolor}%
\pgftext[x=1.420000in,y=0.430778in,,top]{\color{textcolor}\sffamily\fontsize{10.000000}{12.000000}\selectfont 250}%
\end{pgfscope}%
\begin{pgfscope}%
\pgfsetbuttcap%
\pgfsetroundjoin%
\definecolor{currentfill}{rgb}{0.000000,0.000000,0.000000}%
\pgfsetfillcolor{currentfill}%
\pgfsetlinewidth{0.803000pt}%
\definecolor{currentstroke}{rgb}{0.000000,0.000000,0.000000}%
\pgfsetstrokecolor{currentstroke}%
\pgfsetdash{}{0pt}%
\pgfsys@defobject{currentmarker}{\pgfqpoint{0.000000in}{-0.048611in}}{\pgfqpoint{0.000000in}{0.000000in}}{%
\pgfpathmoveto{\pgfqpoint{0.000000in}{0.000000in}}%
\pgfpathlineto{\pgfqpoint{0.000000in}{-0.048611in}}%
\pgfusepath{stroke,fill}%
}%
\begin{pgfscope}%
\pgfsys@transformshift{2.040000in}{0.528000in}%
\pgfsys@useobject{currentmarker}{}%
\end{pgfscope}%
\end{pgfscope}%
\begin{pgfscope}%
\definecolor{textcolor}{rgb}{0.000000,0.000000,0.000000}%
\pgfsetstrokecolor{textcolor}%
\pgfsetfillcolor{textcolor}%
\pgftext[x=2.040000in,y=0.430778in,,top]{\color{textcolor}\sffamily\fontsize{10.000000}{12.000000}\selectfont 500}%
\end{pgfscope}%
\begin{pgfscope}%
\pgfsetbuttcap%
\pgfsetroundjoin%
\definecolor{currentfill}{rgb}{0.000000,0.000000,0.000000}%
\pgfsetfillcolor{currentfill}%
\pgfsetlinewidth{0.803000pt}%
\definecolor{currentstroke}{rgb}{0.000000,0.000000,0.000000}%
\pgfsetstrokecolor{currentstroke}%
\pgfsetdash{}{0pt}%
\pgfsys@defobject{currentmarker}{\pgfqpoint{0.000000in}{-0.048611in}}{\pgfqpoint{0.000000in}{0.000000in}}{%
\pgfpathmoveto{\pgfqpoint{0.000000in}{0.000000in}}%
\pgfpathlineto{\pgfqpoint{0.000000in}{-0.048611in}}%
\pgfusepath{stroke,fill}%
}%
\begin{pgfscope}%
\pgfsys@transformshift{2.660000in}{0.528000in}%
\pgfsys@useobject{currentmarker}{}%
\end{pgfscope}%
\end{pgfscope}%
\begin{pgfscope}%
\definecolor{textcolor}{rgb}{0.000000,0.000000,0.000000}%
\pgfsetstrokecolor{textcolor}%
\pgfsetfillcolor{textcolor}%
\pgftext[x=2.660000in,y=0.430778in,,top]{\color{textcolor}\sffamily\fontsize{10.000000}{12.000000}\selectfont 750}%
\end{pgfscope}%
\begin{pgfscope}%
\pgfsetbuttcap%
\pgfsetroundjoin%
\definecolor{currentfill}{rgb}{0.000000,0.000000,0.000000}%
\pgfsetfillcolor{currentfill}%
\pgfsetlinewidth{0.803000pt}%
\definecolor{currentstroke}{rgb}{0.000000,0.000000,0.000000}%
\pgfsetstrokecolor{currentstroke}%
\pgfsetdash{}{0pt}%
\pgfsys@defobject{currentmarker}{\pgfqpoint{0.000000in}{-0.048611in}}{\pgfqpoint{0.000000in}{0.000000in}}{%
\pgfpathmoveto{\pgfqpoint{0.000000in}{0.000000in}}%
\pgfpathlineto{\pgfqpoint{0.000000in}{-0.048611in}}%
\pgfusepath{stroke,fill}%
}%
\begin{pgfscope}%
\pgfsys@transformshift{3.280000in}{0.528000in}%
\pgfsys@useobject{currentmarker}{}%
\end{pgfscope}%
\end{pgfscope}%
\begin{pgfscope}%
\definecolor{textcolor}{rgb}{0.000000,0.000000,0.000000}%
\pgfsetstrokecolor{textcolor}%
\pgfsetfillcolor{textcolor}%
\pgftext[x=3.280000in,y=0.430778in,,top]{\color{textcolor}\sffamily\fontsize{10.000000}{12.000000}\selectfont 1000}%
\end{pgfscope}%
\begin{pgfscope}%
\pgfsetbuttcap%
\pgfsetroundjoin%
\definecolor{currentfill}{rgb}{0.000000,0.000000,0.000000}%
\pgfsetfillcolor{currentfill}%
\pgfsetlinewidth{0.803000pt}%
\definecolor{currentstroke}{rgb}{0.000000,0.000000,0.000000}%
\pgfsetstrokecolor{currentstroke}%
\pgfsetdash{}{0pt}%
\pgfsys@defobject{currentmarker}{\pgfqpoint{0.000000in}{-0.048611in}}{\pgfqpoint{0.000000in}{0.000000in}}{%
\pgfpathmoveto{\pgfqpoint{0.000000in}{0.000000in}}%
\pgfpathlineto{\pgfqpoint{0.000000in}{-0.048611in}}%
\pgfusepath{stroke,fill}%
}%
\begin{pgfscope}%
\pgfsys@transformshift{3.900000in}{0.528000in}%
\pgfsys@useobject{currentmarker}{}%
\end{pgfscope}%
\end{pgfscope}%
\begin{pgfscope}%
\definecolor{textcolor}{rgb}{0.000000,0.000000,0.000000}%
\pgfsetstrokecolor{textcolor}%
\pgfsetfillcolor{textcolor}%
\pgftext[x=3.900000in,y=0.430778in,,top]{\color{textcolor}\sffamily\fontsize{10.000000}{12.000000}\selectfont 1250}%
\end{pgfscope}%
\begin{pgfscope}%
\pgfsetbuttcap%
\pgfsetroundjoin%
\definecolor{currentfill}{rgb}{0.000000,0.000000,0.000000}%
\pgfsetfillcolor{currentfill}%
\pgfsetlinewidth{0.803000pt}%
\definecolor{currentstroke}{rgb}{0.000000,0.000000,0.000000}%
\pgfsetstrokecolor{currentstroke}%
\pgfsetdash{}{0pt}%
\pgfsys@defobject{currentmarker}{\pgfqpoint{0.000000in}{-0.048611in}}{\pgfqpoint{0.000000in}{0.000000in}}{%
\pgfpathmoveto{\pgfqpoint{0.000000in}{0.000000in}}%
\pgfpathlineto{\pgfqpoint{0.000000in}{-0.048611in}}%
\pgfusepath{stroke,fill}%
}%
\begin{pgfscope}%
\pgfsys@transformshift{4.520000in}{0.528000in}%
\pgfsys@useobject{currentmarker}{}%
\end{pgfscope}%
\end{pgfscope}%
\begin{pgfscope}%
\definecolor{textcolor}{rgb}{0.000000,0.000000,0.000000}%
\pgfsetstrokecolor{textcolor}%
\pgfsetfillcolor{textcolor}%
\pgftext[x=4.520000in,y=0.430778in,,top]{\color{textcolor}\sffamily\fontsize{10.000000}{12.000000}\selectfont 1500}%
\end{pgfscope}%
\begin{pgfscope}%
\pgfsetbuttcap%
\pgfsetroundjoin%
\definecolor{currentfill}{rgb}{0.000000,0.000000,0.000000}%
\pgfsetfillcolor{currentfill}%
\pgfsetlinewidth{0.803000pt}%
\definecolor{currentstroke}{rgb}{0.000000,0.000000,0.000000}%
\pgfsetstrokecolor{currentstroke}%
\pgfsetdash{}{0pt}%
\pgfsys@defobject{currentmarker}{\pgfqpoint{0.000000in}{-0.048611in}}{\pgfqpoint{0.000000in}{0.000000in}}{%
\pgfpathmoveto{\pgfqpoint{0.000000in}{0.000000in}}%
\pgfpathlineto{\pgfqpoint{0.000000in}{-0.048611in}}%
\pgfusepath{stroke,fill}%
}%
\begin{pgfscope}%
\pgfsys@transformshift{5.140000in}{0.528000in}%
\pgfsys@useobject{currentmarker}{}%
\end{pgfscope}%
\end{pgfscope}%
\begin{pgfscope}%
\definecolor{textcolor}{rgb}{0.000000,0.000000,0.000000}%
\pgfsetstrokecolor{textcolor}%
\pgfsetfillcolor{textcolor}%
\pgftext[x=5.140000in,y=0.430778in,,top]{\color{textcolor}\sffamily\fontsize{10.000000}{12.000000}\selectfont 1750}%
\end{pgfscope}%
\begin{pgfscope}%
\pgfsetbuttcap%
\pgfsetroundjoin%
\definecolor{currentfill}{rgb}{0.000000,0.000000,0.000000}%
\pgfsetfillcolor{currentfill}%
\pgfsetlinewidth{0.803000pt}%
\definecolor{currentstroke}{rgb}{0.000000,0.000000,0.000000}%
\pgfsetstrokecolor{currentstroke}%
\pgfsetdash{}{0pt}%
\pgfsys@defobject{currentmarker}{\pgfqpoint{0.000000in}{-0.048611in}}{\pgfqpoint{0.000000in}{0.000000in}}{%
\pgfpathmoveto{\pgfqpoint{0.000000in}{0.000000in}}%
\pgfpathlineto{\pgfqpoint{0.000000in}{-0.048611in}}%
\pgfusepath{stroke,fill}%
}%
\begin{pgfscope}%
\pgfsys@transformshift{5.760000in}{0.528000in}%
\pgfsys@useobject{currentmarker}{}%
\end{pgfscope}%
\end{pgfscope}%
\begin{pgfscope}%
\definecolor{textcolor}{rgb}{0.000000,0.000000,0.000000}%
\pgfsetstrokecolor{textcolor}%
\pgfsetfillcolor{textcolor}%
\pgftext[x=5.760000in,y=0.430778in,,top]{\color{textcolor}\sffamily\fontsize{10.000000}{12.000000}\selectfont 2000}%
\end{pgfscope}%
\begin{pgfscope}%
\definecolor{textcolor}{rgb}{0.000000,0.000000,0.000000}%
\pgfsetstrokecolor{textcolor}%
\pgfsetfillcolor{textcolor}%
\pgftext[x=3.280000in,y=0.240809in,,top]{\color{textcolor}\sffamily\fontsize{10.000000}{12.000000}\selectfont throughput (req/s)}%
\end{pgfscope}%
\begin{pgfscope}%
\pgfsetbuttcap%
\pgfsetroundjoin%
\definecolor{currentfill}{rgb}{0.000000,0.000000,0.000000}%
\pgfsetfillcolor{currentfill}%
\pgfsetlinewidth{0.803000pt}%
\definecolor{currentstroke}{rgb}{0.000000,0.000000,0.000000}%
\pgfsetstrokecolor{currentstroke}%
\pgfsetdash{}{0pt}%
\pgfsys@defobject{currentmarker}{\pgfqpoint{-0.048611in}{0.000000in}}{\pgfqpoint{-0.000000in}{0.000000in}}{%
\pgfpathmoveto{\pgfqpoint{-0.000000in}{0.000000in}}%
\pgfpathlineto{\pgfqpoint{-0.048611in}{0.000000in}}%
\pgfusepath{stroke,fill}%
}%
\begin{pgfscope}%
\pgfsys@transformshift{0.800000in}{0.528000in}%
\pgfsys@useobject{currentmarker}{}%
\end{pgfscope}%
\end{pgfscope}%
\begin{pgfscope}%
\definecolor{textcolor}{rgb}{0.000000,0.000000,0.000000}%
\pgfsetstrokecolor{textcolor}%
\pgfsetfillcolor{textcolor}%
\pgftext[x=0.614412in, y=0.475238in, left, base]{\color{textcolor}\sffamily\fontsize{10.000000}{12.000000}\selectfont 0}%
\end{pgfscope}%
\begin{pgfscope}%
\pgfsetbuttcap%
\pgfsetroundjoin%
\definecolor{currentfill}{rgb}{0.000000,0.000000,0.000000}%
\pgfsetfillcolor{currentfill}%
\pgfsetlinewidth{0.803000pt}%
\definecolor{currentstroke}{rgb}{0.000000,0.000000,0.000000}%
\pgfsetstrokecolor{currentstroke}%
\pgfsetdash{}{0pt}%
\pgfsys@defobject{currentmarker}{\pgfqpoint{-0.048611in}{0.000000in}}{\pgfqpoint{-0.000000in}{0.000000in}}{%
\pgfpathmoveto{\pgfqpoint{-0.000000in}{0.000000in}}%
\pgfpathlineto{\pgfqpoint{-0.048611in}{0.000000in}}%
\pgfusepath{stroke,fill}%
}%
\begin{pgfscope}%
\pgfsys@transformshift{0.800000in}{0.990000in}%
\pgfsys@useobject{currentmarker}{}%
\end{pgfscope}%
\end{pgfscope}%
\begin{pgfscope}%
\definecolor{textcolor}{rgb}{0.000000,0.000000,0.000000}%
\pgfsetstrokecolor{textcolor}%
\pgfsetfillcolor{textcolor}%
\pgftext[x=0.437682in, y=0.937238in, left, base]{\color{textcolor}\sffamily\fontsize{10.000000}{12.000000}\selectfont 100}%
\end{pgfscope}%
\begin{pgfscope}%
\pgfsetbuttcap%
\pgfsetroundjoin%
\definecolor{currentfill}{rgb}{0.000000,0.000000,0.000000}%
\pgfsetfillcolor{currentfill}%
\pgfsetlinewidth{0.803000pt}%
\definecolor{currentstroke}{rgb}{0.000000,0.000000,0.000000}%
\pgfsetstrokecolor{currentstroke}%
\pgfsetdash{}{0pt}%
\pgfsys@defobject{currentmarker}{\pgfqpoint{-0.048611in}{0.000000in}}{\pgfqpoint{-0.000000in}{0.000000in}}{%
\pgfpathmoveto{\pgfqpoint{-0.000000in}{0.000000in}}%
\pgfpathlineto{\pgfqpoint{-0.048611in}{0.000000in}}%
\pgfusepath{stroke,fill}%
}%
\begin{pgfscope}%
\pgfsys@transformshift{0.800000in}{1.452000in}%
\pgfsys@useobject{currentmarker}{}%
\end{pgfscope}%
\end{pgfscope}%
\begin{pgfscope}%
\definecolor{textcolor}{rgb}{0.000000,0.000000,0.000000}%
\pgfsetstrokecolor{textcolor}%
\pgfsetfillcolor{textcolor}%
\pgftext[x=0.437682in, y=1.399238in, left, base]{\color{textcolor}\sffamily\fontsize{10.000000}{12.000000}\selectfont 200}%
\end{pgfscope}%
\begin{pgfscope}%
\pgfsetbuttcap%
\pgfsetroundjoin%
\definecolor{currentfill}{rgb}{0.000000,0.000000,0.000000}%
\pgfsetfillcolor{currentfill}%
\pgfsetlinewidth{0.803000pt}%
\definecolor{currentstroke}{rgb}{0.000000,0.000000,0.000000}%
\pgfsetstrokecolor{currentstroke}%
\pgfsetdash{}{0pt}%
\pgfsys@defobject{currentmarker}{\pgfqpoint{-0.048611in}{0.000000in}}{\pgfqpoint{-0.000000in}{0.000000in}}{%
\pgfpathmoveto{\pgfqpoint{-0.000000in}{0.000000in}}%
\pgfpathlineto{\pgfqpoint{-0.048611in}{0.000000in}}%
\pgfusepath{stroke,fill}%
}%
\begin{pgfscope}%
\pgfsys@transformshift{0.800000in}{1.914000in}%
\pgfsys@useobject{currentmarker}{}%
\end{pgfscope}%
\end{pgfscope}%
\begin{pgfscope}%
\definecolor{textcolor}{rgb}{0.000000,0.000000,0.000000}%
\pgfsetstrokecolor{textcolor}%
\pgfsetfillcolor{textcolor}%
\pgftext[x=0.437682in, y=1.861238in, left, base]{\color{textcolor}\sffamily\fontsize{10.000000}{12.000000}\selectfont 300}%
\end{pgfscope}%
\begin{pgfscope}%
\pgfsetbuttcap%
\pgfsetroundjoin%
\definecolor{currentfill}{rgb}{0.000000,0.000000,0.000000}%
\pgfsetfillcolor{currentfill}%
\pgfsetlinewidth{0.803000pt}%
\definecolor{currentstroke}{rgb}{0.000000,0.000000,0.000000}%
\pgfsetstrokecolor{currentstroke}%
\pgfsetdash{}{0pt}%
\pgfsys@defobject{currentmarker}{\pgfqpoint{-0.048611in}{0.000000in}}{\pgfqpoint{-0.000000in}{0.000000in}}{%
\pgfpathmoveto{\pgfqpoint{-0.000000in}{0.000000in}}%
\pgfpathlineto{\pgfqpoint{-0.048611in}{0.000000in}}%
\pgfusepath{stroke,fill}%
}%
\begin{pgfscope}%
\pgfsys@transformshift{0.800000in}{2.376000in}%
\pgfsys@useobject{currentmarker}{}%
\end{pgfscope}%
\end{pgfscope}%
\begin{pgfscope}%
\definecolor{textcolor}{rgb}{0.000000,0.000000,0.000000}%
\pgfsetstrokecolor{textcolor}%
\pgfsetfillcolor{textcolor}%
\pgftext[x=0.437682in, y=2.323238in, left, base]{\color{textcolor}\sffamily\fontsize{10.000000}{12.000000}\selectfont 400}%
\end{pgfscope}%
\begin{pgfscope}%
\pgfsetbuttcap%
\pgfsetroundjoin%
\definecolor{currentfill}{rgb}{0.000000,0.000000,0.000000}%
\pgfsetfillcolor{currentfill}%
\pgfsetlinewidth{0.803000pt}%
\definecolor{currentstroke}{rgb}{0.000000,0.000000,0.000000}%
\pgfsetstrokecolor{currentstroke}%
\pgfsetdash{}{0pt}%
\pgfsys@defobject{currentmarker}{\pgfqpoint{-0.048611in}{0.000000in}}{\pgfqpoint{-0.000000in}{0.000000in}}{%
\pgfpathmoveto{\pgfqpoint{-0.000000in}{0.000000in}}%
\pgfpathlineto{\pgfqpoint{-0.048611in}{0.000000in}}%
\pgfusepath{stroke,fill}%
}%
\begin{pgfscope}%
\pgfsys@transformshift{0.800000in}{2.838000in}%
\pgfsys@useobject{currentmarker}{}%
\end{pgfscope}%
\end{pgfscope}%
\begin{pgfscope}%
\definecolor{textcolor}{rgb}{0.000000,0.000000,0.000000}%
\pgfsetstrokecolor{textcolor}%
\pgfsetfillcolor{textcolor}%
\pgftext[x=0.437682in, y=2.785238in, left, base]{\color{textcolor}\sffamily\fontsize{10.000000}{12.000000}\selectfont 500}%
\end{pgfscope}%
\begin{pgfscope}%
\pgfsetbuttcap%
\pgfsetroundjoin%
\definecolor{currentfill}{rgb}{0.000000,0.000000,0.000000}%
\pgfsetfillcolor{currentfill}%
\pgfsetlinewidth{0.803000pt}%
\definecolor{currentstroke}{rgb}{0.000000,0.000000,0.000000}%
\pgfsetstrokecolor{currentstroke}%
\pgfsetdash{}{0pt}%
\pgfsys@defobject{currentmarker}{\pgfqpoint{-0.048611in}{0.000000in}}{\pgfqpoint{-0.000000in}{0.000000in}}{%
\pgfpathmoveto{\pgfqpoint{-0.000000in}{0.000000in}}%
\pgfpathlineto{\pgfqpoint{-0.048611in}{0.000000in}}%
\pgfusepath{stroke,fill}%
}%
\begin{pgfscope}%
\pgfsys@transformshift{0.800000in}{3.300000in}%
\pgfsys@useobject{currentmarker}{}%
\end{pgfscope}%
\end{pgfscope}%
\begin{pgfscope}%
\definecolor{textcolor}{rgb}{0.000000,0.000000,0.000000}%
\pgfsetstrokecolor{textcolor}%
\pgfsetfillcolor{textcolor}%
\pgftext[x=0.437682in, y=3.247238in, left, base]{\color{textcolor}\sffamily\fontsize{10.000000}{12.000000}\selectfont 600}%
\end{pgfscope}%
\begin{pgfscope}%
\pgfsetbuttcap%
\pgfsetroundjoin%
\definecolor{currentfill}{rgb}{0.000000,0.000000,0.000000}%
\pgfsetfillcolor{currentfill}%
\pgfsetlinewidth{0.803000pt}%
\definecolor{currentstroke}{rgb}{0.000000,0.000000,0.000000}%
\pgfsetstrokecolor{currentstroke}%
\pgfsetdash{}{0pt}%
\pgfsys@defobject{currentmarker}{\pgfqpoint{-0.048611in}{0.000000in}}{\pgfqpoint{-0.000000in}{0.000000in}}{%
\pgfpathmoveto{\pgfqpoint{-0.000000in}{0.000000in}}%
\pgfpathlineto{\pgfqpoint{-0.048611in}{0.000000in}}%
\pgfusepath{stroke,fill}%
}%
\begin{pgfscope}%
\pgfsys@transformshift{0.800000in}{3.762000in}%
\pgfsys@useobject{currentmarker}{}%
\end{pgfscope}%
\end{pgfscope}%
\begin{pgfscope}%
\definecolor{textcolor}{rgb}{0.000000,0.000000,0.000000}%
\pgfsetstrokecolor{textcolor}%
\pgfsetfillcolor{textcolor}%
\pgftext[x=0.437682in, y=3.709238in, left, base]{\color{textcolor}\sffamily\fontsize{10.000000}{12.000000}\selectfont 700}%
\end{pgfscope}%
\begin{pgfscope}%
\pgfsetbuttcap%
\pgfsetroundjoin%
\definecolor{currentfill}{rgb}{0.000000,0.000000,0.000000}%
\pgfsetfillcolor{currentfill}%
\pgfsetlinewidth{0.803000pt}%
\definecolor{currentstroke}{rgb}{0.000000,0.000000,0.000000}%
\pgfsetstrokecolor{currentstroke}%
\pgfsetdash{}{0pt}%
\pgfsys@defobject{currentmarker}{\pgfqpoint{-0.048611in}{0.000000in}}{\pgfqpoint{-0.000000in}{0.000000in}}{%
\pgfpathmoveto{\pgfqpoint{-0.000000in}{0.000000in}}%
\pgfpathlineto{\pgfqpoint{-0.048611in}{0.000000in}}%
\pgfusepath{stroke,fill}%
}%
\begin{pgfscope}%
\pgfsys@transformshift{0.800000in}{4.224000in}%
\pgfsys@useobject{currentmarker}{}%
\end{pgfscope}%
\end{pgfscope}%
\begin{pgfscope}%
\definecolor{textcolor}{rgb}{0.000000,0.000000,0.000000}%
\pgfsetstrokecolor{textcolor}%
\pgfsetfillcolor{textcolor}%
\pgftext[x=0.437682in, y=4.171238in, left, base]{\color{textcolor}\sffamily\fontsize{10.000000}{12.000000}\selectfont 800}%
\end{pgfscope}%
\begin{pgfscope}%
\definecolor{textcolor}{rgb}{0.000000,0.000000,0.000000}%
\pgfsetstrokecolor{textcolor}%
\pgfsetfillcolor{textcolor}%
\pgftext[x=0.382126in,y=2.376000in,,bottom,rotate=90.000000]{\color{textcolor}\sffamily\fontsize{10.000000}{12.000000}\selectfont goodput (req/s)}%
\end{pgfscope}%
\begin{pgfscope}%
\pgfpathrectangle{\pgfqpoint{0.800000in}{0.528000in}}{\pgfqpoint{4.960000in}{3.696000in}}%
\pgfusepath{clip}%
\pgfsetbuttcap%
\pgfsetroundjoin%
\pgfsetlinewidth{1.505625pt}%
\definecolor{currentstroke}{rgb}{0.003922,0.450980,0.698039}%
\pgfsetstrokecolor{currentstroke}%
\pgfsetdash{{5.550000pt}{2.400000pt}}{0.000000pt}%
\pgfpathmoveto{\pgfqpoint{0.802480in}{0.532620in}}%
\pgfpathlineto{\pgfqpoint{0.862000in}{0.643493in}}%
\pgfpathlineto{\pgfqpoint{0.924000in}{0.759000in}}%
\pgfpathlineto{\pgfqpoint{1.048000in}{0.989852in}}%
\pgfpathlineto{\pgfqpoint{1.296000in}{1.451191in}}%
\pgfpathlineto{\pgfqpoint{1.792000in}{2.339327in}}%
\pgfpathlineto{\pgfqpoint{2.288000in}{2.923077in}}%
\pgfpathlineto{\pgfqpoint{2.784000in}{3.323262in}}%
\pgfpathlineto{\pgfqpoint{3.280000in}{3.669477in}}%
\pgfpathlineto{\pgfqpoint{3.776000in}{3.983840in}}%
\pgfpathlineto{\pgfqpoint{4.272000in}{4.205289in}}%
\pgfpathlineto{\pgfqpoint{4.326523in}{4.234000in}}%
\pgfusepath{stroke}%
\end{pgfscope}%
\begin{pgfscope}%
\pgfsetrectcap%
\pgfsetmiterjoin%
\pgfsetlinewidth{0.803000pt}%
\definecolor{currentstroke}{rgb}{0.000000,0.000000,0.000000}%
\pgfsetstrokecolor{currentstroke}%
\pgfsetdash{}{0pt}%
\pgfpathmoveto{\pgfqpoint{0.800000in}{0.528000in}}%
\pgfpathlineto{\pgfqpoint{0.800000in}{4.224000in}}%
\pgfusepath{stroke}%
\end{pgfscope}%
\begin{pgfscope}%
\pgfsetrectcap%
\pgfsetmiterjoin%
\pgfsetlinewidth{0.803000pt}%
\definecolor{currentstroke}{rgb}{0.000000,0.000000,0.000000}%
\pgfsetstrokecolor{currentstroke}%
\pgfsetdash{}{0pt}%
\pgfpathmoveto{\pgfqpoint{5.760000in}{0.528000in}}%
\pgfpathlineto{\pgfqpoint{5.760000in}{4.224000in}}%
\pgfusepath{stroke}%
\end{pgfscope}%
\begin{pgfscope}%
\pgfsetrectcap%
\pgfsetmiterjoin%
\pgfsetlinewidth{0.803000pt}%
\definecolor{currentstroke}{rgb}{0.000000,0.000000,0.000000}%
\pgfsetstrokecolor{currentstroke}%
\pgfsetdash{}{0pt}%
\pgfpathmoveto{\pgfqpoint{0.800000in}{0.528000in}}%
\pgfpathlineto{\pgfqpoint{5.760000in}{0.528000in}}%
\pgfusepath{stroke}%
\end{pgfscope}%
\begin{pgfscope}%
\pgfsetrectcap%
\pgfsetmiterjoin%
\pgfsetlinewidth{0.803000pt}%
\definecolor{currentstroke}{rgb}{0.000000,0.000000,0.000000}%
\pgfsetstrokecolor{currentstroke}%
\pgfsetdash{}{0pt}%
\pgfpathmoveto{\pgfqpoint{0.800000in}{4.224000in}}%
\pgfpathlineto{\pgfqpoint{5.760000in}{4.224000in}}%
\pgfusepath{stroke}%
\end{pgfscope}%
\begin{pgfscope}%
\pgfsetbuttcap%
\pgfsetmiterjoin%
\definecolor{currentfill}{rgb}{1.000000,1.000000,1.000000}%
\pgfsetfillcolor{currentfill}%
\pgfsetfillopacity{0.800000}%
\pgfsetlinewidth{1.003750pt}%
\definecolor{currentstroke}{rgb}{0.800000,0.800000,0.800000}%
\pgfsetstrokecolor{currentstroke}%
\pgfsetstrokeopacity{0.800000}%
\pgfsetdash{}{0pt}%
\pgfpathmoveto{\pgfqpoint{5.098528in}{3.705174in}}%
\pgfpathlineto{\pgfqpoint{5.662778in}{3.705174in}}%
\pgfpathquadraticcurveto{\pgfqpoint{5.690556in}{3.705174in}}{\pgfqpoint{5.690556in}{3.732952in}}%
\pgfpathlineto{\pgfqpoint{5.690556in}{4.126778in}}%
\pgfpathquadraticcurveto{\pgfqpoint{5.690556in}{4.154556in}}{\pgfqpoint{5.662778in}{4.154556in}}%
\pgfpathlineto{\pgfqpoint{5.098528in}{4.154556in}}%
\pgfpathquadraticcurveto{\pgfqpoint{5.070750in}{4.154556in}}{\pgfqpoint{5.070750in}{4.126778in}}%
\pgfpathlineto{\pgfqpoint{5.070750in}{3.732952in}}%
\pgfpathquadraticcurveto{\pgfqpoint{5.070750in}{3.705174in}}{\pgfqpoint{5.098528in}{3.705174in}}%
\pgfpathlineto{\pgfqpoint{5.098528in}{3.705174in}}%
\pgfpathclose%
\pgfusepath{stroke,fill}%
\end{pgfscope}%
\begin{pgfscope}%
\definecolor{textcolor}{rgb}{0.000000,0.000000,0.000000}%
\pgfsetstrokecolor{textcolor}%
\pgfsetfillcolor{textcolor}%
\pgftext[x=5.126306in,y=3.993477in,left,base]{\color{textcolor}\sffamily\fontsize{10.000000}{12.000000}\selectfont version}%
\end{pgfscope}%
\begin{pgfscope}%
\pgfsetrectcap%
\pgfsetroundjoin%
\pgfsetlinewidth{1.505625pt}%
\definecolor{currentstroke}{rgb}{0.003922,0.450980,0.698039}%
\pgfsetstrokecolor{currentstroke}%
\pgfsetdash{}{0pt}%
\pgfpathmoveto{\pgfqpoint{5.142026in}{3.838231in}}%
\pgfpathlineto{\pgfqpoint{5.280915in}{3.838231in}}%
\pgfpathlineto{\pgfqpoint{5.419804in}{3.838231in}}%
\pgfusepath{stroke}%
\end{pgfscope}%
\begin{pgfscope}%
\definecolor{textcolor}{rgb}{0.000000,0.000000,0.000000}%
\pgfsetstrokecolor{textcolor}%
\pgfsetfillcolor{textcolor}%
\pgftext[x=5.530915in,y=3.789620in,left,base]{\color{textcolor}\sffamily\fontsize{10.000000}{12.000000}\selectfont 1}%
\end{pgfscope}%
\end{pgfpicture}%
\makeatother%
\endgroup%
}
\caption{Benchmarking of goodput for varying throughputs and implementation versions.}
\label{throughputgoodputablation}
\end{figure}

\begin{figure}[h!]
\centering
\resizebox{.6\textwidth}{!}{%% Creator: Matplotlib, PGF backend
%%
%% To include the figure in your LaTeX document, write
%%   \input{<filename>.pgf}
%%
%% Make sure the required packages are loaded in your preamble
%%   \usepackage{pgf}
%%
%% Also ensure that all the required font packages are loaded; for instance,
%% the lmodern package is sometimes necessary when using math font.
%%   \usepackage{lmodern}
%%
%% Figures using additional raster images can only be included by \input if
%% they are in the same directory as the main LaTeX file. For loading figures
%% from other directories you can use the `import` package
%%   \usepackage{import}
%%
%% and then include the figures with
%%   \import{<path to file>}{<filename>.pgf}
%%
%% Matplotlib used the following preamble
%%   
%%   \usepackage{fontspec}
%%   \setmainfont{DejaVuSerif.ttf}[Path=\detokenize{/opt/homebrew/lib/python3.10/site-packages/matplotlib/mpl-data/fonts/ttf/}]
%%   \setsansfont{DejaVuSans.ttf}[Path=\detokenize{/opt/homebrew/lib/python3.10/site-packages/matplotlib/mpl-data/fonts/ttf/}]
%%   \setmonofont{DejaVuSansMono.ttf}[Path=\detokenize{/opt/homebrew/lib/python3.10/site-packages/matplotlib/mpl-data/fonts/ttf/}]
%%   \makeatletter\@ifpackageloaded{underscore}{}{\usepackage[strings]{underscore}}\makeatother
%%
\begingroup%
\makeatletter%
\begin{pgfpicture}%
\pgfpathrectangle{\pgfpointorigin}{\pgfqpoint{5.610000in}{5.000000in}}%
\pgfusepath{use as bounding box, clip}%
\begin{pgfscope}%
\pgfsetbuttcap%
\pgfsetmiterjoin%
\definecolor{currentfill}{rgb}{1.000000,1.000000,1.000000}%
\pgfsetfillcolor{currentfill}%
\pgfsetlinewidth{0.000000pt}%
\definecolor{currentstroke}{rgb}{1.000000,1.000000,1.000000}%
\pgfsetstrokecolor{currentstroke}%
\pgfsetdash{}{0pt}%
\pgfpathmoveto{\pgfqpoint{0.000000in}{0.000000in}}%
\pgfpathlineto{\pgfqpoint{5.610000in}{0.000000in}}%
\pgfpathlineto{\pgfqpoint{5.610000in}{5.000000in}}%
\pgfpathlineto{\pgfqpoint{0.000000in}{5.000000in}}%
\pgfpathlineto{\pgfqpoint{0.000000in}{0.000000in}}%
\pgfpathclose%
\pgfusepath{fill}%
\end{pgfscope}%
\begin{pgfscope}%
\pgfsetbuttcap%
\pgfsetmiterjoin%
\definecolor{currentfill}{rgb}{1.000000,1.000000,1.000000}%
\pgfsetfillcolor{currentfill}%
\pgfsetlinewidth{0.000000pt}%
\definecolor{currentstroke}{rgb}{0.000000,0.000000,0.000000}%
\pgfsetstrokecolor{currentstroke}%
\pgfsetstrokeopacity{0.000000}%
\pgfsetdash{}{0pt}%
\pgfpathmoveto{\pgfqpoint{0.745740in}{0.582778in}}%
\pgfpathlineto{\pgfqpoint{4.935682in}{0.582778in}}%
\pgfpathlineto{\pgfqpoint{4.935682in}{4.850000in}}%
\pgfpathlineto{\pgfqpoint{0.745740in}{4.850000in}}%
\pgfpathlineto{\pgfqpoint{0.745740in}{0.582778in}}%
\pgfpathclose%
\pgfusepath{fill}%
\end{pgfscope}%
\begin{pgfscope}%
\pgfpathrectangle{\pgfqpoint{0.745740in}{0.582778in}}{\pgfqpoint{4.189941in}{4.267222in}}%
\pgfusepath{clip}%
\pgfsetbuttcap%
\pgfsetroundjoin%
\definecolor{currentfill}{rgb}{0.003922,0.450980,0.698039}%
\pgfsetfillcolor{currentfill}%
\pgfsetfillopacity{0.800000}%
\pgfsetlinewidth{1.003750pt}%
\definecolor{currentstroke}{rgb}{0.003922,0.450980,0.698039}%
\pgfsetstrokecolor{currentstroke}%
\pgfsetstrokeopacity{0.800000}%
\pgfsetdash{}{0pt}%
\pgfsys@defobject{currentmarker}{\pgfqpoint{-0.041667in}{-0.041667in}}{\pgfqpoint{0.041667in}{0.041667in}}{%
\pgfpathmoveto{\pgfqpoint{0.000000in}{-0.041667in}}%
\pgfpathcurveto{\pgfqpoint{0.011050in}{-0.041667in}}{\pgfqpoint{0.021649in}{-0.037276in}}{\pgfqpoint{0.029463in}{-0.029463in}}%
\pgfpathcurveto{\pgfqpoint{0.037276in}{-0.021649in}}{\pgfqpoint{0.041667in}{-0.011050in}}{\pgfqpoint{0.041667in}{0.000000in}}%
\pgfpathcurveto{\pgfqpoint{0.041667in}{0.011050in}}{\pgfqpoint{0.037276in}{0.021649in}}{\pgfqpoint{0.029463in}{0.029463in}}%
\pgfpathcurveto{\pgfqpoint{0.021649in}{0.037276in}}{\pgfqpoint{0.011050in}{0.041667in}}{\pgfqpoint{0.000000in}{0.041667in}}%
\pgfpathcurveto{\pgfqpoint{-0.011050in}{0.041667in}}{\pgfqpoint{-0.021649in}{0.037276in}}{\pgfqpoint{-0.029463in}{0.029463in}}%
\pgfpathcurveto{\pgfqpoint{-0.037276in}{0.021649in}}{\pgfqpoint{-0.041667in}{0.011050in}}{\pgfqpoint{-0.041667in}{0.000000in}}%
\pgfpathcurveto{\pgfqpoint{-0.041667in}{-0.011050in}}{\pgfqpoint{-0.037276in}{-0.021649in}}{\pgfqpoint{-0.029463in}{-0.029463in}}%
\pgfpathcurveto{\pgfqpoint{-0.021649in}{-0.037276in}}{\pgfqpoint{-0.011050in}{-0.041667in}}{\pgfqpoint{0.000000in}{-0.041667in}}%
\pgfpathlineto{\pgfqpoint{0.000000in}{-0.041667in}}%
\pgfpathclose%
\pgfusepath{stroke,fill}%
}%
\begin{pgfscope}%
\pgfsys@transformshift{2.700546in}{1.444340in}%
\pgfsys@useobject{currentmarker}{}%
\end{pgfscope}%
\begin{pgfscope}%
\pgfsys@transformshift{0.980981in}{1.806548in}%
\pgfsys@useobject{currentmarker}{}%
\end{pgfscope}%
\begin{pgfscope}%
\pgfsys@transformshift{2.663156in}{1.435060in}%
\pgfsys@useobject{currentmarker}{}%
\end{pgfscope}%
\begin{pgfscope}%
\pgfsys@transformshift{0.936192in}{1.742477in}%
\pgfsys@useobject{currentmarker}{}%
\end{pgfscope}%
\begin{pgfscope}%
\pgfsys@transformshift{4.300646in}{1.842437in}%
\pgfsys@useobject{currentmarker}{}%
\end{pgfscope}%
\begin{pgfscope}%
\pgfsys@transformshift{0.959604in}{1.524820in}%
\pgfsys@useobject{currentmarker}{}%
\end{pgfscope}%
\begin{pgfscope}%
\pgfsys@transformshift{3.993386in}{1.515220in}%
\pgfsys@useobject{currentmarker}{}%
\end{pgfscope}%
\begin{pgfscope}%
\pgfsys@transformshift{4.301573in}{2.136516in}%
\pgfsys@useobject{currentmarker}{}%
\end{pgfscope}%
\begin{pgfscope}%
\pgfsys@transformshift{2.738793in}{1.462927in}%
\pgfsys@useobject{currentmarker}{}%
\end{pgfscope}%
\begin{pgfscope}%
\pgfsys@transformshift{4.745230in}{3.942592in}%
\pgfsys@useobject{currentmarker}{}%
\end{pgfscope}%
\begin{pgfscope}%
\pgfsys@transformshift{4.414488in}{2.620191in}%
\pgfsys@useobject{currentmarker}{}%
\end{pgfscope}%
\begin{pgfscope}%
\pgfsys@transformshift{4.052925in}{3.190900in}%
\pgfsys@useobject{currentmarker}{}%
\end{pgfscope}%
\end{pgfscope}%
\begin{pgfscope}%
\pgfsetbuttcap%
\pgfsetroundjoin%
\definecolor{currentfill}{rgb}{0.000000,0.000000,0.000000}%
\pgfsetfillcolor{currentfill}%
\pgfsetlinewidth{0.803000pt}%
\definecolor{currentstroke}{rgb}{0.000000,0.000000,0.000000}%
\pgfsetstrokecolor{currentstroke}%
\pgfsetdash{}{0pt}%
\pgfsys@defobject{currentmarker}{\pgfqpoint{0.000000in}{-0.048611in}}{\pgfqpoint{0.000000in}{0.000000in}}{%
\pgfpathmoveto{\pgfqpoint{0.000000in}{0.000000in}}%
\pgfpathlineto{\pgfqpoint{0.000000in}{-0.048611in}}%
\pgfusepath{stroke,fill}%
}%
\begin{pgfscope}%
\pgfsys@transformshift{2.821067in}{0.582778in}%
\pgfsys@useobject{currentmarker}{}%
\end{pgfscope}%
\end{pgfscope}%
\begin{pgfscope}%
\definecolor{textcolor}{rgb}{0.000000,0.000000,0.000000}%
\pgfsetstrokecolor{textcolor}%
\pgfsetfillcolor{textcolor}%
\pgftext[x=2.821067in,y=0.485556in,,top]{\color{textcolor}\sffamily\fontsize{10.000000}{12.000000}\selectfont \(\displaystyle {10^{2}}\)}%
\end{pgfscope}%
\begin{pgfscope}%
\pgfsetbuttcap%
\pgfsetroundjoin%
\definecolor{currentfill}{rgb}{0.000000,0.000000,0.000000}%
\pgfsetfillcolor{currentfill}%
\pgfsetlinewidth{0.602250pt}%
\definecolor{currentstroke}{rgb}{0.000000,0.000000,0.000000}%
\pgfsetstrokecolor{currentstroke}%
\pgfsetdash{}{0pt}%
\pgfsys@defobject{currentmarker}{\pgfqpoint{0.000000in}{-0.027778in}}{\pgfqpoint{0.000000in}{0.000000in}}{%
\pgfpathmoveto{\pgfqpoint{0.000000in}{0.000000in}}%
\pgfpathlineto{\pgfqpoint{0.000000in}{-0.027778in}}%
\pgfusepath{stroke,fill}%
}%
\begin{pgfscope}%
\pgfsys@transformshift{1.150731in}{0.582778in}%
\pgfsys@useobject{currentmarker}{}%
\end{pgfscope}%
\end{pgfscope}%
\begin{pgfscope}%
\pgfsetbuttcap%
\pgfsetroundjoin%
\definecolor{currentfill}{rgb}{0.000000,0.000000,0.000000}%
\pgfsetfillcolor{currentfill}%
\pgfsetlinewidth{0.602250pt}%
\definecolor{currentstroke}{rgb}{0.000000,0.000000,0.000000}%
\pgfsetstrokecolor{currentstroke}%
\pgfsetdash{}{0pt}%
\pgfsys@defobject{currentmarker}{\pgfqpoint{0.000000in}{-0.027778in}}{\pgfqpoint{0.000000in}{0.000000in}}{%
\pgfpathmoveto{\pgfqpoint{0.000000in}{0.000000in}}%
\pgfpathlineto{\pgfqpoint{0.000000in}{-0.027778in}}%
\pgfusepath{stroke,fill}%
}%
\begin{pgfscope}%
\pgfsys@transformshift{1.590087in}{0.582778in}%
\pgfsys@useobject{currentmarker}{}%
\end{pgfscope}%
\end{pgfscope}%
\begin{pgfscope}%
\definecolor{textcolor}{rgb}{0.000000,0.000000,0.000000}%
\pgfsetstrokecolor{textcolor}%
\pgfsetfillcolor{textcolor}%
\pgftext[x=1.590087in,y=0.507778in,,top]{\color{textcolor}\sffamily\fontsize{10.000000}{12.000000}\selectfont \(\displaystyle {6\times10^{1}}\)}%
\end{pgfscope}%
\begin{pgfscope}%
\pgfsetbuttcap%
\pgfsetroundjoin%
\definecolor{currentfill}{rgb}{0.000000,0.000000,0.000000}%
\pgfsetfillcolor{currentfill}%
\pgfsetlinewidth{0.602250pt}%
\definecolor{currentstroke}{rgb}{0.000000,0.000000,0.000000}%
\pgfsetstrokecolor{currentstroke}%
\pgfsetdash{}{0pt}%
\pgfsys@defobject{currentmarker}{\pgfqpoint{0.000000in}{-0.027778in}}{\pgfqpoint{0.000000in}{0.000000in}}{%
\pgfpathmoveto{\pgfqpoint{0.000000in}{0.000000in}}%
\pgfpathlineto{\pgfqpoint{0.000000in}{-0.027778in}}%
\pgfusepath{stroke,fill}%
}%
\begin{pgfscope}%
\pgfsys@transformshift{1.961557in}{0.582778in}%
\pgfsys@useobject{currentmarker}{}%
\end{pgfscope}%
\end{pgfscope}%
\begin{pgfscope}%
\pgfsetbuttcap%
\pgfsetroundjoin%
\definecolor{currentfill}{rgb}{0.000000,0.000000,0.000000}%
\pgfsetfillcolor{currentfill}%
\pgfsetlinewidth{0.602250pt}%
\definecolor{currentstroke}{rgb}{0.000000,0.000000,0.000000}%
\pgfsetstrokecolor{currentstroke}%
\pgfsetdash{}{0pt}%
\pgfsys@defobject{currentmarker}{\pgfqpoint{0.000000in}{-0.027778in}}{\pgfqpoint{0.000000in}{0.000000in}}{%
\pgfpathmoveto{\pgfqpoint{0.000000in}{0.000000in}}%
\pgfpathlineto{\pgfqpoint{0.000000in}{-0.027778in}}%
\pgfusepath{stroke,fill}%
}%
\begin{pgfscope}%
\pgfsys@transformshift{2.283339in}{0.582778in}%
\pgfsys@useobject{currentmarker}{}%
\end{pgfscope}%
\end{pgfscope}%
\begin{pgfscope}%
\pgfsetbuttcap%
\pgfsetroundjoin%
\definecolor{currentfill}{rgb}{0.000000,0.000000,0.000000}%
\pgfsetfillcolor{currentfill}%
\pgfsetlinewidth{0.602250pt}%
\definecolor{currentstroke}{rgb}{0.000000,0.000000,0.000000}%
\pgfsetstrokecolor{currentstroke}%
\pgfsetdash{}{0pt}%
\pgfsys@defobject{currentmarker}{\pgfqpoint{0.000000in}{-0.027778in}}{\pgfqpoint{0.000000in}{0.000000in}}{%
\pgfpathmoveto{\pgfqpoint{0.000000in}{0.000000in}}%
\pgfpathlineto{\pgfqpoint{0.000000in}{-0.027778in}}%
\pgfusepath{stroke,fill}%
}%
\begin{pgfscope}%
\pgfsys@transformshift{2.567171in}{0.582778in}%
\pgfsys@useobject{currentmarker}{}%
\end{pgfscope}%
\end{pgfscope}%
\begin{pgfscope}%
\pgfsetbuttcap%
\pgfsetroundjoin%
\definecolor{currentfill}{rgb}{0.000000,0.000000,0.000000}%
\pgfsetfillcolor{currentfill}%
\pgfsetlinewidth{0.602250pt}%
\definecolor{currentstroke}{rgb}{0.000000,0.000000,0.000000}%
\pgfsetstrokecolor{currentstroke}%
\pgfsetdash{}{0pt}%
\pgfsys@defobject{currentmarker}{\pgfqpoint{0.000000in}{-0.027778in}}{\pgfqpoint{0.000000in}{0.000000in}}{%
\pgfpathmoveto{\pgfqpoint{0.000000in}{0.000000in}}%
\pgfpathlineto{\pgfqpoint{0.000000in}{-0.027778in}}%
\pgfusepath{stroke,fill}%
}%
\begin{pgfscope}%
\pgfsys@transformshift{4.491404in}{0.582778in}%
\pgfsys@useobject{currentmarker}{}%
\end{pgfscope}%
\end{pgfscope}%
\begin{pgfscope}%
\definecolor{textcolor}{rgb}{0.000000,0.000000,0.000000}%
\pgfsetstrokecolor{textcolor}%
\pgfsetfillcolor{textcolor}%
\pgftext[x=4.491404in,y=0.507778in,,top]{\color{textcolor}\sffamily\fontsize{10.000000}{12.000000}\selectfont \(\displaystyle {2\times10^{2}}\)}%
\end{pgfscope}%
\begin{pgfscope}%
\definecolor{textcolor}{rgb}{0.000000,0.000000,0.000000}%
\pgfsetstrokecolor{textcolor}%
\pgfsetfillcolor{textcolor}%
\pgftext[x=2.840711in,y=0.295587in,,top]{\color{textcolor}\sffamily\fontsize{10.000000}{12.000000}\selectfont goodput (req/s)}%
\end{pgfscope}%
\begin{pgfscope}%
\pgfsetbuttcap%
\pgfsetroundjoin%
\definecolor{currentfill}{rgb}{0.000000,0.000000,0.000000}%
\pgfsetfillcolor{currentfill}%
\pgfsetlinewidth{0.803000pt}%
\definecolor{currentstroke}{rgb}{0.000000,0.000000,0.000000}%
\pgfsetstrokecolor{currentstroke}%
\pgfsetdash{}{0pt}%
\pgfsys@defobject{currentmarker}{\pgfqpoint{-0.048611in}{0.000000in}}{\pgfqpoint{-0.000000in}{0.000000in}}{%
\pgfpathmoveto{\pgfqpoint{-0.000000in}{0.000000in}}%
\pgfpathlineto{\pgfqpoint{-0.048611in}{0.000000in}}%
\pgfusepath{stroke,fill}%
}%
\begin{pgfscope}%
\pgfsys@transformshift{0.745740in}{0.582778in}%
\pgfsys@useobject{currentmarker}{}%
\end{pgfscope}%
\end{pgfscope}%
\begin{pgfscope}%
\definecolor{textcolor}{rgb}{0.000000,0.000000,0.000000}%
\pgfsetstrokecolor{textcolor}%
\pgfsetfillcolor{textcolor}%
\pgftext[x=0.427639in, y=0.530016in, left, base]{\color{textcolor}\sffamily\fontsize{10.000000}{12.000000}\selectfont 0.0}%
\end{pgfscope}%
\begin{pgfscope}%
\pgfsetbuttcap%
\pgfsetroundjoin%
\definecolor{currentfill}{rgb}{0.000000,0.000000,0.000000}%
\pgfsetfillcolor{currentfill}%
\pgfsetlinewidth{0.803000pt}%
\definecolor{currentstroke}{rgb}{0.000000,0.000000,0.000000}%
\pgfsetstrokecolor{currentstroke}%
\pgfsetdash{}{0pt}%
\pgfsys@defobject{currentmarker}{\pgfqpoint{-0.048611in}{0.000000in}}{\pgfqpoint{-0.000000in}{0.000000in}}{%
\pgfpathmoveto{\pgfqpoint{-0.000000in}{0.000000in}}%
\pgfpathlineto{\pgfqpoint{-0.048611in}{0.000000in}}%
\pgfusepath{stroke,fill}%
}%
\begin{pgfscope}%
\pgfsys@transformshift{0.745740in}{1.436222in}%
\pgfsys@useobject{currentmarker}{}%
\end{pgfscope}%
\end{pgfscope}%
\begin{pgfscope}%
\definecolor{textcolor}{rgb}{0.000000,0.000000,0.000000}%
\pgfsetstrokecolor{textcolor}%
\pgfsetfillcolor{textcolor}%
\pgftext[x=0.427639in, y=1.383461in, left, base]{\color{textcolor}\sffamily\fontsize{10.000000}{12.000000}\selectfont 0.2}%
\end{pgfscope}%
\begin{pgfscope}%
\pgfsetbuttcap%
\pgfsetroundjoin%
\definecolor{currentfill}{rgb}{0.000000,0.000000,0.000000}%
\pgfsetfillcolor{currentfill}%
\pgfsetlinewidth{0.803000pt}%
\definecolor{currentstroke}{rgb}{0.000000,0.000000,0.000000}%
\pgfsetstrokecolor{currentstroke}%
\pgfsetdash{}{0pt}%
\pgfsys@defobject{currentmarker}{\pgfqpoint{-0.048611in}{0.000000in}}{\pgfqpoint{-0.000000in}{0.000000in}}{%
\pgfpathmoveto{\pgfqpoint{-0.000000in}{0.000000in}}%
\pgfpathlineto{\pgfqpoint{-0.048611in}{0.000000in}}%
\pgfusepath{stroke,fill}%
}%
\begin{pgfscope}%
\pgfsys@transformshift{0.745740in}{2.289667in}%
\pgfsys@useobject{currentmarker}{}%
\end{pgfscope}%
\end{pgfscope}%
\begin{pgfscope}%
\definecolor{textcolor}{rgb}{0.000000,0.000000,0.000000}%
\pgfsetstrokecolor{textcolor}%
\pgfsetfillcolor{textcolor}%
\pgftext[x=0.427639in, y=2.236905in, left, base]{\color{textcolor}\sffamily\fontsize{10.000000}{12.000000}\selectfont 0.4}%
\end{pgfscope}%
\begin{pgfscope}%
\pgfsetbuttcap%
\pgfsetroundjoin%
\definecolor{currentfill}{rgb}{0.000000,0.000000,0.000000}%
\pgfsetfillcolor{currentfill}%
\pgfsetlinewidth{0.803000pt}%
\definecolor{currentstroke}{rgb}{0.000000,0.000000,0.000000}%
\pgfsetstrokecolor{currentstroke}%
\pgfsetdash{}{0pt}%
\pgfsys@defobject{currentmarker}{\pgfqpoint{-0.048611in}{0.000000in}}{\pgfqpoint{-0.000000in}{0.000000in}}{%
\pgfpathmoveto{\pgfqpoint{-0.000000in}{0.000000in}}%
\pgfpathlineto{\pgfqpoint{-0.048611in}{0.000000in}}%
\pgfusepath{stroke,fill}%
}%
\begin{pgfscope}%
\pgfsys@transformshift{0.745740in}{3.143111in}%
\pgfsys@useobject{currentmarker}{}%
\end{pgfscope}%
\end{pgfscope}%
\begin{pgfscope}%
\definecolor{textcolor}{rgb}{0.000000,0.000000,0.000000}%
\pgfsetstrokecolor{textcolor}%
\pgfsetfillcolor{textcolor}%
\pgftext[x=0.427639in, y=3.090350in, left, base]{\color{textcolor}\sffamily\fontsize{10.000000}{12.000000}\selectfont 0.6}%
\end{pgfscope}%
\begin{pgfscope}%
\pgfsetbuttcap%
\pgfsetroundjoin%
\definecolor{currentfill}{rgb}{0.000000,0.000000,0.000000}%
\pgfsetfillcolor{currentfill}%
\pgfsetlinewidth{0.803000pt}%
\definecolor{currentstroke}{rgb}{0.000000,0.000000,0.000000}%
\pgfsetstrokecolor{currentstroke}%
\pgfsetdash{}{0pt}%
\pgfsys@defobject{currentmarker}{\pgfqpoint{-0.048611in}{0.000000in}}{\pgfqpoint{-0.000000in}{0.000000in}}{%
\pgfpathmoveto{\pgfqpoint{-0.000000in}{0.000000in}}%
\pgfpathlineto{\pgfqpoint{-0.048611in}{0.000000in}}%
\pgfusepath{stroke,fill}%
}%
\begin{pgfscope}%
\pgfsys@transformshift{0.745740in}{3.996556in}%
\pgfsys@useobject{currentmarker}{}%
\end{pgfscope}%
\end{pgfscope}%
\begin{pgfscope}%
\definecolor{textcolor}{rgb}{0.000000,0.000000,0.000000}%
\pgfsetstrokecolor{textcolor}%
\pgfsetfillcolor{textcolor}%
\pgftext[x=0.427639in, y=3.943794in, left, base]{\color{textcolor}\sffamily\fontsize{10.000000}{12.000000}\selectfont 0.8}%
\end{pgfscope}%
\begin{pgfscope}%
\pgfsetbuttcap%
\pgfsetroundjoin%
\definecolor{currentfill}{rgb}{0.000000,0.000000,0.000000}%
\pgfsetfillcolor{currentfill}%
\pgfsetlinewidth{0.803000pt}%
\definecolor{currentstroke}{rgb}{0.000000,0.000000,0.000000}%
\pgfsetstrokecolor{currentstroke}%
\pgfsetdash{}{0pt}%
\pgfsys@defobject{currentmarker}{\pgfqpoint{-0.048611in}{0.000000in}}{\pgfqpoint{-0.000000in}{0.000000in}}{%
\pgfpathmoveto{\pgfqpoint{-0.000000in}{0.000000in}}%
\pgfpathlineto{\pgfqpoint{-0.048611in}{0.000000in}}%
\pgfusepath{stroke,fill}%
}%
\begin{pgfscope}%
\pgfsys@transformshift{0.745740in}{4.850000in}%
\pgfsys@useobject{currentmarker}{}%
\end{pgfscope}%
\end{pgfscope}%
\begin{pgfscope}%
\definecolor{textcolor}{rgb}{0.000000,0.000000,0.000000}%
\pgfsetstrokecolor{textcolor}%
\pgfsetfillcolor{textcolor}%
\pgftext[x=0.427639in, y=4.797238in, left, base]{\color{textcolor}\sffamily\fontsize{10.000000}{12.000000}\selectfont 1.0}%
\end{pgfscope}%
\begin{pgfscope}%
\definecolor{textcolor}{rgb}{0.000000,0.000000,0.000000}%
\pgfsetstrokecolor{textcolor}%
\pgfsetfillcolor{textcolor}%
\pgftext[x=0.372083in,y=2.716389in,,bottom,rotate=90.000000]{\color{textcolor}\sffamily\fontsize{10.000000}{12.000000}\selectfont median latency (s)}%
\end{pgfscope}%
\begin{pgfscope}%
\pgfpathrectangle{\pgfqpoint{0.745740in}{0.582778in}}{\pgfqpoint{4.189941in}{4.267222in}}%
\pgfusepath{clip}%
\pgfsetrectcap%
\pgfsetroundjoin%
\pgfsetlinewidth{2.258437pt}%
\definecolor{currentstroke}{rgb}{0.003922,0.450980,0.698039}%
\pgfsetstrokecolor{currentstroke}%
\pgfsetdash{}{0pt}%
\pgfpathmoveto{\pgfqpoint{0.936192in}{1.736533in}}%
\pgfpathlineto{\pgfqpoint{1.028320in}{1.711671in}}%
\pgfpathlineto{\pgfqpoint{1.117056in}{1.687701in}}%
\pgfpathlineto{\pgfqpoint{1.202639in}{1.664623in}}%
\pgfpathlineto{\pgfqpoint{1.285287in}{1.642437in}}%
\pgfpathlineto{\pgfqpoint{1.365194in}{1.621144in}}%
\pgfpathlineto{\pgfqpoint{1.442537in}{1.600743in}}%
\pgfpathlineto{\pgfqpoint{1.517473in}{1.581234in}}%
\pgfpathlineto{\pgfqpoint{1.590150in}{1.562618in}}%
\pgfpathlineto{\pgfqpoint{1.660699in}{1.544894in}}%
\pgfpathlineto{\pgfqpoint{1.729241in}{1.528062in}}%
\pgfpathlineto{\pgfqpoint{1.795888in}{1.512122in}}%
\pgfpathlineto{\pgfqpoint{1.860740in}{1.497075in}}%
\pgfpathlineto{\pgfqpoint{1.923893in}{1.482920in}}%
\pgfpathlineto{\pgfqpoint{1.985434in}{1.469657in}}%
\pgfpathlineto{\pgfqpoint{2.045441in}{1.457287in}}%
\pgfpathlineto{\pgfqpoint{2.103991in}{1.445809in}}%
\pgfpathlineto{\pgfqpoint{2.161151in}{1.435223in}}%
\pgfpathlineto{\pgfqpoint{2.216987in}{1.425530in}}%
\pgfpathlineto{\pgfqpoint{2.271559in}{1.416729in}}%
\pgfpathlineto{\pgfqpoint{2.324922in}{1.408820in}}%
\pgfpathlineto{\pgfqpoint{2.377129in}{1.401803in}}%
\pgfpathlineto{\pgfqpoint{2.428229in}{1.395679in}}%
\pgfpathlineto{\pgfqpoint{2.478268in}{1.390447in}}%
\pgfpathlineto{\pgfqpoint{2.527289in}{1.386107in}}%
\pgfpathlineto{\pgfqpoint{2.575332in}{1.382660in}}%
\pgfpathlineto{\pgfqpoint{2.622437in}{1.380104in}}%
\pgfpathlineto{\pgfqpoint{2.668638in}{1.378442in}}%
\pgfpathlineto{\pgfqpoint{2.713970in}{1.377671in}}%
\pgfpathlineto{\pgfqpoint{2.758465in}{1.377793in}}%
\pgfpathlineto{\pgfqpoint{2.802153in}{1.378807in}}%
\pgfpathlineto{\pgfqpoint{2.845064in}{1.380713in}}%
\pgfpathlineto{\pgfqpoint{2.887223in}{1.383512in}}%
\pgfpathlineto{\pgfqpoint{2.928658in}{1.387203in}}%
\pgfpathlineto{\pgfqpoint{2.969392in}{1.391786in}}%
\pgfpathlineto{\pgfqpoint{3.009449in}{1.397261in}}%
\pgfpathlineto{\pgfqpoint{3.048852in}{1.403629in}}%
\pgfpathlineto{\pgfqpoint{3.087620in}{1.410889in}}%
\pgfpathlineto{\pgfqpoint{3.125774in}{1.419041in}}%
\pgfpathlineto{\pgfqpoint{3.163334in}{1.428086in}}%
\pgfpathlineto{\pgfqpoint{3.200317in}{1.438023in}}%
\pgfpathlineto{\pgfqpoint{3.236741in}{1.448852in}}%
\pgfpathlineto{\pgfqpoint{3.272623in}{1.460574in}}%
\pgfpathlineto{\pgfqpoint{3.307979in}{1.473188in}}%
\pgfpathlineto{\pgfqpoint{3.342823in}{1.486694in}}%
\pgfpathlineto{\pgfqpoint{3.377171in}{1.501092in}}%
\pgfpathlineto{\pgfqpoint{3.411035in}{1.516383in}}%
\pgfpathlineto{\pgfqpoint{3.444431in}{1.532566in}}%
\pgfpathlineto{\pgfqpoint{3.477370in}{1.549641in}}%
\pgfpathlineto{\pgfqpoint{3.509865in}{1.567609in}}%
\pgfpathlineto{\pgfqpoint{3.541927in}{1.586469in}}%
\pgfpathlineto{\pgfqpoint{3.573569in}{1.606221in}}%
\pgfpathlineto{\pgfqpoint{3.604800in}{1.626865in}}%
\pgfpathlineto{\pgfqpoint{3.635632in}{1.648402in}}%
\pgfpathlineto{\pgfqpoint{3.666074in}{1.670831in}}%
\pgfpathlineto{\pgfqpoint{3.696137in}{1.694152in}}%
\pgfpathlineto{\pgfqpoint{3.725829in}{1.718366in}}%
\pgfpathlineto{\pgfqpoint{3.755160in}{1.743472in}}%
\pgfpathlineto{\pgfqpoint{3.784138in}{1.769470in}}%
\pgfpathlineto{\pgfqpoint{3.812772in}{1.796361in}}%
\pgfpathlineto{\pgfqpoint{3.841069in}{1.824143in}}%
\pgfpathlineto{\pgfqpoint{3.869038in}{1.852818in}}%
\pgfpathlineto{\pgfqpoint{3.896686in}{1.882386in}}%
\pgfpathlineto{\pgfqpoint{3.924021in}{1.912845in}}%
\pgfpathlineto{\pgfqpoint{3.951049in}{1.944197in}}%
\pgfpathlineto{\pgfqpoint{3.977777in}{1.976442in}}%
\pgfpathlineto{\pgfqpoint{4.004212in}{2.009578in}}%
\pgfpathlineto{\pgfqpoint{4.030360in}{2.043607in}}%
\pgfpathlineto{\pgfqpoint{4.056227in}{2.078528in}}%
\pgfpathlineto{\pgfqpoint{4.081820in}{2.114342in}}%
\pgfpathlineto{\pgfqpoint{4.107144in}{2.151047in}}%
\pgfpathlineto{\pgfqpoint{4.132204in}{2.188645in}}%
\pgfpathlineto{\pgfqpoint{4.157006in}{2.227136in}}%
\pgfpathlineto{\pgfqpoint{4.181556in}{2.266518in}}%
\pgfpathlineto{\pgfqpoint{4.205858in}{2.306793in}}%
\pgfpathlineto{\pgfqpoint{4.229918in}{2.347960in}}%
\pgfpathlineto{\pgfqpoint{4.253740in}{2.390020in}}%
\pgfpathlineto{\pgfqpoint{4.277328in}{2.432971in}}%
\pgfpathlineto{\pgfqpoint{4.300688in}{2.476815in}}%
\pgfpathlineto{\pgfqpoint{4.323824in}{2.521552in}}%
\pgfpathlineto{\pgfqpoint{4.346739in}{2.567180in}}%
\pgfpathlineto{\pgfqpoint{4.369439in}{2.613701in}}%
\pgfpathlineto{\pgfqpoint{4.391927in}{2.661115in}}%
\pgfpathlineto{\pgfqpoint{4.414207in}{2.709420in}}%
\pgfpathlineto{\pgfqpoint{4.436283in}{2.758618in}}%
\pgfpathlineto{\pgfqpoint{4.458158in}{2.808708in}}%
\pgfpathlineto{\pgfqpoint{4.479837in}{2.859690in}}%
\pgfpathlineto{\pgfqpoint{4.501322in}{2.911565in}}%
\pgfpathlineto{\pgfqpoint{4.522618in}{2.964332in}}%
\pgfpathlineto{\pgfqpoint{4.543727in}{3.017991in}}%
\pgfpathlineto{\pgfqpoint{4.564652in}{3.072543in}}%
\pgfpathlineto{\pgfqpoint{4.585398in}{3.127987in}}%
\pgfpathlineto{\pgfqpoint{4.605966in}{3.184323in}}%
\pgfpathlineto{\pgfqpoint{4.626361in}{3.241551in}}%
\pgfpathlineto{\pgfqpoint{4.646584in}{3.299672in}}%
\pgfpathlineto{\pgfqpoint{4.666639in}{3.358685in}}%
\pgfpathlineto{\pgfqpoint{4.686528in}{3.418590in}}%
\pgfpathlineto{\pgfqpoint{4.706255in}{3.479388in}}%
\pgfpathlineto{\pgfqpoint{4.725821in}{3.541078in}}%
\pgfpathlineto{\pgfqpoint{4.745230in}{3.603660in}}%
\pgfusepath{stroke}%
\end{pgfscope}%
\begin{pgfscope}%
\pgfsetrectcap%
\pgfsetmiterjoin%
\pgfsetlinewidth{0.803000pt}%
\definecolor{currentstroke}{rgb}{0.000000,0.000000,0.000000}%
\pgfsetstrokecolor{currentstroke}%
\pgfsetdash{}{0pt}%
\pgfpathmoveto{\pgfqpoint{0.745740in}{0.582778in}}%
\pgfpathlineto{\pgfqpoint{0.745740in}{4.850000in}}%
\pgfusepath{stroke}%
\end{pgfscope}%
\begin{pgfscope}%
\pgfsetrectcap%
\pgfsetmiterjoin%
\pgfsetlinewidth{0.803000pt}%
\definecolor{currentstroke}{rgb}{0.000000,0.000000,0.000000}%
\pgfsetstrokecolor{currentstroke}%
\pgfsetdash{}{0pt}%
\pgfpathmoveto{\pgfqpoint{0.745740in}{0.582778in}}%
\pgfpathlineto{\pgfqpoint{4.935682in}{0.582778in}}%
\pgfusepath{stroke}%
\end{pgfscope}%
\begin{pgfscope}%
\pgfsetbuttcap%
\pgfsetmiterjoin%
\definecolor{currentfill}{rgb}{1.000000,1.000000,1.000000}%
\pgfsetfillcolor{currentfill}%
\pgfsetfillopacity{0.800000}%
\pgfsetlinewidth{1.003750pt}%
\definecolor{currentstroke}{rgb}{0.800000,0.800000,0.800000}%
\pgfsetstrokecolor{currentstroke}%
\pgfsetstrokeopacity{0.800000}%
\pgfsetdash{}{0pt}%
\pgfpathmoveto{\pgfqpoint{4.274210in}{4.331174in}}%
\pgfpathlineto{\pgfqpoint{4.838459in}{4.331174in}}%
\pgfpathquadraticcurveto{\pgfqpoint{4.866237in}{4.331174in}}{\pgfqpoint{4.866237in}{4.358952in}}%
\pgfpathlineto{\pgfqpoint{4.866237in}{4.752778in}}%
\pgfpathquadraticcurveto{\pgfqpoint{4.866237in}{4.780556in}}{\pgfqpoint{4.838459in}{4.780556in}}%
\pgfpathlineto{\pgfqpoint{4.274210in}{4.780556in}}%
\pgfpathquadraticcurveto{\pgfqpoint{4.246432in}{4.780556in}}{\pgfqpoint{4.246432in}{4.752778in}}%
\pgfpathlineto{\pgfqpoint{4.246432in}{4.358952in}}%
\pgfpathquadraticcurveto{\pgfqpoint{4.246432in}{4.331174in}}{\pgfqpoint{4.274210in}{4.331174in}}%
\pgfpathlineto{\pgfqpoint{4.274210in}{4.331174in}}%
\pgfpathclose%
\pgfusepath{stroke,fill}%
\end{pgfscope}%
\begin{pgfscope}%
\definecolor{textcolor}{rgb}{0.000000,0.000000,0.000000}%
\pgfsetstrokecolor{textcolor}%
\pgfsetfillcolor{textcolor}%
\pgftext[x=4.301988in,y=4.619477in,left,base]{\color{textcolor}\sffamily\fontsize{10.000000}{12.000000}\selectfont version}%
\end{pgfscope}%
\begin{pgfscope}%
\pgfsetbuttcap%
\pgfsetroundjoin%
\definecolor{currentfill}{rgb}{0.003922,0.450980,0.698039}%
\pgfsetfillcolor{currentfill}%
\pgfsetfillopacity{0.800000}%
\pgfsetlinewidth{1.003750pt}%
\definecolor{currentstroke}{rgb}{0.003922,0.450980,0.698039}%
\pgfsetstrokecolor{currentstroke}%
\pgfsetstrokeopacity{0.800000}%
\pgfsetdash{}{0pt}%
\pgfsys@defobject{currentmarker}{\pgfqpoint{-0.041667in}{-0.041667in}}{\pgfqpoint{0.041667in}{0.041667in}}{%
\pgfpathmoveto{\pgfqpoint{0.000000in}{-0.041667in}}%
\pgfpathcurveto{\pgfqpoint{0.011050in}{-0.041667in}}{\pgfqpoint{0.021649in}{-0.037276in}}{\pgfqpoint{0.029463in}{-0.029463in}}%
\pgfpathcurveto{\pgfqpoint{0.037276in}{-0.021649in}}{\pgfqpoint{0.041667in}{-0.011050in}}{\pgfqpoint{0.041667in}{0.000000in}}%
\pgfpathcurveto{\pgfqpoint{0.041667in}{0.011050in}}{\pgfqpoint{0.037276in}{0.021649in}}{\pgfqpoint{0.029463in}{0.029463in}}%
\pgfpathcurveto{\pgfqpoint{0.021649in}{0.037276in}}{\pgfqpoint{0.011050in}{0.041667in}}{\pgfqpoint{0.000000in}{0.041667in}}%
\pgfpathcurveto{\pgfqpoint{-0.011050in}{0.041667in}}{\pgfqpoint{-0.021649in}{0.037276in}}{\pgfqpoint{-0.029463in}{0.029463in}}%
\pgfpathcurveto{\pgfqpoint{-0.037276in}{0.021649in}}{\pgfqpoint{-0.041667in}{0.011050in}}{\pgfqpoint{-0.041667in}{0.000000in}}%
\pgfpathcurveto{\pgfqpoint{-0.041667in}{-0.011050in}}{\pgfqpoint{-0.037276in}{-0.021649in}}{\pgfqpoint{-0.029463in}{-0.029463in}}%
\pgfpathcurveto{\pgfqpoint{-0.021649in}{-0.037276in}}{\pgfqpoint{-0.011050in}{-0.041667in}}{\pgfqpoint{0.000000in}{-0.041667in}}%
\pgfpathlineto{\pgfqpoint{0.000000in}{-0.041667in}}%
\pgfpathclose%
\pgfusepath{stroke,fill}%
}%
\begin{pgfscope}%
\pgfsys@transformshift{4.456596in}{4.452078in}%
\pgfsys@useobject{currentmarker}{}%
\end{pgfscope}%
\end{pgfscope}%
\begin{pgfscope}%
\definecolor{textcolor}{rgb}{0.000000,0.000000,0.000000}%
\pgfsetstrokecolor{textcolor}%
\pgfsetfillcolor{textcolor}%
\pgftext[x=4.706596in,y=4.415620in,left,base]{\color{textcolor}\sffamily\fontsize{10.000000}{12.000000}\selectfont 6}%
\end{pgfscope}%
\end{pgfpicture}%
\makeatother%
\endgroup%
}
\caption{Benchmarking of goodput and median latency while varying throughputs and implementation versions.}
\label{goodputlatencablation}
\end{figure}

This study compares the performance of the system with different optimisations enabled. The optimisations explored are chaining (Section~\ref{chaining}), node truncation (Section~\ref{truncation}), and command filtering (Section~\ref{filtering}). Performance is also compared with, and without cryptography enabled. The mapping from version codes to which optimisations are enabled is given in Table~\ref{versiontable}. The experiments displayed in Figure~\ref{throughputgoodputablation} and Figure~\ref{goodputlatencablation} were run for 10s with a network of 4 nodes, and a batch size of 300. Figure~\ref{goodputlatencablation} omits results with latency of above 1s, to show the performance of the system before it becomes overloaded.

The goodput and latency follow similar trends to Section~\ref{batchsizeseval} and Section~\ref{nodecountseval}.

The chained implementation can reach higher goodputs with slightly higher latency than the basic implementation. The higher goodput is a result of pipelining; more requests are processed concurrently. However, pipelining also leads to the size of messages being increased, as each message contains batches of requests from each concurrent phase; this leads to increased latency due to the time taken to serialise larger messages (Section~\ref{capnpbenchmark}).

Filtering significantly increases goodput, and slightly reduces latency. Goodput is increased as there are fewer redundant commands in each batch, so more useful commands can be processed. Latency is slightly reduced as messages are smaller on average due to some requests being filtered, so there is less latency due to serialisation costs.

Truncation dramatically reduces latency which leads to a large increase in goodput. This is because it significantly reduces the size of internal messages, reducing the latency incurred by serialisation costs.

Disabling cryptography reduces latency, leading to increased maximum goodput that can be reached. The reduction in latency is not that significant, implying that serialisation costs are more of a bottleneck than cryptography.

\subsection{Wide area network simulation} \label{minineteval}

\begin{figure}[h!]
\centering
\resizebox{.6\textwidth}{!}{%% Creator: Matplotlib, PGF backend
%%
%% To include the figure in your LaTeX document, write
%%   \input{<filename>.pgf}
%%
%% Make sure the required packages are loaded in your preamble
%%   \usepackage{pgf}
%%
%% Also ensure that all the required font packages are loaded; for instance,
%% the lmodern package is sometimes necessary when using math font.
%%   \usepackage{lmodern}
%%
%% Figures using additional raster images can only be included by \input if
%% they are in the same directory as the main LaTeX file. For loading figures
%% from other directories you can use the `import` package
%%   \usepackage{import}
%%
%% and then include the figures with
%%   \import{<path to file>}{<filename>.pgf}
%%
%% Matplotlib used the following preamble
%%   
%%   \usepackage{fontspec}
%%   \setmainfont{DejaVuSerif.ttf}[Path=\detokenize{/opt/homebrew/lib/python3.10/site-packages/matplotlib/mpl-data/fonts/ttf/}]
%%   \setsansfont{DejaVuSans.ttf}[Path=\detokenize{/opt/homebrew/lib/python3.10/site-packages/matplotlib/mpl-data/fonts/ttf/}]
%%   \setmonofont{DejaVuSansMono.ttf}[Path=\detokenize{/opt/homebrew/lib/python3.10/site-packages/matplotlib/mpl-data/fonts/ttf/}]
%%   \makeatletter\@ifpackageloaded{underscore}{}{\usepackage[strings]{underscore}}\makeatother
%%
\begingroup%
\makeatletter%
\begin{pgfpicture}%
\pgfpathrectangle{\pgfpointorigin}{\pgfqpoint{6.400000in}{4.800000in}}%
\pgfusepath{use as bounding box, clip}%
\begin{pgfscope}%
\pgfsetbuttcap%
\pgfsetmiterjoin%
\definecolor{currentfill}{rgb}{1.000000,1.000000,1.000000}%
\pgfsetfillcolor{currentfill}%
\pgfsetlinewidth{0.000000pt}%
\definecolor{currentstroke}{rgb}{1.000000,1.000000,1.000000}%
\pgfsetstrokecolor{currentstroke}%
\pgfsetdash{}{0pt}%
\pgfpathmoveto{\pgfqpoint{0.000000in}{0.000000in}}%
\pgfpathlineto{\pgfqpoint{6.400000in}{0.000000in}}%
\pgfpathlineto{\pgfqpoint{6.400000in}{4.800000in}}%
\pgfpathlineto{\pgfqpoint{0.000000in}{4.800000in}}%
\pgfpathlineto{\pgfqpoint{0.000000in}{0.000000in}}%
\pgfpathclose%
\pgfusepath{fill}%
\end{pgfscope}%
\begin{pgfscope}%
\pgfsetbuttcap%
\pgfsetmiterjoin%
\definecolor{currentfill}{rgb}{1.000000,1.000000,1.000000}%
\pgfsetfillcolor{currentfill}%
\pgfsetlinewidth{0.000000pt}%
\definecolor{currentstroke}{rgb}{0.000000,0.000000,0.000000}%
\pgfsetstrokecolor{currentstroke}%
\pgfsetstrokeopacity{0.000000}%
\pgfsetdash{}{0pt}%
\pgfpathmoveto{\pgfqpoint{0.800000in}{0.528000in}}%
\pgfpathlineto{\pgfqpoint{5.760000in}{0.528000in}}%
\pgfpathlineto{\pgfqpoint{5.760000in}{4.224000in}}%
\pgfpathlineto{\pgfqpoint{0.800000in}{4.224000in}}%
\pgfpathlineto{\pgfqpoint{0.800000in}{0.528000in}}%
\pgfpathclose%
\pgfusepath{fill}%
\end{pgfscope}%
\begin{pgfscope}%
\pgfpathrectangle{\pgfqpoint{0.800000in}{0.528000in}}{\pgfqpoint{4.960000in}{3.696000in}}%
\pgfusepath{clip}%
\pgfsetbuttcap%
\pgfsetroundjoin%
\definecolor{currentfill}{rgb}{0.121569,0.466667,0.705882}%
\pgfsetfillcolor{currentfill}%
\pgfsetfillopacity{0.200000}%
\pgfsetlinewidth{1.003750pt}%
\definecolor{currentstroke}{rgb}{0.121569,0.466667,0.705882}%
\pgfsetstrokecolor{currentstroke}%
\pgfsetstrokeopacity{0.200000}%
\pgfsetdash{}{0pt}%
\pgfsys@defobject{currentmarker}{\pgfqpoint{1.394012in}{1.259731in}}{\pgfqpoint{5.552096in}{4.082844in}}{%
\pgfpathmoveto{\pgfqpoint{1.394012in}{1.273458in}}%
\pgfpathlineto{\pgfqpoint{1.394012in}{1.259731in}}%
\pgfpathlineto{\pgfqpoint{1.988024in}{2.026259in}}%
\pgfpathlineto{\pgfqpoint{3.176048in}{3.130090in}}%
\pgfpathlineto{\pgfqpoint{5.552096in}{3.195182in}}%
\pgfpathlineto{\pgfqpoint{5.552096in}{4.082844in}}%
\pgfpathlineto{\pgfqpoint{5.552096in}{4.082844in}}%
\pgfpathlineto{\pgfqpoint{3.176048in}{3.485088in}}%
\pgfpathlineto{\pgfqpoint{1.988024in}{2.074031in}}%
\pgfpathlineto{\pgfqpoint{1.394012in}{1.273458in}}%
\pgfpathlineto{\pgfqpoint{1.394012in}{1.273458in}}%
\pgfpathclose%
\pgfusepath{stroke,fill}%
}%
\begin{pgfscope}%
\pgfsys@transformshift{0.000000in}{0.000000in}%
\pgfsys@useobject{currentmarker}{}%
\end{pgfscope}%
\end{pgfscope}%
\begin{pgfscope}%
\pgfsetbuttcap%
\pgfsetroundjoin%
\definecolor{currentfill}{rgb}{0.000000,0.000000,0.000000}%
\pgfsetfillcolor{currentfill}%
\pgfsetlinewidth{0.803000pt}%
\definecolor{currentstroke}{rgb}{0.000000,0.000000,0.000000}%
\pgfsetstrokecolor{currentstroke}%
\pgfsetdash{}{0pt}%
\pgfsys@defobject{currentmarker}{\pgfqpoint{0.000000in}{-0.048611in}}{\pgfqpoint{0.000000in}{0.000000in}}{%
\pgfpathmoveto{\pgfqpoint{0.000000in}{0.000000in}}%
\pgfpathlineto{\pgfqpoint{0.000000in}{-0.048611in}}%
\pgfusepath{stroke,fill}%
}%
\begin{pgfscope}%
\pgfsys@transformshift{0.800000in}{0.528000in}%
\pgfsys@useobject{currentmarker}{}%
\end{pgfscope}%
\end{pgfscope}%
\begin{pgfscope}%
\definecolor{textcolor}{rgb}{0.000000,0.000000,0.000000}%
\pgfsetstrokecolor{textcolor}%
\pgfsetfillcolor{textcolor}%
\pgftext[x=0.800000in,y=0.430778in,,top]{\color{textcolor}\sffamily\fontsize{10.000000}{12.000000}\selectfont 0}%
\end{pgfscope}%
\begin{pgfscope}%
\pgfsetbuttcap%
\pgfsetroundjoin%
\definecolor{currentfill}{rgb}{0.000000,0.000000,0.000000}%
\pgfsetfillcolor{currentfill}%
\pgfsetlinewidth{0.803000pt}%
\definecolor{currentstroke}{rgb}{0.000000,0.000000,0.000000}%
\pgfsetstrokecolor{currentstroke}%
\pgfsetdash{}{0pt}%
\pgfsys@defobject{currentmarker}{\pgfqpoint{0.000000in}{-0.048611in}}{\pgfqpoint{0.000000in}{0.000000in}}{%
\pgfpathmoveto{\pgfqpoint{0.000000in}{0.000000in}}%
\pgfpathlineto{\pgfqpoint{0.000000in}{-0.048611in}}%
\pgfusepath{stroke,fill}%
}%
\begin{pgfscope}%
\pgfsys@transformshift{1.394012in}{0.528000in}%
\pgfsys@useobject{currentmarker}{}%
\end{pgfscope}%
\end{pgfscope}%
\begin{pgfscope}%
\definecolor{textcolor}{rgb}{0.000000,0.000000,0.000000}%
\pgfsetstrokecolor{textcolor}%
\pgfsetfillcolor{textcolor}%
\pgftext[x=1.394012in,y=0.430778in,,top]{\color{textcolor}\sffamily\fontsize{10.000000}{12.000000}\selectfont 50}%
\end{pgfscope}%
\begin{pgfscope}%
\pgfsetbuttcap%
\pgfsetroundjoin%
\definecolor{currentfill}{rgb}{0.000000,0.000000,0.000000}%
\pgfsetfillcolor{currentfill}%
\pgfsetlinewidth{0.803000pt}%
\definecolor{currentstroke}{rgb}{0.000000,0.000000,0.000000}%
\pgfsetstrokecolor{currentstroke}%
\pgfsetdash{}{0pt}%
\pgfsys@defobject{currentmarker}{\pgfqpoint{0.000000in}{-0.048611in}}{\pgfqpoint{0.000000in}{0.000000in}}{%
\pgfpathmoveto{\pgfqpoint{0.000000in}{0.000000in}}%
\pgfpathlineto{\pgfqpoint{0.000000in}{-0.048611in}}%
\pgfusepath{stroke,fill}%
}%
\begin{pgfscope}%
\pgfsys@transformshift{1.988024in}{0.528000in}%
\pgfsys@useobject{currentmarker}{}%
\end{pgfscope}%
\end{pgfscope}%
\begin{pgfscope}%
\definecolor{textcolor}{rgb}{0.000000,0.000000,0.000000}%
\pgfsetstrokecolor{textcolor}%
\pgfsetfillcolor{textcolor}%
\pgftext[x=1.988024in,y=0.430778in,,top]{\color{textcolor}\sffamily\fontsize{10.000000}{12.000000}\selectfont 100}%
\end{pgfscope}%
\begin{pgfscope}%
\pgfsetbuttcap%
\pgfsetroundjoin%
\definecolor{currentfill}{rgb}{0.000000,0.000000,0.000000}%
\pgfsetfillcolor{currentfill}%
\pgfsetlinewidth{0.803000pt}%
\definecolor{currentstroke}{rgb}{0.000000,0.000000,0.000000}%
\pgfsetstrokecolor{currentstroke}%
\pgfsetdash{}{0pt}%
\pgfsys@defobject{currentmarker}{\pgfqpoint{0.000000in}{-0.048611in}}{\pgfqpoint{0.000000in}{0.000000in}}{%
\pgfpathmoveto{\pgfqpoint{0.000000in}{0.000000in}}%
\pgfpathlineto{\pgfqpoint{0.000000in}{-0.048611in}}%
\pgfusepath{stroke,fill}%
}%
\begin{pgfscope}%
\pgfsys@transformshift{2.582036in}{0.528000in}%
\pgfsys@useobject{currentmarker}{}%
\end{pgfscope}%
\end{pgfscope}%
\begin{pgfscope}%
\definecolor{textcolor}{rgb}{0.000000,0.000000,0.000000}%
\pgfsetstrokecolor{textcolor}%
\pgfsetfillcolor{textcolor}%
\pgftext[x=2.582036in,y=0.430778in,,top]{\color{textcolor}\sffamily\fontsize{10.000000}{12.000000}\selectfont 150}%
\end{pgfscope}%
\begin{pgfscope}%
\pgfsetbuttcap%
\pgfsetroundjoin%
\definecolor{currentfill}{rgb}{0.000000,0.000000,0.000000}%
\pgfsetfillcolor{currentfill}%
\pgfsetlinewidth{0.803000pt}%
\definecolor{currentstroke}{rgb}{0.000000,0.000000,0.000000}%
\pgfsetstrokecolor{currentstroke}%
\pgfsetdash{}{0pt}%
\pgfsys@defobject{currentmarker}{\pgfqpoint{0.000000in}{-0.048611in}}{\pgfqpoint{0.000000in}{0.000000in}}{%
\pgfpathmoveto{\pgfqpoint{0.000000in}{0.000000in}}%
\pgfpathlineto{\pgfqpoint{0.000000in}{-0.048611in}}%
\pgfusepath{stroke,fill}%
}%
\begin{pgfscope}%
\pgfsys@transformshift{3.176048in}{0.528000in}%
\pgfsys@useobject{currentmarker}{}%
\end{pgfscope}%
\end{pgfscope}%
\begin{pgfscope}%
\definecolor{textcolor}{rgb}{0.000000,0.000000,0.000000}%
\pgfsetstrokecolor{textcolor}%
\pgfsetfillcolor{textcolor}%
\pgftext[x=3.176048in,y=0.430778in,,top]{\color{textcolor}\sffamily\fontsize{10.000000}{12.000000}\selectfont 200}%
\end{pgfscope}%
\begin{pgfscope}%
\pgfsetbuttcap%
\pgfsetroundjoin%
\definecolor{currentfill}{rgb}{0.000000,0.000000,0.000000}%
\pgfsetfillcolor{currentfill}%
\pgfsetlinewidth{0.803000pt}%
\definecolor{currentstroke}{rgb}{0.000000,0.000000,0.000000}%
\pgfsetstrokecolor{currentstroke}%
\pgfsetdash{}{0pt}%
\pgfsys@defobject{currentmarker}{\pgfqpoint{0.000000in}{-0.048611in}}{\pgfqpoint{0.000000in}{0.000000in}}{%
\pgfpathmoveto{\pgfqpoint{0.000000in}{0.000000in}}%
\pgfpathlineto{\pgfqpoint{0.000000in}{-0.048611in}}%
\pgfusepath{stroke,fill}%
}%
\begin{pgfscope}%
\pgfsys@transformshift{3.770060in}{0.528000in}%
\pgfsys@useobject{currentmarker}{}%
\end{pgfscope}%
\end{pgfscope}%
\begin{pgfscope}%
\definecolor{textcolor}{rgb}{0.000000,0.000000,0.000000}%
\pgfsetstrokecolor{textcolor}%
\pgfsetfillcolor{textcolor}%
\pgftext[x=3.770060in,y=0.430778in,,top]{\color{textcolor}\sffamily\fontsize{10.000000}{12.000000}\selectfont 250}%
\end{pgfscope}%
\begin{pgfscope}%
\pgfsetbuttcap%
\pgfsetroundjoin%
\definecolor{currentfill}{rgb}{0.000000,0.000000,0.000000}%
\pgfsetfillcolor{currentfill}%
\pgfsetlinewidth{0.803000pt}%
\definecolor{currentstroke}{rgb}{0.000000,0.000000,0.000000}%
\pgfsetstrokecolor{currentstroke}%
\pgfsetdash{}{0pt}%
\pgfsys@defobject{currentmarker}{\pgfqpoint{0.000000in}{-0.048611in}}{\pgfqpoint{0.000000in}{0.000000in}}{%
\pgfpathmoveto{\pgfqpoint{0.000000in}{0.000000in}}%
\pgfpathlineto{\pgfqpoint{0.000000in}{-0.048611in}}%
\pgfusepath{stroke,fill}%
}%
\begin{pgfscope}%
\pgfsys@transformshift{4.364072in}{0.528000in}%
\pgfsys@useobject{currentmarker}{}%
\end{pgfscope}%
\end{pgfscope}%
\begin{pgfscope}%
\definecolor{textcolor}{rgb}{0.000000,0.000000,0.000000}%
\pgfsetstrokecolor{textcolor}%
\pgfsetfillcolor{textcolor}%
\pgftext[x=4.364072in,y=0.430778in,,top]{\color{textcolor}\sffamily\fontsize{10.000000}{12.000000}\selectfont 300}%
\end{pgfscope}%
\begin{pgfscope}%
\pgfsetbuttcap%
\pgfsetroundjoin%
\definecolor{currentfill}{rgb}{0.000000,0.000000,0.000000}%
\pgfsetfillcolor{currentfill}%
\pgfsetlinewidth{0.803000pt}%
\definecolor{currentstroke}{rgb}{0.000000,0.000000,0.000000}%
\pgfsetstrokecolor{currentstroke}%
\pgfsetdash{}{0pt}%
\pgfsys@defobject{currentmarker}{\pgfqpoint{0.000000in}{-0.048611in}}{\pgfqpoint{0.000000in}{0.000000in}}{%
\pgfpathmoveto{\pgfqpoint{0.000000in}{0.000000in}}%
\pgfpathlineto{\pgfqpoint{0.000000in}{-0.048611in}}%
\pgfusepath{stroke,fill}%
}%
\begin{pgfscope}%
\pgfsys@transformshift{4.958084in}{0.528000in}%
\pgfsys@useobject{currentmarker}{}%
\end{pgfscope}%
\end{pgfscope}%
\begin{pgfscope}%
\definecolor{textcolor}{rgb}{0.000000,0.000000,0.000000}%
\pgfsetstrokecolor{textcolor}%
\pgfsetfillcolor{textcolor}%
\pgftext[x=4.958084in,y=0.430778in,,top]{\color{textcolor}\sffamily\fontsize{10.000000}{12.000000}\selectfont 350}%
\end{pgfscope}%
\begin{pgfscope}%
\pgfsetbuttcap%
\pgfsetroundjoin%
\definecolor{currentfill}{rgb}{0.000000,0.000000,0.000000}%
\pgfsetfillcolor{currentfill}%
\pgfsetlinewidth{0.803000pt}%
\definecolor{currentstroke}{rgb}{0.000000,0.000000,0.000000}%
\pgfsetstrokecolor{currentstroke}%
\pgfsetdash{}{0pt}%
\pgfsys@defobject{currentmarker}{\pgfqpoint{0.000000in}{-0.048611in}}{\pgfqpoint{0.000000in}{0.000000in}}{%
\pgfpathmoveto{\pgfqpoint{0.000000in}{0.000000in}}%
\pgfpathlineto{\pgfqpoint{0.000000in}{-0.048611in}}%
\pgfusepath{stroke,fill}%
}%
\begin{pgfscope}%
\pgfsys@transformshift{5.552096in}{0.528000in}%
\pgfsys@useobject{currentmarker}{}%
\end{pgfscope}%
\end{pgfscope}%
\begin{pgfscope}%
\definecolor{textcolor}{rgb}{0.000000,0.000000,0.000000}%
\pgfsetstrokecolor{textcolor}%
\pgfsetfillcolor{textcolor}%
\pgftext[x=5.552096in,y=0.430778in,,top]{\color{textcolor}\sffamily\fontsize{10.000000}{12.000000}\selectfont 400}%
\end{pgfscope}%
\begin{pgfscope}%
\definecolor{textcolor}{rgb}{0.000000,0.000000,0.000000}%
\pgfsetstrokecolor{textcolor}%
\pgfsetfillcolor{textcolor}%
\pgftext[x=3.280000in,y=0.240809in,,top]{\color{textcolor}\sffamily\fontsize{10.000000}{12.000000}\selectfont throughput (req/s)}%
\end{pgfscope}%
\begin{pgfscope}%
\pgfsetbuttcap%
\pgfsetroundjoin%
\definecolor{currentfill}{rgb}{0.000000,0.000000,0.000000}%
\pgfsetfillcolor{currentfill}%
\pgfsetlinewidth{0.803000pt}%
\definecolor{currentstroke}{rgb}{0.000000,0.000000,0.000000}%
\pgfsetstrokecolor{currentstroke}%
\pgfsetdash{}{0pt}%
\pgfsys@defobject{currentmarker}{\pgfqpoint{-0.048611in}{0.000000in}}{\pgfqpoint{-0.000000in}{0.000000in}}{%
\pgfpathmoveto{\pgfqpoint{-0.000000in}{0.000000in}}%
\pgfpathlineto{\pgfqpoint{-0.048611in}{0.000000in}}%
\pgfusepath{stroke,fill}%
}%
\begin{pgfscope}%
\pgfsys@transformshift{0.800000in}{0.528000in}%
\pgfsys@useobject{currentmarker}{}%
\end{pgfscope}%
\end{pgfscope}%
\begin{pgfscope}%
\definecolor{textcolor}{rgb}{0.000000,0.000000,0.000000}%
\pgfsetstrokecolor{textcolor}%
\pgfsetfillcolor{textcolor}%
\pgftext[x=0.614412in, y=0.475238in, left, base]{\color{textcolor}\sffamily\fontsize{10.000000}{12.000000}\selectfont 0}%
\end{pgfscope}%
\begin{pgfscope}%
\pgfsetbuttcap%
\pgfsetroundjoin%
\definecolor{currentfill}{rgb}{0.000000,0.000000,0.000000}%
\pgfsetfillcolor{currentfill}%
\pgfsetlinewidth{0.803000pt}%
\definecolor{currentstroke}{rgb}{0.000000,0.000000,0.000000}%
\pgfsetstrokecolor{currentstroke}%
\pgfsetdash{}{0pt}%
\pgfsys@defobject{currentmarker}{\pgfqpoint{-0.048611in}{0.000000in}}{\pgfqpoint{-0.000000in}{0.000000in}}{%
\pgfpathmoveto{\pgfqpoint{-0.000000in}{0.000000in}}%
\pgfpathlineto{\pgfqpoint{-0.048611in}{0.000000in}}%
\pgfusepath{stroke,fill}%
}%
\begin{pgfscope}%
\pgfsys@transformshift{0.800000in}{1.327863in}%
\pgfsys@useobject{currentmarker}{}%
\end{pgfscope}%
\end{pgfscope}%
\begin{pgfscope}%
\definecolor{textcolor}{rgb}{0.000000,0.000000,0.000000}%
\pgfsetstrokecolor{textcolor}%
\pgfsetfillcolor{textcolor}%
\pgftext[x=0.526047in, y=1.275102in, left, base]{\color{textcolor}\sffamily\fontsize{10.000000}{12.000000}\selectfont 50}%
\end{pgfscope}%
\begin{pgfscope}%
\pgfsetbuttcap%
\pgfsetroundjoin%
\definecolor{currentfill}{rgb}{0.000000,0.000000,0.000000}%
\pgfsetfillcolor{currentfill}%
\pgfsetlinewidth{0.803000pt}%
\definecolor{currentstroke}{rgb}{0.000000,0.000000,0.000000}%
\pgfsetstrokecolor{currentstroke}%
\pgfsetdash{}{0pt}%
\pgfsys@defobject{currentmarker}{\pgfqpoint{-0.048611in}{0.000000in}}{\pgfqpoint{-0.000000in}{0.000000in}}{%
\pgfpathmoveto{\pgfqpoint{-0.000000in}{0.000000in}}%
\pgfpathlineto{\pgfqpoint{-0.048611in}{0.000000in}}%
\pgfusepath{stroke,fill}%
}%
\begin{pgfscope}%
\pgfsys@transformshift{0.800000in}{2.127727in}%
\pgfsys@useobject{currentmarker}{}%
\end{pgfscope}%
\end{pgfscope}%
\begin{pgfscope}%
\definecolor{textcolor}{rgb}{0.000000,0.000000,0.000000}%
\pgfsetstrokecolor{textcolor}%
\pgfsetfillcolor{textcolor}%
\pgftext[x=0.437682in, y=2.074965in, left, base]{\color{textcolor}\sffamily\fontsize{10.000000}{12.000000}\selectfont 100}%
\end{pgfscope}%
\begin{pgfscope}%
\pgfsetbuttcap%
\pgfsetroundjoin%
\definecolor{currentfill}{rgb}{0.000000,0.000000,0.000000}%
\pgfsetfillcolor{currentfill}%
\pgfsetlinewidth{0.803000pt}%
\definecolor{currentstroke}{rgb}{0.000000,0.000000,0.000000}%
\pgfsetstrokecolor{currentstroke}%
\pgfsetdash{}{0pt}%
\pgfsys@defobject{currentmarker}{\pgfqpoint{-0.048611in}{0.000000in}}{\pgfqpoint{-0.000000in}{0.000000in}}{%
\pgfpathmoveto{\pgfqpoint{-0.000000in}{0.000000in}}%
\pgfpathlineto{\pgfqpoint{-0.048611in}{0.000000in}}%
\pgfusepath{stroke,fill}%
}%
\begin{pgfscope}%
\pgfsys@transformshift{0.800000in}{2.927590in}%
\pgfsys@useobject{currentmarker}{}%
\end{pgfscope}%
\end{pgfscope}%
\begin{pgfscope}%
\definecolor{textcolor}{rgb}{0.000000,0.000000,0.000000}%
\pgfsetstrokecolor{textcolor}%
\pgfsetfillcolor{textcolor}%
\pgftext[x=0.437682in, y=2.874828in, left, base]{\color{textcolor}\sffamily\fontsize{10.000000}{12.000000}\selectfont 150}%
\end{pgfscope}%
\begin{pgfscope}%
\pgfsetbuttcap%
\pgfsetroundjoin%
\definecolor{currentfill}{rgb}{0.000000,0.000000,0.000000}%
\pgfsetfillcolor{currentfill}%
\pgfsetlinewidth{0.803000pt}%
\definecolor{currentstroke}{rgb}{0.000000,0.000000,0.000000}%
\pgfsetstrokecolor{currentstroke}%
\pgfsetdash{}{0pt}%
\pgfsys@defobject{currentmarker}{\pgfqpoint{-0.048611in}{0.000000in}}{\pgfqpoint{-0.000000in}{0.000000in}}{%
\pgfpathmoveto{\pgfqpoint{-0.000000in}{0.000000in}}%
\pgfpathlineto{\pgfqpoint{-0.048611in}{0.000000in}}%
\pgfusepath{stroke,fill}%
}%
\begin{pgfscope}%
\pgfsys@transformshift{0.800000in}{3.727453in}%
\pgfsys@useobject{currentmarker}{}%
\end{pgfscope}%
\end{pgfscope}%
\begin{pgfscope}%
\definecolor{textcolor}{rgb}{0.000000,0.000000,0.000000}%
\pgfsetstrokecolor{textcolor}%
\pgfsetfillcolor{textcolor}%
\pgftext[x=0.437682in, y=3.674692in, left, base]{\color{textcolor}\sffamily\fontsize{10.000000}{12.000000}\selectfont 200}%
\end{pgfscope}%
\begin{pgfscope}%
\definecolor{textcolor}{rgb}{0.000000,0.000000,0.000000}%
\pgfsetstrokecolor{textcolor}%
\pgfsetfillcolor{textcolor}%
\pgftext[x=0.382126in,y=2.376000in,,bottom,rotate=90.000000]{\color{textcolor}\sffamily\fontsize{10.000000}{12.000000}\selectfont goodput (req/s)}%
\end{pgfscope}%
\begin{pgfscope}%
\pgfpathrectangle{\pgfqpoint{0.800000in}{0.528000in}}{\pgfqpoint{4.960000in}{3.696000in}}%
\pgfusepath{clip}%
\pgfsetbuttcap%
\pgfsetroundjoin%
\pgfsetlinewidth{1.505625pt}%
\definecolor{currentstroke}{rgb}{0.121569,0.466667,0.705882}%
\pgfsetstrokecolor{currentstroke}%
\pgfsetdash{{5.550000pt}{2.400000pt}}{0.000000pt}%
\pgfpathmoveto{\pgfqpoint{1.394012in}{1.266688in}}%
\pgfpathlineto{\pgfqpoint{1.988024in}{2.049992in}}%
\pgfpathlineto{\pgfqpoint{3.176048in}{3.366376in}}%
\pgfpathlineto{\pgfqpoint{5.552096in}{3.634990in}}%
\pgfusepath{stroke}%
\end{pgfscope}%
\begin{pgfscope}%
\pgfsetrectcap%
\pgfsetmiterjoin%
\pgfsetlinewidth{0.803000pt}%
\definecolor{currentstroke}{rgb}{0.000000,0.000000,0.000000}%
\pgfsetstrokecolor{currentstroke}%
\pgfsetdash{}{0pt}%
\pgfpathmoveto{\pgfqpoint{0.800000in}{0.528000in}}%
\pgfpathlineto{\pgfqpoint{0.800000in}{4.224000in}}%
\pgfusepath{stroke}%
\end{pgfscope}%
\begin{pgfscope}%
\pgfsetrectcap%
\pgfsetmiterjoin%
\pgfsetlinewidth{0.803000pt}%
\definecolor{currentstroke}{rgb}{0.000000,0.000000,0.000000}%
\pgfsetstrokecolor{currentstroke}%
\pgfsetdash{}{0pt}%
\pgfpathmoveto{\pgfqpoint{5.760000in}{0.528000in}}%
\pgfpathlineto{\pgfqpoint{5.760000in}{4.224000in}}%
\pgfusepath{stroke}%
\end{pgfscope}%
\begin{pgfscope}%
\pgfsetrectcap%
\pgfsetmiterjoin%
\pgfsetlinewidth{0.803000pt}%
\definecolor{currentstroke}{rgb}{0.000000,0.000000,0.000000}%
\pgfsetstrokecolor{currentstroke}%
\pgfsetdash{}{0pt}%
\pgfpathmoveto{\pgfqpoint{0.800000in}{0.528000in}}%
\pgfpathlineto{\pgfqpoint{5.760000in}{0.528000in}}%
\pgfusepath{stroke}%
\end{pgfscope}%
\begin{pgfscope}%
\pgfsetrectcap%
\pgfsetmiterjoin%
\pgfsetlinewidth{0.803000pt}%
\definecolor{currentstroke}{rgb}{0.000000,0.000000,0.000000}%
\pgfsetstrokecolor{currentstroke}%
\pgfsetdash{}{0pt}%
\pgfpathmoveto{\pgfqpoint{0.800000in}{4.224000in}}%
\pgfpathlineto{\pgfqpoint{5.760000in}{4.224000in}}%
\pgfusepath{stroke}%
\end{pgfscope}%
\end{pgfpicture}%
\makeatother%
\endgroup%
}
\caption{Benchmarking of goodput for varying throughputs.}
\label{throughoutgoodputmininet}
\end{figure}

\begin{figure}[h!]
\centering
\resizebox{.5\textwidth}{!}{%% Creator: Matplotlib, PGF backend
%%
%% To include the figure in your LaTeX document, write
%%   \input{<filename>.pgf}
%%
%% Make sure the required packages are loaded in your preamble
%%   \usepackage{pgf}
%%
%% Also ensure that all the required font packages are loaded; for instance,
%% the lmodern package is sometimes necessary when using math font.
%%   \usepackage{lmodern}
%%
%% Figures using additional raster images can only be included by \input if
%% they are in the same directory as the main LaTeX file. For loading figures
%% from other directories you can use the `import` package
%%   \usepackage{import}
%%
%% and then include the figures with
%%   \import{<path to file>}{<filename>.pgf}
%%
%% Matplotlib used the following preamble
%%   
%%   \usepackage{fontspec}
%%   \setmainfont{DejaVuSerif.ttf}[Path=\detokenize{/opt/homebrew/lib/python3.10/site-packages/matplotlib/mpl-data/fonts/ttf/}]
%%   \setsansfont{DejaVuSans.ttf}[Path=\detokenize{/opt/homebrew/lib/python3.10/site-packages/matplotlib/mpl-data/fonts/ttf/}]
%%   \setmonofont{DejaVuSansMono.ttf}[Path=\detokenize{/opt/homebrew/lib/python3.10/site-packages/matplotlib/mpl-data/fonts/ttf/}]
%%   \makeatletter\@ifpackageloaded{underscore}{}{\usepackage[strings]{underscore}}\makeatother
%%
\begingroup%
\makeatletter%
\begin{pgfpicture}%
\pgfpathrectangle{\pgfpointorigin}{\pgfqpoint{5.000000in}{5.000000in}}%
\pgfusepath{use as bounding box, clip}%
\begin{pgfscope}%
\pgfsetbuttcap%
\pgfsetmiterjoin%
\definecolor{currentfill}{rgb}{1.000000,1.000000,1.000000}%
\pgfsetfillcolor{currentfill}%
\pgfsetlinewidth{0.000000pt}%
\definecolor{currentstroke}{rgb}{1.000000,1.000000,1.000000}%
\pgfsetstrokecolor{currentstroke}%
\pgfsetdash{}{0pt}%
\pgfpathmoveto{\pgfqpoint{0.000000in}{0.000000in}}%
\pgfpathlineto{\pgfqpoint{5.000000in}{0.000000in}}%
\pgfpathlineto{\pgfqpoint{5.000000in}{5.000000in}}%
\pgfpathlineto{\pgfqpoint{0.000000in}{5.000000in}}%
\pgfpathlineto{\pgfqpoint{0.000000in}{0.000000in}}%
\pgfpathclose%
\pgfusepath{fill}%
\end{pgfscope}%
\begin{pgfscope}%
\pgfsetbuttcap%
\pgfsetmiterjoin%
\definecolor{currentfill}{rgb}{1.000000,1.000000,1.000000}%
\pgfsetfillcolor{currentfill}%
\pgfsetlinewidth{0.000000pt}%
\definecolor{currentstroke}{rgb}{0.000000,0.000000,0.000000}%
\pgfsetstrokecolor{currentstroke}%
\pgfsetstrokeopacity{0.000000}%
\pgfsetdash{}{0pt}%
\pgfpathmoveto{\pgfqpoint{0.669653in}{0.582778in}}%
\pgfpathlineto{\pgfqpoint{4.850000in}{0.582778in}}%
\pgfpathlineto{\pgfqpoint{4.850000in}{4.850000in}}%
\pgfpathlineto{\pgfqpoint{0.669653in}{4.850000in}}%
\pgfpathlineto{\pgfqpoint{0.669653in}{0.582778in}}%
\pgfpathclose%
\pgfusepath{fill}%
\end{pgfscope}%
\begin{pgfscope}%
\pgfpathrectangle{\pgfqpoint{0.669653in}{0.582778in}}{\pgfqpoint{4.180347in}{4.267222in}}%
\pgfusepath{clip}%
\pgfsetbuttcap%
\pgfsetroundjoin%
\definecolor{currentfill}{rgb}{0.121569,0.466667,0.705882}%
\pgfsetfillcolor{currentfill}%
\pgfsetfillopacity{0.800000}%
\pgfsetlinewidth{1.003750pt}%
\definecolor{currentstroke}{rgb}{0.121569,0.466667,0.705882}%
\pgfsetstrokecolor{currentstroke}%
\pgfsetstrokeopacity{0.800000}%
\pgfsetdash{}{0pt}%
\pgfsys@defobject{currentmarker}{\pgfqpoint{-0.041667in}{-0.041667in}}{\pgfqpoint{0.041667in}{0.041667in}}{%
\pgfpathmoveto{\pgfqpoint{0.000000in}{-0.041667in}}%
\pgfpathcurveto{\pgfqpoint{0.011050in}{-0.041667in}}{\pgfqpoint{0.021649in}{-0.037276in}}{\pgfqpoint{0.029463in}{-0.029463in}}%
\pgfpathcurveto{\pgfqpoint{0.037276in}{-0.021649in}}{\pgfqpoint{0.041667in}{-0.011050in}}{\pgfqpoint{0.041667in}{0.000000in}}%
\pgfpathcurveto{\pgfqpoint{0.041667in}{0.011050in}}{\pgfqpoint{0.037276in}{0.021649in}}{\pgfqpoint{0.029463in}{0.029463in}}%
\pgfpathcurveto{\pgfqpoint{0.021649in}{0.037276in}}{\pgfqpoint{0.011050in}{0.041667in}}{\pgfqpoint{0.000000in}{0.041667in}}%
\pgfpathcurveto{\pgfqpoint{-0.011050in}{0.041667in}}{\pgfqpoint{-0.021649in}{0.037276in}}{\pgfqpoint{-0.029463in}{0.029463in}}%
\pgfpathcurveto{\pgfqpoint{-0.037276in}{0.021649in}}{\pgfqpoint{-0.041667in}{0.011050in}}{\pgfqpoint{-0.041667in}{0.000000in}}%
\pgfpathcurveto{\pgfqpoint{-0.041667in}{-0.011050in}}{\pgfqpoint{-0.037276in}{-0.021649in}}{\pgfqpoint{-0.029463in}{-0.029463in}}%
\pgfpathcurveto{\pgfqpoint{-0.021649in}{-0.037276in}}{\pgfqpoint{-0.011050in}{-0.041667in}}{\pgfqpoint{0.000000in}{-0.041667in}}%
\pgfpathlineto{\pgfqpoint{0.000000in}{-0.041667in}}%
\pgfpathclose%
\pgfusepath{stroke,fill}%
}%
\begin{pgfscope}%
\pgfsys@transformshift{0.688291in}{1.904547in}%
\pgfsys@useobject{currentmarker}{}%
\end{pgfscope}%
\begin{pgfscope}%
\pgfsys@transformshift{0.688291in}{1.871902in}%
\pgfsys@useobject{currentmarker}{}%
\end{pgfscope}%
\begin{pgfscope}%
\pgfsys@transformshift{0.688291in}{2.001423in}%
\pgfsys@useobject{currentmarker}{}%
\end{pgfscope}%
\begin{pgfscope}%
\pgfsys@transformshift{1.123983in}{2.082616in}%
\pgfsys@useobject{currentmarker}{}%
\end{pgfscope}%
\begin{pgfscope}%
\pgfsys@transformshift{1.135614in}{2.074085in}%
\pgfsys@useobject{currentmarker}{}%
\end{pgfscope}%
\begin{pgfscope}%
\pgfsys@transformshift{1.135614in}{2.025100in}%
\pgfsys@useobject{currentmarker}{}%
\end{pgfscope}%
\begin{pgfscope}%
\pgfsys@transformshift{1.598673in}{2.045214in}%
\pgfsys@useobject{currentmarker}{}%
\end{pgfscope}%
\begin{pgfscope}%
\pgfsys@transformshift{1.601576in}{2.107786in}%
\pgfsys@useobject{currentmarker}{}%
\end{pgfscope}%
\begin{pgfscope}%
\pgfsys@transformshift{1.592033in}{2.045769in}%
\pgfsys@useobject{currentmarker}{}%
\end{pgfscope}%
\begin{pgfscope}%
\pgfsys@transformshift{2.533499in}{2.194150in}%
\pgfsys@useobject{currentmarker}{}%
\end{pgfscope}%
\begin{pgfscope}%
\pgfsys@transformshift{2.485894in}{2.085208in}%
\pgfsys@useobject{currentmarker}{}%
\end{pgfscope}%
\begin{pgfscope}%
\pgfsys@transformshift{2.488778in}{2.108946in}%
\pgfsys@useobject{currentmarker}{}%
\end{pgfscope}%
\begin{pgfscope}%
\pgfsys@transformshift{4.397344in}{2.404816in}%
\pgfsys@useobject{currentmarker}{}%
\end{pgfscope}%
\begin{pgfscope}%
\pgfsys@transformshift{4.180594in}{2.370090in}%
\pgfsys@useobject{currentmarker}{}%
\end{pgfscope}%
\begin{pgfscope}%
\pgfsys@transformshift{4.397344in}{2.250281in}%
\pgfsys@useobject{currentmarker}{}%
\end{pgfscope}%
\begin{pgfscope}%
\pgfsys@transformshift{4.640621in}{4.251742in}%
\pgfsys@useobject{currentmarker}{}%
\end{pgfscope}%
\begin{pgfscope}%
\pgfsys@transformshift{4.651823in}{4.441047in}%
\pgfsys@useobject{currentmarker}{}%
\end{pgfscope}%
\begin{pgfscope}%
\pgfsys@transformshift{4.398625in}{4.569578in}%
\pgfsys@useobject{currentmarker}{}%
\end{pgfscope}%
\end{pgfscope}%
\begin{pgfscope}%
\pgfsetbuttcap%
\pgfsetroundjoin%
\definecolor{currentfill}{rgb}{0.000000,0.000000,0.000000}%
\pgfsetfillcolor{currentfill}%
\pgfsetlinewidth{0.803000pt}%
\definecolor{currentstroke}{rgb}{0.000000,0.000000,0.000000}%
\pgfsetstrokecolor{currentstroke}%
\pgfsetdash{}{0pt}%
\pgfsys@defobject{currentmarker}{\pgfqpoint{0.000000in}{-0.048611in}}{\pgfqpoint{0.000000in}{0.000000in}}{%
\pgfpathmoveto{\pgfqpoint{0.000000in}{0.000000in}}%
\pgfpathlineto{\pgfqpoint{0.000000in}{-0.048611in}}%
\pgfusepath{stroke,fill}%
}%
\begin{pgfscope}%
\pgfsys@transformshift{0.669653in}{0.582778in}%
\pgfsys@useobject{currentmarker}{}%
\end{pgfscope}%
\end{pgfscope}%
\begin{pgfscope}%
\definecolor{textcolor}{rgb}{0.000000,0.000000,0.000000}%
\pgfsetstrokecolor{textcolor}%
\pgfsetfillcolor{textcolor}%
\pgftext[x=0.669653in,y=0.485556in,,top]{\color{textcolor}\sffamily\fontsize{10.000000}{12.000000}\selectfont 0}%
\end{pgfscope}%
\begin{pgfscope}%
\pgfsetbuttcap%
\pgfsetroundjoin%
\definecolor{currentfill}{rgb}{0.000000,0.000000,0.000000}%
\pgfsetfillcolor{currentfill}%
\pgfsetlinewidth{0.803000pt}%
\definecolor{currentstroke}{rgb}{0.000000,0.000000,0.000000}%
\pgfsetstrokecolor{currentstroke}%
\pgfsetdash{}{0pt}%
\pgfsys@defobject{currentmarker}{\pgfqpoint{0.000000in}{-0.048611in}}{\pgfqpoint{0.000000in}{0.000000in}}{%
\pgfpathmoveto{\pgfqpoint{0.000000in}{0.000000in}}%
\pgfpathlineto{\pgfqpoint{0.000000in}{-0.048611in}}%
\pgfusepath{stroke,fill}%
}%
\begin{pgfscope}%
\pgfsys@transformshift{1.135614in}{0.582778in}%
\pgfsys@useobject{currentmarker}{}%
\end{pgfscope}%
\end{pgfscope}%
\begin{pgfscope}%
\definecolor{textcolor}{rgb}{0.000000,0.000000,0.000000}%
\pgfsetstrokecolor{textcolor}%
\pgfsetfillcolor{textcolor}%
\pgftext[x=1.135614in,y=0.485556in,,top]{\color{textcolor}\sffamily\fontsize{10.000000}{12.000000}\selectfont 25}%
\end{pgfscope}%
\begin{pgfscope}%
\pgfsetbuttcap%
\pgfsetroundjoin%
\definecolor{currentfill}{rgb}{0.000000,0.000000,0.000000}%
\pgfsetfillcolor{currentfill}%
\pgfsetlinewidth{0.803000pt}%
\definecolor{currentstroke}{rgb}{0.000000,0.000000,0.000000}%
\pgfsetstrokecolor{currentstroke}%
\pgfsetdash{}{0pt}%
\pgfsys@defobject{currentmarker}{\pgfqpoint{0.000000in}{-0.048611in}}{\pgfqpoint{0.000000in}{0.000000in}}{%
\pgfpathmoveto{\pgfqpoint{0.000000in}{0.000000in}}%
\pgfpathlineto{\pgfqpoint{0.000000in}{-0.048611in}}%
\pgfusepath{stroke,fill}%
}%
\begin{pgfscope}%
\pgfsys@transformshift{1.601576in}{0.582778in}%
\pgfsys@useobject{currentmarker}{}%
\end{pgfscope}%
\end{pgfscope}%
\begin{pgfscope}%
\definecolor{textcolor}{rgb}{0.000000,0.000000,0.000000}%
\pgfsetstrokecolor{textcolor}%
\pgfsetfillcolor{textcolor}%
\pgftext[x=1.601576in,y=0.485556in,,top]{\color{textcolor}\sffamily\fontsize{10.000000}{12.000000}\selectfont 50}%
\end{pgfscope}%
\begin{pgfscope}%
\pgfsetbuttcap%
\pgfsetroundjoin%
\definecolor{currentfill}{rgb}{0.000000,0.000000,0.000000}%
\pgfsetfillcolor{currentfill}%
\pgfsetlinewidth{0.803000pt}%
\definecolor{currentstroke}{rgb}{0.000000,0.000000,0.000000}%
\pgfsetstrokecolor{currentstroke}%
\pgfsetdash{}{0pt}%
\pgfsys@defobject{currentmarker}{\pgfqpoint{0.000000in}{-0.048611in}}{\pgfqpoint{0.000000in}{0.000000in}}{%
\pgfpathmoveto{\pgfqpoint{0.000000in}{0.000000in}}%
\pgfpathlineto{\pgfqpoint{0.000000in}{-0.048611in}}%
\pgfusepath{stroke,fill}%
}%
\begin{pgfscope}%
\pgfsys@transformshift{2.067537in}{0.582778in}%
\pgfsys@useobject{currentmarker}{}%
\end{pgfscope}%
\end{pgfscope}%
\begin{pgfscope}%
\definecolor{textcolor}{rgb}{0.000000,0.000000,0.000000}%
\pgfsetstrokecolor{textcolor}%
\pgfsetfillcolor{textcolor}%
\pgftext[x=2.067537in,y=0.485556in,,top]{\color{textcolor}\sffamily\fontsize{10.000000}{12.000000}\selectfont 75}%
\end{pgfscope}%
\begin{pgfscope}%
\pgfsetbuttcap%
\pgfsetroundjoin%
\definecolor{currentfill}{rgb}{0.000000,0.000000,0.000000}%
\pgfsetfillcolor{currentfill}%
\pgfsetlinewidth{0.803000pt}%
\definecolor{currentstroke}{rgb}{0.000000,0.000000,0.000000}%
\pgfsetstrokecolor{currentstroke}%
\pgfsetdash{}{0pt}%
\pgfsys@defobject{currentmarker}{\pgfqpoint{0.000000in}{-0.048611in}}{\pgfqpoint{0.000000in}{0.000000in}}{%
\pgfpathmoveto{\pgfqpoint{0.000000in}{0.000000in}}%
\pgfpathlineto{\pgfqpoint{0.000000in}{-0.048611in}}%
\pgfusepath{stroke,fill}%
}%
\begin{pgfscope}%
\pgfsys@transformshift{2.533499in}{0.582778in}%
\pgfsys@useobject{currentmarker}{}%
\end{pgfscope}%
\end{pgfscope}%
\begin{pgfscope}%
\definecolor{textcolor}{rgb}{0.000000,0.000000,0.000000}%
\pgfsetstrokecolor{textcolor}%
\pgfsetfillcolor{textcolor}%
\pgftext[x=2.533499in,y=0.485556in,,top]{\color{textcolor}\sffamily\fontsize{10.000000}{12.000000}\selectfont 100}%
\end{pgfscope}%
\begin{pgfscope}%
\pgfsetbuttcap%
\pgfsetroundjoin%
\definecolor{currentfill}{rgb}{0.000000,0.000000,0.000000}%
\pgfsetfillcolor{currentfill}%
\pgfsetlinewidth{0.803000pt}%
\definecolor{currentstroke}{rgb}{0.000000,0.000000,0.000000}%
\pgfsetstrokecolor{currentstroke}%
\pgfsetdash{}{0pt}%
\pgfsys@defobject{currentmarker}{\pgfqpoint{0.000000in}{-0.048611in}}{\pgfqpoint{0.000000in}{0.000000in}}{%
\pgfpathmoveto{\pgfqpoint{0.000000in}{0.000000in}}%
\pgfpathlineto{\pgfqpoint{0.000000in}{-0.048611in}}%
\pgfusepath{stroke,fill}%
}%
\begin{pgfscope}%
\pgfsys@transformshift{2.999460in}{0.582778in}%
\pgfsys@useobject{currentmarker}{}%
\end{pgfscope}%
\end{pgfscope}%
\begin{pgfscope}%
\definecolor{textcolor}{rgb}{0.000000,0.000000,0.000000}%
\pgfsetstrokecolor{textcolor}%
\pgfsetfillcolor{textcolor}%
\pgftext[x=2.999460in,y=0.485556in,,top]{\color{textcolor}\sffamily\fontsize{10.000000}{12.000000}\selectfont 125}%
\end{pgfscope}%
\begin{pgfscope}%
\pgfsetbuttcap%
\pgfsetroundjoin%
\definecolor{currentfill}{rgb}{0.000000,0.000000,0.000000}%
\pgfsetfillcolor{currentfill}%
\pgfsetlinewidth{0.803000pt}%
\definecolor{currentstroke}{rgb}{0.000000,0.000000,0.000000}%
\pgfsetstrokecolor{currentstroke}%
\pgfsetdash{}{0pt}%
\pgfsys@defobject{currentmarker}{\pgfqpoint{0.000000in}{-0.048611in}}{\pgfqpoint{0.000000in}{0.000000in}}{%
\pgfpathmoveto{\pgfqpoint{0.000000in}{0.000000in}}%
\pgfpathlineto{\pgfqpoint{0.000000in}{-0.048611in}}%
\pgfusepath{stroke,fill}%
}%
\begin{pgfscope}%
\pgfsys@transformshift{3.465421in}{0.582778in}%
\pgfsys@useobject{currentmarker}{}%
\end{pgfscope}%
\end{pgfscope}%
\begin{pgfscope}%
\definecolor{textcolor}{rgb}{0.000000,0.000000,0.000000}%
\pgfsetstrokecolor{textcolor}%
\pgfsetfillcolor{textcolor}%
\pgftext[x=3.465421in,y=0.485556in,,top]{\color{textcolor}\sffamily\fontsize{10.000000}{12.000000}\selectfont 150}%
\end{pgfscope}%
\begin{pgfscope}%
\pgfsetbuttcap%
\pgfsetroundjoin%
\definecolor{currentfill}{rgb}{0.000000,0.000000,0.000000}%
\pgfsetfillcolor{currentfill}%
\pgfsetlinewidth{0.803000pt}%
\definecolor{currentstroke}{rgb}{0.000000,0.000000,0.000000}%
\pgfsetstrokecolor{currentstroke}%
\pgfsetdash{}{0pt}%
\pgfsys@defobject{currentmarker}{\pgfqpoint{0.000000in}{-0.048611in}}{\pgfqpoint{0.000000in}{0.000000in}}{%
\pgfpathmoveto{\pgfqpoint{0.000000in}{0.000000in}}%
\pgfpathlineto{\pgfqpoint{0.000000in}{-0.048611in}}%
\pgfusepath{stroke,fill}%
}%
\begin{pgfscope}%
\pgfsys@transformshift{3.931383in}{0.582778in}%
\pgfsys@useobject{currentmarker}{}%
\end{pgfscope}%
\end{pgfscope}%
\begin{pgfscope}%
\definecolor{textcolor}{rgb}{0.000000,0.000000,0.000000}%
\pgfsetstrokecolor{textcolor}%
\pgfsetfillcolor{textcolor}%
\pgftext[x=3.931383in,y=0.485556in,,top]{\color{textcolor}\sffamily\fontsize{10.000000}{12.000000}\selectfont 175}%
\end{pgfscope}%
\begin{pgfscope}%
\pgfsetbuttcap%
\pgfsetroundjoin%
\definecolor{currentfill}{rgb}{0.000000,0.000000,0.000000}%
\pgfsetfillcolor{currentfill}%
\pgfsetlinewidth{0.803000pt}%
\definecolor{currentstroke}{rgb}{0.000000,0.000000,0.000000}%
\pgfsetstrokecolor{currentstroke}%
\pgfsetdash{}{0pt}%
\pgfsys@defobject{currentmarker}{\pgfqpoint{0.000000in}{-0.048611in}}{\pgfqpoint{0.000000in}{0.000000in}}{%
\pgfpathmoveto{\pgfqpoint{0.000000in}{0.000000in}}%
\pgfpathlineto{\pgfqpoint{0.000000in}{-0.048611in}}%
\pgfusepath{stroke,fill}%
}%
\begin{pgfscope}%
\pgfsys@transformshift{4.397344in}{0.582778in}%
\pgfsys@useobject{currentmarker}{}%
\end{pgfscope}%
\end{pgfscope}%
\begin{pgfscope}%
\definecolor{textcolor}{rgb}{0.000000,0.000000,0.000000}%
\pgfsetstrokecolor{textcolor}%
\pgfsetfillcolor{textcolor}%
\pgftext[x=4.397344in,y=0.485556in,,top]{\color{textcolor}\sffamily\fontsize{10.000000}{12.000000}\selectfont 200}%
\end{pgfscope}%
\begin{pgfscope}%
\definecolor{textcolor}{rgb}{0.000000,0.000000,0.000000}%
\pgfsetstrokecolor{textcolor}%
\pgfsetfillcolor{textcolor}%
\pgftext[x=2.759826in,y=0.295587in,,top]{\color{textcolor}\sffamily\fontsize{10.000000}{12.000000}\selectfont goodput (req/s)}%
\end{pgfscope}%
\begin{pgfscope}%
\pgfsetbuttcap%
\pgfsetroundjoin%
\definecolor{currentfill}{rgb}{0.000000,0.000000,0.000000}%
\pgfsetfillcolor{currentfill}%
\pgfsetlinewidth{0.803000pt}%
\definecolor{currentstroke}{rgb}{0.000000,0.000000,0.000000}%
\pgfsetstrokecolor{currentstroke}%
\pgfsetdash{}{0pt}%
\pgfsys@defobject{currentmarker}{\pgfqpoint{-0.048611in}{0.000000in}}{\pgfqpoint{-0.000000in}{0.000000in}}{%
\pgfpathmoveto{\pgfqpoint{-0.000000in}{0.000000in}}%
\pgfpathlineto{\pgfqpoint{-0.048611in}{0.000000in}}%
\pgfusepath{stroke,fill}%
}%
\begin{pgfscope}%
\pgfsys@transformshift{0.669653in}{0.582778in}%
\pgfsys@useobject{currentmarker}{}%
\end{pgfscope}%
\end{pgfscope}%
\begin{pgfscope}%
\definecolor{textcolor}{rgb}{0.000000,0.000000,0.000000}%
\pgfsetstrokecolor{textcolor}%
\pgfsetfillcolor{textcolor}%
\pgftext[x=0.351551in, y=0.530016in, left, base]{\color{textcolor}\sffamily\fontsize{10.000000}{12.000000}\selectfont 0.0}%
\end{pgfscope}%
\begin{pgfscope}%
\pgfsetbuttcap%
\pgfsetroundjoin%
\definecolor{currentfill}{rgb}{0.000000,0.000000,0.000000}%
\pgfsetfillcolor{currentfill}%
\pgfsetlinewidth{0.803000pt}%
\definecolor{currentstroke}{rgb}{0.000000,0.000000,0.000000}%
\pgfsetstrokecolor{currentstroke}%
\pgfsetdash{}{0pt}%
\pgfsys@defobject{currentmarker}{\pgfqpoint{-0.048611in}{0.000000in}}{\pgfqpoint{-0.000000in}{0.000000in}}{%
\pgfpathmoveto{\pgfqpoint{-0.000000in}{0.000000in}}%
\pgfpathlineto{\pgfqpoint{-0.048611in}{0.000000in}}%
\pgfusepath{stroke,fill}%
}%
\begin{pgfscope}%
\pgfsys@transformshift{0.669653in}{1.116181in}%
\pgfsys@useobject{currentmarker}{}%
\end{pgfscope}%
\end{pgfscope}%
\begin{pgfscope}%
\definecolor{textcolor}{rgb}{0.000000,0.000000,0.000000}%
\pgfsetstrokecolor{textcolor}%
\pgfsetfillcolor{textcolor}%
\pgftext[x=0.351551in, y=1.063419in, left, base]{\color{textcolor}\sffamily\fontsize{10.000000}{12.000000}\selectfont 0.5}%
\end{pgfscope}%
\begin{pgfscope}%
\pgfsetbuttcap%
\pgfsetroundjoin%
\definecolor{currentfill}{rgb}{0.000000,0.000000,0.000000}%
\pgfsetfillcolor{currentfill}%
\pgfsetlinewidth{0.803000pt}%
\definecolor{currentstroke}{rgb}{0.000000,0.000000,0.000000}%
\pgfsetstrokecolor{currentstroke}%
\pgfsetdash{}{0pt}%
\pgfsys@defobject{currentmarker}{\pgfqpoint{-0.048611in}{0.000000in}}{\pgfqpoint{-0.000000in}{0.000000in}}{%
\pgfpathmoveto{\pgfqpoint{-0.000000in}{0.000000in}}%
\pgfpathlineto{\pgfqpoint{-0.048611in}{0.000000in}}%
\pgfusepath{stroke,fill}%
}%
\begin{pgfscope}%
\pgfsys@transformshift{0.669653in}{1.649583in}%
\pgfsys@useobject{currentmarker}{}%
\end{pgfscope}%
\end{pgfscope}%
\begin{pgfscope}%
\definecolor{textcolor}{rgb}{0.000000,0.000000,0.000000}%
\pgfsetstrokecolor{textcolor}%
\pgfsetfillcolor{textcolor}%
\pgftext[x=0.351551in, y=1.596822in, left, base]{\color{textcolor}\sffamily\fontsize{10.000000}{12.000000}\selectfont 1.0}%
\end{pgfscope}%
\begin{pgfscope}%
\pgfsetbuttcap%
\pgfsetroundjoin%
\definecolor{currentfill}{rgb}{0.000000,0.000000,0.000000}%
\pgfsetfillcolor{currentfill}%
\pgfsetlinewidth{0.803000pt}%
\definecolor{currentstroke}{rgb}{0.000000,0.000000,0.000000}%
\pgfsetstrokecolor{currentstroke}%
\pgfsetdash{}{0pt}%
\pgfsys@defobject{currentmarker}{\pgfqpoint{-0.048611in}{0.000000in}}{\pgfqpoint{-0.000000in}{0.000000in}}{%
\pgfpathmoveto{\pgfqpoint{-0.000000in}{0.000000in}}%
\pgfpathlineto{\pgfqpoint{-0.048611in}{0.000000in}}%
\pgfusepath{stroke,fill}%
}%
\begin{pgfscope}%
\pgfsys@transformshift{0.669653in}{2.182986in}%
\pgfsys@useobject{currentmarker}{}%
\end{pgfscope}%
\end{pgfscope}%
\begin{pgfscope}%
\definecolor{textcolor}{rgb}{0.000000,0.000000,0.000000}%
\pgfsetstrokecolor{textcolor}%
\pgfsetfillcolor{textcolor}%
\pgftext[x=0.351551in, y=2.130225in, left, base]{\color{textcolor}\sffamily\fontsize{10.000000}{12.000000}\selectfont 1.5}%
\end{pgfscope}%
\begin{pgfscope}%
\pgfsetbuttcap%
\pgfsetroundjoin%
\definecolor{currentfill}{rgb}{0.000000,0.000000,0.000000}%
\pgfsetfillcolor{currentfill}%
\pgfsetlinewidth{0.803000pt}%
\definecolor{currentstroke}{rgb}{0.000000,0.000000,0.000000}%
\pgfsetstrokecolor{currentstroke}%
\pgfsetdash{}{0pt}%
\pgfsys@defobject{currentmarker}{\pgfqpoint{-0.048611in}{0.000000in}}{\pgfqpoint{-0.000000in}{0.000000in}}{%
\pgfpathmoveto{\pgfqpoint{-0.000000in}{0.000000in}}%
\pgfpathlineto{\pgfqpoint{-0.048611in}{0.000000in}}%
\pgfusepath{stroke,fill}%
}%
\begin{pgfscope}%
\pgfsys@transformshift{0.669653in}{2.716389in}%
\pgfsys@useobject{currentmarker}{}%
\end{pgfscope}%
\end{pgfscope}%
\begin{pgfscope}%
\definecolor{textcolor}{rgb}{0.000000,0.000000,0.000000}%
\pgfsetstrokecolor{textcolor}%
\pgfsetfillcolor{textcolor}%
\pgftext[x=0.351551in, y=2.663627in, left, base]{\color{textcolor}\sffamily\fontsize{10.000000}{12.000000}\selectfont 2.0}%
\end{pgfscope}%
\begin{pgfscope}%
\pgfsetbuttcap%
\pgfsetroundjoin%
\definecolor{currentfill}{rgb}{0.000000,0.000000,0.000000}%
\pgfsetfillcolor{currentfill}%
\pgfsetlinewidth{0.803000pt}%
\definecolor{currentstroke}{rgb}{0.000000,0.000000,0.000000}%
\pgfsetstrokecolor{currentstroke}%
\pgfsetdash{}{0pt}%
\pgfsys@defobject{currentmarker}{\pgfqpoint{-0.048611in}{0.000000in}}{\pgfqpoint{-0.000000in}{0.000000in}}{%
\pgfpathmoveto{\pgfqpoint{-0.000000in}{0.000000in}}%
\pgfpathlineto{\pgfqpoint{-0.048611in}{0.000000in}}%
\pgfusepath{stroke,fill}%
}%
\begin{pgfscope}%
\pgfsys@transformshift{0.669653in}{3.249792in}%
\pgfsys@useobject{currentmarker}{}%
\end{pgfscope}%
\end{pgfscope}%
\begin{pgfscope}%
\definecolor{textcolor}{rgb}{0.000000,0.000000,0.000000}%
\pgfsetstrokecolor{textcolor}%
\pgfsetfillcolor{textcolor}%
\pgftext[x=0.351551in, y=3.197030in, left, base]{\color{textcolor}\sffamily\fontsize{10.000000}{12.000000}\selectfont 2.5}%
\end{pgfscope}%
\begin{pgfscope}%
\pgfsetbuttcap%
\pgfsetroundjoin%
\definecolor{currentfill}{rgb}{0.000000,0.000000,0.000000}%
\pgfsetfillcolor{currentfill}%
\pgfsetlinewidth{0.803000pt}%
\definecolor{currentstroke}{rgb}{0.000000,0.000000,0.000000}%
\pgfsetstrokecolor{currentstroke}%
\pgfsetdash{}{0pt}%
\pgfsys@defobject{currentmarker}{\pgfqpoint{-0.048611in}{0.000000in}}{\pgfqpoint{-0.000000in}{0.000000in}}{%
\pgfpathmoveto{\pgfqpoint{-0.000000in}{0.000000in}}%
\pgfpathlineto{\pgfqpoint{-0.048611in}{0.000000in}}%
\pgfusepath{stroke,fill}%
}%
\begin{pgfscope}%
\pgfsys@transformshift{0.669653in}{3.783194in}%
\pgfsys@useobject{currentmarker}{}%
\end{pgfscope}%
\end{pgfscope}%
\begin{pgfscope}%
\definecolor{textcolor}{rgb}{0.000000,0.000000,0.000000}%
\pgfsetstrokecolor{textcolor}%
\pgfsetfillcolor{textcolor}%
\pgftext[x=0.351551in, y=3.730433in, left, base]{\color{textcolor}\sffamily\fontsize{10.000000}{12.000000}\selectfont 3.0}%
\end{pgfscope}%
\begin{pgfscope}%
\pgfsetbuttcap%
\pgfsetroundjoin%
\definecolor{currentfill}{rgb}{0.000000,0.000000,0.000000}%
\pgfsetfillcolor{currentfill}%
\pgfsetlinewidth{0.803000pt}%
\definecolor{currentstroke}{rgb}{0.000000,0.000000,0.000000}%
\pgfsetstrokecolor{currentstroke}%
\pgfsetdash{}{0pt}%
\pgfsys@defobject{currentmarker}{\pgfqpoint{-0.048611in}{0.000000in}}{\pgfqpoint{-0.000000in}{0.000000in}}{%
\pgfpathmoveto{\pgfqpoint{-0.000000in}{0.000000in}}%
\pgfpathlineto{\pgfqpoint{-0.048611in}{0.000000in}}%
\pgfusepath{stroke,fill}%
}%
\begin{pgfscope}%
\pgfsys@transformshift{0.669653in}{4.316597in}%
\pgfsys@useobject{currentmarker}{}%
\end{pgfscope}%
\end{pgfscope}%
\begin{pgfscope}%
\definecolor{textcolor}{rgb}{0.000000,0.000000,0.000000}%
\pgfsetstrokecolor{textcolor}%
\pgfsetfillcolor{textcolor}%
\pgftext[x=0.351551in, y=4.263836in, left, base]{\color{textcolor}\sffamily\fontsize{10.000000}{12.000000}\selectfont 3.5}%
\end{pgfscope}%
\begin{pgfscope}%
\pgfsetbuttcap%
\pgfsetroundjoin%
\definecolor{currentfill}{rgb}{0.000000,0.000000,0.000000}%
\pgfsetfillcolor{currentfill}%
\pgfsetlinewidth{0.803000pt}%
\definecolor{currentstroke}{rgb}{0.000000,0.000000,0.000000}%
\pgfsetstrokecolor{currentstroke}%
\pgfsetdash{}{0pt}%
\pgfsys@defobject{currentmarker}{\pgfqpoint{-0.048611in}{0.000000in}}{\pgfqpoint{-0.000000in}{0.000000in}}{%
\pgfpathmoveto{\pgfqpoint{-0.000000in}{0.000000in}}%
\pgfpathlineto{\pgfqpoint{-0.048611in}{0.000000in}}%
\pgfusepath{stroke,fill}%
}%
\begin{pgfscope}%
\pgfsys@transformshift{0.669653in}{4.850000in}%
\pgfsys@useobject{currentmarker}{}%
\end{pgfscope}%
\end{pgfscope}%
\begin{pgfscope}%
\definecolor{textcolor}{rgb}{0.000000,0.000000,0.000000}%
\pgfsetstrokecolor{textcolor}%
\pgfsetfillcolor{textcolor}%
\pgftext[x=0.351551in, y=4.797238in, left, base]{\color{textcolor}\sffamily\fontsize{10.000000}{12.000000}\selectfont 4.0}%
\end{pgfscope}%
\begin{pgfscope}%
\definecolor{textcolor}{rgb}{0.000000,0.000000,0.000000}%
\pgfsetstrokecolor{textcolor}%
\pgfsetfillcolor{textcolor}%
\pgftext[x=0.295996in,y=2.716389in,,bottom,rotate=90.000000]{\color{textcolor}\sffamily\fontsize{10.000000}{12.000000}\selectfont median latency (s)}%
\end{pgfscope}%
\begin{pgfscope}%
\pgfpathrectangle{\pgfqpoint{0.669653in}{0.582778in}}{\pgfqpoint{4.180347in}{4.267222in}}%
\pgfusepath{clip}%
\pgfsetrectcap%
\pgfsetroundjoin%
\pgfsetlinewidth{2.258437pt}%
\definecolor{currentstroke}{rgb}{0.121569,0.466667,0.705882}%
\pgfsetstrokecolor{currentstroke}%
\pgfsetdash{}{0pt}%
\pgfpathmoveto{\pgfqpoint{0.688291in}{1.927278in}}%
\pgfpathlineto{\pgfqpoint{0.728327in}{1.951775in}}%
\pgfpathlineto{\pgfqpoint{0.768363in}{1.972714in}}%
\pgfpathlineto{\pgfqpoint{0.808398in}{1.990474in}}%
\pgfpathlineto{\pgfqpoint{0.848434in}{2.005411in}}%
\pgfpathlineto{\pgfqpoint{0.888470in}{2.017854in}}%
\pgfpathlineto{\pgfqpoint{0.928505in}{2.028113in}}%
\pgfpathlineto{\pgfqpoint{0.968541in}{2.036474in}}%
\pgfpathlineto{\pgfqpoint{1.008577in}{2.043202in}}%
\pgfpathlineto{\pgfqpoint{1.048612in}{2.048541in}}%
\pgfpathlineto{\pgfqpoint{1.088648in}{2.052717in}}%
\pgfpathlineto{\pgfqpoint{1.128684in}{2.055933in}}%
\pgfpathlineto{\pgfqpoint{1.168719in}{2.058379in}}%
\pgfpathlineto{\pgfqpoint{1.208755in}{2.060221in}}%
\pgfpathlineto{\pgfqpoint{1.248791in}{2.061611in}}%
\pgfpathlineto{\pgfqpoint{1.288826in}{2.062686in}}%
\pgfpathlineto{\pgfqpoint{1.328862in}{2.063562in}}%
\pgfpathlineto{\pgfqpoint{1.368898in}{2.064346in}}%
\pgfpathlineto{\pgfqpoint{1.408933in}{2.065125in}}%
\pgfpathlineto{\pgfqpoint{1.448969in}{2.065975in}}%
\pgfpathlineto{\pgfqpoint{1.489005in}{2.066959in}}%
\pgfpathlineto{\pgfqpoint{1.529040in}{2.068127in}}%
\pgfpathlineto{\pgfqpoint{1.569076in}{2.069515in}}%
\pgfpathlineto{\pgfqpoint{1.609112in}{2.071151in}}%
\pgfpathlineto{\pgfqpoint{1.649148in}{2.073052in}}%
\pgfpathlineto{\pgfqpoint{1.689183in}{2.075223in}}%
\pgfpathlineto{\pgfqpoint{1.729219in}{2.077661in}}%
\pgfpathlineto{\pgfqpoint{1.769255in}{2.080357in}}%
\pgfpathlineto{\pgfqpoint{1.809290in}{2.083289in}}%
\pgfpathlineto{\pgfqpoint{1.849326in}{2.086433in}}%
\pgfpathlineto{\pgfqpoint{1.889362in}{2.089756in}}%
\pgfpathlineto{\pgfqpoint{1.929397in}{2.093219in}}%
\pgfpathlineto{\pgfqpoint{1.969433in}{2.096780in}}%
\pgfpathlineto{\pgfqpoint{2.009469in}{2.100389in}}%
\pgfpathlineto{\pgfqpoint{2.049504in}{2.103996in}}%
\pgfpathlineto{\pgfqpoint{2.089540in}{2.107546in}}%
\pgfpathlineto{\pgfqpoint{2.129576in}{2.110982in}}%
\pgfpathlineto{\pgfqpoint{2.169611in}{2.114244in}}%
\pgfpathlineto{\pgfqpoint{2.209647in}{2.117274in}}%
\pgfpathlineto{\pgfqpoint{2.249683in}{2.120011in}}%
\pgfpathlineto{\pgfqpoint{2.289718in}{2.122394in}}%
\pgfpathlineto{\pgfqpoint{2.329754in}{2.124366in}}%
\pgfpathlineto{\pgfqpoint{2.369790in}{2.125868in}}%
\pgfpathlineto{\pgfqpoint{2.409825in}{2.126847in}}%
\pgfpathlineto{\pgfqpoint{2.449861in}{2.127249in}}%
\pgfpathlineto{\pgfqpoint{2.489897in}{2.127027in}}%
\pgfpathlineto{\pgfqpoint{2.529932in}{2.126136in}}%
\pgfpathlineto{\pgfqpoint{2.569968in}{2.124539in}}%
\pgfpathlineto{\pgfqpoint{2.610004in}{2.122201in}}%
\pgfpathlineto{\pgfqpoint{2.650039in}{2.119096in}}%
\pgfpathlineto{\pgfqpoint{2.690075in}{2.115205in}}%
\pgfpathlineto{\pgfqpoint{2.730111in}{2.110514in}}%
\pgfpathlineto{\pgfqpoint{2.770147in}{2.105021in}}%
\pgfpathlineto{\pgfqpoint{2.810182in}{2.098731in}}%
\pgfpathlineto{\pgfqpoint{2.850218in}{2.091659in}}%
\pgfpathlineto{\pgfqpoint{2.890254in}{2.083832in}}%
\pgfpathlineto{\pgfqpoint{2.930289in}{2.075285in}}%
\pgfpathlineto{\pgfqpoint{2.970325in}{2.066067in}}%
\pgfpathlineto{\pgfqpoint{3.010361in}{2.056239in}}%
\pgfpathlineto{\pgfqpoint{3.050396in}{2.045876in}}%
\pgfpathlineto{\pgfqpoint{3.090432in}{2.035066in}}%
\pgfpathlineto{\pgfqpoint{3.130468in}{2.023910in}}%
\pgfpathlineto{\pgfqpoint{3.170503in}{2.012526in}}%
\pgfpathlineto{\pgfqpoint{3.210539in}{2.001048in}}%
\pgfpathlineto{\pgfqpoint{3.250575in}{1.989626in}}%
\pgfpathlineto{\pgfqpoint{3.290610in}{1.978427in}}%
\pgfpathlineto{\pgfqpoint{3.330646in}{1.967636in}}%
\pgfpathlineto{\pgfqpoint{3.370682in}{1.957455in}}%
\pgfpathlineto{\pgfqpoint{3.410717in}{1.948109in}}%
\pgfpathlineto{\pgfqpoint{3.450753in}{1.939839in}}%
\pgfpathlineto{\pgfqpoint{3.490789in}{1.932909in}}%
\pgfpathlineto{\pgfqpoint{3.530824in}{1.927602in}}%
\pgfpathlineto{\pgfqpoint{3.570860in}{1.924226in}}%
\pgfpathlineto{\pgfqpoint{3.610896in}{1.923109in}}%
\pgfpathlineto{\pgfqpoint{3.650931in}{1.924604in}}%
\pgfpathlineto{\pgfqpoint{3.690967in}{1.929085in}}%
\pgfpathlineto{\pgfqpoint{3.731003in}{1.936954in}}%
\pgfpathlineto{\pgfqpoint{3.771038in}{1.948638in}}%
\pgfpathlineto{\pgfqpoint{3.811074in}{1.964586in}}%
\pgfpathlineto{\pgfqpoint{3.851110in}{1.985279in}}%
\pgfpathlineto{\pgfqpoint{3.891146in}{2.011222in}}%
\pgfpathlineto{\pgfqpoint{3.931181in}{2.042949in}}%
\pgfpathlineto{\pgfqpoint{3.971217in}{2.081022in}}%
\pgfpathlineto{\pgfqpoint{4.011253in}{2.126033in}}%
\pgfpathlineto{\pgfqpoint{4.051288in}{2.178604in}}%
\pgfpathlineto{\pgfqpoint{4.091324in}{2.239387in}}%
\pgfpathlineto{\pgfqpoint{4.131360in}{2.309066in}}%
\pgfpathlineto{\pgfqpoint{4.171395in}{2.388358in}}%
\pgfpathlineto{\pgfqpoint{4.211431in}{2.478010in}}%
\pgfpathlineto{\pgfqpoint{4.251467in}{2.578804in}}%
\pgfpathlineto{\pgfqpoint{4.291502in}{2.691557in}}%
\pgfpathlineto{\pgfqpoint{4.331538in}{2.817119in}}%
\pgfpathlineto{\pgfqpoint{4.371574in}{2.956377in}}%
\pgfpathlineto{\pgfqpoint{4.411609in}{3.110252in}}%
\pgfpathlineto{\pgfqpoint{4.451645in}{3.279704in}}%
\pgfpathlineto{\pgfqpoint{4.491681in}{3.465729in}}%
\pgfpathlineto{\pgfqpoint{4.531716in}{3.669361in}}%
\pgfpathlineto{\pgfqpoint{4.571752in}{3.891675in}}%
\pgfpathlineto{\pgfqpoint{4.611788in}{4.133782in}}%
\pgfpathlineto{\pgfqpoint{4.651823in}{4.396835in}}%
\pgfusepath{stroke}%
\end{pgfscope}%
\begin{pgfscope}%
\pgfsetrectcap%
\pgfsetmiterjoin%
\pgfsetlinewidth{0.803000pt}%
\definecolor{currentstroke}{rgb}{0.000000,0.000000,0.000000}%
\pgfsetstrokecolor{currentstroke}%
\pgfsetdash{}{0pt}%
\pgfpathmoveto{\pgfqpoint{0.669653in}{0.582778in}}%
\pgfpathlineto{\pgfqpoint{0.669653in}{4.850000in}}%
\pgfusepath{stroke}%
\end{pgfscope}%
\begin{pgfscope}%
\pgfsetrectcap%
\pgfsetmiterjoin%
\pgfsetlinewidth{0.803000pt}%
\definecolor{currentstroke}{rgb}{0.000000,0.000000,0.000000}%
\pgfsetstrokecolor{currentstroke}%
\pgfsetdash{}{0pt}%
\pgfpathmoveto{\pgfqpoint{0.669653in}{0.582778in}}%
\pgfpathlineto{\pgfqpoint{4.850000in}{0.582778in}}%
\pgfusepath{stroke}%
\end{pgfscope}%
\end{pgfpicture}%
\makeatother%
\endgroup%
}
\caption{Benchmarking of goodput and median latency while varying throughputs.}
\label{goodputlatencymininet}
\end{figure}

\begin{figure}[h!]
\centering
\resizebox{.6\textwidth}{!}{%% Creator: Matplotlib, PGF backend
%%
%% To include the figure in your LaTeX document, write
%%   \input{<filename>.pgf}
%%
%% Make sure the required packages are loaded in your preamble
%%   \usepackage{pgf}
%%
%% Also ensure that all the required font packages are loaded; for instance,
%% the lmodern package is sometimes necessary when using math font.
%%   \usepackage{lmodern}
%%
%% Figures using additional raster images can only be included by \input if
%% they are in the same directory as the main LaTeX file. For loading figures
%% from other directories you can use the `import` package
%%   \usepackage{import}
%%
%% and then include the figures with
%%   \import{<path to file>}{<filename>.pgf}
%%
%% Matplotlib used the following preamble
%%   
%%   \usepackage{fontspec}
%%   \setmainfont{DejaVuSerif.ttf}[Path=\detokenize{/opt/homebrew/lib/python3.10/site-packages/matplotlib/mpl-data/fonts/ttf/}]
%%   \setsansfont{DejaVuSans.ttf}[Path=\detokenize{/opt/homebrew/lib/python3.10/site-packages/matplotlib/mpl-data/fonts/ttf/}]
%%   \setmonofont{DejaVuSansMono.ttf}[Path=\detokenize{/opt/homebrew/lib/python3.10/site-packages/matplotlib/mpl-data/fonts/ttf/}]
%%   \makeatletter\@ifpackageloaded{underscore}{}{\usepackage[strings]{underscore}}\makeatother
%%
\begingroup%
\makeatletter%
\begin{pgfpicture}%
\pgfpathrectangle{\pgfpointorigin}{\pgfqpoint{6.400000in}{4.800000in}}%
\pgfusepath{use as bounding box, clip}%
\begin{pgfscope}%
\pgfsetbuttcap%
\pgfsetmiterjoin%
\definecolor{currentfill}{rgb}{1.000000,1.000000,1.000000}%
\pgfsetfillcolor{currentfill}%
\pgfsetlinewidth{0.000000pt}%
\definecolor{currentstroke}{rgb}{1.000000,1.000000,1.000000}%
\pgfsetstrokecolor{currentstroke}%
\pgfsetdash{}{0pt}%
\pgfpathmoveto{\pgfqpoint{0.000000in}{0.000000in}}%
\pgfpathlineto{\pgfqpoint{6.400000in}{0.000000in}}%
\pgfpathlineto{\pgfqpoint{6.400000in}{4.800000in}}%
\pgfpathlineto{\pgfqpoint{0.000000in}{4.800000in}}%
\pgfpathlineto{\pgfqpoint{0.000000in}{0.000000in}}%
\pgfpathclose%
\pgfusepath{fill}%
\end{pgfscope}%
\begin{pgfscope}%
\pgfsetbuttcap%
\pgfsetmiterjoin%
\definecolor{currentfill}{rgb}{1.000000,1.000000,1.000000}%
\pgfsetfillcolor{currentfill}%
\pgfsetlinewidth{0.000000pt}%
\definecolor{currentstroke}{rgb}{0.000000,0.000000,0.000000}%
\pgfsetstrokecolor{currentstroke}%
\pgfsetstrokeopacity{0.000000}%
\pgfsetdash{}{0pt}%
\pgfpathmoveto{\pgfqpoint{0.800000in}{0.528000in}}%
\pgfpathlineto{\pgfqpoint{5.760000in}{0.528000in}}%
\pgfpathlineto{\pgfqpoint{5.760000in}{4.224000in}}%
\pgfpathlineto{\pgfqpoint{0.800000in}{4.224000in}}%
\pgfpathlineto{\pgfqpoint{0.800000in}{0.528000in}}%
\pgfpathclose%
\pgfusepath{fill}%
\end{pgfscope}%
\begin{pgfscope}%
\pgfsetbuttcap%
\pgfsetroundjoin%
\definecolor{currentfill}{rgb}{0.000000,0.000000,0.000000}%
\pgfsetfillcolor{currentfill}%
\pgfsetlinewidth{0.602250pt}%
\definecolor{currentstroke}{rgb}{0.000000,0.000000,0.000000}%
\pgfsetstrokecolor{currentstroke}%
\pgfsetdash{}{0pt}%
\pgfsys@defobject{currentmarker}{\pgfqpoint{0.000000in}{-0.027778in}}{\pgfqpoint{0.000000in}{0.000000in}}{%
\pgfpathmoveto{\pgfqpoint{0.000000in}{0.000000in}}%
\pgfpathlineto{\pgfqpoint{0.000000in}{-0.027778in}}%
\pgfusepath{stroke,fill}%
}%
\begin{pgfscope}%
\pgfsys@transformshift{1.549418in}{0.528000in}%
\pgfsys@useobject{currentmarker}{}%
\end{pgfscope}%
\end{pgfscope}%
\begin{pgfscope}%
\definecolor{textcolor}{rgb}{0.000000,0.000000,0.000000}%
\pgfsetstrokecolor{textcolor}%
\pgfsetfillcolor{textcolor}%
\pgftext[x=1.549418in,y=0.453000in,,top]{\color{textcolor}\sffamily\fontsize{10.000000}{12.000000}\selectfont \(\displaystyle {1.4\times10^{0}}\)}%
\end{pgfscope}%
\begin{pgfscope}%
\pgfsetbuttcap%
\pgfsetroundjoin%
\definecolor{currentfill}{rgb}{0.000000,0.000000,0.000000}%
\pgfsetfillcolor{currentfill}%
\pgfsetlinewidth{0.602250pt}%
\definecolor{currentstroke}{rgb}{0.000000,0.000000,0.000000}%
\pgfsetstrokecolor{currentstroke}%
\pgfsetdash{}{0pt}%
\pgfsys@defobject{currentmarker}{\pgfqpoint{0.000000in}{-0.027778in}}{\pgfqpoint{0.000000in}{0.000000in}}{%
\pgfpathmoveto{\pgfqpoint{0.000000in}{0.000000in}}%
\pgfpathlineto{\pgfqpoint{0.000000in}{-0.027778in}}%
\pgfusepath{stroke,fill}%
}%
\begin{pgfscope}%
\pgfsys@transformshift{2.690142in}{0.528000in}%
\pgfsys@useobject{currentmarker}{}%
\end{pgfscope}%
\end{pgfscope}%
\begin{pgfscope}%
\definecolor{textcolor}{rgb}{0.000000,0.000000,0.000000}%
\pgfsetstrokecolor{textcolor}%
\pgfsetfillcolor{textcolor}%
\pgftext[x=2.690142in,y=0.453000in,,top]{\color{textcolor}\sffamily\fontsize{10.000000}{12.000000}\selectfont \(\displaystyle {1.6\times10^{0}}\)}%
\end{pgfscope}%
\begin{pgfscope}%
\pgfsetbuttcap%
\pgfsetroundjoin%
\definecolor{currentfill}{rgb}{0.000000,0.000000,0.000000}%
\pgfsetfillcolor{currentfill}%
\pgfsetlinewidth{0.602250pt}%
\definecolor{currentstroke}{rgb}{0.000000,0.000000,0.000000}%
\pgfsetstrokecolor{currentstroke}%
\pgfsetdash{}{0pt}%
\pgfsys@defobject{currentmarker}{\pgfqpoint{0.000000in}{-0.027778in}}{\pgfqpoint{0.000000in}{0.000000in}}{%
\pgfpathmoveto{\pgfqpoint{0.000000in}{0.000000in}}%
\pgfpathlineto{\pgfqpoint{0.000000in}{-0.027778in}}%
\pgfusepath{stroke,fill}%
}%
\begin{pgfscope}%
\pgfsys@transformshift{3.696331in}{0.528000in}%
\pgfsys@useobject{currentmarker}{}%
\end{pgfscope}%
\end{pgfscope}%
\begin{pgfscope}%
\definecolor{textcolor}{rgb}{0.000000,0.000000,0.000000}%
\pgfsetstrokecolor{textcolor}%
\pgfsetfillcolor{textcolor}%
\pgftext[x=3.696331in,y=0.453000in,,top]{\color{textcolor}\sffamily\fontsize{10.000000}{12.000000}\selectfont \(\displaystyle {1.8\times10^{0}}\)}%
\end{pgfscope}%
\begin{pgfscope}%
\pgfsetbuttcap%
\pgfsetroundjoin%
\definecolor{currentfill}{rgb}{0.000000,0.000000,0.000000}%
\pgfsetfillcolor{currentfill}%
\pgfsetlinewidth{0.602250pt}%
\definecolor{currentstroke}{rgb}{0.000000,0.000000,0.000000}%
\pgfsetstrokecolor{currentstroke}%
\pgfsetdash{}{0pt}%
\pgfsys@defobject{currentmarker}{\pgfqpoint{0.000000in}{-0.027778in}}{\pgfqpoint{0.000000in}{0.000000in}}{%
\pgfpathmoveto{\pgfqpoint{0.000000in}{0.000000in}}%
\pgfpathlineto{\pgfqpoint{0.000000in}{-0.027778in}}%
\pgfusepath{stroke,fill}%
}%
\begin{pgfscope}%
\pgfsys@transformshift{4.596398in}{0.528000in}%
\pgfsys@useobject{currentmarker}{}%
\end{pgfscope}%
\end{pgfscope}%
\begin{pgfscope}%
\definecolor{textcolor}{rgb}{0.000000,0.000000,0.000000}%
\pgfsetstrokecolor{textcolor}%
\pgfsetfillcolor{textcolor}%
\pgftext[x=4.596398in,y=0.453000in,,top]{\color{textcolor}\sffamily\fontsize{10.000000}{12.000000}\selectfont \(\displaystyle {2\times10^{0}}\)}%
\end{pgfscope}%
\begin{pgfscope}%
\pgfsetbuttcap%
\pgfsetroundjoin%
\definecolor{currentfill}{rgb}{0.000000,0.000000,0.000000}%
\pgfsetfillcolor{currentfill}%
\pgfsetlinewidth{0.602250pt}%
\definecolor{currentstroke}{rgb}{0.000000,0.000000,0.000000}%
\pgfsetstrokecolor{currentstroke}%
\pgfsetdash{}{0pt}%
\pgfsys@defobject{currentmarker}{\pgfqpoint{0.000000in}{-0.027778in}}{\pgfqpoint{0.000000in}{0.000000in}}{%
\pgfpathmoveto{\pgfqpoint{0.000000in}{0.000000in}}%
\pgfpathlineto{\pgfqpoint{0.000000in}{-0.027778in}}%
\pgfusepath{stroke,fill}%
}%
\begin{pgfscope}%
\pgfsys@transformshift{5.410608in}{0.528000in}%
\pgfsys@useobject{currentmarker}{}%
\end{pgfscope}%
\end{pgfscope}%
\begin{pgfscope}%
\definecolor{textcolor}{rgb}{0.000000,0.000000,0.000000}%
\pgfsetstrokecolor{textcolor}%
\pgfsetfillcolor{textcolor}%
\pgftext[x=5.410608in,y=0.453000in,,top]{\color{textcolor}\sffamily\fontsize{10.000000}{12.000000}\selectfont \(\displaystyle {2.2\times10^{0}}\)}%
\end{pgfscope}%
\begin{pgfscope}%
\definecolor{textcolor}{rgb}{0.000000,0.000000,0.000000}%
\pgfsetstrokecolor{textcolor}%
\pgfsetfillcolor{textcolor}%
\pgftext[x=3.280000in,y=0.263032in,,top]{\color{textcolor}\sffamily\fontsize{10.000000}{12.000000}\selectfont latency (s)}%
\end{pgfscope}%
\begin{pgfscope}%
\pgfsetbuttcap%
\pgfsetroundjoin%
\definecolor{currentfill}{rgb}{0.000000,0.000000,0.000000}%
\pgfsetfillcolor{currentfill}%
\pgfsetlinewidth{0.803000pt}%
\definecolor{currentstroke}{rgb}{0.000000,0.000000,0.000000}%
\pgfsetstrokecolor{currentstroke}%
\pgfsetdash{}{0pt}%
\pgfsys@defobject{currentmarker}{\pgfqpoint{-0.048611in}{0.000000in}}{\pgfqpoint{-0.000000in}{0.000000in}}{%
\pgfpathmoveto{\pgfqpoint{-0.000000in}{0.000000in}}%
\pgfpathlineto{\pgfqpoint{-0.048611in}{0.000000in}}%
\pgfusepath{stroke,fill}%
}%
\begin{pgfscope}%
\pgfsys@transformshift{0.800000in}{0.528000in}%
\pgfsys@useobject{currentmarker}{}%
\end{pgfscope}%
\end{pgfscope}%
\begin{pgfscope}%
\definecolor{textcolor}{rgb}{0.000000,0.000000,0.000000}%
\pgfsetstrokecolor{textcolor}%
\pgfsetfillcolor{textcolor}%
\pgftext[x=0.481898in, y=0.475238in, left, base]{\color{textcolor}\sffamily\fontsize{10.000000}{12.000000}\selectfont 0.0}%
\end{pgfscope}%
\begin{pgfscope}%
\pgfsetbuttcap%
\pgfsetroundjoin%
\definecolor{currentfill}{rgb}{0.000000,0.000000,0.000000}%
\pgfsetfillcolor{currentfill}%
\pgfsetlinewidth{0.803000pt}%
\definecolor{currentstroke}{rgb}{0.000000,0.000000,0.000000}%
\pgfsetstrokecolor{currentstroke}%
\pgfsetdash{}{0pt}%
\pgfsys@defobject{currentmarker}{\pgfqpoint{-0.048611in}{0.000000in}}{\pgfqpoint{-0.000000in}{0.000000in}}{%
\pgfpathmoveto{\pgfqpoint{-0.000000in}{0.000000in}}%
\pgfpathlineto{\pgfqpoint{-0.048611in}{0.000000in}}%
\pgfusepath{stroke,fill}%
}%
\begin{pgfscope}%
\pgfsys@transformshift{0.800000in}{1.267200in}%
\pgfsys@useobject{currentmarker}{}%
\end{pgfscope}%
\end{pgfscope}%
\begin{pgfscope}%
\definecolor{textcolor}{rgb}{0.000000,0.000000,0.000000}%
\pgfsetstrokecolor{textcolor}%
\pgfsetfillcolor{textcolor}%
\pgftext[x=0.481898in, y=1.214438in, left, base]{\color{textcolor}\sffamily\fontsize{10.000000}{12.000000}\selectfont 0.2}%
\end{pgfscope}%
\begin{pgfscope}%
\pgfsetbuttcap%
\pgfsetroundjoin%
\definecolor{currentfill}{rgb}{0.000000,0.000000,0.000000}%
\pgfsetfillcolor{currentfill}%
\pgfsetlinewidth{0.803000pt}%
\definecolor{currentstroke}{rgb}{0.000000,0.000000,0.000000}%
\pgfsetstrokecolor{currentstroke}%
\pgfsetdash{}{0pt}%
\pgfsys@defobject{currentmarker}{\pgfqpoint{-0.048611in}{0.000000in}}{\pgfqpoint{-0.000000in}{0.000000in}}{%
\pgfpathmoveto{\pgfqpoint{-0.000000in}{0.000000in}}%
\pgfpathlineto{\pgfqpoint{-0.048611in}{0.000000in}}%
\pgfusepath{stroke,fill}%
}%
\begin{pgfscope}%
\pgfsys@transformshift{0.800000in}{2.006400in}%
\pgfsys@useobject{currentmarker}{}%
\end{pgfscope}%
\end{pgfscope}%
\begin{pgfscope}%
\definecolor{textcolor}{rgb}{0.000000,0.000000,0.000000}%
\pgfsetstrokecolor{textcolor}%
\pgfsetfillcolor{textcolor}%
\pgftext[x=0.481898in, y=1.953638in, left, base]{\color{textcolor}\sffamily\fontsize{10.000000}{12.000000}\selectfont 0.4}%
\end{pgfscope}%
\begin{pgfscope}%
\pgfsetbuttcap%
\pgfsetroundjoin%
\definecolor{currentfill}{rgb}{0.000000,0.000000,0.000000}%
\pgfsetfillcolor{currentfill}%
\pgfsetlinewidth{0.803000pt}%
\definecolor{currentstroke}{rgb}{0.000000,0.000000,0.000000}%
\pgfsetstrokecolor{currentstroke}%
\pgfsetdash{}{0pt}%
\pgfsys@defobject{currentmarker}{\pgfqpoint{-0.048611in}{0.000000in}}{\pgfqpoint{-0.000000in}{0.000000in}}{%
\pgfpathmoveto{\pgfqpoint{-0.000000in}{0.000000in}}%
\pgfpathlineto{\pgfqpoint{-0.048611in}{0.000000in}}%
\pgfusepath{stroke,fill}%
}%
\begin{pgfscope}%
\pgfsys@transformshift{0.800000in}{2.745600in}%
\pgfsys@useobject{currentmarker}{}%
\end{pgfscope}%
\end{pgfscope}%
\begin{pgfscope}%
\definecolor{textcolor}{rgb}{0.000000,0.000000,0.000000}%
\pgfsetstrokecolor{textcolor}%
\pgfsetfillcolor{textcolor}%
\pgftext[x=0.481898in, y=2.692838in, left, base]{\color{textcolor}\sffamily\fontsize{10.000000}{12.000000}\selectfont 0.6}%
\end{pgfscope}%
\begin{pgfscope}%
\pgfsetbuttcap%
\pgfsetroundjoin%
\definecolor{currentfill}{rgb}{0.000000,0.000000,0.000000}%
\pgfsetfillcolor{currentfill}%
\pgfsetlinewidth{0.803000pt}%
\definecolor{currentstroke}{rgb}{0.000000,0.000000,0.000000}%
\pgfsetstrokecolor{currentstroke}%
\pgfsetdash{}{0pt}%
\pgfsys@defobject{currentmarker}{\pgfqpoint{-0.048611in}{0.000000in}}{\pgfqpoint{-0.000000in}{0.000000in}}{%
\pgfpathmoveto{\pgfqpoint{-0.000000in}{0.000000in}}%
\pgfpathlineto{\pgfqpoint{-0.048611in}{0.000000in}}%
\pgfusepath{stroke,fill}%
}%
\begin{pgfscope}%
\pgfsys@transformshift{0.800000in}{3.484800in}%
\pgfsys@useobject{currentmarker}{}%
\end{pgfscope}%
\end{pgfscope}%
\begin{pgfscope}%
\definecolor{textcolor}{rgb}{0.000000,0.000000,0.000000}%
\pgfsetstrokecolor{textcolor}%
\pgfsetfillcolor{textcolor}%
\pgftext[x=0.481898in, y=3.432038in, left, base]{\color{textcolor}\sffamily\fontsize{10.000000}{12.000000}\selectfont 0.8}%
\end{pgfscope}%
\begin{pgfscope}%
\pgfsetbuttcap%
\pgfsetroundjoin%
\definecolor{currentfill}{rgb}{0.000000,0.000000,0.000000}%
\pgfsetfillcolor{currentfill}%
\pgfsetlinewidth{0.803000pt}%
\definecolor{currentstroke}{rgb}{0.000000,0.000000,0.000000}%
\pgfsetstrokecolor{currentstroke}%
\pgfsetdash{}{0pt}%
\pgfsys@defobject{currentmarker}{\pgfqpoint{-0.048611in}{0.000000in}}{\pgfqpoint{-0.000000in}{0.000000in}}{%
\pgfpathmoveto{\pgfqpoint{-0.000000in}{0.000000in}}%
\pgfpathlineto{\pgfqpoint{-0.048611in}{0.000000in}}%
\pgfusepath{stroke,fill}%
}%
\begin{pgfscope}%
\pgfsys@transformshift{0.800000in}{4.224000in}%
\pgfsys@useobject{currentmarker}{}%
\end{pgfscope}%
\end{pgfscope}%
\begin{pgfscope}%
\definecolor{textcolor}{rgb}{0.000000,0.000000,0.000000}%
\pgfsetstrokecolor{textcolor}%
\pgfsetfillcolor{textcolor}%
\pgftext[x=0.481898in, y=4.171238in, left, base]{\color{textcolor}\sffamily\fontsize{10.000000}{12.000000}\selectfont 1.0}%
\end{pgfscope}%
\begin{pgfscope}%
\definecolor{textcolor}{rgb}{0.000000,0.000000,0.000000}%
\pgfsetstrokecolor{textcolor}%
\pgfsetfillcolor{textcolor}%
\pgftext[x=0.426343in,y=2.376000in,,bottom,rotate=90.000000]{\color{textcolor}\sffamily\fontsize{10.000000}{12.000000}\selectfont fraction of requests}%
\end{pgfscope}%
\begin{pgfscope}%
\pgfpathrectangle{\pgfqpoint{0.800000in}{0.528000in}}{\pgfqpoint{4.960000in}{3.696000in}}%
\pgfusepath{clip}%
\pgfsetrectcap%
\pgfsetroundjoin%
\pgfsetlinewidth{1.505625pt}%
\definecolor{currentstroke}{rgb}{0.121569,0.466667,0.705882}%
\pgfsetstrokecolor{currentstroke}%
\pgfsetdash{}{0pt}%
\pgfpathmoveto{\pgfqpoint{0.790000in}{0.528000in}}%
\pgfpathlineto{\pgfqpoint{1.025455in}{0.528000in}}%
\pgfpathlineto{\pgfqpoint{1.025455in}{0.529848in}}%
\pgfpathlineto{\pgfqpoint{1.053534in}{0.529848in}}%
\pgfpathlineto{\pgfqpoint{1.053534in}{0.531696in}}%
\pgfpathlineto{\pgfqpoint{1.088010in}{0.531696in}}%
\pgfpathlineto{\pgfqpoint{1.088010in}{0.533544in}}%
\pgfpathlineto{\pgfqpoint{1.115856in}{0.533544in}}%
\pgfpathlineto{\pgfqpoint{1.115856in}{0.535392in}}%
\pgfpathlineto{\pgfqpoint{1.151190in}{0.535392in}}%
\pgfpathlineto{\pgfqpoint{1.151190in}{0.537240in}}%
\pgfpathlineto{\pgfqpoint{1.173182in}{0.537240in}}%
\pgfpathlineto{\pgfqpoint{1.173182in}{0.539088in}}%
\pgfpathlineto{\pgfqpoint{1.185692in}{0.539088in}}%
\pgfpathlineto{\pgfqpoint{1.185692in}{0.540936in}}%
\pgfpathlineto{\pgfqpoint{1.200623in}{0.540936in}}%
\pgfpathlineto{\pgfqpoint{1.200623in}{0.542784in}}%
\pgfpathlineto{\pgfqpoint{1.213267in}{0.542784in}}%
\pgfpathlineto{\pgfqpoint{1.213267in}{0.544632in}}%
\pgfpathlineto{\pgfqpoint{1.234501in}{0.544632in}}%
\pgfpathlineto{\pgfqpoint{1.234501in}{0.546480in}}%
\pgfpathlineto{\pgfqpoint{1.247766in}{0.546480in}}%
\pgfpathlineto{\pgfqpoint{1.247766in}{0.548328in}}%
\pgfpathlineto{\pgfqpoint{1.268078in}{0.548328in}}%
\pgfpathlineto{\pgfqpoint{1.268078in}{0.550176in}}%
\pgfpathlineto{\pgfqpoint{1.281845in}{0.550176in}}%
\pgfpathlineto{\pgfqpoint{1.281845in}{0.552024in}}%
\pgfpathlineto{\pgfqpoint{1.295695in}{0.552024in}}%
\pgfpathlineto{\pgfqpoint{1.295695in}{0.553872in}}%
\pgfpathlineto{\pgfqpoint{1.308743in}{0.553872in}}%
\pgfpathlineto{\pgfqpoint{1.308743in}{0.555720in}}%
\pgfpathlineto{\pgfqpoint{1.329115in}{0.555720in}}%
\pgfpathlineto{\pgfqpoint{1.329115in}{0.557568in}}%
\pgfpathlineto{\pgfqpoint{1.342867in}{0.557568in}}%
\pgfpathlineto{\pgfqpoint{1.342867in}{0.559416in}}%
\pgfpathlineto{\pgfqpoint{1.362665in}{0.559416in}}%
\pgfpathlineto{\pgfqpoint{1.362665in}{0.561264in}}%
\pgfpathlineto{\pgfqpoint{1.376602in}{0.561264in}}%
\pgfpathlineto{\pgfqpoint{1.376602in}{0.563112in}}%
\pgfpathlineto{\pgfqpoint{1.390221in}{0.563112in}}%
\pgfpathlineto{\pgfqpoint{1.390221in}{0.564960in}}%
\pgfpathlineto{\pgfqpoint{1.403548in}{0.564960in}}%
\pgfpathlineto{\pgfqpoint{1.403548in}{0.566808in}}%
\pgfpathlineto{\pgfqpoint{1.421818in}{0.566808in}}%
\pgfpathlineto{\pgfqpoint{1.421818in}{0.568656in}}%
\pgfpathlineto{\pgfqpoint{1.423744in}{0.568656in}}%
\pgfpathlineto{\pgfqpoint{1.423744in}{0.570504in}}%
\pgfpathlineto{\pgfqpoint{1.437065in}{0.570504in}}%
\pgfpathlineto{\pgfqpoint{1.437065in}{0.572352in}}%
\pgfpathlineto{\pgfqpoint{1.450545in}{0.572352in}}%
\pgfpathlineto{\pgfqpoint{1.450545in}{0.574200in}}%
\pgfpathlineto{\pgfqpoint{1.456555in}{0.574200in}}%
\pgfpathlineto{\pgfqpoint{1.456555in}{0.576048in}}%
\pgfpathlineto{\pgfqpoint{1.463784in}{0.576048in}}%
\pgfpathlineto{\pgfqpoint{1.463784in}{0.577896in}}%
\pgfpathlineto{\pgfqpoint{1.481975in}{0.577896in}}%
\pgfpathlineto{\pgfqpoint{1.481975in}{0.579744in}}%
\pgfpathlineto{\pgfqpoint{1.483446in}{0.579744in}}%
\pgfpathlineto{\pgfqpoint{1.484385in}{0.585288in}}%
\pgfpathlineto{\pgfqpoint{1.508584in}{0.585288in}}%
\pgfpathlineto{\pgfqpoint{1.508584in}{0.587136in}}%
\pgfpathlineto{\pgfqpoint{1.516066in}{0.587136in}}%
\pgfpathlineto{\pgfqpoint{1.516679in}{0.590832in}}%
\pgfpathlineto{\pgfqpoint{1.528125in}{0.590832in}}%
\pgfpathlineto{\pgfqpoint{1.528125in}{0.592680in}}%
\pgfpathlineto{\pgfqpoint{1.541298in}{0.592680in}}%
\pgfpathlineto{\pgfqpoint{1.541298in}{0.594528in}}%
\pgfpathlineto{\pgfqpoint{1.543085in}{0.594528in}}%
\pgfpathlineto{\pgfqpoint{1.543085in}{0.596376in}}%
\pgfpathlineto{\pgfqpoint{1.548246in}{0.596376in}}%
\pgfpathlineto{\pgfqpoint{1.548246in}{0.598224in}}%
\pgfpathlineto{\pgfqpoint{1.551233in}{0.598224in}}%
\pgfpathlineto{\pgfqpoint{1.551233in}{0.600072in}}%
\pgfpathlineto{\pgfqpoint{1.561060in}{0.600072in}}%
\pgfpathlineto{\pgfqpoint{1.561060in}{0.601920in}}%
\pgfpathlineto{\pgfqpoint{1.569782in}{0.601920in}}%
\pgfpathlineto{\pgfqpoint{1.569782in}{0.603768in}}%
\pgfpathlineto{\pgfqpoint{1.574095in}{0.603768in}}%
\pgfpathlineto{\pgfqpoint{1.574095in}{0.605616in}}%
\pgfpathlineto{\pgfqpoint{1.580802in}{0.605616in}}%
\pgfpathlineto{\pgfqpoint{1.580802in}{0.607464in}}%
\pgfpathlineto{\pgfqpoint{1.583799in}{0.607464in}}%
\pgfpathlineto{\pgfqpoint{1.583799in}{0.609312in}}%
\pgfpathlineto{\pgfqpoint{1.587550in}{0.609312in}}%
\pgfpathlineto{\pgfqpoint{1.587550in}{0.611160in}}%
\pgfpathlineto{\pgfqpoint{1.591717in}{0.611160in}}%
\pgfpathlineto{\pgfqpoint{1.591717in}{0.613008in}}%
\pgfpathlineto{\pgfqpoint{1.600385in}{0.613008in}}%
\pgfpathlineto{\pgfqpoint{1.600385in}{0.614856in}}%
\pgfpathlineto{\pgfqpoint{1.606864in}{0.614856in}}%
\pgfpathlineto{\pgfqpoint{1.606864in}{0.616704in}}%
\pgfpathlineto{\pgfqpoint{1.609762in}{0.616704in}}%
\pgfpathlineto{\pgfqpoint{1.609762in}{0.618552in}}%
\pgfpathlineto{\pgfqpoint{1.613815in}{0.618552in}}%
\pgfpathlineto{\pgfqpoint{1.613815in}{0.620400in}}%
\pgfpathlineto{\pgfqpoint{1.616653in}{0.620400in}}%
\pgfpathlineto{\pgfqpoint{1.616653in}{0.622248in}}%
\pgfpathlineto{\pgfqpoint{1.618232in}{0.622248in}}%
\pgfpathlineto{\pgfqpoint{1.618232in}{0.624096in}}%
\pgfpathlineto{\pgfqpoint{1.620378in}{0.624096in}}%
\pgfpathlineto{\pgfqpoint{1.620378in}{0.625944in}}%
\pgfpathlineto{\pgfqpoint{1.633117in}{0.625944in}}%
\pgfpathlineto{\pgfqpoint{1.633117in}{0.627792in}}%
\pgfpathlineto{\pgfqpoint{1.638823in}{0.627792in}}%
\pgfpathlineto{\pgfqpoint{1.638962in}{0.631488in}}%
\pgfpathlineto{\pgfqpoint{1.642831in}{0.631488in}}%
\pgfpathlineto{\pgfqpoint{1.642831in}{0.633336in}}%
\pgfpathlineto{\pgfqpoint{1.646374in}{0.633336in}}%
\pgfpathlineto{\pgfqpoint{1.646374in}{0.635184in}}%
\pgfpathlineto{\pgfqpoint{1.650743in}{0.635184in}}%
\pgfpathlineto{\pgfqpoint{1.650743in}{0.637032in}}%
\pgfpathlineto{\pgfqpoint{1.653558in}{0.637032in}}%
\pgfpathlineto{\pgfqpoint{1.653558in}{0.638880in}}%
\pgfpathlineto{\pgfqpoint{1.663109in}{0.638880in}}%
\pgfpathlineto{\pgfqpoint{1.663109in}{0.640728in}}%
\pgfpathlineto{\pgfqpoint{1.664493in}{0.640728in}}%
\pgfpathlineto{\pgfqpoint{1.664493in}{0.642576in}}%
\pgfpathlineto{\pgfqpoint{1.670938in}{0.642576in}}%
\pgfpathlineto{\pgfqpoint{1.670938in}{0.644424in}}%
\pgfpathlineto{\pgfqpoint{1.672652in}{0.644424in}}%
\pgfpathlineto{\pgfqpoint{1.672652in}{0.646272in}}%
\pgfpathlineto{\pgfqpoint{1.675089in}{0.646272in}}%
\pgfpathlineto{\pgfqpoint{1.675089in}{0.648120in}}%
\pgfpathlineto{\pgfqpoint{1.679577in}{0.648120in}}%
\pgfpathlineto{\pgfqpoint{1.679577in}{0.649968in}}%
\pgfpathlineto{\pgfqpoint{1.682858in}{0.649968in}}%
\pgfpathlineto{\pgfqpoint{1.682858in}{0.651816in}}%
\pgfpathlineto{\pgfqpoint{1.688932in}{0.651816in}}%
\pgfpathlineto{\pgfqpoint{1.688932in}{0.653664in}}%
\pgfpathlineto{\pgfqpoint{1.696625in}{0.653664in}}%
\pgfpathlineto{\pgfqpoint{1.696958in}{0.657360in}}%
\pgfpathlineto{\pgfqpoint{1.704892in}{0.657360in}}%
\pgfpathlineto{\pgfqpoint{1.704892in}{0.659208in}}%
\pgfpathlineto{\pgfqpoint{1.707593in}{0.659208in}}%
\pgfpathlineto{\pgfqpoint{1.707593in}{0.661056in}}%
\pgfpathlineto{\pgfqpoint{1.711582in}{0.661056in}}%
\pgfpathlineto{\pgfqpoint{1.711582in}{0.662904in}}%
\pgfpathlineto{\pgfqpoint{1.714584in}{0.662904in}}%
\pgfpathlineto{\pgfqpoint{1.714584in}{0.664752in}}%
\pgfpathlineto{\pgfqpoint{1.725232in}{0.664752in}}%
\pgfpathlineto{\pgfqpoint{1.725232in}{0.666600in}}%
\pgfpathlineto{\pgfqpoint{1.728572in}{0.666600in}}%
\pgfpathlineto{\pgfqpoint{1.728692in}{0.670296in}}%
\pgfpathlineto{\pgfqpoint{1.733447in}{0.670296in}}%
\pgfpathlineto{\pgfqpoint{1.733447in}{0.672144in}}%
\pgfpathlineto{\pgfqpoint{1.736911in}{0.672144in}}%
\pgfpathlineto{\pgfqpoint{1.737885in}{0.675840in}}%
\pgfpathlineto{\pgfqpoint{1.739728in}{0.675840in}}%
\pgfpathlineto{\pgfqpoint{1.739728in}{0.677688in}}%
\pgfpathlineto{\pgfqpoint{1.748102in}{0.677688in}}%
\pgfpathlineto{\pgfqpoint{1.748102in}{0.679536in}}%
\pgfpathlineto{\pgfqpoint{1.756692in}{0.679536in}}%
\pgfpathlineto{\pgfqpoint{1.756692in}{0.681384in}}%
\pgfpathlineto{\pgfqpoint{1.760841in}{0.681384in}}%
\pgfpathlineto{\pgfqpoint{1.760841in}{0.683232in}}%
\pgfpathlineto{\pgfqpoint{1.762903in}{0.683232in}}%
\pgfpathlineto{\pgfqpoint{1.762903in}{0.685080in}}%
\pgfpathlineto{\pgfqpoint{1.765593in}{0.685080in}}%
\pgfpathlineto{\pgfqpoint{1.765593in}{0.686928in}}%
\pgfpathlineto{\pgfqpoint{1.770170in}{0.686928in}}%
\pgfpathlineto{\pgfqpoint{1.770170in}{0.688776in}}%
\pgfpathlineto{\pgfqpoint{1.773470in}{0.688776in}}%
\pgfpathlineto{\pgfqpoint{1.773470in}{0.690624in}}%
\pgfpathlineto{\pgfqpoint{1.782466in}{0.690624in}}%
\pgfpathlineto{\pgfqpoint{1.782852in}{0.694320in}}%
\pgfpathlineto{\pgfqpoint{1.785868in}{0.694320in}}%
\pgfpathlineto{\pgfqpoint{1.786474in}{0.698016in}}%
\pgfpathlineto{\pgfqpoint{1.791524in}{0.698016in}}%
\pgfpathlineto{\pgfqpoint{1.791524in}{0.699864in}}%
\pgfpathlineto{\pgfqpoint{1.798650in}{0.699864in}}%
\pgfpathlineto{\pgfqpoint{1.798650in}{0.701712in}}%
\pgfpathlineto{\pgfqpoint{1.802309in}{0.701712in}}%
\pgfpathlineto{\pgfqpoint{1.802309in}{0.703560in}}%
\pgfpathlineto{\pgfqpoint{1.814560in}{0.703560in}}%
\pgfpathlineto{\pgfqpoint{1.814560in}{0.705408in}}%
\pgfpathlineto{\pgfqpoint{1.817627in}{0.705408in}}%
\pgfpathlineto{\pgfqpoint{1.817989in}{0.709104in}}%
\pgfpathlineto{\pgfqpoint{1.823380in}{0.709104in}}%
\pgfpathlineto{\pgfqpoint{1.823813in}{0.714648in}}%
\pgfpathlineto{\pgfqpoint{1.824571in}{0.714648in}}%
\pgfpathlineto{\pgfqpoint{1.824571in}{0.716496in}}%
\pgfpathlineto{\pgfqpoint{1.827047in}{0.716496in}}%
\pgfpathlineto{\pgfqpoint{1.828136in}{0.722040in}}%
\pgfpathlineto{\pgfqpoint{1.846515in}{0.722040in}}%
\pgfpathlineto{\pgfqpoint{1.846515in}{0.723888in}}%
\pgfpathlineto{\pgfqpoint{1.849015in}{0.723888in}}%
\pgfpathlineto{\pgfqpoint{1.849015in}{0.725736in}}%
\pgfpathlineto{\pgfqpoint{1.852678in}{0.725736in}}%
\pgfpathlineto{\pgfqpoint{1.852943in}{0.729432in}}%
\pgfpathlineto{\pgfqpoint{1.856248in}{0.729432in}}%
\pgfpathlineto{\pgfqpoint{1.856874in}{0.733128in}}%
\pgfpathlineto{\pgfqpoint{1.860313in}{0.733128in}}%
\pgfpathlineto{\pgfqpoint{1.860313in}{0.734976in}}%
\pgfpathlineto{\pgfqpoint{1.861830in}{0.734976in}}%
\pgfpathlineto{\pgfqpoint{1.861830in}{0.736824in}}%
\pgfpathlineto{\pgfqpoint{1.867958in}{0.736824in}}%
\pgfpathlineto{\pgfqpoint{1.867958in}{0.738672in}}%
\pgfpathlineto{\pgfqpoint{1.871762in}{0.738672in}}%
\pgfpathlineto{\pgfqpoint{1.871762in}{0.740520in}}%
\pgfpathlineto{\pgfqpoint{1.874206in}{0.740520in}}%
\pgfpathlineto{\pgfqpoint{1.874206in}{0.742368in}}%
\pgfpathlineto{\pgfqpoint{1.881749in}{0.742368in}}%
\pgfpathlineto{\pgfqpoint{1.881749in}{0.744216in}}%
\pgfpathlineto{\pgfqpoint{1.884274in}{0.744216in}}%
\pgfpathlineto{\pgfqpoint{1.884370in}{0.747912in}}%
\pgfpathlineto{\pgfqpoint{1.888085in}{0.747912in}}%
\pgfpathlineto{\pgfqpoint{1.888085in}{0.749760in}}%
\pgfpathlineto{\pgfqpoint{1.890956in}{0.749760in}}%
\pgfpathlineto{\pgfqpoint{1.890956in}{0.751608in}}%
\pgfpathlineto{\pgfqpoint{1.892084in}{0.751608in}}%
\pgfpathlineto{\pgfqpoint{1.892084in}{0.753456in}}%
\pgfpathlineto{\pgfqpoint{1.893732in}{0.753456in}}%
\pgfpathlineto{\pgfqpoint{1.893732in}{0.755304in}}%
\pgfpathlineto{\pgfqpoint{1.903117in}{0.755304in}}%
\pgfpathlineto{\pgfqpoint{1.903117in}{0.757152in}}%
\pgfpathlineto{\pgfqpoint{1.905526in}{0.757152in}}%
\pgfpathlineto{\pgfqpoint{1.905526in}{0.759000in}}%
\pgfpathlineto{\pgfqpoint{1.913045in}{0.759000in}}%
\pgfpathlineto{\pgfqpoint{1.913085in}{0.762696in}}%
\pgfpathlineto{\pgfqpoint{1.915178in}{0.762696in}}%
\pgfpathlineto{\pgfqpoint{1.915178in}{0.764544in}}%
\pgfpathlineto{\pgfqpoint{1.917448in}{0.764544in}}%
\pgfpathlineto{\pgfqpoint{1.917448in}{0.766392in}}%
\pgfpathlineto{\pgfqpoint{1.918907in}{0.766392in}}%
\pgfpathlineto{\pgfqpoint{1.918907in}{0.768240in}}%
\pgfpathlineto{\pgfqpoint{1.920963in}{0.768240in}}%
\pgfpathlineto{\pgfqpoint{1.920963in}{0.770088in}}%
\pgfpathlineto{\pgfqpoint{1.922754in}{0.770088in}}%
\pgfpathlineto{\pgfqpoint{1.922754in}{0.771936in}}%
\pgfpathlineto{\pgfqpoint{1.924755in}{0.771936in}}%
\pgfpathlineto{\pgfqpoint{1.924755in}{0.773784in}}%
\pgfpathlineto{\pgfqpoint{1.928942in}{0.773784in}}%
\pgfpathlineto{\pgfqpoint{1.928942in}{0.775632in}}%
\pgfpathlineto{\pgfqpoint{1.936963in}{0.775632in}}%
\pgfpathlineto{\pgfqpoint{1.936963in}{0.777480in}}%
\pgfpathlineto{\pgfqpoint{1.938857in}{0.777480in}}%
\pgfpathlineto{\pgfqpoint{1.938857in}{0.779328in}}%
\pgfpathlineto{\pgfqpoint{1.944065in}{0.779328in}}%
\pgfpathlineto{\pgfqpoint{1.944120in}{0.783024in}}%
\pgfpathlineto{\pgfqpoint{1.946125in}{0.783024in}}%
\pgfpathlineto{\pgfqpoint{1.946125in}{0.784872in}}%
\pgfpathlineto{\pgfqpoint{1.948935in}{0.784872in}}%
\pgfpathlineto{\pgfqpoint{1.948935in}{0.786720in}}%
\pgfpathlineto{\pgfqpoint{1.950235in}{0.786720in}}%
\pgfpathlineto{\pgfqpoint{1.950235in}{0.788568in}}%
\pgfpathlineto{\pgfqpoint{1.953944in}{0.788568in}}%
\pgfpathlineto{\pgfqpoint{1.953944in}{0.790416in}}%
\pgfpathlineto{\pgfqpoint{1.955833in}{0.790416in}}%
\pgfpathlineto{\pgfqpoint{1.955833in}{0.792264in}}%
\pgfpathlineto{\pgfqpoint{1.962074in}{0.792264in}}%
\pgfpathlineto{\pgfqpoint{1.962074in}{0.794112in}}%
\pgfpathlineto{\pgfqpoint{1.969298in}{0.794112in}}%
\pgfpathlineto{\pgfqpoint{1.970338in}{0.799656in}}%
\pgfpathlineto{\pgfqpoint{1.975057in}{0.799656in}}%
\pgfpathlineto{\pgfqpoint{1.975888in}{0.803352in}}%
\pgfpathlineto{\pgfqpoint{1.976973in}{0.803352in}}%
\pgfpathlineto{\pgfqpoint{1.976973in}{0.805200in}}%
\pgfpathlineto{\pgfqpoint{1.978520in}{0.805200in}}%
\pgfpathlineto{\pgfqpoint{1.978520in}{0.807048in}}%
\pgfpathlineto{\pgfqpoint{1.980476in}{0.807048in}}%
\pgfpathlineto{\pgfqpoint{1.980476in}{0.808896in}}%
\pgfpathlineto{\pgfqpoint{1.986978in}{0.808896in}}%
\pgfpathlineto{\pgfqpoint{1.986978in}{0.810744in}}%
\pgfpathlineto{\pgfqpoint{1.993251in}{0.810744in}}%
\pgfpathlineto{\pgfqpoint{1.993251in}{0.812592in}}%
\pgfpathlineto{\pgfqpoint{2.000105in}{0.812592in}}%
\pgfpathlineto{\pgfqpoint{2.000433in}{0.816288in}}%
\pgfpathlineto{\pgfqpoint{2.001254in}{0.816288in}}%
\pgfpathlineto{\pgfqpoint{2.001838in}{0.819984in}}%
\pgfpathlineto{\pgfqpoint{2.006456in}{0.819984in}}%
\pgfpathlineto{\pgfqpoint{2.006685in}{0.823680in}}%
\pgfpathlineto{\pgfqpoint{2.008000in}{0.823680in}}%
\pgfpathlineto{\pgfqpoint{2.008000in}{0.825528in}}%
\pgfpathlineto{\pgfqpoint{2.009715in}{0.825528in}}%
\pgfpathlineto{\pgfqpoint{2.009715in}{0.827376in}}%
\pgfpathlineto{\pgfqpoint{2.011785in}{0.827376in}}%
\pgfpathlineto{\pgfqpoint{2.011785in}{0.829224in}}%
\pgfpathlineto{\pgfqpoint{2.014230in}{0.829224in}}%
\pgfpathlineto{\pgfqpoint{2.014230in}{0.831072in}}%
\pgfpathlineto{\pgfqpoint{2.024132in}{0.831072in}}%
\pgfpathlineto{\pgfqpoint{2.024132in}{0.832920in}}%
\pgfpathlineto{\pgfqpoint{2.026846in}{0.832920in}}%
\pgfpathlineto{\pgfqpoint{2.027235in}{0.836616in}}%
\pgfpathlineto{\pgfqpoint{2.030814in}{0.836616in}}%
\pgfpathlineto{\pgfqpoint{2.030814in}{0.838464in}}%
\pgfpathlineto{\pgfqpoint{2.033050in}{0.838464in}}%
\pgfpathlineto{\pgfqpoint{2.033493in}{0.842160in}}%
\pgfpathlineto{\pgfqpoint{2.034662in}{0.842160in}}%
\pgfpathlineto{\pgfqpoint{2.034662in}{0.844008in}}%
\pgfpathlineto{\pgfqpoint{2.037330in}{0.844008in}}%
\pgfpathlineto{\pgfqpoint{2.037646in}{0.847704in}}%
\pgfpathlineto{\pgfqpoint{2.042362in}{0.847704in}}%
\pgfpathlineto{\pgfqpoint{2.042362in}{0.849552in}}%
\pgfpathlineto{\pgfqpoint{2.045386in}{0.849552in}}%
\pgfpathlineto{\pgfqpoint{2.045386in}{0.851400in}}%
\pgfpathlineto{\pgfqpoint{2.049287in}{0.851400in}}%
\pgfpathlineto{\pgfqpoint{2.049287in}{0.853248in}}%
\pgfpathlineto{\pgfqpoint{2.055944in}{0.853248in}}%
\pgfpathlineto{\pgfqpoint{2.055944in}{0.855096in}}%
\pgfpathlineto{\pgfqpoint{2.057960in}{0.855096in}}%
\pgfpathlineto{\pgfqpoint{2.058550in}{0.858792in}}%
\pgfpathlineto{\pgfqpoint{2.059743in}{0.858792in}}%
\pgfpathlineto{\pgfqpoint{2.059743in}{0.860640in}}%
\pgfpathlineto{\pgfqpoint{2.062524in}{0.860640in}}%
\pgfpathlineto{\pgfqpoint{2.063443in}{0.866184in}}%
\pgfpathlineto{\pgfqpoint{2.065579in}{0.866184in}}%
\pgfpathlineto{\pgfqpoint{2.065579in}{0.868032in}}%
\pgfpathlineto{\pgfqpoint{2.070669in}{0.868032in}}%
\pgfpathlineto{\pgfqpoint{2.070669in}{0.869880in}}%
\pgfpathlineto{\pgfqpoint{2.073165in}{0.869880in}}%
\pgfpathlineto{\pgfqpoint{2.073165in}{0.871728in}}%
\pgfpathlineto{\pgfqpoint{2.076220in}{0.871728in}}%
\pgfpathlineto{\pgfqpoint{2.076748in}{0.875424in}}%
\pgfpathlineto{\pgfqpoint{2.080445in}{0.875424in}}%
\pgfpathlineto{\pgfqpoint{2.080445in}{0.877272in}}%
\pgfpathlineto{\pgfqpoint{2.086546in}{0.877272in}}%
\pgfpathlineto{\pgfqpoint{2.086546in}{0.879120in}}%
\pgfpathlineto{\pgfqpoint{2.089145in}{0.879120in}}%
\pgfpathlineto{\pgfqpoint{2.089611in}{0.882816in}}%
\pgfpathlineto{\pgfqpoint{2.090612in}{0.882816in}}%
\pgfpathlineto{\pgfqpoint{2.090612in}{0.884664in}}%
\pgfpathlineto{\pgfqpoint{2.093448in}{0.884664in}}%
\pgfpathlineto{\pgfqpoint{2.093628in}{0.888360in}}%
\pgfpathlineto{\pgfqpoint{2.096513in}{0.888360in}}%
\pgfpathlineto{\pgfqpoint{2.096513in}{0.890208in}}%
\pgfpathlineto{\pgfqpoint{2.098166in}{0.890208in}}%
\pgfpathlineto{\pgfqpoint{2.098166in}{0.892056in}}%
\pgfpathlineto{\pgfqpoint{2.101875in}{0.892056in}}%
\pgfpathlineto{\pgfqpoint{2.101875in}{0.893904in}}%
\pgfpathlineto{\pgfqpoint{2.109275in}{0.893904in}}%
\pgfpathlineto{\pgfqpoint{2.109275in}{0.895752in}}%
\pgfpathlineto{\pgfqpoint{2.111127in}{0.895752in}}%
\pgfpathlineto{\pgfqpoint{2.111127in}{0.897600in}}%
\pgfpathlineto{\pgfqpoint{2.114470in}{0.897600in}}%
\pgfpathlineto{\pgfqpoint{2.114470in}{0.899448in}}%
\pgfpathlineto{\pgfqpoint{2.116966in}{0.899448in}}%
\pgfpathlineto{\pgfqpoint{2.117224in}{0.903144in}}%
\pgfpathlineto{\pgfqpoint{2.121267in}{0.903144in}}%
\pgfpathlineto{\pgfqpoint{2.121842in}{0.906840in}}%
\pgfpathlineto{\pgfqpoint{2.123166in}{0.906840in}}%
\pgfpathlineto{\pgfqpoint{2.124228in}{0.912384in}}%
\pgfpathlineto{\pgfqpoint{2.124768in}{0.912384in}}%
\pgfpathlineto{\pgfqpoint{2.124768in}{0.914232in}}%
\pgfpathlineto{\pgfqpoint{2.128897in}{0.914232in}}%
\pgfpathlineto{\pgfqpoint{2.128897in}{0.916080in}}%
\pgfpathlineto{\pgfqpoint{2.132801in}{0.916080in}}%
\pgfpathlineto{\pgfqpoint{2.133812in}{0.919776in}}%
\pgfpathlineto{\pgfqpoint{2.135722in}{0.919776in}}%
\pgfpathlineto{\pgfqpoint{2.135722in}{0.921624in}}%
\pgfpathlineto{\pgfqpoint{2.142243in}{0.921624in}}%
\pgfpathlineto{\pgfqpoint{2.142243in}{0.923472in}}%
\pgfpathlineto{\pgfqpoint{2.145414in}{0.923472in}}%
\pgfpathlineto{\pgfqpoint{2.145426in}{0.927168in}}%
\pgfpathlineto{\pgfqpoint{2.147193in}{0.927168in}}%
\pgfpathlineto{\pgfqpoint{2.147408in}{0.930864in}}%
\pgfpathlineto{\pgfqpoint{2.149043in}{0.930864in}}%
\pgfpathlineto{\pgfqpoint{2.149117in}{0.934560in}}%
\pgfpathlineto{\pgfqpoint{2.152227in}{0.934560in}}%
\pgfpathlineto{\pgfqpoint{2.152227in}{0.936408in}}%
\pgfpathlineto{\pgfqpoint{2.153805in}{0.936408in}}%
\pgfpathlineto{\pgfqpoint{2.153805in}{0.938256in}}%
\pgfpathlineto{\pgfqpoint{2.155000in}{0.938256in}}%
\pgfpathlineto{\pgfqpoint{2.155000in}{0.940104in}}%
\pgfpathlineto{\pgfqpoint{2.157560in}{0.940104in}}%
\pgfpathlineto{\pgfqpoint{2.157560in}{0.941952in}}%
\pgfpathlineto{\pgfqpoint{2.159148in}{0.941952in}}%
\pgfpathlineto{\pgfqpoint{2.159148in}{0.943800in}}%
\pgfpathlineto{\pgfqpoint{2.164023in}{0.943800in}}%
\pgfpathlineto{\pgfqpoint{2.164023in}{0.945648in}}%
\pgfpathlineto{\pgfqpoint{2.166260in}{0.945648in}}%
\pgfpathlineto{\pgfqpoint{2.166260in}{0.947496in}}%
\pgfpathlineto{\pgfqpoint{2.170423in}{0.947496in}}%
\pgfpathlineto{\pgfqpoint{2.170423in}{0.949344in}}%
\pgfpathlineto{\pgfqpoint{2.171624in}{0.949344in}}%
\pgfpathlineto{\pgfqpoint{2.171624in}{0.951192in}}%
\pgfpathlineto{\pgfqpoint{2.172982in}{0.951192in}}%
\pgfpathlineto{\pgfqpoint{2.172982in}{0.953040in}}%
\pgfpathlineto{\pgfqpoint{2.175728in}{0.953040in}}%
\pgfpathlineto{\pgfqpoint{2.175728in}{0.954888in}}%
\pgfpathlineto{\pgfqpoint{2.177695in}{0.954888in}}%
\pgfpathlineto{\pgfqpoint{2.177695in}{0.956736in}}%
\pgfpathlineto{\pgfqpoint{2.179426in}{0.956736in}}%
\pgfpathlineto{\pgfqpoint{2.179578in}{0.962280in}}%
\pgfpathlineto{\pgfqpoint{2.182448in}{0.962280in}}%
\pgfpathlineto{\pgfqpoint{2.182503in}{0.965976in}}%
\pgfpathlineto{\pgfqpoint{2.183986in}{0.965976in}}%
\pgfpathlineto{\pgfqpoint{2.184087in}{0.969672in}}%
\pgfpathlineto{\pgfqpoint{2.188169in}{0.969672in}}%
\pgfpathlineto{\pgfqpoint{2.188169in}{0.971520in}}%
\pgfpathlineto{\pgfqpoint{2.194195in}{0.971520in}}%
\pgfpathlineto{\pgfqpoint{2.194195in}{0.973368in}}%
\pgfpathlineto{\pgfqpoint{2.196819in}{0.973368in}}%
\pgfpathlineto{\pgfqpoint{2.196819in}{0.975216in}}%
\pgfpathlineto{\pgfqpoint{2.200228in}{0.975216in}}%
\pgfpathlineto{\pgfqpoint{2.200228in}{0.977064in}}%
\pgfpathlineto{\pgfqpoint{2.202239in}{0.977064in}}%
\pgfpathlineto{\pgfqpoint{2.203232in}{0.980760in}}%
\pgfpathlineto{\pgfqpoint{2.205790in}{0.980760in}}%
\pgfpathlineto{\pgfqpoint{2.205790in}{0.982608in}}%
\pgfpathlineto{\pgfqpoint{2.207398in}{0.982608in}}%
\pgfpathlineto{\pgfqpoint{2.207914in}{0.986304in}}%
\pgfpathlineto{\pgfqpoint{2.209900in}{0.986304in}}%
\pgfpathlineto{\pgfqpoint{2.210023in}{0.991848in}}%
\pgfpathlineto{\pgfqpoint{2.213082in}{0.991848in}}%
\pgfpathlineto{\pgfqpoint{2.214049in}{0.995544in}}%
\pgfpathlineto{\pgfqpoint{2.214339in}{0.995544in}}%
\pgfpathlineto{\pgfqpoint{2.214339in}{0.997392in}}%
\pgfpathlineto{\pgfqpoint{2.218459in}{0.997392in}}%
\pgfpathlineto{\pgfqpoint{2.218754in}{1.001088in}}%
\pgfpathlineto{\pgfqpoint{2.221218in}{1.001088in}}%
\pgfpathlineto{\pgfqpoint{2.221218in}{1.002936in}}%
\pgfpathlineto{\pgfqpoint{2.223995in}{1.002936in}}%
\pgfpathlineto{\pgfqpoint{2.223995in}{1.004784in}}%
\pgfpathlineto{\pgfqpoint{2.227714in}{1.004784in}}%
\pgfpathlineto{\pgfqpoint{2.227714in}{1.006632in}}%
\pgfpathlineto{\pgfqpoint{2.230738in}{1.006632in}}%
\pgfpathlineto{\pgfqpoint{2.230738in}{1.008480in}}%
\pgfpathlineto{\pgfqpoint{2.232159in}{1.008480in}}%
\pgfpathlineto{\pgfqpoint{2.233030in}{1.014024in}}%
\pgfpathlineto{\pgfqpoint{2.234166in}{1.014024in}}%
\pgfpathlineto{\pgfqpoint{2.234575in}{1.017720in}}%
\pgfpathlineto{\pgfqpoint{2.238044in}{1.017720in}}%
\pgfpathlineto{\pgfqpoint{2.238725in}{1.021416in}}%
\pgfpathlineto{\pgfqpoint{2.240005in}{1.021416in}}%
\pgfpathlineto{\pgfqpoint{2.240005in}{1.023264in}}%
\pgfpathlineto{\pgfqpoint{2.243116in}{1.023264in}}%
\pgfpathlineto{\pgfqpoint{2.244153in}{1.028808in}}%
\pgfpathlineto{\pgfqpoint{2.250904in}{1.028808in}}%
\pgfpathlineto{\pgfqpoint{2.251178in}{1.032504in}}%
\pgfpathlineto{\pgfqpoint{2.255232in}{1.032504in}}%
\pgfpathlineto{\pgfqpoint{2.255232in}{1.034352in}}%
\pgfpathlineto{\pgfqpoint{2.258039in}{1.034352in}}%
\pgfpathlineto{\pgfqpoint{2.258039in}{1.036200in}}%
\pgfpathlineto{\pgfqpoint{2.261948in}{1.036200in}}%
\pgfpathlineto{\pgfqpoint{2.262937in}{1.039896in}}%
\pgfpathlineto{\pgfqpoint{2.264136in}{1.039896in}}%
\pgfpathlineto{\pgfqpoint{2.265135in}{1.045440in}}%
\pgfpathlineto{\pgfqpoint{2.267827in}{1.045440in}}%
\pgfpathlineto{\pgfqpoint{2.268685in}{1.052832in}}%
\pgfpathlineto{\pgfqpoint{2.273129in}{1.052832in}}%
\pgfpathlineto{\pgfqpoint{2.273129in}{1.054680in}}%
\pgfpathlineto{\pgfqpoint{2.275081in}{1.054680in}}%
\pgfpathlineto{\pgfqpoint{2.275081in}{1.056528in}}%
\pgfpathlineto{\pgfqpoint{2.281122in}{1.056528in}}%
\pgfpathlineto{\pgfqpoint{2.281122in}{1.058376in}}%
\pgfpathlineto{\pgfqpoint{2.284853in}{1.058376in}}%
\pgfpathlineto{\pgfqpoint{2.284853in}{1.060224in}}%
\pgfpathlineto{\pgfqpoint{2.287042in}{1.060224in}}%
\pgfpathlineto{\pgfqpoint{2.287042in}{1.062072in}}%
\pgfpathlineto{\pgfqpoint{2.288647in}{1.062072in}}%
\pgfpathlineto{\pgfqpoint{2.289576in}{1.065768in}}%
\pgfpathlineto{\pgfqpoint{2.291684in}{1.065768in}}%
\pgfpathlineto{\pgfqpoint{2.291684in}{1.067616in}}%
\pgfpathlineto{\pgfqpoint{2.292891in}{1.067616in}}%
\pgfpathlineto{\pgfqpoint{2.293098in}{1.071312in}}%
\pgfpathlineto{\pgfqpoint{2.294011in}{1.071312in}}%
\pgfpathlineto{\pgfqpoint{2.294380in}{1.075008in}}%
\pgfpathlineto{\pgfqpoint{2.298126in}{1.075008in}}%
\pgfpathlineto{\pgfqpoint{2.298980in}{1.080552in}}%
\pgfpathlineto{\pgfqpoint{2.303582in}{1.080552in}}%
\pgfpathlineto{\pgfqpoint{2.303582in}{1.082400in}}%
\pgfpathlineto{\pgfqpoint{2.305306in}{1.082400in}}%
\pgfpathlineto{\pgfqpoint{2.305770in}{1.086096in}}%
\pgfpathlineto{\pgfqpoint{2.313484in}{1.086096in}}%
\pgfpathlineto{\pgfqpoint{2.313484in}{1.087944in}}%
\pgfpathlineto{\pgfqpoint{2.314699in}{1.087944in}}%
\pgfpathlineto{\pgfqpoint{2.315495in}{1.091640in}}%
\pgfpathlineto{\pgfqpoint{2.317388in}{1.091640in}}%
\pgfpathlineto{\pgfqpoint{2.318267in}{1.097184in}}%
\pgfpathlineto{\pgfqpoint{2.319813in}{1.097184in}}%
\pgfpathlineto{\pgfqpoint{2.319813in}{1.099032in}}%
\pgfpathlineto{\pgfqpoint{2.322707in}{1.099032in}}%
\pgfpathlineto{\pgfqpoint{2.322995in}{1.106424in}}%
\pgfpathlineto{\pgfqpoint{2.327879in}{1.106424in}}%
\pgfpathlineto{\pgfqpoint{2.327993in}{1.110120in}}%
\pgfpathlineto{\pgfqpoint{2.335385in}{1.110120in}}%
\pgfpathlineto{\pgfqpoint{2.335853in}{1.113816in}}%
\pgfpathlineto{\pgfqpoint{2.339402in}{1.113816in}}%
\pgfpathlineto{\pgfqpoint{2.339402in}{1.115664in}}%
\pgfpathlineto{\pgfqpoint{2.343197in}{1.115664in}}%
\pgfpathlineto{\pgfqpoint{2.343415in}{1.119360in}}%
\pgfpathlineto{\pgfqpoint{2.345563in}{1.119360in}}%
\pgfpathlineto{\pgfqpoint{2.346209in}{1.123056in}}%
\pgfpathlineto{\pgfqpoint{2.347426in}{1.123056in}}%
\pgfpathlineto{\pgfqpoint{2.347526in}{1.126752in}}%
\pgfpathlineto{\pgfqpoint{2.350149in}{1.126752in}}%
\pgfpathlineto{\pgfqpoint{2.350188in}{1.130448in}}%
\pgfpathlineto{\pgfqpoint{2.352987in}{1.130448in}}%
\pgfpathlineto{\pgfqpoint{2.353539in}{1.134144in}}%
\pgfpathlineto{\pgfqpoint{2.357688in}{1.134144in}}%
\pgfpathlineto{\pgfqpoint{2.357879in}{1.137840in}}%
\pgfpathlineto{\pgfqpoint{2.359321in}{1.137840in}}%
\pgfpathlineto{\pgfqpoint{2.359321in}{1.139688in}}%
\pgfpathlineto{\pgfqpoint{2.365406in}{1.139688in}}%
\pgfpathlineto{\pgfqpoint{2.365406in}{1.141536in}}%
\pgfpathlineto{\pgfqpoint{2.367468in}{1.141536in}}%
\pgfpathlineto{\pgfqpoint{2.367468in}{1.143384in}}%
\pgfpathlineto{\pgfqpoint{2.369101in}{1.143384in}}%
\pgfpathlineto{\pgfqpoint{2.369861in}{1.147080in}}%
\pgfpathlineto{\pgfqpoint{2.372928in}{1.147080in}}%
\pgfpathlineto{\pgfqpoint{2.372928in}{1.148928in}}%
\pgfpathlineto{\pgfqpoint{2.374413in}{1.148928in}}%
\pgfpathlineto{\pgfqpoint{2.374413in}{1.150776in}}%
\pgfpathlineto{\pgfqpoint{2.376916in}{1.150776in}}%
\pgfpathlineto{\pgfqpoint{2.377589in}{1.158168in}}%
\pgfpathlineto{\pgfqpoint{2.379555in}{1.158168in}}%
\pgfpathlineto{\pgfqpoint{2.379589in}{1.161864in}}%
\pgfpathlineto{\pgfqpoint{2.382985in}{1.161864in}}%
\pgfpathlineto{\pgfqpoint{2.383294in}{1.165560in}}%
\pgfpathlineto{\pgfqpoint{2.387316in}{1.165560in}}%
\pgfpathlineto{\pgfqpoint{2.387316in}{1.167408in}}%
\pgfpathlineto{\pgfqpoint{2.388823in}{1.167408in}}%
\pgfpathlineto{\pgfqpoint{2.389591in}{1.171104in}}%
\pgfpathlineto{\pgfqpoint{2.396652in}{1.171104in}}%
\pgfpathlineto{\pgfqpoint{2.397060in}{1.174800in}}%
\pgfpathlineto{\pgfqpoint{2.398936in}{1.174800in}}%
\pgfpathlineto{\pgfqpoint{2.399997in}{1.178496in}}%
\pgfpathlineto{\pgfqpoint{2.401293in}{1.178496in}}%
\pgfpathlineto{\pgfqpoint{2.401293in}{1.180344in}}%
\pgfpathlineto{\pgfqpoint{2.403124in}{1.180344in}}%
\pgfpathlineto{\pgfqpoint{2.404081in}{1.185888in}}%
\pgfpathlineto{\pgfqpoint{2.406744in}{1.185888in}}%
\pgfpathlineto{\pgfqpoint{2.406915in}{1.191432in}}%
\pgfpathlineto{\pgfqpoint{2.409202in}{1.191432in}}%
\pgfpathlineto{\pgfqpoint{2.409274in}{1.195128in}}%
\pgfpathlineto{\pgfqpoint{2.410992in}{1.195128in}}%
\pgfpathlineto{\pgfqpoint{2.410992in}{1.196976in}}%
\pgfpathlineto{\pgfqpoint{2.413231in}{1.196976in}}%
\pgfpathlineto{\pgfqpoint{2.413231in}{1.198824in}}%
\pgfpathlineto{\pgfqpoint{2.418454in}{1.198824in}}%
\pgfpathlineto{\pgfqpoint{2.419213in}{1.202520in}}%
\pgfpathlineto{\pgfqpoint{2.423339in}{1.202520in}}%
\pgfpathlineto{\pgfqpoint{2.423339in}{1.204368in}}%
\pgfpathlineto{\pgfqpoint{2.426434in}{1.204368in}}%
\pgfpathlineto{\pgfqpoint{2.427423in}{1.208064in}}%
\pgfpathlineto{\pgfqpoint{2.429249in}{1.208064in}}%
\pgfpathlineto{\pgfqpoint{2.429249in}{1.209912in}}%
\pgfpathlineto{\pgfqpoint{2.430535in}{1.209912in}}%
\pgfpathlineto{\pgfqpoint{2.431287in}{1.213608in}}%
\pgfpathlineto{\pgfqpoint{2.432686in}{1.213608in}}%
\pgfpathlineto{\pgfqpoint{2.433605in}{1.221000in}}%
\pgfpathlineto{\pgfqpoint{2.433943in}{1.221000in}}%
\pgfpathlineto{\pgfqpoint{2.434595in}{1.224696in}}%
\pgfpathlineto{\pgfqpoint{2.436536in}{1.224696in}}%
\pgfpathlineto{\pgfqpoint{2.436806in}{1.228392in}}%
\pgfpathlineto{\pgfqpoint{2.440826in}{1.228392in}}%
\pgfpathlineto{\pgfqpoint{2.440826in}{1.230240in}}%
\pgfpathlineto{\pgfqpoint{2.442555in}{1.230240in}}%
\pgfpathlineto{\pgfqpoint{2.442555in}{1.232088in}}%
\pgfpathlineto{\pgfqpoint{2.448726in}{1.232088in}}%
\pgfpathlineto{\pgfqpoint{2.448726in}{1.233936in}}%
\pgfpathlineto{\pgfqpoint{2.450783in}{1.233936in}}%
\pgfpathlineto{\pgfqpoint{2.450783in}{1.235784in}}%
\pgfpathlineto{\pgfqpoint{2.452979in}{1.235784in}}%
\pgfpathlineto{\pgfqpoint{2.453518in}{1.239480in}}%
\pgfpathlineto{\pgfqpoint{2.454212in}{1.239480in}}%
\pgfpathlineto{\pgfqpoint{2.454212in}{1.241328in}}%
\pgfpathlineto{\pgfqpoint{2.456994in}{1.241328in}}%
\pgfpathlineto{\pgfqpoint{2.458000in}{1.245024in}}%
\pgfpathlineto{\pgfqpoint{2.458389in}{1.245024in}}%
\pgfpathlineto{\pgfqpoint{2.458389in}{1.246872in}}%
\pgfpathlineto{\pgfqpoint{2.459976in}{1.246872in}}%
\pgfpathlineto{\pgfqpoint{2.460921in}{1.250568in}}%
\pgfpathlineto{\pgfqpoint{2.461996in}{1.250568in}}%
\pgfpathlineto{\pgfqpoint{2.462866in}{1.256112in}}%
\pgfpathlineto{\pgfqpoint{2.466093in}{1.256112in}}%
\pgfpathlineto{\pgfqpoint{2.466230in}{1.259808in}}%
\pgfpathlineto{\pgfqpoint{2.470435in}{1.259808in}}%
\pgfpathlineto{\pgfqpoint{2.470435in}{1.261656in}}%
\pgfpathlineto{\pgfqpoint{2.472101in}{1.261656in}}%
\pgfpathlineto{\pgfqpoint{2.472101in}{1.263504in}}%
\pgfpathlineto{\pgfqpoint{2.477986in}{1.263504in}}%
\pgfpathlineto{\pgfqpoint{2.477986in}{1.265352in}}%
\pgfpathlineto{\pgfqpoint{2.482621in}{1.265352in}}%
\pgfpathlineto{\pgfqpoint{2.483622in}{1.269048in}}%
\pgfpathlineto{\pgfqpoint{2.484230in}{1.269048in}}%
\pgfpathlineto{\pgfqpoint{2.484230in}{1.270896in}}%
\pgfpathlineto{\pgfqpoint{2.486103in}{1.270896in}}%
\pgfpathlineto{\pgfqpoint{2.486640in}{1.276440in}}%
\pgfpathlineto{\pgfqpoint{2.487819in}{1.276440in}}%
\pgfpathlineto{\pgfqpoint{2.488104in}{1.280136in}}%
\pgfpathlineto{\pgfqpoint{2.489938in}{1.280136in}}%
\pgfpathlineto{\pgfqpoint{2.490529in}{1.287528in}}%
\pgfpathlineto{\pgfqpoint{2.491819in}{1.287528in}}%
\pgfpathlineto{\pgfqpoint{2.491819in}{1.289376in}}%
\pgfpathlineto{\pgfqpoint{2.493796in}{1.289376in}}%
\pgfpathlineto{\pgfqpoint{2.493796in}{1.291224in}}%
\pgfpathlineto{\pgfqpoint{2.495133in}{1.291224in}}%
\pgfpathlineto{\pgfqpoint{2.495133in}{1.293072in}}%
\pgfpathlineto{\pgfqpoint{2.500934in}{1.293072in}}%
\pgfpathlineto{\pgfqpoint{2.500988in}{1.296768in}}%
\pgfpathlineto{\pgfqpoint{2.507128in}{1.296768in}}%
\pgfpathlineto{\pgfqpoint{2.507128in}{1.298616in}}%
\pgfpathlineto{\pgfqpoint{2.508750in}{1.298616in}}%
\pgfpathlineto{\pgfqpoint{2.508750in}{1.300464in}}%
\pgfpathlineto{\pgfqpoint{2.511761in}{1.300464in}}%
\pgfpathlineto{\pgfqpoint{2.511761in}{1.302312in}}%
\pgfpathlineto{\pgfqpoint{2.513554in}{1.302312in}}%
\pgfpathlineto{\pgfqpoint{2.514140in}{1.306008in}}%
\pgfpathlineto{\pgfqpoint{2.515238in}{1.306008in}}%
\pgfpathlineto{\pgfqpoint{2.515879in}{1.313400in}}%
\pgfpathlineto{\pgfqpoint{2.516822in}{1.313400in}}%
\pgfpathlineto{\pgfqpoint{2.516822in}{1.315248in}}%
\pgfpathlineto{\pgfqpoint{2.518764in}{1.315248in}}%
\pgfpathlineto{\pgfqpoint{2.519061in}{1.318944in}}%
\pgfpathlineto{\pgfqpoint{2.523404in}{1.318944in}}%
\pgfpathlineto{\pgfqpoint{2.523404in}{1.320792in}}%
\pgfpathlineto{\pgfqpoint{2.528827in}{1.320792in}}%
\pgfpathlineto{\pgfqpoint{2.529836in}{1.324488in}}%
\pgfpathlineto{\pgfqpoint{2.536315in}{1.324488in}}%
\pgfpathlineto{\pgfqpoint{2.536315in}{1.326336in}}%
\pgfpathlineto{\pgfqpoint{2.538046in}{1.326336in}}%
\pgfpathlineto{\pgfqpoint{2.538046in}{1.328184in}}%
\pgfpathlineto{\pgfqpoint{2.539235in}{1.328184in}}%
\pgfpathlineto{\pgfqpoint{2.540248in}{1.333728in}}%
\pgfpathlineto{\pgfqpoint{2.541265in}{1.333728in}}%
\pgfpathlineto{\pgfqpoint{2.541960in}{1.339272in}}%
\pgfpathlineto{\pgfqpoint{2.542441in}{1.339272in}}%
\pgfpathlineto{\pgfqpoint{2.543101in}{1.342968in}}%
\pgfpathlineto{\pgfqpoint{2.543773in}{1.342968in}}%
\pgfpathlineto{\pgfqpoint{2.543773in}{1.344816in}}%
\pgfpathlineto{\pgfqpoint{2.545158in}{1.344816in}}%
\pgfpathlineto{\pgfqpoint{2.545158in}{1.346664in}}%
\pgfpathlineto{\pgfqpoint{2.546977in}{1.346664in}}%
\pgfpathlineto{\pgfqpoint{2.547311in}{1.350360in}}%
\pgfpathlineto{\pgfqpoint{2.548199in}{1.350360in}}%
\pgfpathlineto{\pgfqpoint{2.549150in}{1.354056in}}%
\pgfpathlineto{\pgfqpoint{2.551969in}{1.354056in}}%
\pgfpathlineto{\pgfqpoint{2.552531in}{1.357752in}}%
\pgfpathlineto{\pgfqpoint{2.559784in}{1.357752in}}%
\pgfpathlineto{\pgfqpoint{2.559922in}{1.361448in}}%
\pgfpathlineto{\pgfqpoint{2.564630in}{1.361448in}}%
\pgfpathlineto{\pgfqpoint{2.564630in}{1.363296in}}%
\pgfpathlineto{\pgfqpoint{2.565842in}{1.363296in}}%
\pgfpathlineto{\pgfqpoint{2.565842in}{1.365144in}}%
\pgfpathlineto{\pgfqpoint{2.567210in}{1.365144in}}%
\pgfpathlineto{\pgfqpoint{2.567698in}{1.368840in}}%
\pgfpathlineto{\pgfqpoint{2.568486in}{1.368840in}}%
\pgfpathlineto{\pgfqpoint{2.569055in}{1.374384in}}%
\pgfpathlineto{\pgfqpoint{2.570754in}{1.374384in}}%
\pgfpathlineto{\pgfqpoint{2.571186in}{1.379928in}}%
\pgfpathlineto{\pgfqpoint{2.572125in}{1.379928in}}%
\pgfpathlineto{\pgfqpoint{2.572318in}{1.383624in}}%
\pgfpathlineto{\pgfqpoint{2.574243in}{1.383624in}}%
\pgfpathlineto{\pgfqpoint{2.574243in}{1.385472in}}%
\pgfpathlineto{\pgfqpoint{2.576367in}{1.385472in}}%
\pgfpathlineto{\pgfqpoint{2.577439in}{1.389168in}}%
\pgfpathlineto{\pgfqpoint{2.577802in}{1.389168in}}%
\pgfpathlineto{\pgfqpoint{2.577802in}{1.391016in}}%
\pgfpathlineto{\pgfqpoint{2.583498in}{1.391016in}}%
\pgfpathlineto{\pgfqpoint{2.583498in}{1.392864in}}%
\pgfpathlineto{\pgfqpoint{2.588806in}{1.392864in}}%
\pgfpathlineto{\pgfqpoint{2.589010in}{1.396560in}}%
\pgfpathlineto{\pgfqpoint{2.590601in}{1.396560in}}%
\pgfpathlineto{\pgfqpoint{2.590601in}{1.398408in}}%
\pgfpathlineto{\pgfqpoint{2.592087in}{1.398408in}}%
\pgfpathlineto{\pgfqpoint{2.592087in}{1.400256in}}%
\pgfpathlineto{\pgfqpoint{2.594237in}{1.400256in}}%
\pgfpathlineto{\pgfqpoint{2.595230in}{1.403952in}}%
\pgfpathlineto{\pgfqpoint{2.596806in}{1.403952in}}%
\pgfpathlineto{\pgfqpoint{2.597837in}{1.409496in}}%
\pgfpathlineto{\pgfqpoint{2.598068in}{1.409496in}}%
\pgfpathlineto{\pgfqpoint{2.598068in}{1.411344in}}%
\pgfpathlineto{\pgfqpoint{2.600544in}{1.411344in}}%
\pgfpathlineto{\pgfqpoint{2.601208in}{1.420584in}}%
\pgfpathlineto{\pgfqpoint{2.604990in}{1.420584in}}%
\pgfpathlineto{\pgfqpoint{2.606011in}{1.424280in}}%
\pgfpathlineto{\pgfqpoint{2.606995in}{1.424280in}}%
\pgfpathlineto{\pgfqpoint{2.606995in}{1.426128in}}%
\pgfpathlineto{\pgfqpoint{2.613163in}{1.426128in}}%
\pgfpathlineto{\pgfqpoint{2.613163in}{1.427976in}}%
\pgfpathlineto{\pgfqpoint{2.617519in}{1.427976in}}%
\pgfpathlineto{\pgfqpoint{2.617626in}{1.431672in}}%
\pgfpathlineto{\pgfqpoint{2.621310in}{1.431672in}}%
\pgfpathlineto{\pgfqpoint{2.621393in}{1.435368in}}%
\pgfpathlineto{\pgfqpoint{2.622683in}{1.435368in}}%
\pgfpathlineto{\pgfqpoint{2.623069in}{1.439064in}}%
\pgfpathlineto{\pgfqpoint{2.623824in}{1.439064in}}%
\pgfpathlineto{\pgfqpoint{2.624119in}{1.444608in}}%
\pgfpathlineto{\pgfqpoint{2.625572in}{1.444608in}}%
\pgfpathlineto{\pgfqpoint{2.625572in}{1.446456in}}%
\pgfpathlineto{\pgfqpoint{2.626840in}{1.446456in}}%
\pgfpathlineto{\pgfqpoint{2.626840in}{1.448304in}}%
\pgfpathlineto{\pgfqpoint{2.629539in}{1.448304in}}%
\pgfpathlineto{\pgfqpoint{2.629892in}{1.453848in}}%
\pgfpathlineto{\pgfqpoint{2.633389in}{1.453848in}}%
\pgfpathlineto{\pgfqpoint{2.633389in}{1.455696in}}%
\pgfpathlineto{\pgfqpoint{2.635028in}{1.455696in}}%
\pgfpathlineto{\pgfqpoint{2.635860in}{1.459392in}}%
\pgfpathlineto{\pgfqpoint{2.637459in}{1.459392in}}%
\pgfpathlineto{\pgfqpoint{2.637459in}{1.461240in}}%
\pgfpathlineto{\pgfqpoint{2.640973in}{1.461240in}}%
\pgfpathlineto{\pgfqpoint{2.640973in}{1.463088in}}%
\pgfpathlineto{\pgfqpoint{2.642782in}{1.463088in}}%
\pgfpathlineto{\pgfqpoint{2.643034in}{1.466784in}}%
\pgfpathlineto{\pgfqpoint{2.646959in}{1.466784in}}%
\pgfpathlineto{\pgfqpoint{2.646959in}{1.468632in}}%
\pgfpathlineto{\pgfqpoint{2.649191in}{1.468632in}}%
\pgfpathlineto{\pgfqpoint{2.649948in}{1.472328in}}%
\pgfpathlineto{\pgfqpoint{2.650354in}{1.472328in}}%
\pgfpathlineto{\pgfqpoint{2.650354in}{1.474176in}}%
\pgfpathlineto{\pgfqpoint{2.651512in}{1.474176in}}%
\pgfpathlineto{\pgfqpoint{2.652585in}{1.477872in}}%
\pgfpathlineto{\pgfqpoint{2.652936in}{1.477872in}}%
\pgfpathlineto{\pgfqpoint{2.652949in}{1.481568in}}%
\pgfpathlineto{\pgfqpoint{2.655170in}{1.481568in}}%
\pgfpathlineto{\pgfqpoint{2.655170in}{1.483416in}}%
\pgfpathlineto{\pgfqpoint{2.658139in}{1.483416in}}%
\pgfpathlineto{\pgfqpoint{2.658870in}{1.490808in}}%
\pgfpathlineto{\pgfqpoint{2.663443in}{1.490808in}}%
\pgfpathlineto{\pgfqpoint{2.663443in}{1.492656in}}%
\pgfpathlineto{\pgfqpoint{2.664616in}{1.492656in}}%
\pgfpathlineto{\pgfqpoint{2.664616in}{1.494504in}}%
\pgfpathlineto{\pgfqpoint{2.666353in}{1.494504in}}%
\pgfpathlineto{\pgfqpoint{2.666353in}{1.496352in}}%
\pgfpathlineto{\pgfqpoint{2.669753in}{1.496352in}}%
\pgfpathlineto{\pgfqpoint{2.670178in}{1.500048in}}%
\pgfpathlineto{\pgfqpoint{2.671638in}{1.500048in}}%
\pgfpathlineto{\pgfqpoint{2.671786in}{1.503744in}}%
\pgfpathlineto{\pgfqpoint{2.673365in}{1.503744in}}%
\pgfpathlineto{\pgfqpoint{2.673365in}{1.505592in}}%
\pgfpathlineto{\pgfqpoint{2.674579in}{1.505592in}}%
\pgfpathlineto{\pgfqpoint{2.674579in}{1.507440in}}%
\pgfpathlineto{\pgfqpoint{2.675800in}{1.507440in}}%
\pgfpathlineto{\pgfqpoint{2.676422in}{1.511136in}}%
\pgfpathlineto{\pgfqpoint{2.678063in}{1.511136in}}%
\pgfpathlineto{\pgfqpoint{2.678591in}{1.516680in}}%
\pgfpathlineto{\pgfqpoint{2.681430in}{1.516680in}}%
\pgfpathlineto{\pgfqpoint{2.682388in}{1.525920in}}%
\pgfpathlineto{\pgfqpoint{2.687642in}{1.525920in}}%
\pgfpathlineto{\pgfqpoint{2.687642in}{1.527768in}}%
\pgfpathlineto{\pgfqpoint{2.691729in}{1.527768in}}%
\pgfpathlineto{\pgfqpoint{2.691729in}{1.529616in}}%
\pgfpathlineto{\pgfqpoint{2.694366in}{1.529616in}}%
\pgfpathlineto{\pgfqpoint{2.695104in}{1.535160in}}%
\pgfpathlineto{\pgfqpoint{2.698962in}{1.535160in}}%
\pgfpathlineto{\pgfqpoint{2.698962in}{1.537008in}}%
\pgfpathlineto{\pgfqpoint{2.701237in}{1.537008in}}%
\pgfpathlineto{\pgfqpoint{2.701956in}{1.540704in}}%
\pgfpathlineto{\pgfqpoint{2.702793in}{1.540704in}}%
\pgfpathlineto{\pgfqpoint{2.703429in}{1.544400in}}%
\pgfpathlineto{\pgfqpoint{2.704311in}{1.544400in}}%
\pgfpathlineto{\pgfqpoint{2.705131in}{1.549944in}}%
\pgfpathlineto{\pgfqpoint{2.706277in}{1.549944in}}%
\pgfpathlineto{\pgfqpoint{2.707268in}{1.555488in}}%
\pgfpathlineto{\pgfqpoint{2.709739in}{1.555488in}}%
\pgfpathlineto{\pgfqpoint{2.710353in}{1.559184in}}%
\pgfpathlineto{\pgfqpoint{2.710933in}{1.559184in}}%
\pgfpathlineto{\pgfqpoint{2.710933in}{1.561032in}}%
\pgfpathlineto{\pgfqpoint{2.714352in}{1.561032in}}%
\pgfpathlineto{\pgfqpoint{2.714352in}{1.562880in}}%
\pgfpathlineto{\pgfqpoint{2.715985in}{1.562880in}}%
\pgfpathlineto{\pgfqpoint{2.715985in}{1.564728in}}%
\pgfpathlineto{\pgfqpoint{2.717357in}{1.564728in}}%
\pgfpathlineto{\pgfqpoint{2.717357in}{1.566576in}}%
\pgfpathlineto{\pgfqpoint{2.721896in}{1.566576in}}%
\pgfpathlineto{\pgfqpoint{2.721896in}{1.568424in}}%
\pgfpathlineto{\pgfqpoint{2.723451in}{1.568424in}}%
\pgfpathlineto{\pgfqpoint{2.723539in}{1.572120in}}%
\pgfpathlineto{\pgfqpoint{2.727749in}{1.572120in}}%
\pgfpathlineto{\pgfqpoint{2.728272in}{1.575816in}}%
\pgfpathlineto{\pgfqpoint{2.728906in}{1.575816in}}%
\pgfpathlineto{\pgfqpoint{2.729889in}{1.581360in}}%
\pgfpathlineto{\pgfqpoint{2.732593in}{1.581360in}}%
\pgfpathlineto{\pgfqpoint{2.733505in}{1.585056in}}%
\pgfpathlineto{\pgfqpoint{2.734573in}{1.585056in}}%
\pgfpathlineto{\pgfqpoint{2.735504in}{1.588752in}}%
\pgfpathlineto{\pgfqpoint{2.738379in}{1.588752in}}%
\pgfpathlineto{\pgfqpoint{2.739413in}{1.594296in}}%
\pgfpathlineto{\pgfqpoint{2.742564in}{1.594296in}}%
\pgfpathlineto{\pgfqpoint{2.742564in}{1.596144in}}%
\pgfpathlineto{\pgfqpoint{2.744230in}{1.596144in}}%
\pgfpathlineto{\pgfqpoint{2.744405in}{1.599840in}}%
\pgfpathlineto{\pgfqpoint{2.746301in}{1.599840in}}%
\pgfpathlineto{\pgfqpoint{2.746301in}{1.601688in}}%
\pgfpathlineto{\pgfqpoint{2.749926in}{1.601688in}}%
\pgfpathlineto{\pgfqpoint{2.749926in}{1.603536in}}%
\pgfpathlineto{\pgfqpoint{2.751977in}{1.603536in}}%
\pgfpathlineto{\pgfqpoint{2.752044in}{1.607232in}}%
\pgfpathlineto{\pgfqpoint{2.755947in}{1.607232in}}%
\pgfpathlineto{\pgfqpoint{2.757038in}{1.612776in}}%
\pgfpathlineto{\pgfqpoint{2.757251in}{1.612776in}}%
\pgfpathlineto{\pgfqpoint{2.757683in}{1.616472in}}%
\pgfpathlineto{\pgfqpoint{2.758525in}{1.616472in}}%
\pgfpathlineto{\pgfqpoint{2.758525in}{1.618320in}}%
\pgfpathlineto{\pgfqpoint{2.762062in}{1.618320in}}%
\pgfpathlineto{\pgfqpoint{2.763112in}{1.622016in}}%
\pgfpathlineto{\pgfqpoint{2.764216in}{1.622016in}}%
\pgfpathlineto{\pgfqpoint{2.765224in}{1.625712in}}%
\pgfpathlineto{\pgfqpoint{2.767125in}{1.625712in}}%
\pgfpathlineto{\pgfqpoint{2.767884in}{1.631256in}}%
\pgfpathlineto{\pgfqpoint{2.770869in}{1.631256in}}%
\pgfpathlineto{\pgfqpoint{2.770869in}{1.633104in}}%
\pgfpathlineto{\pgfqpoint{2.774115in}{1.633104in}}%
\pgfpathlineto{\pgfqpoint{2.774858in}{1.638648in}}%
\pgfpathlineto{\pgfqpoint{2.778775in}{1.638648in}}%
\pgfpathlineto{\pgfqpoint{2.778805in}{1.642344in}}%
\pgfpathlineto{\pgfqpoint{2.780564in}{1.642344in}}%
\pgfpathlineto{\pgfqpoint{2.780564in}{1.644192in}}%
\pgfpathlineto{\pgfqpoint{2.784361in}{1.644192in}}%
\pgfpathlineto{\pgfqpoint{2.785400in}{1.649736in}}%
\pgfpathlineto{\pgfqpoint{2.785696in}{1.649736in}}%
\pgfpathlineto{\pgfqpoint{2.785731in}{1.653432in}}%
\pgfpathlineto{\pgfqpoint{2.787050in}{1.653432in}}%
\pgfpathlineto{\pgfqpoint{2.788087in}{1.657128in}}%
\pgfpathlineto{\pgfqpoint{2.790891in}{1.657128in}}%
\pgfpathlineto{\pgfqpoint{2.791323in}{1.660824in}}%
\pgfpathlineto{\pgfqpoint{2.792662in}{1.660824in}}%
\pgfpathlineto{\pgfqpoint{2.792662in}{1.662672in}}%
\pgfpathlineto{\pgfqpoint{2.795480in}{1.662672in}}%
\pgfpathlineto{\pgfqpoint{2.796010in}{1.666368in}}%
\pgfpathlineto{\pgfqpoint{2.797331in}{1.666368in}}%
\pgfpathlineto{\pgfqpoint{2.797331in}{1.668216in}}%
\pgfpathlineto{\pgfqpoint{2.798936in}{1.668216in}}%
\pgfpathlineto{\pgfqpoint{2.798936in}{1.670064in}}%
\pgfpathlineto{\pgfqpoint{2.801834in}{1.670064in}}%
\pgfpathlineto{\pgfqpoint{2.802256in}{1.673760in}}%
\pgfpathlineto{\pgfqpoint{2.804864in}{1.673760in}}%
\pgfpathlineto{\pgfqpoint{2.804864in}{1.675608in}}%
\pgfpathlineto{\pgfqpoint{2.806886in}{1.675608in}}%
\pgfpathlineto{\pgfqpoint{2.807969in}{1.681152in}}%
\pgfpathlineto{\pgfqpoint{2.808302in}{1.681152in}}%
\pgfpathlineto{\pgfqpoint{2.809020in}{1.684848in}}%
\pgfpathlineto{\pgfqpoint{2.809860in}{1.684848in}}%
\pgfpathlineto{\pgfqpoint{2.810045in}{1.688544in}}%
\pgfpathlineto{\pgfqpoint{2.811222in}{1.688544in}}%
\pgfpathlineto{\pgfqpoint{2.811222in}{1.690392in}}%
\pgfpathlineto{\pgfqpoint{2.813896in}{1.690392in}}%
\pgfpathlineto{\pgfqpoint{2.813896in}{1.692240in}}%
\pgfpathlineto{\pgfqpoint{2.815024in}{1.692240in}}%
\pgfpathlineto{\pgfqpoint{2.816016in}{1.697784in}}%
\pgfpathlineto{\pgfqpoint{2.818093in}{1.697784in}}%
\pgfpathlineto{\pgfqpoint{2.819123in}{1.701480in}}%
\pgfpathlineto{\pgfqpoint{2.821735in}{1.701480in}}%
\pgfpathlineto{\pgfqpoint{2.821735in}{1.703328in}}%
\pgfpathlineto{\pgfqpoint{2.823418in}{1.703328in}}%
\pgfpathlineto{\pgfqpoint{2.823697in}{1.707024in}}%
\pgfpathlineto{\pgfqpoint{2.825597in}{1.707024in}}%
\pgfpathlineto{\pgfqpoint{2.825597in}{1.708872in}}%
\pgfpathlineto{\pgfqpoint{2.828063in}{1.708872in}}%
\pgfpathlineto{\pgfqpoint{2.828063in}{1.710720in}}%
\pgfpathlineto{\pgfqpoint{2.829296in}{1.710720in}}%
\pgfpathlineto{\pgfqpoint{2.830241in}{1.716264in}}%
\pgfpathlineto{\pgfqpoint{2.833384in}{1.716264in}}%
\pgfpathlineto{\pgfqpoint{2.833384in}{1.718112in}}%
\pgfpathlineto{\pgfqpoint{2.835809in}{1.718112in}}%
\pgfpathlineto{\pgfqpoint{2.836347in}{1.721808in}}%
\pgfpathlineto{\pgfqpoint{2.837109in}{1.721808in}}%
\pgfpathlineto{\pgfqpoint{2.837706in}{1.725504in}}%
\pgfpathlineto{\pgfqpoint{2.838404in}{1.725504in}}%
\pgfpathlineto{\pgfqpoint{2.838576in}{1.729200in}}%
\pgfpathlineto{\pgfqpoint{2.839613in}{1.729200in}}%
\pgfpathlineto{\pgfqpoint{2.839613in}{1.731048in}}%
\pgfpathlineto{\pgfqpoint{2.842035in}{1.731048in}}%
\pgfpathlineto{\pgfqpoint{2.842930in}{1.734744in}}%
\pgfpathlineto{\pgfqpoint{2.843979in}{1.734744in}}%
\pgfpathlineto{\pgfqpoint{2.844141in}{1.738440in}}%
\pgfpathlineto{\pgfqpoint{2.846265in}{1.738440in}}%
\pgfpathlineto{\pgfqpoint{2.846396in}{1.742136in}}%
\pgfpathlineto{\pgfqpoint{2.849899in}{1.742136in}}%
\pgfpathlineto{\pgfqpoint{2.849899in}{1.743984in}}%
\pgfpathlineto{\pgfqpoint{2.853064in}{1.743984in}}%
\pgfpathlineto{\pgfqpoint{2.853878in}{1.747680in}}%
\pgfpathlineto{\pgfqpoint{2.855708in}{1.747680in}}%
\pgfpathlineto{\pgfqpoint{2.856553in}{1.751376in}}%
\pgfpathlineto{\pgfqpoint{2.857538in}{1.751376in}}%
\pgfpathlineto{\pgfqpoint{2.857999in}{1.755072in}}%
\pgfpathlineto{\pgfqpoint{2.859447in}{1.755072in}}%
\pgfpathlineto{\pgfqpoint{2.860246in}{1.760616in}}%
\pgfpathlineto{\pgfqpoint{2.860926in}{1.760616in}}%
\pgfpathlineto{\pgfqpoint{2.861136in}{1.764312in}}%
\pgfpathlineto{\pgfqpoint{2.866744in}{1.764312in}}%
\pgfpathlineto{\pgfqpoint{2.867672in}{1.769856in}}%
\pgfpathlineto{\pgfqpoint{2.870148in}{1.769856in}}%
\pgfpathlineto{\pgfqpoint{2.870759in}{1.773552in}}%
\pgfpathlineto{\pgfqpoint{2.874074in}{1.773552in}}%
\pgfpathlineto{\pgfqpoint{2.875125in}{1.779096in}}%
\pgfpathlineto{\pgfqpoint{2.876117in}{1.779096in}}%
\pgfpathlineto{\pgfqpoint{2.876455in}{1.782792in}}%
\pgfpathlineto{\pgfqpoint{2.878121in}{1.782792in}}%
\pgfpathlineto{\pgfqpoint{2.878121in}{1.784640in}}%
\pgfpathlineto{\pgfqpoint{2.880196in}{1.784640in}}%
\pgfpathlineto{\pgfqpoint{2.880579in}{1.788336in}}%
\pgfpathlineto{\pgfqpoint{2.881377in}{1.788336in}}%
\pgfpathlineto{\pgfqpoint{2.881377in}{1.790184in}}%
\pgfpathlineto{\pgfqpoint{2.884069in}{1.790184in}}%
\pgfpathlineto{\pgfqpoint{2.884069in}{1.792032in}}%
\pgfpathlineto{\pgfqpoint{2.888017in}{1.792032in}}%
\pgfpathlineto{\pgfqpoint{2.889001in}{1.803120in}}%
\pgfpathlineto{\pgfqpoint{2.889237in}{1.803120in}}%
\pgfpathlineto{\pgfqpoint{2.889237in}{1.804968in}}%
\pgfpathlineto{\pgfqpoint{2.892883in}{1.804968in}}%
\pgfpathlineto{\pgfqpoint{2.892883in}{1.806816in}}%
\pgfpathlineto{\pgfqpoint{2.894904in}{1.806816in}}%
\pgfpathlineto{\pgfqpoint{2.895332in}{1.810512in}}%
\pgfpathlineto{\pgfqpoint{2.896601in}{1.810512in}}%
\pgfpathlineto{\pgfqpoint{2.896601in}{1.812360in}}%
\pgfpathlineto{\pgfqpoint{2.897943in}{1.812360in}}%
\pgfpathlineto{\pgfqpoint{2.897943in}{1.814208in}}%
\pgfpathlineto{\pgfqpoint{2.900244in}{1.814208in}}%
\pgfpathlineto{\pgfqpoint{2.900244in}{1.816056in}}%
\pgfpathlineto{\pgfqpoint{2.902558in}{1.816056in}}%
\pgfpathlineto{\pgfqpoint{2.902873in}{1.819752in}}%
\pgfpathlineto{\pgfqpoint{2.904063in}{1.819752in}}%
\pgfpathlineto{\pgfqpoint{2.904118in}{1.823448in}}%
\pgfpathlineto{\pgfqpoint{2.905668in}{1.823448in}}%
\pgfpathlineto{\pgfqpoint{2.905668in}{1.825296in}}%
\pgfpathlineto{\pgfqpoint{2.908235in}{1.825296in}}%
\pgfpathlineto{\pgfqpoint{2.908893in}{1.828992in}}%
\pgfpathlineto{\pgfqpoint{2.910361in}{1.828992in}}%
\pgfpathlineto{\pgfqpoint{2.911264in}{1.832688in}}%
\pgfpathlineto{\pgfqpoint{2.912258in}{1.832688in}}%
\pgfpathlineto{\pgfqpoint{2.912258in}{1.834536in}}%
\pgfpathlineto{\pgfqpoint{2.915619in}{1.834536in}}%
\pgfpathlineto{\pgfqpoint{2.916573in}{1.841928in}}%
\pgfpathlineto{\pgfqpoint{2.916839in}{1.841928in}}%
\pgfpathlineto{\pgfqpoint{2.917898in}{1.847472in}}%
\pgfpathlineto{\pgfqpoint{2.920569in}{1.847472in}}%
\pgfpathlineto{\pgfqpoint{2.920964in}{1.851168in}}%
\pgfpathlineto{\pgfqpoint{2.922467in}{1.851168in}}%
\pgfpathlineto{\pgfqpoint{2.922467in}{1.853016in}}%
\pgfpathlineto{\pgfqpoint{2.924584in}{1.853016in}}%
\pgfpathlineto{\pgfqpoint{2.924959in}{1.856712in}}%
\pgfpathlineto{\pgfqpoint{2.926252in}{1.856712in}}%
\pgfpathlineto{\pgfqpoint{2.926252in}{1.858560in}}%
\pgfpathlineto{\pgfqpoint{2.927885in}{1.858560in}}%
\pgfpathlineto{\pgfqpoint{2.928265in}{1.862256in}}%
\pgfpathlineto{\pgfqpoint{2.930509in}{1.862256in}}%
\pgfpathlineto{\pgfqpoint{2.931426in}{1.865952in}}%
\pgfpathlineto{\pgfqpoint{2.932079in}{1.865952in}}%
\pgfpathlineto{\pgfqpoint{2.933156in}{1.871496in}}%
\pgfpathlineto{\pgfqpoint{2.934637in}{1.871496in}}%
\pgfpathlineto{\pgfqpoint{2.934637in}{1.873344in}}%
\pgfpathlineto{\pgfqpoint{2.936211in}{1.873344in}}%
\pgfpathlineto{\pgfqpoint{2.936211in}{1.875192in}}%
\pgfpathlineto{\pgfqpoint{2.939151in}{1.875192in}}%
\pgfpathlineto{\pgfqpoint{2.939162in}{1.878888in}}%
\pgfpathlineto{\pgfqpoint{2.941189in}{1.878888in}}%
\pgfpathlineto{\pgfqpoint{2.941548in}{1.882584in}}%
\pgfpathlineto{\pgfqpoint{2.942791in}{1.882584in}}%
\pgfpathlineto{\pgfqpoint{2.943711in}{1.888128in}}%
\pgfpathlineto{\pgfqpoint{2.944222in}{1.888128in}}%
\pgfpathlineto{\pgfqpoint{2.944222in}{1.889976in}}%
\pgfpathlineto{\pgfqpoint{2.945466in}{1.889976in}}%
\pgfpathlineto{\pgfqpoint{2.946428in}{1.893672in}}%
\pgfpathlineto{\pgfqpoint{2.948493in}{1.893672in}}%
\pgfpathlineto{\pgfqpoint{2.948493in}{1.895520in}}%
\pgfpathlineto{\pgfqpoint{2.949990in}{1.895520in}}%
\pgfpathlineto{\pgfqpoint{2.949990in}{1.897368in}}%
\pgfpathlineto{\pgfqpoint{2.951679in}{1.897368in}}%
\pgfpathlineto{\pgfqpoint{2.952452in}{1.902912in}}%
\pgfpathlineto{\pgfqpoint{2.954287in}{1.902912in}}%
\pgfpathlineto{\pgfqpoint{2.954784in}{1.906608in}}%
\pgfpathlineto{\pgfqpoint{2.955615in}{1.906608in}}%
\pgfpathlineto{\pgfqpoint{2.956212in}{1.910304in}}%
\pgfpathlineto{\pgfqpoint{2.958969in}{1.910304in}}%
\pgfpathlineto{\pgfqpoint{2.959311in}{1.914000in}}%
\pgfpathlineto{\pgfqpoint{2.960869in}{1.914000in}}%
\pgfpathlineto{\pgfqpoint{2.960869in}{1.915848in}}%
\pgfpathlineto{\pgfqpoint{2.962394in}{1.915848in}}%
\pgfpathlineto{\pgfqpoint{2.962394in}{1.917696in}}%
\pgfpathlineto{\pgfqpoint{2.963992in}{1.917696in}}%
\pgfpathlineto{\pgfqpoint{2.963996in}{1.921392in}}%
\pgfpathlineto{\pgfqpoint{2.966522in}{1.921392in}}%
\pgfpathlineto{\pgfqpoint{2.967201in}{1.926936in}}%
\pgfpathlineto{\pgfqpoint{2.968146in}{1.926936in}}%
\pgfpathlineto{\pgfqpoint{2.968146in}{1.928784in}}%
\pgfpathlineto{\pgfqpoint{2.970807in}{1.928784in}}%
\pgfpathlineto{\pgfqpoint{2.970807in}{1.930632in}}%
\pgfpathlineto{\pgfqpoint{2.974225in}{1.930632in}}%
\pgfpathlineto{\pgfqpoint{2.974456in}{1.934328in}}%
\pgfpathlineto{\pgfqpoint{2.977014in}{1.934328in}}%
\pgfpathlineto{\pgfqpoint{2.977862in}{1.938024in}}%
\pgfpathlineto{\pgfqpoint{2.979331in}{1.938024in}}%
\pgfpathlineto{\pgfqpoint{2.979995in}{1.941720in}}%
\pgfpathlineto{\pgfqpoint{2.982455in}{1.941720in}}%
\pgfpathlineto{\pgfqpoint{2.982455in}{1.943568in}}%
\pgfpathlineto{\pgfqpoint{2.984232in}{1.943568in}}%
\pgfpathlineto{\pgfqpoint{2.984232in}{1.945416in}}%
\pgfpathlineto{\pgfqpoint{2.986681in}{1.945416in}}%
\pgfpathlineto{\pgfqpoint{2.986949in}{1.949112in}}%
\pgfpathlineto{\pgfqpoint{2.988727in}{1.949112in}}%
\pgfpathlineto{\pgfqpoint{2.988727in}{1.950960in}}%
\pgfpathlineto{\pgfqpoint{2.989986in}{1.950960in}}%
\pgfpathlineto{\pgfqpoint{2.990325in}{1.954656in}}%
\pgfpathlineto{\pgfqpoint{2.991232in}{1.954656in}}%
\pgfpathlineto{\pgfqpoint{2.992089in}{1.960200in}}%
\pgfpathlineto{\pgfqpoint{2.994616in}{1.960200in}}%
\pgfpathlineto{\pgfqpoint{2.995390in}{1.963896in}}%
\pgfpathlineto{\pgfqpoint{2.995926in}{1.963896in}}%
\pgfpathlineto{\pgfqpoint{2.996667in}{1.969440in}}%
\pgfpathlineto{\pgfqpoint{3.001938in}{1.969440in}}%
\pgfpathlineto{\pgfqpoint{3.002638in}{1.974984in}}%
\pgfpathlineto{\pgfqpoint{3.003614in}{1.974984in}}%
\pgfpathlineto{\pgfqpoint{3.003614in}{1.976832in}}%
\pgfpathlineto{\pgfqpoint{3.004748in}{1.976832in}}%
\pgfpathlineto{\pgfqpoint{3.005572in}{1.980528in}}%
\pgfpathlineto{\pgfqpoint{3.006140in}{1.980528in}}%
\pgfpathlineto{\pgfqpoint{3.006870in}{1.986072in}}%
\pgfpathlineto{\pgfqpoint{3.009473in}{1.986072in}}%
\pgfpathlineto{\pgfqpoint{3.009775in}{1.989768in}}%
\pgfpathlineto{\pgfqpoint{3.011129in}{1.989768in}}%
\pgfpathlineto{\pgfqpoint{3.011129in}{1.991616in}}%
\pgfpathlineto{\pgfqpoint{3.012620in}{1.991616in}}%
\pgfpathlineto{\pgfqpoint{3.013060in}{1.995312in}}%
\pgfpathlineto{\pgfqpoint{3.016842in}{1.995312in}}%
\pgfpathlineto{\pgfqpoint{3.017778in}{2.000856in}}%
\pgfpathlineto{\pgfqpoint{3.019002in}{2.000856in}}%
\pgfpathlineto{\pgfqpoint{3.019419in}{2.006400in}}%
\pgfpathlineto{\pgfqpoint{3.023558in}{2.006400in}}%
\pgfpathlineto{\pgfqpoint{3.023697in}{2.010096in}}%
\pgfpathlineto{\pgfqpoint{3.029804in}{2.010096in}}%
\pgfpathlineto{\pgfqpoint{3.030104in}{2.013792in}}%
\pgfpathlineto{\pgfqpoint{3.031479in}{2.013792in}}%
\pgfpathlineto{\pgfqpoint{3.032252in}{2.019336in}}%
\pgfpathlineto{\pgfqpoint{3.033198in}{2.019336in}}%
\pgfpathlineto{\pgfqpoint{3.033198in}{2.021184in}}%
\pgfpathlineto{\pgfqpoint{3.034628in}{2.021184in}}%
\pgfpathlineto{\pgfqpoint{3.035619in}{2.024880in}}%
\pgfpathlineto{\pgfqpoint{3.036242in}{2.024880in}}%
\pgfpathlineto{\pgfqpoint{3.037305in}{2.032272in}}%
\pgfpathlineto{\pgfqpoint{3.037356in}{2.032272in}}%
\pgfpathlineto{\pgfqpoint{3.038069in}{2.035968in}}%
\pgfpathlineto{\pgfqpoint{3.039528in}{2.035968in}}%
\pgfpathlineto{\pgfqpoint{3.039911in}{2.039664in}}%
\pgfpathlineto{\pgfqpoint{3.040965in}{2.039664in}}%
\pgfpathlineto{\pgfqpoint{3.041540in}{2.043360in}}%
\pgfpathlineto{\pgfqpoint{3.044159in}{2.043360in}}%
\pgfpathlineto{\pgfqpoint{3.045250in}{2.047056in}}%
\pgfpathlineto{\pgfqpoint{3.045339in}{2.047056in}}%
\pgfpathlineto{\pgfqpoint{3.046205in}{2.050752in}}%
\pgfpathlineto{\pgfqpoint{3.050760in}{2.050752in}}%
\pgfpathlineto{\pgfqpoint{3.051814in}{2.054448in}}%
\pgfpathlineto{\pgfqpoint{3.054610in}{2.054448in}}%
\pgfpathlineto{\pgfqpoint{3.055395in}{2.059992in}}%
\pgfpathlineto{\pgfqpoint{3.057440in}{2.059992in}}%
\pgfpathlineto{\pgfqpoint{3.057440in}{2.061840in}}%
\pgfpathlineto{\pgfqpoint{3.058647in}{2.061840in}}%
\pgfpathlineto{\pgfqpoint{3.058826in}{2.065536in}}%
\pgfpathlineto{\pgfqpoint{3.062051in}{2.065536in}}%
\pgfpathlineto{\pgfqpoint{3.062270in}{2.069232in}}%
\pgfpathlineto{\pgfqpoint{3.063513in}{2.069232in}}%
\pgfpathlineto{\pgfqpoint{3.064488in}{2.072928in}}%
\pgfpathlineto{\pgfqpoint{3.065007in}{2.072928in}}%
\pgfpathlineto{\pgfqpoint{3.065531in}{2.076624in}}%
\pgfpathlineto{\pgfqpoint{3.066867in}{2.076624in}}%
\pgfpathlineto{\pgfqpoint{3.066867in}{2.078472in}}%
\pgfpathlineto{\pgfqpoint{3.068194in}{2.078472in}}%
\pgfpathlineto{\pgfqpoint{3.069290in}{2.084016in}}%
\pgfpathlineto{\pgfqpoint{3.071605in}{2.084016in}}%
\pgfpathlineto{\pgfqpoint{3.072381in}{2.089560in}}%
\pgfpathlineto{\pgfqpoint{3.073386in}{2.089560in}}%
\pgfpathlineto{\pgfqpoint{3.074278in}{2.093256in}}%
\pgfpathlineto{\pgfqpoint{3.078502in}{2.093256in}}%
\pgfpathlineto{\pgfqpoint{3.078710in}{2.096952in}}%
\pgfpathlineto{\pgfqpoint{3.079667in}{2.096952in}}%
\pgfpathlineto{\pgfqpoint{3.080754in}{2.100648in}}%
\pgfpathlineto{\pgfqpoint{3.082331in}{2.100648in}}%
\pgfpathlineto{\pgfqpoint{3.082331in}{2.102496in}}%
\pgfpathlineto{\pgfqpoint{3.084416in}{2.102496in}}%
\pgfpathlineto{\pgfqpoint{3.084693in}{2.106192in}}%
\pgfpathlineto{\pgfqpoint{3.086207in}{2.106192in}}%
\pgfpathlineto{\pgfqpoint{3.086912in}{2.109888in}}%
\pgfpathlineto{\pgfqpoint{3.087607in}{2.109888in}}%
\pgfpathlineto{\pgfqpoint{3.087607in}{2.111736in}}%
\pgfpathlineto{\pgfqpoint{3.089500in}{2.111736in}}%
\pgfpathlineto{\pgfqpoint{3.089500in}{2.113584in}}%
\pgfpathlineto{\pgfqpoint{3.091077in}{2.113584in}}%
\pgfpathlineto{\pgfqpoint{3.091077in}{2.115432in}}%
\pgfpathlineto{\pgfqpoint{3.092247in}{2.115432in}}%
\pgfpathlineto{\pgfqpoint{3.092247in}{2.117280in}}%
\pgfpathlineto{\pgfqpoint{3.093835in}{2.117280in}}%
\pgfpathlineto{\pgfqpoint{3.093835in}{2.119128in}}%
\pgfpathlineto{\pgfqpoint{3.095412in}{2.119128in}}%
\pgfpathlineto{\pgfqpoint{3.096358in}{2.126520in}}%
\pgfpathlineto{\pgfqpoint{3.096635in}{2.126520in}}%
\pgfpathlineto{\pgfqpoint{3.096635in}{2.128368in}}%
\pgfpathlineto{\pgfqpoint{3.099012in}{2.128368in}}%
\pgfpathlineto{\pgfqpoint{3.099285in}{2.132064in}}%
\pgfpathlineto{\pgfqpoint{3.101648in}{2.132064in}}%
\pgfpathlineto{\pgfqpoint{3.101648in}{2.133912in}}%
\pgfpathlineto{\pgfqpoint{3.105689in}{2.133912in}}%
\pgfpathlineto{\pgfqpoint{3.106723in}{2.139456in}}%
\pgfpathlineto{\pgfqpoint{3.107072in}{2.139456in}}%
\pgfpathlineto{\pgfqpoint{3.107594in}{2.143152in}}%
\pgfpathlineto{\pgfqpoint{3.108271in}{2.143152in}}%
\pgfpathlineto{\pgfqpoint{3.108271in}{2.145000in}}%
\pgfpathlineto{\pgfqpoint{3.109955in}{2.145000in}}%
\pgfpathlineto{\pgfqpoint{3.109955in}{2.146848in}}%
\pgfpathlineto{\pgfqpoint{3.111706in}{2.146848in}}%
\pgfpathlineto{\pgfqpoint{3.111706in}{2.148696in}}%
\pgfpathlineto{\pgfqpoint{3.113297in}{2.148696in}}%
\pgfpathlineto{\pgfqpoint{3.114177in}{2.152392in}}%
\pgfpathlineto{\pgfqpoint{3.114510in}{2.152392in}}%
\pgfpathlineto{\pgfqpoint{3.114595in}{2.156088in}}%
\pgfpathlineto{\pgfqpoint{3.115733in}{2.156088in}}%
\pgfpathlineto{\pgfqpoint{3.115733in}{2.157936in}}%
\pgfpathlineto{\pgfqpoint{3.116914in}{2.157936in}}%
\pgfpathlineto{\pgfqpoint{3.116914in}{2.159784in}}%
\pgfpathlineto{\pgfqpoint{3.118039in}{2.159784in}}%
\pgfpathlineto{\pgfqpoint{3.118562in}{2.165328in}}%
\pgfpathlineto{\pgfqpoint{3.120669in}{2.165328in}}%
\pgfpathlineto{\pgfqpoint{3.121022in}{2.169024in}}%
\pgfpathlineto{\pgfqpoint{3.122569in}{2.169024in}}%
\pgfpathlineto{\pgfqpoint{3.122854in}{2.172720in}}%
\pgfpathlineto{\pgfqpoint{3.126392in}{2.172720in}}%
\pgfpathlineto{\pgfqpoint{3.126392in}{2.174568in}}%
\pgfpathlineto{\pgfqpoint{3.128688in}{2.174568in}}%
\pgfpathlineto{\pgfqpoint{3.129554in}{2.181960in}}%
\pgfpathlineto{\pgfqpoint{3.132194in}{2.181960in}}%
\pgfpathlineto{\pgfqpoint{3.132653in}{2.185656in}}%
\pgfpathlineto{\pgfqpoint{3.133373in}{2.185656in}}%
\pgfpathlineto{\pgfqpoint{3.134185in}{2.191200in}}%
\pgfpathlineto{\pgfqpoint{3.134940in}{2.191200in}}%
\pgfpathlineto{\pgfqpoint{3.135923in}{2.194896in}}%
\pgfpathlineto{\pgfqpoint{3.140157in}{2.194896in}}%
\pgfpathlineto{\pgfqpoint{3.141197in}{2.200440in}}%
\pgfpathlineto{\pgfqpoint{3.142663in}{2.200440in}}%
\pgfpathlineto{\pgfqpoint{3.143406in}{2.204136in}}%
\pgfpathlineto{\pgfqpoint{3.144006in}{2.204136in}}%
\pgfpathlineto{\pgfqpoint{3.145106in}{2.209680in}}%
\pgfpathlineto{\pgfqpoint{3.145222in}{2.209680in}}%
\pgfpathlineto{\pgfqpoint{3.146195in}{2.217072in}}%
\pgfpathlineto{\pgfqpoint{3.147624in}{2.217072in}}%
\pgfpathlineto{\pgfqpoint{3.148235in}{2.220768in}}%
\pgfpathlineto{\pgfqpoint{3.150807in}{2.220768in}}%
\pgfpathlineto{\pgfqpoint{3.151788in}{2.224464in}}%
\pgfpathlineto{\pgfqpoint{3.152618in}{2.224464in}}%
\pgfpathlineto{\pgfqpoint{3.152618in}{2.226312in}}%
\pgfpathlineto{\pgfqpoint{3.154043in}{2.226312in}}%
\pgfpathlineto{\pgfqpoint{3.154973in}{2.230008in}}%
\pgfpathlineto{\pgfqpoint{3.156592in}{2.230008in}}%
\pgfpathlineto{\pgfqpoint{3.156592in}{2.231856in}}%
\pgfpathlineto{\pgfqpoint{3.159323in}{2.231856in}}%
\pgfpathlineto{\pgfqpoint{3.159323in}{2.233704in}}%
\pgfpathlineto{\pgfqpoint{3.161276in}{2.233704in}}%
\pgfpathlineto{\pgfqpoint{3.161450in}{2.237400in}}%
\pgfpathlineto{\pgfqpoint{3.163086in}{2.237400in}}%
\pgfpathlineto{\pgfqpoint{3.163432in}{2.241096in}}%
\pgfpathlineto{\pgfqpoint{3.165457in}{2.241096in}}%
\pgfpathlineto{\pgfqpoint{3.165605in}{2.244792in}}%
\pgfpathlineto{\pgfqpoint{3.166842in}{2.244792in}}%
\pgfpathlineto{\pgfqpoint{3.166842in}{2.246640in}}%
\pgfpathlineto{\pgfqpoint{3.169519in}{2.246640in}}%
\pgfpathlineto{\pgfqpoint{3.169519in}{2.248488in}}%
\pgfpathlineto{\pgfqpoint{3.170856in}{2.248488in}}%
\pgfpathlineto{\pgfqpoint{3.170856in}{2.250336in}}%
\pgfpathlineto{\pgfqpoint{3.172002in}{2.250336in}}%
\pgfpathlineto{\pgfqpoint{3.173095in}{2.259576in}}%
\pgfpathlineto{\pgfqpoint{3.174859in}{2.259576in}}%
\pgfpathlineto{\pgfqpoint{3.175559in}{2.263272in}}%
\pgfpathlineto{\pgfqpoint{3.178506in}{2.263272in}}%
\pgfpathlineto{\pgfqpoint{3.178506in}{2.265120in}}%
\pgfpathlineto{\pgfqpoint{3.180816in}{2.265120in}}%
\pgfpathlineto{\pgfqpoint{3.181160in}{2.268816in}}%
\pgfpathlineto{\pgfqpoint{3.183216in}{2.268816in}}%
\pgfpathlineto{\pgfqpoint{3.183530in}{2.272512in}}%
\pgfpathlineto{\pgfqpoint{3.188187in}{2.272512in}}%
\pgfpathlineto{\pgfqpoint{3.188299in}{2.276208in}}%
\pgfpathlineto{\pgfqpoint{3.190029in}{2.276208in}}%
\pgfpathlineto{\pgfqpoint{3.190669in}{2.281752in}}%
\pgfpathlineto{\pgfqpoint{3.191246in}{2.281752in}}%
\pgfpathlineto{\pgfqpoint{3.191246in}{2.283600in}}%
\pgfpathlineto{\pgfqpoint{3.192645in}{2.283600in}}%
\pgfpathlineto{\pgfqpoint{3.193544in}{2.287296in}}%
\pgfpathlineto{\pgfqpoint{3.194084in}{2.287296in}}%
\pgfpathlineto{\pgfqpoint{3.195149in}{2.292840in}}%
\pgfpathlineto{\pgfqpoint{3.196414in}{2.292840in}}%
\pgfpathlineto{\pgfqpoint{3.196414in}{2.294688in}}%
\pgfpathlineto{\pgfqpoint{3.197750in}{2.294688in}}%
\pgfpathlineto{\pgfqpoint{3.197750in}{2.296536in}}%
\pgfpathlineto{\pgfqpoint{3.199136in}{2.296536in}}%
\pgfpathlineto{\pgfqpoint{3.199499in}{2.300232in}}%
\pgfpathlineto{\pgfqpoint{3.202768in}{2.300232in}}%
\pgfpathlineto{\pgfqpoint{3.202768in}{2.302080in}}%
\pgfpathlineto{\pgfqpoint{3.205363in}{2.302080in}}%
\pgfpathlineto{\pgfqpoint{3.205363in}{2.303928in}}%
\pgfpathlineto{\pgfqpoint{3.207131in}{2.303928in}}%
\pgfpathlineto{\pgfqpoint{3.207810in}{2.307624in}}%
\pgfpathlineto{\pgfqpoint{3.209124in}{2.307624in}}%
\pgfpathlineto{\pgfqpoint{3.210150in}{2.315016in}}%
\pgfpathlineto{\pgfqpoint{3.210577in}{2.315016in}}%
\pgfpathlineto{\pgfqpoint{3.211468in}{2.320560in}}%
\pgfpathlineto{\pgfqpoint{3.212855in}{2.320560in}}%
\pgfpathlineto{\pgfqpoint{3.212855in}{2.322408in}}%
\pgfpathlineto{\pgfqpoint{3.215239in}{2.322408in}}%
\pgfpathlineto{\pgfqpoint{3.215259in}{2.326104in}}%
\pgfpathlineto{\pgfqpoint{3.217595in}{2.326104in}}%
\pgfpathlineto{\pgfqpoint{3.218313in}{2.329800in}}%
\pgfpathlineto{\pgfqpoint{3.219720in}{2.329800in}}%
\pgfpathlineto{\pgfqpoint{3.219720in}{2.331648in}}%
\pgfpathlineto{\pgfqpoint{3.221044in}{2.331648in}}%
\pgfpathlineto{\pgfqpoint{3.222149in}{2.340888in}}%
\pgfpathlineto{\pgfqpoint{3.222830in}{2.340888in}}%
\pgfpathlineto{\pgfqpoint{3.222830in}{2.342736in}}%
\pgfpathlineto{\pgfqpoint{3.224545in}{2.342736in}}%
\pgfpathlineto{\pgfqpoint{3.224847in}{2.346432in}}%
\pgfpathlineto{\pgfqpoint{3.229707in}{2.346432in}}%
\pgfpathlineto{\pgfqpoint{3.230080in}{2.350128in}}%
\pgfpathlineto{\pgfqpoint{3.231983in}{2.350128in}}%
\pgfpathlineto{\pgfqpoint{3.232723in}{2.353824in}}%
\pgfpathlineto{\pgfqpoint{3.234324in}{2.353824in}}%
\pgfpathlineto{\pgfqpoint{3.234324in}{2.355672in}}%
\pgfpathlineto{\pgfqpoint{3.236122in}{2.355672in}}%
\pgfpathlineto{\pgfqpoint{3.236899in}{2.359368in}}%
\pgfpathlineto{\pgfqpoint{3.239739in}{2.359368in}}%
\pgfpathlineto{\pgfqpoint{3.240803in}{2.364912in}}%
\pgfpathlineto{\pgfqpoint{3.241905in}{2.364912in}}%
\pgfpathlineto{\pgfqpoint{3.242244in}{2.368608in}}%
\pgfpathlineto{\pgfqpoint{3.244197in}{2.368608in}}%
\pgfpathlineto{\pgfqpoint{3.245067in}{2.372304in}}%
\pgfpathlineto{\pgfqpoint{3.246633in}{2.372304in}}%
\pgfpathlineto{\pgfqpoint{3.246633in}{2.374152in}}%
\pgfpathlineto{\pgfqpoint{3.247755in}{2.374152in}}%
\pgfpathlineto{\pgfqpoint{3.248193in}{2.383392in}}%
\pgfpathlineto{\pgfqpoint{3.251201in}{2.383392in}}%
\pgfpathlineto{\pgfqpoint{3.251201in}{2.385240in}}%
\pgfpathlineto{\pgfqpoint{3.255559in}{2.385240in}}%
\pgfpathlineto{\pgfqpoint{3.256627in}{2.390784in}}%
\pgfpathlineto{\pgfqpoint{3.257070in}{2.390784in}}%
\pgfpathlineto{\pgfqpoint{3.257797in}{2.394480in}}%
\pgfpathlineto{\pgfqpoint{3.258501in}{2.394480in}}%
\pgfpathlineto{\pgfqpoint{3.258501in}{2.396328in}}%
\pgfpathlineto{\pgfqpoint{3.259815in}{2.396328in}}%
\pgfpathlineto{\pgfqpoint{3.260571in}{2.400024in}}%
\pgfpathlineto{\pgfqpoint{3.261014in}{2.400024in}}%
\pgfpathlineto{\pgfqpoint{3.261014in}{2.401872in}}%
\pgfpathlineto{\pgfqpoint{3.262444in}{2.401872in}}%
\pgfpathlineto{\pgfqpoint{3.262444in}{2.403720in}}%
\pgfpathlineto{\pgfqpoint{3.266356in}{2.403720in}}%
\pgfpathlineto{\pgfqpoint{3.266960in}{2.407416in}}%
\pgfpathlineto{\pgfqpoint{3.268580in}{2.407416in}}%
\pgfpathlineto{\pgfqpoint{3.269515in}{2.412960in}}%
\pgfpathlineto{\pgfqpoint{3.269937in}{2.412960in}}%
\pgfpathlineto{\pgfqpoint{3.269937in}{2.414808in}}%
\pgfpathlineto{\pgfqpoint{3.271111in}{2.414808in}}%
\pgfpathlineto{\pgfqpoint{3.271111in}{2.416656in}}%
\pgfpathlineto{\pgfqpoint{3.273714in}{2.416656in}}%
\pgfpathlineto{\pgfqpoint{3.274571in}{2.422200in}}%
\pgfpathlineto{\pgfqpoint{3.275091in}{2.422200in}}%
\pgfpathlineto{\pgfqpoint{3.275091in}{2.424048in}}%
\pgfpathlineto{\pgfqpoint{3.277582in}{2.424048in}}%
\pgfpathlineto{\pgfqpoint{3.277709in}{2.427744in}}%
\pgfpathlineto{\pgfqpoint{3.282461in}{2.427744in}}%
\pgfpathlineto{\pgfqpoint{3.283227in}{2.431440in}}%
\pgfpathlineto{\pgfqpoint{3.284358in}{2.431440in}}%
\pgfpathlineto{\pgfqpoint{3.285306in}{2.435136in}}%
\pgfpathlineto{\pgfqpoint{3.286801in}{2.435136in}}%
\pgfpathlineto{\pgfqpoint{3.287546in}{2.442528in}}%
\pgfpathlineto{\pgfqpoint{3.288423in}{2.442528in}}%
\pgfpathlineto{\pgfqpoint{3.289180in}{2.448072in}}%
\pgfpathlineto{\pgfqpoint{3.289982in}{2.448072in}}%
\pgfpathlineto{\pgfqpoint{3.290564in}{2.451768in}}%
\pgfpathlineto{\pgfqpoint{3.292314in}{2.451768in}}%
\pgfpathlineto{\pgfqpoint{3.293200in}{2.455464in}}%
\pgfpathlineto{\pgfqpoint{3.296441in}{2.455464in}}%
\pgfpathlineto{\pgfqpoint{3.296804in}{2.461008in}}%
\pgfpathlineto{\pgfqpoint{3.297786in}{2.461008in}}%
\pgfpathlineto{\pgfqpoint{3.298778in}{2.464704in}}%
\pgfpathlineto{\pgfqpoint{3.301016in}{2.464704in}}%
\pgfpathlineto{\pgfqpoint{3.301194in}{2.468400in}}%
\pgfpathlineto{\pgfqpoint{3.304161in}{2.468400in}}%
\pgfpathlineto{\pgfqpoint{3.305013in}{2.473944in}}%
\pgfpathlineto{\pgfqpoint{3.308364in}{2.473944in}}%
\pgfpathlineto{\pgfqpoint{3.309336in}{2.479488in}}%
\pgfpathlineto{\pgfqpoint{3.310833in}{2.479488in}}%
\pgfpathlineto{\pgfqpoint{3.310833in}{2.481336in}}%
\pgfpathlineto{\pgfqpoint{3.312089in}{2.481336in}}%
\pgfpathlineto{\pgfqpoint{3.312089in}{2.483184in}}%
\pgfpathlineto{\pgfqpoint{3.315138in}{2.483184in}}%
\pgfpathlineto{\pgfqpoint{3.315138in}{2.485032in}}%
\pgfpathlineto{\pgfqpoint{3.316723in}{2.485032in}}%
\pgfpathlineto{\pgfqpoint{3.316938in}{2.488728in}}%
\pgfpathlineto{\pgfqpoint{3.318839in}{2.488728in}}%
\pgfpathlineto{\pgfqpoint{3.319903in}{2.494272in}}%
\pgfpathlineto{\pgfqpoint{3.321036in}{2.494272in}}%
\pgfpathlineto{\pgfqpoint{3.322100in}{2.497968in}}%
\pgfpathlineto{\pgfqpoint{3.322252in}{2.497968in}}%
\pgfpathlineto{\pgfqpoint{3.322829in}{2.503512in}}%
\pgfpathlineto{\pgfqpoint{3.323395in}{2.503512in}}%
\pgfpathlineto{\pgfqpoint{3.323463in}{2.507208in}}%
\pgfpathlineto{\pgfqpoint{3.325478in}{2.507208in}}%
\pgfpathlineto{\pgfqpoint{3.325478in}{2.509056in}}%
\pgfpathlineto{\pgfqpoint{3.330726in}{2.509056in}}%
\pgfpathlineto{\pgfqpoint{3.331606in}{2.514600in}}%
\pgfpathlineto{\pgfqpoint{3.333844in}{2.514600in}}%
\pgfpathlineto{\pgfqpoint{3.334690in}{2.520144in}}%
\pgfpathlineto{\pgfqpoint{3.336101in}{2.520144in}}%
\pgfpathlineto{\pgfqpoint{3.337186in}{2.523840in}}%
\pgfpathlineto{\pgfqpoint{3.340750in}{2.523840in}}%
\pgfpathlineto{\pgfqpoint{3.341619in}{2.529384in}}%
\pgfpathlineto{\pgfqpoint{3.342555in}{2.529384in}}%
\pgfpathlineto{\pgfqpoint{3.343479in}{2.534928in}}%
\pgfpathlineto{\pgfqpoint{3.343749in}{2.534928in}}%
\pgfpathlineto{\pgfqpoint{3.344403in}{2.542320in}}%
\pgfpathlineto{\pgfqpoint{3.346159in}{2.542320in}}%
\pgfpathlineto{\pgfqpoint{3.346159in}{2.544168in}}%
\pgfpathlineto{\pgfqpoint{3.348414in}{2.544168in}}%
\pgfpathlineto{\pgfqpoint{3.349303in}{2.551560in}}%
\pgfpathlineto{\pgfqpoint{3.350093in}{2.551560in}}%
\pgfpathlineto{\pgfqpoint{3.350093in}{2.553408in}}%
\pgfpathlineto{\pgfqpoint{3.351727in}{2.553408in}}%
\pgfpathlineto{\pgfqpoint{3.352603in}{2.557104in}}%
\pgfpathlineto{\pgfqpoint{3.357363in}{2.557104in}}%
\pgfpathlineto{\pgfqpoint{3.358071in}{2.560800in}}%
\pgfpathlineto{\pgfqpoint{3.358583in}{2.560800in}}%
\pgfpathlineto{\pgfqpoint{3.358583in}{2.562648in}}%
\pgfpathlineto{\pgfqpoint{3.359861in}{2.562648in}}%
\pgfpathlineto{\pgfqpoint{3.360388in}{2.566344in}}%
\pgfpathlineto{\pgfqpoint{3.361198in}{2.566344in}}%
\pgfpathlineto{\pgfqpoint{3.361198in}{2.568192in}}%
\pgfpathlineto{\pgfqpoint{3.362629in}{2.568192in}}%
\pgfpathlineto{\pgfqpoint{3.362629in}{2.570040in}}%
\pgfpathlineto{\pgfqpoint{3.365054in}{2.570040in}}%
\pgfpathlineto{\pgfqpoint{3.365093in}{2.573736in}}%
\pgfpathlineto{\pgfqpoint{3.367321in}{2.573736in}}%
\pgfpathlineto{\pgfqpoint{3.368027in}{2.579280in}}%
\pgfpathlineto{\pgfqpoint{3.369397in}{2.579280in}}%
\pgfpathlineto{\pgfqpoint{3.370418in}{2.586672in}}%
\pgfpathlineto{\pgfqpoint{3.371057in}{2.586672in}}%
\pgfpathlineto{\pgfqpoint{3.371057in}{2.588520in}}%
\pgfpathlineto{\pgfqpoint{3.372312in}{2.588520in}}%
\pgfpathlineto{\pgfqpoint{3.372312in}{2.590368in}}%
\pgfpathlineto{\pgfqpoint{3.378555in}{2.590368in}}%
\pgfpathlineto{\pgfqpoint{3.379341in}{2.595912in}}%
\pgfpathlineto{\pgfqpoint{3.382820in}{2.595912in}}%
\pgfpathlineto{\pgfqpoint{3.383524in}{2.599608in}}%
\pgfpathlineto{\pgfqpoint{3.384114in}{2.599608in}}%
\pgfpathlineto{\pgfqpoint{3.384114in}{2.601456in}}%
\pgfpathlineto{\pgfqpoint{3.386393in}{2.601456in}}%
\pgfpathlineto{\pgfqpoint{3.386874in}{2.607000in}}%
\pgfpathlineto{\pgfqpoint{3.389040in}{2.607000in}}%
\pgfpathlineto{\pgfqpoint{3.389040in}{2.608848in}}%
\pgfpathlineto{\pgfqpoint{3.391660in}{2.608848in}}%
\pgfpathlineto{\pgfqpoint{3.391660in}{2.610696in}}%
\pgfpathlineto{\pgfqpoint{3.393481in}{2.610696in}}%
\pgfpathlineto{\pgfqpoint{3.394197in}{2.614392in}}%
\pgfpathlineto{\pgfqpoint{3.395418in}{2.614392in}}%
\pgfpathlineto{\pgfqpoint{3.395732in}{2.619936in}}%
\pgfpathlineto{\pgfqpoint{3.396813in}{2.619936in}}%
\pgfpathlineto{\pgfqpoint{3.397869in}{2.625480in}}%
\pgfpathlineto{\pgfqpoint{3.399353in}{2.625480in}}%
\pgfpathlineto{\pgfqpoint{3.399395in}{2.629176in}}%
\pgfpathlineto{\pgfqpoint{3.401124in}{2.629176in}}%
\pgfpathlineto{\pgfqpoint{3.401124in}{2.631024in}}%
\pgfpathlineto{\pgfqpoint{3.404704in}{2.631024in}}%
\pgfpathlineto{\pgfqpoint{3.405689in}{2.634720in}}%
\pgfpathlineto{\pgfqpoint{3.408025in}{2.634720in}}%
\pgfpathlineto{\pgfqpoint{3.408996in}{2.642112in}}%
\pgfpathlineto{\pgfqpoint{3.410621in}{2.642112in}}%
\pgfpathlineto{\pgfqpoint{3.410621in}{2.643960in}}%
\pgfpathlineto{\pgfqpoint{3.412845in}{2.643960in}}%
\pgfpathlineto{\pgfqpoint{3.412845in}{2.645808in}}%
\pgfpathlineto{\pgfqpoint{3.414649in}{2.645808in}}%
\pgfpathlineto{\pgfqpoint{3.415311in}{2.651352in}}%
\pgfpathlineto{\pgfqpoint{3.416697in}{2.651352in}}%
\pgfpathlineto{\pgfqpoint{3.417769in}{2.655048in}}%
\pgfpathlineto{\pgfqpoint{3.421203in}{2.655048in}}%
\pgfpathlineto{\pgfqpoint{3.421845in}{2.660592in}}%
\pgfpathlineto{\pgfqpoint{3.422887in}{2.660592in}}%
\pgfpathlineto{\pgfqpoint{3.423913in}{2.664288in}}%
\pgfpathlineto{\pgfqpoint{3.425890in}{2.664288in}}%
\pgfpathlineto{\pgfqpoint{3.425890in}{2.666136in}}%
\pgfpathlineto{\pgfqpoint{3.427410in}{2.666136in}}%
\pgfpathlineto{\pgfqpoint{3.427410in}{2.667984in}}%
\pgfpathlineto{\pgfqpoint{3.431109in}{2.667984in}}%
\pgfpathlineto{\pgfqpoint{3.432073in}{2.671680in}}%
\pgfpathlineto{\pgfqpoint{3.432261in}{2.671680in}}%
\pgfpathlineto{\pgfqpoint{3.432261in}{2.673528in}}%
\pgfpathlineto{\pgfqpoint{3.433877in}{2.673528in}}%
\pgfpathlineto{\pgfqpoint{3.434982in}{2.680920in}}%
\pgfpathlineto{\pgfqpoint{3.439067in}{2.680920in}}%
\pgfpathlineto{\pgfqpoint{3.439392in}{2.686464in}}%
\pgfpathlineto{\pgfqpoint{3.440764in}{2.686464in}}%
\pgfpathlineto{\pgfqpoint{3.441680in}{2.692008in}}%
\pgfpathlineto{\pgfqpoint{3.442747in}{2.692008in}}%
\pgfpathlineto{\pgfqpoint{3.443688in}{2.695704in}}%
\pgfpathlineto{\pgfqpoint{3.445248in}{2.695704in}}%
\pgfpathlineto{\pgfqpoint{3.445248in}{2.697552in}}%
\pgfpathlineto{\pgfqpoint{3.447261in}{2.697552in}}%
\pgfpathlineto{\pgfqpoint{3.448098in}{2.703096in}}%
\pgfpathlineto{\pgfqpoint{3.449723in}{2.703096in}}%
\pgfpathlineto{\pgfqpoint{3.449723in}{2.704944in}}%
\pgfpathlineto{\pgfqpoint{3.451945in}{2.704944in}}%
\pgfpathlineto{\pgfqpoint{3.451945in}{2.706792in}}%
\pgfpathlineto{\pgfqpoint{3.453214in}{2.706792in}}%
\pgfpathlineto{\pgfqpoint{3.453836in}{2.710488in}}%
\pgfpathlineto{\pgfqpoint{3.456230in}{2.710488in}}%
\pgfpathlineto{\pgfqpoint{3.456230in}{2.712336in}}%
\pgfpathlineto{\pgfqpoint{3.457455in}{2.712336in}}%
\pgfpathlineto{\pgfqpoint{3.457455in}{2.714184in}}%
\pgfpathlineto{\pgfqpoint{3.458697in}{2.714184in}}%
\pgfpathlineto{\pgfqpoint{3.458697in}{2.716032in}}%
\pgfpathlineto{\pgfqpoint{3.460808in}{2.716032in}}%
\pgfpathlineto{\pgfqpoint{3.460808in}{2.717880in}}%
\pgfpathlineto{\pgfqpoint{3.462266in}{2.717880in}}%
\pgfpathlineto{\pgfqpoint{3.462266in}{2.719728in}}%
\pgfpathlineto{\pgfqpoint{3.465320in}{2.719728in}}%
\pgfpathlineto{\pgfqpoint{3.465448in}{2.723424in}}%
\pgfpathlineto{\pgfqpoint{3.467023in}{2.723424in}}%
\pgfpathlineto{\pgfqpoint{3.467889in}{2.727120in}}%
\pgfpathlineto{\pgfqpoint{3.468679in}{2.727120in}}%
\pgfpathlineto{\pgfqpoint{3.468922in}{2.732664in}}%
\pgfpathlineto{\pgfqpoint{3.470207in}{2.732664in}}%
\pgfpathlineto{\pgfqpoint{3.470207in}{2.734512in}}%
\pgfpathlineto{\pgfqpoint{3.471874in}{2.734512in}}%
\pgfpathlineto{\pgfqpoint{3.471874in}{2.736360in}}%
\pgfpathlineto{\pgfqpoint{3.474031in}{2.736360in}}%
\pgfpathlineto{\pgfqpoint{3.474031in}{2.738208in}}%
\pgfpathlineto{\pgfqpoint{3.475321in}{2.738208in}}%
\pgfpathlineto{\pgfqpoint{3.475623in}{2.741904in}}%
\pgfpathlineto{\pgfqpoint{3.478225in}{2.741904in}}%
\pgfpathlineto{\pgfqpoint{3.479057in}{2.747448in}}%
\pgfpathlineto{\pgfqpoint{3.481425in}{2.747448in}}%
\pgfpathlineto{\pgfqpoint{3.482267in}{2.751144in}}%
\pgfpathlineto{\pgfqpoint{3.484707in}{2.751144in}}%
\pgfpathlineto{\pgfqpoint{3.484707in}{2.752992in}}%
\pgfpathlineto{\pgfqpoint{3.486692in}{2.752992in}}%
\pgfpathlineto{\pgfqpoint{3.486692in}{2.754840in}}%
\pgfpathlineto{\pgfqpoint{3.488054in}{2.754840in}}%
\pgfpathlineto{\pgfqpoint{3.488892in}{2.760384in}}%
\pgfpathlineto{\pgfqpoint{3.490517in}{2.760384in}}%
\pgfpathlineto{\pgfqpoint{3.491605in}{2.765928in}}%
\pgfpathlineto{\pgfqpoint{3.493416in}{2.765928in}}%
\pgfpathlineto{\pgfqpoint{3.494434in}{2.769624in}}%
\pgfpathlineto{\pgfqpoint{3.494778in}{2.769624in}}%
\pgfpathlineto{\pgfqpoint{3.494886in}{2.773320in}}%
\pgfpathlineto{\pgfqpoint{3.501494in}{2.773320in}}%
\pgfpathlineto{\pgfqpoint{3.501737in}{2.777016in}}%
\pgfpathlineto{\pgfqpoint{3.504174in}{2.777016in}}%
\pgfpathlineto{\pgfqpoint{3.505089in}{2.782560in}}%
\pgfpathlineto{\pgfqpoint{3.506077in}{2.782560in}}%
\pgfpathlineto{\pgfqpoint{3.506077in}{2.784408in}}%
\pgfpathlineto{\pgfqpoint{3.507456in}{2.784408in}}%
\pgfpathlineto{\pgfqpoint{3.508328in}{2.788104in}}%
\pgfpathlineto{\pgfqpoint{3.512577in}{2.788104in}}%
\pgfpathlineto{\pgfqpoint{3.512875in}{2.791800in}}%
\pgfpathlineto{\pgfqpoint{3.513927in}{2.791800in}}%
\pgfpathlineto{\pgfqpoint{3.514796in}{2.797344in}}%
\pgfpathlineto{\pgfqpoint{3.516658in}{2.797344in}}%
\pgfpathlineto{\pgfqpoint{3.517392in}{2.801040in}}%
\pgfpathlineto{\pgfqpoint{3.517940in}{2.801040in}}%
\pgfpathlineto{\pgfqpoint{3.517940in}{2.802888in}}%
\pgfpathlineto{\pgfqpoint{3.519170in}{2.802888in}}%
\pgfpathlineto{\pgfqpoint{3.519170in}{2.804736in}}%
\pgfpathlineto{\pgfqpoint{3.521019in}{2.804736in}}%
\pgfpathlineto{\pgfqpoint{3.521019in}{2.806584in}}%
\pgfpathlineto{\pgfqpoint{3.522659in}{2.806584in}}%
\pgfpathlineto{\pgfqpoint{3.522659in}{2.808432in}}%
\pgfpathlineto{\pgfqpoint{3.526322in}{2.808432in}}%
\pgfpathlineto{\pgfqpoint{3.527383in}{2.813976in}}%
\pgfpathlineto{\pgfqpoint{3.529940in}{2.813976in}}%
\pgfpathlineto{\pgfqpoint{3.530719in}{2.817672in}}%
\pgfpathlineto{\pgfqpoint{3.531945in}{2.817672in}}%
\pgfpathlineto{\pgfqpoint{3.531945in}{2.819520in}}%
\pgfpathlineto{\pgfqpoint{3.533153in}{2.819520in}}%
\pgfpathlineto{\pgfqpoint{3.533153in}{2.821368in}}%
\pgfpathlineto{\pgfqpoint{3.534464in}{2.821368in}}%
\pgfpathlineto{\pgfqpoint{3.535532in}{2.825064in}}%
\pgfpathlineto{\pgfqpoint{3.538563in}{2.825064in}}%
\pgfpathlineto{\pgfqpoint{3.538654in}{2.828760in}}%
\pgfpathlineto{\pgfqpoint{3.539894in}{2.828760in}}%
\pgfpathlineto{\pgfqpoint{3.540917in}{2.832456in}}%
\pgfpathlineto{\pgfqpoint{3.542495in}{2.832456in}}%
\pgfpathlineto{\pgfqpoint{3.542495in}{2.834304in}}%
\pgfpathlineto{\pgfqpoint{3.543964in}{2.834304in}}%
\pgfpathlineto{\pgfqpoint{3.544836in}{2.839848in}}%
\pgfpathlineto{\pgfqpoint{3.545205in}{2.839848in}}%
\pgfpathlineto{\pgfqpoint{3.545205in}{2.841696in}}%
\pgfpathlineto{\pgfqpoint{3.548481in}{2.841696in}}%
\pgfpathlineto{\pgfqpoint{3.548905in}{2.845392in}}%
\pgfpathlineto{\pgfqpoint{3.550673in}{2.845392in}}%
\pgfpathlineto{\pgfqpoint{3.551571in}{2.850936in}}%
\pgfpathlineto{\pgfqpoint{3.552151in}{2.850936in}}%
\pgfpathlineto{\pgfqpoint{3.552742in}{2.856480in}}%
\pgfpathlineto{\pgfqpoint{3.554929in}{2.856480in}}%
\pgfpathlineto{\pgfqpoint{3.554929in}{2.858328in}}%
\pgfpathlineto{\pgfqpoint{3.558026in}{2.858328in}}%
\pgfpathlineto{\pgfqpoint{3.558026in}{2.860176in}}%
\pgfpathlineto{\pgfqpoint{3.560714in}{2.860176in}}%
\pgfpathlineto{\pgfqpoint{3.561445in}{2.865720in}}%
\pgfpathlineto{\pgfqpoint{3.561868in}{2.865720in}}%
\pgfpathlineto{\pgfqpoint{3.561868in}{2.867568in}}%
\pgfpathlineto{\pgfqpoint{3.563130in}{2.867568in}}%
\pgfpathlineto{\pgfqpoint{3.563130in}{2.869416in}}%
\pgfpathlineto{\pgfqpoint{3.564627in}{2.869416in}}%
\pgfpathlineto{\pgfqpoint{3.564992in}{2.873112in}}%
\pgfpathlineto{\pgfqpoint{3.568232in}{2.873112in}}%
\pgfpathlineto{\pgfqpoint{3.568232in}{2.874960in}}%
\pgfpathlineto{\pgfqpoint{3.569663in}{2.874960in}}%
\pgfpathlineto{\pgfqpoint{3.569663in}{2.876808in}}%
\pgfpathlineto{\pgfqpoint{3.571289in}{2.876808in}}%
\pgfpathlineto{\pgfqpoint{3.571289in}{2.878656in}}%
\pgfpathlineto{\pgfqpoint{3.573090in}{2.878656in}}%
\pgfpathlineto{\pgfqpoint{3.574039in}{2.884200in}}%
\pgfpathlineto{\pgfqpoint{3.576584in}{2.884200in}}%
\pgfpathlineto{\pgfqpoint{3.577161in}{2.889744in}}%
\pgfpathlineto{\pgfqpoint{3.577718in}{2.889744in}}%
\pgfpathlineto{\pgfqpoint{3.578193in}{2.893440in}}%
\pgfpathlineto{\pgfqpoint{3.579241in}{2.893440in}}%
\pgfpathlineto{\pgfqpoint{3.579241in}{2.895288in}}%
\pgfpathlineto{\pgfqpoint{3.580519in}{2.895288in}}%
\pgfpathlineto{\pgfqpoint{3.580519in}{2.897136in}}%
\pgfpathlineto{\pgfqpoint{3.585481in}{2.897136in}}%
\pgfpathlineto{\pgfqpoint{3.585968in}{2.900832in}}%
\pgfpathlineto{\pgfqpoint{3.586692in}{2.900832in}}%
\pgfpathlineto{\pgfqpoint{3.587535in}{2.906376in}}%
\pgfpathlineto{\pgfqpoint{3.588368in}{2.906376in}}%
\pgfpathlineto{\pgfqpoint{3.588368in}{2.908224in}}%
\pgfpathlineto{\pgfqpoint{3.590049in}{2.908224in}}%
\pgfpathlineto{\pgfqpoint{3.590803in}{2.911920in}}%
\pgfpathlineto{\pgfqpoint{3.593652in}{2.911920in}}%
\pgfpathlineto{\pgfqpoint{3.593652in}{2.913768in}}%
\pgfpathlineto{\pgfqpoint{3.598470in}{2.913768in}}%
\pgfpathlineto{\pgfqpoint{3.599269in}{2.921160in}}%
\pgfpathlineto{\pgfqpoint{3.602551in}{2.921160in}}%
\pgfpathlineto{\pgfqpoint{3.602986in}{2.926704in}}%
\pgfpathlineto{\pgfqpoint{3.603790in}{2.926704in}}%
\pgfpathlineto{\pgfqpoint{3.603790in}{2.928552in}}%
\pgfpathlineto{\pgfqpoint{3.605075in}{2.928552in}}%
\pgfpathlineto{\pgfqpoint{3.605075in}{2.930400in}}%
\pgfpathlineto{\pgfqpoint{3.606448in}{2.930400in}}%
\pgfpathlineto{\pgfqpoint{3.606448in}{2.932248in}}%
\pgfpathlineto{\pgfqpoint{3.608161in}{2.932248in}}%
\pgfpathlineto{\pgfqpoint{3.608161in}{2.934096in}}%
\pgfpathlineto{\pgfqpoint{3.610961in}{2.934096in}}%
\pgfpathlineto{\pgfqpoint{3.611659in}{2.937792in}}%
\pgfpathlineto{\pgfqpoint{3.612174in}{2.937792in}}%
\pgfpathlineto{\pgfqpoint{3.613144in}{2.941488in}}%
\pgfpathlineto{\pgfqpoint{3.614201in}{2.941488in}}%
\pgfpathlineto{\pgfqpoint{3.614276in}{2.945184in}}%
\pgfpathlineto{\pgfqpoint{3.615683in}{2.945184in}}%
\pgfpathlineto{\pgfqpoint{3.616091in}{2.948880in}}%
\pgfpathlineto{\pgfqpoint{3.619123in}{2.948880in}}%
\pgfpathlineto{\pgfqpoint{3.619945in}{2.952576in}}%
\pgfpathlineto{\pgfqpoint{3.623299in}{2.952576in}}%
\pgfpathlineto{\pgfqpoint{3.624200in}{2.958120in}}%
\pgfpathlineto{\pgfqpoint{3.625083in}{2.958120in}}%
\pgfpathlineto{\pgfqpoint{3.625083in}{2.959968in}}%
\pgfpathlineto{\pgfqpoint{3.626838in}{2.959968in}}%
\pgfpathlineto{\pgfqpoint{3.626918in}{2.963664in}}%
\pgfpathlineto{\pgfqpoint{3.629227in}{2.963664in}}%
\pgfpathlineto{\pgfqpoint{3.629227in}{2.965512in}}%
\pgfpathlineto{\pgfqpoint{3.632293in}{2.965512in}}%
\pgfpathlineto{\pgfqpoint{3.632826in}{2.969208in}}%
\pgfpathlineto{\pgfqpoint{3.633670in}{2.969208in}}%
\pgfpathlineto{\pgfqpoint{3.633685in}{2.972904in}}%
\pgfpathlineto{\pgfqpoint{3.634940in}{2.972904in}}%
\pgfpathlineto{\pgfqpoint{3.634940in}{2.974752in}}%
\pgfpathlineto{\pgfqpoint{3.636593in}{2.974752in}}%
\pgfpathlineto{\pgfqpoint{3.636593in}{2.976600in}}%
\pgfpathlineto{\pgfqpoint{3.640117in}{2.976600in}}%
\pgfpathlineto{\pgfqpoint{3.640117in}{2.978448in}}%
\pgfpathlineto{\pgfqpoint{3.641244in}{2.978448in}}%
\pgfpathlineto{\pgfqpoint{3.641244in}{2.980296in}}%
\pgfpathlineto{\pgfqpoint{3.644609in}{2.980296in}}%
\pgfpathlineto{\pgfqpoint{3.644609in}{2.982144in}}%
\pgfpathlineto{\pgfqpoint{3.645735in}{2.982144in}}%
\pgfpathlineto{\pgfqpoint{3.645756in}{2.985840in}}%
\pgfpathlineto{\pgfqpoint{3.648641in}{2.985840in}}%
\pgfpathlineto{\pgfqpoint{3.649700in}{2.989536in}}%
\pgfpathlineto{\pgfqpoint{3.650168in}{2.989536in}}%
\pgfpathlineto{\pgfqpoint{3.650168in}{2.991384in}}%
\pgfpathlineto{\pgfqpoint{3.652150in}{2.991384in}}%
\pgfpathlineto{\pgfqpoint{3.652150in}{2.993232in}}%
\pgfpathlineto{\pgfqpoint{3.657693in}{2.993232in}}%
\pgfpathlineto{\pgfqpoint{3.658592in}{2.996928in}}%
\pgfpathlineto{\pgfqpoint{3.659295in}{2.996928in}}%
\pgfpathlineto{\pgfqpoint{3.659497in}{3.000624in}}%
\pgfpathlineto{\pgfqpoint{3.660442in}{3.000624in}}%
\pgfpathlineto{\pgfqpoint{3.661549in}{3.006168in}}%
\pgfpathlineto{\pgfqpoint{3.664470in}{3.006168in}}%
\pgfpathlineto{\pgfqpoint{3.665411in}{3.009864in}}%
\pgfpathlineto{\pgfqpoint{3.670262in}{3.009864in}}%
\pgfpathlineto{\pgfqpoint{3.671364in}{3.015408in}}%
\pgfpathlineto{\pgfqpoint{3.674215in}{3.015408in}}%
\pgfpathlineto{\pgfqpoint{3.674447in}{3.019104in}}%
\pgfpathlineto{\pgfqpoint{3.675745in}{3.019104in}}%
\pgfpathlineto{\pgfqpoint{3.675745in}{3.020952in}}%
\pgfpathlineto{\pgfqpoint{3.676860in}{3.020952in}}%
\pgfpathlineto{\pgfqpoint{3.677761in}{3.028344in}}%
\pgfpathlineto{\pgfqpoint{3.680056in}{3.028344in}}%
\pgfpathlineto{\pgfqpoint{3.680056in}{3.030192in}}%
\pgfpathlineto{\pgfqpoint{3.681467in}{3.030192in}}%
\pgfpathlineto{\pgfqpoint{3.681467in}{3.032040in}}%
\pgfpathlineto{\pgfqpoint{3.683411in}{3.032040in}}%
\pgfpathlineto{\pgfqpoint{3.683896in}{3.035736in}}%
\pgfpathlineto{\pgfqpoint{3.684550in}{3.035736in}}%
\pgfpathlineto{\pgfqpoint{3.685656in}{3.039432in}}%
\pgfpathlineto{\pgfqpoint{3.686142in}{3.039432in}}%
\pgfpathlineto{\pgfqpoint{3.686142in}{3.041280in}}%
\pgfpathlineto{\pgfqpoint{3.688297in}{3.041280in}}%
\pgfpathlineto{\pgfqpoint{3.688297in}{3.043128in}}%
\pgfpathlineto{\pgfqpoint{3.690972in}{3.043128in}}%
\pgfpathlineto{\pgfqpoint{3.690972in}{3.044976in}}%
\pgfpathlineto{\pgfqpoint{3.694733in}{3.044976in}}%
\pgfpathlineto{\pgfqpoint{3.694919in}{3.048672in}}%
\pgfpathlineto{\pgfqpoint{3.696526in}{3.048672in}}%
\pgfpathlineto{\pgfqpoint{3.696961in}{3.052368in}}%
\pgfpathlineto{\pgfqpoint{3.698274in}{3.052368in}}%
\pgfpathlineto{\pgfqpoint{3.698274in}{3.054216in}}%
\pgfpathlineto{\pgfqpoint{3.699541in}{3.054216in}}%
\pgfpathlineto{\pgfqpoint{3.699541in}{3.056064in}}%
\pgfpathlineto{\pgfqpoint{3.703827in}{3.056064in}}%
\pgfpathlineto{\pgfqpoint{3.703976in}{3.059760in}}%
\pgfpathlineto{\pgfqpoint{3.705147in}{3.059760in}}%
\pgfpathlineto{\pgfqpoint{3.705754in}{3.063456in}}%
\pgfpathlineto{\pgfqpoint{3.706676in}{3.063456in}}%
\pgfpathlineto{\pgfqpoint{3.706676in}{3.065304in}}%
\pgfpathlineto{\pgfqpoint{3.708654in}{3.065304in}}%
\pgfpathlineto{\pgfqpoint{3.709024in}{3.069000in}}%
\pgfpathlineto{\pgfqpoint{3.711124in}{3.069000in}}%
\pgfpathlineto{\pgfqpoint{3.711404in}{3.072696in}}%
\pgfpathlineto{\pgfqpoint{3.715512in}{3.072696in}}%
\pgfpathlineto{\pgfqpoint{3.716534in}{3.076392in}}%
\pgfpathlineto{\pgfqpoint{3.717195in}{3.076392in}}%
\pgfpathlineto{\pgfqpoint{3.717195in}{3.078240in}}%
\pgfpathlineto{\pgfqpoint{3.720021in}{3.078240in}}%
\pgfpathlineto{\pgfqpoint{3.720021in}{3.080088in}}%
\pgfpathlineto{\pgfqpoint{3.722712in}{3.080088in}}%
\pgfpathlineto{\pgfqpoint{3.723354in}{3.083784in}}%
\pgfpathlineto{\pgfqpoint{3.723995in}{3.083784in}}%
\pgfpathlineto{\pgfqpoint{3.723995in}{3.085632in}}%
\pgfpathlineto{\pgfqpoint{3.728937in}{3.085632in}}%
\pgfpathlineto{\pgfqpoint{3.728943in}{3.089328in}}%
\pgfpathlineto{\pgfqpoint{3.731088in}{3.089328in}}%
\pgfpathlineto{\pgfqpoint{3.731935in}{3.093024in}}%
\pgfpathlineto{\pgfqpoint{3.732626in}{3.093024in}}%
\pgfpathlineto{\pgfqpoint{3.732626in}{3.094872in}}%
\pgfpathlineto{\pgfqpoint{3.733940in}{3.094872in}}%
\pgfpathlineto{\pgfqpoint{3.733940in}{3.096720in}}%
\pgfpathlineto{\pgfqpoint{3.736643in}{3.096720in}}%
\pgfpathlineto{\pgfqpoint{3.736643in}{3.098568in}}%
\pgfpathlineto{\pgfqpoint{3.740666in}{3.098568in}}%
\pgfpathlineto{\pgfqpoint{3.741659in}{3.102264in}}%
\pgfpathlineto{\pgfqpoint{3.742628in}{3.102264in}}%
\pgfpathlineto{\pgfqpoint{3.742628in}{3.104112in}}%
\pgfpathlineto{\pgfqpoint{3.743911in}{3.104112in}}%
\pgfpathlineto{\pgfqpoint{3.743911in}{3.105960in}}%
\pgfpathlineto{\pgfqpoint{3.745381in}{3.105960in}}%
\pgfpathlineto{\pgfqpoint{3.745381in}{3.107808in}}%
\pgfpathlineto{\pgfqpoint{3.747586in}{3.107808in}}%
\pgfpathlineto{\pgfqpoint{3.748582in}{3.113352in}}%
\pgfpathlineto{\pgfqpoint{3.749083in}{3.113352in}}%
\pgfpathlineto{\pgfqpoint{3.749083in}{3.115200in}}%
\pgfpathlineto{\pgfqpoint{3.754174in}{3.115200in}}%
\pgfpathlineto{\pgfqpoint{3.754328in}{3.118896in}}%
\pgfpathlineto{\pgfqpoint{3.756109in}{3.118896in}}%
\pgfpathlineto{\pgfqpoint{3.756864in}{3.122592in}}%
\pgfpathlineto{\pgfqpoint{3.757296in}{3.122592in}}%
\pgfpathlineto{\pgfqpoint{3.757924in}{3.128136in}}%
\pgfpathlineto{\pgfqpoint{3.759315in}{3.128136in}}%
\pgfpathlineto{\pgfqpoint{3.759315in}{3.129984in}}%
\pgfpathlineto{\pgfqpoint{3.762234in}{3.129984in}}%
\pgfpathlineto{\pgfqpoint{3.762234in}{3.131832in}}%
\pgfpathlineto{\pgfqpoint{3.765710in}{3.131832in}}%
\pgfpathlineto{\pgfqpoint{3.766558in}{3.135528in}}%
\pgfpathlineto{\pgfqpoint{3.767446in}{3.135528in}}%
\pgfpathlineto{\pgfqpoint{3.768242in}{3.139224in}}%
\pgfpathlineto{\pgfqpoint{3.769369in}{3.139224in}}%
\pgfpathlineto{\pgfqpoint{3.769808in}{3.142920in}}%
\pgfpathlineto{\pgfqpoint{3.772698in}{3.142920in}}%
\pgfpathlineto{\pgfqpoint{3.772698in}{3.144768in}}%
\pgfpathlineto{\pgfqpoint{3.774009in}{3.144768in}}%
\pgfpathlineto{\pgfqpoint{3.774009in}{3.146616in}}%
\pgfpathlineto{\pgfqpoint{3.777742in}{3.146616in}}%
\pgfpathlineto{\pgfqpoint{3.777742in}{3.148464in}}%
\pgfpathlineto{\pgfqpoint{3.779322in}{3.148464in}}%
\pgfpathlineto{\pgfqpoint{3.779972in}{3.155856in}}%
\pgfpathlineto{\pgfqpoint{3.780501in}{3.155856in}}%
\pgfpathlineto{\pgfqpoint{3.781207in}{3.159552in}}%
\pgfpathlineto{\pgfqpoint{3.782271in}{3.159552in}}%
\pgfpathlineto{\pgfqpoint{3.783079in}{3.165096in}}%
\pgfpathlineto{\pgfqpoint{3.784490in}{3.165096in}}%
\pgfpathlineto{\pgfqpoint{3.784490in}{3.166944in}}%
\pgfpathlineto{\pgfqpoint{3.787273in}{3.166944in}}%
\pgfpathlineto{\pgfqpoint{3.787532in}{3.170640in}}%
\pgfpathlineto{\pgfqpoint{3.790672in}{3.170640in}}%
\pgfpathlineto{\pgfqpoint{3.790712in}{3.174336in}}%
\pgfpathlineto{\pgfqpoint{3.793269in}{3.174336in}}%
\pgfpathlineto{\pgfqpoint{3.794363in}{3.178032in}}%
\pgfpathlineto{\pgfqpoint{3.794658in}{3.178032in}}%
\pgfpathlineto{\pgfqpoint{3.794658in}{3.179880in}}%
\pgfpathlineto{\pgfqpoint{3.797956in}{3.179880in}}%
\pgfpathlineto{\pgfqpoint{3.797956in}{3.181728in}}%
\pgfpathlineto{\pgfqpoint{3.800437in}{3.181728in}}%
\pgfpathlineto{\pgfqpoint{3.800437in}{3.183576in}}%
\pgfpathlineto{\pgfqpoint{3.803928in}{3.183576in}}%
\pgfpathlineto{\pgfqpoint{3.804426in}{3.187272in}}%
\pgfpathlineto{\pgfqpoint{3.807023in}{3.187272in}}%
\pgfpathlineto{\pgfqpoint{3.807750in}{3.192816in}}%
\pgfpathlineto{\pgfqpoint{3.808286in}{3.192816in}}%
\pgfpathlineto{\pgfqpoint{3.808286in}{3.194664in}}%
\pgfpathlineto{\pgfqpoint{3.812403in}{3.194664in}}%
\pgfpathlineto{\pgfqpoint{3.813021in}{3.200208in}}%
\pgfpathlineto{\pgfqpoint{3.815599in}{3.200208in}}%
\pgfpathlineto{\pgfqpoint{3.816144in}{3.205752in}}%
\pgfpathlineto{\pgfqpoint{3.817495in}{3.205752in}}%
\pgfpathlineto{\pgfqpoint{3.818527in}{3.213144in}}%
\pgfpathlineto{\pgfqpoint{3.819298in}{3.213144in}}%
\pgfpathlineto{\pgfqpoint{3.819298in}{3.214992in}}%
\pgfpathlineto{\pgfqpoint{3.824447in}{3.214992in}}%
\pgfpathlineto{\pgfqpoint{3.825249in}{3.218688in}}%
\pgfpathlineto{\pgfqpoint{3.825794in}{3.218688in}}%
\pgfpathlineto{\pgfqpoint{3.826754in}{3.224232in}}%
\pgfpathlineto{\pgfqpoint{3.827079in}{3.224232in}}%
\pgfpathlineto{\pgfqpoint{3.827579in}{3.227928in}}%
\pgfpathlineto{\pgfqpoint{3.828482in}{3.227928in}}%
\pgfpathlineto{\pgfqpoint{3.828482in}{3.229776in}}%
\pgfpathlineto{\pgfqpoint{3.829631in}{3.229776in}}%
\pgfpathlineto{\pgfqpoint{3.829631in}{3.231624in}}%
\pgfpathlineto{\pgfqpoint{3.833079in}{3.231624in}}%
\pgfpathlineto{\pgfqpoint{3.833079in}{3.233472in}}%
\pgfpathlineto{\pgfqpoint{3.836784in}{3.233472in}}%
\pgfpathlineto{\pgfqpoint{3.837685in}{3.237168in}}%
\pgfpathlineto{\pgfqpoint{3.837993in}{3.237168in}}%
\pgfpathlineto{\pgfqpoint{3.837993in}{3.239016in}}%
\pgfpathlineto{\pgfqpoint{3.839806in}{3.239016in}}%
\pgfpathlineto{\pgfqpoint{3.840648in}{3.242712in}}%
\pgfpathlineto{\pgfqpoint{3.841297in}{3.242712in}}%
\pgfpathlineto{\pgfqpoint{3.841297in}{3.244560in}}%
\pgfpathlineto{\pgfqpoint{3.843680in}{3.244560in}}%
\pgfpathlineto{\pgfqpoint{3.843680in}{3.246408in}}%
\pgfpathlineto{\pgfqpoint{3.845639in}{3.246408in}}%
\pgfpathlineto{\pgfqpoint{3.845639in}{3.248256in}}%
\pgfpathlineto{\pgfqpoint{3.847612in}{3.248256in}}%
\pgfpathlineto{\pgfqpoint{3.847612in}{3.250104in}}%
\pgfpathlineto{\pgfqpoint{3.849363in}{3.250104in}}%
\pgfpathlineto{\pgfqpoint{3.849363in}{3.251952in}}%
\pgfpathlineto{\pgfqpoint{3.852112in}{3.251952in}}%
\pgfpathlineto{\pgfqpoint{3.852524in}{3.255648in}}%
\pgfpathlineto{\pgfqpoint{3.853254in}{3.255648in}}%
\pgfpathlineto{\pgfqpoint{3.853254in}{3.257496in}}%
\pgfpathlineto{\pgfqpoint{3.854664in}{3.257496in}}%
\pgfpathlineto{\pgfqpoint{3.854664in}{3.259344in}}%
\pgfpathlineto{\pgfqpoint{3.857927in}{3.259344in}}%
\pgfpathlineto{\pgfqpoint{3.858322in}{3.264888in}}%
\pgfpathlineto{\pgfqpoint{3.861325in}{3.264888in}}%
\pgfpathlineto{\pgfqpoint{3.862003in}{3.268584in}}%
\pgfpathlineto{\pgfqpoint{3.865041in}{3.268584in}}%
\pgfpathlineto{\pgfqpoint{3.865986in}{3.272280in}}%
\pgfpathlineto{\pgfqpoint{3.868432in}{3.272280in}}%
\pgfpathlineto{\pgfqpoint{3.868432in}{3.274128in}}%
\pgfpathlineto{\pgfqpoint{3.870713in}{3.274128in}}%
\pgfpathlineto{\pgfqpoint{3.870713in}{3.275976in}}%
\pgfpathlineto{\pgfqpoint{3.872273in}{3.275976in}}%
\pgfpathlineto{\pgfqpoint{3.872273in}{3.277824in}}%
\pgfpathlineto{\pgfqpoint{3.873977in}{3.277824in}}%
\pgfpathlineto{\pgfqpoint{3.874731in}{3.281520in}}%
\pgfpathlineto{\pgfqpoint{3.875402in}{3.281520in}}%
\pgfpathlineto{\pgfqpoint{3.875402in}{3.283368in}}%
\pgfpathlineto{\pgfqpoint{3.876876in}{3.283368in}}%
\pgfpathlineto{\pgfqpoint{3.877562in}{3.288912in}}%
\pgfpathlineto{\pgfqpoint{3.882424in}{3.288912in}}%
\pgfpathlineto{\pgfqpoint{3.883505in}{3.296304in}}%
\pgfpathlineto{\pgfqpoint{3.885081in}{3.296304in}}%
\pgfpathlineto{\pgfqpoint{3.885651in}{3.300000in}}%
\pgfpathlineto{\pgfqpoint{3.886896in}{3.300000in}}%
\pgfpathlineto{\pgfqpoint{3.888000in}{3.303696in}}%
\pgfpathlineto{\pgfqpoint{3.888605in}{3.303696in}}%
\pgfpathlineto{\pgfqpoint{3.888605in}{3.305544in}}%
\pgfpathlineto{\pgfqpoint{3.891061in}{3.305544in}}%
\pgfpathlineto{\pgfqpoint{3.891061in}{3.307392in}}%
\pgfpathlineto{\pgfqpoint{3.892377in}{3.307392in}}%
\pgfpathlineto{\pgfqpoint{3.892415in}{3.311088in}}%
\pgfpathlineto{\pgfqpoint{3.894408in}{3.311088in}}%
\pgfpathlineto{\pgfqpoint{3.894408in}{3.312936in}}%
\pgfpathlineto{\pgfqpoint{3.899485in}{3.312936in}}%
\pgfpathlineto{\pgfqpoint{3.900358in}{3.316632in}}%
\pgfpathlineto{\pgfqpoint{3.901880in}{3.316632in}}%
\pgfpathlineto{\pgfqpoint{3.902688in}{3.320328in}}%
\pgfpathlineto{\pgfqpoint{3.903683in}{3.320328in}}%
\pgfpathlineto{\pgfqpoint{3.904713in}{3.324024in}}%
\pgfpathlineto{\pgfqpoint{3.906538in}{3.324024in}}%
\pgfpathlineto{\pgfqpoint{3.907008in}{3.327720in}}%
\pgfpathlineto{\pgfqpoint{3.907651in}{3.327720in}}%
\pgfpathlineto{\pgfqpoint{3.907651in}{3.329568in}}%
\pgfpathlineto{\pgfqpoint{3.910411in}{3.329568in}}%
\pgfpathlineto{\pgfqpoint{3.910838in}{3.333264in}}%
\pgfpathlineto{\pgfqpoint{3.912949in}{3.333264in}}%
\pgfpathlineto{\pgfqpoint{3.912996in}{3.336960in}}%
\pgfpathlineto{\pgfqpoint{3.916345in}{3.336960in}}%
\pgfpathlineto{\pgfqpoint{3.917034in}{3.342504in}}%
\pgfpathlineto{\pgfqpoint{3.919210in}{3.342504in}}%
\pgfpathlineto{\pgfqpoint{3.919210in}{3.344352in}}%
\pgfpathlineto{\pgfqpoint{3.920695in}{3.344352in}}%
\pgfpathlineto{\pgfqpoint{3.921586in}{3.349896in}}%
\pgfpathlineto{\pgfqpoint{3.922451in}{3.349896in}}%
\pgfpathlineto{\pgfqpoint{3.922451in}{3.351744in}}%
\pgfpathlineto{\pgfqpoint{3.924073in}{3.351744in}}%
\pgfpathlineto{\pgfqpoint{3.924073in}{3.353592in}}%
\pgfpathlineto{\pgfqpoint{3.927062in}{3.353592in}}%
\pgfpathlineto{\pgfqpoint{3.927992in}{3.357288in}}%
\pgfpathlineto{\pgfqpoint{3.928350in}{3.357288in}}%
\pgfpathlineto{\pgfqpoint{3.928350in}{3.359136in}}%
\pgfpathlineto{\pgfqpoint{3.930561in}{3.359136in}}%
\pgfpathlineto{\pgfqpoint{3.931623in}{3.364680in}}%
\pgfpathlineto{\pgfqpoint{3.932192in}{3.364680in}}%
\pgfpathlineto{\pgfqpoint{3.932901in}{3.368376in}}%
\pgfpathlineto{\pgfqpoint{3.935333in}{3.368376in}}%
\pgfpathlineto{\pgfqpoint{3.936316in}{3.373920in}}%
\pgfpathlineto{\pgfqpoint{3.937555in}{3.373920in}}%
\pgfpathlineto{\pgfqpoint{3.937555in}{3.375768in}}%
\pgfpathlineto{\pgfqpoint{3.940839in}{3.375768in}}%
\pgfpathlineto{\pgfqpoint{3.941836in}{3.381312in}}%
\pgfpathlineto{\pgfqpoint{3.943989in}{3.381312in}}%
\pgfpathlineto{\pgfqpoint{3.944939in}{3.385008in}}%
\pgfpathlineto{\pgfqpoint{3.945267in}{3.385008in}}%
\pgfpathlineto{\pgfqpoint{3.946351in}{3.388704in}}%
\pgfpathlineto{\pgfqpoint{3.947248in}{3.388704in}}%
\pgfpathlineto{\pgfqpoint{3.947248in}{3.390552in}}%
\pgfpathlineto{\pgfqpoint{3.948730in}{3.390552in}}%
\pgfpathlineto{\pgfqpoint{3.948730in}{3.392400in}}%
\pgfpathlineto{\pgfqpoint{3.951340in}{3.392400in}}%
\pgfpathlineto{\pgfqpoint{3.951554in}{3.396096in}}%
\pgfpathlineto{\pgfqpoint{3.952877in}{3.396096in}}%
\pgfpathlineto{\pgfqpoint{3.952877in}{3.397944in}}%
\pgfpathlineto{\pgfqpoint{3.955223in}{3.397944in}}%
\pgfpathlineto{\pgfqpoint{3.955352in}{3.401640in}}%
\pgfpathlineto{\pgfqpoint{3.957218in}{3.401640in}}%
\pgfpathlineto{\pgfqpoint{3.957616in}{3.407184in}}%
\pgfpathlineto{\pgfqpoint{3.960534in}{3.407184in}}%
\pgfpathlineto{\pgfqpoint{3.961326in}{3.410880in}}%
\pgfpathlineto{\pgfqpoint{3.962215in}{3.410880in}}%
\pgfpathlineto{\pgfqpoint{3.962215in}{3.412728in}}%
\pgfpathlineto{\pgfqpoint{3.964477in}{3.412728in}}%
\pgfpathlineto{\pgfqpoint{3.964477in}{3.414576in}}%
\pgfpathlineto{\pgfqpoint{3.965709in}{3.414576in}}%
\pgfpathlineto{\pgfqpoint{3.966003in}{3.418272in}}%
\pgfpathlineto{\pgfqpoint{3.968038in}{3.418272in}}%
\pgfpathlineto{\pgfqpoint{3.968507in}{3.421968in}}%
\pgfpathlineto{\pgfqpoint{3.969979in}{3.421968in}}%
\pgfpathlineto{\pgfqpoint{3.970827in}{3.425664in}}%
\pgfpathlineto{\pgfqpoint{3.971185in}{3.425664in}}%
\pgfpathlineto{\pgfqpoint{3.971185in}{3.427512in}}%
\pgfpathlineto{\pgfqpoint{3.973561in}{3.427512in}}%
\pgfpathlineto{\pgfqpoint{3.973561in}{3.429360in}}%
\pgfpathlineto{\pgfqpoint{3.975314in}{3.429360in}}%
\pgfpathlineto{\pgfqpoint{3.976181in}{3.433056in}}%
\pgfpathlineto{\pgfqpoint{3.977193in}{3.433056in}}%
\pgfpathlineto{\pgfqpoint{3.977237in}{3.436752in}}%
\pgfpathlineto{\pgfqpoint{3.979251in}{3.436752in}}%
\pgfpathlineto{\pgfqpoint{3.979919in}{3.440448in}}%
\pgfpathlineto{\pgfqpoint{3.980398in}{3.440448in}}%
\pgfpathlineto{\pgfqpoint{3.980398in}{3.442296in}}%
\pgfpathlineto{\pgfqpoint{3.985710in}{3.442296in}}%
\pgfpathlineto{\pgfqpoint{3.986697in}{3.447840in}}%
\pgfpathlineto{\pgfqpoint{3.988944in}{3.447840in}}%
\pgfpathlineto{\pgfqpoint{3.990053in}{3.451536in}}%
\pgfpathlineto{\pgfqpoint{3.990504in}{3.451536in}}%
\pgfpathlineto{\pgfqpoint{3.990555in}{3.455232in}}%
\pgfpathlineto{\pgfqpoint{3.992957in}{3.455232in}}%
\pgfpathlineto{\pgfqpoint{3.992957in}{3.457080in}}%
\pgfpathlineto{\pgfqpoint{3.995809in}{3.457080in}}%
\pgfpathlineto{\pgfqpoint{3.996566in}{3.460776in}}%
\pgfpathlineto{\pgfqpoint{3.996946in}{3.460776in}}%
\pgfpathlineto{\pgfqpoint{3.997579in}{3.464472in}}%
\pgfpathlineto{\pgfqpoint{4.000018in}{3.464472in}}%
\pgfpathlineto{\pgfqpoint{4.000018in}{3.466320in}}%
\pgfpathlineto{\pgfqpoint{4.002891in}{3.466320in}}%
\pgfpathlineto{\pgfqpoint{4.003695in}{3.470016in}}%
\pgfpathlineto{\pgfqpoint{4.004650in}{3.470016in}}%
\pgfpathlineto{\pgfqpoint{4.005528in}{3.475560in}}%
\pgfpathlineto{\pgfqpoint{4.010056in}{3.475560in}}%
\pgfpathlineto{\pgfqpoint{4.011129in}{3.481104in}}%
\pgfpathlineto{\pgfqpoint{4.012781in}{3.481104in}}%
\pgfpathlineto{\pgfqpoint{4.013508in}{3.484800in}}%
\pgfpathlineto{\pgfqpoint{4.014732in}{3.484800in}}%
\pgfpathlineto{\pgfqpoint{4.015670in}{3.488496in}}%
\pgfpathlineto{\pgfqpoint{4.020161in}{3.488496in}}%
\pgfpathlineto{\pgfqpoint{4.020161in}{3.490344in}}%
\pgfpathlineto{\pgfqpoint{4.021687in}{3.490344in}}%
\pgfpathlineto{\pgfqpoint{4.021771in}{3.494040in}}%
\pgfpathlineto{\pgfqpoint{4.022875in}{3.494040in}}%
\pgfpathlineto{\pgfqpoint{4.022875in}{3.495888in}}%
\pgfpathlineto{\pgfqpoint{4.024149in}{3.495888in}}%
\pgfpathlineto{\pgfqpoint{4.024557in}{3.499584in}}%
\pgfpathlineto{\pgfqpoint{4.028309in}{3.499584in}}%
\pgfpathlineto{\pgfqpoint{4.029075in}{3.503280in}}%
\pgfpathlineto{\pgfqpoint{4.029856in}{3.503280in}}%
\pgfpathlineto{\pgfqpoint{4.029856in}{3.505128in}}%
\pgfpathlineto{\pgfqpoint{4.033346in}{3.505128in}}%
\pgfpathlineto{\pgfqpoint{4.033346in}{3.506976in}}%
\pgfpathlineto{\pgfqpoint{4.034524in}{3.506976in}}%
\pgfpathlineto{\pgfqpoint{4.034728in}{3.512520in}}%
\pgfpathlineto{\pgfqpoint{4.038980in}{3.512520in}}%
\pgfpathlineto{\pgfqpoint{4.039835in}{3.518064in}}%
\pgfpathlineto{\pgfqpoint{4.041957in}{3.518064in}}%
\pgfpathlineto{\pgfqpoint{4.041957in}{3.519912in}}%
\pgfpathlineto{\pgfqpoint{4.044055in}{3.519912in}}%
\pgfpathlineto{\pgfqpoint{4.044055in}{3.521760in}}%
\pgfpathlineto{\pgfqpoint{4.047250in}{3.521760in}}%
\pgfpathlineto{\pgfqpoint{4.048357in}{3.527304in}}%
\pgfpathlineto{\pgfqpoint{4.048816in}{3.527304in}}%
\pgfpathlineto{\pgfqpoint{4.049745in}{3.531000in}}%
\pgfpathlineto{\pgfqpoint{4.054099in}{3.531000in}}%
\pgfpathlineto{\pgfqpoint{4.054130in}{3.534696in}}%
\pgfpathlineto{\pgfqpoint{4.057611in}{3.534696in}}%
\pgfpathlineto{\pgfqpoint{4.058385in}{3.538392in}}%
\pgfpathlineto{\pgfqpoint{4.058843in}{3.538392in}}%
\pgfpathlineto{\pgfqpoint{4.059432in}{3.542088in}}%
\pgfpathlineto{\pgfqpoint{4.063997in}{3.542088in}}%
\pgfpathlineto{\pgfqpoint{4.064360in}{3.545784in}}%
\pgfpathlineto{\pgfqpoint{4.066570in}{3.545784in}}%
\pgfpathlineto{\pgfqpoint{4.066570in}{3.547632in}}%
\pgfpathlineto{\pgfqpoint{4.068252in}{3.547632in}}%
\pgfpathlineto{\pgfqpoint{4.068252in}{3.549480in}}%
\pgfpathlineto{\pgfqpoint{4.071566in}{3.549480in}}%
\pgfpathlineto{\pgfqpoint{4.072300in}{3.553176in}}%
\pgfpathlineto{\pgfqpoint{4.072858in}{3.553176in}}%
\pgfpathlineto{\pgfqpoint{4.073836in}{3.556872in}}%
\pgfpathlineto{\pgfqpoint{4.078399in}{3.556872in}}%
\pgfpathlineto{\pgfqpoint{4.078466in}{3.560568in}}%
\pgfpathlineto{\pgfqpoint{4.081777in}{3.560568in}}%
\pgfpathlineto{\pgfqpoint{4.082581in}{3.564264in}}%
\pgfpathlineto{\pgfqpoint{4.088328in}{3.564264in}}%
\pgfpathlineto{\pgfqpoint{4.088436in}{3.567960in}}%
\pgfpathlineto{\pgfqpoint{4.090817in}{3.567960in}}%
\pgfpathlineto{\pgfqpoint{4.091561in}{3.573504in}}%
\pgfpathlineto{\pgfqpoint{4.092662in}{3.573504in}}%
\pgfpathlineto{\pgfqpoint{4.092678in}{3.577200in}}%
\pgfpathlineto{\pgfqpoint{4.096622in}{3.577200in}}%
\pgfpathlineto{\pgfqpoint{4.096622in}{3.579048in}}%
\pgfpathlineto{\pgfqpoint{4.098222in}{3.579048in}}%
\pgfpathlineto{\pgfqpoint{4.098222in}{3.580896in}}%
\pgfpathlineto{\pgfqpoint{4.101109in}{3.580896in}}%
\pgfpathlineto{\pgfqpoint{4.101109in}{3.582744in}}%
\pgfpathlineto{\pgfqpoint{4.102526in}{3.582744in}}%
\pgfpathlineto{\pgfqpoint{4.102606in}{3.586440in}}%
\pgfpathlineto{\pgfqpoint{4.107082in}{3.586440in}}%
\pgfpathlineto{\pgfqpoint{4.107978in}{3.590136in}}%
\pgfpathlineto{\pgfqpoint{4.112322in}{3.590136in}}%
\pgfpathlineto{\pgfqpoint{4.112443in}{3.593832in}}%
\pgfpathlineto{\pgfqpoint{4.115086in}{3.593832in}}%
\pgfpathlineto{\pgfqpoint{4.116197in}{3.601224in}}%
\pgfpathlineto{\pgfqpoint{4.116761in}{3.601224in}}%
\pgfpathlineto{\pgfqpoint{4.116761in}{3.603072in}}%
\pgfpathlineto{\pgfqpoint{4.122034in}{3.603072in}}%
\pgfpathlineto{\pgfqpoint{4.122398in}{3.606768in}}%
\pgfpathlineto{\pgfqpoint{4.125302in}{3.606768in}}%
\pgfpathlineto{\pgfqpoint{4.125302in}{3.608616in}}%
\pgfpathlineto{\pgfqpoint{4.129837in}{3.608616in}}%
\pgfpathlineto{\pgfqpoint{4.129837in}{3.610464in}}%
\pgfpathlineto{\pgfqpoint{4.131465in}{3.610464in}}%
\pgfpathlineto{\pgfqpoint{4.132345in}{3.617856in}}%
\pgfpathlineto{\pgfqpoint{4.135031in}{3.617856in}}%
\pgfpathlineto{\pgfqpoint{4.135031in}{3.619704in}}%
\pgfpathlineto{\pgfqpoint{4.136433in}{3.619704in}}%
\pgfpathlineto{\pgfqpoint{4.136433in}{3.621552in}}%
\pgfpathlineto{\pgfqpoint{4.139980in}{3.621552in}}%
\pgfpathlineto{\pgfqpoint{4.141084in}{3.627096in}}%
\pgfpathlineto{\pgfqpoint{4.146589in}{3.627096in}}%
\pgfpathlineto{\pgfqpoint{4.146589in}{3.628944in}}%
\pgfpathlineto{\pgfqpoint{4.149231in}{3.628944in}}%
\pgfpathlineto{\pgfqpoint{4.149231in}{3.630792in}}%
\pgfpathlineto{\pgfqpoint{4.151251in}{3.630792in}}%
\pgfpathlineto{\pgfqpoint{4.151456in}{3.634488in}}%
\pgfpathlineto{\pgfqpoint{4.154089in}{3.634488in}}%
\pgfpathlineto{\pgfqpoint{4.154089in}{3.636336in}}%
\pgfpathlineto{\pgfqpoint{4.156163in}{3.636336in}}%
\pgfpathlineto{\pgfqpoint{4.156703in}{3.640032in}}%
\pgfpathlineto{\pgfqpoint{4.159306in}{3.640032in}}%
\pgfpathlineto{\pgfqpoint{4.160231in}{3.647424in}}%
\pgfpathlineto{\pgfqpoint{4.160687in}{3.647424in}}%
\pgfpathlineto{\pgfqpoint{4.160687in}{3.649272in}}%
\pgfpathlineto{\pgfqpoint{4.170835in}{3.649272in}}%
\pgfpathlineto{\pgfqpoint{4.170835in}{3.651120in}}%
\pgfpathlineto{\pgfqpoint{4.173493in}{3.651120in}}%
\pgfpathlineto{\pgfqpoint{4.173493in}{3.652968in}}%
\pgfpathlineto{\pgfqpoint{4.175249in}{3.652968in}}%
\pgfpathlineto{\pgfqpoint{4.176139in}{3.656664in}}%
\pgfpathlineto{\pgfqpoint{4.176698in}{3.656664in}}%
\pgfpathlineto{\pgfqpoint{4.176698in}{3.658512in}}%
\pgfpathlineto{\pgfqpoint{4.178095in}{3.658512in}}%
\pgfpathlineto{\pgfqpoint{4.178095in}{3.660360in}}%
\pgfpathlineto{\pgfqpoint{4.180230in}{3.660360in}}%
\pgfpathlineto{\pgfqpoint{4.180230in}{3.662208in}}%
\pgfpathlineto{\pgfqpoint{4.183432in}{3.662208in}}%
\pgfpathlineto{\pgfqpoint{4.184146in}{3.667752in}}%
\pgfpathlineto{\pgfqpoint{4.186397in}{3.667752in}}%
\pgfpathlineto{\pgfqpoint{4.186397in}{3.669600in}}%
\pgfpathlineto{\pgfqpoint{4.190264in}{3.669600in}}%
\pgfpathlineto{\pgfqpoint{4.190264in}{3.671448in}}%
\pgfpathlineto{\pgfqpoint{4.192993in}{3.671448in}}%
\pgfpathlineto{\pgfqpoint{4.192993in}{3.673296in}}%
\pgfpathlineto{\pgfqpoint{4.197537in}{3.673296in}}%
\pgfpathlineto{\pgfqpoint{4.197537in}{3.675144in}}%
\pgfpathlineto{\pgfqpoint{4.198992in}{3.675144in}}%
\pgfpathlineto{\pgfqpoint{4.199989in}{3.680688in}}%
\pgfpathlineto{\pgfqpoint{4.200795in}{3.680688in}}%
\pgfpathlineto{\pgfqpoint{4.200795in}{3.682536in}}%
\pgfpathlineto{\pgfqpoint{4.202713in}{3.682536in}}%
\pgfpathlineto{\pgfqpoint{4.202713in}{3.684384in}}%
\pgfpathlineto{\pgfqpoint{4.206617in}{3.684384in}}%
\pgfpathlineto{\pgfqpoint{4.206617in}{3.686232in}}%
\pgfpathlineto{\pgfqpoint{4.208173in}{3.686232in}}%
\pgfpathlineto{\pgfqpoint{4.208173in}{3.688080in}}%
\pgfpathlineto{\pgfqpoint{4.210281in}{3.688080in}}%
\pgfpathlineto{\pgfqpoint{4.210281in}{3.689928in}}%
\pgfpathlineto{\pgfqpoint{4.214294in}{3.689928in}}%
\pgfpathlineto{\pgfqpoint{4.214294in}{3.691776in}}%
\pgfpathlineto{\pgfqpoint{4.216774in}{3.691776in}}%
\pgfpathlineto{\pgfqpoint{4.216774in}{3.693624in}}%
\pgfpathlineto{\pgfqpoint{4.218597in}{3.693624in}}%
\pgfpathlineto{\pgfqpoint{4.218651in}{3.697320in}}%
\pgfpathlineto{\pgfqpoint{4.220191in}{3.697320in}}%
\pgfpathlineto{\pgfqpoint{4.220191in}{3.699168in}}%
\pgfpathlineto{\pgfqpoint{4.222773in}{3.699168in}}%
\pgfpathlineto{\pgfqpoint{4.223276in}{3.702864in}}%
\pgfpathlineto{\pgfqpoint{4.224111in}{3.702864in}}%
\pgfpathlineto{\pgfqpoint{4.224111in}{3.704712in}}%
\pgfpathlineto{\pgfqpoint{4.225438in}{3.704712in}}%
\pgfpathlineto{\pgfqpoint{4.226419in}{3.710256in}}%
\pgfpathlineto{\pgfqpoint{4.231959in}{3.710256in}}%
\pgfpathlineto{\pgfqpoint{4.231959in}{3.712104in}}%
\pgfpathlineto{\pgfqpoint{4.238277in}{3.712104in}}%
\pgfpathlineto{\pgfqpoint{4.238277in}{3.713952in}}%
\pgfpathlineto{\pgfqpoint{4.240669in}{3.713952in}}%
\pgfpathlineto{\pgfqpoint{4.240669in}{3.715800in}}%
\pgfpathlineto{\pgfqpoint{4.242073in}{3.715800in}}%
\pgfpathlineto{\pgfqpoint{4.242731in}{3.721344in}}%
\pgfpathlineto{\pgfqpoint{4.243381in}{3.721344in}}%
\pgfpathlineto{\pgfqpoint{4.243381in}{3.723192in}}%
\pgfpathlineto{\pgfqpoint{4.247159in}{3.723192in}}%
\pgfpathlineto{\pgfqpoint{4.247159in}{3.725040in}}%
\pgfpathlineto{\pgfqpoint{4.249665in}{3.725040in}}%
\pgfpathlineto{\pgfqpoint{4.249665in}{3.726888in}}%
\pgfpathlineto{\pgfqpoint{4.250797in}{3.726888in}}%
\pgfpathlineto{\pgfqpoint{4.250942in}{3.730584in}}%
\pgfpathlineto{\pgfqpoint{4.257114in}{3.730584in}}%
\pgfpathlineto{\pgfqpoint{4.257114in}{3.732432in}}%
\pgfpathlineto{\pgfqpoint{4.259618in}{3.732432in}}%
\pgfpathlineto{\pgfqpoint{4.259618in}{3.734280in}}%
\pgfpathlineto{\pgfqpoint{4.261970in}{3.734280in}}%
\pgfpathlineto{\pgfqpoint{4.261970in}{3.736128in}}%
\pgfpathlineto{\pgfqpoint{4.265733in}{3.736128in}}%
\pgfpathlineto{\pgfqpoint{4.266659in}{3.741672in}}%
\pgfpathlineto{\pgfqpoint{4.266875in}{3.741672in}}%
\pgfpathlineto{\pgfqpoint{4.267136in}{3.745368in}}%
\pgfpathlineto{\pgfqpoint{4.269033in}{3.745368in}}%
\pgfpathlineto{\pgfqpoint{4.269033in}{3.747216in}}%
\pgfpathlineto{\pgfqpoint{4.273510in}{3.747216in}}%
\pgfpathlineto{\pgfqpoint{4.273510in}{3.749064in}}%
\pgfpathlineto{\pgfqpoint{4.274639in}{3.749064in}}%
\pgfpathlineto{\pgfqpoint{4.274639in}{3.750912in}}%
\pgfpathlineto{\pgfqpoint{4.280569in}{3.750912in}}%
\pgfpathlineto{\pgfqpoint{4.280569in}{3.752760in}}%
\pgfpathlineto{\pgfqpoint{4.283525in}{3.752760in}}%
\pgfpathlineto{\pgfqpoint{4.283525in}{3.754608in}}%
\pgfpathlineto{\pgfqpoint{4.285302in}{3.754608in}}%
\pgfpathlineto{\pgfqpoint{4.285801in}{3.758304in}}%
\pgfpathlineto{\pgfqpoint{4.292895in}{3.758304in}}%
\pgfpathlineto{\pgfqpoint{4.293685in}{3.763848in}}%
\pgfpathlineto{\pgfqpoint{4.295954in}{3.763848in}}%
\pgfpathlineto{\pgfqpoint{4.295954in}{3.765696in}}%
\pgfpathlineto{\pgfqpoint{4.303893in}{3.765696in}}%
\pgfpathlineto{\pgfqpoint{4.303893in}{3.767544in}}%
\pgfpathlineto{\pgfqpoint{4.306960in}{3.767544in}}%
\pgfpathlineto{\pgfqpoint{4.306960in}{3.769392in}}%
\pgfpathlineto{\pgfqpoint{4.308828in}{3.769392in}}%
\pgfpathlineto{\pgfqpoint{4.308828in}{3.771240in}}%
\pgfpathlineto{\pgfqpoint{4.313625in}{3.771240in}}%
\pgfpathlineto{\pgfqpoint{4.313625in}{3.773088in}}%
\pgfpathlineto{\pgfqpoint{4.316350in}{3.773088in}}%
\pgfpathlineto{\pgfqpoint{4.316350in}{3.774936in}}%
\pgfpathlineto{\pgfqpoint{4.319053in}{3.774936in}}%
\pgfpathlineto{\pgfqpoint{4.319929in}{3.778632in}}%
\pgfpathlineto{\pgfqpoint{4.327293in}{3.778632in}}%
\pgfpathlineto{\pgfqpoint{4.327293in}{3.780480in}}%
\pgfpathlineto{\pgfqpoint{4.335729in}{3.780480in}}%
\pgfpathlineto{\pgfqpoint{4.335729in}{3.782328in}}%
\pgfpathlineto{\pgfqpoint{4.336981in}{3.782328in}}%
\pgfpathlineto{\pgfqpoint{4.337884in}{3.787872in}}%
\pgfpathlineto{\pgfqpoint{4.343206in}{3.787872in}}%
\pgfpathlineto{\pgfqpoint{4.343206in}{3.789720in}}%
\pgfpathlineto{\pgfqpoint{4.346329in}{3.789720in}}%
\pgfpathlineto{\pgfqpoint{4.346329in}{3.791568in}}%
\pgfpathlineto{\pgfqpoint{4.351580in}{3.791568in}}%
\pgfpathlineto{\pgfqpoint{4.351912in}{3.795264in}}%
\pgfpathlineto{\pgfqpoint{4.359323in}{3.795264in}}%
\pgfpathlineto{\pgfqpoint{4.359323in}{3.797112in}}%
\pgfpathlineto{\pgfqpoint{4.360620in}{3.797112in}}%
\pgfpathlineto{\pgfqpoint{4.361557in}{3.804504in}}%
\pgfpathlineto{\pgfqpoint{4.368622in}{3.804504in}}%
\pgfpathlineto{\pgfqpoint{4.368622in}{3.806352in}}%
\pgfpathlineto{\pgfqpoint{4.369905in}{3.806352in}}%
\pgfpathlineto{\pgfqpoint{4.369905in}{3.808200in}}%
\pgfpathlineto{\pgfqpoint{4.378501in}{3.808200in}}%
\pgfpathlineto{\pgfqpoint{4.378501in}{3.810048in}}%
\pgfpathlineto{\pgfqpoint{4.380110in}{3.810048in}}%
\pgfpathlineto{\pgfqpoint{4.380110in}{3.811896in}}%
\pgfpathlineto{\pgfqpoint{4.382821in}{3.811896in}}%
\pgfpathlineto{\pgfqpoint{4.382821in}{3.813744in}}%
\pgfpathlineto{\pgfqpoint{4.385136in}{3.813744in}}%
\pgfpathlineto{\pgfqpoint{4.385136in}{3.815592in}}%
\pgfpathlineto{\pgfqpoint{4.388699in}{3.815592in}}%
\pgfpathlineto{\pgfqpoint{4.388699in}{3.817440in}}%
\pgfpathlineto{\pgfqpoint{4.394658in}{3.817440in}}%
\pgfpathlineto{\pgfqpoint{4.395092in}{3.821136in}}%
\pgfpathlineto{\pgfqpoint{4.401628in}{3.821136in}}%
\pgfpathlineto{\pgfqpoint{4.401628in}{3.822984in}}%
\pgfpathlineto{\pgfqpoint{4.403978in}{3.822984in}}%
\pgfpathlineto{\pgfqpoint{4.403978in}{3.824832in}}%
\pgfpathlineto{\pgfqpoint{4.406473in}{3.824832in}}%
\pgfpathlineto{\pgfqpoint{4.406473in}{3.826680in}}%
\pgfpathlineto{\pgfqpoint{4.408037in}{3.826680in}}%
\pgfpathlineto{\pgfqpoint{4.408037in}{3.828528in}}%
\pgfpathlineto{\pgfqpoint{4.410908in}{3.828528in}}%
\pgfpathlineto{\pgfqpoint{4.410908in}{3.830376in}}%
\pgfpathlineto{\pgfqpoint{4.413252in}{3.830376in}}%
\pgfpathlineto{\pgfqpoint{4.413252in}{3.832224in}}%
\pgfpathlineto{\pgfqpoint{4.424767in}{3.832224in}}%
\pgfpathlineto{\pgfqpoint{4.425170in}{3.837768in}}%
\pgfpathlineto{\pgfqpoint{4.427379in}{3.837768in}}%
\pgfpathlineto{\pgfqpoint{4.427379in}{3.839616in}}%
\pgfpathlineto{\pgfqpoint{4.428952in}{3.839616in}}%
\pgfpathlineto{\pgfqpoint{4.428952in}{3.841464in}}%
\pgfpathlineto{\pgfqpoint{4.431580in}{3.841464in}}%
\pgfpathlineto{\pgfqpoint{4.431580in}{3.843312in}}%
\pgfpathlineto{\pgfqpoint{4.432692in}{3.843312in}}%
\pgfpathlineto{\pgfqpoint{4.432692in}{3.845160in}}%
\pgfpathlineto{\pgfqpoint{4.434146in}{3.845160in}}%
\pgfpathlineto{\pgfqpoint{4.434146in}{3.847008in}}%
\pgfpathlineto{\pgfqpoint{4.436602in}{3.847008in}}%
\pgfpathlineto{\pgfqpoint{4.436602in}{3.848856in}}%
\pgfpathlineto{\pgfqpoint{4.447094in}{3.848856in}}%
\pgfpathlineto{\pgfqpoint{4.447961in}{3.852552in}}%
\pgfpathlineto{\pgfqpoint{4.448586in}{3.852552in}}%
\pgfpathlineto{\pgfqpoint{4.448586in}{3.854400in}}%
\pgfpathlineto{\pgfqpoint{4.450716in}{3.854400in}}%
\pgfpathlineto{\pgfqpoint{4.450716in}{3.856248in}}%
\pgfpathlineto{\pgfqpoint{4.452236in}{3.856248in}}%
\pgfpathlineto{\pgfqpoint{4.452236in}{3.858096in}}%
\pgfpathlineto{\pgfqpoint{4.455712in}{3.858096in}}%
\pgfpathlineto{\pgfqpoint{4.455712in}{3.859944in}}%
\pgfpathlineto{\pgfqpoint{4.457695in}{3.859944in}}%
\pgfpathlineto{\pgfqpoint{4.457695in}{3.861792in}}%
\pgfpathlineto{\pgfqpoint{4.469903in}{3.861792in}}%
\pgfpathlineto{\pgfqpoint{4.470948in}{3.865488in}}%
\pgfpathlineto{\pgfqpoint{4.471192in}{3.865488in}}%
\pgfpathlineto{\pgfqpoint{4.472090in}{3.869184in}}%
\pgfpathlineto{\pgfqpoint{4.473951in}{3.869184in}}%
\pgfpathlineto{\pgfqpoint{4.473951in}{3.871032in}}%
\pgfpathlineto{\pgfqpoint{4.476101in}{3.871032in}}%
\pgfpathlineto{\pgfqpoint{4.476101in}{3.872880in}}%
\pgfpathlineto{\pgfqpoint{4.478866in}{3.872880in}}%
\pgfpathlineto{\pgfqpoint{4.478866in}{3.874728in}}%
\pgfpathlineto{\pgfqpoint{4.489856in}{3.874728in}}%
\pgfpathlineto{\pgfqpoint{4.490867in}{3.878424in}}%
\pgfpathlineto{\pgfqpoint{4.493165in}{3.878424in}}%
\pgfpathlineto{\pgfqpoint{4.493165in}{3.880272in}}%
\pgfpathlineto{\pgfqpoint{4.494867in}{3.880272in}}%
\pgfpathlineto{\pgfqpoint{4.494867in}{3.882120in}}%
\pgfpathlineto{\pgfqpoint{4.496000in}{3.882120in}}%
\pgfpathlineto{\pgfqpoint{4.496000in}{3.883968in}}%
\pgfpathlineto{\pgfqpoint{4.499102in}{3.883968in}}%
\pgfpathlineto{\pgfqpoint{4.499725in}{3.887664in}}%
\pgfpathlineto{\pgfqpoint{4.513046in}{3.887664in}}%
\pgfpathlineto{\pgfqpoint{4.513934in}{3.891360in}}%
\pgfpathlineto{\pgfqpoint{4.516159in}{3.891360in}}%
\pgfpathlineto{\pgfqpoint{4.517046in}{3.896904in}}%
\pgfpathlineto{\pgfqpoint{4.519083in}{3.896904in}}%
\pgfpathlineto{\pgfqpoint{4.519083in}{3.898752in}}%
\pgfpathlineto{\pgfqpoint{4.522711in}{3.898752in}}%
\pgfpathlineto{\pgfqpoint{4.522711in}{3.900600in}}%
\pgfpathlineto{\pgfqpoint{4.526013in}{3.900600in}}%
\pgfpathlineto{\pgfqpoint{4.526013in}{3.902448in}}%
\pgfpathlineto{\pgfqpoint{4.531625in}{3.902448in}}%
\pgfpathlineto{\pgfqpoint{4.531625in}{3.904296in}}%
\pgfpathlineto{\pgfqpoint{4.535909in}{3.904296in}}%
\pgfpathlineto{\pgfqpoint{4.536304in}{3.909840in}}%
\pgfpathlineto{\pgfqpoint{4.537958in}{3.909840in}}%
\pgfpathlineto{\pgfqpoint{4.537958in}{3.911688in}}%
\pgfpathlineto{\pgfqpoint{4.540415in}{3.911688in}}%
\pgfpathlineto{\pgfqpoint{4.540415in}{3.913536in}}%
\pgfpathlineto{\pgfqpoint{4.544953in}{3.913536in}}%
\pgfpathlineto{\pgfqpoint{4.544953in}{3.915384in}}%
\pgfpathlineto{\pgfqpoint{4.548389in}{3.915384in}}%
\pgfpathlineto{\pgfqpoint{4.548389in}{3.917232in}}%
\pgfpathlineto{\pgfqpoint{4.550601in}{3.917232in}}%
\pgfpathlineto{\pgfqpoint{4.550601in}{3.919080in}}%
\pgfpathlineto{\pgfqpoint{4.554662in}{3.919080in}}%
\pgfpathlineto{\pgfqpoint{4.554662in}{3.920928in}}%
\pgfpathlineto{\pgfqpoint{4.558806in}{3.920928in}}%
\pgfpathlineto{\pgfqpoint{4.559517in}{3.926472in}}%
\pgfpathlineto{\pgfqpoint{4.561148in}{3.926472in}}%
\pgfpathlineto{\pgfqpoint{4.561148in}{3.928320in}}%
\pgfpathlineto{\pgfqpoint{4.569328in}{3.928320in}}%
\pgfpathlineto{\pgfqpoint{4.569328in}{3.930168in}}%
\pgfpathlineto{\pgfqpoint{4.571232in}{3.930168in}}%
\pgfpathlineto{\pgfqpoint{4.571232in}{3.932016in}}%
\pgfpathlineto{\pgfqpoint{4.575117in}{3.932016in}}%
\pgfpathlineto{\pgfqpoint{4.575117in}{3.933864in}}%
\pgfpathlineto{\pgfqpoint{4.578180in}{3.933864in}}%
\pgfpathlineto{\pgfqpoint{4.578180in}{3.935712in}}%
\pgfpathlineto{\pgfqpoint{4.582004in}{3.935712in}}%
\pgfpathlineto{\pgfqpoint{4.582004in}{3.937560in}}%
\pgfpathlineto{\pgfqpoint{4.584014in}{3.937560in}}%
\pgfpathlineto{\pgfqpoint{4.584014in}{3.939408in}}%
\pgfpathlineto{\pgfqpoint{4.592455in}{3.939408in}}%
\pgfpathlineto{\pgfqpoint{4.592455in}{3.941256in}}%
\pgfpathlineto{\pgfqpoint{4.594083in}{3.941256in}}%
\pgfpathlineto{\pgfqpoint{4.594083in}{3.943104in}}%
\pgfpathlineto{\pgfqpoint{4.596548in}{3.943104in}}%
\pgfpathlineto{\pgfqpoint{4.596548in}{3.944952in}}%
\pgfpathlineto{\pgfqpoint{4.601043in}{3.944952in}}%
\pgfpathlineto{\pgfqpoint{4.601043in}{3.946800in}}%
\pgfpathlineto{\pgfqpoint{4.602847in}{3.946800in}}%
\pgfpathlineto{\pgfqpoint{4.602847in}{3.948648in}}%
\pgfpathlineto{\pgfqpoint{4.612535in}{3.948648in}}%
\pgfpathlineto{\pgfqpoint{4.612535in}{3.950496in}}%
\pgfpathlineto{\pgfqpoint{4.615078in}{3.950496in}}%
\pgfpathlineto{\pgfqpoint{4.615078in}{3.952344in}}%
\pgfpathlineto{\pgfqpoint{4.619393in}{3.952344in}}%
\pgfpathlineto{\pgfqpoint{4.619642in}{3.956040in}}%
\pgfpathlineto{\pgfqpoint{4.625986in}{3.956040in}}%
\pgfpathlineto{\pgfqpoint{4.625986in}{3.957888in}}%
\pgfpathlineto{\pgfqpoint{4.635332in}{3.957888in}}%
\pgfpathlineto{\pgfqpoint{4.635332in}{3.959736in}}%
\pgfpathlineto{\pgfqpoint{4.637712in}{3.959736in}}%
\pgfpathlineto{\pgfqpoint{4.638142in}{3.965280in}}%
\pgfpathlineto{\pgfqpoint{4.642460in}{3.965280in}}%
\pgfpathlineto{\pgfqpoint{4.642460in}{3.967128in}}%
\pgfpathlineto{\pgfqpoint{4.645110in}{3.967128in}}%
\pgfpathlineto{\pgfqpoint{4.645110in}{3.968976in}}%
\pgfpathlineto{\pgfqpoint{4.656057in}{3.968976in}}%
\pgfpathlineto{\pgfqpoint{4.656057in}{3.970824in}}%
\pgfpathlineto{\pgfqpoint{4.658165in}{3.970824in}}%
\pgfpathlineto{\pgfqpoint{4.658165in}{3.972672in}}%
\pgfpathlineto{\pgfqpoint{4.660246in}{3.972672in}}%
\pgfpathlineto{\pgfqpoint{4.660659in}{3.976368in}}%
\pgfpathlineto{\pgfqpoint{4.665200in}{3.976368in}}%
\pgfpathlineto{\pgfqpoint{4.666069in}{3.980064in}}%
\pgfpathlineto{\pgfqpoint{4.667523in}{3.980064in}}%
\pgfpathlineto{\pgfqpoint{4.667523in}{3.981912in}}%
\pgfpathlineto{\pgfqpoint{4.676218in}{3.981912in}}%
\pgfpathlineto{\pgfqpoint{4.676802in}{3.985608in}}%
\pgfpathlineto{\pgfqpoint{4.678941in}{3.985608in}}%
\pgfpathlineto{\pgfqpoint{4.679104in}{3.989304in}}%
\pgfpathlineto{\pgfqpoint{4.683101in}{3.989304in}}%
\pgfpathlineto{\pgfqpoint{4.683459in}{3.993000in}}%
\pgfpathlineto{\pgfqpoint{4.686615in}{3.993000in}}%
\pgfpathlineto{\pgfqpoint{4.686615in}{3.994848in}}%
\pgfpathlineto{\pgfqpoint{4.697193in}{3.994848in}}%
\pgfpathlineto{\pgfqpoint{4.697193in}{3.996696in}}%
\pgfpathlineto{\pgfqpoint{4.699652in}{3.996696in}}%
\pgfpathlineto{\pgfqpoint{4.699652in}{3.998544in}}%
\pgfpathlineto{\pgfqpoint{4.701357in}{3.998544in}}%
\pgfpathlineto{\pgfqpoint{4.701906in}{4.002240in}}%
\pgfpathlineto{\pgfqpoint{4.702729in}{4.002240in}}%
\pgfpathlineto{\pgfqpoint{4.702729in}{4.004088in}}%
\pgfpathlineto{\pgfqpoint{4.705886in}{4.004088in}}%
\pgfpathlineto{\pgfqpoint{4.705886in}{4.005936in}}%
\pgfpathlineto{\pgfqpoint{4.709341in}{4.005936in}}%
\pgfpathlineto{\pgfqpoint{4.709341in}{4.007784in}}%
\pgfpathlineto{\pgfqpoint{4.714823in}{4.007784in}}%
\pgfpathlineto{\pgfqpoint{4.715208in}{4.011480in}}%
\pgfpathlineto{\pgfqpoint{4.722096in}{4.011480in}}%
\pgfpathlineto{\pgfqpoint{4.722096in}{4.013328in}}%
\pgfpathlineto{\pgfqpoint{4.724425in}{4.013328in}}%
\pgfpathlineto{\pgfqpoint{4.724717in}{4.017024in}}%
\pgfpathlineto{\pgfqpoint{4.727233in}{4.017024in}}%
\pgfpathlineto{\pgfqpoint{4.727233in}{4.018872in}}%
\pgfpathlineto{\pgfqpoint{4.728479in}{4.018872in}}%
\pgfpathlineto{\pgfqpoint{4.728479in}{4.020720in}}%
\pgfpathlineto{\pgfqpoint{4.738161in}{4.020720in}}%
\pgfpathlineto{\pgfqpoint{4.738161in}{4.022568in}}%
\pgfpathlineto{\pgfqpoint{4.740617in}{4.022568in}}%
\pgfpathlineto{\pgfqpoint{4.740617in}{4.024416in}}%
\pgfpathlineto{\pgfqpoint{4.743971in}{4.024416in}}%
\pgfpathlineto{\pgfqpoint{4.743971in}{4.026264in}}%
\pgfpathlineto{\pgfqpoint{4.745981in}{4.026264in}}%
\pgfpathlineto{\pgfqpoint{4.747048in}{4.031808in}}%
\pgfpathlineto{\pgfqpoint{4.747471in}{4.031808in}}%
\pgfpathlineto{\pgfqpoint{4.747471in}{4.033656in}}%
\pgfpathlineto{\pgfqpoint{4.749540in}{4.033656in}}%
\pgfpathlineto{\pgfqpoint{4.749540in}{4.035504in}}%
\pgfpathlineto{\pgfqpoint{4.751057in}{4.035504in}}%
\pgfpathlineto{\pgfqpoint{4.751057in}{4.037352in}}%
\pgfpathlineto{\pgfqpoint{4.760751in}{4.037352in}}%
\pgfpathlineto{\pgfqpoint{4.760751in}{4.039200in}}%
\pgfpathlineto{\pgfqpoint{4.762186in}{4.039200in}}%
\pgfpathlineto{\pgfqpoint{4.763217in}{4.042896in}}%
\pgfpathlineto{\pgfqpoint{4.764148in}{4.042896in}}%
\pgfpathlineto{\pgfqpoint{4.764575in}{4.046592in}}%
\pgfpathlineto{\pgfqpoint{4.765545in}{4.046592in}}%
\pgfpathlineto{\pgfqpoint{4.765545in}{4.048440in}}%
\pgfpathlineto{\pgfqpoint{4.768940in}{4.048440in}}%
\pgfpathlineto{\pgfqpoint{4.768940in}{4.050288in}}%
\pgfpathlineto{\pgfqpoint{4.771778in}{4.050288in}}%
\pgfpathlineto{\pgfqpoint{4.771778in}{4.052136in}}%
\pgfpathlineto{\pgfqpoint{4.779133in}{4.052136in}}%
\pgfpathlineto{\pgfqpoint{4.779133in}{4.053984in}}%
\pgfpathlineto{\pgfqpoint{4.784411in}{4.053984in}}%
\pgfpathlineto{\pgfqpoint{4.785433in}{4.057680in}}%
\pgfpathlineto{\pgfqpoint{4.786635in}{4.057680in}}%
\pgfpathlineto{\pgfqpoint{4.786635in}{4.059528in}}%
\pgfpathlineto{\pgfqpoint{4.790110in}{4.059528in}}%
\pgfpathlineto{\pgfqpoint{4.790110in}{4.061376in}}%
\pgfpathlineto{\pgfqpoint{4.791415in}{4.061376in}}%
\pgfpathlineto{\pgfqpoint{4.791415in}{4.063224in}}%
\pgfpathlineto{\pgfqpoint{4.801548in}{4.063224in}}%
\pgfpathlineto{\pgfqpoint{4.801548in}{4.065072in}}%
\pgfpathlineto{\pgfqpoint{4.803417in}{4.065072in}}%
\pgfpathlineto{\pgfqpoint{4.803417in}{4.066920in}}%
\pgfpathlineto{\pgfqpoint{4.806823in}{4.066920in}}%
\pgfpathlineto{\pgfqpoint{4.806823in}{4.068768in}}%
\pgfpathlineto{\pgfqpoint{4.808784in}{4.068768in}}%
\pgfpathlineto{\pgfqpoint{4.809890in}{4.072464in}}%
\pgfpathlineto{\pgfqpoint{4.812243in}{4.072464in}}%
\pgfpathlineto{\pgfqpoint{4.812243in}{4.074312in}}%
\pgfpathlineto{\pgfqpoint{4.823751in}{4.074312in}}%
\pgfpathlineto{\pgfqpoint{4.823751in}{4.076160in}}%
\pgfpathlineto{\pgfqpoint{4.825508in}{4.076160in}}%
\pgfpathlineto{\pgfqpoint{4.826618in}{4.079856in}}%
\pgfpathlineto{\pgfqpoint{4.832368in}{4.079856in}}%
\pgfpathlineto{\pgfqpoint{4.832368in}{4.081704in}}%
\pgfpathlineto{\pgfqpoint{4.834158in}{4.081704in}}%
\pgfpathlineto{\pgfqpoint{4.834158in}{4.083552in}}%
\pgfpathlineto{\pgfqpoint{4.842000in}{4.083552in}}%
\pgfpathlineto{\pgfqpoint{4.842000in}{4.085400in}}%
\pgfpathlineto{\pgfqpoint{4.847905in}{4.085400in}}%
\pgfpathlineto{\pgfqpoint{4.848876in}{4.089096in}}%
\pgfpathlineto{\pgfqpoint{4.854368in}{4.089096in}}%
\pgfpathlineto{\pgfqpoint{4.854368in}{4.090944in}}%
\pgfpathlineto{\pgfqpoint{4.859846in}{4.090944in}}%
\pgfpathlineto{\pgfqpoint{4.859846in}{4.092792in}}%
\pgfpathlineto{\pgfqpoint{4.864297in}{4.092792in}}%
\pgfpathlineto{\pgfqpoint{4.864297in}{4.094640in}}%
\pgfpathlineto{\pgfqpoint{4.869903in}{4.094640in}}%
\pgfpathlineto{\pgfqpoint{4.869903in}{4.096488in}}%
\pgfpathlineto{\pgfqpoint{4.871063in}{4.096488in}}%
\pgfpathlineto{\pgfqpoint{4.871063in}{4.098336in}}%
\pgfpathlineto{\pgfqpoint{4.872210in}{4.098336in}}%
\pgfpathlineto{\pgfqpoint{4.872580in}{4.102032in}}%
\pgfpathlineto{\pgfqpoint{4.877963in}{4.102032in}}%
\pgfpathlineto{\pgfqpoint{4.877963in}{4.103880in}}%
\pgfpathlineto{\pgfqpoint{4.886468in}{4.103880in}}%
\pgfpathlineto{\pgfqpoint{4.886468in}{4.105728in}}%
\pgfpathlineto{\pgfqpoint{4.887606in}{4.105728in}}%
\pgfpathlineto{\pgfqpoint{4.887606in}{4.107576in}}%
\pgfpathlineto{\pgfqpoint{4.888870in}{4.107576in}}%
\pgfpathlineto{\pgfqpoint{4.888870in}{4.109424in}}%
\pgfpathlineto{\pgfqpoint{4.894753in}{4.109424in}}%
\pgfpathlineto{\pgfqpoint{4.894753in}{4.111272in}}%
\pgfpathlineto{\pgfqpoint{4.899842in}{4.111272in}}%
\pgfpathlineto{\pgfqpoint{4.899842in}{4.113120in}}%
\pgfpathlineto{\pgfqpoint{4.904257in}{4.113120in}}%
\pgfpathlineto{\pgfqpoint{4.904257in}{4.114968in}}%
\pgfpathlineto{\pgfqpoint{4.909883in}{4.114968in}}%
\pgfpathlineto{\pgfqpoint{4.910785in}{4.118664in}}%
\pgfpathlineto{\pgfqpoint{4.913674in}{4.118664in}}%
\pgfpathlineto{\pgfqpoint{4.913674in}{4.120512in}}%
\pgfpathlineto{\pgfqpoint{4.916808in}{4.120512in}}%
\pgfpathlineto{\pgfqpoint{4.916808in}{4.122360in}}%
\pgfpathlineto{\pgfqpoint{4.926398in}{4.122360in}}%
\pgfpathlineto{\pgfqpoint{4.926398in}{4.124208in}}%
\pgfpathlineto{\pgfqpoint{4.932010in}{4.124208in}}%
\pgfpathlineto{\pgfqpoint{4.932590in}{4.127904in}}%
\pgfpathlineto{\pgfqpoint{4.934486in}{4.127904in}}%
\pgfpathlineto{\pgfqpoint{4.934486in}{4.129752in}}%
\pgfpathlineto{\pgfqpoint{4.942040in}{4.129752in}}%
\pgfpathlineto{\pgfqpoint{4.942040in}{4.131600in}}%
\pgfpathlineto{\pgfqpoint{4.945445in}{4.131600in}}%
\pgfpathlineto{\pgfqpoint{4.945445in}{4.133448in}}%
\pgfpathlineto{\pgfqpoint{4.946975in}{4.133448in}}%
\pgfpathlineto{\pgfqpoint{4.946975in}{4.135296in}}%
\pgfpathlineto{\pgfqpoint{4.949585in}{4.135296in}}%
\pgfpathlineto{\pgfqpoint{4.950614in}{4.138992in}}%
\pgfpathlineto{\pgfqpoint{4.956448in}{4.138992in}}%
\pgfpathlineto{\pgfqpoint{4.956448in}{4.140840in}}%
\pgfpathlineto{\pgfqpoint{4.967407in}{4.140840in}}%
\pgfpathlineto{\pgfqpoint{4.967407in}{4.142688in}}%
\pgfpathlineto{\pgfqpoint{4.972580in}{4.142688in}}%
\pgfpathlineto{\pgfqpoint{4.973152in}{4.146384in}}%
\pgfpathlineto{\pgfqpoint{4.978509in}{4.146384in}}%
\pgfpathlineto{\pgfqpoint{4.978509in}{4.148232in}}%
\pgfpathlineto{\pgfqpoint{4.982786in}{4.148232in}}%
\pgfpathlineto{\pgfqpoint{4.982786in}{4.150080in}}%
\pgfpathlineto{\pgfqpoint{4.989129in}{4.150080in}}%
\pgfpathlineto{\pgfqpoint{4.989129in}{4.151928in}}%
\pgfpathlineto{\pgfqpoint{4.992083in}{4.151928in}}%
\pgfpathlineto{\pgfqpoint{4.992083in}{4.153776in}}%
\pgfpathlineto{\pgfqpoint{4.994341in}{4.153776in}}%
\pgfpathlineto{\pgfqpoint{4.994341in}{4.155624in}}%
\pgfpathlineto{\pgfqpoint{5.000494in}{4.155624in}}%
\pgfpathlineto{\pgfqpoint{5.000494in}{4.157472in}}%
\pgfpathlineto{\pgfqpoint{5.006890in}{4.157472in}}%
\pgfpathlineto{\pgfqpoint{5.006890in}{4.159320in}}%
\pgfpathlineto{\pgfqpoint{5.009853in}{4.159320in}}%
\pgfpathlineto{\pgfqpoint{5.009853in}{4.161168in}}%
\pgfpathlineto{\pgfqpoint{5.012218in}{4.161168in}}%
\pgfpathlineto{\pgfqpoint{5.012218in}{4.163016in}}%
\pgfpathlineto{\pgfqpoint{5.022222in}{4.163016in}}%
\pgfpathlineto{\pgfqpoint{5.022222in}{4.164864in}}%
\pgfpathlineto{\pgfqpoint{5.028741in}{4.164864in}}%
\pgfpathlineto{\pgfqpoint{5.028741in}{4.166712in}}%
\pgfpathlineto{\pgfqpoint{5.033105in}{4.166712in}}%
\pgfpathlineto{\pgfqpoint{5.033105in}{4.168560in}}%
\pgfpathlineto{\pgfqpoint{5.043577in}{4.168560in}}%
\pgfpathlineto{\pgfqpoint{5.043577in}{4.170408in}}%
\pgfpathlineto{\pgfqpoint{5.050633in}{4.170408in}}%
\pgfpathlineto{\pgfqpoint{5.051051in}{4.174104in}}%
\pgfpathlineto{\pgfqpoint{5.062118in}{4.174104in}}%
\pgfpathlineto{\pgfqpoint{5.062118in}{4.175952in}}%
\pgfpathlineto{\pgfqpoint{5.072840in}{4.175952in}}%
\pgfpathlineto{\pgfqpoint{5.072840in}{4.177800in}}%
\pgfpathlineto{\pgfqpoint{5.083318in}{4.177800in}}%
\pgfpathlineto{\pgfqpoint{5.083318in}{4.179648in}}%
\pgfpathlineto{\pgfqpoint{5.094464in}{4.179648in}}%
\pgfpathlineto{\pgfqpoint{5.094464in}{4.181496in}}%
\pgfpathlineto{\pgfqpoint{5.104933in}{4.181496in}}%
\pgfpathlineto{\pgfqpoint{5.104933in}{4.183344in}}%
\pgfpathlineto{\pgfqpoint{5.111975in}{4.183344in}}%
\pgfpathlineto{\pgfqpoint{5.111975in}{4.185192in}}%
\pgfpathlineto{\pgfqpoint{5.133459in}{4.185192in}}%
\pgfpathlineto{\pgfqpoint{5.133459in}{4.187040in}}%
\pgfpathlineto{\pgfqpoint{5.154999in}{4.187040in}}%
\pgfpathlineto{\pgfqpoint{5.154999in}{4.188888in}}%
\pgfpathlineto{\pgfqpoint{5.171277in}{4.188888in}}%
\pgfpathlineto{\pgfqpoint{5.171277in}{4.190736in}}%
\pgfpathlineto{\pgfqpoint{5.172464in}{4.190736in}}%
\pgfpathlineto{\pgfqpoint{5.172464in}{4.192584in}}%
\pgfpathlineto{\pgfqpoint{5.193975in}{4.192584in}}%
\pgfpathlineto{\pgfqpoint{5.193975in}{4.194432in}}%
\pgfpathlineto{\pgfqpoint{5.215116in}{4.194432in}}%
\pgfpathlineto{\pgfqpoint{5.215116in}{4.196280in}}%
\pgfpathlineto{\pgfqpoint{5.232886in}{4.196280in}}%
\pgfpathlineto{\pgfqpoint{5.232886in}{4.198128in}}%
\pgfpathlineto{\pgfqpoint{5.254281in}{4.198128in}}%
\pgfpathlineto{\pgfqpoint{5.254281in}{4.199976in}}%
\pgfpathlineto{\pgfqpoint{5.275571in}{4.199976in}}%
\pgfpathlineto{\pgfqpoint{5.275571in}{4.201824in}}%
\pgfpathlineto{\pgfqpoint{5.292252in}{4.201824in}}%
\pgfpathlineto{\pgfqpoint{5.292252in}{4.203672in}}%
\pgfpathlineto{\pgfqpoint{5.313347in}{4.203672in}}%
\pgfpathlineto{\pgfqpoint{5.313347in}{4.205520in}}%
\pgfpathlineto{\pgfqpoint{5.334440in}{4.205520in}}%
\pgfpathlineto{\pgfqpoint{5.334440in}{4.207368in}}%
\pgfpathlineto{\pgfqpoint{5.351381in}{4.207368in}}%
\pgfpathlineto{\pgfqpoint{5.351381in}{4.209216in}}%
\pgfpathlineto{\pgfqpoint{5.372154in}{4.209216in}}%
\pgfpathlineto{\pgfqpoint{5.372154in}{4.211064in}}%
\pgfpathlineto{\pgfqpoint{5.416387in}{4.211064in}}%
\pgfpathlineto{\pgfqpoint{5.416387in}{4.212912in}}%
\pgfpathlineto{\pgfqpoint{5.437035in}{4.212912in}}%
\pgfpathlineto{\pgfqpoint{5.437035in}{4.214760in}}%
\pgfpathlineto{\pgfqpoint{5.457589in}{4.214760in}}%
\pgfpathlineto{\pgfqpoint{5.457589in}{4.216608in}}%
\pgfpathlineto{\pgfqpoint{5.474574in}{4.216608in}}%
\pgfpathlineto{\pgfqpoint{5.474574in}{4.218456in}}%
\pgfpathlineto{\pgfqpoint{5.495286in}{4.218456in}}%
\pgfpathlineto{\pgfqpoint{5.495286in}{4.220304in}}%
\pgfpathlineto{\pgfqpoint{5.514191in}{4.220304in}}%
\pgfpathlineto{\pgfqpoint{5.514191in}{4.222152in}}%
\pgfpathlineto{\pgfqpoint{5.534545in}{4.222152in}}%
\pgfpathlineto{\pgfqpoint{5.534545in}{4.224000in}}%
\pgfpathlineto{\pgfqpoint{5.534545in}{4.224000in}}%
\pgfusepath{stroke}%
\end{pgfscope}%
\begin{pgfscope}%
\pgfsetrectcap%
\pgfsetmiterjoin%
\pgfsetlinewidth{0.803000pt}%
\definecolor{currentstroke}{rgb}{0.000000,0.000000,0.000000}%
\pgfsetstrokecolor{currentstroke}%
\pgfsetdash{}{0pt}%
\pgfpathmoveto{\pgfqpoint{0.800000in}{0.528000in}}%
\pgfpathlineto{\pgfqpoint{0.800000in}{4.224000in}}%
\pgfusepath{stroke}%
\end{pgfscope}%
\begin{pgfscope}%
\pgfsetrectcap%
\pgfsetmiterjoin%
\pgfsetlinewidth{0.803000pt}%
\definecolor{currentstroke}{rgb}{0.000000,0.000000,0.000000}%
\pgfsetstrokecolor{currentstroke}%
\pgfsetdash{}{0pt}%
\pgfpathmoveto{\pgfqpoint{5.760000in}{0.528000in}}%
\pgfpathlineto{\pgfqpoint{5.760000in}{4.224000in}}%
\pgfusepath{stroke}%
\end{pgfscope}%
\begin{pgfscope}%
\pgfsetrectcap%
\pgfsetmiterjoin%
\pgfsetlinewidth{0.803000pt}%
\definecolor{currentstroke}{rgb}{0.000000,0.000000,0.000000}%
\pgfsetstrokecolor{currentstroke}%
\pgfsetdash{}{0pt}%
\pgfpathmoveto{\pgfqpoint{0.800000in}{0.528000in}}%
\pgfpathlineto{\pgfqpoint{5.760000in}{0.528000in}}%
\pgfusepath{stroke}%
\end{pgfscope}%
\begin{pgfscope}%
\pgfsetrectcap%
\pgfsetmiterjoin%
\pgfsetlinewidth{0.803000pt}%
\definecolor{currentstroke}{rgb}{0.000000,0.000000,0.000000}%
\pgfsetstrokecolor{currentstroke}%
\pgfsetdash{}{0pt}%
\pgfpathmoveto{\pgfqpoint{0.800000in}{4.224000in}}%
\pgfpathlineto{\pgfqpoint{5.760000in}{4.224000in}}%
\pgfusepath{stroke}%
\end{pgfscope}%
\end{pgfpicture}%
\makeatother%
\endgroup%
}
\caption{Cumulative latency plot for the system when exhibiting stable latency.}
\label{ecdfmininet}
\end{figure}

I benchmarked the performance of the system in a simulated Mininet network \cite{mininet,lantzNetworkLaptopRapid2010} with link latency of 100ms between each node. All experiments were run for 10s on a system with 4 nodes and a batch size of 300. Figure~\ref{goodputlatencymininet} omits results with latency of above 4s, to show the performance of the system as it becomes overloaded. Figure~\ref{ecdfstable} is a cumulative latency plot of an experiment with a throughput of 200req/s.

The system has significantly higher latency in the simulator due to the five round trips that are required by the HotStuff protocol: this is an inherent bottleneck. We observe a median latency of around 1.3s (Figure~\ref{goodputlatencymininet}) until the system becomes overloaded when it reaches a maximum goodput of 200req/s (Figure~\ref{throughoutgoodputmininet}), and latency increases rapidly as commands queue on the nodes. One could predict a latency of around 1.1s in the simulated network, as a request takes 5 round trips to commit (1s total), and Figure~\ref{goodputlatencynodes} shows that the latency with negligible link latency was around 0.1s. The observed latency may be slightly greater (by 0.2s) as the nodes do not have synchronised clocks, so the increased link latency could cause delays in the view being advanced.

\subsection{View-changes} \label{viewchangeeval}

\begin{figure}[h!]
\centering
\resizebox{\textwidth}{!}{%% Creator: Matplotlib, PGF backend
%%
%% To include the figure in your LaTeX document, write
%%   \input{<filename>.pgf}
%%
%% Make sure the required packages are loaded in your preamble
%%   \usepackage{pgf}
%%
%% Also ensure that all the required font packages are loaded; for instance,
%% the lmodern package is sometimes necessary when using math font.
%%   \usepackage{lmodern}
%%
%% Figures using additional raster images can only be included by \input if
%% they are in the same directory as the main LaTeX file. For loading figures
%% from other directories you can use the `import` package
%%   \usepackage{import}
%%
%% and then include the figures with
%%   \import{<path to file>}{<filename>.pgf}
%%
%% Matplotlib used the following preamble
%%   
%%   \usepackage{fontspec}
%%   \setmainfont{DejaVuSerif.ttf}[Path=\detokenize{/opt/homebrew/lib/python3.10/site-packages/matplotlib/mpl-data/fonts/ttf/}]
%%   \setsansfont{DejaVuSans.ttf}[Path=\detokenize{/opt/homebrew/lib/python3.10/site-packages/matplotlib/mpl-data/fonts/ttf/}]
%%   \setmonofont{DejaVuSansMono.ttf}[Path=\detokenize{/opt/homebrew/lib/python3.10/site-packages/matplotlib/mpl-data/fonts/ttf/}]
%%   \makeatletter\@ifpackageloaded{underscore}{}{\usepackage[strings]{underscore}}\makeatother
%%
\begingroup%
\makeatletter%
\begin{pgfpicture}%
\pgfpathrectangle{\pgfpointorigin}{\pgfqpoint{10.000000in}{5.000000in}}%
\pgfusepath{use as bounding box, clip}%
\begin{pgfscope}%
\pgfsetbuttcap%
\pgfsetmiterjoin%
\definecolor{currentfill}{rgb}{1.000000,1.000000,1.000000}%
\pgfsetfillcolor{currentfill}%
\pgfsetlinewidth{0.000000pt}%
\definecolor{currentstroke}{rgb}{1.000000,1.000000,1.000000}%
\pgfsetstrokecolor{currentstroke}%
\pgfsetdash{}{0pt}%
\pgfpathmoveto{\pgfqpoint{0.000000in}{0.000000in}}%
\pgfpathlineto{\pgfqpoint{10.000000in}{0.000000in}}%
\pgfpathlineto{\pgfqpoint{10.000000in}{5.000000in}}%
\pgfpathlineto{\pgfqpoint{0.000000in}{5.000000in}}%
\pgfpathlineto{\pgfqpoint{0.000000in}{0.000000in}}%
\pgfpathclose%
\pgfusepath{fill}%
\end{pgfscope}%
\begin{pgfscope}%
\pgfsetbuttcap%
\pgfsetmiterjoin%
\definecolor{currentfill}{rgb}{1.000000,1.000000,1.000000}%
\pgfsetfillcolor{currentfill}%
\pgfsetlinewidth{0.000000pt}%
\definecolor{currentstroke}{rgb}{0.000000,0.000000,0.000000}%
\pgfsetstrokecolor{currentstroke}%
\pgfsetstrokeopacity{0.000000}%
\pgfsetdash{}{0pt}%
\pgfpathmoveto{\pgfqpoint{0.549740in}{0.463273in}}%
\pgfpathlineto{\pgfqpoint{9.869965in}{0.463273in}}%
\pgfpathlineto{\pgfqpoint{9.869965in}{4.958330in}}%
\pgfpathlineto{\pgfqpoint{0.549740in}{4.958330in}}%
\pgfpathlineto{\pgfqpoint{0.549740in}{0.463273in}}%
\pgfpathclose%
\pgfusepath{fill}%
\end{pgfscope}%
\begin{pgfscope}%
\pgfpathrectangle{\pgfqpoint{0.549740in}{0.463273in}}{\pgfqpoint{9.320225in}{4.495057in}}%
\pgfusepath{clip}%
\pgfsetbuttcap%
\pgfsetroundjoin%
\pgfsetlinewidth{0.000000pt}%
\definecolor{currentstroke}{rgb}{0.000000,0.000000,0.000000}%
\pgfsetstrokecolor{currentstroke}%
\pgfsetdash{}{0pt}%
\pgfpathmoveto{\pgfqpoint{0.549761in}{0.667594in}}%
\pgfpathlineto{\pgfqpoint{0.735988in}{0.667594in}}%
\pgfpathlineto{\pgfqpoint{0.735988in}{0.749322in}}%
\pgfpathlineto{\pgfqpoint{0.549761in}{0.749322in}}%
\pgfpathlineto{\pgfqpoint{0.549761in}{0.667594in}}%
\pgfusepath{}%
\end{pgfscope}%
\begin{pgfscope}%
\pgfpathrectangle{\pgfqpoint{0.549740in}{0.463273in}}{\pgfqpoint{9.320225in}{4.495057in}}%
\pgfusepath{clip}%
\pgfsetbuttcap%
\pgfsetroundjoin%
\pgfsetlinewidth{0.000000pt}%
\definecolor{currentstroke}{rgb}{0.000000,0.000000,0.000000}%
\pgfsetstrokecolor{currentstroke}%
\pgfsetdash{}{0pt}%
\pgfpathmoveto{\pgfqpoint{0.735988in}{0.667594in}}%
\pgfpathlineto{\pgfqpoint{0.922214in}{0.667594in}}%
\pgfpathlineto{\pgfqpoint{0.922214in}{0.749322in}}%
\pgfpathlineto{\pgfqpoint{0.735988in}{0.749322in}}%
\pgfpathlineto{\pgfqpoint{0.735988in}{0.667594in}}%
\pgfusepath{}%
\end{pgfscope}%
\begin{pgfscope}%
\pgfpathrectangle{\pgfqpoint{0.549740in}{0.463273in}}{\pgfqpoint{9.320225in}{4.495057in}}%
\pgfusepath{clip}%
\pgfsetbuttcap%
\pgfsetroundjoin%
\pgfsetlinewidth{0.000000pt}%
\definecolor{currentstroke}{rgb}{0.000000,0.000000,0.000000}%
\pgfsetstrokecolor{currentstroke}%
\pgfsetdash{}{0pt}%
\pgfpathmoveto{\pgfqpoint{0.922214in}{0.667594in}}%
\pgfpathlineto{\pgfqpoint{1.108441in}{0.667594in}}%
\pgfpathlineto{\pgfqpoint{1.108441in}{0.749322in}}%
\pgfpathlineto{\pgfqpoint{0.922214in}{0.749322in}}%
\pgfpathlineto{\pgfqpoint{0.922214in}{0.667594in}}%
\pgfusepath{}%
\end{pgfscope}%
\begin{pgfscope}%
\pgfpathrectangle{\pgfqpoint{0.549740in}{0.463273in}}{\pgfqpoint{9.320225in}{4.495057in}}%
\pgfusepath{clip}%
\pgfsetbuttcap%
\pgfsetroundjoin%
\pgfsetlinewidth{0.000000pt}%
\definecolor{currentstroke}{rgb}{0.000000,0.000000,0.000000}%
\pgfsetstrokecolor{currentstroke}%
\pgfsetdash{}{0pt}%
\pgfpathmoveto{\pgfqpoint{1.108441in}{0.667594in}}%
\pgfpathlineto{\pgfqpoint{1.294667in}{0.667594in}}%
\pgfpathlineto{\pgfqpoint{1.294667in}{0.749322in}}%
\pgfpathlineto{\pgfqpoint{1.108441in}{0.749322in}}%
\pgfpathlineto{\pgfqpoint{1.108441in}{0.667594in}}%
\pgfusepath{}%
\end{pgfscope}%
\begin{pgfscope}%
\pgfpathrectangle{\pgfqpoint{0.549740in}{0.463273in}}{\pgfqpoint{9.320225in}{4.495057in}}%
\pgfusepath{clip}%
\pgfsetbuttcap%
\pgfsetroundjoin%
\pgfsetlinewidth{0.000000pt}%
\definecolor{currentstroke}{rgb}{0.000000,0.000000,0.000000}%
\pgfsetstrokecolor{currentstroke}%
\pgfsetdash{}{0pt}%
\pgfpathmoveto{\pgfqpoint{1.294667in}{0.667594in}}%
\pgfpathlineto{\pgfqpoint{1.480894in}{0.667594in}}%
\pgfpathlineto{\pgfqpoint{1.480894in}{0.749322in}}%
\pgfpathlineto{\pgfqpoint{1.294667in}{0.749322in}}%
\pgfpathlineto{\pgfqpoint{1.294667in}{0.667594in}}%
\pgfusepath{}%
\end{pgfscope}%
\begin{pgfscope}%
\pgfpathrectangle{\pgfqpoint{0.549740in}{0.463273in}}{\pgfqpoint{9.320225in}{4.495057in}}%
\pgfusepath{clip}%
\pgfsetbuttcap%
\pgfsetroundjoin%
\pgfsetlinewidth{0.000000pt}%
\definecolor{currentstroke}{rgb}{0.000000,0.000000,0.000000}%
\pgfsetstrokecolor{currentstroke}%
\pgfsetdash{}{0pt}%
\pgfpathmoveto{\pgfqpoint{1.480894in}{0.667594in}}%
\pgfpathlineto{\pgfqpoint{1.667120in}{0.667594in}}%
\pgfpathlineto{\pgfqpoint{1.667120in}{0.749322in}}%
\pgfpathlineto{\pgfqpoint{1.480894in}{0.749322in}}%
\pgfpathlineto{\pgfqpoint{1.480894in}{0.667594in}}%
\pgfusepath{}%
\end{pgfscope}%
\begin{pgfscope}%
\pgfpathrectangle{\pgfqpoint{0.549740in}{0.463273in}}{\pgfqpoint{9.320225in}{4.495057in}}%
\pgfusepath{clip}%
\pgfsetbuttcap%
\pgfsetroundjoin%
\pgfsetlinewidth{0.000000pt}%
\definecolor{currentstroke}{rgb}{0.000000,0.000000,0.000000}%
\pgfsetstrokecolor{currentstroke}%
\pgfsetdash{}{0pt}%
\pgfpathmoveto{\pgfqpoint{1.667120in}{0.667594in}}%
\pgfpathlineto{\pgfqpoint{1.853347in}{0.667594in}}%
\pgfpathlineto{\pgfqpoint{1.853347in}{0.749322in}}%
\pgfpathlineto{\pgfqpoint{1.667120in}{0.749322in}}%
\pgfpathlineto{\pgfqpoint{1.667120in}{0.667594in}}%
\pgfusepath{}%
\end{pgfscope}%
\begin{pgfscope}%
\pgfpathrectangle{\pgfqpoint{0.549740in}{0.463273in}}{\pgfqpoint{9.320225in}{4.495057in}}%
\pgfusepath{clip}%
\pgfsetbuttcap%
\pgfsetroundjoin%
\pgfsetlinewidth{0.000000pt}%
\definecolor{currentstroke}{rgb}{0.000000,0.000000,0.000000}%
\pgfsetstrokecolor{currentstroke}%
\pgfsetdash{}{0pt}%
\pgfpathmoveto{\pgfqpoint{1.853347in}{0.667594in}}%
\pgfpathlineto{\pgfqpoint{2.039573in}{0.667594in}}%
\pgfpathlineto{\pgfqpoint{2.039573in}{0.749322in}}%
\pgfpathlineto{\pgfqpoint{1.853347in}{0.749322in}}%
\pgfpathlineto{\pgfqpoint{1.853347in}{0.667594in}}%
\pgfusepath{}%
\end{pgfscope}%
\begin{pgfscope}%
\pgfpathrectangle{\pgfqpoint{0.549740in}{0.463273in}}{\pgfqpoint{9.320225in}{4.495057in}}%
\pgfusepath{clip}%
\pgfsetbuttcap%
\pgfsetroundjoin%
\pgfsetlinewidth{0.000000pt}%
\definecolor{currentstroke}{rgb}{0.000000,0.000000,0.000000}%
\pgfsetstrokecolor{currentstroke}%
\pgfsetdash{}{0pt}%
\pgfpathmoveto{\pgfqpoint{2.039573in}{0.667594in}}%
\pgfpathlineto{\pgfqpoint{2.225800in}{0.667594in}}%
\pgfpathlineto{\pgfqpoint{2.225800in}{0.749322in}}%
\pgfpathlineto{\pgfqpoint{2.039573in}{0.749322in}}%
\pgfpathlineto{\pgfqpoint{2.039573in}{0.667594in}}%
\pgfusepath{}%
\end{pgfscope}%
\begin{pgfscope}%
\pgfpathrectangle{\pgfqpoint{0.549740in}{0.463273in}}{\pgfqpoint{9.320225in}{4.495057in}}%
\pgfusepath{clip}%
\pgfsetbuttcap%
\pgfsetroundjoin%
\definecolor{currentfill}{rgb}{0.472869,0.711325,0.955316}%
\pgfsetfillcolor{currentfill}%
\pgfsetlinewidth{0.000000pt}%
\definecolor{currentstroke}{rgb}{0.000000,0.000000,0.000000}%
\pgfsetstrokecolor{currentstroke}%
\pgfsetdash{}{0pt}%
\pgfpathmoveto{\pgfqpoint{2.225800in}{0.667594in}}%
\pgfpathlineto{\pgfqpoint{2.412027in}{0.667594in}}%
\pgfpathlineto{\pgfqpoint{2.412027in}{0.749322in}}%
\pgfpathlineto{\pgfqpoint{2.225800in}{0.749322in}}%
\pgfpathlineto{\pgfqpoint{2.225800in}{0.667594in}}%
\pgfusepath{fill}%
\end{pgfscope}%
\begin{pgfscope}%
\pgfpathrectangle{\pgfqpoint{0.549740in}{0.463273in}}{\pgfqpoint{9.320225in}{4.495057in}}%
\pgfusepath{clip}%
\pgfsetbuttcap%
\pgfsetroundjoin%
\definecolor{currentfill}{rgb}{0.547810,0.736432,0.947518}%
\pgfsetfillcolor{currentfill}%
\pgfsetlinewidth{0.000000pt}%
\definecolor{currentstroke}{rgb}{0.000000,0.000000,0.000000}%
\pgfsetstrokecolor{currentstroke}%
\pgfsetdash{}{0pt}%
\pgfpathmoveto{\pgfqpoint{2.412027in}{0.667594in}}%
\pgfpathlineto{\pgfqpoint{2.598253in}{0.667594in}}%
\pgfpathlineto{\pgfqpoint{2.598253in}{0.749322in}}%
\pgfpathlineto{\pgfqpoint{2.412027in}{0.749322in}}%
\pgfpathlineto{\pgfqpoint{2.412027in}{0.667594in}}%
\pgfusepath{fill}%
\end{pgfscope}%
\begin{pgfscope}%
\pgfpathrectangle{\pgfqpoint{0.549740in}{0.463273in}}{\pgfqpoint{9.320225in}{4.495057in}}%
\pgfusepath{clip}%
\pgfsetbuttcap%
\pgfsetroundjoin%
\definecolor{currentfill}{rgb}{0.547810,0.736432,0.947518}%
\pgfsetfillcolor{currentfill}%
\pgfsetlinewidth{0.000000pt}%
\definecolor{currentstroke}{rgb}{0.000000,0.000000,0.000000}%
\pgfsetstrokecolor{currentstroke}%
\pgfsetdash{}{0pt}%
\pgfpathmoveto{\pgfqpoint{2.598253in}{0.667594in}}%
\pgfpathlineto{\pgfqpoint{2.784480in}{0.667594in}}%
\pgfpathlineto{\pgfqpoint{2.784480in}{0.749322in}}%
\pgfpathlineto{\pgfqpoint{2.598253in}{0.749322in}}%
\pgfpathlineto{\pgfqpoint{2.598253in}{0.667594in}}%
\pgfusepath{fill}%
\end{pgfscope}%
\begin{pgfscope}%
\pgfpathrectangle{\pgfqpoint{0.549740in}{0.463273in}}{\pgfqpoint{9.320225in}{4.495057in}}%
\pgfusepath{clip}%
\pgfsetbuttcap%
\pgfsetroundjoin%
\definecolor{currentfill}{rgb}{0.547810,0.736432,0.947518}%
\pgfsetfillcolor{currentfill}%
\pgfsetlinewidth{0.000000pt}%
\definecolor{currentstroke}{rgb}{0.000000,0.000000,0.000000}%
\pgfsetstrokecolor{currentstroke}%
\pgfsetdash{}{0pt}%
\pgfpathmoveto{\pgfqpoint{2.784480in}{0.667594in}}%
\pgfpathlineto{\pgfqpoint{2.970706in}{0.667594in}}%
\pgfpathlineto{\pgfqpoint{2.970706in}{0.749322in}}%
\pgfpathlineto{\pgfqpoint{2.784480in}{0.749322in}}%
\pgfpathlineto{\pgfqpoint{2.784480in}{0.667594in}}%
\pgfusepath{fill}%
\end{pgfscope}%
\begin{pgfscope}%
\pgfpathrectangle{\pgfqpoint{0.549740in}{0.463273in}}{\pgfqpoint{9.320225in}{4.495057in}}%
\pgfusepath{clip}%
\pgfsetbuttcap%
\pgfsetroundjoin%
\definecolor{currentfill}{rgb}{0.472869,0.711325,0.955316}%
\pgfsetfillcolor{currentfill}%
\pgfsetlinewidth{0.000000pt}%
\definecolor{currentstroke}{rgb}{0.000000,0.000000,0.000000}%
\pgfsetstrokecolor{currentstroke}%
\pgfsetdash{}{0pt}%
\pgfpathmoveto{\pgfqpoint{2.970706in}{0.667594in}}%
\pgfpathlineto{\pgfqpoint{3.156933in}{0.667594in}}%
\pgfpathlineto{\pgfqpoint{3.156933in}{0.749322in}}%
\pgfpathlineto{\pgfqpoint{2.970706in}{0.749322in}}%
\pgfpathlineto{\pgfqpoint{2.970706in}{0.667594in}}%
\pgfusepath{fill}%
\end{pgfscope}%
\begin{pgfscope}%
\pgfpathrectangle{\pgfqpoint{0.549740in}{0.463273in}}{\pgfqpoint{9.320225in}{4.495057in}}%
\pgfusepath{clip}%
\pgfsetbuttcap%
\pgfsetroundjoin%
\definecolor{currentfill}{rgb}{0.273225,0.662144,0.968515}%
\pgfsetfillcolor{currentfill}%
\pgfsetlinewidth{0.000000pt}%
\definecolor{currentstroke}{rgb}{0.000000,0.000000,0.000000}%
\pgfsetstrokecolor{currentstroke}%
\pgfsetdash{}{0pt}%
\pgfpathmoveto{\pgfqpoint{3.156933in}{0.667594in}}%
\pgfpathlineto{\pgfqpoint{3.343159in}{0.667594in}}%
\pgfpathlineto{\pgfqpoint{3.343159in}{0.749322in}}%
\pgfpathlineto{\pgfqpoint{3.156933in}{0.749322in}}%
\pgfpathlineto{\pgfqpoint{3.156933in}{0.667594in}}%
\pgfusepath{fill}%
\end{pgfscope}%
\begin{pgfscope}%
\pgfpathrectangle{\pgfqpoint{0.549740in}{0.463273in}}{\pgfqpoint{9.320225in}{4.495057in}}%
\pgfusepath{clip}%
\pgfsetbuttcap%
\pgfsetroundjoin%
\definecolor{currentfill}{rgb}{0.547810,0.736432,0.947518}%
\pgfsetfillcolor{currentfill}%
\pgfsetlinewidth{0.000000pt}%
\definecolor{currentstroke}{rgb}{0.000000,0.000000,0.000000}%
\pgfsetstrokecolor{currentstroke}%
\pgfsetdash{}{0pt}%
\pgfpathmoveto{\pgfqpoint{3.343159in}{0.667594in}}%
\pgfpathlineto{\pgfqpoint{3.529386in}{0.667594in}}%
\pgfpathlineto{\pgfqpoint{3.529386in}{0.749322in}}%
\pgfpathlineto{\pgfqpoint{3.343159in}{0.749322in}}%
\pgfpathlineto{\pgfqpoint{3.343159in}{0.667594in}}%
\pgfusepath{fill}%
\end{pgfscope}%
\begin{pgfscope}%
\pgfpathrectangle{\pgfqpoint{0.549740in}{0.463273in}}{\pgfqpoint{9.320225in}{4.495057in}}%
\pgfusepath{clip}%
\pgfsetbuttcap%
\pgfsetroundjoin%
\definecolor{currentfill}{rgb}{0.385185,0.686583,0.962589}%
\pgfsetfillcolor{currentfill}%
\pgfsetlinewidth{0.000000pt}%
\definecolor{currentstroke}{rgb}{0.000000,0.000000,0.000000}%
\pgfsetstrokecolor{currentstroke}%
\pgfsetdash{}{0pt}%
\pgfpathmoveto{\pgfqpoint{3.529386in}{0.667594in}}%
\pgfpathlineto{\pgfqpoint{3.715612in}{0.667594in}}%
\pgfpathlineto{\pgfqpoint{3.715612in}{0.749322in}}%
\pgfpathlineto{\pgfqpoint{3.529386in}{0.749322in}}%
\pgfpathlineto{\pgfqpoint{3.529386in}{0.667594in}}%
\pgfusepath{fill}%
\end{pgfscope}%
\begin{pgfscope}%
\pgfpathrectangle{\pgfqpoint{0.549740in}{0.463273in}}{\pgfqpoint{9.320225in}{4.495057in}}%
\pgfusepath{clip}%
\pgfsetbuttcap%
\pgfsetroundjoin%
\definecolor{currentfill}{rgb}{0.189527,0.635753,0.950228}%
\pgfsetfillcolor{currentfill}%
\pgfsetlinewidth{0.000000pt}%
\definecolor{currentstroke}{rgb}{0.000000,0.000000,0.000000}%
\pgfsetstrokecolor{currentstroke}%
\pgfsetdash{}{0pt}%
\pgfpathmoveto{\pgfqpoint{3.715612in}{0.667594in}}%
\pgfpathlineto{\pgfqpoint{3.901839in}{0.667594in}}%
\pgfpathlineto{\pgfqpoint{3.901839in}{0.749322in}}%
\pgfpathlineto{\pgfqpoint{3.715612in}{0.749322in}}%
\pgfpathlineto{\pgfqpoint{3.715612in}{0.667594in}}%
\pgfusepath{fill}%
\end{pgfscope}%
\begin{pgfscope}%
\pgfpathrectangle{\pgfqpoint{0.549740in}{0.463273in}}{\pgfqpoint{9.320225in}{4.495057in}}%
\pgfusepath{clip}%
\pgfsetbuttcap%
\pgfsetroundjoin%
\definecolor{currentfill}{rgb}{0.169712,0.607583,0.911334}%
\pgfsetfillcolor{currentfill}%
\pgfsetlinewidth{0.000000pt}%
\definecolor{currentstroke}{rgb}{0.000000,0.000000,0.000000}%
\pgfsetstrokecolor{currentstroke}%
\pgfsetdash{}{0pt}%
\pgfpathmoveto{\pgfqpoint{3.901839in}{0.667594in}}%
\pgfpathlineto{\pgfqpoint{4.088065in}{0.667594in}}%
\pgfpathlineto{\pgfqpoint{4.088065in}{0.749322in}}%
\pgfpathlineto{\pgfqpoint{3.901839in}{0.749322in}}%
\pgfpathlineto{\pgfqpoint{3.901839in}{0.667594in}}%
\pgfusepath{fill}%
\end{pgfscope}%
\begin{pgfscope}%
\pgfpathrectangle{\pgfqpoint{0.549740in}{0.463273in}}{\pgfqpoint{9.320225in}{4.495057in}}%
\pgfusepath{clip}%
\pgfsetbuttcap%
\pgfsetroundjoin%
\definecolor{currentfill}{rgb}{0.385185,0.686583,0.962589}%
\pgfsetfillcolor{currentfill}%
\pgfsetlinewidth{0.000000pt}%
\definecolor{currentstroke}{rgb}{0.000000,0.000000,0.000000}%
\pgfsetstrokecolor{currentstroke}%
\pgfsetdash{}{0pt}%
\pgfpathmoveto{\pgfqpoint{4.088065in}{0.667594in}}%
\pgfpathlineto{\pgfqpoint{4.274292in}{0.667594in}}%
\pgfpathlineto{\pgfqpoint{4.274292in}{0.749322in}}%
\pgfpathlineto{\pgfqpoint{4.088065in}{0.749322in}}%
\pgfpathlineto{\pgfqpoint{4.088065in}{0.667594in}}%
\pgfusepath{fill}%
\end{pgfscope}%
\begin{pgfscope}%
\pgfpathrectangle{\pgfqpoint{0.549740in}{0.463273in}}{\pgfqpoint{9.320225in}{4.495057in}}%
\pgfusepath{clip}%
\pgfsetbuttcap%
\pgfsetroundjoin%
\definecolor{currentfill}{rgb}{0.169712,0.607583,0.911334}%
\pgfsetfillcolor{currentfill}%
\pgfsetlinewidth{0.000000pt}%
\definecolor{currentstroke}{rgb}{0.000000,0.000000,0.000000}%
\pgfsetstrokecolor{currentstroke}%
\pgfsetdash{}{0pt}%
\pgfpathmoveto{\pgfqpoint{4.274292in}{0.667594in}}%
\pgfpathlineto{\pgfqpoint{4.460519in}{0.667594in}}%
\pgfpathlineto{\pgfqpoint{4.460519in}{0.749322in}}%
\pgfpathlineto{\pgfqpoint{4.274292in}{0.749322in}}%
\pgfpathlineto{\pgfqpoint{4.274292in}{0.667594in}}%
\pgfusepath{fill}%
\end{pgfscope}%
\begin{pgfscope}%
\pgfpathrectangle{\pgfqpoint{0.549740in}{0.463273in}}{\pgfqpoint{9.320225in}{4.495057in}}%
\pgfusepath{clip}%
\pgfsetbuttcap%
\pgfsetroundjoin%
\definecolor{currentfill}{rgb}{0.163625,0.579322,0.869386}%
\pgfsetfillcolor{currentfill}%
\pgfsetlinewidth{0.000000pt}%
\definecolor{currentstroke}{rgb}{0.000000,0.000000,0.000000}%
\pgfsetstrokecolor{currentstroke}%
\pgfsetdash{}{0pt}%
\pgfpathmoveto{\pgfqpoint{4.460519in}{0.667594in}}%
\pgfpathlineto{\pgfqpoint{4.646745in}{0.667594in}}%
\pgfpathlineto{\pgfqpoint{4.646745in}{0.749322in}}%
\pgfpathlineto{\pgfqpoint{4.460519in}{0.749322in}}%
\pgfpathlineto{\pgfqpoint{4.460519in}{0.667594in}}%
\pgfusepath{fill}%
\end{pgfscope}%
\begin{pgfscope}%
\pgfpathrectangle{\pgfqpoint{0.549740in}{0.463273in}}{\pgfqpoint{9.320225in}{4.495057in}}%
\pgfusepath{clip}%
\pgfsetbuttcap%
\pgfsetroundjoin%
\definecolor{currentfill}{rgb}{0.273225,0.662144,0.968515}%
\pgfsetfillcolor{currentfill}%
\pgfsetlinewidth{0.000000pt}%
\definecolor{currentstroke}{rgb}{0.000000,0.000000,0.000000}%
\pgfsetstrokecolor{currentstroke}%
\pgfsetdash{}{0pt}%
\pgfpathmoveto{\pgfqpoint{4.646745in}{0.667594in}}%
\pgfpathlineto{\pgfqpoint{4.832972in}{0.667594in}}%
\pgfpathlineto{\pgfqpoint{4.832972in}{0.749322in}}%
\pgfpathlineto{\pgfqpoint{4.646745in}{0.749322in}}%
\pgfpathlineto{\pgfqpoint{4.646745in}{0.667594in}}%
\pgfusepath{fill}%
\end{pgfscope}%
\begin{pgfscope}%
\pgfpathrectangle{\pgfqpoint{0.549740in}{0.463273in}}{\pgfqpoint{9.320225in}{4.495057in}}%
\pgfusepath{clip}%
\pgfsetbuttcap%
\pgfsetroundjoin%
\definecolor{currentfill}{rgb}{0.189527,0.635753,0.950228}%
\pgfsetfillcolor{currentfill}%
\pgfsetlinewidth{0.000000pt}%
\definecolor{currentstroke}{rgb}{0.000000,0.000000,0.000000}%
\pgfsetstrokecolor{currentstroke}%
\pgfsetdash{}{0pt}%
\pgfpathmoveto{\pgfqpoint{4.832972in}{0.667594in}}%
\pgfpathlineto{\pgfqpoint{5.019198in}{0.667594in}}%
\pgfpathlineto{\pgfqpoint{5.019198in}{0.749322in}}%
\pgfpathlineto{\pgfqpoint{4.832972in}{0.749322in}}%
\pgfpathlineto{\pgfqpoint{4.832972in}{0.667594in}}%
\pgfusepath{fill}%
\end{pgfscope}%
\begin{pgfscope}%
\pgfpathrectangle{\pgfqpoint{0.549740in}{0.463273in}}{\pgfqpoint{9.320225in}{4.495057in}}%
\pgfusepath{clip}%
\pgfsetbuttcap%
\pgfsetroundjoin%
\pgfsetlinewidth{0.000000pt}%
\definecolor{currentstroke}{rgb}{0.000000,0.000000,0.000000}%
\pgfsetstrokecolor{currentstroke}%
\pgfsetdash{}{0pt}%
\pgfpathmoveto{\pgfqpoint{5.019198in}{0.667594in}}%
\pgfpathlineto{\pgfqpoint{5.205425in}{0.667594in}}%
\pgfpathlineto{\pgfqpoint{5.205425in}{0.749322in}}%
\pgfpathlineto{\pgfqpoint{5.019198in}{0.749322in}}%
\pgfpathlineto{\pgfqpoint{5.019198in}{0.667594in}}%
\pgfusepath{}%
\end{pgfscope}%
\begin{pgfscope}%
\pgfpathrectangle{\pgfqpoint{0.549740in}{0.463273in}}{\pgfqpoint{9.320225in}{4.495057in}}%
\pgfusepath{clip}%
\pgfsetbuttcap%
\pgfsetroundjoin%
\pgfsetlinewidth{0.000000pt}%
\definecolor{currentstroke}{rgb}{0.000000,0.000000,0.000000}%
\pgfsetstrokecolor{currentstroke}%
\pgfsetdash{}{0pt}%
\pgfpathmoveto{\pgfqpoint{5.205425in}{0.667594in}}%
\pgfpathlineto{\pgfqpoint{5.391651in}{0.667594in}}%
\pgfpathlineto{\pgfqpoint{5.391651in}{0.749322in}}%
\pgfpathlineto{\pgfqpoint{5.205425in}{0.749322in}}%
\pgfpathlineto{\pgfqpoint{5.205425in}{0.667594in}}%
\pgfusepath{}%
\end{pgfscope}%
\begin{pgfscope}%
\pgfpathrectangle{\pgfqpoint{0.549740in}{0.463273in}}{\pgfqpoint{9.320225in}{4.495057in}}%
\pgfusepath{clip}%
\pgfsetbuttcap%
\pgfsetroundjoin%
\pgfsetlinewidth{0.000000pt}%
\definecolor{currentstroke}{rgb}{0.000000,0.000000,0.000000}%
\pgfsetstrokecolor{currentstroke}%
\pgfsetdash{}{0pt}%
\pgfpathmoveto{\pgfqpoint{5.391651in}{0.667594in}}%
\pgfpathlineto{\pgfqpoint{5.577878in}{0.667594in}}%
\pgfpathlineto{\pgfqpoint{5.577878in}{0.749322in}}%
\pgfpathlineto{\pgfqpoint{5.391651in}{0.749322in}}%
\pgfpathlineto{\pgfqpoint{5.391651in}{0.667594in}}%
\pgfusepath{}%
\end{pgfscope}%
\begin{pgfscope}%
\pgfpathrectangle{\pgfqpoint{0.549740in}{0.463273in}}{\pgfqpoint{9.320225in}{4.495057in}}%
\pgfusepath{clip}%
\pgfsetbuttcap%
\pgfsetroundjoin%
\pgfsetlinewidth{0.000000pt}%
\definecolor{currentstroke}{rgb}{0.000000,0.000000,0.000000}%
\pgfsetstrokecolor{currentstroke}%
\pgfsetdash{}{0pt}%
\pgfpathmoveto{\pgfqpoint{5.577878in}{0.667594in}}%
\pgfpathlineto{\pgfqpoint{5.764104in}{0.667594in}}%
\pgfpathlineto{\pgfqpoint{5.764104in}{0.749322in}}%
\pgfpathlineto{\pgfqpoint{5.577878in}{0.749322in}}%
\pgfpathlineto{\pgfqpoint{5.577878in}{0.667594in}}%
\pgfusepath{}%
\end{pgfscope}%
\begin{pgfscope}%
\pgfpathrectangle{\pgfqpoint{0.549740in}{0.463273in}}{\pgfqpoint{9.320225in}{4.495057in}}%
\pgfusepath{clip}%
\pgfsetbuttcap%
\pgfsetroundjoin%
\pgfsetlinewidth{0.000000pt}%
\definecolor{currentstroke}{rgb}{0.000000,0.000000,0.000000}%
\pgfsetstrokecolor{currentstroke}%
\pgfsetdash{}{0pt}%
\pgfpathmoveto{\pgfqpoint{5.764104in}{0.667594in}}%
\pgfpathlineto{\pgfqpoint{5.950331in}{0.667594in}}%
\pgfpathlineto{\pgfqpoint{5.950331in}{0.749322in}}%
\pgfpathlineto{\pgfqpoint{5.764104in}{0.749322in}}%
\pgfpathlineto{\pgfqpoint{5.764104in}{0.667594in}}%
\pgfusepath{}%
\end{pgfscope}%
\begin{pgfscope}%
\pgfpathrectangle{\pgfqpoint{0.549740in}{0.463273in}}{\pgfqpoint{9.320225in}{4.495057in}}%
\pgfusepath{clip}%
\pgfsetbuttcap%
\pgfsetroundjoin%
\pgfsetlinewidth{0.000000pt}%
\definecolor{currentstroke}{rgb}{0.000000,0.000000,0.000000}%
\pgfsetstrokecolor{currentstroke}%
\pgfsetdash{}{0pt}%
\pgfpathmoveto{\pgfqpoint{5.950331in}{0.667594in}}%
\pgfpathlineto{\pgfqpoint{6.136557in}{0.667594in}}%
\pgfpathlineto{\pgfqpoint{6.136557in}{0.749322in}}%
\pgfpathlineto{\pgfqpoint{5.950331in}{0.749322in}}%
\pgfpathlineto{\pgfqpoint{5.950331in}{0.667594in}}%
\pgfusepath{}%
\end{pgfscope}%
\begin{pgfscope}%
\pgfpathrectangle{\pgfqpoint{0.549740in}{0.463273in}}{\pgfqpoint{9.320225in}{4.495057in}}%
\pgfusepath{clip}%
\pgfsetbuttcap%
\pgfsetroundjoin%
\pgfsetlinewidth{0.000000pt}%
\definecolor{currentstroke}{rgb}{0.000000,0.000000,0.000000}%
\pgfsetstrokecolor{currentstroke}%
\pgfsetdash{}{0pt}%
\pgfpathmoveto{\pgfqpoint{6.136557in}{0.667594in}}%
\pgfpathlineto{\pgfqpoint{6.322784in}{0.667594in}}%
\pgfpathlineto{\pgfqpoint{6.322784in}{0.749322in}}%
\pgfpathlineto{\pgfqpoint{6.136557in}{0.749322in}}%
\pgfpathlineto{\pgfqpoint{6.136557in}{0.667594in}}%
\pgfusepath{}%
\end{pgfscope}%
\begin{pgfscope}%
\pgfpathrectangle{\pgfqpoint{0.549740in}{0.463273in}}{\pgfqpoint{9.320225in}{4.495057in}}%
\pgfusepath{clip}%
\pgfsetbuttcap%
\pgfsetroundjoin%
\pgfsetlinewidth{0.000000pt}%
\definecolor{currentstroke}{rgb}{0.000000,0.000000,0.000000}%
\pgfsetstrokecolor{currentstroke}%
\pgfsetdash{}{0pt}%
\pgfpathmoveto{\pgfqpoint{6.322784in}{0.667594in}}%
\pgfpathlineto{\pgfqpoint{6.509011in}{0.667594in}}%
\pgfpathlineto{\pgfqpoint{6.509011in}{0.749322in}}%
\pgfpathlineto{\pgfqpoint{6.322784in}{0.749322in}}%
\pgfpathlineto{\pgfqpoint{6.322784in}{0.667594in}}%
\pgfusepath{}%
\end{pgfscope}%
\begin{pgfscope}%
\pgfpathrectangle{\pgfqpoint{0.549740in}{0.463273in}}{\pgfqpoint{9.320225in}{4.495057in}}%
\pgfusepath{clip}%
\pgfsetbuttcap%
\pgfsetroundjoin%
\pgfsetlinewidth{0.000000pt}%
\definecolor{currentstroke}{rgb}{0.000000,0.000000,0.000000}%
\pgfsetstrokecolor{currentstroke}%
\pgfsetdash{}{0pt}%
\pgfpathmoveto{\pgfqpoint{6.509011in}{0.667594in}}%
\pgfpathlineto{\pgfqpoint{6.695237in}{0.667594in}}%
\pgfpathlineto{\pgfqpoint{6.695237in}{0.749322in}}%
\pgfpathlineto{\pgfqpoint{6.509011in}{0.749322in}}%
\pgfpathlineto{\pgfqpoint{6.509011in}{0.667594in}}%
\pgfusepath{}%
\end{pgfscope}%
\begin{pgfscope}%
\pgfpathrectangle{\pgfqpoint{0.549740in}{0.463273in}}{\pgfqpoint{9.320225in}{4.495057in}}%
\pgfusepath{clip}%
\pgfsetbuttcap%
\pgfsetroundjoin%
\pgfsetlinewidth{0.000000pt}%
\definecolor{currentstroke}{rgb}{0.000000,0.000000,0.000000}%
\pgfsetstrokecolor{currentstroke}%
\pgfsetdash{}{0pt}%
\pgfpathmoveto{\pgfqpoint{6.695237in}{0.667594in}}%
\pgfpathlineto{\pgfqpoint{6.881464in}{0.667594in}}%
\pgfpathlineto{\pgfqpoint{6.881464in}{0.749322in}}%
\pgfpathlineto{\pgfqpoint{6.695237in}{0.749322in}}%
\pgfpathlineto{\pgfqpoint{6.695237in}{0.667594in}}%
\pgfusepath{}%
\end{pgfscope}%
\begin{pgfscope}%
\pgfpathrectangle{\pgfqpoint{0.549740in}{0.463273in}}{\pgfqpoint{9.320225in}{4.495057in}}%
\pgfusepath{clip}%
\pgfsetbuttcap%
\pgfsetroundjoin%
\pgfsetlinewidth{0.000000pt}%
\definecolor{currentstroke}{rgb}{0.000000,0.000000,0.000000}%
\pgfsetstrokecolor{currentstroke}%
\pgfsetdash{}{0pt}%
\pgfpathmoveto{\pgfqpoint{6.881464in}{0.667594in}}%
\pgfpathlineto{\pgfqpoint{7.067690in}{0.667594in}}%
\pgfpathlineto{\pgfqpoint{7.067690in}{0.749322in}}%
\pgfpathlineto{\pgfqpoint{6.881464in}{0.749322in}}%
\pgfpathlineto{\pgfqpoint{6.881464in}{0.667594in}}%
\pgfusepath{}%
\end{pgfscope}%
\begin{pgfscope}%
\pgfpathrectangle{\pgfqpoint{0.549740in}{0.463273in}}{\pgfqpoint{9.320225in}{4.495057in}}%
\pgfusepath{clip}%
\pgfsetbuttcap%
\pgfsetroundjoin%
\pgfsetlinewidth{0.000000pt}%
\definecolor{currentstroke}{rgb}{0.000000,0.000000,0.000000}%
\pgfsetstrokecolor{currentstroke}%
\pgfsetdash{}{0pt}%
\pgfpathmoveto{\pgfqpoint{7.067690in}{0.667594in}}%
\pgfpathlineto{\pgfqpoint{7.253917in}{0.667594in}}%
\pgfpathlineto{\pgfqpoint{7.253917in}{0.749322in}}%
\pgfpathlineto{\pgfqpoint{7.067690in}{0.749322in}}%
\pgfpathlineto{\pgfqpoint{7.067690in}{0.667594in}}%
\pgfusepath{}%
\end{pgfscope}%
\begin{pgfscope}%
\pgfpathrectangle{\pgfqpoint{0.549740in}{0.463273in}}{\pgfqpoint{9.320225in}{4.495057in}}%
\pgfusepath{clip}%
\pgfsetbuttcap%
\pgfsetroundjoin%
\pgfsetlinewidth{0.000000pt}%
\definecolor{currentstroke}{rgb}{0.000000,0.000000,0.000000}%
\pgfsetstrokecolor{currentstroke}%
\pgfsetdash{}{0pt}%
\pgfpathmoveto{\pgfqpoint{7.253917in}{0.667594in}}%
\pgfpathlineto{\pgfqpoint{7.440143in}{0.667594in}}%
\pgfpathlineto{\pgfqpoint{7.440143in}{0.749322in}}%
\pgfpathlineto{\pgfqpoint{7.253917in}{0.749322in}}%
\pgfpathlineto{\pgfqpoint{7.253917in}{0.667594in}}%
\pgfusepath{}%
\end{pgfscope}%
\begin{pgfscope}%
\pgfpathrectangle{\pgfqpoint{0.549740in}{0.463273in}}{\pgfqpoint{9.320225in}{4.495057in}}%
\pgfusepath{clip}%
\pgfsetbuttcap%
\pgfsetroundjoin%
\pgfsetlinewidth{0.000000pt}%
\definecolor{currentstroke}{rgb}{0.000000,0.000000,0.000000}%
\pgfsetstrokecolor{currentstroke}%
\pgfsetdash{}{0pt}%
\pgfpathmoveto{\pgfqpoint{7.440143in}{0.667594in}}%
\pgfpathlineto{\pgfqpoint{7.626370in}{0.667594in}}%
\pgfpathlineto{\pgfqpoint{7.626370in}{0.749322in}}%
\pgfpathlineto{\pgfqpoint{7.440143in}{0.749322in}}%
\pgfpathlineto{\pgfqpoint{7.440143in}{0.667594in}}%
\pgfusepath{}%
\end{pgfscope}%
\begin{pgfscope}%
\pgfpathrectangle{\pgfqpoint{0.549740in}{0.463273in}}{\pgfqpoint{9.320225in}{4.495057in}}%
\pgfusepath{clip}%
\pgfsetbuttcap%
\pgfsetroundjoin%
\pgfsetlinewidth{0.000000pt}%
\definecolor{currentstroke}{rgb}{0.000000,0.000000,0.000000}%
\pgfsetstrokecolor{currentstroke}%
\pgfsetdash{}{0pt}%
\pgfpathmoveto{\pgfqpoint{7.626370in}{0.667594in}}%
\pgfpathlineto{\pgfqpoint{7.812596in}{0.667594in}}%
\pgfpathlineto{\pgfqpoint{7.812596in}{0.749322in}}%
\pgfpathlineto{\pgfqpoint{7.626370in}{0.749322in}}%
\pgfpathlineto{\pgfqpoint{7.626370in}{0.667594in}}%
\pgfusepath{}%
\end{pgfscope}%
\begin{pgfscope}%
\pgfpathrectangle{\pgfqpoint{0.549740in}{0.463273in}}{\pgfqpoint{9.320225in}{4.495057in}}%
\pgfusepath{clip}%
\pgfsetbuttcap%
\pgfsetroundjoin%
\pgfsetlinewidth{0.000000pt}%
\definecolor{currentstroke}{rgb}{0.000000,0.000000,0.000000}%
\pgfsetstrokecolor{currentstroke}%
\pgfsetdash{}{0pt}%
\pgfpathmoveto{\pgfqpoint{7.812596in}{0.667594in}}%
\pgfpathlineto{\pgfqpoint{7.998823in}{0.667594in}}%
\pgfpathlineto{\pgfqpoint{7.998823in}{0.749322in}}%
\pgfpathlineto{\pgfqpoint{7.812596in}{0.749322in}}%
\pgfpathlineto{\pgfqpoint{7.812596in}{0.667594in}}%
\pgfusepath{}%
\end{pgfscope}%
\begin{pgfscope}%
\pgfpathrectangle{\pgfqpoint{0.549740in}{0.463273in}}{\pgfqpoint{9.320225in}{4.495057in}}%
\pgfusepath{clip}%
\pgfsetbuttcap%
\pgfsetroundjoin%
\pgfsetlinewidth{0.000000pt}%
\definecolor{currentstroke}{rgb}{0.000000,0.000000,0.000000}%
\pgfsetstrokecolor{currentstroke}%
\pgfsetdash{}{0pt}%
\pgfpathmoveto{\pgfqpoint{7.998823in}{0.667594in}}%
\pgfpathlineto{\pgfqpoint{8.185049in}{0.667594in}}%
\pgfpathlineto{\pgfqpoint{8.185049in}{0.749322in}}%
\pgfpathlineto{\pgfqpoint{7.998823in}{0.749322in}}%
\pgfpathlineto{\pgfqpoint{7.998823in}{0.667594in}}%
\pgfusepath{}%
\end{pgfscope}%
\begin{pgfscope}%
\pgfpathrectangle{\pgfqpoint{0.549740in}{0.463273in}}{\pgfqpoint{9.320225in}{4.495057in}}%
\pgfusepath{clip}%
\pgfsetbuttcap%
\pgfsetroundjoin%
\pgfsetlinewidth{0.000000pt}%
\definecolor{currentstroke}{rgb}{0.000000,0.000000,0.000000}%
\pgfsetstrokecolor{currentstroke}%
\pgfsetdash{}{0pt}%
\pgfpathmoveto{\pgfqpoint{8.185049in}{0.667594in}}%
\pgfpathlineto{\pgfqpoint{8.371276in}{0.667594in}}%
\pgfpathlineto{\pgfqpoint{8.371276in}{0.749322in}}%
\pgfpathlineto{\pgfqpoint{8.185049in}{0.749322in}}%
\pgfpathlineto{\pgfqpoint{8.185049in}{0.667594in}}%
\pgfusepath{}%
\end{pgfscope}%
\begin{pgfscope}%
\pgfpathrectangle{\pgfqpoint{0.549740in}{0.463273in}}{\pgfqpoint{9.320225in}{4.495057in}}%
\pgfusepath{clip}%
\pgfsetbuttcap%
\pgfsetroundjoin%
\pgfsetlinewidth{0.000000pt}%
\definecolor{currentstroke}{rgb}{0.000000,0.000000,0.000000}%
\pgfsetstrokecolor{currentstroke}%
\pgfsetdash{}{0pt}%
\pgfpathmoveto{\pgfqpoint{8.371276in}{0.667594in}}%
\pgfpathlineto{\pgfqpoint{8.557503in}{0.667594in}}%
\pgfpathlineto{\pgfqpoint{8.557503in}{0.749322in}}%
\pgfpathlineto{\pgfqpoint{8.371276in}{0.749322in}}%
\pgfpathlineto{\pgfqpoint{8.371276in}{0.667594in}}%
\pgfusepath{}%
\end{pgfscope}%
\begin{pgfscope}%
\pgfpathrectangle{\pgfqpoint{0.549740in}{0.463273in}}{\pgfqpoint{9.320225in}{4.495057in}}%
\pgfusepath{clip}%
\pgfsetbuttcap%
\pgfsetroundjoin%
\pgfsetlinewidth{0.000000pt}%
\definecolor{currentstroke}{rgb}{0.000000,0.000000,0.000000}%
\pgfsetstrokecolor{currentstroke}%
\pgfsetdash{}{0pt}%
\pgfpathmoveto{\pgfqpoint{8.557503in}{0.667594in}}%
\pgfpathlineto{\pgfqpoint{8.743729in}{0.667594in}}%
\pgfpathlineto{\pgfqpoint{8.743729in}{0.749322in}}%
\pgfpathlineto{\pgfqpoint{8.557503in}{0.749322in}}%
\pgfpathlineto{\pgfqpoint{8.557503in}{0.667594in}}%
\pgfusepath{}%
\end{pgfscope}%
\begin{pgfscope}%
\pgfpathrectangle{\pgfqpoint{0.549740in}{0.463273in}}{\pgfqpoint{9.320225in}{4.495057in}}%
\pgfusepath{clip}%
\pgfsetbuttcap%
\pgfsetroundjoin%
\pgfsetlinewidth{0.000000pt}%
\definecolor{currentstroke}{rgb}{0.000000,0.000000,0.000000}%
\pgfsetstrokecolor{currentstroke}%
\pgfsetdash{}{0pt}%
\pgfpathmoveto{\pgfqpoint{8.743729in}{0.667594in}}%
\pgfpathlineto{\pgfqpoint{8.929956in}{0.667594in}}%
\pgfpathlineto{\pgfqpoint{8.929956in}{0.749322in}}%
\pgfpathlineto{\pgfqpoint{8.743729in}{0.749322in}}%
\pgfpathlineto{\pgfqpoint{8.743729in}{0.667594in}}%
\pgfusepath{}%
\end{pgfscope}%
\begin{pgfscope}%
\pgfpathrectangle{\pgfqpoint{0.549740in}{0.463273in}}{\pgfqpoint{9.320225in}{4.495057in}}%
\pgfusepath{clip}%
\pgfsetbuttcap%
\pgfsetroundjoin%
\pgfsetlinewidth{0.000000pt}%
\definecolor{currentstroke}{rgb}{0.000000,0.000000,0.000000}%
\pgfsetstrokecolor{currentstroke}%
\pgfsetdash{}{0pt}%
\pgfpathmoveto{\pgfqpoint{8.929956in}{0.667594in}}%
\pgfpathlineto{\pgfqpoint{9.116182in}{0.667594in}}%
\pgfpathlineto{\pgfqpoint{9.116182in}{0.749322in}}%
\pgfpathlineto{\pgfqpoint{8.929956in}{0.749322in}}%
\pgfpathlineto{\pgfqpoint{8.929956in}{0.667594in}}%
\pgfusepath{}%
\end{pgfscope}%
\begin{pgfscope}%
\pgfpathrectangle{\pgfqpoint{0.549740in}{0.463273in}}{\pgfqpoint{9.320225in}{4.495057in}}%
\pgfusepath{clip}%
\pgfsetbuttcap%
\pgfsetroundjoin%
\pgfsetlinewidth{0.000000pt}%
\definecolor{currentstroke}{rgb}{0.000000,0.000000,0.000000}%
\pgfsetstrokecolor{currentstroke}%
\pgfsetdash{}{0pt}%
\pgfpathmoveto{\pgfqpoint{9.116182in}{0.667594in}}%
\pgfpathlineto{\pgfqpoint{9.302409in}{0.667594in}}%
\pgfpathlineto{\pgfqpoint{9.302409in}{0.749322in}}%
\pgfpathlineto{\pgfqpoint{9.116182in}{0.749322in}}%
\pgfpathlineto{\pgfqpoint{9.116182in}{0.667594in}}%
\pgfusepath{}%
\end{pgfscope}%
\begin{pgfscope}%
\pgfpathrectangle{\pgfqpoint{0.549740in}{0.463273in}}{\pgfqpoint{9.320225in}{4.495057in}}%
\pgfusepath{clip}%
\pgfsetbuttcap%
\pgfsetroundjoin%
\pgfsetlinewidth{0.000000pt}%
\definecolor{currentstroke}{rgb}{0.000000,0.000000,0.000000}%
\pgfsetstrokecolor{currentstroke}%
\pgfsetdash{}{0pt}%
\pgfpathmoveto{\pgfqpoint{9.302409in}{0.667594in}}%
\pgfpathlineto{\pgfqpoint{9.488635in}{0.667594in}}%
\pgfpathlineto{\pgfqpoint{9.488635in}{0.749322in}}%
\pgfpathlineto{\pgfqpoint{9.302409in}{0.749322in}}%
\pgfpathlineto{\pgfqpoint{9.302409in}{0.667594in}}%
\pgfusepath{}%
\end{pgfscope}%
\begin{pgfscope}%
\pgfpathrectangle{\pgfqpoint{0.549740in}{0.463273in}}{\pgfqpoint{9.320225in}{4.495057in}}%
\pgfusepath{clip}%
\pgfsetbuttcap%
\pgfsetroundjoin%
\pgfsetlinewidth{0.000000pt}%
\definecolor{currentstroke}{rgb}{0.000000,0.000000,0.000000}%
\pgfsetstrokecolor{currentstroke}%
\pgfsetdash{}{0pt}%
\pgfpathmoveto{\pgfqpoint{9.488635in}{0.667594in}}%
\pgfpathlineto{\pgfqpoint{9.674862in}{0.667594in}}%
\pgfpathlineto{\pgfqpoint{9.674862in}{0.749322in}}%
\pgfpathlineto{\pgfqpoint{9.488635in}{0.749322in}}%
\pgfpathlineto{\pgfqpoint{9.488635in}{0.667594in}}%
\pgfusepath{}%
\end{pgfscope}%
\begin{pgfscope}%
\pgfpathrectangle{\pgfqpoint{0.549740in}{0.463273in}}{\pgfqpoint{9.320225in}{4.495057in}}%
\pgfusepath{clip}%
\pgfsetbuttcap%
\pgfsetroundjoin%
\pgfsetlinewidth{0.000000pt}%
\definecolor{currentstroke}{rgb}{0.000000,0.000000,0.000000}%
\pgfsetstrokecolor{currentstroke}%
\pgfsetdash{}{0pt}%
\pgfpathmoveto{\pgfqpoint{9.674862in}{0.667594in}}%
\pgfpathlineto{\pgfqpoint{9.861088in}{0.667594in}}%
\pgfpathlineto{\pgfqpoint{9.861088in}{0.749322in}}%
\pgfpathlineto{\pgfqpoint{9.674862in}{0.749322in}}%
\pgfpathlineto{\pgfqpoint{9.674862in}{0.667594in}}%
\pgfusepath{}%
\end{pgfscope}%
\begin{pgfscope}%
\pgfpathrectangle{\pgfqpoint{0.549740in}{0.463273in}}{\pgfqpoint{9.320225in}{4.495057in}}%
\pgfusepath{clip}%
\pgfsetbuttcap%
\pgfsetroundjoin%
\definecolor{currentfill}{rgb}{0.385185,0.686583,0.962589}%
\pgfsetfillcolor{currentfill}%
\pgfsetlinewidth{0.000000pt}%
\definecolor{currentstroke}{rgb}{0.000000,0.000000,0.000000}%
\pgfsetstrokecolor{currentstroke}%
\pgfsetdash{}{0pt}%
\pgfpathmoveto{\pgfqpoint{0.549761in}{0.749322in}}%
\pgfpathlineto{\pgfqpoint{0.735988in}{0.749322in}}%
\pgfpathlineto{\pgfqpoint{0.735988in}{0.831051in}}%
\pgfpathlineto{\pgfqpoint{0.549761in}{0.831051in}}%
\pgfpathlineto{\pgfqpoint{0.549761in}{0.749322in}}%
\pgfusepath{fill}%
\end{pgfscope}%
\begin{pgfscope}%
\pgfpathrectangle{\pgfqpoint{0.549740in}{0.463273in}}{\pgfqpoint{9.320225in}{4.495057in}}%
\pgfusepath{clip}%
\pgfsetbuttcap%
\pgfsetroundjoin%
\definecolor{currentfill}{rgb}{0.614330,0.761948,0.940009}%
\pgfsetfillcolor{currentfill}%
\pgfsetlinewidth{0.000000pt}%
\definecolor{currentstroke}{rgb}{0.000000,0.000000,0.000000}%
\pgfsetstrokecolor{currentstroke}%
\pgfsetdash{}{0pt}%
\pgfpathmoveto{\pgfqpoint{0.735988in}{0.749322in}}%
\pgfpathlineto{\pgfqpoint{0.922214in}{0.749322in}}%
\pgfpathlineto{\pgfqpoint{0.922214in}{0.831051in}}%
\pgfpathlineto{\pgfqpoint{0.735988in}{0.831051in}}%
\pgfpathlineto{\pgfqpoint{0.735988in}{0.749322in}}%
\pgfusepath{fill}%
\end{pgfscope}%
\begin{pgfscope}%
\pgfpathrectangle{\pgfqpoint{0.549740in}{0.463273in}}{\pgfqpoint{9.320225in}{4.495057in}}%
\pgfusepath{clip}%
\pgfsetbuttcap%
\pgfsetroundjoin%
\pgfsetlinewidth{0.000000pt}%
\definecolor{currentstroke}{rgb}{0.000000,0.000000,0.000000}%
\pgfsetstrokecolor{currentstroke}%
\pgfsetdash{}{0pt}%
\pgfpathmoveto{\pgfqpoint{0.922214in}{0.749322in}}%
\pgfpathlineto{\pgfqpoint{1.108441in}{0.749322in}}%
\pgfpathlineto{\pgfqpoint{1.108441in}{0.831051in}}%
\pgfpathlineto{\pgfqpoint{0.922214in}{0.831051in}}%
\pgfpathlineto{\pgfqpoint{0.922214in}{0.749322in}}%
\pgfusepath{}%
\end{pgfscope}%
\begin{pgfscope}%
\pgfpathrectangle{\pgfqpoint{0.549740in}{0.463273in}}{\pgfqpoint{9.320225in}{4.495057in}}%
\pgfusepath{clip}%
\pgfsetbuttcap%
\pgfsetroundjoin%
\definecolor{currentfill}{rgb}{0.547810,0.736432,0.947518}%
\pgfsetfillcolor{currentfill}%
\pgfsetlinewidth{0.000000pt}%
\definecolor{currentstroke}{rgb}{0.000000,0.000000,0.000000}%
\pgfsetstrokecolor{currentstroke}%
\pgfsetdash{}{0pt}%
\pgfpathmoveto{\pgfqpoint{1.108441in}{0.749322in}}%
\pgfpathlineto{\pgfqpoint{1.294667in}{0.749322in}}%
\pgfpathlineto{\pgfqpoint{1.294667in}{0.831051in}}%
\pgfpathlineto{\pgfqpoint{1.108441in}{0.831051in}}%
\pgfpathlineto{\pgfqpoint{1.108441in}{0.749322in}}%
\pgfusepath{fill}%
\end{pgfscope}%
\begin{pgfscope}%
\pgfpathrectangle{\pgfqpoint{0.549740in}{0.463273in}}{\pgfqpoint{9.320225in}{4.495057in}}%
\pgfusepath{clip}%
\pgfsetbuttcap%
\pgfsetroundjoin%
\definecolor{currentfill}{rgb}{0.169712,0.607583,0.911334}%
\pgfsetfillcolor{currentfill}%
\pgfsetlinewidth{0.000000pt}%
\definecolor{currentstroke}{rgb}{0.000000,0.000000,0.000000}%
\pgfsetstrokecolor{currentstroke}%
\pgfsetdash{}{0pt}%
\pgfpathmoveto{\pgfqpoint{1.294667in}{0.749322in}}%
\pgfpathlineto{\pgfqpoint{1.480894in}{0.749322in}}%
\pgfpathlineto{\pgfqpoint{1.480894in}{0.831051in}}%
\pgfpathlineto{\pgfqpoint{1.294667in}{0.831051in}}%
\pgfpathlineto{\pgfqpoint{1.294667in}{0.749322in}}%
\pgfusepath{fill}%
\end{pgfscope}%
\begin{pgfscope}%
\pgfpathrectangle{\pgfqpoint{0.549740in}{0.463273in}}{\pgfqpoint{9.320225in}{4.495057in}}%
\pgfusepath{clip}%
\pgfsetbuttcap%
\pgfsetroundjoin%
\definecolor{currentfill}{rgb}{0.194981,0.494518,0.729769}%
\pgfsetfillcolor{currentfill}%
\pgfsetlinewidth{0.000000pt}%
\definecolor{currentstroke}{rgb}{0.000000,0.000000,0.000000}%
\pgfsetstrokecolor{currentstroke}%
\pgfsetdash{}{0pt}%
\pgfpathmoveto{\pgfqpoint{1.480894in}{0.749322in}}%
\pgfpathlineto{\pgfqpoint{1.667120in}{0.749322in}}%
\pgfpathlineto{\pgfqpoint{1.667120in}{0.831051in}}%
\pgfpathlineto{\pgfqpoint{1.480894in}{0.831051in}}%
\pgfpathlineto{\pgfqpoint{1.480894in}{0.749322in}}%
\pgfusepath{fill}%
\end{pgfscope}%
\begin{pgfscope}%
\pgfpathrectangle{\pgfqpoint{0.549740in}{0.463273in}}{\pgfqpoint{9.320225in}{4.495057in}}%
\pgfusepath{clip}%
\pgfsetbuttcap%
\pgfsetroundjoin%
\definecolor{currentfill}{rgb}{0.614330,0.761948,0.940009}%
\pgfsetfillcolor{currentfill}%
\pgfsetlinewidth{0.000000pt}%
\definecolor{currentstroke}{rgb}{0.000000,0.000000,0.000000}%
\pgfsetstrokecolor{currentstroke}%
\pgfsetdash{}{0pt}%
\pgfpathmoveto{\pgfqpoint{1.667120in}{0.749322in}}%
\pgfpathlineto{\pgfqpoint{1.853347in}{0.749322in}}%
\pgfpathlineto{\pgfqpoint{1.853347in}{0.831051in}}%
\pgfpathlineto{\pgfqpoint{1.667120in}{0.831051in}}%
\pgfpathlineto{\pgfqpoint{1.667120in}{0.749322in}}%
\pgfusepath{fill}%
\end{pgfscope}%
\begin{pgfscope}%
\pgfpathrectangle{\pgfqpoint{0.549740in}{0.463273in}}{\pgfqpoint{9.320225in}{4.495057in}}%
\pgfusepath{clip}%
\pgfsetbuttcap%
\pgfsetroundjoin%
\definecolor{currentfill}{rgb}{0.273225,0.662144,0.968515}%
\pgfsetfillcolor{currentfill}%
\pgfsetlinewidth{0.000000pt}%
\definecolor{currentstroke}{rgb}{0.000000,0.000000,0.000000}%
\pgfsetstrokecolor{currentstroke}%
\pgfsetdash{}{0pt}%
\pgfpathmoveto{\pgfqpoint{1.853347in}{0.749322in}}%
\pgfpathlineto{\pgfqpoint{2.039573in}{0.749322in}}%
\pgfpathlineto{\pgfqpoint{2.039573in}{0.831051in}}%
\pgfpathlineto{\pgfqpoint{1.853347in}{0.831051in}}%
\pgfpathlineto{\pgfqpoint{1.853347in}{0.749322in}}%
\pgfusepath{fill}%
\end{pgfscope}%
\begin{pgfscope}%
\pgfpathrectangle{\pgfqpoint{0.549740in}{0.463273in}}{\pgfqpoint{9.320225in}{4.495057in}}%
\pgfusepath{clip}%
\pgfsetbuttcap%
\pgfsetroundjoin%
\definecolor{currentfill}{rgb}{0.273225,0.662144,0.968515}%
\pgfsetfillcolor{currentfill}%
\pgfsetlinewidth{0.000000pt}%
\definecolor{currentstroke}{rgb}{0.000000,0.000000,0.000000}%
\pgfsetstrokecolor{currentstroke}%
\pgfsetdash{}{0pt}%
\pgfpathmoveto{\pgfqpoint{2.039573in}{0.749322in}}%
\pgfpathlineto{\pgfqpoint{2.225800in}{0.749322in}}%
\pgfpathlineto{\pgfqpoint{2.225800in}{0.831051in}}%
\pgfpathlineto{\pgfqpoint{2.039573in}{0.831051in}}%
\pgfpathlineto{\pgfqpoint{2.039573in}{0.749322in}}%
\pgfusepath{fill}%
\end{pgfscope}%
\begin{pgfscope}%
\pgfpathrectangle{\pgfqpoint{0.549740in}{0.463273in}}{\pgfqpoint{9.320225in}{4.495057in}}%
\pgfusepath{clip}%
\pgfsetbuttcap%
\pgfsetroundjoin%
\definecolor{currentfill}{rgb}{0.189527,0.635753,0.950228}%
\pgfsetfillcolor{currentfill}%
\pgfsetlinewidth{0.000000pt}%
\definecolor{currentstroke}{rgb}{0.000000,0.000000,0.000000}%
\pgfsetstrokecolor{currentstroke}%
\pgfsetdash{}{0pt}%
\pgfpathmoveto{\pgfqpoint{2.225800in}{0.749322in}}%
\pgfpathlineto{\pgfqpoint{2.412027in}{0.749322in}}%
\pgfpathlineto{\pgfqpoint{2.412027in}{0.831051in}}%
\pgfpathlineto{\pgfqpoint{2.225800in}{0.831051in}}%
\pgfpathlineto{\pgfqpoint{2.225800in}{0.749322in}}%
\pgfusepath{fill}%
\end{pgfscope}%
\begin{pgfscope}%
\pgfpathrectangle{\pgfqpoint{0.549740in}{0.463273in}}{\pgfqpoint{9.320225in}{4.495057in}}%
\pgfusepath{clip}%
\pgfsetbuttcap%
\pgfsetroundjoin%
\definecolor{currentfill}{rgb}{0.221438,0.438563,0.630024}%
\pgfsetfillcolor{currentfill}%
\pgfsetlinewidth{0.000000pt}%
\definecolor{currentstroke}{rgb}{0.000000,0.000000,0.000000}%
\pgfsetstrokecolor{currentstroke}%
\pgfsetdash{}{0pt}%
\pgfpathmoveto{\pgfqpoint{2.412027in}{0.749322in}}%
\pgfpathlineto{\pgfqpoint{2.598253in}{0.749322in}}%
\pgfpathlineto{\pgfqpoint{2.598253in}{0.831051in}}%
\pgfpathlineto{\pgfqpoint{2.412027in}{0.831051in}}%
\pgfpathlineto{\pgfqpoint{2.412027in}{0.749322in}}%
\pgfusepath{fill}%
\end{pgfscope}%
\begin{pgfscope}%
\pgfpathrectangle{\pgfqpoint{0.549740in}{0.463273in}}{\pgfqpoint{9.320225in}{4.495057in}}%
\pgfusepath{clip}%
\pgfsetbuttcap%
\pgfsetroundjoin%
\definecolor{currentfill}{rgb}{0.221438,0.438563,0.630024}%
\pgfsetfillcolor{currentfill}%
\pgfsetlinewidth{0.000000pt}%
\definecolor{currentstroke}{rgb}{0.000000,0.000000,0.000000}%
\pgfsetstrokecolor{currentstroke}%
\pgfsetdash{}{0pt}%
\pgfpathmoveto{\pgfqpoint{2.598253in}{0.749322in}}%
\pgfpathlineto{\pgfqpoint{2.784480in}{0.749322in}}%
\pgfpathlineto{\pgfqpoint{2.784480in}{0.831051in}}%
\pgfpathlineto{\pgfqpoint{2.598253in}{0.831051in}}%
\pgfpathlineto{\pgfqpoint{2.598253in}{0.749322in}}%
\pgfusepath{fill}%
\end{pgfscope}%
\begin{pgfscope}%
\pgfpathrectangle{\pgfqpoint{0.549740in}{0.463273in}}{\pgfqpoint{9.320225in}{4.495057in}}%
\pgfusepath{clip}%
\pgfsetbuttcap%
\pgfsetroundjoin%
\definecolor{currentfill}{rgb}{0.240633,0.330170,0.431120}%
\pgfsetfillcolor{currentfill}%
\pgfsetlinewidth{0.000000pt}%
\definecolor{currentstroke}{rgb}{0.000000,0.000000,0.000000}%
\pgfsetstrokecolor{currentstroke}%
\pgfsetdash{}{0pt}%
\pgfpathmoveto{\pgfqpoint{2.784480in}{0.749322in}}%
\pgfpathlineto{\pgfqpoint{2.970706in}{0.749322in}}%
\pgfpathlineto{\pgfqpoint{2.970706in}{0.831051in}}%
\pgfpathlineto{\pgfqpoint{2.784480in}{0.831051in}}%
\pgfpathlineto{\pgfqpoint{2.784480in}{0.749322in}}%
\pgfusepath{fill}%
\end{pgfscope}%
\begin{pgfscope}%
\pgfpathrectangle{\pgfqpoint{0.549740in}{0.463273in}}{\pgfqpoint{9.320225in}{4.495057in}}%
\pgfusepath{clip}%
\pgfsetbuttcap%
\pgfsetroundjoin%
\definecolor{currentfill}{rgb}{0.240633,0.330170,0.431120}%
\pgfsetfillcolor{currentfill}%
\pgfsetlinewidth{0.000000pt}%
\definecolor{currentstroke}{rgb}{0.000000,0.000000,0.000000}%
\pgfsetstrokecolor{currentstroke}%
\pgfsetdash{}{0pt}%
\pgfpathmoveto{\pgfqpoint{2.970706in}{0.749322in}}%
\pgfpathlineto{\pgfqpoint{3.156933in}{0.749322in}}%
\pgfpathlineto{\pgfqpoint{3.156933in}{0.831051in}}%
\pgfpathlineto{\pgfqpoint{2.970706in}{0.831051in}}%
\pgfpathlineto{\pgfqpoint{2.970706in}{0.749322in}}%
\pgfusepath{fill}%
\end{pgfscope}%
\begin{pgfscope}%
\pgfpathrectangle{\pgfqpoint{0.549740in}{0.463273in}}{\pgfqpoint{9.320225in}{4.495057in}}%
\pgfusepath{clip}%
\pgfsetbuttcap%
\pgfsetroundjoin%
\definecolor{currentfill}{rgb}{0.221438,0.438563,0.630024}%
\pgfsetfillcolor{currentfill}%
\pgfsetlinewidth{0.000000pt}%
\definecolor{currentstroke}{rgb}{0.000000,0.000000,0.000000}%
\pgfsetstrokecolor{currentstroke}%
\pgfsetdash{}{0pt}%
\pgfpathmoveto{\pgfqpoint{3.156933in}{0.749322in}}%
\pgfpathlineto{\pgfqpoint{3.343159in}{0.749322in}}%
\pgfpathlineto{\pgfqpoint{3.343159in}{0.831051in}}%
\pgfpathlineto{\pgfqpoint{3.156933in}{0.831051in}}%
\pgfpathlineto{\pgfqpoint{3.156933in}{0.749322in}}%
\pgfusepath{fill}%
\end{pgfscope}%
\begin{pgfscope}%
\pgfpathrectangle{\pgfqpoint{0.549740in}{0.463273in}}{\pgfqpoint{9.320225in}{4.495057in}}%
\pgfusepath{clip}%
\pgfsetbuttcap%
\pgfsetroundjoin%
\definecolor{currentfill}{rgb}{0.237426,0.383637,0.529350}%
\pgfsetfillcolor{currentfill}%
\pgfsetlinewidth{0.000000pt}%
\definecolor{currentstroke}{rgb}{0.000000,0.000000,0.000000}%
\pgfsetstrokecolor{currentstroke}%
\pgfsetdash{}{0pt}%
\pgfpathmoveto{\pgfqpoint{3.343159in}{0.749322in}}%
\pgfpathlineto{\pgfqpoint{3.529386in}{0.749322in}}%
\pgfpathlineto{\pgfqpoint{3.529386in}{0.831051in}}%
\pgfpathlineto{\pgfqpoint{3.343159in}{0.831051in}}%
\pgfpathlineto{\pgfqpoint{3.343159in}{0.749322in}}%
\pgfusepath{fill}%
\end{pgfscope}%
\begin{pgfscope}%
\pgfpathrectangle{\pgfqpoint{0.549740in}{0.463273in}}{\pgfqpoint{9.320225in}{4.495057in}}%
\pgfusepath{clip}%
\pgfsetbuttcap%
\pgfsetroundjoin%
\definecolor{currentfill}{rgb}{0.168741,0.551020,0.824827}%
\pgfsetfillcolor{currentfill}%
\pgfsetlinewidth{0.000000pt}%
\definecolor{currentstroke}{rgb}{0.000000,0.000000,0.000000}%
\pgfsetstrokecolor{currentstroke}%
\pgfsetdash{}{0pt}%
\pgfpathmoveto{\pgfqpoint{3.529386in}{0.749322in}}%
\pgfpathlineto{\pgfqpoint{3.715612in}{0.749322in}}%
\pgfpathlineto{\pgfqpoint{3.715612in}{0.831051in}}%
\pgfpathlineto{\pgfqpoint{3.529386in}{0.831051in}}%
\pgfpathlineto{\pgfqpoint{3.529386in}{0.749322in}}%
\pgfusepath{fill}%
\end{pgfscope}%
\begin{pgfscope}%
\pgfpathrectangle{\pgfqpoint{0.549740in}{0.463273in}}{\pgfqpoint{9.320225in}{4.495057in}}%
\pgfusepath{clip}%
\pgfsetbuttcap%
\pgfsetroundjoin%
\definecolor{currentfill}{rgb}{0.168741,0.551020,0.824827}%
\pgfsetfillcolor{currentfill}%
\pgfsetlinewidth{0.000000pt}%
\definecolor{currentstroke}{rgb}{0.000000,0.000000,0.000000}%
\pgfsetstrokecolor{currentstroke}%
\pgfsetdash{}{0pt}%
\pgfpathmoveto{\pgfqpoint{3.715612in}{0.749322in}}%
\pgfpathlineto{\pgfqpoint{3.901839in}{0.749322in}}%
\pgfpathlineto{\pgfqpoint{3.901839in}{0.831051in}}%
\pgfpathlineto{\pgfqpoint{3.715612in}{0.831051in}}%
\pgfpathlineto{\pgfqpoint{3.715612in}{0.749322in}}%
\pgfusepath{fill}%
\end{pgfscope}%
\begin{pgfscope}%
\pgfpathrectangle{\pgfqpoint{0.549740in}{0.463273in}}{\pgfqpoint{9.320225in}{4.495057in}}%
\pgfusepath{clip}%
\pgfsetbuttcap%
\pgfsetroundjoin%
\definecolor{currentfill}{rgb}{0.221438,0.438563,0.630024}%
\pgfsetfillcolor{currentfill}%
\pgfsetlinewidth{0.000000pt}%
\definecolor{currentstroke}{rgb}{0.000000,0.000000,0.000000}%
\pgfsetstrokecolor{currentstroke}%
\pgfsetdash{}{0pt}%
\pgfpathmoveto{\pgfqpoint{3.901839in}{0.749322in}}%
\pgfpathlineto{\pgfqpoint{4.088065in}{0.749322in}}%
\pgfpathlineto{\pgfqpoint{4.088065in}{0.831051in}}%
\pgfpathlineto{\pgfqpoint{3.901839in}{0.831051in}}%
\pgfpathlineto{\pgfqpoint{3.901839in}{0.749322in}}%
\pgfusepath{fill}%
\end{pgfscope}%
\begin{pgfscope}%
\pgfpathrectangle{\pgfqpoint{0.549740in}{0.463273in}}{\pgfqpoint{9.320225in}{4.495057in}}%
\pgfusepath{clip}%
\pgfsetbuttcap%
\pgfsetroundjoin%
\definecolor{currentfill}{rgb}{0.194981,0.494518,0.729769}%
\pgfsetfillcolor{currentfill}%
\pgfsetlinewidth{0.000000pt}%
\definecolor{currentstroke}{rgb}{0.000000,0.000000,0.000000}%
\pgfsetstrokecolor{currentstroke}%
\pgfsetdash{}{0pt}%
\pgfpathmoveto{\pgfqpoint{4.088065in}{0.749322in}}%
\pgfpathlineto{\pgfqpoint{4.274292in}{0.749322in}}%
\pgfpathlineto{\pgfqpoint{4.274292in}{0.831051in}}%
\pgfpathlineto{\pgfqpoint{4.088065in}{0.831051in}}%
\pgfpathlineto{\pgfqpoint{4.088065in}{0.749322in}}%
\pgfusepath{fill}%
\end{pgfscope}%
\begin{pgfscope}%
\pgfpathrectangle{\pgfqpoint{0.549740in}{0.463273in}}{\pgfqpoint{9.320225in}{4.495057in}}%
\pgfusepath{clip}%
\pgfsetbuttcap%
\pgfsetroundjoin%
\definecolor{currentfill}{rgb}{0.221438,0.438563,0.630024}%
\pgfsetfillcolor{currentfill}%
\pgfsetlinewidth{0.000000pt}%
\definecolor{currentstroke}{rgb}{0.000000,0.000000,0.000000}%
\pgfsetstrokecolor{currentstroke}%
\pgfsetdash{}{0pt}%
\pgfpathmoveto{\pgfqpoint{4.274292in}{0.749322in}}%
\pgfpathlineto{\pgfqpoint{4.460519in}{0.749322in}}%
\pgfpathlineto{\pgfqpoint{4.460519in}{0.831051in}}%
\pgfpathlineto{\pgfqpoint{4.274292in}{0.831051in}}%
\pgfpathlineto{\pgfqpoint{4.274292in}{0.749322in}}%
\pgfusepath{fill}%
\end{pgfscope}%
\begin{pgfscope}%
\pgfpathrectangle{\pgfqpoint{0.549740in}{0.463273in}}{\pgfqpoint{9.320225in}{4.495057in}}%
\pgfusepath{clip}%
\pgfsetbuttcap%
\pgfsetroundjoin%
\definecolor{currentfill}{rgb}{0.209148,0.466441,0.680241}%
\pgfsetfillcolor{currentfill}%
\pgfsetlinewidth{0.000000pt}%
\definecolor{currentstroke}{rgb}{0.000000,0.000000,0.000000}%
\pgfsetstrokecolor{currentstroke}%
\pgfsetdash{}{0pt}%
\pgfpathmoveto{\pgfqpoint{4.460519in}{0.749322in}}%
\pgfpathlineto{\pgfqpoint{4.646745in}{0.749322in}}%
\pgfpathlineto{\pgfqpoint{4.646745in}{0.831051in}}%
\pgfpathlineto{\pgfqpoint{4.460519in}{0.831051in}}%
\pgfpathlineto{\pgfqpoint{4.460519in}{0.749322in}}%
\pgfusepath{fill}%
\end{pgfscope}%
\begin{pgfscope}%
\pgfpathrectangle{\pgfqpoint{0.549740in}{0.463273in}}{\pgfqpoint{9.320225in}{4.495057in}}%
\pgfusepath{clip}%
\pgfsetbuttcap%
\pgfsetroundjoin%
\definecolor{currentfill}{rgb}{0.221438,0.438563,0.630024}%
\pgfsetfillcolor{currentfill}%
\pgfsetlinewidth{0.000000pt}%
\definecolor{currentstroke}{rgb}{0.000000,0.000000,0.000000}%
\pgfsetstrokecolor{currentstroke}%
\pgfsetdash{}{0pt}%
\pgfpathmoveto{\pgfqpoint{4.646745in}{0.749322in}}%
\pgfpathlineto{\pgfqpoint{4.832972in}{0.749322in}}%
\pgfpathlineto{\pgfqpoint{4.832972in}{0.831051in}}%
\pgfpathlineto{\pgfqpoint{4.646745in}{0.831051in}}%
\pgfpathlineto{\pgfqpoint{4.646745in}{0.749322in}}%
\pgfusepath{fill}%
\end{pgfscope}%
\begin{pgfscope}%
\pgfpathrectangle{\pgfqpoint{0.549740in}{0.463273in}}{\pgfqpoint{9.320225in}{4.495057in}}%
\pgfusepath{clip}%
\pgfsetbuttcap%
\pgfsetroundjoin%
\definecolor{currentfill}{rgb}{0.230994,0.410942,0.579580}%
\pgfsetfillcolor{currentfill}%
\pgfsetlinewidth{0.000000pt}%
\definecolor{currentstroke}{rgb}{0.000000,0.000000,0.000000}%
\pgfsetstrokecolor{currentstroke}%
\pgfsetdash{}{0pt}%
\pgfpathmoveto{\pgfqpoint{4.832972in}{0.749322in}}%
\pgfpathlineto{\pgfqpoint{5.019198in}{0.749322in}}%
\pgfpathlineto{\pgfqpoint{5.019198in}{0.831051in}}%
\pgfpathlineto{\pgfqpoint{4.832972in}{0.831051in}}%
\pgfpathlineto{\pgfqpoint{4.832972in}{0.749322in}}%
\pgfusepath{fill}%
\end{pgfscope}%
\begin{pgfscope}%
\pgfpathrectangle{\pgfqpoint{0.549740in}{0.463273in}}{\pgfqpoint{9.320225in}{4.495057in}}%
\pgfusepath{clip}%
\pgfsetbuttcap%
\pgfsetroundjoin%
\definecolor{currentfill}{rgb}{0.237426,0.383637,0.529350}%
\pgfsetfillcolor{currentfill}%
\pgfsetlinewidth{0.000000pt}%
\definecolor{currentstroke}{rgb}{0.000000,0.000000,0.000000}%
\pgfsetstrokecolor{currentstroke}%
\pgfsetdash{}{0pt}%
\pgfpathmoveto{\pgfqpoint{5.019198in}{0.749322in}}%
\pgfpathlineto{\pgfqpoint{5.205425in}{0.749322in}}%
\pgfpathlineto{\pgfqpoint{5.205425in}{0.831051in}}%
\pgfpathlineto{\pgfqpoint{5.019198in}{0.831051in}}%
\pgfpathlineto{\pgfqpoint{5.019198in}{0.749322in}}%
\pgfusepath{fill}%
\end{pgfscope}%
\begin{pgfscope}%
\pgfpathrectangle{\pgfqpoint{0.549740in}{0.463273in}}{\pgfqpoint{9.320225in}{4.495057in}}%
\pgfusepath{clip}%
\pgfsetbuttcap%
\pgfsetroundjoin%
\pgfsetlinewidth{0.000000pt}%
\definecolor{currentstroke}{rgb}{0.000000,0.000000,0.000000}%
\pgfsetstrokecolor{currentstroke}%
\pgfsetdash{}{0pt}%
\pgfpathmoveto{\pgfqpoint{5.205425in}{0.749322in}}%
\pgfpathlineto{\pgfqpoint{5.391651in}{0.749322in}}%
\pgfpathlineto{\pgfqpoint{5.391651in}{0.831051in}}%
\pgfpathlineto{\pgfqpoint{5.205425in}{0.831051in}}%
\pgfpathlineto{\pgfqpoint{5.205425in}{0.749322in}}%
\pgfusepath{}%
\end{pgfscope}%
\begin{pgfscope}%
\pgfpathrectangle{\pgfqpoint{0.549740in}{0.463273in}}{\pgfqpoint{9.320225in}{4.495057in}}%
\pgfusepath{clip}%
\pgfsetbuttcap%
\pgfsetroundjoin%
\pgfsetlinewidth{0.000000pt}%
\definecolor{currentstroke}{rgb}{0.000000,0.000000,0.000000}%
\pgfsetstrokecolor{currentstroke}%
\pgfsetdash{}{0pt}%
\pgfpathmoveto{\pgfqpoint{5.391651in}{0.749322in}}%
\pgfpathlineto{\pgfqpoint{5.577878in}{0.749322in}}%
\pgfpathlineto{\pgfqpoint{5.577878in}{0.831051in}}%
\pgfpathlineto{\pgfqpoint{5.391651in}{0.831051in}}%
\pgfpathlineto{\pgfqpoint{5.391651in}{0.749322in}}%
\pgfusepath{}%
\end{pgfscope}%
\begin{pgfscope}%
\pgfpathrectangle{\pgfqpoint{0.549740in}{0.463273in}}{\pgfqpoint{9.320225in}{4.495057in}}%
\pgfusepath{clip}%
\pgfsetbuttcap%
\pgfsetroundjoin%
\pgfsetlinewidth{0.000000pt}%
\definecolor{currentstroke}{rgb}{0.000000,0.000000,0.000000}%
\pgfsetstrokecolor{currentstroke}%
\pgfsetdash{}{0pt}%
\pgfpathmoveto{\pgfqpoint{5.577878in}{0.749322in}}%
\pgfpathlineto{\pgfqpoint{5.764104in}{0.749322in}}%
\pgfpathlineto{\pgfqpoint{5.764104in}{0.831051in}}%
\pgfpathlineto{\pgfqpoint{5.577878in}{0.831051in}}%
\pgfpathlineto{\pgfqpoint{5.577878in}{0.749322in}}%
\pgfusepath{}%
\end{pgfscope}%
\begin{pgfscope}%
\pgfpathrectangle{\pgfqpoint{0.549740in}{0.463273in}}{\pgfqpoint{9.320225in}{4.495057in}}%
\pgfusepath{clip}%
\pgfsetbuttcap%
\pgfsetroundjoin%
\pgfsetlinewidth{0.000000pt}%
\definecolor{currentstroke}{rgb}{0.000000,0.000000,0.000000}%
\pgfsetstrokecolor{currentstroke}%
\pgfsetdash{}{0pt}%
\pgfpathmoveto{\pgfqpoint{5.764104in}{0.749322in}}%
\pgfpathlineto{\pgfqpoint{5.950331in}{0.749322in}}%
\pgfpathlineto{\pgfqpoint{5.950331in}{0.831051in}}%
\pgfpathlineto{\pgfqpoint{5.764104in}{0.831051in}}%
\pgfpathlineto{\pgfqpoint{5.764104in}{0.749322in}}%
\pgfusepath{}%
\end{pgfscope}%
\begin{pgfscope}%
\pgfpathrectangle{\pgfqpoint{0.549740in}{0.463273in}}{\pgfqpoint{9.320225in}{4.495057in}}%
\pgfusepath{clip}%
\pgfsetbuttcap%
\pgfsetroundjoin%
\pgfsetlinewidth{0.000000pt}%
\definecolor{currentstroke}{rgb}{0.000000,0.000000,0.000000}%
\pgfsetstrokecolor{currentstroke}%
\pgfsetdash{}{0pt}%
\pgfpathmoveto{\pgfqpoint{5.950331in}{0.749322in}}%
\pgfpathlineto{\pgfqpoint{6.136557in}{0.749322in}}%
\pgfpathlineto{\pgfqpoint{6.136557in}{0.831051in}}%
\pgfpathlineto{\pgfqpoint{5.950331in}{0.831051in}}%
\pgfpathlineto{\pgfqpoint{5.950331in}{0.749322in}}%
\pgfusepath{}%
\end{pgfscope}%
\begin{pgfscope}%
\pgfpathrectangle{\pgfqpoint{0.549740in}{0.463273in}}{\pgfqpoint{9.320225in}{4.495057in}}%
\pgfusepath{clip}%
\pgfsetbuttcap%
\pgfsetroundjoin%
\pgfsetlinewidth{0.000000pt}%
\definecolor{currentstroke}{rgb}{0.000000,0.000000,0.000000}%
\pgfsetstrokecolor{currentstroke}%
\pgfsetdash{}{0pt}%
\pgfpathmoveto{\pgfqpoint{6.136557in}{0.749322in}}%
\pgfpathlineto{\pgfqpoint{6.322784in}{0.749322in}}%
\pgfpathlineto{\pgfqpoint{6.322784in}{0.831051in}}%
\pgfpathlineto{\pgfqpoint{6.136557in}{0.831051in}}%
\pgfpathlineto{\pgfqpoint{6.136557in}{0.749322in}}%
\pgfusepath{}%
\end{pgfscope}%
\begin{pgfscope}%
\pgfpathrectangle{\pgfqpoint{0.549740in}{0.463273in}}{\pgfqpoint{9.320225in}{4.495057in}}%
\pgfusepath{clip}%
\pgfsetbuttcap%
\pgfsetroundjoin%
\pgfsetlinewidth{0.000000pt}%
\definecolor{currentstroke}{rgb}{0.000000,0.000000,0.000000}%
\pgfsetstrokecolor{currentstroke}%
\pgfsetdash{}{0pt}%
\pgfpathmoveto{\pgfqpoint{6.322784in}{0.749322in}}%
\pgfpathlineto{\pgfqpoint{6.509011in}{0.749322in}}%
\pgfpathlineto{\pgfqpoint{6.509011in}{0.831051in}}%
\pgfpathlineto{\pgfqpoint{6.322784in}{0.831051in}}%
\pgfpathlineto{\pgfqpoint{6.322784in}{0.749322in}}%
\pgfusepath{}%
\end{pgfscope}%
\begin{pgfscope}%
\pgfpathrectangle{\pgfqpoint{0.549740in}{0.463273in}}{\pgfqpoint{9.320225in}{4.495057in}}%
\pgfusepath{clip}%
\pgfsetbuttcap%
\pgfsetroundjoin%
\pgfsetlinewidth{0.000000pt}%
\definecolor{currentstroke}{rgb}{0.000000,0.000000,0.000000}%
\pgfsetstrokecolor{currentstroke}%
\pgfsetdash{}{0pt}%
\pgfpathmoveto{\pgfqpoint{6.509011in}{0.749322in}}%
\pgfpathlineto{\pgfqpoint{6.695237in}{0.749322in}}%
\pgfpathlineto{\pgfqpoint{6.695237in}{0.831051in}}%
\pgfpathlineto{\pgfqpoint{6.509011in}{0.831051in}}%
\pgfpathlineto{\pgfqpoint{6.509011in}{0.749322in}}%
\pgfusepath{}%
\end{pgfscope}%
\begin{pgfscope}%
\pgfpathrectangle{\pgfqpoint{0.549740in}{0.463273in}}{\pgfqpoint{9.320225in}{4.495057in}}%
\pgfusepath{clip}%
\pgfsetbuttcap%
\pgfsetroundjoin%
\pgfsetlinewidth{0.000000pt}%
\definecolor{currentstroke}{rgb}{0.000000,0.000000,0.000000}%
\pgfsetstrokecolor{currentstroke}%
\pgfsetdash{}{0pt}%
\pgfpathmoveto{\pgfqpoint{6.695237in}{0.749322in}}%
\pgfpathlineto{\pgfqpoint{6.881464in}{0.749322in}}%
\pgfpathlineto{\pgfqpoint{6.881464in}{0.831051in}}%
\pgfpathlineto{\pgfqpoint{6.695237in}{0.831051in}}%
\pgfpathlineto{\pgfqpoint{6.695237in}{0.749322in}}%
\pgfusepath{}%
\end{pgfscope}%
\begin{pgfscope}%
\pgfpathrectangle{\pgfqpoint{0.549740in}{0.463273in}}{\pgfqpoint{9.320225in}{4.495057in}}%
\pgfusepath{clip}%
\pgfsetbuttcap%
\pgfsetroundjoin%
\pgfsetlinewidth{0.000000pt}%
\definecolor{currentstroke}{rgb}{0.000000,0.000000,0.000000}%
\pgfsetstrokecolor{currentstroke}%
\pgfsetdash{}{0pt}%
\pgfpathmoveto{\pgfqpoint{6.881464in}{0.749322in}}%
\pgfpathlineto{\pgfqpoint{7.067690in}{0.749322in}}%
\pgfpathlineto{\pgfqpoint{7.067690in}{0.831051in}}%
\pgfpathlineto{\pgfqpoint{6.881464in}{0.831051in}}%
\pgfpathlineto{\pgfqpoint{6.881464in}{0.749322in}}%
\pgfusepath{}%
\end{pgfscope}%
\begin{pgfscope}%
\pgfpathrectangle{\pgfqpoint{0.549740in}{0.463273in}}{\pgfqpoint{9.320225in}{4.495057in}}%
\pgfusepath{clip}%
\pgfsetbuttcap%
\pgfsetroundjoin%
\pgfsetlinewidth{0.000000pt}%
\definecolor{currentstroke}{rgb}{0.000000,0.000000,0.000000}%
\pgfsetstrokecolor{currentstroke}%
\pgfsetdash{}{0pt}%
\pgfpathmoveto{\pgfqpoint{7.067690in}{0.749322in}}%
\pgfpathlineto{\pgfqpoint{7.253917in}{0.749322in}}%
\pgfpathlineto{\pgfqpoint{7.253917in}{0.831051in}}%
\pgfpathlineto{\pgfqpoint{7.067690in}{0.831051in}}%
\pgfpathlineto{\pgfqpoint{7.067690in}{0.749322in}}%
\pgfusepath{}%
\end{pgfscope}%
\begin{pgfscope}%
\pgfpathrectangle{\pgfqpoint{0.549740in}{0.463273in}}{\pgfqpoint{9.320225in}{4.495057in}}%
\pgfusepath{clip}%
\pgfsetbuttcap%
\pgfsetroundjoin%
\pgfsetlinewidth{0.000000pt}%
\definecolor{currentstroke}{rgb}{0.000000,0.000000,0.000000}%
\pgfsetstrokecolor{currentstroke}%
\pgfsetdash{}{0pt}%
\pgfpathmoveto{\pgfqpoint{7.253917in}{0.749322in}}%
\pgfpathlineto{\pgfqpoint{7.440143in}{0.749322in}}%
\pgfpathlineto{\pgfqpoint{7.440143in}{0.831051in}}%
\pgfpathlineto{\pgfqpoint{7.253917in}{0.831051in}}%
\pgfpathlineto{\pgfqpoint{7.253917in}{0.749322in}}%
\pgfusepath{}%
\end{pgfscope}%
\begin{pgfscope}%
\pgfpathrectangle{\pgfqpoint{0.549740in}{0.463273in}}{\pgfqpoint{9.320225in}{4.495057in}}%
\pgfusepath{clip}%
\pgfsetbuttcap%
\pgfsetroundjoin%
\pgfsetlinewidth{0.000000pt}%
\definecolor{currentstroke}{rgb}{0.000000,0.000000,0.000000}%
\pgfsetstrokecolor{currentstroke}%
\pgfsetdash{}{0pt}%
\pgfpathmoveto{\pgfqpoint{7.440143in}{0.749322in}}%
\pgfpathlineto{\pgfqpoint{7.626370in}{0.749322in}}%
\pgfpathlineto{\pgfqpoint{7.626370in}{0.831051in}}%
\pgfpathlineto{\pgfqpoint{7.440143in}{0.831051in}}%
\pgfpathlineto{\pgfqpoint{7.440143in}{0.749322in}}%
\pgfusepath{}%
\end{pgfscope}%
\begin{pgfscope}%
\pgfpathrectangle{\pgfqpoint{0.549740in}{0.463273in}}{\pgfqpoint{9.320225in}{4.495057in}}%
\pgfusepath{clip}%
\pgfsetbuttcap%
\pgfsetroundjoin%
\pgfsetlinewidth{0.000000pt}%
\definecolor{currentstroke}{rgb}{0.000000,0.000000,0.000000}%
\pgfsetstrokecolor{currentstroke}%
\pgfsetdash{}{0pt}%
\pgfpathmoveto{\pgfqpoint{7.626370in}{0.749322in}}%
\pgfpathlineto{\pgfqpoint{7.812596in}{0.749322in}}%
\pgfpathlineto{\pgfqpoint{7.812596in}{0.831051in}}%
\pgfpathlineto{\pgfqpoint{7.626370in}{0.831051in}}%
\pgfpathlineto{\pgfqpoint{7.626370in}{0.749322in}}%
\pgfusepath{}%
\end{pgfscope}%
\begin{pgfscope}%
\pgfpathrectangle{\pgfqpoint{0.549740in}{0.463273in}}{\pgfqpoint{9.320225in}{4.495057in}}%
\pgfusepath{clip}%
\pgfsetbuttcap%
\pgfsetroundjoin%
\pgfsetlinewidth{0.000000pt}%
\definecolor{currentstroke}{rgb}{0.000000,0.000000,0.000000}%
\pgfsetstrokecolor{currentstroke}%
\pgfsetdash{}{0pt}%
\pgfpathmoveto{\pgfqpoint{7.812596in}{0.749322in}}%
\pgfpathlineto{\pgfqpoint{7.998823in}{0.749322in}}%
\pgfpathlineto{\pgfqpoint{7.998823in}{0.831051in}}%
\pgfpathlineto{\pgfqpoint{7.812596in}{0.831051in}}%
\pgfpathlineto{\pgfqpoint{7.812596in}{0.749322in}}%
\pgfusepath{}%
\end{pgfscope}%
\begin{pgfscope}%
\pgfpathrectangle{\pgfqpoint{0.549740in}{0.463273in}}{\pgfqpoint{9.320225in}{4.495057in}}%
\pgfusepath{clip}%
\pgfsetbuttcap%
\pgfsetroundjoin%
\pgfsetlinewidth{0.000000pt}%
\definecolor{currentstroke}{rgb}{0.000000,0.000000,0.000000}%
\pgfsetstrokecolor{currentstroke}%
\pgfsetdash{}{0pt}%
\pgfpathmoveto{\pgfqpoint{7.998823in}{0.749322in}}%
\pgfpathlineto{\pgfqpoint{8.185049in}{0.749322in}}%
\pgfpathlineto{\pgfqpoint{8.185049in}{0.831051in}}%
\pgfpathlineto{\pgfqpoint{7.998823in}{0.831051in}}%
\pgfpathlineto{\pgfqpoint{7.998823in}{0.749322in}}%
\pgfusepath{}%
\end{pgfscope}%
\begin{pgfscope}%
\pgfpathrectangle{\pgfqpoint{0.549740in}{0.463273in}}{\pgfqpoint{9.320225in}{4.495057in}}%
\pgfusepath{clip}%
\pgfsetbuttcap%
\pgfsetroundjoin%
\pgfsetlinewidth{0.000000pt}%
\definecolor{currentstroke}{rgb}{0.000000,0.000000,0.000000}%
\pgfsetstrokecolor{currentstroke}%
\pgfsetdash{}{0pt}%
\pgfpathmoveto{\pgfqpoint{8.185049in}{0.749322in}}%
\pgfpathlineto{\pgfqpoint{8.371276in}{0.749322in}}%
\pgfpathlineto{\pgfqpoint{8.371276in}{0.831051in}}%
\pgfpathlineto{\pgfqpoint{8.185049in}{0.831051in}}%
\pgfpathlineto{\pgfqpoint{8.185049in}{0.749322in}}%
\pgfusepath{}%
\end{pgfscope}%
\begin{pgfscope}%
\pgfpathrectangle{\pgfqpoint{0.549740in}{0.463273in}}{\pgfqpoint{9.320225in}{4.495057in}}%
\pgfusepath{clip}%
\pgfsetbuttcap%
\pgfsetroundjoin%
\pgfsetlinewidth{0.000000pt}%
\definecolor{currentstroke}{rgb}{0.000000,0.000000,0.000000}%
\pgfsetstrokecolor{currentstroke}%
\pgfsetdash{}{0pt}%
\pgfpathmoveto{\pgfqpoint{8.371276in}{0.749322in}}%
\pgfpathlineto{\pgfqpoint{8.557503in}{0.749322in}}%
\pgfpathlineto{\pgfqpoint{8.557503in}{0.831051in}}%
\pgfpathlineto{\pgfqpoint{8.371276in}{0.831051in}}%
\pgfpathlineto{\pgfqpoint{8.371276in}{0.749322in}}%
\pgfusepath{}%
\end{pgfscope}%
\begin{pgfscope}%
\pgfpathrectangle{\pgfqpoint{0.549740in}{0.463273in}}{\pgfqpoint{9.320225in}{4.495057in}}%
\pgfusepath{clip}%
\pgfsetbuttcap%
\pgfsetroundjoin%
\pgfsetlinewidth{0.000000pt}%
\definecolor{currentstroke}{rgb}{0.000000,0.000000,0.000000}%
\pgfsetstrokecolor{currentstroke}%
\pgfsetdash{}{0pt}%
\pgfpathmoveto{\pgfqpoint{8.557503in}{0.749322in}}%
\pgfpathlineto{\pgfqpoint{8.743729in}{0.749322in}}%
\pgfpathlineto{\pgfqpoint{8.743729in}{0.831051in}}%
\pgfpathlineto{\pgfqpoint{8.557503in}{0.831051in}}%
\pgfpathlineto{\pgfqpoint{8.557503in}{0.749322in}}%
\pgfusepath{}%
\end{pgfscope}%
\begin{pgfscope}%
\pgfpathrectangle{\pgfqpoint{0.549740in}{0.463273in}}{\pgfqpoint{9.320225in}{4.495057in}}%
\pgfusepath{clip}%
\pgfsetbuttcap%
\pgfsetroundjoin%
\pgfsetlinewidth{0.000000pt}%
\definecolor{currentstroke}{rgb}{0.000000,0.000000,0.000000}%
\pgfsetstrokecolor{currentstroke}%
\pgfsetdash{}{0pt}%
\pgfpathmoveto{\pgfqpoint{8.743729in}{0.749322in}}%
\pgfpathlineto{\pgfqpoint{8.929956in}{0.749322in}}%
\pgfpathlineto{\pgfqpoint{8.929956in}{0.831051in}}%
\pgfpathlineto{\pgfqpoint{8.743729in}{0.831051in}}%
\pgfpathlineto{\pgfqpoint{8.743729in}{0.749322in}}%
\pgfusepath{}%
\end{pgfscope}%
\begin{pgfscope}%
\pgfpathrectangle{\pgfqpoint{0.549740in}{0.463273in}}{\pgfqpoint{9.320225in}{4.495057in}}%
\pgfusepath{clip}%
\pgfsetbuttcap%
\pgfsetroundjoin%
\pgfsetlinewidth{0.000000pt}%
\definecolor{currentstroke}{rgb}{0.000000,0.000000,0.000000}%
\pgfsetstrokecolor{currentstroke}%
\pgfsetdash{}{0pt}%
\pgfpathmoveto{\pgfqpoint{8.929956in}{0.749322in}}%
\pgfpathlineto{\pgfqpoint{9.116182in}{0.749322in}}%
\pgfpathlineto{\pgfqpoint{9.116182in}{0.831051in}}%
\pgfpathlineto{\pgfqpoint{8.929956in}{0.831051in}}%
\pgfpathlineto{\pgfqpoint{8.929956in}{0.749322in}}%
\pgfusepath{}%
\end{pgfscope}%
\begin{pgfscope}%
\pgfpathrectangle{\pgfqpoint{0.549740in}{0.463273in}}{\pgfqpoint{9.320225in}{4.495057in}}%
\pgfusepath{clip}%
\pgfsetbuttcap%
\pgfsetroundjoin%
\pgfsetlinewidth{0.000000pt}%
\definecolor{currentstroke}{rgb}{0.000000,0.000000,0.000000}%
\pgfsetstrokecolor{currentstroke}%
\pgfsetdash{}{0pt}%
\pgfpathmoveto{\pgfqpoint{9.116182in}{0.749322in}}%
\pgfpathlineto{\pgfqpoint{9.302409in}{0.749322in}}%
\pgfpathlineto{\pgfqpoint{9.302409in}{0.831051in}}%
\pgfpathlineto{\pgfqpoint{9.116182in}{0.831051in}}%
\pgfpathlineto{\pgfqpoint{9.116182in}{0.749322in}}%
\pgfusepath{}%
\end{pgfscope}%
\begin{pgfscope}%
\pgfpathrectangle{\pgfqpoint{0.549740in}{0.463273in}}{\pgfqpoint{9.320225in}{4.495057in}}%
\pgfusepath{clip}%
\pgfsetbuttcap%
\pgfsetroundjoin%
\pgfsetlinewidth{0.000000pt}%
\definecolor{currentstroke}{rgb}{0.000000,0.000000,0.000000}%
\pgfsetstrokecolor{currentstroke}%
\pgfsetdash{}{0pt}%
\pgfpathmoveto{\pgfqpoint{9.302409in}{0.749322in}}%
\pgfpathlineto{\pgfqpoint{9.488635in}{0.749322in}}%
\pgfpathlineto{\pgfqpoint{9.488635in}{0.831051in}}%
\pgfpathlineto{\pgfqpoint{9.302409in}{0.831051in}}%
\pgfpathlineto{\pgfqpoint{9.302409in}{0.749322in}}%
\pgfusepath{}%
\end{pgfscope}%
\begin{pgfscope}%
\pgfpathrectangle{\pgfqpoint{0.549740in}{0.463273in}}{\pgfqpoint{9.320225in}{4.495057in}}%
\pgfusepath{clip}%
\pgfsetbuttcap%
\pgfsetroundjoin%
\pgfsetlinewidth{0.000000pt}%
\definecolor{currentstroke}{rgb}{0.000000,0.000000,0.000000}%
\pgfsetstrokecolor{currentstroke}%
\pgfsetdash{}{0pt}%
\pgfpathmoveto{\pgfqpoint{9.488635in}{0.749322in}}%
\pgfpathlineto{\pgfqpoint{9.674862in}{0.749322in}}%
\pgfpathlineto{\pgfqpoint{9.674862in}{0.831051in}}%
\pgfpathlineto{\pgfqpoint{9.488635in}{0.831051in}}%
\pgfpathlineto{\pgfqpoint{9.488635in}{0.749322in}}%
\pgfusepath{}%
\end{pgfscope}%
\begin{pgfscope}%
\pgfpathrectangle{\pgfqpoint{0.549740in}{0.463273in}}{\pgfqpoint{9.320225in}{4.495057in}}%
\pgfusepath{clip}%
\pgfsetbuttcap%
\pgfsetroundjoin%
\pgfsetlinewidth{0.000000pt}%
\definecolor{currentstroke}{rgb}{0.000000,0.000000,0.000000}%
\pgfsetstrokecolor{currentstroke}%
\pgfsetdash{}{0pt}%
\pgfpathmoveto{\pgfqpoint{9.674862in}{0.749322in}}%
\pgfpathlineto{\pgfqpoint{9.861088in}{0.749322in}}%
\pgfpathlineto{\pgfqpoint{9.861088in}{0.831051in}}%
\pgfpathlineto{\pgfqpoint{9.674862in}{0.831051in}}%
\pgfpathlineto{\pgfqpoint{9.674862in}{0.749322in}}%
\pgfusepath{}%
\end{pgfscope}%
\begin{pgfscope}%
\pgfpathrectangle{\pgfqpoint{0.549740in}{0.463273in}}{\pgfqpoint{9.320225in}{4.495057in}}%
\pgfusepath{clip}%
\pgfsetbuttcap%
\pgfsetroundjoin%
\definecolor{currentfill}{rgb}{0.273225,0.662144,0.968515}%
\pgfsetfillcolor{currentfill}%
\pgfsetlinewidth{0.000000pt}%
\definecolor{currentstroke}{rgb}{0.000000,0.000000,0.000000}%
\pgfsetstrokecolor{currentstroke}%
\pgfsetdash{}{0pt}%
\pgfpathmoveto{\pgfqpoint{0.549761in}{0.831051in}}%
\pgfpathlineto{\pgfqpoint{0.735988in}{0.831051in}}%
\pgfpathlineto{\pgfqpoint{0.735988in}{0.912779in}}%
\pgfpathlineto{\pgfqpoint{0.549761in}{0.912779in}}%
\pgfpathlineto{\pgfqpoint{0.549761in}{0.831051in}}%
\pgfusepath{fill}%
\end{pgfscope}%
\begin{pgfscope}%
\pgfpathrectangle{\pgfqpoint{0.549740in}{0.463273in}}{\pgfqpoint{9.320225in}{4.495057in}}%
\pgfusepath{clip}%
\pgfsetbuttcap%
\pgfsetroundjoin%
\definecolor{currentfill}{rgb}{0.273225,0.662144,0.968515}%
\pgfsetfillcolor{currentfill}%
\pgfsetlinewidth{0.000000pt}%
\definecolor{currentstroke}{rgb}{0.000000,0.000000,0.000000}%
\pgfsetstrokecolor{currentstroke}%
\pgfsetdash{}{0pt}%
\pgfpathmoveto{\pgfqpoint{0.735988in}{0.831051in}}%
\pgfpathlineto{\pgfqpoint{0.922214in}{0.831051in}}%
\pgfpathlineto{\pgfqpoint{0.922214in}{0.912779in}}%
\pgfpathlineto{\pgfqpoint{0.735988in}{0.912779in}}%
\pgfpathlineto{\pgfqpoint{0.735988in}{0.831051in}}%
\pgfusepath{fill}%
\end{pgfscope}%
\begin{pgfscope}%
\pgfpathrectangle{\pgfqpoint{0.549740in}{0.463273in}}{\pgfqpoint{9.320225in}{4.495057in}}%
\pgfusepath{clip}%
\pgfsetbuttcap%
\pgfsetroundjoin%
\definecolor{currentfill}{rgb}{0.163625,0.579322,0.869386}%
\pgfsetfillcolor{currentfill}%
\pgfsetlinewidth{0.000000pt}%
\definecolor{currentstroke}{rgb}{0.000000,0.000000,0.000000}%
\pgfsetstrokecolor{currentstroke}%
\pgfsetdash{}{0pt}%
\pgfpathmoveto{\pgfqpoint{0.922214in}{0.831051in}}%
\pgfpathlineto{\pgfqpoint{1.108441in}{0.831051in}}%
\pgfpathlineto{\pgfqpoint{1.108441in}{0.912779in}}%
\pgfpathlineto{\pgfqpoint{0.922214in}{0.912779in}}%
\pgfpathlineto{\pgfqpoint{0.922214in}{0.831051in}}%
\pgfusepath{fill}%
\end{pgfscope}%
\begin{pgfscope}%
\pgfpathrectangle{\pgfqpoint{0.549740in}{0.463273in}}{\pgfqpoint{9.320225in}{4.495057in}}%
\pgfusepath{clip}%
\pgfsetbuttcap%
\pgfsetroundjoin%
\definecolor{currentfill}{rgb}{0.163625,0.579322,0.869386}%
\pgfsetfillcolor{currentfill}%
\pgfsetlinewidth{0.000000pt}%
\definecolor{currentstroke}{rgb}{0.000000,0.000000,0.000000}%
\pgfsetstrokecolor{currentstroke}%
\pgfsetdash{}{0pt}%
\pgfpathmoveto{\pgfqpoint{1.108441in}{0.831051in}}%
\pgfpathlineto{\pgfqpoint{1.294667in}{0.831051in}}%
\pgfpathlineto{\pgfqpoint{1.294667in}{0.912779in}}%
\pgfpathlineto{\pgfqpoint{1.108441in}{0.912779in}}%
\pgfpathlineto{\pgfqpoint{1.108441in}{0.831051in}}%
\pgfusepath{fill}%
\end{pgfscope}%
\begin{pgfscope}%
\pgfpathrectangle{\pgfqpoint{0.549740in}{0.463273in}}{\pgfqpoint{9.320225in}{4.495057in}}%
\pgfusepath{clip}%
\pgfsetbuttcap%
\pgfsetroundjoin%
\definecolor{currentfill}{rgb}{0.194981,0.494518,0.729769}%
\pgfsetfillcolor{currentfill}%
\pgfsetlinewidth{0.000000pt}%
\definecolor{currentstroke}{rgb}{0.000000,0.000000,0.000000}%
\pgfsetstrokecolor{currentstroke}%
\pgfsetdash{}{0pt}%
\pgfpathmoveto{\pgfqpoint{1.294667in}{0.831051in}}%
\pgfpathlineto{\pgfqpoint{1.480894in}{0.831051in}}%
\pgfpathlineto{\pgfqpoint{1.480894in}{0.912779in}}%
\pgfpathlineto{\pgfqpoint{1.294667in}{0.912779in}}%
\pgfpathlineto{\pgfqpoint{1.294667in}{0.831051in}}%
\pgfusepath{fill}%
\end{pgfscope}%
\begin{pgfscope}%
\pgfpathrectangle{\pgfqpoint{0.549740in}{0.463273in}}{\pgfqpoint{9.320225in}{4.495057in}}%
\pgfusepath{clip}%
\pgfsetbuttcap%
\pgfsetroundjoin%
\definecolor{currentfill}{rgb}{0.168741,0.551020,0.824827}%
\pgfsetfillcolor{currentfill}%
\pgfsetlinewidth{0.000000pt}%
\definecolor{currentstroke}{rgb}{0.000000,0.000000,0.000000}%
\pgfsetstrokecolor{currentstroke}%
\pgfsetdash{}{0pt}%
\pgfpathmoveto{\pgfqpoint{1.480894in}{0.831051in}}%
\pgfpathlineto{\pgfqpoint{1.667120in}{0.831051in}}%
\pgfpathlineto{\pgfqpoint{1.667120in}{0.912779in}}%
\pgfpathlineto{\pgfqpoint{1.480894in}{0.912779in}}%
\pgfpathlineto{\pgfqpoint{1.480894in}{0.831051in}}%
\pgfusepath{fill}%
\end{pgfscope}%
\begin{pgfscope}%
\pgfpathrectangle{\pgfqpoint{0.549740in}{0.463273in}}{\pgfqpoint{9.320225in}{4.495057in}}%
\pgfusepath{clip}%
\pgfsetbuttcap%
\pgfsetroundjoin%
\definecolor{currentfill}{rgb}{0.168741,0.551020,0.824827}%
\pgfsetfillcolor{currentfill}%
\pgfsetlinewidth{0.000000pt}%
\definecolor{currentstroke}{rgb}{0.000000,0.000000,0.000000}%
\pgfsetstrokecolor{currentstroke}%
\pgfsetdash{}{0pt}%
\pgfpathmoveto{\pgfqpoint{1.667120in}{0.831051in}}%
\pgfpathlineto{\pgfqpoint{1.853347in}{0.831051in}}%
\pgfpathlineto{\pgfqpoint{1.853347in}{0.912779in}}%
\pgfpathlineto{\pgfqpoint{1.667120in}{0.912779in}}%
\pgfpathlineto{\pgfqpoint{1.667120in}{0.831051in}}%
\pgfusepath{fill}%
\end{pgfscope}%
\begin{pgfscope}%
\pgfpathrectangle{\pgfqpoint{0.549740in}{0.463273in}}{\pgfqpoint{9.320225in}{4.495057in}}%
\pgfusepath{clip}%
\pgfsetbuttcap%
\pgfsetroundjoin%
\definecolor{currentfill}{rgb}{0.194981,0.494518,0.729769}%
\pgfsetfillcolor{currentfill}%
\pgfsetlinewidth{0.000000pt}%
\definecolor{currentstroke}{rgb}{0.000000,0.000000,0.000000}%
\pgfsetstrokecolor{currentstroke}%
\pgfsetdash{}{0pt}%
\pgfpathmoveto{\pgfqpoint{1.853347in}{0.831051in}}%
\pgfpathlineto{\pgfqpoint{2.039573in}{0.831051in}}%
\pgfpathlineto{\pgfqpoint{2.039573in}{0.912779in}}%
\pgfpathlineto{\pgfqpoint{1.853347in}{0.912779in}}%
\pgfpathlineto{\pgfqpoint{1.853347in}{0.831051in}}%
\pgfusepath{fill}%
\end{pgfscope}%
\begin{pgfscope}%
\pgfpathrectangle{\pgfqpoint{0.549740in}{0.463273in}}{\pgfqpoint{9.320225in}{4.495057in}}%
\pgfusepath{clip}%
\pgfsetbuttcap%
\pgfsetroundjoin%
\definecolor{currentfill}{rgb}{0.180573,0.522732,0.778127}%
\pgfsetfillcolor{currentfill}%
\pgfsetlinewidth{0.000000pt}%
\definecolor{currentstroke}{rgb}{0.000000,0.000000,0.000000}%
\pgfsetstrokecolor{currentstroke}%
\pgfsetdash{}{0pt}%
\pgfpathmoveto{\pgfqpoint{2.039573in}{0.831051in}}%
\pgfpathlineto{\pgfqpoint{2.225800in}{0.831051in}}%
\pgfpathlineto{\pgfqpoint{2.225800in}{0.912779in}}%
\pgfpathlineto{\pgfqpoint{2.039573in}{0.912779in}}%
\pgfpathlineto{\pgfqpoint{2.039573in}{0.831051in}}%
\pgfusepath{fill}%
\end{pgfscope}%
\begin{pgfscope}%
\pgfpathrectangle{\pgfqpoint{0.549740in}{0.463273in}}{\pgfqpoint{9.320225in}{4.495057in}}%
\pgfusepath{clip}%
\pgfsetbuttcap%
\pgfsetroundjoin%
\definecolor{currentfill}{rgb}{0.273225,0.662144,0.968515}%
\pgfsetfillcolor{currentfill}%
\pgfsetlinewidth{0.000000pt}%
\definecolor{currentstroke}{rgb}{0.000000,0.000000,0.000000}%
\pgfsetstrokecolor{currentstroke}%
\pgfsetdash{}{0pt}%
\pgfpathmoveto{\pgfqpoint{2.225800in}{0.831051in}}%
\pgfpathlineto{\pgfqpoint{2.412027in}{0.831051in}}%
\pgfpathlineto{\pgfqpoint{2.412027in}{0.912779in}}%
\pgfpathlineto{\pgfqpoint{2.225800in}{0.912779in}}%
\pgfpathlineto{\pgfqpoint{2.225800in}{0.831051in}}%
\pgfusepath{fill}%
\end{pgfscope}%
\begin{pgfscope}%
\pgfpathrectangle{\pgfqpoint{0.549740in}{0.463273in}}{\pgfqpoint{9.320225in}{4.495057in}}%
\pgfusepath{clip}%
\pgfsetbuttcap%
\pgfsetroundjoin%
\definecolor{currentfill}{rgb}{0.273225,0.662144,0.968515}%
\pgfsetfillcolor{currentfill}%
\pgfsetlinewidth{0.000000pt}%
\definecolor{currentstroke}{rgb}{0.000000,0.000000,0.000000}%
\pgfsetstrokecolor{currentstroke}%
\pgfsetdash{}{0pt}%
\pgfpathmoveto{\pgfqpoint{2.412027in}{0.831051in}}%
\pgfpathlineto{\pgfqpoint{2.598253in}{0.831051in}}%
\pgfpathlineto{\pgfqpoint{2.598253in}{0.912779in}}%
\pgfpathlineto{\pgfqpoint{2.412027in}{0.912779in}}%
\pgfpathlineto{\pgfqpoint{2.412027in}{0.831051in}}%
\pgfusepath{fill}%
\end{pgfscope}%
\begin{pgfscope}%
\pgfpathrectangle{\pgfqpoint{0.549740in}{0.463273in}}{\pgfqpoint{9.320225in}{4.495057in}}%
\pgfusepath{clip}%
\pgfsetbuttcap%
\pgfsetroundjoin%
\definecolor{currentfill}{rgb}{0.273225,0.662144,0.968515}%
\pgfsetfillcolor{currentfill}%
\pgfsetlinewidth{0.000000pt}%
\definecolor{currentstroke}{rgb}{0.000000,0.000000,0.000000}%
\pgfsetstrokecolor{currentstroke}%
\pgfsetdash{}{0pt}%
\pgfpathmoveto{\pgfqpoint{2.598253in}{0.831051in}}%
\pgfpathlineto{\pgfqpoint{2.784480in}{0.831051in}}%
\pgfpathlineto{\pgfqpoint{2.784480in}{0.912779in}}%
\pgfpathlineto{\pgfqpoint{2.598253in}{0.912779in}}%
\pgfpathlineto{\pgfqpoint{2.598253in}{0.831051in}}%
\pgfusepath{fill}%
\end{pgfscope}%
\begin{pgfscope}%
\pgfpathrectangle{\pgfqpoint{0.549740in}{0.463273in}}{\pgfqpoint{9.320225in}{4.495057in}}%
\pgfusepath{clip}%
\pgfsetbuttcap%
\pgfsetroundjoin%
\definecolor{currentfill}{rgb}{0.614330,0.761948,0.940009}%
\pgfsetfillcolor{currentfill}%
\pgfsetlinewidth{0.000000pt}%
\definecolor{currentstroke}{rgb}{0.000000,0.000000,0.000000}%
\pgfsetstrokecolor{currentstroke}%
\pgfsetdash{}{0pt}%
\pgfpathmoveto{\pgfqpoint{2.784480in}{0.831051in}}%
\pgfpathlineto{\pgfqpoint{2.970706in}{0.831051in}}%
\pgfpathlineto{\pgfqpoint{2.970706in}{0.912779in}}%
\pgfpathlineto{\pgfqpoint{2.784480in}{0.912779in}}%
\pgfpathlineto{\pgfqpoint{2.784480in}{0.831051in}}%
\pgfusepath{fill}%
\end{pgfscope}%
\begin{pgfscope}%
\pgfpathrectangle{\pgfqpoint{0.549740in}{0.463273in}}{\pgfqpoint{9.320225in}{4.495057in}}%
\pgfusepath{clip}%
\pgfsetbuttcap%
\pgfsetroundjoin%
\pgfsetlinewidth{0.000000pt}%
\definecolor{currentstroke}{rgb}{0.000000,0.000000,0.000000}%
\pgfsetstrokecolor{currentstroke}%
\pgfsetdash{}{0pt}%
\pgfpathmoveto{\pgfqpoint{2.970706in}{0.831051in}}%
\pgfpathlineto{\pgfqpoint{3.156933in}{0.831051in}}%
\pgfpathlineto{\pgfqpoint{3.156933in}{0.912779in}}%
\pgfpathlineto{\pgfqpoint{2.970706in}{0.912779in}}%
\pgfpathlineto{\pgfqpoint{2.970706in}{0.831051in}}%
\pgfusepath{}%
\end{pgfscope}%
\begin{pgfscope}%
\pgfpathrectangle{\pgfqpoint{0.549740in}{0.463273in}}{\pgfqpoint{9.320225in}{4.495057in}}%
\pgfusepath{clip}%
\pgfsetbuttcap%
\pgfsetroundjoin%
\definecolor{currentfill}{rgb}{0.547810,0.736432,0.947518}%
\pgfsetfillcolor{currentfill}%
\pgfsetlinewidth{0.000000pt}%
\definecolor{currentstroke}{rgb}{0.000000,0.000000,0.000000}%
\pgfsetstrokecolor{currentstroke}%
\pgfsetdash{}{0pt}%
\pgfpathmoveto{\pgfqpoint{3.156933in}{0.831051in}}%
\pgfpathlineto{\pgfqpoint{3.343159in}{0.831051in}}%
\pgfpathlineto{\pgfqpoint{3.343159in}{0.912779in}}%
\pgfpathlineto{\pgfqpoint{3.156933in}{0.912779in}}%
\pgfpathlineto{\pgfqpoint{3.156933in}{0.831051in}}%
\pgfusepath{fill}%
\end{pgfscope}%
\begin{pgfscope}%
\pgfpathrectangle{\pgfqpoint{0.549740in}{0.463273in}}{\pgfqpoint{9.320225in}{4.495057in}}%
\pgfusepath{clip}%
\pgfsetbuttcap%
\pgfsetroundjoin%
\definecolor{currentfill}{rgb}{0.472869,0.711325,0.955316}%
\pgfsetfillcolor{currentfill}%
\pgfsetlinewidth{0.000000pt}%
\definecolor{currentstroke}{rgb}{0.000000,0.000000,0.000000}%
\pgfsetstrokecolor{currentstroke}%
\pgfsetdash{}{0pt}%
\pgfpathmoveto{\pgfqpoint{3.343159in}{0.831051in}}%
\pgfpathlineto{\pgfqpoint{3.529386in}{0.831051in}}%
\pgfpathlineto{\pgfqpoint{3.529386in}{0.912779in}}%
\pgfpathlineto{\pgfqpoint{3.343159in}{0.912779in}}%
\pgfpathlineto{\pgfqpoint{3.343159in}{0.831051in}}%
\pgfusepath{fill}%
\end{pgfscope}%
\begin{pgfscope}%
\pgfpathrectangle{\pgfqpoint{0.549740in}{0.463273in}}{\pgfqpoint{9.320225in}{4.495057in}}%
\pgfusepath{clip}%
\pgfsetbuttcap%
\pgfsetroundjoin%
\definecolor{currentfill}{rgb}{0.189527,0.635753,0.950228}%
\pgfsetfillcolor{currentfill}%
\pgfsetlinewidth{0.000000pt}%
\definecolor{currentstroke}{rgb}{0.000000,0.000000,0.000000}%
\pgfsetstrokecolor{currentstroke}%
\pgfsetdash{}{0pt}%
\pgfpathmoveto{\pgfqpoint{3.529386in}{0.831051in}}%
\pgfpathlineto{\pgfqpoint{3.715612in}{0.831051in}}%
\pgfpathlineto{\pgfqpoint{3.715612in}{0.912779in}}%
\pgfpathlineto{\pgfqpoint{3.529386in}{0.912779in}}%
\pgfpathlineto{\pgfqpoint{3.529386in}{0.831051in}}%
\pgfusepath{fill}%
\end{pgfscope}%
\begin{pgfscope}%
\pgfpathrectangle{\pgfqpoint{0.549740in}{0.463273in}}{\pgfqpoint{9.320225in}{4.495057in}}%
\pgfusepath{clip}%
\pgfsetbuttcap%
\pgfsetroundjoin%
\definecolor{currentfill}{rgb}{0.273225,0.662144,0.968515}%
\pgfsetfillcolor{currentfill}%
\pgfsetlinewidth{0.000000pt}%
\definecolor{currentstroke}{rgb}{0.000000,0.000000,0.000000}%
\pgfsetstrokecolor{currentstroke}%
\pgfsetdash{}{0pt}%
\pgfpathmoveto{\pgfqpoint{3.715612in}{0.831051in}}%
\pgfpathlineto{\pgfqpoint{3.901839in}{0.831051in}}%
\pgfpathlineto{\pgfqpoint{3.901839in}{0.912779in}}%
\pgfpathlineto{\pgfqpoint{3.715612in}{0.912779in}}%
\pgfpathlineto{\pgfqpoint{3.715612in}{0.831051in}}%
\pgfusepath{fill}%
\end{pgfscope}%
\begin{pgfscope}%
\pgfpathrectangle{\pgfqpoint{0.549740in}{0.463273in}}{\pgfqpoint{9.320225in}{4.495057in}}%
\pgfusepath{clip}%
\pgfsetbuttcap%
\pgfsetroundjoin%
\pgfsetlinewidth{0.000000pt}%
\definecolor{currentstroke}{rgb}{0.000000,0.000000,0.000000}%
\pgfsetstrokecolor{currentstroke}%
\pgfsetdash{}{0pt}%
\pgfpathmoveto{\pgfqpoint{3.901839in}{0.831051in}}%
\pgfpathlineto{\pgfqpoint{4.088065in}{0.831051in}}%
\pgfpathlineto{\pgfqpoint{4.088065in}{0.912779in}}%
\pgfpathlineto{\pgfqpoint{3.901839in}{0.912779in}}%
\pgfpathlineto{\pgfqpoint{3.901839in}{0.831051in}}%
\pgfusepath{}%
\end{pgfscope}%
\begin{pgfscope}%
\pgfpathrectangle{\pgfqpoint{0.549740in}{0.463273in}}{\pgfqpoint{9.320225in}{4.495057in}}%
\pgfusepath{clip}%
\pgfsetbuttcap%
\pgfsetroundjoin%
\definecolor{currentfill}{rgb}{0.273225,0.662144,0.968515}%
\pgfsetfillcolor{currentfill}%
\pgfsetlinewidth{0.000000pt}%
\definecolor{currentstroke}{rgb}{0.000000,0.000000,0.000000}%
\pgfsetstrokecolor{currentstroke}%
\pgfsetdash{}{0pt}%
\pgfpathmoveto{\pgfqpoint{4.088065in}{0.831051in}}%
\pgfpathlineto{\pgfqpoint{4.274292in}{0.831051in}}%
\pgfpathlineto{\pgfqpoint{4.274292in}{0.912779in}}%
\pgfpathlineto{\pgfqpoint{4.088065in}{0.912779in}}%
\pgfpathlineto{\pgfqpoint{4.088065in}{0.831051in}}%
\pgfusepath{fill}%
\end{pgfscope}%
\begin{pgfscope}%
\pgfpathrectangle{\pgfqpoint{0.549740in}{0.463273in}}{\pgfqpoint{9.320225in}{4.495057in}}%
\pgfusepath{clip}%
\pgfsetbuttcap%
\pgfsetroundjoin%
\pgfsetlinewidth{0.000000pt}%
\definecolor{currentstroke}{rgb}{0.000000,0.000000,0.000000}%
\pgfsetstrokecolor{currentstroke}%
\pgfsetdash{}{0pt}%
\pgfpathmoveto{\pgfqpoint{4.274292in}{0.831051in}}%
\pgfpathlineto{\pgfqpoint{4.460519in}{0.831051in}}%
\pgfpathlineto{\pgfqpoint{4.460519in}{0.912779in}}%
\pgfpathlineto{\pgfqpoint{4.274292in}{0.912779in}}%
\pgfpathlineto{\pgfqpoint{4.274292in}{0.831051in}}%
\pgfusepath{}%
\end{pgfscope}%
\begin{pgfscope}%
\pgfpathrectangle{\pgfqpoint{0.549740in}{0.463273in}}{\pgfqpoint{9.320225in}{4.495057in}}%
\pgfusepath{clip}%
\pgfsetbuttcap%
\pgfsetroundjoin%
\pgfsetlinewidth{0.000000pt}%
\definecolor{currentstroke}{rgb}{0.000000,0.000000,0.000000}%
\pgfsetstrokecolor{currentstroke}%
\pgfsetdash{}{0pt}%
\pgfpathmoveto{\pgfqpoint{4.460519in}{0.831051in}}%
\pgfpathlineto{\pgfqpoint{4.646745in}{0.831051in}}%
\pgfpathlineto{\pgfqpoint{4.646745in}{0.912779in}}%
\pgfpathlineto{\pgfqpoint{4.460519in}{0.912779in}}%
\pgfpathlineto{\pgfqpoint{4.460519in}{0.831051in}}%
\pgfusepath{}%
\end{pgfscope}%
\begin{pgfscope}%
\pgfpathrectangle{\pgfqpoint{0.549740in}{0.463273in}}{\pgfqpoint{9.320225in}{4.495057in}}%
\pgfusepath{clip}%
\pgfsetbuttcap%
\pgfsetroundjoin%
\definecolor{currentfill}{rgb}{0.547810,0.736432,0.947518}%
\pgfsetfillcolor{currentfill}%
\pgfsetlinewidth{0.000000pt}%
\definecolor{currentstroke}{rgb}{0.000000,0.000000,0.000000}%
\pgfsetstrokecolor{currentstroke}%
\pgfsetdash{}{0pt}%
\pgfpathmoveto{\pgfqpoint{4.646745in}{0.831051in}}%
\pgfpathlineto{\pgfqpoint{4.832972in}{0.831051in}}%
\pgfpathlineto{\pgfqpoint{4.832972in}{0.912779in}}%
\pgfpathlineto{\pgfqpoint{4.646745in}{0.912779in}}%
\pgfpathlineto{\pgfqpoint{4.646745in}{0.831051in}}%
\pgfusepath{fill}%
\end{pgfscope}%
\begin{pgfscope}%
\pgfpathrectangle{\pgfqpoint{0.549740in}{0.463273in}}{\pgfqpoint{9.320225in}{4.495057in}}%
\pgfusepath{clip}%
\pgfsetbuttcap%
\pgfsetroundjoin%
\pgfsetlinewidth{0.000000pt}%
\definecolor{currentstroke}{rgb}{0.000000,0.000000,0.000000}%
\pgfsetstrokecolor{currentstroke}%
\pgfsetdash{}{0pt}%
\pgfpathmoveto{\pgfqpoint{4.832972in}{0.831051in}}%
\pgfpathlineto{\pgfqpoint{5.019198in}{0.831051in}}%
\pgfpathlineto{\pgfqpoint{5.019198in}{0.912779in}}%
\pgfpathlineto{\pgfqpoint{4.832972in}{0.912779in}}%
\pgfpathlineto{\pgfqpoint{4.832972in}{0.831051in}}%
\pgfusepath{}%
\end{pgfscope}%
\begin{pgfscope}%
\pgfpathrectangle{\pgfqpoint{0.549740in}{0.463273in}}{\pgfqpoint{9.320225in}{4.495057in}}%
\pgfusepath{clip}%
\pgfsetbuttcap%
\pgfsetroundjoin%
\definecolor{currentfill}{rgb}{0.547810,0.736432,0.947518}%
\pgfsetfillcolor{currentfill}%
\pgfsetlinewidth{0.000000pt}%
\definecolor{currentstroke}{rgb}{0.000000,0.000000,0.000000}%
\pgfsetstrokecolor{currentstroke}%
\pgfsetdash{}{0pt}%
\pgfpathmoveto{\pgfqpoint{5.019198in}{0.831051in}}%
\pgfpathlineto{\pgfqpoint{5.205425in}{0.831051in}}%
\pgfpathlineto{\pgfqpoint{5.205425in}{0.912779in}}%
\pgfpathlineto{\pgfqpoint{5.019198in}{0.912779in}}%
\pgfpathlineto{\pgfqpoint{5.019198in}{0.831051in}}%
\pgfusepath{fill}%
\end{pgfscope}%
\begin{pgfscope}%
\pgfpathrectangle{\pgfqpoint{0.549740in}{0.463273in}}{\pgfqpoint{9.320225in}{4.495057in}}%
\pgfusepath{clip}%
\pgfsetbuttcap%
\pgfsetroundjoin%
\pgfsetlinewidth{0.000000pt}%
\definecolor{currentstroke}{rgb}{0.000000,0.000000,0.000000}%
\pgfsetstrokecolor{currentstroke}%
\pgfsetdash{}{0pt}%
\pgfpathmoveto{\pgfqpoint{5.205425in}{0.831051in}}%
\pgfpathlineto{\pgfqpoint{5.391651in}{0.831051in}}%
\pgfpathlineto{\pgfqpoint{5.391651in}{0.912779in}}%
\pgfpathlineto{\pgfqpoint{5.205425in}{0.912779in}}%
\pgfpathlineto{\pgfqpoint{5.205425in}{0.831051in}}%
\pgfusepath{}%
\end{pgfscope}%
\begin{pgfscope}%
\pgfpathrectangle{\pgfqpoint{0.549740in}{0.463273in}}{\pgfqpoint{9.320225in}{4.495057in}}%
\pgfusepath{clip}%
\pgfsetbuttcap%
\pgfsetroundjoin%
\pgfsetlinewidth{0.000000pt}%
\definecolor{currentstroke}{rgb}{0.000000,0.000000,0.000000}%
\pgfsetstrokecolor{currentstroke}%
\pgfsetdash{}{0pt}%
\pgfpathmoveto{\pgfqpoint{5.391651in}{0.831051in}}%
\pgfpathlineto{\pgfqpoint{5.577878in}{0.831051in}}%
\pgfpathlineto{\pgfqpoint{5.577878in}{0.912779in}}%
\pgfpathlineto{\pgfqpoint{5.391651in}{0.912779in}}%
\pgfpathlineto{\pgfqpoint{5.391651in}{0.831051in}}%
\pgfusepath{}%
\end{pgfscope}%
\begin{pgfscope}%
\pgfpathrectangle{\pgfqpoint{0.549740in}{0.463273in}}{\pgfqpoint{9.320225in}{4.495057in}}%
\pgfusepath{clip}%
\pgfsetbuttcap%
\pgfsetroundjoin%
\pgfsetlinewidth{0.000000pt}%
\definecolor{currentstroke}{rgb}{0.000000,0.000000,0.000000}%
\pgfsetstrokecolor{currentstroke}%
\pgfsetdash{}{0pt}%
\pgfpathmoveto{\pgfqpoint{5.577878in}{0.831051in}}%
\pgfpathlineto{\pgfqpoint{5.764104in}{0.831051in}}%
\pgfpathlineto{\pgfqpoint{5.764104in}{0.912779in}}%
\pgfpathlineto{\pgfqpoint{5.577878in}{0.912779in}}%
\pgfpathlineto{\pgfqpoint{5.577878in}{0.831051in}}%
\pgfusepath{}%
\end{pgfscope}%
\begin{pgfscope}%
\pgfpathrectangle{\pgfqpoint{0.549740in}{0.463273in}}{\pgfqpoint{9.320225in}{4.495057in}}%
\pgfusepath{clip}%
\pgfsetbuttcap%
\pgfsetroundjoin%
\pgfsetlinewidth{0.000000pt}%
\definecolor{currentstroke}{rgb}{0.000000,0.000000,0.000000}%
\pgfsetstrokecolor{currentstroke}%
\pgfsetdash{}{0pt}%
\pgfpathmoveto{\pgfqpoint{5.764104in}{0.831051in}}%
\pgfpathlineto{\pgfqpoint{5.950331in}{0.831051in}}%
\pgfpathlineto{\pgfqpoint{5.950331in}{0.912779in}}%
\pgfpathlineto{\pgfqpoint{5.764104in}{0.912779in}}%
\pgfpathlineto{\pgfqpoint{5.764104in}{0.831051in}}%
\pgfusepath{}%
\end{pgfscope}%
\begin{pgfscope}%
\pgfpathrectangle{\pgfqpoint{0.549740in}{0.463273in}}{\pgfqpoint{9.320225in}{4.495057in}}%
\pgfusepath{clip}%
\pgfsetbuttcap%
\pgfsetroundjoin%
\definecolor{currentfill}{rgb}{0.614330,0.761948,0.940009}%
\pgfsetfillcolor{currentfill}%
\pgfsetlinewidth{0.000000pt}%
\definecolor{currentstroke}{rgb}{0.000000,0.000000,0.000000}%
\pgfsetstrokecolor{currentstroke}%
\pgfsetdash{}{0pt}%
\pgfpathmoveto{\pgfqpoint{5.950331in}{0.831051in}}%
\pgfpathlineto{\pgfqpoint{6.136557in}{0.831051in}}%
\pgfpathlineto{\pgfqpoint{6.136557in}{0.912779in}}%
\pgfpathlineto{\pgfqpoint{5.950331in}{0.912779in}}%
\pgfpathlineto{\pgfqpoint{5.950331in}{0.831051in}}%
\pgfusepath{fill}%
\end{pgfscope}%
\begin{pgfscope}%
\pgfpathrectangle{\pgfqpoint{0.549740in}{0.463273in}}{\pgfqpoint{9.320225in}{4.495057in}}%
\pgfusepath{clip}%
\pgfsetbuttcap%
\pgfsetroundjoin%
\pgfsetlinewidth{0.000000pt}%
\definecolor{currentstroke}{rgb}{0.000000,0.000000,0.000000}%
\pgfsetstrokecolor{currentstroke}%
\pgfsetdash{}{0pt}%
\pgfpathmoveto{\pgfqpoint{6.136557in}{0.831051in}}%
\pgfpathlineto{\pgfqpoint{6.322784in}{0.831051in}}%
\pgfpathlineto{\pgfqpoint{6.322784in}{0.912779in}}%
\pgfpathlineto{\pgfqpoint{6.136557in}{0.912779in}}%
\pgfpathlineto{\pgfqpoint{6.136557in}{0.831051in}}%
\pgfusepath{}%
\end{pgfscope}%
\begin{pgfscope}%
\pgfpathrectangle{\pgfqpoint{0.549740in}{0.463273in}}{\pgfqpoint{9.320225in}{4.495057in}}%
\pgfusepath{clip}%
\pgfsetbuttcap%
\pgfsetroundjoin%
\pgfsetlinewidth{0.000000pt}%
\definecolor{currentstroke}{rgb}{0.000000,0.000000,0.000000}%
\pgfsetstrokecolor{currentstroke}%
\pgfsetdash{}{0pt}%
\pgfpathmoveto{\pgfqpoint{6.322784in}{0.831051in}}%
\pgfpathlineto{\pgfqpoint{6.509011in}{0.831051in}}%
\pgfpathlineto{\pgfqpoint{6.509011in}{0.912779in}}%
\pgfpathlineto{\pgfqpoint{6.322784in}{0.912779in}}%
\pgfpathlineto{\pgfqpoint{6.322784in}{0.831051in}}%
\pgfusepath{}%
\end{pgfscope}%
\begin{pgfscope}%
\pgfpathrectangle{\pgfqpoint{0.549740in}{0.463273in}}{\pgfqpoint{9.320225in}{4.495057in}}%
\pgfusepath{clip}%
\pgfsetbuttcap%
\pgfsetroundjoin%
\pgfsetlinewidth{0.000000pt}%
\definecolor{currentstroke}{rgb}{0.000000,0.000000,0.000000}%
\pgfsetstrokecolor{currentstroke}%
\pgfsetdash{}{0pt}%
\pgfpathmoveto{\pgfqpoint{6.509011in}{0.831051in}}%
\pgfpathlineto{\pgfqpoint{6.695237in}{0.831051in}}%
\pgfpathlineto{\pgfqpoint{6.695237in}{0.912779in}}%
\pgfpathlineto{\pgfqpoint{6.509011in}{0.912779in}}%
\pgfpathlineto{\pgfqpoint{6.509011in}{0.831051in}}%
\pgfusepath{}%
\end{pgfscope}%
\begin{pgfscope}%
\pgfpathrectangle{\pgfqpoint{0.549740in}{0.463273in}}{\pgfqpoint{9.320225in}{4.495057in}}%
\pgfusepath{clip}%
\pgfsetbuttcap%
\pgfsetroundjoin%
\pgfsetlinewidth{0.000000pt}%
\definecolor{currentstroke}{rgb}{0.000000,0.000000,0.000000}%
\pgfsetstrokecolor{currentstroke}%
\pgfsetdash{}{0pt}%
\pgfpathmoveto{\pgfqpoint{6.695237in}{0.831051in}}%
\pgfpathlineto{\pgfqpoint{6.881464in}{0.831051in}}%
\pgfpathlineto{\pgfqpoint{6.881464in}{0.912779in}}%
\pgfpathlineto{\pgfqpoint{6.695237in}{0.912779in}}%
\pgfpathlineto{\pgfqpoint{6.695237in}{0.831051in}}%
\pgfusepath{}%
\end{pgfscope}%
\begin{pgfscope}%
\pgfpathrectangle{\pgfqpoint{0.549740in}{0.463273in}}{\pgfqpoint{9.320225in}{4.495057in}}%
\pgfusepath{clip}%
\pgfsetbuttcap%
\pgfsetroundjoin%
\pgfsetlinewidth{0.000000pt}%
\definecolor{currentstroke}{rgb}{0.000000,0.000000,0.000000}%
\pgfsetstrokecolor{currentstroke}%
\pgfsetdash{}{0pt}%
\pgfpathmoveto{\pgfqpoint{6.881464in}{0.831051in}}%
\pgfpathlineto{\pgfqpoint{7.067690in}{0.831051in}}%
\pgfpathlineto{\pgfqpoint{7.067690in}{0.912779in}}%
\pgfpathlineto{\pgfqpoint{6.881464in}{0.912779in}}%
\pgfpathlineto{\pgfqpoint{6.881464in}{0.831051in}}%
\pgfusepath{}%
\end{pgfscope}%
\begin{pgfscope}%
\pgfpathrectangle{\pgfqpoint{0.549740in}{0.463273in}}{\pgfqpoint{9.320225in}{4.495057in}}%
\pgfusepath{clip}%
\pgfsetbuttcap%
\pgfsetroundjoin%
\pgfsetlinewidth{0.000000pt}%
\definecolor{currentstroke}{rgb}{0.000000,0.000000,0.000000}%
\pgfsetstrokecolor{currentstroke}%
\pgfsetdash{}{0pt}%
\pgfpathmoveto{\pgfqpoint{7.067690in}{0.831051in}}%
\pgfpathlineto{\pgfqpoint{7.253917in}{0.831051in}}%
\pgfpathlineto{\pgfqpoint{7.253917in}{0.912779in}}%
\pgfpathlineto{\pgfqpoint{7.067690in}{0.912779in}}%
\pgfpathlineto{\pgfqpoint{7.067690in}{0.831051in}}%
\pgfusepath{}%
\end{pgfscope}%
\begin{pgfscope}%
\pgfpathrectangle{\pgfqpoint{0.549740in}{0.463273in}}{\pgfqpoint{9.320225in}{4.495057in}}%
\pgfusepath{clip}%
\pgfsetbuttcap%
\pgfsetroundjoin%
\pgfsetlinewidth{0.000000pt}%
\definecolor{currentstroke}{rgb}{0.000000,0.000000,0.000000}%
\pgfsetstrokecolor{currentstroke}%
\pgfsetdash{}{0pt}%
\pgfpathmoveto{\pgfqpoint{7.253917in}{0.831051in}}%
\pgfpathlineto{\pgfqpoint{7.440143in}{0.831051in}}%
\pgfpathlineto{\pgfqpoint{7.440143in}{0.912779in}}%
\pgfpathlineto{\pgfqpoint{7.253917in}{0.912779in}}%
\pgfpathlineto{\pgfqpoint{7.253917in}{0.831051in}}%
\pgfusepath{}%
\end{pgfscope}%
\begin{pgfscope}%
\pgfpathrectangle{\pgfqpoint{0.549740in}{0.463273in}}{\pgfqpoint{9.320225in}{4.495057in}}%
\pgfusepath{clip}%
\pgfsetbuttcap%
\pgfsetroundjoin%
\pgfsetlinewidth{0.000000pt}%
\definecolor{currentstroke}{rgb}{0.000000,0.000000,0.000000}%
\pgfsetstrokecolor{currentstroke}%
\pgfsetdash{}{0pt}%
\pgfpathmoveto{\pgfqpoint{7.440143in}{0.831051in}}%
\pgfpathlineto{\pgfqpoint{7.626370in}{0.831051in}}%
\pgfpathlineto{\pgfqpoint{7.626370in}{0.912779in}}%
\pgfpathlineto{\pgfqpoint{7.440143in}{0.912779in}}%
\pgfpathlineto{\pgfqpoint{7.440143in}{0.831051in}}%
\pgfusepath{}%
\end{pgfscope}%
\begin{pgfscope}%
\pgfpathrectangle{\pgfqpoint{0.549740in}{0.463273in}}{\pgfqpoint{9.320225in}{4.495057in}}%
\pgfusepath{clip}%
\pgfsetbuttcap%
\pgfsetroundjoin%
\pgfsetlinewidth{0.000000pt}%
\definecolor{currentstroke}{rgb}{0.000000,0.000000,0.000000}%
\pgfsetstrokecolor{currentstroke}%
\pgfsetdash{}{0pt}%
\pgfpathmoveto{\pgfqpoint{7.626370in}{0.831051in}}%
\pgfpathlineto{\pgfqpoint{7.812596in}{0.831051in}}%
\pgfpathlineto{\pgfqpoint{7.812596in}{0.912779in}}%
\pgfpathlineto{\pgfqpoint{7.626370in}{0.912779in}}%
\pgfpathlineto{\pgfqpoint{7.626370in}{0.831051in}}%
\pgfusepath{}%
\end{pgfscope}%
\begin{pgfscope}%
\pgfpathrectangle{\pgfqpoint{0.549740in}{0.463273in}}{\pgfqpoint{9.320225in}{4.495057in}}%
\pgfusepath{clip}%
\pgfsetbuttcap%
\pgfsetroundjoin%
\pgfsetlinewidth{0.000000pt}%
\definecolor{currentstroke}{rgb}{0.000000,0.000000,0.000000}%
\pgfsetstrokecolor{currentstroke}%
\pgfsetdash{}{0pt}%
\pgfpathmoveto{\pgfqpoint{7.812596in}{0.831051in}}%
\pgfpathlineto{\pgfqpoint{7.998823in}{0.831051in}}%
\pgfpathlineto{\pgfqpoint{7.998823in}{0.912779in}}%
\pgfpathlineto{\pgfqpoint{7.812596in}{0.912779in}}%
\pgfpathlineto{\pgfqpoint{7.812596in}{0.831051in}}%
\pgfusepath{}%
\end{pgfscope}%
\begin{pgfscope}%
\pgfpathrectangle{\pgfqpoint{0.549740in}{0.463273in}}{\pgfqpoint{9.320225in}{4.495057in}}%
\pgfusepath{clip}%
\pgfsetbuttcap%
\pgfsetroundjoin%
\pgfsetlinewidth{0.000000pt}%
\definecolor{currentstroke}{rgb}{0.000000,0.000000,0.000000}%
\pgfsetstrokecolor{currentstroke}%
\pgfsetdash{}{0pt}%
\pgfpathmoveto{\pgfqpoint{7.998823in}{0.831051in}}%
\pgfpathlineto{\pgfqpoint{8.185049in}{0.831051in}}%
\pgfpathlineto{\pgfqpoint{8.185049in}{0.912779in}}%
\pgfpathlineto{\pgfqpoint{7.998823in}{0.912779in}}%
\pgfpathlineto{\pgfqpoint{7.998823in}{0.831051in}}%
\pgfusepath{}%
\end{pgfscope}%
\begin{pgfscope}%
\pgfpathrectangle{\pgfqpoint{0.549740in}{0.463273in}}{\pgfqpoint{9.320225in}{4.495057in}}%
\pgfusepath{clip}%
\pgfsetbuttcap%
\pgfsetroundjoin%
\pgfsetlinewidth{0.000000pt}%
\definecolor{currentstroke}{rgb}{0.000000,0.000000,0.000000}%
\pgfsetstrokecolor{currentstroke}%
\pgfsetdash{}{0pt}%
\pgfpathmoveto{\pgfqpoint{8.185049in}{0.831051in}}%
\pgfpathlineto{\pgfqpoint{8.371276in}{0.831051in}}%
\pgfpathlineto{\pgfqpoint{8.371276in}{0.912779in}}%
\pgfpathlineto{\pgfqpoint{8.185049in}{0.912779in}}%
\pgfpathlineto{\pgfqpoint{8.185049in}{0.831051in}}%
\pgfusepath{}%
\end{pgfscope}%
\begin{pgfscope}%
\pgfpathrectangle{\pgfqpoint{0.549740in}{0.463273in}}{\pgfqpoint{9.320225in}{4.495057in}}%
\pgfusepath{clip}%
\pgfsetbuttcap%
\pgfsetroundjoin%
\pgfsetlinewidth{0.000000pt}%
\definecolor{currentstroke}{rgb}{0.000000,0.000000,0.000000}%
\pgfsetstrokecolor{currentstroke}%
\pgfsetdash{}{0pt}%
\pgfpathmoveto{\pgfqpoint{8.371276in}{0.831051in}}%
\pgfpathlineto{\pgfqpoint{8.557503in}{0.831051in}}%
\pgfpathlineto{\pgfqpoint{8.557503in}{0.912779in}}%
\pgfpathlineto{\pgfqpoint{8.371276in}{0.912779in}}%
\pgfpathlineto{\pgfqpoint{8.371276in}{0.831051in}}%
\pgfusepath{}%
\end{pgfscope}%
\begin{pgfscope}%
\pgfpathrectangle{\pgfqpoint{0.549740in}{0.463273in}}{\pgfqpoint{9.320225in}{4.495057in}}%
\pgfusepath{clip}%
\pgfsetbuttcap%
\pgfsetroundjoin%
\pgfsetlinewidth{0.000000pt}%
\definecolor{currentstroke}{rgb}{0.000000,0.000000,0.000000}%
\pgfsetstrokecolor{currentstroke}%
\pgfsetdash{}{0pt}%
\pgfpathmoveto{\pgfqpoint{8.557503in}{0.831051in}}%
\pgfpathlineto{\pgfqpoint{8.743729in}{0.831051in}}%
\pgfpathlineto{\pgfqpoint{8.743729in}{0.912779in}}%
\pgfpathlineto{\pgfqpoint{8.557503in}{0.912779in}}%
\pgfpathlineto{\pgfqpoint{8.557503in}{0.831051in}}%
\pgfusepath{}%
\end{pgfscope}%
\begin{pgfscope}%
\pgfpathrectangle{\pgfqpoint{0.549740in}{0.463273in}}{\pgfqpoint{9.320225in}{4.495057in}}%
\pgfusepath{clip}%
\pgfsetbuttcap%
\pgfsetroundjoin%
\pgfsetlinewidth{0.000000pt}%
\definecolor{currentstroke}{rgb}{0.000000,0.000000,0.000000}%
\pgfsetstrokecolor{currentstroke}%
\pgfsetdash{}{0pt}%
\pgfpathmoveto{\pgfqpoint{8.743729in}{0.831051in}}%
\pgfpathlineto{\pgfqpoint{8.929956in}{0.831051in}}%
\pgfpathlineto{\pgfqpoint{8.929956in}{0.912779in}}%
\pgfpathlineto{\pgfqpoint{8.743729in}{0.912779in}}%
\pgfpathlineto{\pgfqpoint{8.743729in}{0.831051in}}%
\pgfusepath{}%
\end{pgfscope}%
\begin{pgfscope}%
\pgfpathrectangle{\pgfqpoint{0.549740in}{0.463273in}}{\pgfqpoint{9.320225in}{4.495057in}}%
\pgfusepath{clip}%
\pgfsetbuttcap%
\pgfsetroundjoin%
\pgfsetlinewidth{0.000000pt}%
\definecolor{currentstroke}{rgb}{0.000000,0.000000,0.000000}%
\pgfsetstrokecolor{currentstroke}%
\pgfsetdash{}{0pt}%
\pgfpathmoveto{\pgfqpoint{8.929956in}{0.831051in}}%
\pgfpathlineto{\pgfqpoint{9.116182in}{0.831051in}}%
\pgfpathlineto{\pgfqpoint{9.116182in}{0.912779in}}%
\pgfpathlineto{\pgfqpoint{8.929956in}{0.912779in}}%
\pgfpathlineto{\pgfqpoint{8.929956in}{0.831051in}}%
\pgfusepath{}%
\end{pgfscope}%
\begin{pgfscope}%
\pgfpathrectangle{\pgfqpoint{0.549740in}{0.463273in}}{\pgfqpoint{9.320225in}{4.495057in}}%
\pgfusepath{clip}%
\pgfsetbuttcap%
\pgfsetroundjoin%
\pgfsetlinewidth{0.000000pt}%
\definecolor{currentstroke}{rgb}{0.000000,0.000000,0.000000}%
\pgfsetstrokecolor{currentstroke}%
\pgfsetdash{}{0pt}%
\pgfpathmoveto{\pgfqpoint{9.116182in}{0.831051in}}%
\pgfpathlineto{\pgfqpoint{9.302409in}{0.831051in}}%
\pgfpathlineto{\pgfqpoint{9.302409in}{0.912779in}}%
\pgfpathlineto{\pgfqpoint{9.116182in}{0.912779in}}%
\pgfpathlineto{\pgfqpoint{9.116182in}{0.831051in}}%
\pgfusepath{}%
\end{pgfscope}%
\begin{pgfscope}%
\pgfpathrectangle{\pgfqpoint{0.549740in}{0.463273in}}{\pgfqpoint{9.320225in}{4.495057in}}%
\pgfusepath{clip}%
\pgfsetbuttcap%
\pgfsetroundjoin%
\pgfsetlinewidth{0.000000pt}%
\definecolor{currentstroke}{rgb}{0.000000,0.000000,0.000000}%
\pgfsetstrokecolor{currentstroke}%
\pgfsetdash{}{0pt}%
\pgfpathmoveto{\pgfqpoint{9.302409in}{0.831051in}}%
\pgfpathlineto{\pgfqpoint{9.488635in}{0.831051in}}%
\pgfpathlineto{\pgfqpoint{9.488635in}{0.912779in}}%
\pgfpathlineto{\pgfqpoint{9.302409in}{0.912779in}}%
\pgfpathlineto{\pgfqpoint{9.302409in}{0.831051in}}%
\pgfusepath{}%
\end{pgfscope}%
\begin{pgfscope}%
\pgfpathrectangle{\pgfqpoint{0.549740in}{0.463273in}}{\pgfqpoint{9.320225in}{4.495057in}}%
\pgfusepath{clip}%
\pgfsetbuttcap%
\pgfsetroundjoin%
\pgfsetlinewidth{0.000000pt}%
\definecolor{currentstroke}{rgb}{0.000000,0.000000,0.000000}%
\pgfsetstrokecolor{currentstroke}%
\pgfsetdash{}{0pt}%
\pgfpathmoveto{\pgfqpoint{9.488635in}{0.831051in}}%
\pgfpathlineto{\pgfqpoint{9.674862in}{0.831051in}}%
\pgfpathlineto{\pgfqpoint{9.674862in}{0.912779in}}%
\pgfpathlineto{\pgfqpoint{9.488635in}{0.912779in}}%
\pgfpathlineto{\pgfqpoint{9.488635in}{0.831051in}}%
\pgfusepath{}%
\end{pgfscope}%
\begin{pgfscope}%
\pgfpathrectangle{\pgfqpoint{0.549740in}{0.463273in}}{\pgfqpoint{9.320225in}{4.495057in}}%
\pgfusepath{clip}%
\pgfsetbuttcap%
\pgfsetroundjoin%
\pgfsetlinewidth{0.000000pt}%
\definecolor{currentstroke}{rgb}{0.000000,0.000000,0.000000}%
\pgfsetstrokecolor{currentstroke}%
\pgfsetdash{}{0pt}%
\pgfpathmoveto{\pgfqpoint{9.674862in}{0.831051in}}%
\pgfpathlineto{\pgfqpoint{9.861088in}{0.831051in}}%
\pgfpathlineto{\pgfqpoint{9.861088in}{0.912779in}}%
\pgfpathlineto{\pgfqpoint{9.674862in}{0.912779in}}%
\pgfpathlineto{\pgfqpoint{9.674862in}{0.831051in}}%
\pgfusepath{}%
\end{pgfscope}%
\begin{pgfscope}%
\pgfpathrectangle{\pgfqpoint{0.549740in}{0.463273in}}{\pgfqpoint{9.320225in}{4.495057in}}%
\pgfusepath{clip}%
\pgfsetbuttcap%
\pgfsetroundjoin%
\definecolor{currentfill}{rgb}{0.180573,0.522732,0.778127}%
\pgfsetfillcolor{currentfill}%
\pgfsetlinewidth{0.000000pt}%
\definecolor{currentstroke}{rgb}{0.000000,0.000000,0.000000}%
\pgfsetstrokecolor{currentstroke}%
\pgfsetdash{}{0pt}%
\pgfpathmoveto{\pgfqpoint{0.549761in}{0.912779in}}%
\pgfpathlineto{\pgfqpoint{0.735988in}{0.912779in}}%
\pgfpathlineto{\pgfqpoint{0.735988in}{0.994507in}}%
\pgfpathlineto{\pgfqpoint{0.549761in}{0.994507in}}%
\pgfpathlineto{\pgfqpoint{0.549761in}{0.912779in}}%
\pgfusepath{fill}%
\end{pgfscope}%
\begin{pgfscope}%
\pgfpathrectangle{\pgfqpoint{0.549740in}{0.463273in}}{\pgfqpoint{9.320225in}{4.495057in}}%
\pgfusepath{clip}%
\pgfsetbuttcap%
\pgfsetroundjoin%
\definecolor{currentfill}{rgb}{0.168741,0.551020,0.824827}%
\pgfsetfillcolor{currentfill}%
\pgfsetlinewidth{0.000000pt}%
\definecolor{currentstroke}{rgb}{0.000000,0.000000,0.000000}%
\pgfsetstrokecolor{currentstroke}%
\pgfsetdash{}{0pt}%
\pgfpathmoveto{\pgfqpoint{0.735988in}{0.912779in}}%
\pgfpathlineto{\pgfqpoint{0.922214in}{0.912779in}}%
\pgfpathlineto{\pgfqpoint{0.922214in}{0.994507in}}%
\pgfpathlineto{\pgfqpoint{0.735988in}{0.994507in}}%
\pgfpathlineto{\pgfqpoint{0.735988in}{0.912779in}}%
\pgfusepath{fill}%
\end{pgfscope}%
\begin{pgfscope}%
\pgfpathrectangle{\pgfqpoint{0.549740in}{0.463273in}}{\pgfqpoint{9.320225in}{4.495057in}}%
\pgfusepath{clip}%
\pgfsetbuttcap%
\pgfsetroundjoin%
\definecolor{currentfill}{rgb}{0.194981,0.494518,0.729769}%
\pgfsetfillcolor{currentfill}%
\pgfsetlinewidth{0.000000pt}%
\definecolor{currentstroke}{rgb}{0.000000,0.000000,0.000000}%
\pgfsetstrokecolor{currentstroke}%
\pgfsetdash{}{0pt}%
\pgfpathmoveto{\pgfqpoint{0.922214in}{0.912779in}}%
\pgfpathlineto{\pgfqpoint{1.108441in}{0.912779in}}%
\pgfpathlineto{\pgfqpoint{1.108441in}{0.994507in}}%
\pgfpathlineto{\pgfqpoint{0.922214in}{0.994507in}}%
\pgfpathlineto{\pgfqpoint{0.922214in}{0.912779in}}%
\pgfusepath{fill}%
\end{pgfscope}%
\begin{pgfscope}%
\pgfpathrectangle{\pgfqpoint{0.549740in}{0.463273in}}{\pgfqpoint{9.320225in}{4.495057in}}%
\pgfusepath{clip}%
\pgfsetbuttcap%
\pgfsetroundjoin%
\definecolor{currentfill}{rgb}{0.180573,0.522732,0.778127}%
\pgfsetfillcolor{currentfill}%
\pgfsetlinewidth{0.000000pt}%
\definecolor{currentstroke}{rgb}{0.000000,0.000000,0.000000}%
\pgfsetstrokecolor{currentstroke}%
\pgfsetdash{}{0pt}%
\pgfpathmoveto{\pgfqpoint{1.108441in}{0.912779in}}%
\pgfpathlineto{\pgfqpoint{1.294667in}{0.912779in}}%
\pgfpathlineto{\pgfqpoint{1.294667in}{0.994507in}}%
\pgfpathlineto{\pgfqpoint{1.108441in}{0.994507in}}%
\pgfpathlineto{\pgfqpoint{1.108441in}{0.912779in}}%
\pgfusepath{fill}%
\end{pgfscope}%
\begin{pgfscope}%
\pgfpathrectangle{\pgfqpoint{0.549740in}{0.463273in}}{\pgfqpoint{9.320225in}{4.495057in}}%
\pgfusepath{clip}%
\pgfsetbuttcap%
\pgfsetroundjoin%
\definecolor{currentfill}{rgb}{0.547810,0.736432,0.947518}%
\pgfsetfillcolor{currentfill}%
\pgfsetlinewidth{0.000000pt}%
\definecolor{currentstroke}{rgb}{0.000000,0.000000,0.000000}%
\pgfsetstrokecolor{currentstroke}%
\pgfsetdash{}{0pt}%
\pgfpathmoveto{\pgfqpoint{1.294667in}{0.912779in}}%
\pgfpathlineto{\pgfqpoint{1.480894in}{0.912779in}}%
\pgfpathlineto{\pgfqpoint{1.480894in}{0.994507in}}%
\pgfpathlineto{\pgfqpoint{1.294667in}{0.994507in}}%
\pgfpathlineto{\pgfqpoint{1.294667in}{0.912779in}}%
\pgfusepath{fill}%
\end{pgfscope}%
\begin{pgfscope}%
\pgfpathrectangle{\pgfqpoint{0.549740in}{0.463273in}}{\pgfqpoint{9.320225in}{4.495057in}}%
\pgfusepath{clip}%
\pgfsetbuttcap%
\pgfsetroundjoin%
\pgfsetlinewidth{0.000000pt}%
\definecolor{currentstroke}{rgb}{0.000000,0.000000,0.000000}%
\pgfsetstrokecolor{currentstroke}%
\pgfsetdash{}{0pt}%
\pgfpathmoveto{\pgfqpoint{1.480894in}{0.912779in}}%
\pgfpathlineto{\pgfqpoint{1.667120in}{0.912779in}}%
\pgfpathlineto{\pgfqpoint{1.667120in}{0.994507in}}%
\pgfpathlineto{\pgfqpoint{1.480894in}{0.994507in}}%
\pgfpathlineto{\pgfqpoint{1.480894in}{0.912779in}}%
\pgfusepath{}%
\end{pgfscope}%
\begin{pgfscope}%
\pgfpathrectangle{\pgfqpoint{0.549740in}{0.463273in}}{\pgfqpoint{9.320225in}{4.495057in}}%
\pgfusepath{clip}%
\pgfsetbuttcap%
\pgfsetroundjoin%
\definecolor{currentfill}{rgb}{0.180573,0.522732,0.778127}%
\pgfsetfillcolor{currentfill}%
\pgfsetlinewidth{0.000000pt}%
\definecolor{currentstroke}{rgb}{0.000000,0.000000,0.000000}%
\pgfsetstrokecolor{currentstroke}%
\pgfsetdash{}{0pt}%
\pgfpathmoveto{\pgfqpoint{1.667120in}{0.912779in}}%
\pgfpathlineto{\pgfqpoint{1.853347in}{0.912779in}}%
\pgfpathlineto{\pgfqpoint{1.853347in}{0.994507in}}%
\pgfpathlineto{\pgfqpoint{1.667120in}{0.994507in}}%
\pgfpathlineto{\pgfqpoint{1.667120in}{0.912779in}}%
\pgfusepath{fill}%
\end{pgfscope}%
\begin{pgfscope}%
\pgfpathrectangle{\pgfqpoint{0.549740in}{0.463273in}}{\pgfqpoint{9.320225in}{4.495057in}}%
\pgfusepath{clip}%
\pgfsetbuttcap%
\pgfsetroundjoin%
\definecolor{currentfill}{rgb}{0.385185,0.686583,0.962589}%
\pgfsetfillcolor{currentfill}%
\pgfsetlinewidth{0.000000pt}%
\definecolor{currentstroke}{rgb}{0.000000,0.000000,0.000000}%
\pgfsetstrokecolor{currentstroke}%
\pgfsetdash{}{0pt}%
\pgfpathmoveto{\pgfqpoint{1.853347in}{0.912779in}}%
\pgfpathlineto{\pgfqpoint{2.039573in}{0.912779in}}%
\pgfpathlineto{\pgfqpoint{2.039573in}{0.994507in}}%
\pgfpathlineto{\pgfqpoint{1.853347in}{0.994507in}}%
\pgfpathlineto{\pgfqpoint{1.853347in}{0.912779in}}%
\pgfusepath{fill}%
\end{pgfscope}%
\begin{pgfscope}%
\pgfpathrectangle{\pgfqpoint{0.549740in}{0.463273in}}{\pgfqpoint{9.320225in}{4.495057in}}%
\pgfusepath{clip}%
\pgfsetbuttcap%
\pgfsetroundjoin%
\definecolor{currentfill}{rgb}{0.273225,0.662144,0.968515}%
\pgfsetfillcolor{currentfill}%
\pgfsetlinewidth{0.000000pt}%
\definecolor{currentstroke}{rgb}{0.000000,0.000000,0.000000}%
\pgfsetstrokecolor{currentstroke}%
\pgfsetdash{}{0pt}%
\pgfpathmoveto{\pgfqpoint{2.039573in}{0.912779in}}%
\pgfpathlineto{\pgfqpoint{2.225800in}{0.912779in}}%
\pgfpathlineto{\pgfqpoint{2.225800in}{0.994507in}}%
\pgfpathlineto{\pgfqpoint{2.039573in}{0.994507in}}%
\pgfpathlineto{\pgfqpoint{2.039573in}{0.912779in}}%
\pgfusepath{fill}%
\end{pgfscope}%
\begin{pgfscope}%
\pgfpathrectangle{\pgfqpoint{0.549740in}{0.463273in}}{\pgfqpoint{9.320225in}{4.495057in}}%
\pgfusepath{clip}%
\pgfsetbuttcap%
\pgfsetroundjoin%
\definecolor{currentfill}{rgb}{0.273225,0.662144,0.968515}%
\pgfsetfillcolor{currentfill}%
\pgfsetlinewidth{0.000000pt}%
\definecolor{currentstroke}{rgb}{0.000000,0.000000,0.000000}%
\pgfsetstrokecolor{currentstroke}%
\pgfsetdash{}{0pt}%
\pgfpathmoveto{\pgfqpoint{2.225800in}{0.912779in}}%
\pgfpathlineto{\pgfqpoint{2.412027in}{0.912779in}}%
\pgfpathlineto{\pgfqpoint{2.412027in}{0.994507in}}%
\pgfpathlineto{\pgfqpoint{2.225800in}{0.994507in}}%
\pgfpathlineto{\pgfqpoint{2.225800in}{0.912779in}}%
\pgfusepath{fill}%
\end{pgfscope}%
\begin{pgfscope}%
\pgfpathrectangle{\pgfqpoint{0.549740in}{0.463273in}}{\pgfqpoint{9.320225in}{4.495057in}}%
\pgfusepath{clip}%
\pgfsetbuttcap%
\pgfsetroundjoin%
\pgfsetlinewidth{0.000000pt}%
\definecolor{currentstroke}{rgb}{0.000000,0.000000,0.000000}%
\pgfsetstrokecolor{currentstroke}%
\pgfsetdash{}{0pt}%
\pgfpathmoveto{\pgfqpoint{2.412027in}{0.912779in}}%
\pgfpathlineto{\pgfqpoint{2.598253in}{0.912779in}}%
\pgfpathlineto{\pgfqpoint{2.598253in}{0.994507in}}%
\pgfpathlineto{\pgfqpoint{2.412027in}{0.994507in}}%
\pgfpathlineto{\pgfqpoint{2.412027in}{0.912779in}}%
\pgfusepath{}%
\end{pgfscope}%
\begin{pgfscope}%
\pgfpathrectangle{\pgfqpoint{0.549740in}{0.463273in}}{\pgfqpoint{9.320225in}{4.495057in}}%
\pgfusepath{clip}%
\pgfsetbuttcap%
\pgfsetroundjoin%
\pgfsetlinewidth{0.000000pt}%
\definecolor{currentstroke}{rgb}{0.000000,0.000000,0.000000}%
\pgfsetstrokecolor{currentstroke}%
\pgfsetdash{}{0pt}%
\pgfpathmoveto{\pgfqpoint{2.598253in}{0.912779in}}%
\pgfpathlineto{\pgfqpoint{2.784480in}{0.912779in}}%
\pgfpathlineto{\pgfqpoint{2.784480in}{0.994507in}}%
\pgfpathlineto{\pgfqpoint{2.598253in}{0.994507in}}%
\pgfpathlineto{\pgfqpoint{2.598253in}{0.912779in}}%
\pgfusepath{}%
\end{pgfscope}%
\begin{pgfscope}%
\pgfpathrectangle{\pgfqpoint{0.549740in}{0.463273in}}{\pgfqpoint{9.320225in}{4.495057in}}%
\pgfusepath{clip}%
\pgfsetbuttcap%
\pgfsetroundjoin%
\pgfsetlinewidth{0.000000pt}%
\definecolor{currentstroke}{rgb}{0.000000,0.000000,0.000000}%
\pgfsetstrokecolor{currentstroke}%
\pgfsetdash{}{0pt}%
\pgfpathmoveto{\pgfqpoint{2.784480in}{0.912779in}}%
\pgfpathlineto{\pgfqpoint{2.970706in}{0.912779in}}%
\pgfpathlineto{\pgfqpoint{2.970706in}{0.994507in}}%
\pgfpathlineto{\pgfqpoint{2.784480in}{0.994507in}}%
\pgfpathlineto{\pgfqpoint{2.784480in}{0.912779in}}%
\pgfusepath{}%
\end{pgfscope}%
\begin{pgfscope}%
\pgfpathrectangle{\pgfqpoint{0.549740in}{0.463273in}}{\pgfqpoint{9.320225in}{4.495057in}}%
\pgfusepath{clip}%
\pgfsetbuttcap%
\pgfsetroundjoin%
\pgfsetlinewidth{0.000000pt}%
\definecolor{currentstroke}{rgb}{0.000000,0.000000,0.000000}%
\pgfsetstrokecolor{currentstroke}%
\pgfsetdash{}{0pt}%
\pgfpathmoveto{\pgfqpoint{2.970706in}{0.912779in}}%
\pgfpathlineto{\pgfqpoint{3.156933in}{0.912779in}}%
\pgfpathlineto{\pgfqpoint{3.156933in}{0.994507in}}%
\pgfpathlineto{\pgfqpoint{2.970706in}{0.994507in}}%
\pgfpathlineto{\pgfqpoint{2.970706in}{0.912779in}}%
\pgfusepath{}%
\end{pgfscope}%
\begin{pgfscope}%
\pgfpathrectangle{\pgfqpoint{0.549740in}{0.463273in}}{\pgfqpoint{9.320225in}{4.495057in}}%
\pgfusepath{clip}%
\pgfsetbuttcap%
\pgfsetroundjoin%
\pgfsetlinewidth{0.000000pt}%
\definecolor{currentstroke}{rgb}{0.000000,0.000000,0.000000}%
\pgfsetstrokecolor{currentstroke}%
\pgfsetdash{}{0pt}%
\pgfpathmoveto{\pgfqpoint{3.156933in}{0.912779in}}%
\pgfpathlineto{\pgfqpoint{3.343159in}{0.912779in}}%
\pgfpathlineto{\pgfqpoint{3.343159in}{0.994507in}}%
\pgfpathlineto{\pgfqpoint{3.156933in}{0.994507in}}%
\pgfpathlineto{\pgfqpoint{3.156933in}{0.912779in}}%
\pgfusepath{}%
\end{pgfscope}%
\begin{pgfscope}%
\pgfpathrectangle{\pgfqpoint{0.549740in}{0.463273in}}{\pgfqpoint{9.320225in}{4.495057in}}%
\pgfusepath{clip}%
\pgfsetbuttcap%
\pgfsetroundjoin%
\pgfsetlinewidth{0.000000pt}%
\definecolor{currentstroke}{rgb}{0.000000,0.000000,0.000000}%
\pgfsetstrokecolor{currentstroke}%
\pgfsetdash{}{0pt}%
\pgfpathmoveto{\pgfqpoint{3.343159in}{0.912779in}}%
\pgfpathlineto{\pgfqpoint{3.529386in}{0.912779in}}%
\pgfpathlineto{\pgfqpoint{3.529386in}{0.994507in}}%
\pgfpathlineto{\pgfqpoint{3.343159in}{0.994507in}}%
\pgfpathlineto{\pgfqpoint{3.343159in}{0.912779in}}%
\pgfusepath{}%
\end{pgfscope}%
\begin{pgfscope}%
\pgfpathrectangle{\pgfqpoint{0.549740in}{0.463273in}}{\pgfqpoint{9.320225in}{4.495057in}}%
\pgfusepath{clip}%
\pgfsetbuttcap%
\pgfsetroundjoin%
\definecolor{currentfill}{rgb}{0.614330,0.761948,0.940009}%
\pgfsetfillcolor{currentfill}%
\pgfsetlinewidth{0.000000pt}%
\definecolor{currentstroke}{rgb}{0.000000,0.000000,0.000000}%
\pgfsetstrokecolor{currentstroke}%
\pgfsetdash{}{0pt}%
\pgfpathmoveto{\pgfqpoint{3.529386in}{0.912779in}}%
\pgfpathlineto{\pgfqpoint{3.715612in}{0.912779in}}%
\pgfpathlineto{\pgfqpoint{3.715612in}{0.994507in}}%
\pgfpathlineto{\pgfqpoint{3.529386in}{0.994507in}}%
\pgfpathlineto{\pgfqpoint{3.529386in}{0.912779in}}%
\pgfusepath{fill}%
\end{pgfscope}%
\begin{pgfscope}%
\pgfpathrectangle{\pgfqpoint{0.549740in}{0.463273in}}{\pgfqpoint{9.320225in}{4.495057in}}%
\pgfusepath{clip}%
\pgfsetbuttcap%
\pgfsetroundjoin%
\pgfsetlinewidth{0.000000pt}%
\definecolor{currentstroke}{rgb}{0.000000,0.000000,0.000000}%
\pgfsetstrokecolor{currentstroke}%
\pgfsetdash{}{0pt}%
\pgfpathmoveto{\pgfqpoint{3.715612in}{0.912779in}}%
\pgfpathlineto{\pgfqpoint{3.901839in}{0.912779in}}%
\pgfpathlineto{\pgfqpoint{3.901839in}{0.994507in}}%
\pgfpathlineto{\pgfqpoint{3.715612in}{0.994507in}}%
\pgfpathlineto{\pgfqpoint{3.715612in}{0.912779in}}%
\pgfusepath{}%
\end{pgfscope}%
\begin{pgfscope}%
\pgfpathrectangle{\pgfqpoint{0.549740in}{0.463273in}}{\pgfqpoint{9.320225in}{4.495057in}}%
\pgfusepath{clip}%
\pgfsetbuttcap%
\pgfsetroundjoin%
\pgfsetlinewidth{0.000000pt}%
\definecolor{currentstroke}{rgb}{0.000000,0.000000,0.000000}%
\pgfsetstrokecolor{currentstroke}%
\pgfsetdash{}{0pt}%
\pgfpathmoveto{\pgfqpoint{3.901839in}{0.912779in}}%
\pgfpathlineto{\pgfqpoint{4.088065in}{0.912779in}}%
\pgfpathlineto{\pgfqpoint{4.088065in}{0.994507in}}%
\pgfpathlineto{\pgfqpoint{3.901839in}{0.994507in}}%
\pgfpathlineto{\pgfqpoint{3.901839in}{0.912779in}}%
\pgfusepath{}%
\end{pgfscope}%
\begin{pgfscope}%
\pgfpathrectangle{\pgfqpoint{0.549740in}{0.463273in}}{\pgfqpoint{9.320225in}{4.495057in}}%
\pgfusepath{clip}%
\pgfsetbuttcap%
\pgfsetroundjoin%
\pgfsetlinewidth{0.000000pt}%
\definecolor{currentstroke}{rgb}{0.000000,0.000000,0.000000}%
\pgfsetstrokecolor{currentstroke}%
\pgfsetdash{}{0pt}%
\pgfpathmoveto{\pgfqpoint{4.088065in}{0.912779in}}%
\pgfpathlineto{\pgfqpoint{4.274292in}{0.912779in}}%
\pgfpathlineto{\pgfqpoint{4.274292in}{0.994507in}}%
\pgfpathlineto{\pgfqpoint{4.088065in}{0.994507in}}%
\pgfpathlineto{\pgfqpoint{4.088065in}{0.912779in}}%
\pgfusepath{}%
\end{pgfscope}%
\begin{pgfscope}%
\pgfpathrectangle{\pgfqpoint{0.549740in}{0.463273in}}{\pgfqpoint{9.320225in}{4.495057in}}%
\pgfusepath{clip}%
\pgfsetbuttcap%
\pgfsetroundjoin%
\pgfsetlinewidth{0.000000pt}%
\definecolor{currentstroke}{rgb}{0.000000,0.000000,0.000000}%
\pgfsetstrokecolor{currentstroke}%
\pgfsetdash{}{0pt}%
\pgfpathmoveto{\pgfqpoint{4.274292in}{0.912779in}}%
\pgfpathlineto{\pgfqpoint{4.460519in}{0.912779in}}%
\pgfpathlineto{\pgfqpoint{4.460519in}{0.994507in}}%
\pgfpathlineto{\pgfqpoint{4.274292in}{0.994507in}}%
\pgfpathlineto{\pgfqpoint{4.274292in}{0.912779in}}%
\pgfusepath{}%
\end{pgfscope}%
\begin{pgfscope}%
\pgfpathrectangle{\pgfqpoint{0.549740in}{0.463273in}}{\pgfqpoint{9.320225in}{4.495057in}}%
\pgfusepath{clip}%
\pgfsetbuttcap%
\pgfsetroundjoin%
\pgfsetlinewidth{0.000000pt}%
\definecolor{currentstroke}{rgb}{0.000000,0.000000,0.000000}%
\pgfsetstrokecolor{currentstroke}%
\pgfsetdash{}{0pt}%
\pgfpathmoveto{\pgfqpoint{4.460519in}{0.912779in}}%
\pgfpathlineto{\pgfqpoint{4.646745in}{0.912779in}}%
\pgfpathlineto{\pgfqpoint{4.646745in}{0.994507in}}%
\pgfpathlineto{\pgfqpoint{4.460519in}{0.994507in}}%
\pgfpathlineto{\pgfqpoint{4.460519in}{0.912779in}}%
\pgfusepath{}%
\end{pgfscope}%
\begin{pgfscope}%
\pgfpathrectangle{\pgfqpoint{0.549740in}{0.463273in}}{\pgfqpoint{9.320225in}{4.495057in}}%
\pgfusepath{clip}%
\pgfsetbuttcap%
\pgfsetroundjoin%
\pgfsetlinewidth{0.000000pt}%
\definecolor{currentstroke}{rgb}{0.000000,0.000000,0.000000}%
\pgfsetstrokecolor{currentstroke}%
\pgfsetdash{}{0pt}%
\pgfpathmoveto{\pgfqpoint{4.646745in}{0.912779in}}%
\pgfpathlineto{\pgfqpoint{4.832972in}{0.912779in}}%
\pgfpathlineto{\pgfqpoint{4.832972in}{0.994507in}}%
\pgfpathlineto{\pgfqpoint{4.646745in}{0.994507in}}%
\pgfpathlineto{\pgfqpoint{4.646745in}{0.912779in}}%
\pgfusepath{}%
\end{pgfscope}%
\begin{pgfscope}%
\pgfpathrectangle{\pgfqpoint{0.549740in}{0.463273in}}{\pgfqpoint{9.320225in}{4.495057in}}%
\pgfusepath{clip}%
\pgfsetbuttcap%
\pgfsetroundjoin%
\pgfsetlinewidth{0.000000pt}%
\definecolor{currentstroke}{rgb}{0.000000,0.000000,0.000000}%
\pgfsetstrokecolor{currentstroke}%
\pgfsetdash{}{0pt}%
\pgfpathmoveto{\pgfqpoint{4.832972in}{0.912779in}}%
\pgfpathlineto{\pgfqpoint{5.019198in}{0.912779in}}%
\pgfpathlineto{\pgfqpoint{5.019198in}{0.994507in}}%
\pgfpathlineto{\pgfqpoint{4.832972in}{0.994507in}}%
\pgfpathlineto{\pgfqpoint{4.832972in}{0.912779in}}%
\pgfusepath{}%
\end{pgfscope}%
\begin{pgfscope}%
\pgfpathrectangle{\pgfqpoint{0.549740in}{0.463273in}}{\pgfqpoint{9.320225in}{4.495057in}}%
\pgfusepath{clip}%
\pgfsetbuttcap%
\pgfsetroundjoin%
\pgfsetlinewidth{0.000000pt}%
\definecolor{currentstroke}{rgb}{0.000000,0.000000,0.000000}%
\pgfsetstrokecolor{currentstroke}%
\pgfsetdash{}{0pt}%
\pgfpathmoveto{\pgfqpoint{5.019198in}{0.912779in}}%
\pgfpathlineto{\pgfqpoint{5.205425in}{0.912779in}}%
\pgfpathlineto{\pgfqpoint{5.205425in}{0.994507in}}%
\pgfpathlineto{\pgfqpoint{5.019198in}{0.994507in}}%
\pgfpathlineto{\pgfqpoint{5.019198in}{0.912779in}}%
\pgfusepath{}%
\end{pgfscope}%
\begin{pgfscope}%
\pgfpathrectangle{\pgfqpoint{0.549740in}{0.463273in}}{\pgfqpoint{9.320225in}{4.495057in}}%
\pgfusepath{clip}%
\pgfsetbuttcap%
\pgfsetroundjoin%
\pgfsetlinewidth{0.000000pt}%
\definecolor{currentstroke}{rgb}{0.000000,0.000000,0.000000}%
\pgfsetstrokecolor{currentstroke}%
\pgfsetdash{}{0pt}%
\pgfpathmoveto{\pgfqpoint{5.205425in}{0.912779in}}%
\pgfpathlineto{\pgfqpoint{5.391651in}{0.912779in}}%
\pgfpathlineto{\pgfqpoint{5.391651in}{0.994507in}}%
\pgfpathlineto{\pgfqpoint{5.205425in}{0.994507in}}%
\pgfpathlineto{\pgfqpoint{5.205425in}{0.912779in}}%
\pgfusepath{}%
\end{pgfscope}%
\begin{pgfscope}%
\pgfpathrectangle{\pgfqpoint{0.549740in}{0.463273in}}{\pgfqpoint{9.320225in}{4.495057in}}%
\pgfusepath{clip}%
\pgfsetbuttcap%
\pgfsetroundjoin%
\pgfsetlinewidth{0.000000pt}%
\definecolor{currentstroke}{rgb}{0.000000,0.000000,0.000000}%
\pgfsetstrokecolor{currentstroke}%
\pgfsetdash{}{0pt}%
\pgfpathmoveto{\pgfqpoint{5.391651in}{0.912779in}}%
\pgfpathlineto{\pgfqpoint{5.577878in}{0.912779in}}%
\pgfpathlineto{\pgfqpoint{5.577878in}{0.994507in}}%
\pgfpathlineto{\pgfqpoint{5.391651in}{0.994507in}}%
\pgfpathlineto{\pgfqpoint{5.391651in}{0.912779in}}%
\pgfusepath{}%
\end{pgfscope}%
\begin{pgfscope}%
\pgfpathrectangle{\pgfqpoint{0.549740in}{0.463273in}}{\pgfqpoint{9.320225in}{4.495057in}}%
\pgfusepath{clip}%
\pgfsetbuttcap%
\pgfsetroundjoin%
\pgfsetlinewidth{0.000000pt}%
\definecolor{currentstroke}{rgb}{0.000000,0.000000,0.000000}%
\pgfsetstrokecolor{currentstroke}%
\pgfsetdash{}{0pt}%
\pgfpathmoveto{\pgfqpoint{5.577878in}{0.912779in}}%
\pgfpathlineto{\pgfqpoint{5.764104in}{0.912779in}}%
\pgfpathlineto{\pgfqpoint{5.764104in}{0.994507in}}%
\pgfpathlineto{\pgfqpoint{5.577878in}{0.994507in}}%
\pgfpathlineto{\pgfqpoint{5.577878in}{0.912779in}}%
\pgfusepath{}%
\end{pgfscope}%
\begin{pgfscope}%
\pgfpathrectangle{\pgfqpoint{0.549740in}{0.463273in}}{\pgfqpoint{9.320225in}{4.495057in}}%
\pgfusepath{clip}%
\pgfsetbuttcap%
\pgfsetroundjoin%
\pgfsetlinewidth{0.000000pt}%
\definecolor{currentstroke}{rgb}{0.000000,0.000000,0.000000}%
\pgfsetstrokecolor{currentstroke}%
\pgfsetdash{}{0pt}%
\pgfpathmoveto{\pgfqpoint{5.764104in}{0.912779in}}%
\pgfpathlineto{\pgfqpoint{5.950331in}{0.912779in}}%
\pgfpathlineto{\pgfqpoint{5.950331in}{0.994507in}}%
\pgfpathlineto{\pgfqpoint{5.764104in}{0.994507in}}%
\pgfpathlineto{\pgfqpoint{5.764104in}{0.912779in}}%
\pgfusepath{}%
\end{pgfscope}%
\begin{pgfscope}%
\pgfpathrectangle{\pgfqpoint{0.549740in}{0.463273in}}{\pgfqpoint{9.320225in}{4.495057in}}%
\pgfusepath{clip}%
\pgfsetbuttcap%
\pgfsetroundjoin%
\definecolor{currentfill}{rgb}{0.472869,0.711325,0.955316}%
\pgfsetfillcolor{currentfill}%
\pgfsetlinewidth{0.000000pt}%
\definecolor{currentstroke}{rgb}{0.000000,0.000000,0.000000}%
\pgfsetstrokecolor{currentstroke}%
\pgfsetdash{}{0pt}%
\pgfpathmoveto{\pgfqpoint{5.950331in}{0.912779in}}%
\pgfpathlineto{\pgfqpoint{6.136557in}{0.912779in}}%
\pgfpathlineto{\pgfqpoint{6.136557in}{0.994507in}}%
\pgfpathlineto{\pgfqpoint{5.950331in}{0.994507in}}%
\pgfpathlineto{\pgfqpoint{5.950331in}{0.912779in}}%
\pgfusepath{fill}%
\end{pgfscope}%
\begin{pgfscope}%
\pgfpathrectangle{\pgfqpoint{0.549740in}{0.463273in}}{\pgfqpoint{9.320225in}{4.495057in}}%
\pgfusepath{clip}%
\pgfsetbuttcap%
\pgfsetroundjoin%
\pgfsetlinewidth{0.000000pt}%
\definecolor{currentstroke}{rgb}{0.000000,0.000000,0.000000}%
\pgfsetstrokecolor{currentstroke}%
\pgfsetdash{}{0pt}%
\pgfpathmoveto{\pgfqpoint{6.136557in}{0.912779in}}%
\pgfpathlineto{\pgfqpoint{6.322784in}{0.912779in}}%
\pgfpathlineto{\pgfqpoint{6.322784in}{0.994507in}}%
\pgfpathlineto{\pgfqpoint{6.136557in}{0.994507in}}%
\pgfpathlineto{\pgfqpoint{6.136557in}{0.912779in}}%
\pgfusepath{}%
\end{pgfscope}%
\begin{pgfscope}%
\pgfpathrectangle{\pgfqpoint{0.549740in}{0.463273in}}{\pgfqpoint{9.320225in}{4.495057in}}%
\pgfusepath{clip}%
\pgfsetbuttcap%
\pgfsetroundjoin%
\pgfsetlinewidth{0.000000pt}%
\definecolor{currentstroke}{rgb}{0.000000,0.000000,0.000000}%
\pgfsetstrokecolor{currentstroke}%
\pgfsetdash{}{0pt}%
\pgfpathmoveto{\pgfqpoint{6.322784in}{0.912779in}}%
\pgfpathlineto{\pgfqpoint{6.509011in}{0.912779in}}%
\pgfpathlineto{\pgfqpoint{6.509011in}{0.994507in}}%
\pgfpathlineto{\pgfqpoint{6.322784in}{0.994507in}}%
\pgfpathlineto{\pgfqpoint{6.322784in}{0.912779in}}%
\pgfusepath{}%
\end{pgfscope}%
\begin{pgfscope}%
\pgfpathrectangle{\pgfqpoint{0.549740in}{0.463273in}}{\pgfqpoint{9.320225in}{4.495057in}}%
\pgfusepath{clip}%
\pgfsetbuttcap%
\pgfsetroundjoin%
\pgfsetlinewidth{0.000000pt}%
\definecolor{currentstroke}{rgb}{0.000000,0.000000,0.000000}%
\pgfsetstrokecolor{currentstroke}%
\pgfsetdash{}{0pt}%
\pgfpathmoveto{\pgfqpoint{6.509011in}{0.912779in}}%
\pgfpathlineto{\pgfqpoint{6.695237in}{0.912779in}}%
\pgfpathlineto{\pgfqpoint{6.695237in}{0.994507in}}%
\pgfpathlineto{\pgfqpoint{6.509011in}{0.994507in}}%
\pgfpathlineto{\pgfqpoint{6.509011in}{0.912779in}}%
\pgfusepath{}%
\end{pgfscope}%
\begin{pgfscope}%
\pgfpathrectangle{\pgfqpoint{0.549740in}{0.463273in}}{\pgfqpoint{9.320225in}{4.495057in}}%
\pgfusepath{clip}%
\pgfsetbuttcap%
\pgfsetroundjoin%
\pgfsetlinewidth{0.000000pt}%
\definecolor{currentstroke}{rgb}{0.000000,0.000000,0.000000}%
\pgfsetstrokecolor{currentstroke}%
\pgfsetdash{}{0pt}%
\pgfpathmoveto{\pgfqpoint{6.695237in}{0.912779in}}%
\pgfpathlineto{\pgfqpoint{6.881464in}{0.912779in}}%
\pgfpathlineto{\pgfqpoint{6.881464in}{0.994507in}}%
\pgfpathlineto{\pgfqpoint{6.695237in}{0.994507in}}%
\pgfpathlineto{\pgfqpoint{6.695237in}{0.912779in}}%
\pgfusepath{}%
\end{pgfscope}%
\begin{pgfscope}%
\pgfpathrectangle{\pgfqpoint{0.549740in}{0.463273in}}{\pgfqpoint{9.320225in}{4.495057in}}%
\pgfusepath{clip}%
\pgfsetbuttcap%
\pgfsetroundjoin%
\pgfsetlinewidth{0.000000pt}%
\definecolor{currentstroke}{rgb}{0.000000,0.000000,0.000000}%
\pgfsetstrokecolor{currentstroke}%
\pgfsetdash{}{0pt}%
\pgfpathmoveto{\pgfqpoint{6.881464in}{0.912779in}}%
\pgfpathlineto{\pgfqpoint{7.067690in}{0.912779in}}%
\pgfpathlineto{\pgfqpoint{7.067690in}{0.994507in}}%
\pgfpathlineto{\pgfqpoint{6.881464in}{0.994507in}}%
\pgfpathlineto{\pgfqpoint{6.881464in}{0.912779in}}%
\pgfusepath{}%
\end{pgfscope}%
\begin{pgfscope}%
\pgfpathrectangle{\pgfqpoint{0.549740in}{0.463273in}}{\pgfqpoint{9.320225in}{4.495057in}}%
\pgfusepath{clip}%
\pgfsetbuttcap%
\pgfsetroundjoin%
\pgfsetlinewidth{0.000000pt}%
\definecolor{currentstroke}{rgb}{0.000000,0.000000,0.000000}%
\pgfsetstrokecolor{currentstroke}%
\pgfsetdash{}{0pt}%
\pgfpathmoveto{\pgfqpoint{7.067690in}{0.912779in}}%
\pgfpathlineto{\pgfqpoint{7.253917in}{0.912779in}}%
\pgfpathlineto{\pgfqpoint{7.253917in}{0.994507in}}%
\pgfpathlineto{\pgfqpoint{7.067690in}{0.994507in}}%
\pgfpathlineto{\pgfqpoint{7.067690in}{0.912779in}}%
\pgfusepath{}%
\end{pgfscope}%
\begin{pgfscope}%
\pgfpathrectangle{\pgfqpoint{0.549740in}{0.463273in}}{\pgfqpoint{9.320225in}{4.495057in}}%
\pgfusepath{clip}%
\pgfsetbuttcap%
\pgfsetroundjoin%
\pgfsetlinewidth{0.000000pt}%
\definecolor{currentstroke}{rgb}{0.000000,0.000000,0.000000}%
\pgfsetstrokecolor{currentstroke}%
\pgfsetdash{}{0pt}%
\pgfpathmoveto{\pgfqpoint{7.253917in}{0.912779in}}%
\pgfpathlineto{\pgfqpoint{7.440143in}{0.912779in}}%
\pgfpathlineto{\pgfqpoint{7.440143in}{0.994507in}}%
\pgfpathlineto{\pgfqpoint{7.253917in}{0.994507in}}%
\pgfpathlineto{\pgfqpoint{7.253917in}{0.912779in}}%
\pgfusepath{}%
\end{pgfscope}%
\begin{pgfscope}%
\pgfpathrectangle{\pgfqpoint{0.549740in}{0.463273in}}{\pgfqpoint{9.320225in}{4.495057in}}%
\pgfusepath{clip}%
\pgfsetbuttcap%
\pgfsetroundjoin%
\pgfsetlinewidth{0.000000pt}%
\definecolor{currentstroke}{rgb}{0.000000,0.000000,0.000000}%
\pgfsetstrokecolor{currentstroke}%
\pgfsetdash{}{0pt}%
\pgfpathmoveto{\pgfqpoint{7.440143in}{0.912779in}}%
\pgfpathlineto{\pgfqpoint{7.626370in}{0.912779in}}%
\pgfpathlineto{\pgfqpoint{7.626370in}{0.994507in}}%
\pgfpathlineto{\pgfqpoint{7.440143in}{0.994507in}}%
\pgfpathlineto{\pgfqpoint{7.440143in}{0.912779in}}%
\pgfusepath{}%
\end{pgfscope}%
\begin{pgfscope}%
\pgfpathrectangle{\pgfqpoint{0.549740in}{0.463273in}}{\pgfqpoint{9.320225in}{4.495057in}}%
\pgfusepath{clip}%
\pgfsetbuttcap%
\pgfsetroundjoin%
\pgfsetlinewidth{0.000000pt}%
\definecolor{currentstroke}{rgb}{0.000000,0.000000,0.000000}%
\pgfsetstrokecolor{currentstroke}%
\pgfsetdash{}{0pt}%
\pgfpathmoveto{\pgfqpoint{7.626370in}{0.912779in}}%
\pgfpathlineto{\pgfqpoint{7.812596in}{0.912779in}}%
\pgfpathlineto{\pgfqpoint{7.812596in}{0.994507in}}%
\pgfpathlineto{\pgfqpoint{7.626370in}{0.994507in}}%
\pgfpathlineto{\pgfqpoint{7.626370in}{0.912779in}}%
\pgfusepath{}%
\end{pgfscope}%
\begin{pgfscope}%
\pgfpathrectangle{\pgfqpoint{0.549740in}{0.463273in}}{\pgfqpoint{9.320225in}{4.495057in}}%
\pgfusepath{clip}%
\pgfsetbuttcap%
\pgfsetroundjoin%
\pgfsetlinewidth{0.000000pt}%
\definecolor{currentstroke}{rgb}{0.000000,0.000000,0.000000}%
\pgfsetstrokecolor{currentstroke}%
\pgfsetdash{}{0pt}%
\pgfpathmoveto{\pgfqpoint{7.812596in}{0.912779in}}%
\pgfpathlineto{\pgfqpoint{7.998823in}{0.912779in}}%
\pgfpathlineto{\pgfqpoint{7.998823in}{0.994507in}}%
\pgfpathlineto{\pgfqpoint{7.812596in}{0.994507in}}%
\pgfpathlineto{\pgfqpoint{7.812596in}{0.912779in}}%
\pgfusepath{}%
\end{pgfscope}%
\begin{pgfscope}%
\pgfpathrectangle{\pgfqpoint{0.549740in}{0.463273in}}{\pgfqpoint{9.320225in}{4.495057in}}%
\pgfusepath{clip}%
\pgfsetbuttcap%
\pgfsetroundjoin%
\pgfsetlinewidth{0.000000pt}%
\definecolor{currentstroke}{rgb}{0.000000,0.000000,0.000000}%
\pgfsetstrokecolor{currentstroke}%
\pgfsetdash{}{0pt}%
\pgfpathmoveto{\pgfqpoint{7.998823in}{0.912779in}}%
\pgfpathlineto{\pgfqpoint{8.185049in}{0.912779in}}%
\pgfpathlineto{\pgfqpoint{8.185049in}{0.994507in}}%
\pgfpathlineto{\pgfqpoint{7.998823in}{0.994507in}}%
\pgfpathlineto{\pgfqpoint{7.998823in}{0.912779in}}%
\pgfusepath{}%
\end{pgfscope}%
\begin{pgfscope}%
\pgfpathrectangle{\pgfqpoint{0.549740in}{0.463273in}}{\pgfqpoint{9.320225in}{4.495057in}}%
\pgfusepath{clip}%
\pgfsetbuttcap%
\pgfsetroundjoin%
\pgfsetlinewidth{0.000000pt}%
\definecolor{currentstroke}{rgb}{0.000000,0.000000,0.000000}%
\pgfsetstrokecolor{currentstroke}%
\pgfsetdash{}{0pt}%
\pgfpathmoveto{\pgfqpoint{8.185049in}{0.912779in}}%
\pgfpathlineto{\pgfqpoint{8.371276in}{0.912779in}}%
\pgfpathlineto{\pgfqpoint{8.371276in}{0.994507in}}%
\pgfpathlineto{\pgfqpoint{8.185049in}{0.994507in}}%
\pgfpathlineto{\pgfqpoint{8.185049in}{0.912779in}}%
\pgfusepath{}%
\end{pgfscope}%
\begin{pgfscope}%
\pgfpathrectangle{\pgfqpoint{0.549740in}{0.463273in}}{\pgfqpoint{9.320225in}{4.495057in}}%
\pgfusepath{clip}%
\pgfsetbuttcap%
\pgfsetroundjoin%
\pgfsetlinewidth{0.000000pt}%
\definecolor{currentstroke}{rgb}{0.000000,0.000000,0.000000}%
\pgfsetstrokecolor{currentstroke}%
\pgfsetdash{}{0pt}%
\pgfpathmoveto{\pgfqpoint{8.371276in}{0.912779in}}%
\pgfpathlineto{\pgfqpoint{8.557503in}{0.912779in}}%
\pgfpathlineto{\pgfqpoint{8.557503in}{0.994507in}}%
\pgfpathlineto{\pgfqpoint{8.371276in}{0.994507in}}%
\pgfpathlineto{\pgfqpoint{8.371276in}{0.912779in}}%
\pgfusepath{}%
\end{pgfscope}%
\begin{pgfscope}%
\pgfpathrectangle{\pgfqpoint{0.549740in}{0.463273in}}{\pgfqpoint{9.320225in}{4.495057in}}%
\pgfusepath{clip}%
\pgfsetbuttcap%
\pgfsetroundjoin%
\pgfsetlinewidth{0.000000pt}%
\definecolor{currentstroke}{rgb}{0.000000,0.000000,0.000000}%
\pgfsetstrokecolor{currentstroke}%
\pgfsetdash{}{0pt}%
\pgfpathmoveto{\pgfqpoint{8.557503in}{0.912779in}}%
\pgfpathlineto{\pgfqpoint{8.743729in}{0.912779in}}%
\pgfpathlineto{\pgfqpoint{8.743729in}{0.994507in}}%
\pgfpathlineto{\pgfqpoint{8.557503in}{0.994507in}}%
\pgfpathlineto{\pgfqpoint{8.557503in}{0.912779in}}%
\pgfusepath{}%
\end{pgfscope}%
\begin{pgfscope}%
\pgfpathrectangle{\pgfqpoint{0.549740in}{0.463273in}}{\pgfqpoint{9.320225in}{4.495057in}}%
\pgfusepath{clip}%
\pgfsetbuttcap%
\pgfsetroundjoin%
\pgfsetlinewidth{0.000000pt}%
\definecolor{currentstroke}{rgb}{0.000000,0.000000,0.000000}%
\pgfsetstrokecolor{currentstroke}%
\pgfsetdash{}{0pt}%
\pgfpathmoveto{\pgfqpoint{8.743729in}{0.912779in}}%
\pgfpathlineto{\pgfqpoint{8.929956in}{0.912779in}}%
\pgfpathlineto{\pgfqpoint{8.929956in}{0.994507in}}%
\pgfpathlineto{\pgfqpoint{8.743729in}{0.994507in}}%
\pgfpathlineto{\pgfqpoint{8.743729in}{0.912779in}}%
\pgfusepath{}%
\end{pgfscope}%
\begin{pgfscope}%
\pgfpathrectangle{\pgfqpoint{0.549740in}{0.463273in}}{\pgfqpoint{9.320225in}{4.495057in}}%
\pgfusepath{clip}%
\pgfsetbuttcap%
\pgfsetroundjoin%
\pgfsetlinewidth{0.000000pt}%
\definecolor{currentstroke}{rgb}{0.000000,0.000000,0.000000}%
\pgfsetstrokecolor{currentstroke}%
\pgfsetdash{}{0pt}%
\pgfpathmoveto{\pgfqpoint{8.929956in}{0.912779in}}%
\pgfpathlineto{\pgfqpoint{9.116182in}{0.912779in}}%
\pgfpathlineto{\pgfqpoint{9.116182in}{0.994507in}}%
\pgfpathlineto{\pgfqpoint{8.929956in}{0.994507in}}%
\pgfpathlineto{\pgfqpoint{8.929956in}{0.912779in}}%
\pgfusepath{}%
\end{pgfscope}%
\begin{pgfscope}%
\pgfpathrectangle{\pgfqpoint{0.549740in}{0.463273in}}{\pgfqpoint{9.320225in}{4.495057in}}%
\pgfusepath{clip}%
\pgfsetbuttcap%
\pgfsetroundjoin%
\pgfsetlinewidth{0.000000pt}%
\definecolor{currentstroke}{rgb}{0.000000,0.000000,0.000000}%
\pgfsetstrokecolor{currentstroke}%
\pgfsetdash{}{0pt}%
\pgfpathmoveto{\pgfqpoint{9.116182in}{0.912779in}}%
\pgfpathlineto{\pgfqpoint{9.302409in}{0.912779in}}%
\pgfpathlineto{\pgfqpoint{9.302409in}{0.994507in}}%
\pgfpathlineto{\pgfqpoint{9.116182in}{0.994507in}}%
\pgfpathlineto{\pgfqpoint{9.116182in}{0.912779in}}%
\pgfusepath{}%
\end{pgfscope}%
\begin{pgfscope}%
\pgfpathrectangle{\pgfqpoint{0.549740in}{0.463273in}}{\pgfqpoint{9.320225in}{4.495057in}}%
\pgfusepath{clip}%
\pgfsetbuttcap%
\pgfsetroundjoin%
\pgfsetlinewidth{0.000000pt}%
\definecolor{currentstroke}{rgb}{0.000000,0.000000,0.000000}%
\pgfsetstrokecolor{currentstroke}%
\pgfsetdash{}{0pt}%
\pgfpathmoveto{\pgfqpoint{9.302409in}{0.912779in}}%
\pgfpathlineto{\pgfqpoint{9.488635in}{0.912779in}}%
\pgfpathlineto{\pgfqpoint{9.488635in}{0.994507in}}%
\pgfpathlineto{\pgfqpoint{9.302409in}{0.994507in}}%
\pgfpathlineto{\pgfqpoint{9.302409in}{0.912779in}}%
\pgfusepath{}%
\end{pgfscope}%
\begin{pgfscope}%
\pgfpathrectangle{\pgfqpoint{0.549740in}{0.463273in}}{\pgfqpoint{9.320225in}{4.495057in}}%
\pgfusepath{clip}%
\pgfsetbuttcap%
\pgfsetroundjoin%
\pgfsetlinewidth{0.000000pt}%
\definecolor{currentstroke}{rgb}{0.000000,0.000000,0.000000}%
\pgfsetstrokecolor{currentstroke}%
\pgfsetdash{}{0pt}%
\pgfpathmoveto{\pgfqpoint{9.488635in}{0.912779in}}%
\pgfpathlineto{\pgfqpoint{9.674862in}{0.912779in}}%
\pgfpathlineto{\pgfqpoint{9.674862in}{0.994507in}}%
\pgfpathlineto{\pgfqpoint{9.488635in}{0.994507in}}%
\pgfpathlineto{\pgfqpoint{9.488635in}{0.912779in}}%
\pgfusepath{}%
\end{pgfscope}%
\begin{pgfscope}%
\pgfpathrectangle{\pgfqpoint{0.549740in}{0.463273in}}{\pgfqpoint{9.320225in}{4.495057in}}%
\pgfusepath{clip}%
\pgfsetbuttcap%
\pgfsetroundjoin%
\pgfsetlinewidth{0.000000pt}%
\definecolor{currentstroke}{rgb}{0.000000,0.000000,0.000000}%
\pgfsetstrokecolor{currentstroke}%
\pgfsetdash{}{0pt}%
\pgfpathmoveto{\pgfqpoint{9.674862in}{0.912779in}}%
\pgfpathlineto{\pgfqpoint{9.861088in}{0.912779in}}%
\pgfpathlineto{\pgfqpoint{9.861088in}{0.994507in}}%
\pgfpathlineto{\pgfqpoint{9.674862in}{0.994507in}}%
\pgfpathlineto{\pgfqpoint{9.674862in}{0.912779in}}%
\pgfusepath{}%
\end{pgfscope}%
\begin{pgfscope}%
\pgfpathrectangle{\pgfqpoint{0.549740in}{0.463273in}}{\pgfqpoint{9.320225in}{4.495057in}}%
\pgfusepath{clip}%
\pgfsetbuttcap%
\pgfsetroundjoin%
\definecolor{currentfill}{rgb}{0.614330,0.761948,0.940009}%
\pgfsetfillcolor{currentfill}%
\pgfsetlinewidth{0.000000pt}%
\definecolor{currentstroke}{rgb}{0.000000,0.000000,0.000000}%
\pgfsetstrokecolor{currentstroke}%
\pgfsetdash{}{0pt}%
\pgfpathmoveto{\pgfqpoint{0.549761in}{0.994507in}}%
\pgfpathlineto{\pgfqpoint{0.735988in}{0.994507in}}%
\pgfpathlineto{\pgfqpoint{0.735988in}{1.076236in}}%
\pgfpathlineto{\pgfqpoint{0.549761in}{1.076236in}}%
\pgfpathlineto{\pgfqpoint{0.549761in}{0.994507in}}%
\pgfusepath{fill}%
\end{pgfscope}%
\begin{pgfscope}%
\pgfpathrectangle{\pgfqpoint{0.549740in}{0.463273in}}{\pgfqpoint{9.320225in}{4.495057in}}%
\pgfusepath{clip}%
\pgfsetbuttcap%
\pgfsetroundjoin%
\definecolor{currentfill}{rgb}{0.472869,0.711325,0.955316}%
\pgfsetfillcolor{currentfill}%
\pgfsetlinewidth{0.000000pt}%
\definecolor{currentstroke}{rgb}{0.000000,0.000000,0.000000}%
\pgfsetstrokecolor{currentstroke}%
\pgfsetdash{}{0pt}%
\pgfpathmoveto{\pgfqpoint{0.735988in}{0.994507in}}%
\pgfpathlineto{\pgfqpoint{0.922214in}{0.994507in}}%
\pgfpathlineto{\pgfqpoint{0.922214in}{1.076236in}}%
\pgfpathlineto{\pgfqpoint{0.735988in}{1.076236in}}%
\pgfpathlineto{\pgfqpoint{0.735988in}{0.994507in}}%
\pgfusepath{fill}%
\end{pgfscope}%
\begin{pgfscope}%
\pgfpathrectangle{\pgfqpoint{0.549740in}{0.463273in}}{\pgfqpoint{9.320225in}{4.495057in}}%
\pgfusepath{clip}%
\pgfsetbuttcap%
\pgfsetroundjoin%
\definecolor{currentfill}{rgb}{0.614330,0.761948,0.940009}%
\pgfsetfillcolor{currentfill}%
\pgfsetlinewidth{0.000000pt}%
\definecolor{currentstroke}{rgb}{0.000000,0.000000,0.000000}%
\pgfsetstrokecolor{currentstroke}%
\pgfsetdash{}{0pt}%
\pgfpathmoveto{\pgfqpoint{0.922214in}{0.994507in}}%
\pgfpathlineto{\pgfqpoint{1.108441in}{0.994507in}}%
\pgfpathlineto{\pgfqpoint{1.108441in}{1.076236in}}%
\pgfpathlineto{\pgfqpoint{0.922214in}{1.076236in}}%
\pgfpathlineto{\pgfqpoint{0.922214in}{0.994507in}}%
\pgfusepath{fill}%
\end{pgfscope}%
\begin{pgfscope}%
\pgfpathrectangle{\pgfqpoint{0.549740in}{0.463273in}}{\pgfqpoint{9.320225in}{4.495057in}}%
\pgfusepath{clip}%
\pgfsetbuttcap%
\pgfsetroundjoin%
\pgfsetlinewidth{0.000000pt}%
\definecolor{currentstroke}{rgb}{0.000000,0.000000,0.000000}%
\pgfsetstrokecolor{currentstroke}%
\pgfsetdash{}{0pt}%
\pgfpathmoveto{\pgfqpoint{1.108441in}{0.994507in}}%
\pgfpathlineto{\pgfqpoint{1.294667in}{0.994507in}}%
\pgfpathlineto{\pgfqpoint{1.294667in}{1.076236in}}%
\pgfpathlineto{\pgfqpoint{1.108441in}{1.076236in}}%
\pgfpathlineto{\pgfqpoint{1.108441in}{0.994507in}}%
\pgfusepath{}%
\end{pgfscope}%
\begin{pgfscope}%
\pgfpathrectangle{\pgfqpoint{0.549740in}{0.463273in}}{\pgfqpoint{9.320225in}{4.495057in}}%
\pgfusepath{clip}%
\pgfsetbuttcap%
\pgfsetroundjoin%
\pgfsetlinewidth{0.000000pt}%
\definecolor{currentstroke}{rgb}{0.000000,0.000000,0.000000}%
\pgfsetstrokecolor{currentstroke}%
\pgfsetdash{}{0pt}%
\pgfpathmoveto{\pgfqpoint{1.294667in}{0.994507in}}%
\pgfpathlineto{\pgfqpoint{1.480894in}{0.994507in}}%
\pgfpathlineto{\pgfqpoint{1.480894in}{1.076236in}}%
\pgfpathlineto{\pgfqpoint{1.294667in}{1.076236in}}%
\pgfpathlineto{\pgfqpoint{1.294667in}{0.994507in}}%
\pgfusepath{}%
\end{pgfscope}%
\begin{pgfscope}%
\pgfpathrectangle{\pgfqpoint{0.549740in}{0.463273in}}{\pgfqpoint{9.320225in}{4.495057in}}%
\pgfusepath{clip}%
\pgfsetbuttcap%
\pgfsetroundjoin%
\pgfsetlinewidth{0.000000pt}%
\definecolor{currentstroke}{rgb}{0.000000,0.000000,0.000000}%
\pgfsetstrokecolor{currentstroke}%
\pgfsetdash{}{0pt}%
\pgfpathmoveto{\pgfqpoint{1.480894in}{0.994507in}}%
\pgfpathlineto{\pgfqpoint{1.667120in}{0.994507in}}%
\pgfpathlineto{\pgfqpoint{1.667120in}{1.076236in}}%
\pgfpathlineto{\pgfqpoint{1.480894in}{1.076236in}}%
\pgfpathlineto{\pgfqpoint{1.480894in}{0.994507in}}%
\pgfusepath{}%
\end{pgfscope}%
\begin{pgfscope}%
\pgfpathrectangle{\pgfqpoint{0.549740in}{0.463273in}}{\pgfqpoint{9.320225in}{4.495057in}}%
\pgfusepath{clip}%
\pgfsetbuttcap%
\pgfsetroundjoin%
\pgfsetlinewidth{0.000000pt}%
\definecolor{currentstroke}{rgb}{0.000000,0.000000,0.000000}%
\pgfsetstrokecolor{currentstroke}%
\pgfsetdash{}{0pt}%
\pgfpathmoveto{\pgfqpoint{1.667120in}{0.994507in}}%
\pgfpathlineto{\pgfqpoint{1.853347in}{0.994507in}}%
\pgfpathlineto{\pgfqpoint{1.853347in}{1.076236in}}%
\pgfpathlineto{\pgfqpoint{1.667120in}{1.076236in}}%
\pgfpathlineto{\pgfqpoint{1.667120in}{0.994507in}}%
\pgfusepath{}%
\end{pgfscope}%
\begin{pgfscope}%
\pgfpathrectangle{\pgfqpoint{0.549740in}{0.463273in}}{\pgfqpoint{9.320225in}{4.495057in}}%
\pgfusepath{clip}%
\pgfsetbuttcap%
\pgfsetroundjoin%
\pgfsetlinewidth{0.000000pt}%
\definecolor{currentstroke}{rgb}{0.000000,0.000000,0.000000}%
\pgfsetstrokecolor{currentstroke}%
\pgfsetdash{}{0pt}%
\pgfpathmoveto{\pgfqpoint{1.853347in}{0.994507in}}%
\pgfpathlineto{\pgfqpoint{2.039573in}{0.994507in}}%
\pgfpathlineto{\pgfqpoint{2.039573in}{1.076236in}}%
\pgfpathlineto{\pgfqpoint{1.853347in}{1.076236in}}%
\pgfpathlineto{\pgfqpoint{1.853347in}{0.994507in}}%
\pgfusepath{}%
\end{pgfscope}%
\begin{pgfscope}%
\pgfpathrectangle{\pgfqpoint{0.549740in}{0.463273in}}{\pgfqpoint{9.320225in}{4.495057in}}%
\pgfusepath{clip}%
\pgfsetbuttcap%
\pgfsetroundjoin%
\pgfsetlinewidth{0.000000pt}%
\definecolor{currentstroke}{rgb}{0.000000,0.000000,0.000000}%
\pgfsetstrokecolor{currentstroke}%
\pgfsetdash{}{0pt}%
\pgfpathmoveto{\pgfqpoint{2.039573in}{0.994507in}}%
\pgfpathlineto{\pgfqpoint{2.225800in}{0.994507in}}%
\pgfpathlineto{\pgfqpoint{2.225800in}{1.076236in}}%
\pgfpathlineto{\pgfqpoint{2.039573in}{1.076236in}}%
\pgfpathlineto{\pgfqpoint{2.039573in}{0.994507in}}%
\pgfusepath{}%
\end{pgfscope}%
\begin{pgfscope}%
\pgfpathrectangle{\pgfqpoint{0.549740in}{0.463273in}}{\pgfqpoint{9.320225in}{4.495057in}}%
\pgfusepath{clip}%
\pgfsetbuttcap%
\pgfsetroundjoin%
\definecolor{currentfill}{rgb}{0.614330,0.761948,0.940009}%
\pgfsetfillcolor{currentfill}%
\pgfsetlinewidth{0.000000pt}%
\definecolor{currentstroke}{rgb}{0.000000,0.000000,0.000000}%
\pgfsetstrokecolor{currentstroke}%
\pgfsetdash{}{0pt}%
\pgfpathmoveto{\pgfqpoint{2.225800in}{0.994507in}}%
\pgfpathlineto{\pgfqpoint{2.412027in}{0.994507in}}%
\pgfpathlineto{\pgfqpoint{2.412027in}{1.076236in}}%
\pgfpathlineto{\pgfqpoint{2.225800in}{1.076236in}}%
\pgfpathlineto{\pgfqpoint{2.225800in}{0.994507in}}%
\pgfusepath{fill}%
\end{pgfscope}%
\begin{pgfscope}%
\pgfpathrectangle{\pgfqpoint{0.549740in}{0.463273in}}{\pgfqpoint{9.320225in}{4.495057in}}%
\pgfusepath{clip}%
\pgfsetbuttcap%
\pgfsetroundjoin%
\pgfsetlinewidth{0.000000pt}%
\definecolor{currentstroke}{rgb}{0.000000,0.000000,0.000000}%
\pgfsetstrokecolor{currentstroke}%
\pgfsetdash{}{0pt}%
\pgfpathmoveto{\pgfqpoint{2.412027in}{0.994507in}}%
\pgfpathlineto{\pgfqpoint{2.598253in}{0.994507in}}%
\pgfpathlineto{\pgfqpoint{2.598253in}{1.076236in}}%
\pgfpathlineto{\pgfqpoint{2.412027in}{1.076236in}}%
\pgfpathlineto{\pgfqpoint{2.412027in}{0.994507in}}%
\pgfusepath{}%
\end{pgfscope}%
\begin{pgfscope}%
\pgfpathrectangle{\pgfqpoint{0.549740in}{0.463273in}}{\pgfqpoint{9.320225in}{4.495057in}}%
\pgfusepath{clip}%
\pgfsetbuttcap%
\pgfsetroundjoin%
\pgfsetlinewidth{0.000000pt}%
\definecolor{currentstroke}{rgb}{0.000000,0.000000,0.000000}%
\pgfsetstrokecolor{currentstroke}%
\pgfsetdash{}{0pt}%
\pgfpathmoveto{\pgfqpoint{2.598253in}{0.994507in}}%
\pgfpathlineto{\pgfqpoint{2.784480in}{0.994507in}}%
\pgfpathlineto{\pgfqpoint{2.784480in}{1.076236in}}%
\pgfpathlineto{\pgfqpoint{2.598253in}{1.076236in}}%
\pgfpathlineto{\pgfqpoint{2.598253in}{0.994507in}}%
\pgfusepath{}%
\end{pgfscope}%
\begin{pgfscope}%
\pgfpathrectangle{\pgfqpoint{0.549740in}{0.463273in}}{\pgfqpoint{9.320225in}{4.495057in}}%
\pgfusepath{clip}%
\pgfsetbuttcap%
\pgfsetroundjoin%
\pgfsetlinewidth{0.000000pt}%
\definecolor{currentstroke}{rgb}{0.000000,0.000000,0.000000}%
\pgfsetstrokecolor{currentstroke}%
\pgfsetdash{}{0pt}%
\pgfpathmoveto{\pgfqpoint{2.784480in}{0.994507in}}%
\pgfpathlineto{\pgfqpoint{2.970706in}{0.994507in}}%
\pgfpathlineto{\pgfqpoint{2.970706in}{1.076236in}}%
\pgfpathlineto{\pgfqpoint{2.784480in}{1.076236in}}%
\pgfpathlineto{\pgfqpoint{2.784480in}{0.994507in}}%
\pgfusepath{}%
\end{pgfscope}%
\begin{pgfscope}%
\pgfpathrectangle{\pgfqpoint{0.549740in}{0.463273in}}{\pgfqpoint{9.320225in}{4.495057in}}%
\pgfusepath{clip}%
\pgfsetbuttcap%
\pgfsetroundjoin%
\pgfsetlinewidth{0.000000pt}%
\definecolor{currentstroke}{rgb}{0.000000,0.000000,0.000000}%
\pgfsetstrokecolor{currentstroke}%
\pgfsetdash{}{0pt}%
\pgfpathmoveto{\pgfqpoint{2.970706in}{0.994507in}}%
\pgfpathlineto{\pgfqpoint{3.156933in}{0.994507in}}%
\pgfpathlineto{\pgfqpoint{3.156933in}{1.076236in}}%
\pgfpathlineto{\pgfqpoint{2.970706in}{1.076236in}}%
\pgfpathlineto{\pgfqpoint{2.970706in}{0.994507in}}%
\pgfusepath{}%
\end{pgfscope}%
\begin{pgfscope}%
\pgfpathrectangle{\pgfqpoint{0.549740in}{0.463273in}}{\pgfqpoint{9.320225in}{4.495057in}}%
\pgfusepath{clip}%
\pgfsetbuttcap%
\pgfsetroundjoin%
\pgfsetlinewidth{0.000000pt}%
\definecolor{currentstroke}{rgb}{0.000000,0.000000,0.000000}%
\pgfsetstrokecolor{currentstroke}%
\pgfsetdash{}{0pt}%
\pgfpathmoveto{\pgfqpoint{3.156933in}{0.994507in}}%
\pgfpathlineto{\pgfqpoint{3.343159in}{0.994507in}}%
\pgfpathlineto{\pgfqpoint{3.343159in}{1.076236in}}%
\pgfpathlineto{\pgfqpoint{3.156933in}{1.076236in}}%
\pgfpathlineto{\pgfqpoint{3.156933in}{0.994507in}}%
\pgfusepath{}%
\end{pgfscope}%
\begin{pgfscope}%
\pgfpathrectangle{\pgfqpoint{0.549740in}{0.463273in}}{\pgfqpoint{9.320225in}{4.495057in}}%
\pgfusepath{clip}%
\pgfsetbuttcap%
\pgfsetroundjoin%
\pgfsetlinewidth{0.000000pt}%
\definecolor{currentstroke}{rgb}{0.000000,0.000000,0.000000}%
\pgfsetstrokecolor{currentstroke}%
\pgfsetdash{}{0pt}%
\pgfpathmoveto{\pgfqpoint{3.343159in}{0.994507in}}%
\pgfpathlineto{\pgfqpoint{3.529386in}{0.994507in}}%
\pgfpathlineto{\pgfqpoint{3.529386in}{1.076236in}}%
\pgfpathlineto{\pgfqpoint{3.343159in}{1.076236in}}%
\pgfpathlineto{\pgfqpoint{3.343159in}{0.994507in}}%
\pgfusepath{}%
\end{pgfscope}%
\begin{pgfscope}%
\pgfpathrectangle{\pgfqpoint{0.549740in}{0.463273in}}{\pgfqpoint{9.320225in}{4.495057in}}%
\pgfusepath{clip}%
\pgfsetbuttcap%
\pgfsetroundjoin%
\pgfsetlinewidth{0.000000pt}%
\definecolor{currentstroke}{rgb}{0.000000,0.000000,0.000000}%
\pgfsetstrokecolor{currentstroke}%
\pgfsetdash{}{0pt}%
\pgfpathmoveto{\pgfqpoint{3.529386in}{0.994507in}}%
\pgfpathlineto{\pgfqpoint{3.715612in}{0.994507in}}%
\pgfpathlineto{\pgfqpoint{3.715612in}{1.076236in}}%
\pgfpathlineto{\pgfqpoint{3.529386in}{1.076236in}}%
\pgfpathlineto{\pgfqpoint{3.529386in}{0.994507in}}%
\pgfusepath{}%
\end{pgfscope}%
\begin{pgfscope}%
\pgfpathrectangle{\pgfqpoint{0.549740in}{0.463273in}}{\pgfqpoint{9.320225in}{4.495057in}}%
\pgfusepath{clip}%
\pgfsetbuttcap%
\pgfsetroundjoin%
\pgfsetlinewidth{0.000000pt}%
\definecolor{currentstroke}{rgb}{0.000000,0.000000,0.000000}%
\pgfsetstrokecolor{currentstroke}%
\pgfsetdash{}{0pt}%
\pgfpathmoveto{\pgfqpoint{3.715612in}{0.994507in}}%
\pgfpathlineto{\pgfqpoint{3.901839in}{0.994507in}}%
\pgfpathlineto{\pgfqpoint{3.901839in}{1.076236in}}%
\pgfpathlineto{\pgfqpoint{3.715612in}{1.076236in}}%
\pgfpathlineto{\pgfqpoint{3.715612in}{0.994507in}}%
\pgfusepath{}%
\end{pgfscope}%
\begin{pgfscope}%
\pgfpathrectangle{\pgfqpoint{0.549740in}{0.463273in}}{\pgfqpoint{9.320225in}{4.495057in}}%
\pgfusepath{clip}%
\pgfsetbuttcap%
\pgfsetroundjoin%
\pgfsetlinewidth{0.000000pt}%
\definecolor{currentstroke}{rgb}{0.000000,0.000000,0.000000}%
\pgfsetstrokecolor{currentstroke}%
\pgfsetdash{}{0pt}%
\pgfpathmoveto{\pgfqpoint{3.901839in}{0.994507in}}%
\pgfpathlineto{\pgfqpoint{4.088065in}{0.994507in}}%
\pgfpathlineto{\pgfqpoint{4.088065in}{1.076236in}}%
\pgfpathlineto{\pgfqpoint{3.901839in}{1.076236in}}%
\pgfpathlineto{\pgfqpoint{3.901839in}{0.994507in}}%
\pgfusepath{}%
\end{pgfscope}%
\begin{pgfscope}%
\pgfpathrectangle{\pgfqpoint{0.549740in}{0.463273in}}{\pgfqpoint{9.320225in}{4.495057in}}%
\pgfusepath{clip}%
\pgfsetbuttcap%
\pgfsetroundjoin%
\pgfsetlinewidth{0.000000pt}%
\definecolor{currentstroke}{rgb}{0.000000,0.000000,0.000000}%
\pgfsetstrokecolor{currentstroke}%
\pgfsetdash{}{0pt}%
\pgfpathmoveto{\pgfqpoint{4.088065in}{0.994507in}}%
\pgfpathlineto{\pgfqpoint{4.274292in}{0.994507in}}%
\pgfpathlineto{\pgfqpoint{4.274292in}{1.076236in}}%
\pgfpathlineto{\pgfqpoint{4.088065in}{1.076236in}}%
\pgfpathlineto{\pgfqpoint{4.088065in}{0.994507in}}%
\pgfusepath{}%
\end{pgfscope}%
\begin{pgfscope}%
\pgfpathrectangle{\pgfqpoint{0.549740in}{0.463273in}}{\pgfqpoint{9.320225in}{4.495057in}}%
\pgfusepath{clip}%
\pgfsetbuttcap%
\pgfsetroundjoin%
\pgfsetlinewidth{0.000000pt}%
\definecolor{currentstroke}{rgb}{0.000000,0.000000,0.000000}%
\pgfsetstrokecolor{currentstroke}%
\pgfsetdash{}{0pt}%
\pgfpathmoveto{\pgfqpoint{4.274292in}{0.994507in}}%
\pgfpathlineto{\pgfqpoint{4.460519in}{0.994507in}}%
\pgfpathlineto{\pgfqpoint{4.460519in}{1.076236in}}%
\pgfpathlineto{\pgfqpoint{4.274292in}{1.076236in}}%
\pgfpathlineto{\pgfqpoint{4.274292in}{0.994507in}}%
\pgfusepath{}%
\end{pgfscope}%
\begin{pgfscope}%
\pgfpathrectangle{\pgfqpoint{0.549740in}{0.463273in}}{\pgfqpoint{9.320225in}{4.495057in}}%
\pgfusepath{clip}%
\pgfsetbuttcap%
\pgfsetroundjoin%
\pgfsetlinewidth{0.000000pt}%
\definecolor{currentstroke}{rgb}{0.000000,0.000000,0.000000}%
\pgfsetstrokecolor{currentstroke}%
\pgfsetdash{}{0pt}%
\pgfpathmoveto{\pgfqpoint{4.460519in}{0.994507in}}%
\pgfpathlineto{\pgfqpoint{4.646745in}{0.994507in}}%
\pgfpathlineto{\pgfqpoint{4.646745in}{1.076236in}}%
\pgfpathlineto{\pgfqpoint{4.460519in}{1.076236in}}%
\pgfpathlineto{\pgfqpoint{4.460519in}{0.994507in}}%
\pgfusepath{}%
\end{pgfscope}%
\begin{pgfscope}%
\pgfpathrectangle{\pgfqpoint{0.549740in}{0.463273in}}{\pgfqpoint{9.320225in}{4.495057in}}%
\pgfusepath{clip}%
\pgfsetbuttcap%
\pgfsetroundjoin%
\pgfsetlinewidth{0.000000pt}%
\definecolor{currentstroke}{rgb}{0.000000,0.000000,0.000000}%
\pgfsetstrokecolor{currentstroke}%
\pgfsetdash{}{0pt}%
\pgfpathmoveto{\pgfqpoint{4.646745in}{0.994507in}}%
\pgfpathlineto{\pgfqpoint{4.832972in}{0.994507in}}%
\pgfpathlineto{\pgfqpoint{4.832972in}{1.076236in}}%
\pgfpathlineto{\pgfqpoint{4.646745in}{1.076236in}}%
\pgfpathlineto{\pgfqpoint{4.646745in}{0.994507in}}%
\pgfusepath{}%
\end{pgfscope}%
\begin{pgfscope}%
\pgfpathrectangle{\pgfqpoint{0.549740in}{0.463273in}}{\pgfqpoint{9.320225in}{4.495057in}}%
\pgfusepath{clip}%
\pgfsetbuttcap%
\pgfsetroundjoin%
\pgfsetlinewidth{0.000000pt}%
\definecolor{currentstroke}{rgb}{0.000000,0.000000,0.000000}%
\pgfsetstrokecolor{currentstroke}%
\pgfsetdash{}{0pt}%
\pgfpathmoveto{\pgfqpoint{4.832972in}{0.994507in}}%
\pgfpathlineto{\pgfqpoint{5.019198in}{0.994507in}}%
\pgfpathlineto{\pgfqpoint{5.019198in}{1.076236in}}%
\pgfpathlineto{\pgfqpoint{4.832972in}{1.076236in}}%
\pgfpathlineto{\pgfqpoint{4.832972in}{0.994507in}}%
\pgfusepath{}%
\end{pgfscope}%
\begin{pgfscope}%
\pgfpathrectangle{\pgfqpoint{0.549740in}{0.463273in}}{\pgfqpoint{9.320225in}{4.495057in}}%
\pgfusepath{clip}%
\pgfsetbuttcap%
\pgfsetroundjoin%
\pgfsetlinewidth{0.000000pt}%
\definecolor{currentstroke}{rgb}{0.000000,0.000000,0.000000}%
\pgfsetstrokecolor{currentstroke}%
\pgfsetdash{}{0pt}%
\pgfpathmoveto{\pgfqpoint{5.019198in}{0.994507in}}%
\pgfpathlineto{\pgfqpoint{5.205425in}{0.994507in}}%
\pgfpathlineto{\pgfqpoint{5.205425in}{1.076236in}}%
\pgfpathlineto{\pgfqpoint{5.019198in}{1.076236in}}%
\pgfpathlineto{\pgfqpoint{5.019198in}{0.994507in}}%
\pgfusepath{}%
\end{pgfscope}%
\begin{pgfscope}%
\pgfpathrectangle{\pgfqpoint{0.549740in}{0.463273in}}{\pgfqpoint{9.320225in}{4.495057in}}%
\pgfusepath{clip}%
\pgfsetbuttcap%
\pgfsetroundjoin%
\pgfsetlinewidth{0.000000pt}%
\definecolor{currentstroke}{rgb}{0.000000,0.000000,0.000000}%
\pgfsetstrokecolor{currentstroke}%
\pgfsetdash{}{0pt}%
\pgfpathmoveto{\pgfqpoint{5.205425in}{0.994507in}}%
\pgfpathlineto{\pgfqpoint{5.391651in}{0.994507in}}%
\pgfpathlineto{\pgfqpoint{5.391651in}{1.076236in}}%
\pgfpathlineto{\pgfqpoint{5.205425in}{1.076236in}}%
\pgfpathlineto{\pgfqpoint{5.205425in}{0.994507in}}%
\pgfusepath{}%
\end{pgfscope}%
\begin{pgfscope}%
\pgfpathrectangle{\pgfqpoint{0.549740in}{0.463273in}}{\pgfqpoint{9.320225in}{4.495057in}}%
\pgfusepath{clip}%
\pgfsetbuttcap%
\pgfsetroundjoin%
\pgfsetlinewidth{0.000000pt}%
\definecolor{currentstroke}{rgb}{0.000000,0.000000,0.000000}%
\pgfsetstrokecolor{currentstroke}%
\pgfsetdash{}{0pt}%
\pgfpathmoveto{\pgfqpoint{5.391651in}{0.994507in}}%
\pgfpathlineto{\pgfqpoint{5.577878in}{0.994507in}}%
\pgfpathlineto{\pgfqpoint{5.577878in}{1.076236in}}%
\pgfpathlineto{\pgfqpoint{5.391651in}{1.076236in}}%
\pgfpathlineto{\pgfqpoint{5.391651in}{0.994507in}}%
\pgfusepath{}%
\end{pgfscope}%
\begin{pgfscope}%
\pgfpathrectangle{\pgfqpoint{0.549740in}{0.463273in}}{\pgfqpoint{9.320225in}{4.495057in}}%
\pgfusepath{clip}%
\pgfsetbuttcap%
\pgfsetroundjoin%
\pgfsetlinewidth{0.000000pt}%
\definecolor{currentstroke}{rgb}{0.000000,0.000000,0.000000}%
\pgfsetstrokecolor{currentstroke}%
\pgfsetdash{}{0pt}%
\pgfpathmoveto{\pgfqpoint{5.577878in}{0.994507in}}%
\pgfpathlineto{\pgfqpoint{5.764104in}{0.994507in}}%
\pgfpathlineto{\pgfqpoint{5.764104in}{1.076236in}}%
\pgfpathlineto{\pgfqpoint{5.577878in}{1.076236in}}%
\pgfpathlineto{\pgfqpoint{5.577878in}{0.994507in}}%
\pgfusepath{}%
\end{pgfscope}%
\begin{pgfscope}%
\pgfpathrectangle{\pgfqpoint{0.549740in}{0.463273in}}{\pgfqpoint{9.320225in}{4.495057in}}%
\pgfusepath{clip}%
\pgfsetbuttcap%
\pgfsetroundjoin%
\definecolor{currentfill}{rgb}{0.614330,0.761948,0.940009}%
\pgfsetfillcolor{currentfill}%
\pgfsetlinewidth{0.000000pt}%
\definecolor{currentstroke}{rgb}{0.000000,0.000000,0.000000}%
\pgfsetstrokecolor{currentstroke}%
\pgfsetdash{}{0pt}%
\pgfpathmoveto{\pgfqpoint{5.764104in}{0.994507in}}%
\pgfpathlineto{\pgfqpoint{5.950331in}{0.994507in}}%
\pgfpathlineto{\pgfqpoint{5.950331in}{1.076236in}}%
\pgfpathlineto{\pgfqpoint{5.764104in}{1.076236in}}%
\pgfpathlineto{\pgfqpoint{5.764104in}{0.994507in}}%
\pgfusepath{fill}%
\end{pgfscope}%
\begin{pgfscope}%
\pgfpathrectangle{\pgfqpoint{0.549740in}{0.463273in}}{\pgfqpoint{9.320225in}{4.495057in}}%
\pgfusepath{clip}%
\pgfsetbuttcap%
\pgfsetroundjoin%
\definecolor{currentfill}{rgb}{0.472869,0.711325,0.955316}%
\pgfsetfillcolor{currentfill}%
\pgfsetlinewidth{0.000000pt}%
\definecolor{currentstroke}{rgb}{0.000000,0.000000,0.000000}%
\pgfsetstrokecolor{currentstroke}%
\pgfsetdash{}{0pt}%
\pgfpathmoveto{\pgfqpoint{5.950331in}{0.994507in}}%
\pgfpathlineto{\pgfqpoint{6.136557in}{0.994507in}}%
\pgfpathlineto{\pgfqpoint{6.136557in}{1.076236in}}%
\pgfpathlineto{\pgfqpoint{5.950331in}{1.076236in}}%
\pgfpathlineto{\pgfqpoint{5.950331in}{0.994507in}}%
\pgfusepath{fill}%
\end{pgfscope}%
\begin{pgfscope}%
\pgfpathrectangle{\pgfqpoint{0.549740in}{0.463273in}}{\pgfqpoint{9.320225in}{4.495057in}}%
\pgfusepath{clip}%
\pgfsetbuttcap%
\pgfsetroundjoin%
\pgfsetlinewidth{0.000000pt}%
\definecolor{currentstroke}{rgb}{0.000000,0.000000,0.000000}%
\pgfsetstrokecolor{currentstroke}%
\pgfsetdash{}{0pt}%
\pgfpathmoveto{\pgfqpoint{6.136557in}{0.994507in}}%
\pgfpathlineto{\pgfqpoint{6.322784in}{0.994507in}}%
\pgfpathlineto{\pgfqpoint{6.322784in}{1.076236in}}%
\pgfpathlineto{\pgfqpoint{6.136557in}{1.076236in}}%
\pgfpathlineto{\pgfqpoint{6.136557in}{0.994507in}}%
\pgfusepath{}%
\end{pgfscope}%
\begin{pgfscope}%
\pgfpathrectangle{\pgfqpoint{0.549740in}{0.463273in}}{\pgfqpoint{9.320225in}{4.495057in}}%
\pgfusepath{clip}%
\pgfsetbuttcap%
\pgfsetroundjoin%
\pgfsetlinewidth{0.000000pt}%
\definecolor{currentstroke}{rgb}{0.000000,0.000000,0.000000}%
\pgfsetstrokecolor{currentstroke}%
\pgfsetdash{}{0pt}%
\pgfpathmoveto{\pgfqpoint{6.322784in}{0.994507in}}%
\pgfpathlineto{\pgfqpoint{6.509011in}{0.994507in}}%
\pgfpathlineto{\pgfqpoint{6.509011in}{1.076236in}}%
\pgfpathlineto{\pgfqpoint{6.322784in}{1.076236in}}%
\pgfpathlineto{\pgfqpoint{6.322784in}{0.994507in}}%
\pgfusepath{}%
\end{pgfscope}%
\begin{pgfscope}%
\pgfpathrectangle{\pgfqpoint{0.549740in}{0.463273in}}{\pgfqpoint{9.320225in}{4.495057in}}%
\pgfusepath{clip}%
\pgfsetbuttcap%
\pgfsetroundjoin%
\pgfsetlinewidth{0.000000pt}%
\definecolor{currentstroke}{rgb}{0.000000,0.000000,0.000000}%
\pgfsetstrokecolor{currentstroke}%
\pgfsetdash{}{0pt}%
\pgfpathmoveto{\pgfqpoint{6.509011in}{0.994507in}}%
\pgfpathlineto{\pgfqpoint{6.695237in}{0.994507in}}%
\pgfpathlineto{\pgfqpoint{6.695237in}{1.076236in}}%
\pgfpathlineto{\pgfqpoint{6.509011in}{1.076236in}}%
\pgfpathlineto{\pgfqpoint{6.509011in}{0.994507in}}%
\pgfusepath{}%
\end{pgfscope}%
\begin{pgfscope}%
\pgfpathrectangle{\pgfqpoint{0.549740in}{0.463273in}}{\pgfqpoint{9.320225in}{4.495057in}}%
\pgfusepath{clip}%
\pgfsetbuttcap%
\pgfsetroundjoin%
\definecolor{currentfill}{rgb}{0.547810,0.736432,0.947518}%
\pgfsetfillcolor{currentfill}%
\pgfsetlinewidth{0.000000pt}%
\definecolor{currentstroke}{rgb}{0.000000,0.000000,0.000000}%
\pgfsetstrokecolor{currentstroke}%
\pgfsetdash{}{0pt}%
\pgfpathmoveto{\pgfqpoint{6.695237in}{0.994507in}}%
\pgfpathlineto{\pgfqpoint{6.881464in}{0.994507in}}%
\pgfpathlineto{\pgfqpoint{6.881464in}{1.076236in}}%
\pgfpathlineto{\pgfqpoint{6.695237in}{1.076236in}}%
\pgfpathlineto{\pgfqpoint{6.695237in}{0.994507in}}%
\pgfusepath{fill}%
\end{pgfscope}%
\begin{pgfscope}%
\pgfpathrectangle{\pgfqpoint{0.549740in}{0.463273in}}{\pgfqpoint{9.320225in}{4.495057in}}%
\pgfusepath{clip}%
\pgfsetbuttcap%
\pgfsetroundjoin%
\definecolor{currentfill}{rgb}{0.547810,0.736432,0.947518}%
\pgfsetfillcolor{currentfill}%
\pgfsetlinewidth{0.000000pt}%
\definecolor{currentstroke}{rgb}{0.000000,0.000000,0.000000}%
\pgfsetstrokecolor{currentstroke}%
\pgfsetdash{}{0pt}%
\pgfpathmoveto{\pgfqpoint{6.881464in}{0.994507in}}%
\pgfpathlineto{\pgfqpoint{7.067690in}{0.994507in}}%
\pgfpathlineto{\pgfqpoint{7.067690in}{1.076236in}}%
\pgfpathlineto{\pgfqpoint{6.881464in}{1.076236in}}%
\pgfpathlineto{\pgfqpoint{6.881464in}{0.994507in}}%
\pgfusepath{fill}%
\end{pgfscope}%
\begin{pgfscope}%
\pgfpathrectangle{\pgfqpoint{0.549740in}{0.463273in}}{\pgfqpoint{9.320225in}{4.495057in}}%
\pgfusepath{clip}%
\pgfsetbuttcap%
\pgfsetroundjoin%
\pgfsetlinewidth{0.000000pt}%
\definecolor{currentstroke}{rgb}{0.000000,0.000000,0.000000}%
\pgfsetstrokecolor{currentstroke}%
\pgfsetdash{}{0pt}%
\pgfpathmoveto{\pgfqpoint{7.067690in}{0.994507in}}%
\pgfpathlineto{\pgfqpoint{7.253917in}{0.994507in}}%
\pgfpathlineto{\pgfqpoint{7.253917in}{1.076236in}}%
\pgfpathlineto{\pgfqpoint{7.067690in}{1.076236in}}%
\pgfpathlineto{\pgfqpoint{7.067690in}{0.994507in}}%
\pgfusepath{}%
\end{pgfscope}%
\begin{pgfscope}%
\pgfpathrectangle{\pgfqpoint{0.549740in}{0.463273in}}{\pgfqpoint{9.320225in}{4.495057in}}%
\pgfusepath{clip}%
\pgfsetbuttcap%
\pgfsetroundjoin%
\pgfsetlinewidth{0.000000pt}%
\definecolor{currentstroke}{rgb}{0.000000,0.000000,0.000000}%
\pgfsetstrokecolor{currentstroke}%
\pgfsetdash{}{0pt}%
\pgfpathmoveto{\pgfqpoint{7.253917in}{0.994507in}}%
\pgfpathlineto{\pgfqpoint{7.440143in}{0.994507in}}%
\pgfpathlineto{\pgfqpoint{7.440143in}{1.076236in}}%
\pgfpathlineto{\pgfqpoint{7.253917in}{1.076236in}}%
\pgfpathlineto{\pgfqpoint{7.253917in}{0.994507in}}%
\pgfusepath{}%
\end{pgfscope}%
\begin{pgfscope}%
\pgfpathrectangle{\pgfqpoint{0.549740in}{0.463273in}}{\pgfqpoint{9.320225in}{4.495057in}}%
\pgfusepath{clip}%
\pgfsetbuttcap%
\pgfsetroundjoin%
\pgfsetlinewidth{0.000000pt}%
\definecolor{currentstroke}{rgb}{0.000000,0.000000,0.000000}%
\pgfsetstrokecolor{currentstroke}%
\pgfsetdash{}{0pt}%
\pgfpathmoveto{\pgfqpoint{7.440143in}{0.994507in}}%
\pgfpathlineto{\pgfqpoint{7.626370in}{0.994507in}}%
\pgfpathlineto{\pgfqpoint{7.626370in}{1.076236in}}%
\pgfpathlineto{\pgfqpoint{7.440143in}{1.076236in}}%
\pgfpathlineto{\pgfqpoint{7.440143in}{0.994507in}}%
\pgfusepath{}%
\end{pgfscope}%
\begin{pgfscope}%
\pgfpathrectangle{\pgfqpoint{0.549740in}{0.463273in}}{\pgfqpoint{9.320225in}{4.495057in}}%
\pgfusepath{clip}%
\pgfsetbuttcap%
\pgfsetroundjoin%
\pgfsetlinewidth{0.000000pt}%
\definecolor{currentstroke}{rgb}{0.000000,0.000000,0.000000}%
\pgfsetstrokecolor{currentstroke}%
\pgfsetdash{}{0pt}%
\pgfpathmoveto{\pgfqpoint{7.626370in}{0.994507in}}%
\pgfpathlineto{\pgfqpoint{7.812596in}{0.994507in}}%
\pgfpathlineto{\pgfqpoint{7.812596in}{1.076236in}}%
\pgfpathlineto{\pgfqpoint{7.626370in}{1.076236in}}%
\pgfpathlineto{\pgfqpoint{7.626370in}{0.994507in}}%
\pgfusepath{}%
\end{pgfscope}%
\begin{pgfscope}%
\pgfpathrectangle{\pgfqpoint{0.549740in}{0.463273in}}{\pgfqpoint{9.320225in}{4.495057in}}%
\pgfusepath{clip}%
\pgfsetbuttcap%
\pgfsetroundjoin%
\pgfsetlinewidth{0.000000pt}%
\definecolor{currentstroke}{rgb}{0.000000,0.000000,0.000000}%
\pgfsetstrokecolor{currentstroke}%
\pgfsetdash{}{0pt}%
\pgfpathmoveto{\pgfqpoint{7.812596in}{0.994507in}}%
\pgfpathlineto{\pgfqpoint{7.998823in}{0.994507in}}%
\pgfpathlineto{\pgfqpoint{7.998823in}{1.076236in}}%
\pgfpathlineto{\pgfqpoint{7.812596in}{1.076236in}}%
\pgfpathlineto{\pgfqpoint{7.812596in}{0.994507in}}%
\pgfusepath{}%
\end{pgfscope}%
\begin{pgfscope}%
\pgfpathrectangle{\pgfqpoint{0.549740in}{0.463273in}}{\pgfqpoint{9.320225in}{4.495057in}}%
\pgfusepath{clip}%
\pgfsetbuttcap%
\pgfsetroundjoin%
\pgfsetlinewidth{0.000000pt}%
\definecolor{currentstroke}{rgb}{0.000000,0.000000,0.000000}%
\pgfsetstrokecolor{currentstroke}%
\pgfsetdash{}{0pt}%
\pgfpathmoveto{\pgfqpoint{7.998823in}{0.994507in}}%
\pgfpathlineto{\pgfqpoint{8.185049in}{0.994507in}}%
\pgfpathlineto{\pgfqpoint{8.185049in}{1.076236in}}%
\pgfpathlineto{\pgfqpoint{7.998823in}{1.076236in}}%
\pgfpathlineto{\pgfqpoint{7.998823in}{0.994507in}}%
\pgfusepath{}%
\end{pgfscope}%
\begin{pgfscope}%
\pgfpathrectangle{\pgfqpoint{0.549740in}{0.463273in}}{\pgfqpoint{9.320225in}{4.495057in}}%
\pgfusepath{clip}%
\pgfsetbuttcap%
\pgfsetroundjoin%
\pgfsetlinewidth{0.000000pt}%
\definecolor{currentstroke}{rgb}{0.000000,0.000000,0.000000}%
\pgfsetstrokecolor{currentstroke}%
\pgfsetdash{}{0pt}%
\pgfpathmoveto{\pgfqpoint{8.185049in}{0.994507in}}%
\pgfpathlineto{\pgfqpoint{8.371276in}{0.994507in}}%
\pgfpathlineto{\pgfqpoint{8.371276in}{1.076236in}}%
\pgfpathlineto{\pgfqpoint{8.185049in}{1.076236in}}%
\pgfpathlineto{\pgfqpoint{8.185049in}{0.994507in}}%
\pgfusepath{}%
\end{pgfscope}%
\begin{pgfscope}%
\pgfpathrectangle{\pgfqpoint{0.549740in}{0.463273in}}{\pgfqpoint{9.320225in}{4.495057in}}%
\pgfusepath{clip}%
\pgfsetbuttcap%
\pgfsetroundjoin%
\pgfsetlinewidth{0.000000pt}%
\definecolor{currentstroke}{rgb}{0.000000,0.000000,0.000000}%
\pgfsetstrokecolor{currentstroke}%
\pgfsetdash{}{0pt}%
\pgfpathmoveto{\pgfqpoint{8.371276in}{0.994507in}}%
\pgfpathlineto{\pgfqpoint{8.557503in}{0.994507in}}%
\pgfpathlineto{\pgfqpoint{8.557503in}{1.076236in}}%
\pgfpathlineto{\pgfqpoint{8.371276in}{1.076236in}}%
\pgfpathlineto{\pgfqpoint{8.371276in}{0.994507in}}%
\pgfusepath{}%
\end{pgfscope}%
\begin{pgfscope}%
\pgfpathrectangle{\pgfqpoint{0.549740in}{0.463273in}}{\pgfqpoint{9.320225in}{4.495057in}}%
\pgfusepath{clip}%
\pgfsetbuttcap%
\pgfsetroundjoin%
\pgfsetlinewidth{0.000000pt}%
\definecolor{currentstroke}{rgb}{0.000000,0.000000,0.000000}%
\pgfsetstrokecolor{currentstroke}%
\pgfsetdash{}{0pt}%
\pgfpathmoveto{\pgfqpoint{8.557503in}{0.994507in}}%
\pgfpathlineto{\pgfqpoint{8.743729in}{0.994507in}}%
\pgfpathlineto{\pgfqpoint{8.743729in}{1.076236in}}%
\pgfpathlineto{\pgfqpoint{8.557503in}{1.076236in}}%
\pgfpathlineto{\pgfqpoint{8.557503in}{0.994507in}}%
\pgfusepath{}%
\end{pgfscope}%
\begin{pgfscope}%
\pgfpathrectangle{\pgfqpoint{0.549740in}{0.463273in}}{\pgfqpoint{9.320225in}{4.495057in}}%
\pgfusepath{clip}%
\pgfsetbuttcap%
\pgfsetroundjoin%
\pgfsetlinewidth{0.000000pt}%
\definecolor{currentstroke}{rgb}{0.000000,0.000000,0.000000}%
\pgfsetstrokecolor{currentstroke}%
\pgfsetdash{}{0pt}%
\pgfpathmoveto{\pgfqpoint{8.743729in}{0.994507in}}%
\pgfpathlineto{\pgfqpoint{8.929956in}{0.994507in}}%
\pgfpathlineto{\pgfqpoint{8.929956in}{1.076236in}}%
\pgfpathlineto{\pgfqpoint{8.743729in}{1.076236in}}%
\pgfpathlineto{\pgfqpoint{8.743729in}{0.994507in}}%
\pgfusepath{}%
\end{pgfscope}%
\begin{pgfscope}%
\pgfpathrectangle{\pgfqpoint{0.549740in}{0.463273in}}{\pgfqpoint{9.320225in}{4.495057in}}%
\pgfusepath{clip}%
\pgfsetbuttcap%
\pgfsetroundjoin%
\pgfsetlinewidth{0.000000pt}%
\definecolor{currentstroke}{rgb}{0.000000,0.000000,0.000000}%
\pgfsetstrokecolor{currentstroke}%
\pgfsetdash{}{0pt}%
\pgfpathmoveto{\pgfqpoint{8.929956in}{0.994507in}}%
\pgfpathlineto{\pgfqpoint{9.116182in}{0.994507in}}%
\pgfpathlineto{\pgfqpoint{9.116182in}{1.076236in}}%
\pgfpathlineto{\pgfqpoint{8.929956in}{1.076236in}}%
\pgfpathlineto{\pgfqpoint{8.929956in}{0.994507in}}%
\pgfusepath{}%
\end{pgfscope}%
\begin{pgfscope}%
\pgfpathrectangle{\pgfqpoint{0.549740in}{0.463273in}}{\pgfqpoint{9.320225in}{4.495057in}}%
\pgfusepath{clip}%
\pgfsetbuttcap%
\pgfsetroundjoin%
\pgfsetlinewidth{0.000000pt}%
\definecolor{currentstroke}{rgb}{0.000000,0.000000,0.000000}%
\pgfsetstrokecolor{currentstroke}%
\pgfsetdash{}{0pt}%
\pgfpathmoveto{\pgfqpoint{9.116182in}{0.994507in}}%
\pgfpathlineto{\pgfqpoint{9.302409in}{0.994507in}}%
\pgfpathlineto{\pgfqpoint{9.302409in}{1.076236in}}%
\pgfpathlineto{\pgfqpoint{9.116182in}{1.076236in}}%
\pgfpathlineto{\pgfqpoint{9.116182in}{0.994507in}}%
\pgfusepath{}%
\end{pgfscope}%
\begin{pgfscope}%
\pgfpathrectangle{\pgfqpoint{0.549740in}{0.463273in}}{\pgfqpoint{9.320225in}{4.495057in}}%
\pgfusepath{clip}%
\pgfsetbuttcap%
\pgfsetroundjoin%
\pgfsetlinewidth{0.000000pt}%
\definecolor{currentstroke}{rgb}{0.000000,0.000000,0.000000}%
\pgfsetstrokecolor{currentstroke}%
\pgfsetdash{}{0pt}%
\pgfpathmoveto{\pgfqpoint{9.302409in}{0.994507in}}%
\pgfpathlineto{\pgfqpoint{9.488635in}{0.994507in}}%
\pgfpathlineto{\pgfqpoint{9.488635in}{1.076236in}}%
\pgfpathlineto{\pgfqpoint{9.302409in}{1.076236in}}%
\pgfpathlineto{\pgfqpoint{9.302409in}{0.994507in}}%
\pgfusepath{}%
\end{pgfscope}%
\begin{pgfscope}%
\pgfpathrectangle{\pgfqpoint{0.549740in}{0.463273in}}{\pgfqpoint{9.320225in}{4.495057in}}%
\pgfusepath{clip}%
\pgfsetbuttcap%
\pgfsetroundjoin%
\pgfsetlinewidth{0.000000pt}%
\definecolor{currentstroke}{rgb}{0.000000,0.000000,0.000000}%
\pgfsetstrokecolor{currentstroke}%
\pgfsetdash{}{0pt}%
\pgfpathmoveto{\pgfqpoint{9.488635in}{0.994507in}}%
\pgfpathlineto{\pgfqpoint{9.674862in}{0.994507in}}%
\pgfpathlineto{\pgfqpoint{9.674862in}{1.076236in}}%
\pgfpathlineto{\pgfqpoint{9.488635in}{1.076236in}}%
\pgfpathlineto{\pgfqpoint{9.488635in}{0.994507in}}%
\pgfusepath{}%
\end{pgfscope}%
\begin{pgfscope}%
\pgfpathrectangle{\pgfqpoint{0.549740in}{0.463273in}}{\pgfqpoint{9.320225in}{4.495057in}}%
\pgfusepath{clip}%
\pgfsetbuttcap%
\pgfsetroundjoin%
\pgfsetlinewidth{0.000000pt}%
\definecolor{currentstroke}{rgb}{0.000000,0.000000,0.000000}%
\pgfsetstrokecolor{currentstroke}%
\pgfsetdash{}{0pt}%
\pgfpathmoveto{\pgfqpoint{9.674862in}{0.994507in}}%
\pgfpathlineto{\pgfqpoint{9.861088in}{0.994507in}}%
\pgfpathlineto{\pgfqpoint{9.861088in}{1.076236in}}%
\pgfpathlineto{\pgfqpoint{9.674862in}{1.076236in}}%
\pgfpathlineto{\pgfqpoint{9.674862in}{0.994507in}}%
\pgfusepath{}%
\end{pgfscope}%
\begin{pgfscope}%
\pgfpathrectangle{\pgfqpoint{0.549740in}{0.463273in}}{\pgfqpoint{9.320225in}{4.495057in}}%
\pgfusepath{clip}%
\pgfsetbuttcap%
\pgfsetroundjoin%
\pgfsetlinewidth{0.000000pt}%
\definecolor{currentstroke}{rgb}{0.000000,0.000000,0.000000}%
\pgfsetstrokecolor{currentstroke}%
\pgfsetdash{}{0pt}%
\pgfpathmoveto{\pgfqpoint{0.549761in}{1.076236in}}%
\pgfpathlineto{\pgfqpoint{0.735988in}{1.076236in}}%
\pgfpathlineto{\pgfqpoint{0.735988in}{1.157964in}}%
\pgfpathlineto{\pgfqpoint{0.549761in}{1.157964in}}%
\pgfpathlineto{\pgfqpoint{0.549761in}{1.076236in}}%
\pgfusepath{}%
\end{pgfscope}%
\begin{pgfscope}%
\pgfpathrectangle{\pgfqpoint{0.549740in}{0.463273in}}{\pgfqpoint{9.320225in}{4.495057in}}%
\pgfusepath{clip}%
\pgfsetbuttcap%
\pgfsetroundjoin%
\definecolor{currentfill}{rgb}{0.547810,0.736432,0.947518}%
\pgfsetfillcolor{currentfill}%
\pgfsetlinewidth{0.000000pt}%
\definecolor{currentstroke}{rgb}{0.000000,0.000000,0.000000}%
\pgfsetstrokecolor{currentstroke}%
\pgfsetdash{}{0pt}%
\pgfpathmoveto{\pgfqpoint{0.735988in}{1.076236in}}%
\pgfpathlineto{\pgfqpoint{0.922214in}{1.076236in}}%
\pgfpathlineto{\pgfqpoint{0.922214in}{1.157964in}}%
\pgfpathlineto{\pgfqpoint{0.735988in}{1.157964in}}%
\pgfpathlineto{\pgfqpoint{0.735988in}{1.076236in}}%
\pgfusepath{fill}%
\end{pgfscope}%
\begin{pgfscope}%
\pgfpathrectangle{\pgfqpoint{0.549740in}{0.463273in}}{\pgfqpoint{9.320225in}{4.495057in}}%
\pgfusepath{clip}%
\pgfsetbuttcap%
\pgfsetroundjoin%
\pgfsetlinewidth{0.000000pt}%
\definecolor{currentstroke}{rgb}{0.000000,0.000000,0.000000}%
\pgfsetstrokecolor{currentstroke}%
\pgfsetdash{}{0pt}%
\pgfpathmoveto{\pgfqpoint{0.922214in}{1.076236in}}%
\pgfpathlineto{\pgfqpoint{1.108441in}{1.076236in}}%
\pgfpathlineto{\pgfqpoint{1.108441in}{1.157964in}}%
\pgfpathlineto{\pgfqpoint{0.922214in}{1.157964in}}%
\pgfpathlineto{\pgfqpoint{0.922214in}{1.076236in}}%
\pgfusepath{}%
\end{pgfscope}%
\begin{pgfscope}%
\pgfpathrectangle{\pgfqpoint{0.549740in}{0.463273in}}{\pgfqpoint{9.320225in}{4.495057in}}%
\pgfusepath{clip}%
\pgfsetbuttcap%
\pgfsetroundjoin%
\pgfsetlinewidth{0.000000pt}%
\definecolor{currentstroke}{rgb}{0.000000,0.000000,0.000000}%
\pgfsetstrokecolor{currentstroke}%
\pgfsetdash{}{0pt}%
\pgfpathmoveto{\pgfqpoint{1.108441in}{1.076236in}}%
\pgfpathlineto{\pgfqpoint{1.294667in}{1.076236in}}%
\pgfpathlineto{\pgfqpoint{1.294667in}{1.157964in}}%
\pgfpathlineto{\pgfqpoint{1.108441in}{1.157964in}}%
\pgfpathlineto{\pgfqpoint{1.108441in}{1.076236in}}%
\pgfusepath{}%
\end{pgfscope}%
\begin{pgfscope}%
\pgfpathrectangle{\pgfqpoint{0.549740in}{0.463273in}}{\pgfqpoint{9.320225in}{4.495057in}}%
\pgfusepath{clip}%
\pgfsetbuttcap%
\pgfsetroundjoin%
\pgfsetlinewidth{0.000000pt}%
\definecolor{currentstroke}{rgb}{0.000000,0.000000,0.000000}%
\pgfsetstrokecolor{currentstroke}%
\pgfsetdash{}{0pt}%
\pgfpathmoveto{\pgfqpoint{1.294667in}{1.076236in}}%
\pgfpathlineto{\pgfqpoint{1.480894in}{1.076236in}}%
\pgfpathlineto{\pgfqpoint{1.480894in}{1.157964in}}%
\pgfpathlineto{\pgfqpoint{1.294667in}{1.157964in}}%
\pgfpathlineto{\pgfqpoint{1.294667in}{1.076236in}}%
\pgfusepath{}%
\end{pgfscope}%
\begin{pgfscope}%
\pgfpathrectangle{\pgfqpoint{0.549740in}{0.463273in}}{\pgfqpoint{9.320225in}{4.495057in}}%
\pgfusepath{clip}%
\pgfsetbuttcap%
\pgfsetroundjoin%
\pgfsetlinewidth{0.000000pt}%
\definecolor{currentstroke}{rgb}{0.000000,0.000000,0.000000}%
\pgfsetstrokecolor{currentstroke}%
\pgfsetdash{}{0pt}%
\pgfpathmoveto{\pgfqpoint{1.480894in}{1.076236in}}%
\pgfpathlineto{\pgfqpoint{1.667120in}{1.076236in}}%
\pgfpathlineto{\pgfqpoint{1.667120in}{1.157964in}}%
\pgfpathlineto{\pgfqpoint{1.480894in}{1.157964in}}%
\pgfpathlineto{\pgfqpoint{1.480894in}{1.076236in}}%
\pgfusepath{}%
\end{pgfscope}%
\begin{pgfscope}%
\pgfpathrectangle{\pgfqpoint{0.549740in}{0.463273in}}{\pgfqpoint{9.320225in}{4.495057in}}%
\pgfusepath{clip}%
\pgfsetbuttcap%
\pgfsetroundjoin%
\pgfsetlinewidth{0.000000pt}%
\definecolor{currentstroke}{rgb}{0.000000,0.000000,0.000000}%
\pgfsetstrokecolor{currentstroke}%
\pgfsetdash{}{0pt}%
\pgfpathmoveto{\pgfqpoint{1.667120in}{1.076236in}}%
\pgfpathlineto{\pgfqpoint{1.853347in}{1.076236in}}%
\pgfpathlineto{\pgfqpoint{1.853347in}{1.157964in}}%
\pgfpathlineto{\pgfqpoint{1.667120in}{1.157964in}}%
\pgfpathlineto{\pgfqpoint{1.667120in}{1.076236in}}%
\pgfusepath{}%
\end{pgfscope}%
\begin{pgfscope}%
\pgfpathrectangle{\pgfqpoint{0.549740in}{0.463273in}}{\pgfqpoint{9.320225in}{4.495057in}}%
\pgfusepath{clip}%
\pgfsetbuttcap%
\pgfsetroundjoin%
\pgfsetlinewidth{0.000000pt}%
\definecolor{currentstroke}{rgb}{0.000000,0.000000,0.000000}%
\pgfsetstrokecolor{currentstroke}%
\pgfsetdash{}{0pt}%
\pgfpathmoveto{\pgfqpoint{1.853347in}{1.076236in}}%
\pgfpathlineto{\pgfqpoint{2.039573in}{1.076236in}}%
\pgfpathlineto{\pgfqpoint{2.039573in}{1.157964in}}%
\pgfpathlineto{\pgfqpoint{1.853347in}{1.157964in}}%
\pgfpathlineto{\pgfqpoint{1.853347in}{1.076236in}}%
\pgfusepath{}%
\end{pgfscope}%
\begin{pgfscope}%
\pgfpathrectangle{\pgfqpoint{0.549740in}{0.463273in}}{\pgfqpoint{9.320225in}{4.495057in}}%
\pgfusepath{clip}%
\pgfsetbuttcap%
\pgfsetroundjoin%
\pgfsetlinewidth{0.000000pt}%
\definecolor{currentstroke}{rgb}{0.000000,0.000000,0.000000}%
\pgfsetstrokecolor{currentstroke}%
\pgfsetdash{}{0pt}%
\pgfpathmoveto{\pgfqpoint{2.039573in}{1.076236in}}%
\pgfpathlineto{\pgfqpoint{2.225800in}{1.076236in}}%
\pgfpathlineto{\pgfqpoint{2.225800in}{1.157964in}}%
\pgfpathlineto{\pgfqpoint{2.039573in}{1.157964in}}%
\pgfpathlineto{\pgfqpoint{2.039573in}{1.076236in}}%
\pgfusepath{}%
\end{pgfscope}%
\begin{pgfscope}%
\pgfpathrectangle{\pgfqpoint{0.549740in}{0.463273in}}{\pgfqpoint{9.320225in}{4.495057in}}%
\pgfusepath{clip}%
\pgfsetbuttcap%
\pgfsetroundjoin%
\pgfsetlinewidth{0.000000pt}%
\definecolor{currentstroke}{rgb}{0.000000,0.000000,0.000000}%
\pgfsetstrokecolor{currentstroke}%
\pgfsetdash{}{0pt}%
\pgfpathmoveto{\pgfqpoint{2.225800in}{1.076236in}}%
\pgfpathlineto{\pgfqpoint{2.412027in}{1.076236in}}%
\pgfpathlineto{\pgfqpoint{2.412027in}{1.157964in}}%
\pgfpathlineto{\pgfqpoint{2.225800in}{1.157964in}}%
\pgfpathlineto{\pgfqpoint{2.225800in}{1.076236in}}%
\pgfusepath{}%
\end{pgfscope}%
\begin{pgfscope}%
\pgfpathrectangle{\pgfqpoint{0.549740in}{0.463273in}}{\pgfqpoint{9.320225in}{4.495057in}}%
\pgfusepath{clip}%
\pgfsetbuttcap%
\pgfsetroundjoin%
\pgfsetlinewidth{0.000000pt}%
\definecolor{currentstroke}{rgb}{0.000000,0.000000,0.000000}%
\pgfsetstrokecolor{currentstroke}%
\pgfsetdash{}{0pt}%
\pgfpathmoveto{\pgfqpoint{2.412027in}{1.076236in}}%
\pgfpathlineto{\pgfqpoint{2.598253in}{1.076236in}}%
\pgfpathlineto{\pgfqpoint{2.598253in}{1.157964in}}%
\pgfpathlineto{\pgfqpoint{2.412027in}{1.157964in}}%
\pgfpathlineto{\pgfqpoint{2.412027in}{1.076236in}}%
\pgfusepath{}%
\end{pgfscope}%
\begin{pgfscope}%
\pgfpathrectangle{\pgfqpoint{0.549740in}{0.463273in}}{\pgfqpoint{9.320225in}{4.495057in}}%
\pgfusepath{clip}%
\pgfsetbuttcap%
\pgfsetroundjoin%
\pgfsetlinewidth{0.000000pt}%
\definecolor{currentstroke}{rgb}{0.000000,0.000000,0.000000}%
\pgfsetstrokecolor{currentstroke}%
\pgfsetdash{}{0pt}%
\pgfpathmoveto{\pgfqpoint{2.598253in}{1.076236in}}%
\pgfpathlineto{\pgfqpoint{2.784480in}{1.076236in}}%
\pgfpathlineto{\pgfqpoint{2.784480in}{1.157964in}}%
\pgfpathlineto{\pgfqpoint{2.598253in}{1.157964in}}%
\pgfpathlineto{\pgfqpoint{2.598253in}{1.076236in}}%
\pgfusepath{}%
\end{pgfscope}%
\begin{pgfscope}%
\pgfpathrectangle{\pgfqpoint{0.549740in}{0.463273in}}{\pgfqpoint{9.320225in}{4.495057in}}%
\pgfusepath{clip}%
\pgfsetbuttcap%
\pgfsetroundjoin%
\pgfsetlinewidth{0.000000pt}%
\definecolor{currentstroke}{rgb}{0.000000,0.000000,0.000000}%
\pgfsetstrokecolor{currentstroke}%
\pgfsetdash{}{0pt}%
\pgfpathmoveto{\pgfqpoint{2.784480in}{1.076236in}}%
\pgfpathlineto{\pgfqpoint{2.970706in}{1.076236in}}%
\pgfpathlineto{\pgfqpoint{2.970706in}{1.157964in}}%
\pgfpathlineto{\pgfqpoint{2.784480in}{1.157964in}}%
\pgfpathlineto{\pgfqpoint{2.784480in}{1.076236in}}%
\pgfusepath{}%
\end{pgfscope}%
\begin{pgfscope}%
\pgfpathrectangle{\pgfqpoint{0.549740in}{0.463273in}}{\pgfqpoint{9.320225in}{4.495057in}}%
\pgfusepath{clip}%
\pgfsetbuttcap%
\pgfsetroundjoin%
\pgfsetlinewidth{0.000000pt}%
\definecolor{currentstroke}{rgb}{0.000000,0.000000,0.000000}%
\pgfsetstrokecolor{currentstroke}%
\pgfsetdash{}{0pt}%
\pgfpathmoveto{\pgfqpoint{2.970706in}{1.076236in}}%
\pgfpathlineto{\pgfqpoint{3.156933in}{1.076236in}}%
\pgfpathlineto{\pgfqpoint{3.156933in}{1.157964in}}%
\pgfpathlineto{\pgfqpoint{2.970706in}{1.157964in}}%
\pgfpathlineto{\pgfqpoint{2.970706in}{1.076236in}}%
\pgfusepath{}%
\end{pgfscope}%
\begin{pgfscope}%
\pgfpathrectangle{\pgfqpoint{0.549740in}{0.463273in}}{\pgfqpoint{9.320225in}{4.495057in}}%
\pgfusepath{clip}%
\pgfsetbuttcap%
\pgfsetroundjoin%
\pgfsetlinewidth{0.000000pt}%
\definecolor{currentstroke}{rgb}{0.000000,0.000000,0.000000}%
\pgfsetstrokecolor{currentstroke}%
\pgfsetdash{}{0pt}%
\pgfpathmoveto{\pgfqpoint{3.156933in}{1.076236in}}%
\pgfpathlineto{\pgfqpoint{3.343159in}{1.076236in}}%
\pgfpathlineto{\pgfqpoint{3.343159in}{1.157964in}}%
\pgfpathlineto{\pgfqpoint{3.156933in}{1.157964in}}%
\pgfpathlineto{\pgfqpoint{3.156933in}{1.076236in}}%
\pgfusepath{}%
\end{pgfscope}%
\begin{pgfscope}%
\pgfpathrectangle{\pgfqpoint{0.549740in}{0.463273in}}{\pgfqpoint{9.320225in}{4.495057in}}%
\pgfusepath{clip}%
\pgfsetbuttcap%
\pgfsetroundjoin%
\pgfsetlinewidth{0.000000pt}%
\definecolor{currentstroke}{rgb}{0.000000,0.000000,0.000000}%
\pgfsetstrokecolor{currentstroke}%
\pgfsetdash{}{0pt}%
\pgfpathmoveto{\pgfqpoint{3.343159in}{1.076236in}}%
\pgfpathlineto{\pgfqpoint{3.529386in}{1.076236in}}%
\pgfpathlineto{\pgfqpoint{3.529386in}{1.157964in}}%
\pgfpathlineto{\pgfqpoint{3.343159in}{1.157964in}}%
\pgfpathlineto{\pgfqpoint{3.343159in}{1.076236in}}%
\pgfusepath{}%
\end{pgfscope}%
\begin{pgfscope}%
\pgfpathrectangle{\pgfqpoint{0.549740in}{0.463273in}}{\pgfqpoint{9.320225in}{4.495057in}}%
\pgfusepath{clip}%
\pgfsetbuttcap%
\pgfsetroundjoin%
\pgfsetlinewidth{0.000000pt}%
\definecolor{currentstroke}{rgb}{0.000000,0.000000,0.000000}%
\pgfsetstrokecolor{currentstroke}%
\pgfsetdash{}{0pt}%
\pgfpathmoveto{\pgfqpoint{3.529386in}{1.076236in}}%
\pgfpathlineto{\pgfqpoint{3.715612in}{1.076236in}}%
\pgfpathlineto{\pgfqpoint{3.715612in}{1.157964in}}%
\pgfpathlineto{\pgfqpoint{3.529386in}{1.157964in}}%
\pgfpathlineto{\pgfqpoint{3.529386in}{1.076236in}}%
\pgfusepath{}%
\end{pgfscope}%
\begin{pgfscope}%
\pgfpathrectangle{\pgfqpoint{0.549740in}{0.463273in}}{\pgfqpoint{9.320225in}{4.495057in}}%
\pgfusepath{clip}%
\pgfsetbuttcap%
\pgfsetroundjoin%
\pgfsetlinewidth{0.000000pt}%
\definecolor{currentstroke}{rgb}{0.000000,0.000000,0.000000}%
\pgfsetstrokecolor{currentstroke}%
\pgfsetdash{}{0pt}%
\pgfpathmoveto{\pgfqpoint{3.715612in}{1.076236in}}%
\pgfpathlineto{\pgfqpoint{3.901839in}{1.076236in}}%
\pgfpathlineto{\pgfqpoint{3.901839in}{1.157964in}}%
\pgfpathlineto{\pgfqpoint{3.715612in}{1.157964in}}%
\pgfpathlineto{\pgfqpoint{3.715612in}{1.076236in}}%
\pgfusepath{}%
\end{pgfscope}%
\begin{pgfscope}%
\pgfpathrectangle{\pgfqpoint{0.549740in}{0.463273in}}{\pgfqpoint{9.320225in}{4.495057in}}%
\pgfusepath{clip}%
\pgfsetbuttcap%
\pgfsetroundjoin%
\pgfsetlinewidth{0.000000pt}%
\definecolor{currentstroke}{rgb}{0.000000,0.000000,0.000000}%
\pgfsetstrokecolor{currentstroke}%
\pgfsetdash{}{0pt}%
\pgfpathmoveto{\pgfqpoint{3.901839in}{1.076236in}}%
\pgfpathlineto{\pgfqpoint{4.088065in}{1.076236in}}%
\pgfpathlineto{\pgfqpoint{4.088065in}{1.157964in}}%
\pgfpathlineto{\pgfqpoint{3.901839in}{1.157964in}}%
\pgfpathlineto{\pgfqpoint{3.901839in}{1.076236in}}%
\pgfusepath{}%
\end{pgfscope}%
\begin{pgfscope}%
\pgfpathrectangle{\pgfqpoint{0.549740in}{0.463273in}}{\pgfqpoint{9.320225in}{4.495057in}}%
\pgfusepath{clip}%
\pgfsetbuttcap%
\pgfsetroundjoin%
\pgfsetlinewidth{0.000000pt}%
\definecolor{currentstroke}{rgb}{0.000000,0.000000,0.000000}%
\pgfsetstrokecolor{currentstroke}%
\pgfsetdash{}{0pt}%
\pgfpathmoveto{\pgfqpoint{4.088065in}{1.076236in}}%
\pgfpathlineto{\pgfqpoint{4.274292in}{1.076236in}}%
\pgfpathlineto{\pgfqpoint{4.274292in}{1.157964in}}%
\pgfpathlineto{\pgfqpoint{4.088065in}{1.157964in}}%
\pgfpathlineto{\pgfqpoint{4.088065in}{1.076236in}}%
\pgfusepath{}%
\end{pgfscope}%
\begin{pgfscope}%
\pgfpathrectangle{\pgfqpoint{0.549740in}{0.463273in}}{\pgfqpoint{9.320225in}{4.495057in}}%
\pgfusepath{clip}%
\pgfsetbuttcap%
\pgfsetroundjoin%
\pgfsetlinewidth{0.000000pt}%
\definecolor{currentstroke}{rgb}{0.000000,0.000000,0.000000}%
\pgfsetstrokecolor{currentstroke}%
\pgfsetdash{}{0pt}%
\pgfpathmoveto{\pgfqpoint{4.274292in}{1.076236in}}%
\pgfpathlineto{\pgfqpoint{4.460519in}{1.076236in}}%
\pgfpathlineto{\pgfqpoint{4.460519in}{1.157964in}}%
\pgfpathlineto{\pgfqpoint{4.274292in}{1.157964in}}%
\pgfpathlineto{\pgfqpoint{4.274292in}{1.076236in}}%
\pgfusepath{}%
\end{pgfscope}%
\begin{pgfscope}%
\pgfpathrectangle{\pgfqpoint{0.549740in}{0.463273in}}{\pgfqpoint{9.320225in}{4.495057in}}%
\pgfusepath{clip}%
\pgfsetbuttcap%
\pgfsetroundjoin%
\pgfsetlinewidth{0.000000pt}%
\definecolor{currentstroke}{rgb}{0.000000,0.000000,0.000000}%
\pgfsetstrokecolor{currentstroke}%
\pgfsetdash{}{0pt}%
\pgfpathmoveto{\pgfqpoint{4.460519in}{1.076236in}}%
\pgfpathlineto{\pgfqpoint{4.646745in}{1.076236in}}%
\pgfpathlineto{\pgfqpoint{4.646745in}{1.157964in}}%
\pgfpathlineto{\pgfqpoint{4.460519in}{1.157964in}}%
\pgfpathlineto{\pgfqpoint{4.460519in}{1.076236in}}%
\pgfusepath{}%
\end{pgfscope}%
\begin{pgfscope}%
\pgfpathrectangle{\pgfqpoint{0.549740in}{0.463273in}}{\pgfqpoint{9.320225in}{4.495057in}}%
\pgfusepath{clip}%
\pgfsetbuttcap%
\pgfsetroundjoin%
\pgfsetlinewidth{0.000000pt}%
\definecolor{currentstroke}{rgb}{0.000000,0.000000,0.000000}%
\pgfsetstrokecolor{currentstroke}%
\pgfsetdash{}{0pt}%
\pgfpathmoveto{\pgfqpoint{4.646745in}{1.076236in}}%
\pgfpathlineto{\pgfqpoint{4.832972in}{1.076236in}}%
\pgfpathlineto{\pgfqpoint{4.832972in}{1.157964in}}%
\pgfpathlineto{\pgfqpoint{4.646745in}{1.157964in}}%
\pgfpathlineto{\pgfqpoint{4.646745in}{1.076236in}}%
\pgfusepath{}%
\end{pgfscope}%
\begin{pgfscope}%
\pgfpathrectangle{\pgfqpoint{0.549740in}{0.463273in}}{\pgfqpoint{9.320225in}{4.495057in}}%
\pgfusepath{clip}%
\pgfsetbuttcap%
\pgfsetroundjoin%
\pgfsetlinewidth{0.000000pt}%
\definecolor{currentstroke}{rgb}{0.000000,0.000000,0.000000}%
\pgfsetstrokecolor{currentstroke}%
\pgfsetdash{}{0pt}%
\pgfpathmoveto{\pgfqpoint{4.832972in}{1.076236in}}%
\pgfpathlineto{\pgfqpoint{5.019198in}{1.076236in}}%
\pgfpathlineto{\pgfqpoint{5.019198in}{1.157964in}}%
\pgfpathlineto{\pgfqpoint{4.832972in}{1.157964in}}%
\pgfpathlineto{\pgfqpoint{4.832972in}{1.076236in}}%
\pgfusepath{}%
\end{pgfscope}%
\begin{pgfscope}%
\pgfpathrectangle{\pgfqpoint{0.549740in}{0.463273in}}{\pgfqpoint{9.320225in}{4.495057in}}%
\pgfusepath{clip}%
\pgfsetbuttcap%
\pgfsetroundjoin%
\pgfsetlinewidth{0.000000pt}%
\definecolor{currentstroke}{rgb}{0.000000,0.000000,0.000000}%
\pgfsetstrokecolor{currentstroke}%
\pgfsetdash{}{0pt}%
\pgfpathmoveto{\pgfqpoint{5.019198in}{1.076236in}}%
\pgfpathlineto{\pgfqpoint{5.205425in}{1.076236in}}%
\pgfpathlineto{\pgfqpoint{5.205425in}{1.157964in}}%
\pgfpathlineto{\pgfqpoint{5.019198in}{1.157964in}}%
\pgfpathlineto{\pgfqpoint{5.019198in}{1.076236in}}%
\pgfusepath{}%
\end{pgfscope}%
\begin{pgfscope}%
\pgfpathrectangle{\pgfqpoint{0.549740in}{0.463273in}}{\pgfqpoint{9.320225in}{4.495057in}}%
\pgfusepath{clip}%
\pgfsetbuttcap%
\pgfsetroundjoin%
\pgfsetlinewidth{0.000000pt}%
\definecolor{currentstroke}{rgb}{0.000000,0.000000,0.000000}%
\pgfsetstrokecolor{currentstroke}%
\pgfsetdash{}{0pt}%
\pgfpathmoveto{\pgfqpoint{5.205425in}{1.076236in}}%
\pgfpathlineto{\pgfqpoint{5.391651in}{1.076236in}}%
\pgfpathlineto{\pgfqpoint{5.391651in}{1.157964in}}%
\pgfpathlineto{\pgfqpoint{5.205425in}{1.157964in}}%
\pgfpathlineto{\pgfqpoint{5.205425in}{1.076236in}}%
\pgfusepath{}%
\end{pgfscope}%
\begin{pgfscope}%
\pgfpathrectangle{\pgfqpoint{0.549740in}{0.463273in}}{\pgfqpoint{9.320225in}{4.495057in}}%
\pgfusepath{clip}%
\pgfsetbuttcap%
\pgfsetroundjoin%
\pgfsetlinewidth{0.000000pt}%
\definecolor{currentstroke}{rgb}{0.000000,0.000000,0.000000}%
\pgfsetstrokecolor{currentstroke}%
\pgfsetdash{}{0pt}%
\pgfpathmoveto{\pgfqpoint{5.391651in}{1.076236in}}%
\pgfpathlineto{\pgfqpoint{5.577878in}{1.076236in}}%
\pgfpathlineto{\pgfqpoint{5.577878in}{1.157964in}}%
\pgfpathlineto{\pgfqpoint{5.391651in}{1.157964in}}%
\pgfpathlineto{\pgfqpoint{5.391651in}{1.076236in}}%
\pgfusepath{}%
\end{pgfscope}%
\begin{pgfscope}%
\pgfpathrectangle{\pgfqpoint{0.549740in}{0.463273in}}{\pgfqpoint{9.320225in}{4.495057in}}%
\pgfusepath{clip}%
\pgfsetbuttcap%
\pgfsetroundjoin%
\pgfsetlinewidth{0.000000pt}%
\definecolor{currentstroke}{rgb}{0.000000,0.000000,0.000000}%
\pgfsetstrokecolor{currentstroke}%
\pgfsetdash{}{0pt}%
\pgfpathmoveto{\pgfqpoint{5.577878in}{1.076236in}}%
\pgfpathlineto{\pgfqpoint{5.764104in}{1.076236in}}%
\pgfpathlineto{\pgfqpoint{5.764104in}{1.157964in}}%
\pgfpathlineto{\pgfqpoint{5.577878in}{1.157964in}}%
\pgfpathlineto{\pgfqpoint{5.577878in}{1.076236in}}%
\pgfusepath{}%
\end{pgfscope}%
\begin{pgfscope}%
\pgfpathrectangle{\pgfqpoint{0.549740in}{0.463273in}}{\pgfqpoint{9.320225in}{4.495057in}}%
\pgfusepath{clip}%
\pgfsetbuttcap%
\pgfsetroundjoin%
\definecolor{currentfill}{rgb}{0.472869,0.711325,0.955316}%
\pgfsetfillcolor{currentfill}%
\pgfsetlinewidth{0.000000pt}%
\definecolor{currentstroke}{rgb}{0.000000,0.000000,0.000000}%
\pgfsetstrokecolor{currentstroke}%
\pgfsetdash{}{0pt}%
\pgfpathmoveto{\pgfqpoint{5.764104in}{1.076236in}}%
\pgfpathlineto{\pgfqpoint{5.950331in}{1.076236in}}%
\pgfpathlineto{\pgfqpoint{5.950331in}{1.157964in}}%
\pgfpathlineto{\pgfqpoint{5.764104in}{1.157964in}}%
\pgfpathlineto{\pgfqpoint{5.764104in}{1.076236in}}%
\pgfusepath{fill}%
\end{pgfscope}%
\begin{pgfscope}%
\pgfpathrectangle{\pgfqpoint{0.549740in}{0.463273in}}{\pgfqpoint{9.320225in}{4.495057in}}%
\pgfusepath{clip}%
\pgfsetbuttcap%
\pgfsetroundjoin%
\pgfsetlinewidth{0.000000pt}%
\definecolor{currentstroke}{rgb}{0.000000,0.000000,0.000000}%
\pgfsetstrokecolor{currentstroke}%
\pgfsetdash{}{0pt}%
\pgfpathmoveto{\pgfqpoint{5.950331in}{1.076236in}}%
\pgfpathlineto{\pgfqpoint{6.136557in}{1.076236in}}%
\pgfpathlineto{\pgfqpoint{6.136557in}{1.157964in}}%
\pgfpathlineto{\pgfqpoint{5.950331in}{1.157964in}}%
\pgfpathlineto{\pgfqpoint{5.950331in}{1.076236in}}%
\pgfusepath{}%
\end{pgfscope}%
\begin{pgfscope}%
\pgfpathrectangle{\pgfqpoint{0.549740in}{0.463273in}}{\pgfqpoint{9.320225in}{4.495057in}}%
\pgfusepath{clip}%
\pgfsetbuttcap%
\pgfsetroundjoin%
\pgfsetlinewidth{0.000000pt}%
\definecolor{currentstroke}{rgb}{0.000000,0.000000,0.000000}%
\pgfsetstrokecolor{currentstroke}%
\pgfsetdash{}{0pt}%
\pgfpathmoveto{\pgfqpoint{6.136557in}{1.076236in}}%
\pgfpathlineto{\pgfqpoint{6.322784in}{1.076236in}}%
\pgfpathlineto{\pgfqpoint{6.322784in}{1.157964in}}%
\pgfpathlineto{\pgfqpoint{6.136557in}{1.157964in}}%
\pgfpathlineto{\pgfqpoint{6.136557in}{1.076236in}}%
\pgfusepath{}%
\end{pgfscope}%
\begin{pgfscope}%
\pgfpathrectangle{\pgfqpoint{0.549740in}{0.463273in}}{\pgfqpoint{9.320225in}{4.495057in}}%
\pgfusepath{clip}%
\pgfsetbuttcap%
\pgfsetroundjoin%
\pgfsetlinewidth{0.000000pt}%
\definecolor{currentstroke}{rgb}{0.000000,0.000000,0.000000}%
\pgfsetstrokecolor{currentstroke}%
\pgfsetdash{}{0pt}%
\pgfpathmoveto{\pgfqpoint{6.322784in}{1.076236in}}%
\pgfpathlineto{\pgfqpoint{6.509011in}{1.076236in}}%
\pgfpathlineto{\pgfqpoint{6.509011in}{1.157964in}}%
\pgfpathlineto{\pgfqpoint{6.322784in}{1.157964in}}%
\pgfpathlineto{\pgfqpoint{6.322784in}{1.076236in}}%
\pgfusepath{}%
\end{pgfscope}%
\begin{pgfscope}%
\pgfpathrectangle{\pgfqpoint{0.549740in}{0.463273in}}{\pgfqpoint{9.320225in}{4.495057in}}%
\pgfusepath{clip}%
\pgfsetbuttcap%
\pgfsetroundjoin%
\pgfsetlinewidth{0.000000pt}%
\definecolor{currentstroke}{rgb}{0.000000,0.000000,0.000000}%
\pgfsetstrokecolor{currentstroke}%
\pgfsetdash{}{0pt}%
\pgfpathmoveto{\pgfqpoint{6.509011in}{1.076236in}}%
\pgfpathlineto{\pgfqpoint{6.695237in}{1.076236in}}%
\pgfpathlineto{\pgfqpoint{6.695237in}{1.157964in}}%
\pgfpathlineto{\pgfqpoint{6.509011in}{1.157964in}}%
\pgfpathlineto{\pgfqpoint{6.509011in}{1.076236in}}%
\pgfusepath{}%
\end{pgfscope}%
\begin{pgfscope}%
\pgfpathrectangle{\pgfqpoint{0.549740in}{0.463273in}}{\pgfqpoint{9.320225in}{4.495057in}}%
\pgfusepath{clip}%
\pgfsetbuttcap%
\pgfsetroundjoin%
\definecolor{currentfill}{rgb}{0.189527,0.635753,0.950228}%
\pgfsetfillcolor{currentfill}%
\pgfsetlinewidth{0.000000pt}%
\definecolor{currentstroke}{rgb}{0.000000,0.000000,0.000000}%
\pgfsetstrokecolor{currentstroke}%
\pgfsetdash{}{0pt}%
\pgfpathmoveto{\pgfqpoint{6.695237in}{1.076236in}}%
\pgfpathlineto{\pgfqpoint{6.881464in}{1.076236in}}%
\pgfpathlineto{\pgfqpoint{6.881464in}{1.157964in}}%
\pgfpathlineto{\pgfqpoint{6.695237in}{1.157964in}}%
\pgfpathlineto{\pgfqpoint{6.695237in}{1.076236in}}%
\pgfusepath{fill}%
\end{pgfscope}%
\begin{pgfscope}%
\pgfpathrectangle{\pgfqpoint{0.549740in}{0.463273in}}{\pgfqpoint{9.320225in}{4.495057in}}%
\pgfusepath{clip}%
\pgfsetbuttcap%
\pgfsetroundjoin%
\pgfsetlinewidth{0.000000pt}%
\definecolor{currentstroke}{rgb}{0.000000,0.000000,0.000000}%
\pgfsetstrokecolor{currentstroke}%
\pgfsetdash{}{0pt}%
\pgfpathmoveto{\pgfqpoint{6.881464in}{1.076236in}}%
\pgfpathlineto{\pgfqpoint{7.067690in}{1.076236in}}%
\pgfpathlineto{\pgfqpoint{7.067690in}{1.157964in}}%
\pgfpathlineto{\pgfqpoint{6.881464in}{1.157964in}}%
\pgfpathlineto{\pgfqpoint{6.881464in}{1.076236in}}%
\pgfusepath{}%
\end{pgfscope}%
\begin{pgfscope}%
\pgfpathrectangle{\pgfqpoint{0.549740in}{0.463273in}}{\pgfqpoint{9.320225in}{4.495057in}}%
\pgfusepath{clip}%
\pgfsetbuttcap%
\pgfsetroundjoin%
\pgfsetlinewidth{0.000000pt}%
\definecolor{currentstroke}{rgb}{0.000000,0.000000,0.000000}%
\pgfsetstrokecolor{currentstroke}%
\pgfsetdash{}{0pt}%
\pgfpathmoveto{\pgfqpoint{7.067690in}{1.076236in}}%
\pgfpathlineto{\pgfqpoint{7.253917in}{1.076236in}}%
\pgfpathlineto{\pgfqpoint{7.253917in}{1.157964in}}%
\pgfpathlineto{\pgfqpoint{7.067690in}{1.157964in}}%
\pgfpathlineto{\pgfqpoint{7.067690in}{1.076236in}}%
\pgfusepath{}%
\end{pgfscope}%
\begin{pgfscope}%
\pgfpathrectangle{\pgfqpoint{0.549740in}{0.463273in}}{\pgfqpoint{9.320225in}{4.495057in}}%
\pgfusepath{clip}%
\pgfsetbuttcap%
\pgfsetroundjoin%
\pgfsetlinewidth{0.000000pt}%
\definecolor{currentstroke}{rgb}{0.000000,0.000000,0.000000}%
\pgfsetstrokecolor{currentstroke}%
\pgfsetdash{}{0pt}%
\pgfpathmoveto{\pgfqpoint{7.253917in}{1.076236in}}%
\pgfpathlineto{\pgfqpoint{7.440143in}{1.076236in}}%
\pgfpathlineto{\pgfqpoint{7.440143in}{1.157964in}}%
\pgfpathlineto{\pgfqpoint{7.253917in}{1.157964in}}%
\pgfpathlineto{\pgfqpoint{7.253917in}{1.076236in}}%
\pgfusepath{}%
\end{pgfscope}%
\begin{pgfscope}%
\pgfpathrectangle{\pgfqpoint{0.549740in}{0.463273in}}{\pgfqpoint{9.320225in}{4.495057in}}%
\pgfusepath{clip}%
\pgfsetbuttcap%
\pgfsetroundjoin%
\pgfsetlinewidth{0.000000pt}%
\definecolor{currentstroke}{rgb}{0.000000,0.000000,0.000000}%
\pgfsetstrokecolor{currentstroke}%
\pgfsetdash{}{0pt}%
\pgfpathmoveto{\pgfqpoint{7.440143in}{1.076236in}}%
\pgfpathlineto{\pgfqpoint{7.626370in}{1.076236in}}%
\pgfpathlineto{\pgfqpoint{7.626370in}{1.157964in}}%
\pgfpathlineto{\pgfqpoint{7.440143in}{1.157964in}}%
\pgfpathlineto{\pgfqpoint{7.440143in}{1.076236in}}%
\pgfusepath{}%
\end{pgfscope}%
\begin{pgfscope}%
\pgfpathrectangle{\pgfqpoint{0.549740in}{0.463273in}}{\pgfqpoint{9.320225in}{4.495057in}}%
\pgfusepath{clip}%
\pgfsetbuttcap%
\pgfsetroundjoin%
\pgfsetlinewidth{0.000000pt}%
\definecolor{currentstroke}{rgb}{0.000000,0.000000,0.000000}%
\pgfsetstrokecolor{currentstroke}%
\pgfsetdash{}{0pt}%
\pgfpathmoveto{\pgfqpoint{7.626370in}{1.076236in}}%
\pgfpathlineto{\pgfqpoint{7.812596in}{1.076236in}}%
\pgfpathlineto{\pgfqpoint{7.812596in}{1.157964in}}%
\pgfpathlineto{\pgfqpoint{7.626370in}{1.157964in}}%
\pgfpathlineto{\pgfqpoint{7.626370in}{1.076236in}}%
\pgfusepath{}%
\end{pgfscope}%
\begin{pgfscope}%
\pgfpathrectangle{\pgfqpoint{0.549740in}{0.463273in}}{\pgfqpoint{9.320225in}{4.495057in}}%
\pgfusepath{clip}%
\pgfsetbuttcap%
\pgfsetroundjoin%
\pgfsetlinewidth{0.000000pt}%
\definecolor{currentstroke}{rgb}{0.000000,0.000000,0.000000}%
\pgfsetstrokecolor{currentstroke}%
\pgfsetdash{}{0pt}%
\pgfpathmoveto{\pgfqpoint{7.812596in}{1.076236in}}%
\pgfpathlineto{\pgfqpoint{7.998823in}{1.076236in}}%
\pgfpathlineto{\pgfqpoint{7.998823in}{1.157964in}}%
\pgfpathlineto{\pgfqpoint{7.812596in}{1.157964in}}%
\pgfpathlineto{\pgfqpoint{7.812596in}{1.076236in}}%
\pgfusepath{}%
\end{pgfscope}%
\begin{pgfscope}%
\pgfpathrectangle{\pgfqpoint{0.549740in}{0.463273in}}{\pgfqpoint{9.320225in}{4.495057in}}%
\pgfusepath{clip}%
\pgfsetbuttcap%
\pgfsetroundjoin%
\pgfsetlinewidth{0.000000pt}%
\definecolor{currentstroke}{rgb}{0.000000,0.000000,0.000000}%
\pgfsetstrokecolor{currentstroke}%
\pgfsetdash{}{0pt}%
\pgfpathmoveto{\pgfqpoint{7.998823in}{1.076236in}}%
\pgfpathlineto{\pgfqpoint{8.185049in}{1.076236in}}%
\pgfpathlineto{\pgfqpoint{8.185049in}{1.157964in}}%
\pgfpathlineto{\pgfqpoint{7.998823in}{1.157964in}}%
\pgfpathlineto{\pgfqpoint{7.998823in}{1.076236in}}%
\pgfusepath{}%
\end{pgfscope}%
\begin{pgfscope}%
\pgfpathrectangle{\pgfqpoint{0.549740in}{0.463273in}}{\pgfqpoint{9.320225in}{4.495057in}}%
\pgfusepath{clip}%
\pgfsetbuttcap%
\pgfsetroundjoin%
\pgfsetlinewidth{0.000000pt}%
\definecolor{currentstroke}{rgb}{0.000000,0.000000,0.000000}%
\pgfsetstrokecolor{currentstroke}%
\pgfsetdash{}{0pt}%
\pgfpathmoveto{\pgfqpoint{8.185049in}{1.076236in}}%
\pgfpathlineto{\pgfqpoint{8.371276in}{1.076236in}}%
\pgfpathlineto{\pgfqpoint{8.371276in}{1.157964in}}%
\pgfpathlineto{\pgfqpoint{8.185049in}{1.157964in}}%
\pgfpathlineto{\pgfqpoint{8.185049in}{1.076236in}}%
\pgfusepath{}%
\end{pgfscope}%
\begin{pgfscope}%
\pgfpathrectangle{\pgfqpoint{0.549740in}{0.463273in}}{\pgfqpoint{9.320225in}{4.495057in}}%
\pgfusepath{clip}%
\pgfsetbuttcap%
\pgfsetroundjoin%
\pgfsetlinewidth{0.000000pt}%
\definecolor{currentstroke}{rgb}{0.000000,0.000000,0.000000}%
\pgfsetstrokecolor{currentstroke}%
\pgfsetdash{}{0pt}%
\pgfpathmoveto{\pgfqpoint{8.371276in}{1.076236in}}%
\pgfpathlineto{\pgfqpoint{8.557503in}{1.076236in}}%
\pgfpathlineto{\pgfqpoint{8.557503in}{1.157964in}}%
\pgfpathlineto{\pgfqpoint{8.371276in}{1.157964in}}%
\pgfpathlineto{\pgfqpoint{8.371276in}{1.076236in}}%
\pgfusepath{}%
\end{pgfscope}%
\begin{pgfscope}%
\pgfpathrectangle{\pgfqpoint{0.549740in}{0.463273in}}{\pgfqpoint{9.320225in}{4.495057in}}%
\pgfusepath{clip}%
\pgfsetbuttcap%
\pgfsetroundjoin%
\pgfsetlinewidth{0.000000pt}%
\definecolor{currentstroke}{rgb}{0.000000,0.000000,0.000000}%
\pgfsetstrokecolor{currentstroke}%
\pgfsetdash{}{0pt}%
\pgfpathmoveto{\pgfqpoint{8.557503in}{1.076236in}}%
\pgfpathlineto{\pgfqpoint{8.743729in}{1.076236in}}%
\pgfpathlineto{\pgfqpoint{8.743729in}{1.157964in}}%
\pgfpathlineto{\pgfqpoint{8.557503in}{1.157964in}}%
\pgfpathlineto{\pgfqpoint{8.557503in}{1.076236in}}%
\pgfusepath{}%
\end{pgfscope}%
\begin{pgfscope}%
\pgfpathrectangle{\pgfqpoint{0.549740in}{0.463273in}}{\pgfqpoint{9.320225in}{4.495057in}}%
\pgfusepath{clip}%
\pgfsetbuttcap%
\pgfsetroundjoin%
\pgfsetlinewidth{0.000000pt}%
\definecolor{currentstroke}{rgb}{0.000000,0.000000,0.000000}%
\pgfsetstrokecolor{currentstroke}%
\pgfsetdash{}{0pt}%
\pgfpathmoveto{\pgfqpoint{8.743729in}{1.076236in}}%
\pgfpathlineto{\pgfqpoint{8.929956in}{1.076236in}}%
\pgfpathlineto{\pgfqpoint{8.929956in}{1.157964in}}%
\pgfpathlineto{\pgfqpoint{8.743729in}{1.157964in}}%
\pgfpathlineto{\pgfqpoint{8.743729in}{1.076236in}}%
\pgfusepath{}%
\end{pgfscope}%
\begin{pgfscope}%
\pgfpathrectangle{\pgfqpoint{0.549740in}{0.463273in}}{\pgfqpoint{9.320225in}{4.495057in}}%
\pgfusepath{clip}%
\pgfsetbuttcap%
\pgfsetroundjoin%
\pgfsetlinewidth{0.000000pt}%
\definecolor{currentstroke}{rgb}{0.000000,0.000000,0.000000}%
\pgfsetstrokecolor{currentstroke}%
\pgfsetdash{}{0pt}%
\pgfpathmoveto{\pgfqpoint{8.929956in}{1.076236in}}%
\pgfpathlineto{\pgfqpoint{9.116182in}{1.076236in}}%
\pgfpathlineto{\pgfqpoint{9.116182in}{1.157964in}}%
\pgfpathlineto{\pgfqpoint{8.929956in}{1.157964in}}%
\pgfpathlineto{\pgfqpoint{8.929956in}{1.076236in}}%
\pgfusepath{}%
\end{pgfscope}%
\begin{pgfscope}%
\pgfpathrectangle{\pgfqpoint{0.549740in}{0.463273in}}{\pgfqpoint{9.320225in}{4.495057in}}%
\pgfusepath{clip}%
\pgfsetbuttcap%
\pgfsetroundjoin%
\pgfsetlinewidth{0.000000pt}%
\definecolor{currentstroke}{rgb}{0.000000,0.000000,0.000000}%
\pgfsetstrokecolor{currentstroke}%
\pgfsetdash{}{0pt}%
\pgfpathmoveto{\pgfqpoint{9.116182in}{1.076236in}}%
\pgfpathlineto{\pgfqpoint{9.302409in}{1.076236in}}%
\pgfpathlineto{\pgfqpoint{9.302409in}{1.157964in}}%
\pgfpathlineto{\pgfqpoint{9.116182in}{1.157964in}}%
\pgfpathlineto{\pgfqpoint{9.116182in}{1.076236in}}%
\pgfusepath{}%
\end{pgfscope}%
\begin{pgfscope}%
\pgfpathrectangle{\pgfqpoint{0.549740in}{0.463273in}}{\pgfqpoint{9.320225in}{4.495057in}}%
\pgfusepath{clip}%
\pgfsetbuttcap%
\pgfsetroundjoin%
\pgfsetlinewidth{0.000000pt}%
\definecolor{currentstroke}{rgb}{0.000000,0.000000,0.000000}%
\pgfsetstrokecolor{currentstroke}%
\pgfsetdash{}{0pt}%
\pgfpathmoveto{\pgfqpoint{9.302409in}{1.076236in}}%
\pgfpathlineto{\pgfqpoint{9.488635in}{1.076236in}}%
\pgfpathlineto{\pgfqpoint{9.488635in}{1.157964in}}%
\pgfpathlineto{\pgfqpoint{9.302409in}{1.157964in}}%
\pgfpathlineto{\pgfqpoint{9.302409in}{1.076236in}}%
\pgfusepath{}%
\end{pgfscope}%
\begin{pgfscope}%
\pgfpathrectangle{\pgfqpoint{0.549740in}{0.463273in}}{\pgfqpoint{9.320225in}{4.495057in}}%
\pgfusepath{clip}%
\pgfsetbuttcap%
\pgfsetroundjoin%
\pgfsetlinewidth{0.000000pt}%
\definecolor{currentstroke}{rgb}{0.000000,0.000000,0.000000}%
\pgfsetstrokecolor{currentstroke}%
\pgfsetdash{}{0pt}%
\pgfpathmoveto{\pgfqpoint{9.488635in}{1.076236in}}%
\pgfpathlineto{\pgfqpoint{9.674862in}{1.076236in}}%
\pgfpathlineto{\pgfqpoint{9.674862in}{1.157964in}}%
\pgfpathlineto{\pgfqpoint{9.488635in}{1.157964in}}%
\pgfpathlineto{\pgfqpoint{9.488635in}{1.076236in}}%
\pgfusepath{}%
\end{pgfscope}%
\begin{pgfscope}%
\pgfpathrectangle{\pgfqpoint{0.549740in}{0.463273in}}{\pgfqpoint{9.320225in}{4.495057in}}%
\pgfusepath{clip}%
\pgfsetbuttcap%
\pgfsetroundjoin%
\pgfsetlinewidth{0.000000pt}%
\definecolor{currentstroke}{rgb}{0.000000,0.000000,0.000000}%
\pgfsetstrokecolor{currentstroke}%
\pgfsetdash{}{0pt}%
\pgfpathmoveto{\pgfqpoint{9.674862in}{1.076236in}}%
\pgfpathlineto{\pgfqpoint{9.861088in}{1.076236in}}%
\pgfpathlineto{\pgfqpoint{9.861088in}{1.157964in}}%
\pgfpathlineto{\pgfqpoint{9.674862in}{1.157964in}}%
\pgfpathlineto{\pgfqpoint{9.674862in}{1.076236in}}%
\pgfusepath{}%
\end{pgfscope}%
\begin{pgfscope}%
\pgfpathrectangle{\pgfqpoint{0.549740in}{0.463273in}}{\pgfqpoint{9.320225in}{4.495057in}}%
\pgfusepath{clip}%
\pgfsetbuttcap%
\pgfsetroundjoin%
\pgfsetlinewidth{0.000000pt}%
\definecolor{currentstroke}{rgb}{0.000000,0.000000,0.000000}%
\pgfsetstrokecolor{currentstroke}%
\pgfsetdash{}{0pt}%
\pgfpathmoveto{\pgfqpoint{0.549761in}{1.157964in}}%
\pgfpathlineto{\pgfqpoint{0.735988in}{1.157964in}}%
\pgfpathlineto{\pgfqpoint{0.735988in}{1.239692in}}%
\pgfpathlineto{\pgfqpoint{0.549761in}{1.239692in}}%
\pgfpathlineto{\pgfqpoint{0.549761in}{1.157964in}}%
\pgfusepath{}%
\end{pgfscope}%
\begin{pgfscope}%
\pgfpathrectangle{\pgfqpoint{0.549740in}{0.463273in}}{\pgfqpoint{9.320225in}{4.495057in}}%
\pgfusepath{clip}%
\pgfsetbuttcap%
\pgfsetroundjoin%
\pgfsetlinewidth{0.000000pt}%
\definecolor{currentstroke}{rgb}{0.000000,0.000000,0.000000}%
\pgfsetstrokecolor{currentstroke}%
\pgfsetdash{}{0pt}%
\pgfpathmoveto{\pgfqpoint{0.735988in}{1.157964in}}%
\pgfpathlineto{\pgfqpoint{0.922214in}{1.157964in}}%
\pgfpathlineto{\pgfqpoint{0.922214in}{1.239692in}}%
\pgfpathlineto{\pgfqpoint{0.735988in}{1.239692in}}%
\pgfpathlineto{\pgfqpoint{0.735988in}{1.157964in}}%
\pgfusepath{}%
\end{pgfscope}%
\begin{pgfscope}%
\pgfpathrectangle{\pgfqpoint{0.549740in}{0.463273in}}{\pgfqpoint{9.320225in}{4.495057in}}%
\pgfusepath{clip}%
\pgfsetbuttcap%
\pgfsetroundjoin%
\pgfsetlinewidth{0.000000pt}%
\definecolor{currentstroke}{rgb}{0.000000,0.000000,0.000000}%
\pgfsetstrokecolor{currentstroke}%
\pgfsetdash{}{0pt}%
\pgfpathmoveto{\pgfqpoint{0.922214in}{1.157964in}}%
\pgfpathlineto{\pgfqpoint{1.108441in}{1.157964in}}%
\pgfpathlineto{\pgfqpoint{1.108441in}{1.239692in}}%
\pgfpathlineto{\pgfqpoint{0.922214in}{1.239692in}}%
\pgfpathlineto{\pgfqpoint{0.922214in}{1.157964in}}%
\pgfusepath{}%
\end{pgfscope}%
\begin{pgfscope}%
\pgfpathrectangle{\pgfqpoint{0.549740in}{0.463273in}}{\pgfqpoint{9.320225in}{4.495057in}}%
\pgfusepath{clip}%
\pgfsetbuttcap%
\pgfsetroundjoin%
\pgfsetlinewidth{0.000000pt}%
\definecolor{currentstroke}{rgb}{0.000000,0.000000,0.000000}%
\pgfsetstrokecolor{currentstroke}%
\pgfsetdash{}{0pt}%
\pgfpathmoveto{\pgfqpoint{1.108441in}{1.157964in}}%
\pgfpathlineto{\pgfqpoint{1.294667in}{1.157964in}}%
\pgfpathlineto{\pgfqpoint{1.294667in}{1.239692in}}%
\pgfpathlineto{\pgfqpoint{1.108441in}{1.239692in}}%
\pgfpathlineto{\pgfqpoint{1.108441in}{1.157964in}}%
\pgfusepath{}%
\end{pgfscope}%
\begin{pgfscope}%
\pgfpathrectangle{\pgfqpoint{0.549740in}{0.463273in}}{\pgfqpoint{9.320225in}{4.495057in}}%
\pgfusepath{clip}%
\pgfsetbuttcap%
\pgfsetroundjoin%
\pgfsetlinewidth{0.000000pt}%
\definecolor{currentstroke}{rgb}{0.000000,0.000000,0.000000}%
\pgfsetstrokecolor{currentstroke}%
\pgfsetdash{}{0pt}%
\pgfpathmoveto{\pgfqpoint{1.294667in}{1.157964in}}%
\pgfpathlineto{\pgfqpoint{1.480894in}{1.157964in}}%
\pgfpathlineto{\pgfqpoint{1.480894in}{1.239692in}}%
\pgfpathlineto{\pgfqpoint{1.294667in}{1.239692in}}%
\pgfpathlineto{\pgfqpoint{1.294667in}{1.157964in}}%
\pgfusepath{}%
\end{pgfscope}%
\begin{pgfscope}%
\pgfpathrectangle{\pgfqpoint{0.549740in}{0.463273in}}{\pgfqpoint{9.320225in}{4.495057in}}%
\pgfusepath{clip}%
\pgfsetbuttcap%
\pgfsetroundjoin%
\pgfsetlinewidth{0.000000pt}%
\definecolor{currentstroke}{rgb}{0.000000,0.000000,0.000000}%
\pgfsetstrokecolor{currentstroke}%
\pgfsetdash{}{0pt}%
\pgfpathmoveto{\pgfqpoint{1.480894in}{1.157964in}}%
\pgfpathlineto{\pgfqpoint{1.667120in}{1.157964in}}%
\pgfpathlineto{\pgfqpoint{1.667120in}{1.239692in}}%
\pgfpathlineto{\pgfqpoint{1.480894in}{1.239692in}}%
\pgfpathlineto{\pgfqpoint{1.480894in}{1.157964in}}%
\pgfusepath{}%
\end{pgfscope}%
\begin{pgfscope}%
\pgfpathrectangle{\pgfqpoint{0.549740in}{0.463273in}}{\pgfqpoint{9.320225in}{4.495057in}}%
\pgfusepath{clip}%
\pgfsetbuttcap%
\pgfsetroundjoin%
\pgfsetlinewidth{0.000000pt}%
\definecolor{currentstroke}{rgb}{0.000000,0.000000,0.000000}%
\pgfsetstrokecolor{currentstroke}%
\pgfsetdash{}{0pt}%
\pgfpathmoveto{\pgfqpoint{1.667120in}{1.157964in}}%
\pgfpathlineto{\pgfqpoint{1.853347in}{1.157964in}}%
\pgfpathlineto{\pgfqpoint{1.853347in}{1.239692in}}%
\pgfpathlineto{\pgfqpoint{1.667120in}{1.239692in}}%
\pgfpathlineto{\pgfqpoint{1.667120in}{1.157964in}}%
\pgfusepath{}%
\end{pgfscope}%
\begin{pgfscope}%
\pgfpathrectangle{\pgfqpoint{0.549740in}{0.463273in}}{\pgfqpoint{9.320225in}{4.495057in}}%
\pgfusepath{clip}%
\pgfsetbuttcap%
\pgfsetroundjoin%
\pgfsetlinewidth{0.000000pt}%
\definecolor{currentstroke}{rgb}{0.000000,0.000000,0.000000}%
\pgfsetstrokecolor{currentstroke}%
\pgfsetdash{}{0pt}%
\pgfpathmoveto{\pgfqpoint{1.853347in}{1.157964in}}%
\pgfpathlineto{\pgfqpoint{2.039573in}{1.157964in}}%
\pgfpathlineto{\pgfqpoint{2.039573in}{1.239692in}}%
\pgfpathlineto{\pgfqpoint{1.853347in}{1.239692in}}%
\pgfpathlineto{\pgfqpoint{1.853347in}{1.157964in}}%
\pgfusepath{}%
\end{pgfscope}%
\begin{pgfscope}%
\pgfpathrectangle{\pgfqpoint{0.549740in}{0.463273in}}{\pgfqpoint{9.320225in}{4.495057in}}%
\pgfusepath{clip}%
\pgfsetbuttcap%
\pgfsetroundjoin%
\pgfsetlinewidth{0.000000pt}%
\definecolor{currentstroke}{rgb}{0.000000,0.000000,0.000000}%
\pgfsetstrokecolor{currentstroke}%
\pgfsetdash{}{0pt}%
\pgfpathmoveto{\pgfqpoint{2.039573in}{1.157964in}}%
\pgfpathlineto{\pgfqpoint{2.225800in}{1.157964in}}%
\pgfpathlineto{\pgfqpoint{2.225800in}{1.239692in}}%
\pgfpathlineto{\pgfqpoint{2.039573in}{1.239692in}}%
\pgfpathlineto{\pgfqpoint{2.039573in}{1.157964in}}%
\pgfusepath{}%
\end{pgfscope}%
\begin{pgfscope}%
\pgfpathrectangle{\pgfqpoint{0.549740in}{0.463273in}}{\pgfqpoint{9.320225in}{4.495057in}}%
\pgfusepath{clip}%
\pgfsetbuttcap%
\pgfsetroundjoin%
\pgfsetlinewidth{0.000000pt}%
\definecolor{currentstroke}{rgb}{0.000000,0.000000,0.000000}%
\pgfsetstrokecolor{currentstroke}%
\pgfsetdash{}{0pt}%
\pgfpathmoveto{\pgfqpoint{2.225800in}{1.157964in}}%
\pgfpathlineto{\pgfqpoint{2.412027in}{1.157964in}}%
\pgfpathlineto{\pgfqpoint{2.412027in}{1.239692in}}%
\pgfpathlineto{\pgfqpoint{2.225800in}{1.239692in}}%
\pgfpathlineto{\pgfqpoint{2.225800in}{1.157964in}}%
\pgfusepath{}%
\end{pgfscope}%
\begin{pgfscope}%
\pgfpathrectangle{\pgfqpoint{0.549740in}{0.463273in}}{\pgfqpoint{9.320225in}{4.495057in}}%
\pgfusepath{clip}%
\pgfsetbuttcap%
\pgfsetroundjoin%
\pgfsetlinewidth{0.000000pt}%
\definecolor{currentstroke}{rgb}{0.000000,0.000000,0.000000}%
\pgfsetstrokecolor{currentstroke}%
\pgfsetdash{}{0pt}%
\pgfpathmoveto{\pgfqpoint{2.412027in}{1.157964in}}%
\pgfpathlineto{\pgfqpoint{2.598253in}{1.157964in}}%
\pgfpathlineto{\pgfqpoint{2.598253in}{1.239692in}}%
\pgfpathlineto{\pgfqpoint{2.412027in}{1.239692in}}%
\pgfpathlineto{\pgfqpoint{2.412027in}{1.157964in}}%
\pgfusepath{}%
\end{pgfscope}%
\begin{pgfscope}%
\pgfpathrectangle{\pgfqpoint{0.549740in}{0.463273in}}{\pgfqpoint{9.320225in}{4.495057in}}%
\pgfusepath{clip}%
\pgfsetbuttcap%
\pgfsetroundjoin%
\pgfsetlinewidth{0.000000pt}%
\definecolor{currentstroke}{rgb}{0.000000,0.000000,0.000000}%
\pgfsetstrokecolor{currentstroke}%
\pgfsetdash{}{0pt}%
\pgfpathmoveto{\pgfqpoint{2.598253in}{1.157964in}}%
\pgfpathlineto{\pgfqpoint{2.784480in}{1.157964in}}%
\pgfpathlineto{\pgfqpoint{2.784480in}{1.239692in}}%
\pgfpathlineto{\pgfqpoint{2.598253in}{1.239692in}}%
\pgfpathlineto{\pgfqpoint{2.598253in}{1.157964in}}%
\pgfusepath{}%
\end{pgfscope}%
\begin{pgfscope}%
\pgfpathrectangle{\pgfqpoint{0.549740in}{0.463273in}}{\pgfqpoint{9.320225in}{4.495057in}}%
\pgfusepath{clip}%
\pgfsetbuttcap%
\pgfsetroundjoin%
\pgfsetlinewidth{0.000000pt}%
\definecolor{currentstroke}{rgb}{0.000000,0.000000,0.000000}%
\pgfsetstrokecolor{currentstroke}%
\pgfsetdash{}{0pt}%
\pgfpathmoveto{\pgfqpoint{2.784480in}{1.157964in}}%
\pgfpathlineto{\pgfqpoint{2.970706in}{1.157964in}}%
\pgfpathlineto{\pgfqpoint{2.970706in}{1.239692in}}%
\pgfpathlineto{\pgfqpoint{2.784480in}{1.239692in}}%
\pgfpathlineto{\pgfqpoint{2.784480in}{1.157964in}}%
\pgfusepath{}%
\end{pgfscope}%
\begin{pgfscope}%
\pgfpathrectangle{\pgfqpoint{0.549740in}{0.463273in}}{\pgfqpoint{9.320225in}{4.495057in}}%
\pgfusepath{clip}%
\pgfsetbuttcap%
\pgfsetroundjoin%
\pgfsetlinewidth{0.000000pt}%
\definecolor{currentstroke}{rgb}{0.000000,0.000000,0.000000}%
\pgfsetstrokecolor{currentstroke}%
\pgfsetdash{}{0pt}%
\pgfpathmoveto{\pgfqpoint{2.970706in}{1.157964in}}%
\pgfpathlineto{\pgfqpoint{3.156933in}{1.157964in}}%
\pgfpathlineto{\pgfqpoint{3.156933in}{1.239692in}}%
\pgfpathlineto{\pgfqpoint{2.970706in}{1.239692in}}%
\pgfpathlineto{\pgfqpoint{2.970706in}{1.157964in}}%
\pgfusepath{}%
\end{pgfscope}%
\begin{pgfscope}%
\pgfpathrectangle{\pgfqpoint{0.549740in}{0.463273in}}{\pgfqpoint{9.320225in}{4.495057in}}%
\pgfusepath{clip}%
\pgfsetbuttcap%
\pgfsetroundjoin%
\pgfsetlinewidth{0.000000pt}%
\definecolor{currentstroke}{rgb}{0.000000,0.000000,0.000000}%
\pgfsetstrokecolor{currentstroke}%
\pgfsetdash{}{0pt}%
\pgfpathmoveto{\pgfqpoint{3.156933in}{1.157964in}}%
\pgfpathlineto{\pgfqpoint{3.343159in}{1.157964in}}%
\pgfpathlineto{\pgfqpoint{3.343159in}{1.239692in}}%
\pgfpathlineto{\pgfqpoint{3.156933in}{1.239692in}}%
\pgfpathlineto{\pgfqpoint{3.156933in}{1.157964in}}%
\pgfusepath{}%
\end{pgfscope}%
\begin{pgfscope}%
\pgfpathrectangle{\pgfqpoint{0.549740in}{0.463273in}}{\pgfqpoint{9.320225in}{4.495057in}}%
\pgfusepath{clip}%
\pgfsetbuttcap%
\pgfsetroundjoin%
\pgfsetlinewidth{0.000000pt}%
\definecolor{currentstroke}{rgb}{0.000000,0.000000,0.000000}%
\pgfsetstrokecolor{currentstroke}%
\pgfsetdash{}{0pt}%
\pgfpathmoveto{\pgfqpoint{3.343159in}{1.157964in}}%
\pgfpathlineto{\pgfqpoint{3.529386in}{1.157964in}}%
\pgfpathlineto{\pgfqpoint{3.529386in}{1.239692in}}%
\pgfpathlineto{\pgfqpoint{3.343159in}{1.239692in}}%
\pgfpathlineto{\pgfqpoint{3.343159in}{1.157964in}}%
\pgfusepath{}%
\end{pgfscope}%
\begin{pgfscope}%
\pgfpathrectangle{\pgfqpoint{0.549740in}{0.463273in}}{\pgfqpoint{9.320225in}{4.495057in}}%
\pgfusepath{clip}%
\pgfsetbuttcap%
\pgfsetroundjoin%
\pgfsetlinewidth{0.000000pt}%
\definecolor{currentstroke}{rgb}{0.000000,0.000000,0.000000}%
\pgfsetstrokecolor{currentstroke}%
\pgfsetdash{}{0pt}%
\pgfpathmoveto{\pgfqpoint{3.529386in}{1.157964in}}%
\pgfpathlineto{\pgfqpoint{3.715612in}{1.157964in}}%
\pgfpathlineto{\pgfqpoint{3.715612in}{1.239692in}}%
\pgfpathlineto{\pgfqpoint{3.529386in}{1.239692in}}%
\pgfpathlineto{\pgfqpoint{3.529386in}{1.157964in}}%
\pgfusepath{}%
\end{pgfscope}%
\begin{pgfscope}%
\pgfpathrectangle{\pgfqpoint{0.549740in}{0.463273in}}{\pgfqpoint{9.320225in}{4.495057in}}%
\pgfusepath{clip}%
\pgfsetbuttcap%
\pgfsetroundjoin%
\pgfsetlinewidth{0.000000pt}%
\definecolor{currentstroke}{rgb}{0.000000,0.000000,0.000000}%
\pgfsetstrokecolor{currentstroke}%
\pgfsetdash{}{0pt}%
\pgfpathmoveto{\pgfqpoint{3.715612in}{1.157964in}}%
\pgfpathlineto{\pgfqpoint{3.901839in}{1.157964in}}%
\pgfpathlineto{\pgfqpoint{3.901839in}{1.239692in}}%
\pgfpathlineto{\pgfqpoint{3.715612in}{1.239692in}}%
\pgfpathlineto{\pgfqpoint{3.715612in}{1.157964in}}%
\pgfusepath{}%
\end{pgfscope}%
\begin{pgfscope}%
\pgfpathrectangle{\pgfqpoint{0.549740in}{0.463273in}}{\pgfqpoint{9.320225in}{4.495057in}}%
\pgfusepath{clip}%
\pgfsetbuttcap%
\pgfsetroundjoin%
\pgfsetlinewidth{0.000000pt}%
\definecolor{currentstroke}{rgb}{0.000000,0.000000,0.000000}%
\pgfsetstrokecolor{currentstroke}%
\pgfsetdash{}{0pt}%
\pgfpathmoveto{\pgfqpoint{3.901839in}{1.157964in}}%
\pgfpathlineto{\pgfqpoint{4.088065in}{1.157964in}}%
\pgfpathlineto{\pgfqpoint{4.088065in}{1.239692in}}%
\pgfpathlineto{\pgfqpoint{3.901839in}{1.239692in}}%
\pgfpathlineto{\pgfqpoint{3.901839in}{1.157964in}}%
\pgfusepath{}%
\end{pgfscope}%
\begin{pgfscope}%
\pgfpathrectangle{\pgfqpoint{0.549740in}{0.463273in}}{\pgfqpoint{9.320225in}{4.495057in}}%
\pgfusepath{clip}%
\pgfsetbuttcap%
\pgfsetroundjoin%
\pgfsetlinewidth{0.000000pt}%
\definecolor{currentstroke}{rgb}{0.000000,0.000000,0.000000}%
\pgfsetstrokecolor{currentstroke}%
\pgfsetdash{}{0pt}%
\pgfpathmoveto{\pgfqpoint{4.088065in}{1.157964in}}%
\pgfpathlineto{\pgfqpoint{4.274292in}{1.157964in}}%
\pgfpathlineto{\pgfqpoint{4.274292in}{1.239692in}}%
\pgfpathlineto{\pgfqpoint{4.088065in}{1.239692in}}%
\pgfpathlineto{\pgfqpoint{4.088065in}{1.157964in}}%
\pgfusepath{}%
\end{pgfscope}%
\begin{pgfscope}%
\pgfpathrectangle{\pgfqpoint{0.549740in}{0.463273in}}{\pgfqpoint{9.320225in}{4.495057in}}%
\pgfusepath{clip}%
\pgfsetbuttcap%
\pgfsetroundjoin%
\pgfsetlinewidth{0.000000pt}%
\definecolor{currentstroke}{rgb}{0.000000,0.000000,0.000000}%
\pgfsetstrokecolor{currentstroke}%
\pgfsetdash{}{0pt}%
\pgfpathmoveto{\pgfqpoint{4.274292in}{1.157964in}}%
\pgfpathlineto{\pgfqpoint{4.460519in}{1.157964in}}%
\pgfpathlineto{\pgfqpoint{4.460519in}{1.239692in}}%
\pgfpathlineto{\pgfqpoint{4.274292in}{1.239692in}}%
\pgfpathlineto{\pgfqpoint{4.274292in}{1.157964in}}%
\pgfusepath{}%
\end{pgfscope}%
\begin{pgfscope}%
\pgfpathrectangle{\pgfqpoint{0.549740in}{0.463273in}}{\pgfqpoint{9.320225in}{4.495057in}}%
\pgfusepath{clip}%
\pgfsetbuttcap%
\pgfsetroundjoin%
\pgfsetlinewidth{0.000000pt}%
\definecolor{currentstroke}{rgb}{0.000000,0.000000,0.000000}%
\pgfsetstrokecolor{currentstroke}%
\pgfsetdash{}{0pt}%
\pgfpathmoveto{\pgfqpoint{4.460519in}{1.157964in}}%
\pgfpathlineto{\pgfqpoint{4.646745in}{1.157964in}}%
\pgfpathlineto{\pgfqpoint{4.646745in}{1.239692in}}%
\pgfpathlineto{\pgfqpoint{4.460519in}{1.239692in}}%
\pgfpathlineto{\pgfqpoint{4.460519in}{1.157964in}}%
\pgfusepath{}%
\end{pgfscope}%
\begin{pgfscope}%
\pgfpathrectangle{\pgfqpoint{0.549740in}{0.463273in}}{\pgfqpoint{9.320225in}{4.495057in}}%
\pgfusepath{clip}%
\pgfsetbuttcap%
\pgfsetroundjoin%
\pgfsetlinewidth{0.000000pt}%
\definecolor{currentstroke}{rgb}{0.000000,0.000000,0.000000}%
\pgfsetstrokecolor{currentstroke}%
\pgfsetdash{}{0pt}%
\pgfpathmoveto{\pgfqpoint{4.646745in}{1.157964in}}%
\pgfpathlineto{\pgfqpoint{4.832972in}{1.157964in}}%
\pgfpathlineto{\pgfqpoint{4.832972in}{1.239692in}}%
\pgfpathlineto{\pgfqpoint{4.646745in}{1.239692in}}%
\pgfpathlineto{\pgfqpoint{4.646745in}{1.157964in}}%
\pgfusepath{}%
\end{pgfscope}%
\begin{pgfscope}%
\pgfpathrectangle{\pgfqpoint{0.549740in}{0.463273in}}{\pgfqpoint{9.320225in}{4.495057in}}%
\pgfusepath{clip}%
\pgfsetbuttcap%
\pgfsetroundjoin%
\pgfsetlinewidth{0.000000pt}%
\definecolor{currentstroke}{rgb}{0.000000,0.000000,0.000000}%
\pgfsetstrokecolor{currentstroke}%
\pgfsetdash{}{0pt}%
\pgfpathmoveto{\pgfqpoint{4.832972in}{1.157964in}}%
\pgfpathlineto{\pgfqpoint{5.019198in}{1.157964in}}%
\pgfpathlineto{\pgfqpoint{5.019198in}{1.239692in}}%
\pgfpathlineto{\pgfqpoint{4.832972in}{1.239692in}}%
\pgfpathlineto{\pgfqpoint{4.832972in}{1.157964in}}%
\pgfusepath{}%
\end{pgfscope}%
\begin{pgfscope}%
\pgfpathrectangle{\pgfqpoint{0.549740in}{0.463273in}}{\pgfqpoint{9.320225in}{4.495057in}}%
\pgfusepath{clip}%
\pgfsetbuttcap%
\pgfsetroundjoin%
\pgfsetlinewidth{0.000000pt}%
\definecolor{currentstroke}{rgb}{0.000000,0.000000,0.000000}%
\pgfsetstrokecolor{currentstroke}%
\pgfsetdash{}{0pt}%
\pgfpathmoveto{\pgfqpoint{5.019198in}{1.157964in}}%
\pgfpathlineto{\pgfqpoint{5.205425in}{1.157964in}}%
\pgfpathlineto{\pgfqpoint{5.205425in}{1.239692in}}%
\pgfpathlineto{\pgfqpoint{5.019198in}{1.239692in}}%
\pgfpathlineto{\pgfqpoint{5.019198in}{1.157964in}}%
\pgfusepath{}%
\end{pgfscope}%
\begin{pgfscope}%
\pgfpathrectangle{\pgfqpoint{0.549740in}{0.463273in}}{\pgfqpoint{9.320225in}{4.495057in}}%
\pgfusepath{clip}%
\pgfsetbuttcap%
\pgfsetroundjoin%
\pgfsetlinewidth{0.000000pt}%
\definecolor{currentstroke}{rgb}{0.000000,0.000000,0.000000}%
\pgfsetstrokecolor{currentstroke}%
\pgfsetdash{}{0pt}%
\pgfpathmoveto{\pgfqpoint{5.205425in}{1.157964in}}%
\pgfpathlineto{\pgfqpoint{5.391651in}{1.157964in}}%
\pgfpathlineto{\pgfqpoint{5.391651in}{1.239692in}}%
\pgfpathlineto{\pgfqpoint{5.205425in}{1.239692in}}%
\pgfpathlineto{\pgfqpoint{5.205425in}{1.157964in}}%
\pgfusepath{}%
\end{pgfscope}%
\begin{pgfscope}%
\pgfpathrectangle{\pgfqpoint{0.549740in}{0.463273in}}{\pgfqpoint{9.320225in}{4.495057in}}%
\pgfusepath{clip}%
\pgfsetbuttcap%
\pgfsetroundjoin%
\pgfsetlinewidth{0.000000pt}%
\definecolor{currentstroke}{rgb}{0.000000,0.000000,0.000000}%
\pgfsetstrokecolor{currentstroke}%
\pgfsetdash{}{0pt}%
\pgfpathmoveto{\pgfqpoint{5.391651in}{1.157964in}}%
\pgfpathlineto{\pgfqpoint{5.577878in}{1.157964in}}%
\pgfpathlineto{\pgfqpoint{5.577878in}{1.239692in}}%
\pgfpathlineto{\pgfqpoint{5.391651in}{1.239692in}}%
\pgfpathlineto{\pgfqpoint{5.391651in}{1.157964in}}%
\pgfusepath{}%
\end{pgfscope}%
\begin{pgfscope}%
\pgfpathrectangle{\pgfqpoint{0.549740in}{0.463273in}}{\pgfqpoint{9.320225in}{4.495057in}}%
\pgfusepath{clip}%
\pgfsetbuttcap%
\pgfsetroundjoin%
\pgfsetlinewidth{0.000000pt}%
\definecolor{currentstroke}{rgb}{0.000000,0.000000,0.000000}%
\pgfsetstrokecolor{currentstroke}%
\pgfsetdash{}{0pt}%
\pgfpathmoveto{\pgfqpoint{5.577878in}{1.157964in}}%
\pgfpathlineto{\pgfqpoint{5.764104in}{1.157964in}}%
\pgfpathlineto{\pgfqpoint{5.764104in}{1.239692in}}%
\pgfpathlineto{\pgfqpoint{5.577878in}{1.239692in}}%
\pgfpathlineto{\pgfqpoint{5.577878in}{1.157964in}}%
\pgfusepath{}%
\end{pgfscope}%
\begin{pgfscope}%
\pgfpathrectangle{\pgfqpoint{0.549740in}{0.463273in}}{\pgfqpoint{9.320225in}{4.495057in}}%
\pgfusepath{clip}%
\pgfsetbuttcap%
\pgfsetroundjoin%
\definecolor{currentfill}{rgb}{0.472869,0.711325,0.955316}%
\pgfsetfillcolor{currentfill}%
\pgfsetlinewidth{0.000000pt}%
\definecolor{currentstroke}{rgb}{0.000000,0.000000,0.000000}%
\pgfsetstrokecolor{currentstroke}%
\pgfsetdash{}{0pt}%
\pgfpathmoveto{\pgfqpoint{5.764104in}{1.157964in}}%
\pgfpathlineto{\pgfqpoint{5.950331in}{1.157964in}}%
\pgfpathlineto{\pgfqpoint{5.950331in}{1.239692in}}%
\pgfpathlineto{\pgfqpoint{5.764104in}{1.239692in}}%
\pgfpathlineto{\pgfqpoint{5.764104in}{1.157964in}}%
\pgfusepath{fill}%
\end{pgfscope}%
\begin{pgfscope}%
\pgfpathrectangle{\pgfqpoint{0.549740in}{0.463273in}}{\pgfqpoint{9.320225in}{4.495057in}}%
\pgfusepath{clip}%
\pgfsetbuttcap%
\pgfsetroundjoin%
\pgfsetlinewidth{0.000000pt}%
\definecolor{currentstroke}{rgb}{0.000000,0.000000,0.000000}%
\pgfsetstrokecolor{currentstroke}%
\pgfsetdash{}{0pt}%
\pgfpathmoveto{\pgfqpoint{5.950331in}{1.157964in}}%
\pgfpathlineto{\pgfqpoint{6.136557in}{1.157964in}}%
\pgfpathlineto{\pgfqpoint{6.136557in}{1.239692in}}%
\pgfpathlineto{\pgfqpoint{5.950331in}{1.239692in}}%
\pgfpathlineto{\pgfqpoint{5.950331in}{1.157964in}}%
\pgfusepath{}%
\end{pgfscope}%
\begin{pgfscope}%
\pgfpathrectangle{\pgfqpoint{0.549740in}{0.463273in}}{\pgfqpoint{9.320225in}{4.495057in}}%
\pgfusepath{clip}%
\pgfsetbuttcap%
\pgfsetroundjoin%
\pgfsetlinewidth{0.000000pt}%
\definecolor{currentstroke}{rgb}{0.000000,0.000000,0.000000}%
\pgfsetstrokecolor{currentstroke}%
\pgfsetdash{}{0pt}%
\pgfpathmoveto{\pgfqpoint{6.136557in}{1.157964in}}%
\pgfpathlineto{\pgfqpoint{6.322784in}{1.157964in}}%
\pgfpathlineto{\pgfqpoint{6.322784in}{1.239692in}}%
\pgfpathlineto{\pgfqpoint{6.136557in}{1.239692in}}%
\pgfpathlineto{\pgfqpoint{6.136557in}{1.157964in}}%
\pgfusepath{}%
\end{pgfscope}%
\begin{pgfscope}%
\pgfpathrectangle{\pgfqpoint{0.549740in}{0.463273in}}{\pgfqpoint{9.320225in}{4.495057in}}%
\pgfusepath{clip}%
\pgfsetbuttcap%
\pgfsetroundjoin%
\pgfsetlinewidth{0.000000pt}%
\definecolor{currentstroke}{rgb}{0.000000,0.000000,0.000000}%
\pgfsetstrokecolor{currentstroke}%
\pgfsetdash{}{0pt}%
\pgfpathmoveto{\pgfqpoint{6.322784in}{1.157964in}}%
\pgfpathlineto{\pgfqpoint{6.509011in}{1.157964in}}%
\pgfpathlineto{\pgfqpoint{6.509011in}{1.239692in}}%
\pgfpathlineto{\pgfqpoint{6.322784in}{1.239692in}}%
\pgfpathlineto{\pgfqpoint{6.322784in}{1.157964in}}%
\pgfusepath{}%
\end{pgfscope}%
\begin{pgfscope}%
\pgfpathrectangle{\pgfqpoint{0.549740in}{0.463273in}}{\pgfqpoint{9.320225in}{4.495057in}}%
\pgfusepath{clip}%
\pgfsetbuttcap%
\pgfsetroundjoin%
\pgfsetlinewidth{0.000000pt}%
\definecolor{currentstroke}{rgb}{0.000000,0.000000,0.000000}%
\pgfsetstrokecolor{currentstroke}%
\pgfsetdash{}{0pt}%
\pgfpathmoveto{\pgfqpoint{6.509011in}{1.157964in}}%
\pgfpathlineto{\pgfqpoint{6.695237in}{1.157964in}}%
\pgfpathlineto{\pgfqpoint{6.695237in}{1.239692in}}%
\pgfpathlineto{\pgfqpoint{6.509011in}{1.239692in}}%
\pgfpathlineto{\pgfqpoint{6.509011in}{1.157964in}}%
\pgfusepath{}%
\end{pgfscope}%
\begin{pgfscope}%
\pgfpathrectangle{\pgfqpoint{0.549740in}{0.463273in}}{\pgfqpoint{9.320225in}{4.495057in}}%
\pgfusepath{clip}%
\pgfsetbuttcap%
\pgfsetroundjoin%
\definecolor{currentfill}{rgb}{0.472869,0.711325,0.955316}%
\pgfsetfillcolor{currentfill}%
\pgfsetlinewidth{0.000000pt}%
\definecolor{currentstroke}{rgb}{0.000000,0.000000,0.000000}%
\pgfsetstrokecolor{currentstroke}%
\pgfsetdash{}{0pt}%
\pgfpathmoveto{\pgfqpoint{6.695237in}{1.157964in}}%
\pgfpathlineto{\pgfqpoint{6.881464in}{1.157964in}}%
\pgfpathlineto{\pgfqpoint{6.881464in}{1.239692in}}%
\pgfpathlineto{\pgfqpoint{6.695237in}{1.239692in}}%
\pgfpathlineto{\pgfqpoint{6.695237in}{1.157964in}}%
\pgfusepath{fill}%
\end{pgfscope}%
\begin{pgfscope}%
\pgfpathrectangle{\pgfqpoint{0.549740in}{0.463273in}}{\pgfqpoint{9.320225in}{4.495057in}}%
\pgfusepath{clip}%
\pgfsetbuttcap%
\pgfsetroundjoin%
\pgfsetlinewidth{0.000000pt}%
\definecolor{currentstroke}{rgb}{0.000000,0.000000,0.000000}%
\pgfsetstrokecolor{currentstroke}%
\pgfsetdash{}{0pt}%
\pgfpathmoveto{\pgfqpoint{6.881464in}{1.157964in}}%
\pgfpathlineto{\pgfqpoint{7.067690in}{1.157964in}}%
\pgfpathlineto{\pgfqpoint{7.067690in}{1.239692in}}%
\pgfpathlineto{\pgfqpoint{6.881464in}{1.239692in}}%
\pgfpathlineto{\pgfqpoint{6.881464in}{1.157964in}}%
\pgfusepath{}%
\end{pgfscope}%
\begin{pgfscope}%
\pgfpathrectangle{\pgfqpoint{0.549740in}{0.463273in}}{\pgfqpoint{9.320225in}{4.495057in}}%
\pgfusepath{clip}%
\pgfsetbuttcap%
\pgfsetroundjoin%
\pgfsetlinewidth{0.000000pt}%
\definecolor{currentstroke}{rgb}{0.000000,0.000000,0.000000}%
\pgfsetstrokecolor{currentstroke}%
\pgfsetdash{}{0pt}%
\pgfpathmoveto{\pgfqpoint{7.067690in}{1.157964in}}%
\pgfpathlineto{\pgfqpoint{7.253917in}{1.157964in}}%
\pgfpathlineto{\pgfqpoint{7.253917in}{1.239692in}}%
\pgfpathlineto{\pgfqpoint{7.067690in}{1.239692in}}%
\pgfpathlineto{\pgfqpoint{7.067690in}{1.157964in}}%
\pgfusepath{}%
\end{pgfscope}%
\begin{pgfscope}%
\pgfpathrectangle{\pgfqpoint{0.549740in}{0.463273in}}{\pgfqpoint{9.320225in}{4.495057in}}%
\pgfusepath{clip}%
\pgfsetbuttcap%
\pgfsetroundjoin%
\pgfsetlinewidth{0.000000pt}%
\definecolor{currentstroke}{rgb}{0.000000,0.000000,0.000000}%
\pgfsetstrokecolor{currentstroke}%
\pgfsetdash{}{0pt}%
\pgfpathmoveto{\pgfqpoint{7.253917in}{1.157964in}}%
\pgfpathlineto{\pgfqpoint{7.440143in}{1.157964in}}%
\pgfpathlineto{\pgfqpoint{7.440143in}{1.239692in}}%
\pgfpathlineto{\pgfqpoint{7.253917in}{1.239692in}}%
\pgfpathlineto{\pgfqpoint{7.253917in}{1.157964in}}%
\pgfusepath{}%
\end{pgfscope}%
\begin{pgfscope}%
\pgfpathrectangle{\pgfqpoint{0.549740in}{0.463273in}}{\pgfqpoint{9.320225in}{4.495057in}}%
\pgfusepath{clip}%
\pgfsetbuttcap%
\pgfsetroundjoin%
\pgfsetlinewidth{0.000000pt}%
\definecolor{currentstroke}{rgb}{0.000000,0.000000,0.000000}%
\pgfsetstrokecolor{currentstroke}%
\pgfsetdash{}{0pt}%
\pgfpathmoveto{\pgfqpoint{7.440143in}{1.157964in}}%
\pgfpathlineto{\pgfqpoint{7.626370in}{1.157964in}}%
\pgfpathlineto{\pgfqpoint{7.626370in}{1.239692in}}%
\pgfpathlineto{\pgfqpoint{7.440143in}{1.239692in}}%
\pgfpathlineto{\pgfqpoint{7.440143in}{1.157964in}}%
\pgfusepath{}%
\end{pgfscope}%
\begin{pgfscope}%
\pgfpathrectangle{\pgfqpoint{0.549740in}{0.463273in}}{\pgfqpoint{9.320225in}{4.495057in}}%
\pgfusepath{clip}%
\pgfsetbuttcap%
\pgfsetroundjoin%
\pgfsetlinewidth{0.000000pt}%
\definecolor{currentstroke}{rgb}{0.000000,0.000000,0.000000}%
\pgfsetstrokecolor{currentstroke}%
\pgfsetdash{}{0pt}%
\pgfpathmoveto{\pgfqpoint{7.626370in}{1.157964in}}%
\pgfpathlineto{\pgfqpoint{7.812596in}{1.157964in}}%
\pgfpathlineto{\pgfqpoint{7.812596in}{1.239692in}}%
\pgfpathlineto{\pgfqpoint{7.626370in}{1.239692in}}%
\pgfpathlineto{\pgfqpoint{7.626370in}{1.157964in}}%
\pgfusepath{}%
\end{pgfscope}%
\begin{pgfscope}%
\pgfpathrectangle{\pgfqpoint{0.549740in}{0.463273in}}{\pgfqpoint{9.320225in}{4.495057in}}%
\pgfusepath{clip}%
\pgfsetbuttcap%
\pgfsetroundjoin%
\pgfsetlinewidth{0.000000pt}%
\definecolor{currentstroke}{rgb}{0.000000,0.000000,0.000000}%
\pgfsetstrokecolor{currentstroke}%
\pgfsetdash{}{0pt}%
\pgfpathmoveto{\pgfqpoint{7.812596in}{1.157964in}}%
\pgfpathlineto{\pgfqpoint{7.998823in}{1.157964in}}%
\pgfpathlineto{\pgfqpoint{7.998823in}{1.239692in}}%
\pgfpathlineto{\pgfqpoint{7.812596in}{1.239692in}}%
\pgfpathlineto{\pgfqpoint{7.812596in}{1.157964in}}%
\pgfusepath{}%
\end{pgfscope}%
\begin{pgfscope}%
\pgfpathrectangle{\pgfqpoint{0.549740in}{0.463273in}}{\pgfqpoint{9.320225in}{4.495057in}}%
\pgfusepath{clip}%
\pgfsetbuttcap%
\pgfsetroundjoin%
\pgfsetlinewidth{0.000000pt}%
\definecolor{currentstroke}{rgb}{0.000000,0.000000,0.000000}%
\pgfsetstrokecolor{currentstroke}%
\pgfsetdash{}{0pt}%
\pgfpathmoveto{\pgfqpoint{7.998823in}{1.157964in}}%
\pgfpathlineto{\pgfqpoint{8.185049in}{1.157964in}}%
\pgfpathlineto{\pgfqpoint{8.185049in}{1.239692in}}%
\pgfpathlineto{\pgfqpoint{7.998823in}{1.239692in}}%
\pgfpathlineto{\pgfqpoint{7.998823in}{1.157964in}}%
\pgfusepath{}%
\end{pgfscope}%
\begin{pgfscope}%
\pgfpathrectangle{\pgfqpoint{0.549740in}{0.463273in}}{\pgfqpoint{9.320225in}{4.495057in}}%
\pgfusepath{clip}%
\pgfsetbuttcap%
\pgfsetroundjoin%
\pgfsetlinewidth{0.000000pt}%
\definecolor{currentstroke}{rgb}{0.000000,0.000000,0.000000}%
\pgfsetstrokecolor{currentstroke}%
\pgfsetdash{}{0pt}%
\pgfpathmoveto{\pgfqpoint{8.185049in}{1.157964in}}%
\pgfpathlineto{\pgfqpoint{8.371276in}{1.157964in}}%
\pgfpathlineto{\pgfqpoint{8.371276in}{1.239692in}}%
\pgfpathlineto{\pgfqpoint{8.185049in}{1.239692in}}%
\pgfpathlineto{\pgfqpoint{8.185049in}{1.157964in}}%
\pgfusepath{}%
\end{pgfscope}%
\begin{pgfscope}%
\pgfpathrectangle{\pgfqpoint{0.549740in}{0.463273in}}{\pgfqpoint{9.320225in}{4.495057in}}%
\pgfusepath{clip}%
\pgfsetbuttcap%
\pgfsetroundjoin%
\pgfsetlinewidth{0.000000pt}%
\definecolor{currentstroke}{rgb}{0.000000,0.000000,0.000000}%
\pgfsetstrokecolor{currentstroke}%
\pgfsetdash{}{0pt}%
\pgfpathmoveto{\pgfqpoint{8.371276in}{1.157964in}}%
\pgfpathlineto{\pgfqpoint{8.557503in}{1.157964in}}%
\pgfpathlineto{\pgfqpoint{8.557503in}{1.239692in}}%
\pgfpathlineto{\pgfqpoint{8.371276in}{1.239692in}}%
\pgfpathlineto{\pgfqpoint{8.371276in}{1.157964in}}%
\pgfusepath{}%
\end{pgfscope}%
\begin{pgfscope}%
\pgfpathrectangle{\pgfqpoint{0.549740in}{0.463273in}}{\pgfqpoint{9.320225in}{4.495057in}}%
\pgfusepath{clip}%
\pgfsetbuttcap%
\pgfsetroundjoin%
\pgfsetlinewidth{0.000000pt}%
\definecolor{currentstroke}{rgb}{0.000000,0.000000,0.000000}%
\pgfsetstrokecolor{currentstroke}%
\pgfsetdash{}{0pt}%
\pgfpathmoveto{\pgfqpoint{8.557503in}{1.157964in}}%
\pgfpathlineto{\pgfqpoint{8.743729in}{1.157964in}}%
\pgfpathlineto{\pgfqpoint{8.743729in}{1.239692in}}%
\pgfpathlineto{\pgfqpoint{8.557503in}{1.239692in}}%
\pgfpathlineto{\pgfqpoint{8.557503in}{1.157964in}}%
\pgfusepath{}%
\end{pgfscope}%
\begin{pgfscope}%
\pgfpathrectangle{\pgfqpoint{0.549740in}{0.463273in}}{\pgfqpoint{9.320225in}{4.495057in}}%
\pgfusepath{clip}%
\pgfsetbuttcap%
\pgfsetroundjoin%
\pgfsetlinewidth{0.000000pt}%
\definecolor{currentstroke}{rgb}{0.000000,0.000000,0.000000}%
\pgfsetstrokecolor{currentstroke}%
\pgfsetdash{}{0pt}%
\pgfpathmoveto{\pgfqpoint{8.743729in}{1.157964in}}%
\pgfpathlineto{\pgfqpoint{8.929956in}{1.157964in}}%
\pgfpathlineto{\pgfqpoint{8.929956in}{1.239692in}}%
\pgfpathlineto{\pgfqpoint{8.743729in}{1.239692in}}%
\pgfpathlineto{\pgfqpoint{8.743729in}{1.157964in}}%
\pgfusepath{}%
\end{pgfscope}%
\begin{pgfscope}%
\pgfpathrectangle{\pgfqpoint{0.549740in}{0.463273in}}{\pgfqpoint{9.320225in}{4.495057in}}%
\pgfusepath{clip}%
\pgfsetbuttcap%
\pgfsetroundjoin%
\pgfsetlinewidth{0.000000pt}%
\definecolor{currentstroke}{rgb}{0.000000,0.000000,0.000000}%
\pgfsetstrokecolor{currentstroke}%
\pgfsetdash{}{0pt}%
\pgfpathmoveto{\pgfqpoint{8.929956in}{1.157964in}}%
\pgfpathlineto{\pgfqpoint{9.116182in}{1.157964in}}%
\pgfpathlineto{\pgfqpoint{9.116182in}{1.239692in}}%
\pgfpathlineto{\pgfqpoint{8.929956in}{1.239692in}}%
\pgfpathlineto{\pgfqpoint{8.929956in}{1.157964in}}%
\pgfusepath{}%
\end{pgfscope}%
\begin{pgfscope}%
\pgfpathrectangle{\pgfqpoint{0.549740in}{0.463273in}}{\pgfqpoint{9.320225in}{4.495057in}}%
\pgfusepath{clip}%
\pgfsetbuttcap%
\pgfsetroundjoin%
\pgfsetlinewidth{0.000000pt}%
\definecolor{currentstroke}{rgb}{0.000000,0.000000,0.000000}%
\pgfsetstrokecolor{currentstroke}%
\pgfsetdash{}{0pt}%
\pgfpathmoveto{\pgfqpoint{9.116182in}{1.157964in}}%
\pgfpathlineto{\pgfqpoint{9.302409in}{1.157964in}}%
\pgfpathlineto{\pgfqpoint{9.302409in}{1.239692in}}%
\pgfpathlineto{\pgfqpoint{9.116182in}{1.239692in}}%
\pgfpathlineto{\pgfqpoint{9.116182in}{1.157964in}}%
\pgfusepath{}%
\end{pgfscope}%
\begin{pgfscope}%
\pgfpathrectangle{\pgfqpoint{0.549740in}{0.463273in}}{\pgfqpoint{9.320225in}{4.495057in}}%
\pgfusepath{clip}%
\pgfsetbuttcap%
\pgfsetroundjoin%
\pgfsetlinewidth{0.000000pt}%
\definecolor{currentstroke}{rgb}{0.000000,0.000000,0.000000}%
\pgfsetstrokecolor{currentstroke}%
\pgfsetdash{}{0pt}%
\pgfpathmoveto{\pgfqpoint{9.302409in}{1.157964in}}%
\pgfpathlineto{\pgfqpoint{9.488635in}{1.157964in}}%
\pgfpathlineto{\pgfqpoint{9.488635in}{1.239692in}}%
\pgfpathlineto{\pgfqpoint{9.302409in}{1.239692in}}%
\pgfpathlineto{\pgfqpoint{9.302409in}{1.157964in}}%
\pgfusepath{}%
\end{pgfscope}%
\begin{pgfscope}%
\pgfpathrectangle{\pgfqpoint{0.549740in}{0.463273in}}{\pgfqpoint{9.320225in}{4.495057in}}%
\pgfusepath{clip}%
\pgfsetbuttcap%
\pgfsetroundjoin%
\pgfsetlinewidth{0.000000pt}%
\definecolor{currentstroke}{rgb}{0.000000,0.000000,0.000000}%
\pgfsetstrokecolor{currentstroke}%
\pgfsetdash{}{0pt}%
\pgfpathmoveto{\pgfqpoint{9.488635in}{1.157964in}}%
\pgfpathlineto{\pgfqpoint{9.674862in}{1.157964in}}%
\pgfpathlineto{\pgfqpoint{9.674862in}{1.239692in}}%
\pgfpathlineto{\pgfqpoint{9.488635in}{1.239692in}}%
\pgfpathlineto{\pgfqpoint{9.488635in}{1.157964in}}%
\pgfusepath{}%
\end{pgfscope}%
\begin{pgfscope}%
\pgfpathrectangle{\pgfqpoint{0.549740in}{0.463273in}}{\pgfqpoint{9.320225in}{4.495057in}}%
\pgfusepath{clip}%
\pgfsetbuttcap%
\pgfsetroundjoin%
\pgfsetlinewidth{0.000000pt}%
\definecolor{currentstroke}{rgb}{0.000000,0.000000,0.000000}%
\pgfsetstrokecolor{currentstroke}%
\pgfsetdash{}{0pt}%
\pgfpathmoveto{\pgfqpoint{9.674862in}{1.157964in}}%
\pgfpathlineto{\pgfqpoint{9.861088in}{1.157964in}}%
\pgfpathlineto{\pgfqpoint{9.861088in}{1.239692in}}%
\pgfpathlineto{\pgfqpoint{9.674862in}{1.239692in}}%
\pgfpathlineto{\pgfqpoint{9.674862in}{1.157964in}}%
\pgfusepath{}%
\end{pgfscope}%
\begin{pgfscope}%
\pgfpathrectangle{\pgfqpoint{0.549740in}{0.463273in}}{\pgfqpoint{9.320225in}{4.495057in}}%
\pgfusepath{clip}%
\pgfsetbuttcap%
\pgfsetroundjoin%
\pgfsetlinewidth{0.000000pt}%
\definecolor{currentstroke}{rgb}{0.000000,0.000000,0.000000}%
\pgfsetstrokecolor{currentstroke}%
\pgfsetdash{}{0pt}%
\pgfpathmoveto{\pgfqpoint{0.549761in}{1.239692in}}%
\pgfpathlineto{\pgfqpoint{0.735988in}{1.239692in}}%
\pgfpathlineto{\pgfqpoint{0.735988in}{1.321421in}}%
\pgfpathlineto{\pgfqpoint{0.549761in}{1.321421in}}%
\pgfpathlineto{\pgfqpoint{0.549761in}{1.239692in}}%
\pgfusepath{}%
\end{pgfscope}%
\begin{pgfscope}%
\pgfpathrectangle{\pgfqpoint{0.549740in}{0.463273in}}{\pgfqpoint{9.320225in}{4.495057in}}%
\pgfusepath{clip}%
\pgfsetbuttcap%
\pgfsetroundjoin%
\pgfsetlinewidth{0.000000pt}%
\definecolor{currentstroke}{rgb}{0.000000,0.000000,0.000000}%
\pgfsetstrokecolor{currentstroke}%
\pgfsetdash{}{0pt}%
\pgfpathmoveto{\pgfqpoint{0.735988in}{1.239692in}}%
\pgfpathlineto{\pgfqpoint{0.922214in}{1.239692in}}%
\pgfpathlineto{\pgfqpoint{0.922214in}{1.321421in}}%
\pgfpathlineto{\pgfqpoint{0.735988in}{1.321421in}}%
\pgfpathlineto{\pgfqpoint{0.735988in}{1.239692in}}%
\pgfusepath{}%
\end{pgfscope}%
\begin{pgfscope}%
\pgfpathrectangle{\pgfqpoint{0.549740in}{0.463273in}}{\pgfqpoint{9.320225in}{4.495057in}}%
\pgfusepath{clip}%
\pgfsetbuttcap%
\pgfsetroundjoin%
\pgfsetlinewidth{0.000000pt}%
\definecolor{currentstroke}{rgb}{0.000000,0.000000,0.000000}%
\pgfsetstrokecolor{currentstroke}%
\pgfsetdash{}{0pt}%
\pgfpathmoveto{\pgfqpoint{0.922214in}{1.239692in}}%
\pgfpathlineto{\pgfqpoint{1.108441in}{1.239692in}}%
\pgfpathlineto{\pgfqpoint{1.108441in}{1.321421in}}%
\pgfpathlineto{\pgfqpoint{0.922214in}{1.321421in}}%
\pgfpathlineto{\pgfqpoint{0.922214in}{1.239692in}}%
\pgfusepath{}%
\end{pgfscope}%
\begin{pgfscope}%
\pgfpathrectangle{\pgfqpoint{0.549740in}{0.463273in}}{\pgfqpoint{9.320225in}{4.495057in}}%
\pgfusepath{clip}%
\pgfsetbuttcap%
\pgfsetroundjoin%
\pgfsetlinewidth{0.000000pt}%
\definecolor{currentstroke}{rgb}{0.000000,0.000000,0.000000}%
\pgfsetstrokecolor{currentstroke}%
\pgfsetdash{}{0pt}%
\pgfpathmoveto{\pgfqpoint{1.108441in}{1.239692in}}%
\pgfpathlineto{\pgfqpoint{1.294667in}{1.239692in}}%
\pgfpathlineto{\pgfqpoint{1.294667in}{1.321421in}}%
\pgfpathlineto{\pgfqpoint{1.108441in}{1.321421in}}%
\pgfpathlineto{\pgfqpoint{1.108441in}{1.239692in}}%
\pgfusepath{}%
\end{pgfscope}%
\begin{pgfscope}%
\pgfpathrectangle{\pgfqpoint{0.549740in}{0.463273in}}{\pgfqpoint{9.320225in}{4.495057in}}%
\pgfusepath{clip}%
\pgfsetbuttcap%
\pgfsetroundjoin%
\pgfsetlinewidth{0.000000pt}%
\definecolor{currentstroke}{rgb}{0.000000,0.000000,0.000000}%
\pgfsetstrokecolor{currentstroke}%
\pgfsetdash{}{0pt}%
\pgfpathmoveto{\pgfqpoint{1.294667in}{1.239692in}}%
\pgfpathlineto{\pgfqpoint{1.480894in}{1.239692in}}%
\pgfpathlineto{\pgfqpoint{1.480894in}{1.321421in}}%
\pgfpathlineto{\pgfqpoint{1.294667in}{1.321421in}}%
\pgfpathlineto{\pgfqpoint{1.294667in}{1.239692in}}%
\pgfusepath{}%
\end{pgfscope}%
\begin{pgfscope}%
\pgfpathrectangle{\pgfqpoint{0.549740in}{0.463273in}}{\pgfqpoint{9.320225in}{4.495057in}}%
\pgfusepath{clip}%
\pgfsetbuttcap%
\pgfsetroundjoin%
\pgfsetlinewidth{0.000000pt}%
\definecolor{currentstroke}{rgb}{0.000000,0.000000,0.000000}%
\pgfsetstrokecolor{currentstroke}%
\pgfsetdash{}{0pt}%
\pgfpathmoveto{\pgfqpoint{1.480894in}{1.239692in}}%
\pgfpathlineto{\pgfqpoint{1.667120in}{1.239692in}}%
\pgfpathlineto{\pgfqpoint{1.667120in}{1.321421in}}%
\pgfpathlineto{\pgfqpoint{1.480894in}{1.321421in}}%
\pgfpathlineto{\pgfqpoint{1.480894in}{1.239692in}}%
\pgfusepath{}%
\end{pgfscope}%
\begin{pgfscope}%
\pgfpathrectangle{\pgfqpoint{0.549740in}{0.463273in}}{\pgfqpoint{9.320225in}{4.495057in}}%
\pgfusepath{clip}%
\pgfsetbuttcap%
\pgfsetroundjoin%
\pgfsetlinewidth{0.000000pt}%
\definecolor{currentstroke}{rgb}{0.000000,0.000000,0.000000}%
\pgfsetstrokecolor{currentstroke}%
\pgfsetdash{}{0pt}%
\pgfpathmoveto{\pgfqpoint{1.667120in}{1.239692in}}%
\pgfpathlineto{\pgfqpoint{1.853347in}{1.239692in}}%
\pgfpathlineto{\pgfqpoint{1.853347in}{1.321421in}}%
\pgfpathlineto{\pgfqpoint{1.667120in}{1.321421in}}%
\pgfpathlineto{\pgfqpoint{1.667120in}{1.239692in}}%
\pgfusepath{}%
\end{pgfscope}%
\begin{pgfscope}%
\pgfpathrectangle{\pgfqpoint{0.549740in}{0.463273in}}{\pgfqpoint{9.320225in}{4.495057in}}%
\pgfusepath{clip}%
\pgfsetbuttcap%
\pgfsetroundjoin%
\pgfsetlinewidth{0.000000pt}%
\definecolor{currentstroke}{rgb}{0.000000,0.000000,0.000000}%
\pgfsetstrokecolor{currentstroke}%
\pgfsetdash{}{0pt}%
\pgfpathmoveto{\pgfqpoint{1.853347in}{1.239692in}}%
\pgfpathlineto{\pgfqpoint{2.039573in}{1.239692in}}%
\pgfpathlineto{\pgfqpoint{2.039573in}{1.321421in}}%
\pgfpathlineto{\pgfqpoint{1.853347in}{1.321421in}}%
\pgfpathlineto{\pgfqpoint{1.853347in}{1.239692in}}%
\pgfusepath{}%
\end{pgfscope}%
\begin{pgfscope}%
\pgfpathrectangle{\pgfqpoint{0.549740in}{0.463273in}}{\pgfqpoint{9.320225in}{4.495057in}}%
\pgfusepath{clip}%
\pgfsetbuttcap%
\pgfsetroundjoin%
\pgfsetlinewidth{0.000000pt}%
\definecolor{currentstroke}{rgb}{0.000000,0.000000,0.000000}%
\pgfsetstrokecolor{currentstroke}%
\pgfsetdash{}{0pt}%
\pgfpathmoveto{\pgfqpoint{2.039573in}{1.239692in}}%
\pgfpathlineto{\pgfqpoint{2.225800in}{1.239692in}}%
\pgfpathlineto{\pgfqpoint{2.225800in}{1.321421in}}%
\pgfpathlineto{\pgfqpoint{2.039573in}{1.321421in}}%
\pgfpathlineto{\pgfqpoint{2.039573in}{1.239692in}}%
\pgfusepath{}%
\end{pgfscope}%
\begin{pgfscope}%
\pgfpathrectangle{\pgfqpoint{0.549740in}{0.463273in}}{\pgfqpoint{9.320225in}{4.495057in}}%
\pgfusepath{clip}%
\pgfsetbuttcap%
\pgfsetroundjoin%
\pgfsetlinewidth{0.000000pt}%
\definecolor{currentstroke}{rgb}{0.000000,0.000000,0.000000}%
\pgfsetstrokecolor{currentstroke}%
\pgfsetdash{}{0pt}%
\pgfpathmoveto{\pgfqpoint{2.225800in}{1.239692in}}%
\pgfpathlineto{\pgfqpoint{2.412027in}{1.239692in}}%
\pgfpathlineto{\pgfqpoint{2.412027in}{1.321421in}}%
\pgfpathlineto{\pgfqpoint{2.225800in}{1.321421in}}%
\pgfpathlineto{\pgfqpoint{2.225800in}{1.239692in}}%
\pgfusepath{}%
\end{pgfscope}%
\begin{pgfscope}%
\pgfpathrectangle{\pgfqpoint{0.549740in}{0.463273in}}{\pgfqpoint{9.320225in}{4.495057in}}%
\pgfusepath{clip}%
\pgfsetbuttcap%
\pgfsetroundjoin%
\pgfsetlinewidth{0.000000pt}%
\definecolor{currentstroke}{rgb}{0.000000,0.000000,0.000000}%
\pgfsetstrokecolor{currentstroke}%
\pgfsetdash{}{0pt}%
\pgfpathmoveto{\pgfqpoint{2.412027in}{1.239692in}}%
\pgfpathlineto{\pgfqpoint{2.598253in}{1.239692in}}%
\pgfpathlineto{\pgfqpoint{2.598253in}{1.321421in}}%
\pgfpathlineto{\pgfqpoint{2.412027in}{1.321421in}}%
\pgfpathlineto{\pgfqpoint{2.412027in}{1.239692in}}%
\pgfusepath{}%
\end{pgfscope}%
\begin{pgfscope}%
\pgfpathrectangle{\pgfqpoint{0.549740in}{0.463273in}}{\pgfqpoint{9.320225in}{4.495057in}}%
\pgfusepath{clip}%
\pgfsetbuttcap%
\pgfsetroundjoin%
\pgfsetlinewidth{0.000000pt}%
\definecolor{currentstroke}{rgb}{0.000000,0.000000,0.000000}%
\pgfsetstrokecolor{currentstroke}%
\pgfsetdash{}{0pt}%
\pgfpathmoveto{\pgfqpoint{2.598253in}{1.239692in}}%
\pgfpathlineto{\pgfqpoint{2.784480in}{1.239692in}}%
\pgfpathlineto{\pgfqpoint{2.784480in}{1.321421in}}%
\pgfpathlineto{\pgfqpoint{2.598253in}{1.321421in}}%
\pgfpathlineto{\pgfqpoint{2.598253in}{1.239692in}}%
\pgfusepath{}%
\end{pgfscope}%
\begin{pgfscope}%
\pgfpathrectangle{\pgfqpoint{0.549740in}{0.463273in}}{\pgfqpoint{9.320225in}{4.495057in}}%
\pgfusepath{clip}%
\pgfsetbuttcap%
\pgfsetroundjoin%
\pgfsetlinewidth{0.000000pt}%
\definecolor{currentstroke}{rgb}{0.000000,0.000000,0.000000}%
\pgfsetstrokecolor{currentstroke}%
\pgfsetdash{}{0pt}%
\pgfpathmoveto{\pgfqpoint{2.784480in}{1.239692in}}%
\pgfpathlineto{\pgfqpoint{2.970706in}{1.239692in}}%
\pgfpathlineto{\pgfqpoint{2.970706in}{1.321421in}}%
\pgfpathlineto{\pgfqpoint{2.784480in}{1.321421in}}%
\pgfpathlineto{\pgfqpoint{2.784480in}{1.239692in}}%
\pgfusepath{}%
\end{pgfscope}%
\begin{pgfscope}%
\pgfpathrectangle{\pgfqpoint{0.549740in}{0.463273in}}{\pgfqpoint{9.320225in}{4.495057in}}%
\pgfusepath{clip}%
\pgfsetbuttcap%
\pgfsetroundjoin%
\pgfsetlinewidth{0.000000pt}%
\definecolor{currentstroke}{rgb}{0.000000,0.000000,0.000000}%
\pgfsetstrokecolor{currentstroke}%
\pgfsetdash{}{0pt}%
\pgfpathmoveto{\pgfqpoint{2.970706in}{1.239692in}}%
\pgfpathlineto{\pgfqpoint{3.156933in}{1.239692in}}%
\pgfpathlineto{\pgfqpoint{3.156933in}{1.321421in}}%
\pgfpathlineto{\pgfqpoint{2.970706in}{1.321421in}}%
\pgfpathlineto{\pgfqpoint{2.970706in}{1.239692in}}%
\pgfusepath{}%
\end{pgfscope}%
\begin{pgfscope}%
\pgfpathrectangle{\pgfqpoint{0.549740in}{0.463273in}}{\pgfqpoint{9.320225in}{4.495057in}}%
\pgfusepath{clip}%
\pgfsetbuttcap%
\pgfsetroundjoin%
\pgfsetlinewidth{0.000000pt}%
\definecolor{currentstroke}{rgb}{0.000000,0.000000,0.000000}%
\pgfsetstrokecolor{currentstroke}%
\pgfsetdash{}{0pt}%
\pgfpathmoveto{\pgfqpoint{3.156933in}{1.239692in}}%
\pgfpathlineto{\pgfqpoint{3.343159in}{1.239692in}}%
\pgfpathlineto{\pgfqpoint{3.343159in}{1.321421in}}%
\pgfpathlineto{\pgfqpoint{3.156933in}{1.321421in}}%
\pgfpathlineto{\pgfqpoint{3.156933in}{1.239692in}}%
\pgfusepath{}%
\end{pgfscope}%
\begin{pgfscope}%
\pgfpathrectangle{\pgfqpoint{0.549740in}{0.463273in}}{\pgfqpoint{9.320225in}{4.495057in}}%
\pgfusepath{clip}%
\pgfsetbuttcap%
\pgfsetroundjoin%
\pgfsetlinewidth{0.000000pt}%
\definecolor{currentstroke}{rgb}{0.000000,0.000000,0.000000}%
\pgfsetstrokecolor{currentstroke}%
\pgfsetdash{}{0pt}%
\pgfpathmoveto{\pgfqpoint{3.343159in}{1.239692in}}%
\pgfpathlineto{\pgfqpoint{3.529386in}{1.239692in}}%
\pgfpathlineto{\pgfqpoint{3.529386in}{1.321421in}}%
\pgfpathlineto{\pgfqpoint{3.343159in}{1.321421in}}%
\pgfpathlineto{\pgfqpoint{3.343159in}{1.239692in}}%
\pgfusepath{}%
\end{pgfscope}%
\begin{pgfscope}%
\pgfpathrectangle{\pgfqpoint{0.549740in}{0.463273in}}{\pgfqpoint{9.320225in}{4.495057in}}%
\pgfusepath{clip}%
\pgfsetbuttcap%
\pgfsetroundjoin%
\pgfsetlinewidth{0.000000pt}%
\definecolor{currentstroke}{rgb}{0.000000,0.000000,0.000000}%
\pgfsetstrokecolor{currentstroke}%
\pgfsetdash{}{0pt}%
\pgfpathmoveto{\pgfqpoint{3.529386in}{1.239692in}}%
\pgfpathlineto{\pgfqpoint{3.715612in}{1.239692in}}%
\pgfpathlineto{\pgfqpoint{3.715612in}{1.321421in}}%
\pgfpathlineto{\pgfqpoint{3.529386in}{1.321421in}}%
\pgfpathlineto{\pgfqpoint{3.529386in}{1.239692in}}%
\pgfusepath{}%
\end{pgfscope}%
\begin{pgfscope}%
\pgfpathrectangle{\pgfqpoint{0.549740in}{0.463273in}}{\pgfqpoint{9.320225in}{4.495057in}}%
\pgfusepath{clip}%
\pgfsetbuttcap%
\pgfsetroundjoin%
\pgfsetlinewidth{0.000000pt}%
\definecolor{currentstroke}{rgb}{0.000000,0.000000,0.000000}%
\pgfsetstrokecolor{currentstroke}%
\pgfsetdash{}{0pt}%
\pgfpathmoveto{\pgfqpoint{3.715612in}{1.239692in}}%
\pgfpathlineto{\pgfqpoint{3.901839in}{1.239692in}}%
\pgfpathlineto{\pgfqpoint{3.901839in}{1.321421in}}%
\pgfpathlineto{\pgfqpoint{3.715612in}{1.321421in}}%
\pgfpathlineto{\pgfqpoint{3.715612in}{1.239692in}}%
\pgfusepath{}%
\end{pgfscope}%
\begin{pgfscope}%
\pgfpathrectangle{\pgfqpoint{0.549740in}{0.463273in}}{\pgfqpoint{9.320225in}{4.495057in}}%
\pgfusepath{clip}%
\pgfsetbuttcap%
\pgfsetroundjoin%
\pgfsetlinewidth{0.000000pt}%
\definecolor{currentstroke}{rgb}{0.000000,0.000000,0.000000}%
\pgfsetstrokecolor{currentstroke}%
\pgfsetdash{}{0pt}%
\pgfpathmoveto{\pgfqpoint{3.901839in}{1.239692in}}%
\pgfpathlineto{\pgfqpoint{4.088065in}{1.239692in}}%
\pgfpathlineto{\pgfqpoint{4.088065in}{1.321421in}}%
\pgfpathlineto{\pgfqpoint{3.901839in}{1.321421in}}%
\pgfpathlineto{\pgfqpoint{3.901839in}{1.239692in}}%
\pgfusepath{}%
\end{pgfscope}%
\begin{pgfscope}%
\pgfpathrectangle{\pgfqpoint{0.549740in}{0.463273in}}{\pgfqpoint{9.320225in}{4.495057in}}%
\pgfusepath{clip}%
\pgfsetbuttcap%
\pgfsetroundjoin%
\pgfsetlinewidth{0.000000pt}%
\definecolor{currentstroke}{rgb}{0.000000,0.000000,0.000000}%
\pgfsetstrokecolor{currentstroke}%
\pgfsetdash{}{0pt}%
\pgfpathmoveto{\pgfqpoint{4.088065in}{1.239692in}}%
\pgfpathlineto{\pgfqpoint{4.274292in}{1.239692in}}%
\pgfpathlineto{\pgfqpoint{4.274292in}{1.321421in}}%
\pgfpathlineto{\pgfqpoint{4.088065in}{1.321421in}}%
\pgfpathlineto{\pgfqpoint{4.088065in}{1.239692in}}%
\pgfusepath{}%
\end{pgfscope}%
\begin{pgfscope}%
\pgfpathrectangle{\pgfqpoint{0.549740in}{0.463273in}}{\pgfqpoint{9.320225in}{4.495057in}}%
\pgfusepath{clip}%
\pgfsetbuttcap%
\pgfsetroundjoin%
\pgfsetlinewidth{0.000000pt}%
\definecolor{currentstroke}{rgb}{0.000000,0.000000,0.000000}%
\pgfsetstrokecolor{currentstroke}%
\pgfsetdash{}{0pt}%
\pgfpathmoveto{\pgfqpoint{4.274292in}{1.239692in}}%
\pgfpathlineto{\pgfqpoint{4.460519in}{1.239692in}}%
\pgfpathlineto{\pgfqpoint{4.460519in}{1.321421in}}%
\pgfpathlineto{\pgfqpoint{4.274292in}{1.321421in}}%
\pgfpathlineto{\pgfqpoint{4.274292in}{1.239692in}}%
\pgfusepath{}%
\end{pgfscope}%
\begin{pgfscope}%
\pgfpathrectangle{\pgfqpoint{0.549740in}{0.463273in}}{\pgfqpoint{9.320225in}{4.495057in}}%
\pgfusepath{clip}%
\pgfsetbuttcap%
\pgfsetroundjoin%
\pgfsetlinewidth{0.000000pt}%
\definecolor{currentstroke}{rgb}{0.000000,0.000000,0.000000}%
\pgfsetstrokecolor{currentstroke}%
\pgfsetdash{}{0pt}%
\pgfpathmoveto{\pgfqpoint{4.460519in}{1.239692in}}%
\pgfpathlineto{\pgfqpoint{4.646745in}{1.239692in}}%
\pgfpathlineto{\pgfqpoint{4.646745in}{1.321421in}}%
\pgfpathlineto{\pgfqpoint{4.460519in}{1.321421in}}%
\pgfpathlineto{\pgfqpoint{4.460519in}{1.239692in}}%
\pgfusepath{}%
\end{pgfscope}%
\begin{pgfscope}%
\pgfpathrectangle{\pgfqpoint{0.549740in}{0.463273in}}{\pgfqpoint{9.320225in}{4.495057in}}%
\pgfusepath{clip}%
\pgfsetbuttcap%
\pgfsetroundjoin%
\pgfsetlinewidth{0.000000pt}%
\definecolor{currentstroke}{rgb}{0.000000,0.000000,0.000000}%
\pgfsetstrokecolor{currentstroke}%
\pgfsetdash{}{0pt}%
\pgfpathmoveto{\pgfqpoint{4.646745in}{1.239692in}}%
\pgfpathlineto{\pgfqpoint{4.832972in}{1.239692in}}%
\pgfpathlineto{\pgfqpoint{4.832972in}{1.321421in}}%
\pgfpathlineto{\pgfqpoint{4.646745in}{1.321421in}}%
\pgfpathlineto{\pgfqpoint{4.646745in}{1.239692in}}%
\pgfusepath{}%
\end{pgfscope}%
\begin{pgfscope}%
\pgfpathrectangle{\pgfqpoint{0.549740in}{0.463273in}}{\pgfqpoint{9.320225in}{4.495057in}}%
\pgfusepath{clip}%
\pgfsetbuttcap%
\pgfsetroundjoin%
\pgfsetlinewidth{0.000000pt}%
\definecolor{currentstroke}{rgb}{0.000000,0.000000,0.000000}%
\pgfsetstrokecolor{currentstroke}%
\pgfsetdash{}{0pt}%
\pgfpathmoveto{\pgfqpoint{4.832972in}{1.239692in}}%
\pgfpathlineto{\pgfqpoint{5.019198in}{1.239692in}}%
\pgfpathlineto{\pgfqpoint{5.019198in}{1.321421in}}%
\pgfpathlineto{\pgfqpoint{4.832972in}{1.321421in}}%
\pgfpathlineto{\pgfqpoint{4.832972in}{1.239692in}}%
\pgfusepath{}%
\end{pgfscope}%
\begin{pgfscope}%
\pgfpathrectangle{\pgfqpoint{0.549740in}{0.463273in}}{\pgfqpoint{9.320225in}{4.495057in}}%
\pgfusepath{clip}%
\pgfsetbuttcap%
\pgfsetroundjoin%
\pgfsetlinewidth{0.000000pt}%
\definecolor{currentstroke}{rgb}{0.000000,0.000000,0.000000}%
\pgfsetstrokecolor{currentstroke}%
\pgfsetdash{}{0pt}%
\pgfpathmoveto{\pgfqpoint{5.019198in}{1.239692in}}%
\pgfpathlineto{\pgfqpoint{5.205425in}{1.239692in}}%
\pgfpathlineto{\pgfqpoint{5.205425in}{1.321421in}}%
\pgfpathlineto{\pgfqpoint{5.019198in}{1.321421in}}%
\pgfpathlineto{\pgfqpoint{5.019198in}{1.239692in}}%
\pgfusepath{}%
\end{pgfscope}%
\begin{pgfscope}%
\pgfpathrectangle{\pgfqpoint{0.549740in}{0.463273in}}{\pgfqpoint{9.320225in}{4.495057in}}%
\pgfusepath{clip}%
\pgfsetbuttcap%
\pgfsetroundjoin%
\pgfsetlinewidth{0.000000pt}%
\definecolor{currentstroke}{rgb}{0.000000,0.000000,0.000000}%
\pgfsetstrokecolor{currentstroke}%
\pgfsetdash{}{0pt}%
\pgfpathmoveto{\pgfqpoint{5.205425in}{1.239692in}}%
\pgfpathlineto{\pgfqpoint{5.391651in}{1.239692in}}%
\pgfpathlineto{\pgfqpoint{5.391651in}{1.321421in}}%
\pgfpathlineto{\pgfqpoint{5.205425in}{1.321421in}}%
\pgfpathlineto{\pgfqpoint{5.205425in}{1.239692in}}%
\pgfusepath{}%
\end{pgfscope}%
\begin{pgfscope}%
\pgfpathrectangle{\pgfqpoint{0.549740in}{0.463273in}}{\pgfqpoint{9.320225in}{4.495057in}}%
\pgfusepath{clip}%
\pgfsetbuttcap%
\pgfsetroundjoin%
\pgfsetlinewidth{0.000000pt}%
\definecolor{currentstroke}{rgb}{0.000000,0.000000,0.000000}%
\pgfsetstrokecolor{currentstroke}%
\pgfsetdash{}{0pt}%
\pgfpathmoveto{\pgfqpoint{5.391651in}{1.239692in}}%
\pgfpathlineto{\pgfqpoint{5.577878in}{1.239692in}}%
\pgfpathlineto{\pgfqpoint{5.577878in}{1.321421in}}%
\pgfpathlineto{\pgfqpoint{5.391651in}{1.321421in}}%
\pgfpathlineto{\pgfqpoint{5.391651in}{1.239692in}}%
\pgfusepath{}%
\end{pgfscope}%
\begin{pgfscope}%
\pgfpathrectangle{\pgfqpoint{0.549740in}{0.463273in}}{\pgfqpoint{9.320225in}{4.495057in}}%
\pgfusepath{clip}%
\pgfsetbuttcap%
\pgfsetroundjoin%
\pgfsetlinewidth{0.000000pt}%
\definecolor{currentstroke}{rgb}{0.000000,0.000000,0.000000}%
\pgfsetstrokecolor{currentstroke}%
\pgfsetdash{}{0pt}%
\pgfpathmoveto{\pgfqpoint{5.577878in}{1.239692in}}%
\pgfpathlineto{\pgfqpoint{5.764104in}{1.239692in}}%
\pgfpathlineto{\pgfqpoint{5.764104in}{1.321421in}}%
\pgfpathlineto{\pgfqpoint{5.577878in}{1.321421in}}%
\pgfpathlineto{\pgfqpoint{5.577878in}{1.239692in}}%
\pgfusepath{}%
\end{pgfscope}%
\begin{pgfscope}%
\pgfpathrectangle{\pgfqpoint{0.549740in}{0.463273in}}{\pgfqpoint{9.320225in}{4.495057in}}%
\pgfusepath{clip}%
\pgfsetbuttcap%
\pgfsetroundjoin%
\definecolor{currentfill}{rgb}{0.472869,0.711325,0.955316}%
\pgfsetfillcolor{currentfill}%
\pgfsetlinewidth{0.000000pt}%
\definecolor{currentstroke}{rgb}{0.000000,0.000000,0.000000}%
\pgfsetstrokecolor{currentstroke}%
\pgfsetdash{}{0pt}%
\pgfpathmoveto{\pgfqpoint{5.764104in}{1.239692in}}%
\pgfpathlineto{\pgfqpoint{5.950331in}{1.239692in}}%
\pgfpathlineto{\pgfqpoint{5.950331in}{1.321421in}}%
\pgfpathlineto{\pgfqpoint{5.764104in}{1.321421in}}%
\pgfpathlineto{\pgfqpoint{5.764104in}{1.239692in}}%
\pgfusepath{fill}%
\end{pgfscope}%
\begin{pgfscope}%
\pgfpathrectangle{\pgfqpoint{0.549740in}{0.463273in}}{\pgfqpoint{9.320225in}{4.495057in}}%
\pgfusepath{clip}%
\pgfsetbuttcap%
\pgfsetroundjoin%
\pgfsetlinewidth{0.000000pt}%
\definecolor{currentstroke}{rgb}{0.000000,0.000000,0.000000}%
\pgfsetstrokecolor{currentstroke}%
\pgfsetdash{}{0pt}%
\pgfpathmoveto{\pgfqpoint{5.950331in}{1.239692in}}%
\pgfpathlineto{\pgfqpoint{6.136557in}{1.239692in}}%
\pgfpathlineto{\pgfqpoint{6.136557in}{1.321421in}}%
\pgfpathlineto{\pgfqpoint{5.950331in}{1.321421in}}%
\pgfpathlineto{\pgfqpoint{5.950331in}{1.239692in}}%
\pgfusepath{}%
\end{pgfscope}%
\begin{pgfscope}%
\pgfpathrectangle{\pgfqpoint{0.549740in}{0.463273in}}{\pgfqpoint{9.320225in}{4.495057in}}%
\pgfusepath{clip}%
\pgfsetbuttcap%
\pgfsetroundjoin%
\pgfsetlinewidth{0.000000pt}%
\definecolor{currentstroke}{rgb}{0.000000,0.000000,0.000000}%
\pgfsetstrokecolor{currentstroke}%
\pgfsetdash{}{0pt}%
\pgfpathmoveto{\pgfqpoint{6.136557in}{1.239692in}}%
\pgfpathlineto{\pgfqpoint{6.322784in}{1.239692in}}%
\pgfpathlineto{\pgfqpoint{6.322784in}{1.321421in}}%
\pgfpathlineto{\pgfqpoint{6.136557in}{1.321421in}}%
\pgfpathlineto{\pgfqpoint{6.136557in}{1.239692in}}%
\pgfusepath{}%
\end{pgfscope}%
\begin{pgfscope}%
\pgfpathrectangle{\pgfqpoint{0.549740in}{0.463273in}}{\pgfqpoint{9.320225in}{4.495057in}}%
\pgfusepath{clip}%
\pgfsetbuttcap%
\pgfsetroundjoin%
\pgfsetlinewidth{0.000000pt}%
\definecolor{currentstroke}{rgb}{0.000000,0.000000,0.000000}%
\pgfsetstrokecolor{currentstroke}%
\pgfsetdash{}{0pt}%
\pgfpathmoveto{\pgfqpoint{6.322784in}{1.239692in}}%
\pgfpathlineto{\pgfqpoint{6.509011in}{1.239692in}}%
\pgfpathlineto{\pgfqpoint{6.509011in}{1.321421in}}%
\pgfpathlineto{\pgfqpoint{6.322784in}{1.321421in}}%
\pgfpathlineto{\pgfqpoint{6.322784in}{1.239692in}}%
\pgfusepath{}%
\end{pgfscope}%
\begin{pgfscope}%
\pgfpathrectangle{\pgfqpoint{0.549740in}{0.463273in}}{\pgfqpoint{9.320225in}{4.495057in}}%
\pgfusepath{clip}%
\pgfsetbuttcap%
\pgfsetroundjoin%
\pgfsetlinewidth{0.000000pt}%
\definecolor{currentstroke}{rgb}{0.000000,0.000000,0.000000}%
\pgfsetstrokecolor{currentstroke}%
\pgfsetdash{}{0pt}%
\pgfpathmoveto{\pgfqpoint{6.509011in}{1.239692in}}%
\pgfpathlineto{\pgfqpoint{6.695237in}{1.239692in}}%
\pgfpathlineto{\pgfqpoint{6.695237in}{1.321421in}}%
\pgfpathlineto{\pgfqpoint{6.509011in}{1.321421in}}%
\pgfpathlineto{\pgfqpoint{6.509011in}{1.239692in}}%
\pgfusepath{}%
\end{pgfscope}%
\begin{pgfscope}%
\pgfpathrectangle{\pgfqpoint{0.549740in}{0.463273in}}{\pgfqpoint{9.320225in}{4.495057in}}%
\pgfusepath{clip}%
\pgfsetbuttcap%
\pgfsetroundjoin%
\definecolor{currentfill}{rgb}{0.385185,0.686583,0.962589}%
\pgfsetfillcolor{currentfill}%
\pgfsetlinewidth{0.000000pt}%
\definecolor{currentstroke}{rgb}{0.000000,0.000000,0.000000}%
\pgfsetstrokecolor{currentstroke}%
\pgfsetdash{}{0pt}%
\pgfpathmoveto{\pgfqpoint{6.695237in}{1.239692in}}%
\pgfpathlineto{\pgfqpoint{6.881464in}{1.239692in}}%
\pgfpathlineto{\pgfqpoint{6.881464in}{1.321421in}}%
\pgfpathlineto{\pgfqpoint{6.695237in}{1.321421in}}%
\pgfpathlineto{\pgfqpoint{6.695237in}{1.239692in}}%
\pgfusepath{fill}%
\end{pgfscope}%
\begin{pgfscope}%
\pgfpathrectangle{\pgfqpoint{0.549740in}{0.463273in}}{\pgfqpoint{9.320225in}{4.495057in}}%
\pgfusepath{clip}%
\pgfsetbuttcap%
\pgfsetroundjoin%
\pgfsetlinewidth{0.000000pt}%
\definecolor{currentstroke}{rgb}{0.000000,0.000000,0.000000}%
\pgfsetstrokecolor{currentstroke}%
\pgfsetdash{}{0pt}%
\pgfpathmoveto{\pgfqpoint{6.881464in}{1.239692in}}%
\pgfpathlineto{\pgfqpoint{7.067690in}{1.239692in}}%
\pgfpathlineto{\pgfqpoint{7.067690in}{1.321421in}}%
\pgfpathlineto{\pgfqpoint{6.881464in}{1.321421in}}%
\pgfpathlineto{\pgfqpoint{6.881464in}{1.239692in}}%
\pgfusepath{}%
\end{pgfscope}%
\begin{pgfscope}%
\pgfpathrectangle{\pgfqpoint{0.549740in}{0.463273in}}{\pgfqpoint{9.320225in}{4.495057in}}%
\pgfusepath{clip}%
\pgfsetbuttcap%
\pgfsetroundjoin%
\pgfsetlinewidth{0.000000pt}%
\definecolor{currentstroke}{rgb}{0.000000,0.000000,0.000000}%
\pgfsetstrokecolor{currentstroke}%
\pgfsetdash{}{0pt}%
\pgfpathmoveto{\pgfqpoint{7.067690in}{1.239692in}}%
\pgfpathlineto{\pgfqpoint{7.253917in}{1.239692in}}%
\pgfpathlineto{\pgfqpoint{7.253917in}{1.321421in}}%
\pgfpathlineto{\pgfqpoint{7.067690in}{1.321421in}}%
\pgfpathlineto{\pgfqpoint{7.067690in}{1.239692in}}%
\pgfusepath{}%
\end{pgfscope}%
\begin{pgfscope}%
\pgfpathrectangle{\pgfqpoint{0.549740in}{0.463273in}}{\pgfqpoint{9.320225in}{4.495057in}}%
\pgfusepath{clip}%
\pgfsetbuttcap%
\pgfsetroundjoin%
\pgfsetlinewidth{0.000000pt}%
\definecolor{currentstroke}{rgb}{0.000000,0.000000,0.000000}%
\pgfsetstrokecolor{currentstroke}%
\pgfsetdash{}{0pt}%
\pgfpathmoveto{\pgfqpoint{7.253917in}{1.239692in}}%
\pgfpathlineto{\pgfqpoint{7.440143in}{1.239692in}}%
\pgfpathlineto{\pgfqpoint{7.440143in}{1.321421in}}%
\pgfpathlineto{\pgfqpoint{7.253917in}{1.321421in}}%
\pgfpathlineto{\pgfqpoint{7.253917in}{1.239692in}}%
\pgfusepath{}%
\end{pgfscope}%
\begin{pgfscope}%
\pgfpathrectangle{\pgfqpoint{0.549740in}{0.463273in}}{\pgfqpoint{9.320225in}{4.495057in}}%
\pgfusepath{clip}%
\pgfsetbuttcap%
\pgfsetroundjoin%
\pgfsetlinewidth{0.000000pt}%
\definecolor{currentstroke}{rgb}{0.000000,0.000000,0.000000}%
\pgfsetstrokecolor{currentstroke}%
\pgfsetdash{}{0pt}%
\pgfpathmoveto{\pgfqpoint{7.440143in}{1.239692in}}%
\pgfpathlineto{\pgfqpoint{7.626370in}{1.239692in}}%
\pgfpathlineto{\pgfqpoint{7.626370in}{1.321421in}}%
\pgfpathlineto{\pgfqpoint{7.440143in}{1.321421in}}%
\pgfpathlineto{\pgfqpoint{7.440143in}{1.239692in}}%
\pgfusepath{}%
\end{pgfscope}%
\begin{pgfscope}%
\pgfpathrectangle{\pgfqpoint{0.549740in}{0.463273in}}{\pgfqpoint{9.320225in}{4.495057in}}%
\pgfusepath{clip}%
\pgfsetbuttcap%
\pgfsetroundjoin%
\pgfsetlinewidth{0.000000pt}%
\definecolor{currentstroke}{rgb}{0.000000,0.000000,0.000000}%
\pgfsetstrokecolor{currentstroke}%
\pgfsetdash{}{0pt}%
\pgfpathmoveto{\pgfqpoint{7.626370in}{1.239692in}}%
\pgfpathlineto{\pgfqpoint{7.812596in}{1.239692in}}%
\pgfpathlineto{\pgfqpoint{7.812596in}{1.321421in}}%
\pgfpathlineto{\pgfqpoint{7.626370in}{1.321421in}}%
\pgfpathlineto{\pgfqpoint{7.626370in}{1.239692in}}%
\pgfusepath{}%
\end{pgfscope}%
\begin{pgfscope}%
\pgfpathrectangle{\pgfqpoint{0.549740in}{0.463273in}}{\pgfqpoint{9.320225in}{4.495057in}}%
\pgfusepath{clip}%
\pgfsetbuttcap%
\pgfsetroundjoin%
\definecolor{currentfill}{rgb}{0.614330,0.761948,0.940009}%
\pgfsetfillcolor{currentfill}%
\pgfsetlinewidth{0.000000pt}%
\definecolor{currentstroke}{rgb}{0.000000,0.000000,0.000000}%
\pgfsetstrokecolor{currentstroke}%
\pgfsetdash{}{0pt}%
\pgfpathmoveto{\pgfqpoint{7.812596in}{1.239692in}}%
\pgfpathlineto{\pgfqpoint{7.998823in}{1.239692in}}%
\pgfpathlineto{\pgfqpoint{7.998823in}{1.321421in}}%
\pgfpathlineto{\pgfqpoint{7.812596in}{1.321421in}}%
\pgfpathlineto{\pgfqpoint{7.812596in}{1.239692in}}%
\pgfusepath{fill}%
\end{pgfscope}%
\begin{pgfscope}%
\pgfpathrectangle{\pgfqpoint{0.549740in}{0.463273in}}{\pgfqpoint{9.320225in}{4.495057in}}%
\pgfusepath{clip}%
\pgfsetbuttcap%
\pgfsetroundjoin%
\pgfsetlinewidth{0.000000pt}%
\definecolor{currentstroke}{rgb}{0.000000,0.000000,0.000000}%
\pgfsetstrokecolor{currentstroke}%
\pgfsetdash{}{0pt}%
\pgfpathmoveto{\pgfqpoint{7.998823in}{1.239692in}}%
\pgfpathlineto{\pgfqpoint{8.185049in}{1.239692in}}%
\pgfpathlineto{\pgfqpoint{8.185049in}{1.321421in}}%
\pgfpathlineto{\pgfqpoint{7.998823in}{1.321421in}}%
\pgfpathlineto{\pgfqpoint{7.998823in}{1.239692in}}%
\pgfusepath{}%
\end{pgfscope}%
\begin{pgfscope}%
\pgfpathrectangle{\pgfqpoint{0.549740in}{0.463273in}}{\pgfqpoint{9.320225in}{4.495057in}}%
\pgfusepath{clip}%
\pgfsetbuttcap%
\pgfsetroundjoin%
\pgfsetlinewidth{0.000000pt}%
\definecolor{currentstroke}{rgb}{0.000000,0.000000,0.000000}%
\pgfsetstrokecolor{currentstroke}%
\pgfsetdash{}{0pt}%
\pgfpathmoveto{\pgfqpoint{8.185049in}{1.239692in}}%
\pgfpathlineto{\pgfqpoint{8.371276in}{1.239692in}}%
\pgfpathlineto{\pgfqpoint{8.371276in}{1.321421in}}%
\pgfpathlineto{\pgfqpoint{8.185049in}{1.321421in}}%
\pgfpathlineto{\pgfqpoint{8.185049in}{1.239692in}}%
\pgfusepath{}%
\end{pgfscope}%
\begin{pgfscope}%
\pgfpathrectangle{\pgfqpoint{0.549740in}{0.463273in}}{\pgfqpoint{9.320225in}{4.495057in}}%
\pgfusepath{clip}%
\pgfsetbuttcap%
\pgfsetroundjoin%
\pgfsetlinewidth{0.000000pt}%
\definecolor{currentstroke}{rgb}{0.000000,0.000000,0.000000}%
\pgfsetstrokecolor{currentstroke}%
\pgfsetdash{}{0pt}%
\pgfpathmoveto{\pgfqpoint{8.371276in}{1.239692in}}%
\pgfpathlineto{\pgfqpoint{8.557503in}{1.239692in}}%
\pgfpathlineto{\pgfqpoint{8.557503in}{1.321421in}}%
\pgfpathlineto{\pgfqpoint{8.371276in}{1.321421in}}%
\pgfpathlineto{\pgfqpoint{8.371276in}{1.239692in}}%
\pgfusepath{}%
\end{pgfscope}%
\begin{pgfscope}%
\pgfpathrectangle{\pgfqpoint{0.549740in}{0.463273in}}{\pgfqpoint{9.320225in}{4.495057in}}%
\pgfusepath{clip}%
\pgfsetbuttcap%
\pgfsetroundjoin%
\pgfsetlinewidth{0.000000pt}%
\definecolor{currentstroke}{rgb}{0.000000,0.000000,0.000000}%
\pgfsetstrokecolor{currentstroke}%
\pgfsetdash{}{0pt}%
\pgfpathmoveto{\pgfqpoint{8.557503in}{1.239692in}}%
\pgfpathlineto{\pgfqpoint{8.743729in}{1.239692in}}%
\pgfpathlineto{\pgfqpoint{8.743729in}{1.321421in}}%
\pgfpathlineto{\pgfqpoint{8.557503in}{1.321421in}}%
\pgfpathlineto{\pgfqpoint{8.557503in}{1.239692in}}%
\pgfusepath{}%
\end{pgfscope}%
\begin{pgfscope}%
\pgfpathrectangle{\pgfqpoint{0.549740in}{0.463273in}}{\pgfqpoint{9.320225in}{4.495057in}}%
\pgfusepath{clip}%
\pgfsetbuttcap%
\pgfsetroundjoin%
\definecolor{currentfill}{rgb}{0.614330,0.761948,0.940009}%
\pgfsetfillcolor{currentfill}%
\pgfsetlinewidth{0.000000pt}%
\definecolor{currentstroke}{rgb}{0.000000,0.000000,0.000000}%
\pgfsetstrokecolor{currentstroke}%
\pgfsetdash{}{0pt}%
\pgfpathmoveto{\pgfqpoint{8.743729in}{1.239692in}}%
\pgfpathlineto{\pgfqpoint{8.929956in}{1.239692in}}%
\pgfpathlineto{\pgfqpoint{8.929956in}{1.321421in}}%
\pgfpathlineto{\pgfqpoint{8.743729in}{1.321421in}}%
\pgfpathlineto{\pgfqpoint{8.743729in}{1.239692in}}%
\pgfusepath{fill}%
\end{pgfscope}%
\begin{pgfscope}%
\pgfpathrectangle{\pgfqpoint{0.549740in}{0.463273in}}{\pgfqpoint{9.320225in}{4.495057in}}%
\pgfusepath{clip}%
\pgfsetbuttcap%
\pgfsetroundjoin%
\pgfsetlinewidth{0.000000pt}%
\definecolor{currentstroke}{rgb}{0.000000,0.000000,0.000000}%
\pgfsetstrokecolor{currentstroke}%
\pgfsetdash{}{0pt}%
\pgfpathmoveto{\pgfqpoint{8.929956in}{1.239692in}}%
\pgfpathlineto{\pgfqpoint{9.116182in}{1.239692in}}%
\pgfpathlineto{\pgfqpoint{9.116182in}{1.321421in}}%
\pgfpathlineto{\pgfqpoint{8.929956in}{1.321421in}}%
\pgfpathlineto{\pgfqpoint{8.929956in}{1.239692in}}%
\pgfusepath{}%
\end{pgfscope}%
\begin{pgfscope}%
\pgfpathrectangle{\pgfqpoint{0.549740in}{0.463273in}}{\pgfqpoint{9.320225in}{4.495057in}}%
\pgfusepath{clip}%
\pgfsetbuttcap%
\pgfsetroundjoin%
\pgfsetlinewidth{0.000000pt}%
\definecolor{currentstroke}{rgb}{0.000000,0.000000,0.000000}%
\pgfsetstrokecolor{currentstroke}%
\pgfsetdash{}{0pt}%
\pgfpathmoveto{\pgfqpoint{9.116182in}{1.239692in}}%
\pgfpathlineto{\pgfqpoint{9.302409in}{1.239692in}}%
\pgfpathlineto{\pgfqpoint{9.302409in}{1.321421in}}%
\pgfpathlineto{\pgfqpoint{9.116182in}{1.321421in}}%
\pgfpathlineto{\pgfqpoint{9.116182in}{1.239692in}}%
\pgfusepath{}%
\end{pgfscope}%
\begin{pgfscope}%
\pgfpathrectangle{\pgfqpoint{0.549740in}{0.463273in}}{\pgfqpoint{9.320225in}{4.495057in}}%
\pgfusepath{clip}%
\pgfsetbuttcap%
\pgfsetroundjoin%
\pgfsetlinewidth{0.000000pt}%
\definecolor{currentstroke}{rgb}{0.000000,0.000000,0.000000}%
\pgfsetstrokecolor{currentstroke}%
\pgfsetdash{}{0pt}%
\pgfpathmoveto{\pgfqpoint{9.302409in}{1.239692in}}%
\pgfpathlineto{\pgfqpoint{9.488635in}{1.239692in}}%
\pgfpathlineto{\pgfqpoint{9.488635in}{1.321421in}}%
\pgfpathlineto{\pgfqpoint{9.302409in}{1.321421in}}%
\pgfpathlineto{\pgfqpoint{9.302409in}{1.239692in}}%
\pgfusepath{}%
\end{pgfscope}%
\begin{pgfscope}%
\pgfpathrectangle{\pgfqpoint{0.549740in}{0.463273in}}{\pgfqpoint{9.320225in}{4.495057in}}%
\pgfusepath{clip}%
\pgfsetbuttcap%
\pgfsetroundjoin%
\pgfsetlinewidth{0.000000pt}%
\definecolor{currentstroke}{rgb}{0.000000,0.000000,0.000000}%
\pgfsetstrokecolor{currentstroke}%
\pgfsetdash{}{0pt}%
\pgfpathmoveto{\pgfqpoint{9.488635in}{1.239692in}}%
\pgfpathlineto{\pgfqpoint{9.674862in}{1.239692in}}%
\pgfpathlineto{\pgfqpoint{9.674862in}{1.321421in}}%
\pgfpathlineto{\pgfqpoint{9.488635in}{1.321421in}}%
\pgfpathlineto{\pgfqpoint{9.488635in}{1.239692in}}%
\pgfusepath{}%
\end{pgfscope}%
\begin{pgfscope}%
\pgfpathrectangle{\pgfqpoint{0.549740in}{0.463273in}}{\pgfqpoint{9.320225in}{4.495057in}}%
\pgfusepath{clip}%
\pgfsetbuttcap%
\pgfsetroundjoin%
\pgfsetlinewidth{0.000000pt}%
\definecolor{currentstroke}{rgb}{0.000000,0.000000,0.000000}%
\pgfsetstrokecolor{currentstroke}%
\pgfsetdash{}{0pt}%
\pgfpathmoveto{\pgfqpoint{9.674862in}{1.239692in}}%
\pgfpathlineto{\pgfqpoint{9.861088in}{1.239692in}}%
\pgfpathlineto{\pgfqpoint{9.861088in}{1.321421in}}%
\pgfpathlineto{\pgfqpoint{9.674862in}{1.321421in}}%
\pgfpathlineto{\pgfqpoint{9.674862in}{1.239692in}}%
\pgfusepath{}%
\end{pgfscope}%
\begin{pgfscope}%
\pgfpathrectangle{\pgfqpoint{0.549740in}{0.463273in}}{\pgfqpoint{9.320225in}{4.495057in}}%
\pgfusepath{clip}%
\pgfsetbuttcap%
\pgfsetroundjoin%
\pgfsetlinewidth{0.000000pt}%
\definecolor{currentstroke}{rgb}{0.000000,0.000000,0.000000}%
\pgfsetstrokecolor{currentstroke}%
\pgfsetdash{}{0pt}%
\pgfpathmoveto{\pgfqpoint{0.549761in}{1.321421in}}%
\pgfpathlineto{\pgfqpoint{0.735988in}{1.321421in}}%
\pgfpathlineto{\pgfqpoint{0.735988in}{1.403149in}}%
\pgfpathlineto{\pgfqpoint{0.549761in}{1.403149in}}%
\pgfpathlineto{\pgfqpoint{0.549761in}{1.321421in}}%
\pgfusepath{}%
\end{pgfscope}%
\begin{pgfscope}%
\pgfpathrectangle{\pgfqpoint{0.549740in}{0.463273in}}{\pgfqpoint{9.320225in}{4.495057in}}%
\pgfusepath{clip}%
\pgfsetbuttcap%
\pgfsetroundjoin%
\pgfsetlinewidth{0.000000pt}%
\definecolor{currentstroke}{rgb}{0.000000,0.000000,0.000000}%
\pgfsetstrokecolor{currentstroke}%
\pgfsetdash{}{0pt}%
\pgfpathmoveto{\pgfqpoint{0.735988in}{1.321421in}}%
\pgfpathlineto{\pgfqpoint{0.922214in}{1.321421in}}%
\pgfpathlineto{\pgfqpoint{0.922214in}{1.403149in}}%
\pgfpathlineto{\pgfqpoint{0.735988in}{1.403149in}}%
\pgfpathlineto{\pgfqpoint{0.735988in}{1.321421in}}%
\pgfusepath{}%
\end{pgfscope}%
\begin{pgfscope}%
\pgfpathrectangle{\pgfqpoint{0.549740in}{0.463273in}}{\pgfqpoint{9.320225in}{4.495057in}}%
\pgfusepath{clip}%
\pgfsetbuttcap%
\pgfsetroundjoin%
\pgfsetlinewidth{0.000000pt}%
\definecolor{currentstroke}{rgb}{0.000000,0.000000,0.000000}%
\pgfsetstrokecolor{currentstroke}%
\pgfsetdash{}{0pt}%
\pgfpathmoveto{\pgfqpoint{0.922214in}{1.321421in}}%
\pgfpathlineto{\pgfqpoint{1.108441in}{1.321421in}}%
\pgfpathlineto{\pgfqpoint{1.108441in}{1.403149in}}%
\pgfpathlineto{\pgfqpoint{0.922214in}{1.403149in}}%
\pgfpathlineto{\pgfqpoint{0.922214in}{1.321421in}}%
\pgfusepath{}%
\end{pgfscope}%
\begin{pgfscope}%
\pgfpathrectangle{\pgfqpoint{0.549740in}{0.463273in}}{\pgfqpoint{9.320225in}{4.495057in}}%
\pgfusepath{clip}%
\pgfsetbuttcap%
\pgfsetroundjoin%
\pgfsetlinewidth{0.000000pt}%
\definecolor{currentstroke}{rgb}{0.000000,0.000000,0.000000}%
\pgfsetstrokecolor{currentstroke}%
\pgfsetdash{}{0pt}%
\pgfpathmoveto{\pgfqpoint{1.108441in}{1.321421in}}%
\pgfpathlineto{\pgfqpoint{1.294667in}{1.321421in}}%
\pgfpathlineto{\pgfqpoint{1.294667in}{1.403149in}}%
\pgfpathlineto{\pgfqpoint{1.108441in}{1.403149in}}%
\pgfpathlineto{\pgfqpoint{1.108441in}{1.321421in}}%
\pgfusepath{}%
\end{pgfscope}%
\begin{pgfscope}%
\pgfpathrectangle{\pgfqpoint{0.549740in}{0.463273in}}{\pgfqpoint{9.320225in}{4.495057in}}%
\pgfusepath{clip}%
\pgfsetbuttcap%
\pgfsetroundjoin%
\pgfsetlinewidth{0.000000pt}%
\definecolor{currentstroke}{rgb}{0.000000,0.000000,0.000000}%
\pgfsetstrokecolor{currentstroke}%
\pgfsetdash{}{0pt}%
\pgfpathmoveto{\pgfqpoint{1.294667in}{1.321421in}}%
\pgfpathlineto{\pgfqpoint{1.480894in}{1.321421in}}%
\pgfpathlineto{\pgfqpoint{1.480894in}{1.403149in}}%
\pgfpathlineto{\pgfqpoint{1.294667in}{1.403149in}}%
\pgfpathlineto{\pgfqpoint{1.294667in}{1.321421in}}%
\pgfusepath{}%
\end{pgfscope}%
\begin{pgfscope}%
\pgfpathrectangle{\pgfqpoint{0.549740in}{0.463273in}}{\pgfqpoint{9.320225in}{4.495057in}}%
\pgfusepath{clip}%
\pgfsetbuttcap%
\pgfsetroundjoin%
\pgfsetlinewidth{0.000000pt}%
\definecolor{currentstroke}{rgb}{0.000000,0.000000,0.000000}%
\pgfsetstrokecolor{currentstroke}%
\pgfsetdash{}{0pt}%
\pgfpathmoveto{\pgfqpoint{1.480894in}{1.321421in}}%
\pgfpathlineto{\pgfqpoint{1.667120in}{1.321421in}}%
\pgfpathlineto{\pgfqpoint{1.667120in}{1.403149in}}%
\pgfpathlineto{\pgfqpoint{1.480894in}{1.403149in}}%
\pgfpathlineto{\pgfqpoint{1.480894in}{1.321421in}}%
\pgfusepath{}%
\end{pgfscope}%
\begin{pgfscope}%
\pgfpathrectangle{\pgfqpoint{0.549740in}{0.463273in}}{\pgfqpoint{9.320225in}{4.495057in}}%
\pgfusepath{clip}%
\pgfsetbuttcap%
\pgfsetroundjoin%
\pgfsetlinewidth{0.000000pt}%
\definecolor{currentstroke}{rgb}{0.000000,0.000000,0.000000}%
\pgfsetstrokecolor{currentstroke}%
\pgfsetdash{}{0pt}%
\pgfpathmoveto{\pgfqpoint{1.667120in}{1.321421in}}%
\pgfpathlineto{\pgfqpoint{1.853347in}{1.321421in}}%
\pgfpathlineto{\pgfqpoint{1.853347in}{1.403149in}}%
\pgfpathlineto{\pgfqpoint{1.667120in}{1.403149in}}%
\pgfpathlineto{\pgfqpoint{1.667120in}{1.321421in}}%
\pgfusepath{}%
\end{pgfscope}%
\begin{pgfscope}%
\pgfpathrectangle{\pgfqpoint{0.549740in}{0.463273in}}{\pgfqpoint{9.320225in}{4.495057in}}%
\pgfusepath{clip}%
\pgfsetbuttcap%
\pgfsetroundjoin%
\pgfsetlinewidth{0.000000pt}%
\definecolor{currentstroke}{rgb}{0.000000,0.000000,0.000000}%
\pgfsetstrokecolor{currentstroke}%
\pgfsetdash{}{0pt}%
\pgfpathmoveto{\pgfqpoint{1.853347in}{1.321421in}}%
\pgfpathlineto{\pgfqpoint{2.039573in}{1.321421in}}%
\pgfpathlineto{\pgfqpoint{2.039573in}{1.403149in}}%
\pgfpathlineto{\pgfqpoint{1.853347in}{1.403149in}}%
\pgfpathlineto{\pgfqpoint{1.853347in}{1.321421in}}%
\pgfusepath{}%
\end{pgfscope}%
\begin{pgfscope}%
\pgfpathrectangle{\pgfqpoint{0.549740in}{0.463273in}}{\pgfqpoint{9.320225in}{4.495057in}}%
\pgfusepath{clip}%
\pgfsetbuttcap%
\pgfsetroundjoin%
\pgfsetlinewidth{0.000000pt}%
\definecolor{currentstroke}{rgb}{0.000000,0.000000,0.000000}%
\pgfsetstrokecolor{currentstroke}%
\pgfsetdash{}{0pt}%
\pgfpathmoveto{\pgfqpoint{2.039573in}{1.321421in}}%
\pgfpathlineto{\pgfqpoint{2.225800in}{1.321421in}}%
\pgfpathlineto{\pgfqpoint{2.225800in}{1.403149in}}%
\pgfpathlineto{\pgfqpoint{2.039573in}{1.403149in}}%
\pgfpathlineto{\pgfqpoint{2.039573in}{1.321421in}}%
\pgfusepath{}%
\end{pgfscope}%
\begin{pgfscope}%
\pgfpathrectangle{\pgfqpoint{0.549740in}{0.463273in}}{\pgfqpoint{9.320225in}{4.495057in}}%
\pgfusepath{clip}%
\pgfsetbuttcap%
\pgfsetroundjoin%
\pgfsetlinewidth{0.000000pt}%
\definecolor{currentstroke}{rgb}{0.000000,0.000000,0.000000}%
\pgfsetstrokecolor{currentstroke}%
\pgfsetdash{}{0pt}%
\pgfpathmoveto{\pgfqpoint{2.225800in}{1.321421in}}%
\pgfpathlineto{\pgfqpoint{2.412027in}{1.321421in}}%
\pgfpathlineto{\pgfqpoint{2.412027in}{1.403149in}}%
\pgfpathlineto{\pgfqpoint{2.225800in}{1.403149in}}%
\pgfpathlineto{\pgfqpoint{2.225800in}{1.321421in}}%
\pgfusepath{}%
\end{pgfscope}%
\begin{pgfscope}%
\pgfpathrectangle{\pgfqpoint{0.549740in}{0.463273in}}{\pgfqpoint{9.320225in}{4.495057in}}%
\pgfusepath{clip}%
\pgfsetbuttcap%
\pgfsetroundjoin%
\pgfsetlinewidth{0.000000pt}%
\definecolor{currentstroke}{rgb}{0.000000,0.000000,0.000000}%
\pgfsetstrokecolor{currentstroke}%
\pgfsetdash{}{0pt}%
\pgfpathmoveto{\pgfqpoint{2.412027in}{1.321421in}}%
\pgfpathlineto{\pgfqpoint{2.598253in}{1.321421in}}%
\pgfpathlineto{\pgfqpoint{2.598253in}{1.403149in}}%
\pgfpathlineto{\pgfqpoint{2.412027in}{1.403149in}}%
\pgfpathlineto{\pgfqpoint{2.412027in}{1.321421in}}%
\pgfusepath{}%
\end{pgfscope}%
\begin{pgfscope}%
\pgfpathrectangle{\pgfqpoint{0.549740in}{0.463273in}}{\pgfqpoint{9.320225in}{4.495057in}}%
\pgfusepath{clip}%
\pgfsetbuttcap%
\pgfsetroundjoin%
\pgfsetlinewidth{0.000000pt}%
\definecolor{currentstroke}{rgb}{0.000000,0.000000,0.000000}%
\pgfsetstrokecolor{currentstroke}%
\pgfsetdash{}{0pt}%
\pgfpathmoveto{\pgfqpoint{2.598253in}{1.321421in}}%
\pgfpathlineto{\pgfqpoint{2.784480in}{1.321421in}}%
\pgfpathlineto{\pgfqpoint{2.784480in}{1.403149in}}%
\pgfpathlineto{\pgfqpoint{2.598253in}{1.403149in}}%
\pgfpathlineto{\pgfqpoint{2.598253in}{1.321421in}}%
\pgfusepath{}%
\end{pgfscope}%
\begin{pgfscope}%
\pgfpathrectangle{\pgfqpoint{0.549740in}{0.463273in}}{\pgfqpoint{9.320225in}{4.495057in}}%
\pgfusepath{clip}%
\pgfsetbuttcap%
\pgfsetroundjoin%
\pgfsetlinewidth{0.000000pt}%
\definecolor{currentstroke}{rgb}{0.000000,0.000000,0.000000}%
\pgfsetstrokecolor{currentstroke}%
\pgfsetdash{}{0pt}%
\pgfpathmoveto{\pgfqpoint{2.784480in}{1.321421in}}%
\pgfpathlineto{\pgfqpoint{2.970706in}{1.321421in}}%
\pgfpathlineto{\pgfqpoint{2.970706in}{1.403149in}}%
\pgfpathlineto{\pgfqpoint{2.784480in}{1.403149in}}%
\pgfpathlineto{\pgfqpoint{2.784480in}{1.321421in}}%
\pgfusepath{}%
\end{pgfscope}%
\begin{pgfscope}%
\pgfpathrectangle{\pgfqpoint{0.549740in}{0.463273in}}{\pgfqpoint{9.320225in}{4.495057in}}%
\pgfusepath{clip}%
\pgfsetbuttcap%
\pgfsetroundjoin%
\pgfsetlinewidth{0.000000pt}%
\definecolor{currentstroke}{rgb}{0.000000,0.000000,0.000000}%
\pgfsetstrokecolor{currentstroke}%
\pgfsetdash{}{0pt}%
\pgfpathmoveto{\pgfqpoint{2.970706in}{1.321421in}}%
\pgfpathlineto{\pgfqpoint{3.156933in}{1.321421in}}%
\pgfpathlineto{\pgfqpoint{3.156933in}{1.403149in}}%
\pgfpathlineto{\pgfqpoint{2.970706in}{1.403149in}}%
\pgfpathlineto{\pgfqpoint{2.970706in}{1.321421in}}%
\pgfusepath{}%
\end{pgfscope}%
\begin{pgfscope}%
\pgfpathrectangle{\pgfqpoint{0.549740in}{0.463273in}}{\pgfqpoint{9.320225in}{4.495057in}}%
\pgfusepath{clip}%
\pgfsetbuttcap%
\pgfsetroundjoin%
\pgfsetlinewidth{0.000000pt}%
\definecolor{currentstroke}{rgb}{0.000000,0.000000,0.000000}%
\pgfsetstrokecolor{currentstroke}%
\pgfsetdash{}{0pt}%
\pgfpathmoveto{\pgfqpoint{3.156933in}{1.321421in}}%
\pgfpathlineto{\pgfqpoint{3.343159in}{1.321421in}}%
\pgfpathlineto{\pgfqpoint{3.343159in}{1.403149in}}%
\pgfpathlineto{\pgfqpoint{3.156933in}{1.403149in}}%
\pgfpathlineto{\pgfqpoint{3.156933in}{1.321421in}}%
\pgfusepath{}%
\end{pgfscope}%
\begin{pgfscope}%
\pgfpathrectangle{\pgfqpoint{0.549740in}{0.463273in}}{\pgfqpoint{9.320225in}{4.495057in}}%
\pgfusepath{clip}%
\pgfsetbuttcap%
\pgfsetroundjoin%
\pgfsetlinewidth{0.000000pt}%
\definecolor{currentstroke}{rgb}{0.000000,0.000000,0.000000}%
\pgfsetstrokecolor{currentstroke}%
\pgfsetdash{}{0pt}%
\pgfpathmoveto{\pgfqpoint{3.343159in}{1.321421in}}%
\pgfpathlineto{\pgfqpoint{3.529386in}{1.321421in}}%
\pgfpathlineto{\pgfqpoint{3.529386in}{1.403149in}}%
\pgfpathlineto{\pgfqpoint{3.343159in}{1.403149in}}%
\pgfpathlineto{\pgfqpoint{3.343159in}{1.321421in}}%
\pgfusepath{}%
\end{pgfscope}%
\begin{pgfscope}%
\pgfpathrectangle{\pgfqpoint{0.549740in}{0.463273in}}{\pgfqpoint{9.320225in}{4.495057in}}%
\pgfusepath{clip}%
\pgfsetbuttcap%
\pgfsetroundjoin%
\pgfsetlinewidth{0.000000pt}%
\definecolor{currentstroke}{rgb}{0.000000,0.000000,0.000000}%
\pgfsetstrokecolor{currentstroke}%
\pgfsetdash{}{0pt}%
\pgfpathmoveto{\pgfqpoint{3.529386in}{1.321421in}}%
\pgfpathlineto{\pgfqpoint{3.715612in}{1.321421in}}%
\pgfpathlineto{\pgfqpoint{3.715612in}{1.403149in}}%
\pgfpathlineto{\pgfqpoint{3.529386in}{1.403149in}}%
\pgfpathlineto{\pgfqpoint{3.529386in}{1.321421in}}%
\pgfusepath{}%
\end{pgfscope}%
\begin{pgfscope}%
\pgfpathrectangle{\pgfqpoint{0.549740in}{0.463273in}}{\pgfqpoint{9.320225in}{4.495057in}}%
\pgfusepath{clip}%
\pgfsetbuttcap%
\pgfsetroundjoin%
\pgfsetlinewidth{0.000000pt}%
\definecolor{currentstroke}{rgb}{0.000000,0.000000,0.000000}%
\pgfsetstrokecolor{currentstroke}%
\pgfsetdash{}{0pt}%
\pgfpathmoveto{\pgfqpoint{3.715612in}{1.321421in}}%
\pgfpathlineto{\pgfqpoint{3.901839in}{1.321421in}}%
\pgfpathlineto{\pgfqpoint{3.901839in}{1.403149in}}%
\pgfpathlineto{\pgfqpoint{3.715612in}{1.403149in}}%
\pgfpathlineto{\pgfqpoint{3.715612in}{1.321421in}}%
\pgfusepath{}%
\end{pgfscope}%
\begin{pgfscope}%
\pgfpathrectangle{\pgfqpoint{0.549740in}{0.463273in}}{\pgfqpoint{9.320225in}{4.495057in}}%
\pgfusepath{clip}%
\pgfsetbuttcap%
\pgfsetroundjoin%
\pgfsetlinewidth{0.000000pt}%
\definecolor{currentstroke}{rgb}{0.000000,0.000000,0.000000}%
\pgfsetstrokecolor{currentstroke}%
\pgfsetdash{}{0pt}%
\pgfpathmoveto{\pgfqpoint{3.901839in}{1.321421in}}%
\pgfpathlineto{\pgfqpoint{4.088065in}{1.321421in}}%
\pgfpathlineto{\pgfqpoint{4.088065in}{1.403149in}}%
\pgfpathlineto{\pgfqpoint{3.901839in}{1.403149in}}%
\pgfpathlineto{\pgfqpoint{3.901839in}{1.321421in}}%
\pgfusepath{}%
\end{pgfscope}%
\begin{pgfscope}%
\pgfpathrectangle{\pgfqpoint{0.549740in}{0.463273in}}{\pgfqpoint{9.320225in}{4.495057in}}%
\pgfusepath{clip}%
\pgfsetbuttcap%
\pgfsetroundjoin%
\pgfsetlinewidth{0.000000pt}%
\definecolor{currentstroke}{rgb}{0.000000,0.000000,0.000000}%
\pgfsetstrokecolor{currentstroke}%
\pgfsetdash{}{0pt}%
\pgfpathmoveto{\pgfqpoint{4.088065in}{1.321421in}}%
\pgfpathlineto{\pgfqpoint{4.274292in}{1.321421in}}%
\pgfpathlineto{\pgfqpoint{4.274292in}{1.403149in}}%
\pgfpathlineto{\pgfqpoint{4.088065in}{1.403149in}}%
\pgfpathlineto{\pgfqpoint{4.088065in}{1.321421in}}%
\pgfusepath{}%
\end{pgfscope}%
\begin{pgfscope}%
\pgfpathrectangle{\pgfqpoint{0.549740in}{0.463273in}}{\pgfqpoint{9.320225in}{4.495057in}}%
\pgfusepath{clip}%
\pgfsetbuttcap%
\pgfsetroundjoin%
\pgfsetlinewidth{0.000000pt}%
\definecolor{currentstroke}{rgb}{0.000000,0.000000,0.000000}%
\pgfsetstrokecolor{currentstroke}%
\pgfsetdash{}{0pt}%
\pgfpathmoveto{\pgfqpoint{4.274292in}{1.321421in}}%
\pgfpathlineto{\pgfqpoint{4.460519in}{1.321421in}}%
\pgfpathlineto{\pgfqpoint{4.460519in}{1.403149in}}%
\pgfpathlineto{\pgfqpoint{4.274292in}{1.403149in}}%
\pgfpathlineto{\pgfqpoint{4.274292in}{1.321421in}}%
\pgfusepath{}%
\end{pgfscope}%
\begin{pgfscope}%
\pgfpathrectangle{\pgfqpoint{0.549740in}{0.463273in}}{\pgfqpoint{9.320225in}{4.495057in}}%
\pgfusepath{clip}%
\pgfsetbuttcap%
\pgfsetroundjoin%
\pgfsetlinewidth{0.000000pt}%
\definecolor{currentstroke}{rgb}{0.000000,0.000000,0.000000}%
\pgfsetstrokecolor{currentstroke}%
\pgfsetdash{}{0pt}%
\pgfpathmoveto{\pgfqpoint{4.460519in}{1.321421in}}%
\pgfpathlineto{\pgfqpoint{4.646745in}{1.321421in}}%
\pgfpathlineto{\pgfqpoint{4.646745in}{1.403149in}}%
\pgfpathlineto{\pgfqpoint{4.460519in}{1.403149in}}%
\pgfpathlineto{\pgfqpoint{4.460519in}{1.321421in}}%
\pgfusepath{}%
\end{pgfscope}%
\begin{pgfscope}%
\pgfpathrectangle{\pgfqpoint{0.549740in}{0.463273in}}{\pgfqpoint{9.320225in}{4.495057in}}%
\pgfusepath{clip}%
\pgfsetbuttcap%
\pgfsetroundjoin%
\pgfsetlinewidth{0.000000pt}%
\definecolor{currentstroke}{rgb}{0.000000,0.000000,0.000000}%
\pgfsetstrokecolor{currentstroke}%
\pgfsetdash{}{0pt}%
\pgfpathmoveto{\pgfqpoint{4.646745in}{1.321421in}}%
\pgfpathlineto{\pgfqpoint{4.832972in}{1.321421in}}%
\pgfpathlineto{\pgfqpoint{4.832972in}{1.403149in}}%
\pgfpathlineto{\pgfqpoint{4.646745in}{1.403149in}}%
\pgfpathlineto{\pgfqpoint{4.646745in}{1.321421in}}%
\pgfusepath{}%
\end{pgfscope}%
\begin{pgfscope}%
\pgfpathrectangle{\pgfqpoint{0.549740in}{0.463273in}}{\pgfqpoint{9.320225in}{4.495057in}}%
\pgfusepath{clip}%
\pgfsetbuttcap%
\pgfsetroundjoin%
\pgfsetlinewidth{0.000000pt}%
\definecolor{currentstroke}{rgb}{0.000000,0.000000,0.000000}%
\pgfsetstrokecolor{currentstroke}%
\pgfsetdash{}{0pt}%
\pgfpathmoveto{\pgfqpoint{4.832972in}{1.321421in}}%
\pgfpathlineto{\pgfqpoint{5.019198in}{1.321421in}}%
\pgfpathlineto{\pgfqpoint{5.019198in}{1.403149in}}%
\pgfpathlineto{\pgfqpoint{4.832972in}{1.403149in}}%
\pgfpathlineto{\pgfqpoint{4.832972in}{1.321421in}}%
\pgfusepath{}%
\end{pgfscope}%
\begin{pgfscope}%
\pgfpathrectangle{\pgfqpoint{0.549740in}{0.463273in}}{\pgfqpoint{9.320225in}{4.495057in}}%
\pgfusepath{clip}%
\pgfsetbuttcap%
\pgfsetroundjoin%
\pgfsetlinewidth{0.000000pt}%
\definecolor{currentstroke}{rgb}{0.000000,0.000000,0.000000}%
\pgfsetstrokecolor{currentstroke}%
\pgfsetdash{}{0pt}%
\pgfpathmoveto{\pgfqpoint{5.019198in}{1.321421in}}%
\pgfpathlineto{\pgfqpoint{5.205425in}{1.321421in}}%
\pgfpathlineto{\pgfqpoint{5.205425in}{1.403149in}}%
\pgfpathlineto{\pgfqpoint{5.019198in}{1.403149in}}%
\pgfpathlineto{\pgfqpoint{5.019198in}{1.321421in}}%
\pgfusepath{}%
\end{pgfscope}%
\begin{pgfscope}%
\pgfpathrectangle{\pgfqpoint{0.549740in}{0.463273in}}{\pgfqpoint{9.320225in}{4.495057in}}%
\pgfusepath{clip}%
\pgfsetbuttcap%
\pgfsetroundjoin%
\pgfsetlinewidth{0.000000pt}%
\definecolor{currentstroke}{rgb}{0.000000,0.000000,0.000000}%
\pgfsetstrokecolor{currentstroke}%
\pgfsetdash{}{0pt}%
\pgfpathmoveto{\pgfqpoint{5.205425in}{1.321421in}}%
\pgfpathlineto{\pgfqpoint{5.391651in}{1.321421in}}%
\pgfpathlineto{\pgfqpoint{5.391651in}{1.403149in}}%
\pgfpathlineto{\pgfqpoint{5.205425in}{1.403149in}}%
\pgfpathlineto{\pgfqpoint{5.205425in}{1.321421in}}%
\pgfusepath{}%
\end{pgfscope}%
\begin{pgfscope}%
\pgfpathrectangle{\pgfqpoint{0.549740in}{0.463273in}}{\pgfqpoint{9.320225in}{4.495057in}}%
\pgfusepath{clip}%
\pgfsetbuttcap%
\pgfsetroundjoin%
\pgfsetlinewidth{0.000000pt}%
\definecolor{currentstroke}{rgb}{0.000000,0.000000,0.000000}%
\pgfsetstrokecolor{currentstroke}%
\pgfsetdash{}{0pt}%
\pgfpathmoveto{\pgfqpoint{5.391651in}{1.321421in}}%
\pgfpathlineto{\pgfqpoint{5.577878in}{1.321421in}}%
\pgfpathlineto{\pgfqpoint{5.577878in}{1.403149in}}%
\pgfpathlineto{\pgfqpoint{5.391651in}{1.403149in}}%
\pgfpathlineto{\pgfqpoint{5.391651in}{1.321421in}}%
\pgfusepath{}%
\end{pgfscope}%
\begin{pgfscope}%
\pgfpathrectangle{\pgfqpoint{0.549740in}{0.463273in}}{\pgfqpoint{9.320225in}{4.495057in}}%
\pgfusepath{clip}%
\pgfsetbuttcap%
\pgfsetroundjoin%
\pgfsetlinewidth{0.000000pt}%
\definecolor{currentstroke}{rgb}{0.000000,0.000000,0.000000}%
\pgfsetstrokecolor{currentstroke}%
\pgfsetdash{}{0pt}%
\pgfpathmoveto{\pgfqpoint{5.577878in}{1.321421in}}%
\pgfpathlineto{\pgfqpoint{5.764104in}{1.321421in}}%
\pgfpathlineto{\pgfqpoint{5.764104in}{1.403149in}}%
\pgfpathlineto{\pgfqpoint{5.577878in}{1.403149in}}%
\pgfpathlineto{\pgfqpoint{5.577878in}{1.321421in}}%
\pgfusepath{}%
\end{pgfscope}%
\begin{pgfscope}%
\pgfpathrectangle{\pgfqpoint{0.549740in}{0.463273in}}{\pgfqpoint{9.320225in}{4.495057in}}%
\pgfusepath{clip}%
\pgfsetbuttcap%
\pgfsetroundjoin%
\definecolor{currentfill}{rgb}{0.273225,0.662144,0.968515}%
\pgfsetfillcolor{currentfill}%
\pgfsetlinewidth{0.000000pt}%
\definecolor{currentstroke}{rgb}{0.000000,0.000000,0.000000}%
\pgfsetstrokecolor{currentstroke}%
\pgfsetdash{}{0pt}%
\pgfpathmoveto{\pgfqpoint{5.764104in}{1.321421in}}%
\pgfpathlineto{\pgfqpoint{5.950331in}{1.321421in}}%
\pgfpathlineto{\pgfqpoint{5.950331in}{1.403149in}}%
\pgfpathlineto{\pgfqpoint{5.764104in}{1.403149in}}%
\pgfpathlineto{\pgfqpoint{5.764104in}{1.321421in}}%
\pgfusepath{fill}%
\end{pgfscope}%
\begin{pgfscope}%
\pgfpathrectangle{\pgfqpoint{0.549740in}{0.463273in}}{\pgfqpoint{9.320225in}{4.495057in}}%
\pgfusepath{clip}%
\pgfsetbuttcap%
\pgfsetroundjoin%
\pgfsetlinewidth{0.000000pt}%
\definecolor{currentstroke}{rgb}{0.000000,0.000000,0.000000}%
\pgfsetstrokecolor{currentstroke}%
\pgfsetdash{}{0pt}%
\pgfpathmoveto{\pgfqpoint{5.950331in}{1.321421in}}%
\pgfpathlineto{\pgfqpoint{6.136557in}{1.321421in}}%
\pgfpathlineto{\pgfqpoint{6.136557in}{1.403149in}}%
\pgfpathlineto{\pgfqpoint{5.950331in}{1.403149in}}%
\pgfpathlineto{\pgfqpoint{5.950331in}{1.321421in}}%
\pgfusepath{}%
\end{pgfscope}%
\begin{pgfscope}%
\pgfpathrectangle{\pgfqpoint{0.549740in}{0.463273in}}{\pgfqpoint{9.320225in}{4.495057in}}%
\pgfusepath{clip}%
\pgfsetbuttcap%
\pgfsetroundjoin%
\pgfsetlinewidth{0.000000pt}%
\definecolor{currentstroke}{rgb}{0.000000,0.000000,0.000000}%
\pgfsetstrokecolor{currentstroke}%
\pgfsetdash{}{0pt}%
\pgfpathmoveto{\pgfqpoint{6.136557in}{1.321421in}}%
\pgfpathlineto{\pgfqpoint{6.322784in}{1.321421in}}%
\pgfpathlineto{\pgfqpoint{6.322784in}{1.403149in}}%
\pgfpathlineto{\pgfqpoint{6.136557in}{1.403149in}}%
\pgfpathlineto{\pgfqpoint{6.136557in}{1.321421in}}%
\pgfusepath{}%
\end{pgfscope}%
\begin{pgfscope}%
\pgfpathrectangle{\pgfqpoint{0.549740in}{0.463273in}}{\pgfqpoint{9.320225in}{4.495057in}}%
\pgfusepath{clip}%
\pgfsetbuttcap%
\pgfsetroundjoin%
\pgfsetlinewidth{0.000000pt}%
\definecolor{currentstroke}{rgb}{0.000000,0.000000,0.000000}%
\pgfsetstrokecolor{currentstroke}%
\pgfsetdash{}{0pt}%
\pgfpathmoveto{\pgfqpoint{6.322784in}{1.321421in}}%
\pgfpathlineto{\pgfqpoint{6.509011in}{1.321421in}}%
\pgfpathlineto{\pgfqpoint{6.509011in}{1.403149in}}%
\pgfpathlineto{\pgfqpoint{6.322784in}{1.403149in}}%
\pgfpathlineto{\pgfqpoint{6.322784in}{1.321421in}}%
\pgfusepath{}%
\end{pgfscope}%
\begin{pgfscope}%
\pgfpathrectangle{\pgfqpoint{0.549740in}{0.463273in}}{\pgfqpoint{9.320225in}{4.495057in}}%
\pgfusepath{clip}%
\pgfsetbuttcap%
\pgfsetroundjoin%
\definecolor{currentfill}{rgb}{0.472869,0.711325,0.955316}%
\pgfsetfillcolor{currentfill}%
\pgfsetlinewidth{0.000000pt}%
\definecolor{currentstroke}{rgb}{0.000000,0.000000,0.000000}%
\pgfsetstrokecolor{currentstroke}%
\pgfsetdash{}{0pt}%
\pgfpathmoveto{\pgfqpoint{6.509011in}{1.321421in}}%
\pgfpathlineto{\pgfqpoint{6.695237in}{1.321421in}}%
\pgfpathlineto{\pgfqpoint{6.695237in}{1.403149in}}%
\pgfpathlineto{\pgfqpoint{6.509011in}{1.403149in}}%
\pgfpathlineto{\pgfqpoint{6.509011in}{1.321421in}}%
\pgfusepath{fill}%
\end{pgfscope}%
\begin{pgfscope}%
\pgfpathrectangle{\pgfqpoint{0.549740in}{0.463273in}}{\pgfqpoint{9.320225in}{4.495057in}}%
\pgfusepath{clip}%
\pgfsetbuttcap%
\pgfsetroundjoin%
\definecolor{currentfill}{rgb}{0.472869,0.711325,0.955316}%
\pgfsetfillcolor{currentfill}%
\pgfsetlinewidth{0.000000pt}%
\definecolor{currentstroke}{rgb}{0.000000,0.000000,0.000000}%
\pgfsetstrokecolor{currentstroke}%
\pgfsetdash{}{0pt}%
\pgfpathmoveto{\pgfqpoint{6.695237in}{1.321421in}}%
\pgfpathlineto{\pgfqpoint{6.881464in}{1.321421in}}%
\pgfpathlineto{\pgfqpoint{6.881464in}{1.403149in}}%
\pgfpathlineto{\pgfqpoint{6.695237in}{1.403149in}}%
\pgfpathlineto{\pgfqpoint{6.695237in}{1.321421in}}%
\pgfusepath{fill}%
\end{pgfscope}%
\begin{pgfscope}%
\pgfpathrectangle{\pgfqpoint{0.549740in}{0.463273in}}{\pgfqpoint{9.320225in}{4.495057in}}%
\pgfusepath{clip}%
\pgfsetbuttcap%
\pgfsetroundjoin%
\pgfsetlinewidth{0.000000pt}%
\definecolor{currentstroke}{rgb}{0.000000,0.000000,0.000000}%
\pgfsetstrokecolor{currentstroke}%
\pgfsetdash{}{0pt}%
\pgfpathmoveto{\pgfqpoint{6.881464in}{1.321421in}}%
\pgfpathlineto{\pgfqpoint{7.067690in}{1.321421in}}%
\pgfpathlineto{\pgfqpoint{7.067690in}{1.403149in}}%
\pgfpathlineto{\pgfqpoint{6.881464in}{1.403149in}}%
\pgfpathlineto{\pgfqpoint{6.881464in}{1.321421in}}%
\pgfusepath{}%
\end{pgfscope}%
\begin{pgfscope}%
\pgfpathrectangle{\pgfqpoint{0.549740in}{0.463273in}}{\pgfqpoint{9.320225in}{4.495057in}}%
\pgfusepath{clip}%
\pgfsetbuttcap%
\pgfsetroundjoin%
\pgfsetlinewidth{0.000000pt}%
\definecolor{currentstroke}{rgb}{0.000000,0.000000,0.000000}%
\pgfsetstrokecolor{currentstroke}%
\pgfsetdash{}{0pt}%
\pgfpathmoveto{\pgfqpoint{7.067690in}{1.321421in}}%
\pgfpathlineto{\pgfqpoint{7.253917in}{1.321421in}}%
\pgfpathlineto{\pgfqpoint{7.253917in}{1.403149in}}%
\pgfpathlineto{\pgfqpoint{7.067690in}{1.403149in}}%
\pgfpathlineto{\pgfqpoint{7.067690in}{1.321421in}}%
\pgfusepath{}%
\end{pgfscope}%
\begin{pgfscope}%
\pgfpathrectangle{\pgfqpoint{0.549740in}{0.463273in}}{\pgfqpoint{9.320225in}{4.495057in}}%
\pgfusepath{clip}%
\pgfsetbuttcap%
\pgfsetroundjoin%
\pgfsetlinewidth{0.000000pt}%
\definecolor{currentstroke}{rgb}{0.000000,0.000000,0.000000}%
\pgfsetstrokecolor{currentstroke}%
\pgfsetdash{}{0pt}%
\pgfpathmoveto{\pgfqpoint{7.253917in}{1.321421in}}%
\pgfpathlineto{\pgfqpoint{7.440143in}{1.321421in}}%
\pgfpathlineto{\pgfqpoint{7.440143in}{1.403149in}}%
\pgfpathlineto{\pgfqpoint{7.253917in}{1.403149in}}%
\pgfpathlineto{\pgfqpoint{7.253917in}{1.321421in}}%
\pgfusepath{}%
\end{pgfscope}%
\begin{pgfscope}%
\pgfpathrectangle{\pgfqpoint{0.549740in}{0.463273in}}{\pgfqpoint{9.320225in}{4.495057in}}%
\pgfusepath{clip}%
\pgfsetbuttcap%
\pgfsetroundjoin%
\pgfsetlinewidth{0.000000pt}%
\definecolor{currentstroke}{rgb}{0.000000,0.000000,0.000000}%
\pgfsetstrokecolor{currentstroke}%
\pgfsetdash{}{0pt}%
\pgfpathmoveto{\pgfqpoint{7.440143in}{1.321421in}}%
\pgfpathlineto{\pgfqpoint{7.626370in}{1.321421in}}%
\pgfpathlineto{\pgfqpoint{7.626370in}{1.403149in}}%
\pgfpathlineto{\pgfqpoint{7.440143in}{1.403149in}}%
\pgfpathlineto{\pgfqpoint{7.440143in}{1.321421in}}%
\pgfusepath{}%
\end{pgfscope}%
\begin{pgfscope}%
\pgfpathrectangle{\pgfqpoint{0.549740in}{0.463273in}}{\pgfqpoint{9.320225in}{4.495057in}}%
\pgfusepath{clip}%
\pgfsetbuttcap%
\pgfsetroundjoin%
\pgfsetlinewidth{0.000000pt}%
\definecolor{currentstroke}{rgb}{0.000000,0.000000,0.000000}%
\pgfsetstrokecolor{currentstroke}%
\pgfsetdash{}{0pt}%
\pgfpathmoveto{\pgfqpoint{7.626370in}{1.321421in}}%
\pgfpathlineto{\pgfqpoint{7.812596in}{1.321421in}}%
\pgfpathlineto{\pgfqpoint{7.812596in}{1.403149in}}%
\pgfpathlineto{\pgfqpoint{7.626370in}{1.403149in}}%
\pgfpathlineto{\pgfqpoint{7.626370in}{1.321421in}}%
\pgfusepath{}%
\end{pgfscope}%
\begin{pgfscope}%
\pgfpathrectangle{\pgfqpoint{0.549740in}{0.463273in}}{\pgfqpoint{9.320225in}{4.495057in}}%
\pgfusepath{clip}%
\pgfsetbuttcap%
\pgfsetroundjoin%
\definecolor{currentfill}{rgb}{0.472869,0.711325,0.955316}%
\pgfsetfillcolor{currentfill}%
\pgfsetlinewidth{0.000000pt}%
\definecolor{currentstroke}{rgb}{0.000000,0.000000,0.000000}%
\pgfsetstrokecolor{currentstroke}%
\pgfsetdash{}{0pt}%
\pgfpathmoveto{\pgfqpoint{7.812596in}{1.321421in}}%
\pgfpathlineto{\pgfqpoint{7.998823in}{1.321421in}}%
\pgfpathlineto{\pgfqpoint{7.998823in}{1.403149in}}%
\pgfpathlineto{\pgfqpoint{7.812596in}{1.403149in}}%
\pgfpathlineto{\pgfqpoint{7.812596in}{1.321421in}}%
\pgfusepath{fill}%
\end{pgfscope}%
\begin{pgfscope}%
\pgfpathrectangle{\pgfqpoint{0.549740in}{0.463273in}}{\pgfqpoint{9.320225in}{4.495057in}}%
\pgfusepath{clip}%
\pgfsetbuttcap%
\pgfsetroundjoin%
\pgfsetlinewidth{0.000000pt}%
\definecolor{currentstroke}{rgb}{0.000000,0.000000,0.000000}%
\pgfsetstrokecolor{currentstroke}%
\pgfsetdash{}{0pt}%
\pgfpathmoveto{\pgfqpoint{7.998823in}{1.321421in}}%
\pgfpathlineto{\pgfqpoint{8.185049in}{1.321421in}}%
\pgfpathlineto{\pgfqpoint{8.185049in}{1.403149in}}%
\pgfpathlineto{\pgfqpoint{7.998823in}{1.403149in}}%
\pgfpathlineto{\pgfqpoint{7.998823in}{1.321421in}}%
\pgfusepath{}%
\end{pgfscope}%
\begin{pgfscope}%
\pgfpathrectangle{\pgfqpoint{0.549740in}{0.463273in}}{\pgfqpoint{9.320225in}{4.495057in}}%
\pgfusepath{clip}%
\pgfsetbuttcap%
\pgfsetroundjoin%
\pgfsetlinewidth{0.000000pt}%
\definecolor{currentstroke}{rgb}{0.000000,0.000000,0.000000}%
\pgfsetstrokecolor{currentstroke}%
\pgfsetdash{}{0pt}%
\pgfpathmoveto{\pgfqpoint{8.185049in}{1.321421in}}%
\pgfpathlineto{\pgfqpoint{8.371276in}{1.321421in}}%
\pgfpathlineto{\pgfqpoint{8.371276in}{1.403149in}}%
\pgfpathlineto{\pgfqpoint{8.185049in}{1.403149in}}%
\pgfpathlineto{\pgfqpoint{8.185049in}{1.321421in}}%
\pgfusepath{}%
\end{pgfscope}%
\begin{pgfscope}%
\pgfpathrectangle{\pgfqpoint{0.549740in}{0.463273in}}{\pgfqpoint{9.320225in}{4.495057in}}%
\pgfusepath{clip}%
\pgfsetbuttcap%
\pgfsetroundjoin%
\pgfsetlinewidth{0.000000pt}%
\definecolor{currentstroke}{rgb}{0.000000,0.000000,0.000000}%
\pgfsetstrokecolor{currentstroke}%
\pgfsetdash{}{0pt}%
\pgfpathmoveto{\pgfqpoint{8.371276in}{1.321421in}}%
\pgfpathlineto{\pgfqpoint{8.557503in}{1.321421in}}%
\pgfpathlineto{\pgfqpoint{8.557503in}{1.403149in}}%
\pgfpathlineto{\pgfqpoint{8.371276in}{1.403149in}}%
\pgfpathlineto{\pgfqpoint{8.371276in}{1.321421in}}%
\pgfusepath{}%
\end{pgfscope}%
\begin{pgfscope}%
\pgfpathrectangle{\pgfqpoint{0.549740in}{0.463273in}}{\pgfqpoint{9.320225in}{4.495057in}}%
\pgfusepath{clip}%
\pgfsetbuttcap%
\pgfsetroundjoin%
\pgfsetlinewidth{0.000000pt}%
\definecolor{currentstroke}{rgb}{0.000000,0.000000,0.000000}%
\pgfsetstrokecolor{currentstroke}%
\pgfsetdash{}{0pt}%
\pgfpathmoveto{\pgfqpoint{8.557503in}{1.321421in}}%
\pgfpathlineto{\pgfqpoint{8.743729in}{1.321421in}}%
\pgfpathlineto{\pgfqpoint{8.743729in}{1.403149in}}%
\pgfpathlineto{\pgfqpoint{8.557503in}{1.403149in}}%
\pgfpathlineto{\pgfqpoint{8.557503in}{1.321421in}}%
\pgfusepath{}%
\end{pgfscope}%
\begin{pgfscope}%
\pgfpathrectangle{\pgfqpoint{0.549740in}{0.463273in}}{\pgfqpoint{9.320225in}{4.495057in}}%
\pgfusepath{clip}%
\pgfsetbuttcap%
\pgfsetroundjoin%
\definecolor{currentfill}{rgb}{0.472869,0.711325,0.955316}%
\pgfsetfillcolor{currentfill}%
\pgfsetlinewidth{0.000000pt}%
\definecolor{currentstroke}{rgb}{0.000000,0.000000,0.000000}%
\pgfsetstrokecolor{currentstroke}%
\pgfsetdash{}{0pt}%
\pgfpathmoveto{\pgfqpoint{8.743729in}{1.321421in}}%
\pgfpathlineto{\pgfqpoint{8.929956in}{1.321421in}}%
\pgfpathlineto{\pgfqpoint{8.929956in}{1.403149in}}%
\pgfpathlineto{\pgfqpoint{8.743729in}{1.403149in}}%
\pgfpathlineto{\pgfqpoint{8.743729in}{1.321421in}}%
\pgfusepath{fill}%
\end{pgfscope}%
\begin{pgfscope}%
\pgfpathrectangle{\pgfqpoint{0.549740in}{0.463273in}}{\pgfqpoint{9.320225in}{4.495057in}}%
\pgfusepath{clip}%
\pgfsetbuttcap%
\pgfsetroundjoin%
\pgfsetlinewidth{0.000000pt}%
\definecolor{currentstroke}{rgb}{0.000000,0.000000,0.000000}%
\pgfsetstrokecolor{currentstroke}%
\pgfsetdash{}{0pt}%
\pgfpathmoveto{\pgfqpoint{8.929956in}{1.321421in}}%
\pgfpathlineto{\pgfqpoint{9.116182in}{1.321421in}}%
\pgfpathlineto{\pgfqpoint{9.116182in}{1.403149in}}%
\pgfpathlineto{\pgfqpoint{8.929956in}{1.403149in}}%
\pgfpathlineto{\pgfqpoint{8.929956in}{1.321421in}}%
\pgfusepath{}%
\end{pgfscope}%
\begin{pgfscope}%
\pgfpathrectangle{\pgfqpoint{0.549740in}{0.463273in}}{\pgfqpoint{9.320225in}{4.495057in}}%
\pgfusepath{clip}%
\pgfsetbuttcap%
\pgfsetroundjoin%
\pgfsetlinewidth{0.000000pt}%
\definecolor{currentstroke}{rgb}{0.000000,0.000000,0.000000}%
\pgfsetstrokecolor{currentstroke}%
\pgfsetdash{}{0pt}%
\pgfpathmoveto{\pgfqpoint{9.116182in}{1.321421in}}%
\pgfpathlineto{\pgfqpoint{9.302409in}{1.321421in}}%
\pgfpathlineto{\pgfqpoint{9.302409in}{1.403149in}}%
\pgfpathlineto{\pgfqpoint{9.116182in}{1.403149in}}%
\pgfpathlineto{\pgfqpoint{9.116182in}{1.321421in}}%
\pgfusepath{}%
\end{pgfscope}%
\begin{pgfscope}%
\pgfpathrectangle{\pgfqpoint{0.549740in}{0.463273in}}{\pgfqpoint{9.320225in}{4.495057in}}%
\pgfusepath{clip}%
\pgfsetbuttcap%
\pgfsetroundjoin%
\pgfsetlinewidth{0.000000pt}%
\definecolor{currentstroke}{rgb}{0.000000,0.000000,0.000000}%
\pgfsetstrokecolor{currentstroke}%
\pgfsetdash{}{0pt}%
\pgfpathmoveto{\pgfqpoint{9.302409in}{1.321421in}}%
\pgfpathlineto{\pgfqpoint{9.488635in}{1.321421in}}%
\pgfpathlineto{\pgfqpoint{9.488635in}{1.403149in}}%
\pgfpathlineto{\pgfqpoint{9.302409in}{1.403149in}}%
\pgfpathlineto{\pgfqpoint{9.302409in}{1.321421in}}%
\pgfusepath{}%
\end{pgfscope}%
\begin{pgfscope}%
\pgfpathrectangle{\pgfqpoint{0.549740in}{0.463273in}}{\pgfqpoint{9.320225in}{4.495057in}}%
\pgfusepath{clip}%
\pgfsetbuttcap%
\pgfsetroundjoin%
\pgfsetlinewidth{0.000000pt}%
\definecolor{currentstroke}{rgb}{0.000000,0.000000,0.000000}%
\pgfsetstrokecolor{currentstroke}%
\pgfsetdash{}{0pt}%
\pgfpathmoveto{\pgfqpoint{9.488635in}{1.321421in}}%
\pgfpathlineto{\pgfqpoint{9.674862in}{1.321421in}}%
\pgfpathlineto{\pgfqpoint{9.674862in}{1.403149in}}%
\pgfpathlineto{\pgfqpoint{9.488635in}{1.403149in}}%
\pgfpathlineto{\pgfqpoint{9.488635in}{1.321421in}}%
\pgfusepath{}%
\end{pgfscope}%
\begin{pgfscope}%
\pgfpathrectangle{\pgfqpoint{0.549740in}{0.463273in}}{\pgfqpoint{9.320225in}{4.495057in}}%
\pgfusepath{clip}%
\pgfsetbuttcap%
\pgfsetroundjoin%
\pgfsetlinewidth{0.000000pt}%
\definecolor{currentstroke}{rgb}{0.000000,0.000000,0.000000}%
\pgfsetstrokecolor{currentstroke}%
\pgfsetdash{}{0pt}%
\pgfpathmoveto{\pgfqpoint{9.674862in}{1.321421in}}%
\pgfpathlineto{\pgfqpoint{9.861088in}{1.321421in}}%
\pgfpathlineto{\pgfqpoint{9.861088in}{1.403149in}}%
\pgfpathlineto{\pgfqpoint{9.674862in}{1.403149in}}%
\pgfpathlineto{\pgfqpoint{9.674862in}{1.321421in}}%
\pgfusepath{}%
\end{pgfscope}%
\begin{pgfscope}%
\pgfpathrectangle{\pgfqpoint{0.549740in}{0.463273in}}{\pgfqpoint{9.320225in}{4.495057in}}%
\pgfusepath{clip}%
\pgfsetbuttcap%
\pgfsetroundjoin%
\pgfsetlinewidth{0.000000pt}%
\definecolor{currentstroke}{rgb}{0.000000,0.000000,0.000000}%
\pgfsetstrokecolor{currentstroke}%
\pgfsetdash{}{0pt}%
\pgfpathmoveto{\pgfqpoint{0.549761in}{1.403149in}}%
\pgfpathlineto{\pgfqpoint{0.735988in}{1.403149in}}%
\pgfpathlineto{\pgfqpoint{0.735988in}{1.484877in}}%
\pgfpathlineto{\pgfqpoint{0.549761in}{1.484877in}}%
\pgfpathlineto{\pgfqpoint{0.549761in}{1.403149in}}%
\pgfusepath{}%
\end{pgfscope}%
\begin{pgfscope}%
\pgfpathrectangle{\pgfqpoint{0.549740in}{0.463273in}}{\pgfqpoint{9.320225in}{4.495057in}}%
\pgfusepath{clip}%
\pgfsetbuttcap%
\pgfsetroundjoin%
\pgfsetlinewidth{0.000000pt}%
\definecolor{currentstroke}{rgb}{0.000000,0.000000,0.000000}%
\pgfsetstrokecolor{currentstroke}%
\pgfsetdash{}{0pt}%
\pgfpathmoveto{\pgfqpoint{0.735988in}{1.403149in}}%
\pgfpathlineto{\pgfqpoint{0.922214in}{1.403149in}}%
\pgfpathlineto{\pgfqpoint{0.922214in}{1.484877in}}%
\pgfpathlineto{\pgfqpoint{0.735988in}{1.484877in}}%
\pgfpathlineto{\pgfqpoint{0.735988in}{1.403149in}}%
\pgfusepath{}%
\end{pgfscope}%
\begin{pgfscope}%
\pgfpathrectangle{\pgfqpoint{0.549740in}{0.463273in}}{\pgfqpoint{9.320225in}{4.495057in}}%
\pgfusepath{clip}%
\pgfsetbuttcap%
\pgfsetroundjoin%
\pgfsetlinewidth{0.000000pt}%
\definecolor{currentstroke}{rgb}{0.000000,0.000000,0.000000}%
\pgfsetstrokecolor{currentstroke}%
\pgfsetdash{}{0pt}%
\pgfpathmoveto{\pgfqpoint{0.922214in}{1.403149in}}%
\pgfpathlineto{\pgfqpoint{1.108441in}{1.403149in}}%
\pgfpathlineto{\pgfqpoint{1.108441in}{1.484877in}}%
\pgfpathlineto{\pgfqpoint{0.922214in}{1.484877in}}%
\pgfpathlineto{\pgfqpoint{0.922214in}{1.403149in}}%
\pgfusepath{}%
\end{pgfscope}%
\begin{pgfscope}%
\pgfpathrectangle{\pgfqpoint{0.549740in}{0.463273in}}{\pgfqpoint{9.320225in}{4.495057in}}%
\pgfusepath{clip}%
\pgfsetbuttcap%
\pgfsetroundjoin%
\pgfsetlinewidth{0.000000pt}%
\definecolor{currentstroke}{rgb}{0.000000,0.000000,0.000000}%
\pgfsetstrokecolor{currentstroke}%
\pgfsetdash{}{0pt}%
\pgfpathmoveto{\pgfqpoint{1.108441in}{1.403149in}}%
\pgfpathlineto{\pgfqpoint{1.294667in}{1.403149in}}%
\pgfpathlineto{\pgfqpoint{1.294667in}{1.484877in}}%
\pgfpathlineto{\pgfqpoint{1.108441in}{1.484877in}}%
\pgfpathlineto{\pgfqpoint{1.108441in}{1.403149in}}%
\pgfusepath{}%
\end{pgfscope}%
\begin{pgfscope}%
\pgfpathrectangle{\pgfqpoint{0.549740in}{0.463273in}}{\pgfqpoint{9.320225in}{4.495057in}}%
\pgfusepath{clip}%
\pgfsetbuttcap%
\pgfsetroundjoin%
\pgfsetlinewidth{0.000000pt}%
\definecolor{currentstroke}{rgb}{0.000000,0.000000,0.000000}%
\pgfsetstrokecolor{currentstroke}%
\pgfsetdash{}{0pt}%
\pgfpathmoveto{\pgfqpoint{1.294667in}{1.403149in}}%
\pgfpathlineto{\pgfqpoint{1.480894in}{1.403149in}}%
\pgfpathlineto{\pgfqpoint{1.480894in}{1.484877in}}%
\pgfpathlineto{\pgfqpoint{1.294667in}{1.484877in}}%
\pgfpathlineto{\pgfqpoint{1.294667in}{1.403149in}}%
\pgfusepath{}%
\end{pgfscope}%
\begin{pgfscope}%
\pgfpathrectangle{\pgfqpoint{0.549740in}{0.463273in}}{\pgfqpoint{9.320225in}{4.495057in}}%
\pgfusepath{clip}%
\pgfsetbuttcap%
\pgfsetroundjoin%
\pgfsetlinewidth{0.000000pt}%
\definecolor{currentstroke}{rgb}{0.000000,0.000000,0.000000}%
\pgfsetstrokecolor{currentstroke}%
\pgfsetdash{}{0pt}%
\pgfpathmoveto{\pgfqpoint{1.480894in}{1.403149in}}%
\pgfpathlineto{\pgfqpoint{1.667120in}{1.403149in}}%
\pgfpathlineto{\pgfqpoint{1.667120in}{1.484877in}}%
\pgfpathlineto{\pgfqpoint{1.480894in}{1.484877in}}%
\pgfpathlineto{\pgfqpoint{1.480894in}{1.403149in}}%
\pgfusepath{}%
\end{pgfscope}%
\begin{pgfscope}%
\pgfpathrectangle{\pgfqpoint{0.549740in}{0.463273in}}{\pgfqpoint{9.320225in}{4.495057in}}%
\pgfusepath{clip}%
\pgfsetbuttcap%
\pgfsetroundjoin%
\pgfsetlinewidth{0.000000pt}%
\definecolor{currentstroke}{rgb}{0.000000,0.000000,0.000000}%
\pgfsetstrokecolor{currentstroke}%
\pgfsetdash{}{0pt}%
\pgfpathmoveto{\pgfqpoint{1.667120in}{1.403149in}}%
\pgfpathlineto{\pgfqpoint{1.853347in}{1.403149in}}%
\pgfpathlineto{\pgfqpoint{1.853347in}{1.484877in}}%
\pgfpathlineto{\pgfqpoint{1.667120in}{1.484877in}}%
\pgfpathlineto{\pgfqpoint{1.667120in}{1.403149in}}%
\pgfusepath{}%
\end{pgfscope}%
\begin{pgfscope}%
\pgfpathrectangle{\pgfqpoint{0.549740in}{0.463273in}}{\pgfqpoint{9.320225in}{4.495057in}}%
\pgfusepath{clip}%
\pgfsetbuttcap%
\pgfsetroundjoin%
\pgfsetlinewidth{0.000000pt}%
\definecolor{currentstroke}{rgb}{0.000000,0.000000,0.000000}%
\pgfsetstrokecolor{currentstroke}%
\pgfsetdash{}{0pt}%
\pgfpathmoveto{\pgfqpoint{1.853347in}{1.403149in}}%
\pgfpathlineto{\pgfqpoint{2.039573in}{1.403149in}}%
\pgfpathlineto{\pgfqpoint{2.039573in}{1.484877in}}%
\pgfpathlineto{\pgfqpoint{1.853347in}{1.484877in}}%
\pgfpathlineto{\pgfqpoint{1.853347in}{1.403149in}}%
\pgfusepath{}%
\end{pgfscope}%
\begin{pgfscope}%
\pgfpathrectangle{\pgfqpoint{0.549740in}{0.463273in}}{\pgfqpoint{9.320225in}{4.495057in}}%
\pgfusepath{clip}%
\pgfsetbuttcap%
\pgfsetroundjoin%
\pgfsetlinewidth{0.000000pt}%
\definecolor{currentstroke}{rgb}{0.000000,0.000000,0.000000}%
\pgfsetstrokecolor{currentstroke}%
\pgfsetdash{}{0pt}%
\pgfpathmoveto{\pgfqpoint{2.039573in}{1.403149in}}%
\pgfpathlineto{\pgfqpoint{2.225800in}{1.403149in}}%
\pgfpathlineto{\pgfqpoint{2.225800in}{1.484877in}}%
\pgfpathlineto{\pgfqpoint{2.039573in}{1.484877in}}%
\pgfpathlineto{\pgfqpoint{2.039573in}{1.403149in}}%
\pgfusepath{}%
\end{pgfscope}%
\begin{pgfscope}%
\pgfpathrectangle{\pgfqpoint{0.549740in}{0.463273in}}{\pgfqpoint{9.320225in}{4.495057in}}%
\pgfusepath{clip}%
\pgfsetbuttcap%
\pgfsetroundjoin%
\pgfsetlinewidth{0.000000pt}%
\definecolor{currentstroke}{rgb}{0.000000,0.000000,0.000000}%
\pgfsetstrokecolor{currentstroke}%
\pgfsetdash{}{0pt}%
\pgfpathmoveto{\pgfqpoint{2.225800in}{1.403149in}}%
\pgfpathlineto{\pgfqpoint{2.412027in}{1.403149in}}%
\pgfpathlineto{\pgfqpoint{2.412027in}{1.484877in}}%
\pgfpathlineto{\pgfqpoint{2.225800in}{1.484877in}}%
\pgfpathlineto{\pgfqpoint{2.225800in}{1.403149in}}%
\pgfusepath{}%
\end{pgfscope}%
\begin{pgfscope}%
\pgfpathrectangle{\pgfqpoint{0.549740in}{0.463273in}}{\pgfqpoint{9.320225in}{4.495057in}}%
\pgfusepath{clip}%
\pgfsetbuttcap%
\pgfsetroundjoin%
\pgfsetlinewidth{0.000000pt}%
\definecolor{currentstroke}{rgb}{0.000000,0.000000,0.000000}%
\pgfsetstrokecolor{currentstroke}%
\pgfsetdash{}{0pt}%
\pgfpathmoveto{\pgfqpoint{2.412027in}{1.403149in}}%
\pgfpathlineto{\pgfqpoint{2.598253in}{1.403149in}}%
\pgfpathlineto{\pgfqpoint{2.598253in}{1.484877in}}%
\pgfpathlineto{\pgfqpoint{2.412027in}{1.484877in}}%
\pgfpathlineto{\pgfqpoint{2.412027in}{1.403149in}}%
\pgfusepath{}%
\end{pgfscope}%
\begin{pgfscope}%
\pgfpathrectangle{\pgfqpoint{0.549740in}{0.463273in}}{\pgfqpoint{9.320225in}{4.495057in}}%
\pgfusepath{clip}%
\pgfsetbuttcap%
\pgfsetroundjoin%
\pgfsetlinewidth{0.000000pt}%
\definecolor{currentstroke}{rgb}{0.000000,0.000000,0.000000}%
\pgfsetstrokecolor{currentstroke}%
\pgfsetdash{}{0pt}%
\pgfpathmoveto{\pgfqpoint{2.598253in}{1.403149in}}%
\pgfpathlineto{\pgfqpoint{2.784480in}{1.403149in}}%
\pgfpathlineto{\pgfqpoint{2.784480in}{1.484877in}}%
\pgfpathlineto{\pgfqpoint{2.598253in}{1.484877in}}%
\pgfpathlineto{\pgfqpoint{2.598253in}{1.403149in}}%
\pgfusepath{}%
\end{pgfscope}%
\begin{pgfscope}%
\pgfpathrectangle{\pgfqpoint{0.549740in}{0.463273in}}{\pgfqpoint{9.320225in}{4.495057in}}%
\pgfusepath{clip}%
\pgfsetbuttcap%
\pgfsetroundjoin%
\pgfsetlinewidth{0.000000pt}%
\definecolor{currentstroke}{rgb}{0.000000,0.000000,0.000000}%
\pgfsetstrokecolor{currentstroke}%
\pgfsetdash{}{0pt}%
\pgfpathmoveto{\pgfqpoint{2.784480in}{1.403149in}}%
\pgfpathlineto{\pgfqpoint{2.970706in}{1.403149in}}%
\pgfpathlineto{\pgfqpoint{2.970706in}{1.484877in}}%
\pgfpathlineto{\pgfqpoint{2.784480in}{1.484877in}}%
\pgfpathlineto{\pgfqpoint{2.784480in}{1.403149in}}%
\pgfusepath{}%
\end{pgfscope}%
\begin{pgfscope}%
\pgfpathrectangle{\pgfqpoint{0.549740in}{0.463273in}}{\pgfqpoint{9.320225in}{4.495057in}}%
\pgfusepath{clip}%
\pgfsetbuttcap%
\pgfsetroundjoin%
\pgfsetlinewidth{0.000000pt}%
\definecolor{currentstroke}{rgb}{0.000000,0.000000,0.000000}%
\pgfsetstrokecolor{currentstroke}%
\pgfsetdash{}{0pt}%
\pgfpathmoveto{\pgfqpoint{2.970706in}{1.403149in}}%
\pgfpathlineto{\pgfqpoint{3.156933in}{1.403149in}}%
\pgfpathlineto{\pgfqpoint{3.156933in}{1.484877in}}%
\pgfpathlineto{\pgfqpoint{2.970706in}{1.484877in}}%
\pgfpathlineto{\pgfqpoint{2.970706in}{1.403149in}}%
\pgfusepath{}%
\end{pgfscope}%
\begin{pgfscope}%
\pgfpathrectangle{\pgfqpoint{0.549740in}{0.463273in}}{\pgfqpoint{9.320225in}{4.495057in}}%
\pgfusepath{clip}%
\pgfsetbuttcap%
\pgfsetroundjoin%
\pgfsetlinewidth{0.000000pt}%
\definecolor{currentstroke}{rgb}{0.000000,0.000000,0.000000}%
\pgfsetstrokecolor{currentstroke}%
\pgfsetdash{}{0pt}%
\pgfpathmoveto{\pgfqpoint{3.156933in}{1.403149in}}%
\pgfpathlineto{\pgfqpoint{3.343159in}{1.403149in}}%
\pgfpathlineto{\pgfqpoint{3.343159in}{1.484877in}}%
\pgfpathlineto{\pgfqpoint{3.156933in}{1.484877in}}%
\pgfpathlineto{\pgfqpoint{3.156933in}{1.403149in}}%
\pgfusepath{}%
\end{pgfscope}%
\begin{pgfscope}%
\pgfpathrectangle{\pgfqpoint{0.549740in}{0.463273in}}{\pgfqpoint{9.320225in}{4.495057in}}%
\pgfusepath{clip}%
\pgfsetbuttcap%
\pgfsetroundjoin%
\pgfsetlinewidth{0.000000pt}%
\definecolor{currentstroke}{rgb}{0.000000,0.000000,0.000000}%
\pgfsetstrokecolor{currentstroke}%
\pgfsetdash{}{0pt}%
\pgfpathmoveto{\pgfqpoint{3.343159in}{1.403149in}}%
\pgfpathlineto{\pgfqpoint{3.529386in}{1.403149in}}%
\pgfpathlineto{\pgfqpoint{3.529386in}{1.484877in}}%
\pgfpathlineto{\pgfqpoint{3.343159in}{1.484877in}}%
\pgfpathlineto{\pgfqpoint{3.343159in}{1.403149in}}%
\pgfusepath{}%
\end{pgfscope}%
\begin{pgfscope}%
\pgfpathrectangle{\pgfqpoint{0.549740in}{0.463273in}}{\pgfqpoint{9.320225in}{4.495057in}}%
\pgfusepath{clip}%
\pgfsetbuttcap%
\pgfsetroundjoin%
\pgfsetlinewidth{0.000000pt}%
\definecolor{currentstroke}{rgb}{0.000000,0.000000,0.000000}%
\pgfsetstrokecolor{currentstroke}%
\pgfsetdash{}{0pt}%
\pgfpathmoveto{\pgfqpoint{3.529386in}{1.403149in}}%
\pgfpathlineto{\pgfqpoint{3.715612in}{1.403149in}}%
\pgfpathlineto{\pgfqpoint{3.715612in}{1.484877in}}%
\pgfpathlineto{\pgfqpoint{3.529386in}{1.484877in}}%
\pgfpathlineto{\pgfqpoint{3.529386in}{1.403149in}}%
\pgfusepath{}%
\end{pgfscope}%
\begin{pgfscope}%
\pgfpathrectangle{\pgfqpoint{0.549740in}{0.463273in}}{\pgfqpoint{9.320225in}{4.495057in}}%
\pgfusepath{clip}%
\pgfsetbuttcap%
\pgfsetroundjoin%
\pgfsetlinewidth{0.000000pt}%
\definecolor{currentstroke}{rgb}{0.000000,0.000000,0.000000}%
\pgfsetstrokecolor{currentstroke}%
\pgfsetdash{}{0pt}%
\pgfpathmoveto{\pgfqpoint{3.715612in}{1.403149in}}%
\pgfpathlineto{\pgfqpoint{3.901839in}{1.403149in}}%
\pgfpathlineto{\pgfqpoint{3.901839in}{1.484877in}}%
\pgfpathlineto{\pgfqpoint{3.715612in}{1.484877in}}%
\pgfpathlineto{\pgfqpoint{3.715612in}{1.403149in}}%
\pgfusepath{}%
\end{pgfscope}%
\begin{pgfscope}%
\pgfpathrectangle{\pgfqpoint{0.549740in}{0.463273in}}{\pgfqpoint{9.320225in}{4.495057in}}%
\pgfusepath{clip}%
\pgfsetbuttcap%
\pgfsetroundjoin%
\pgfsetlinewidth{0.000000pt}%
\definecolor{currentstroke}{rgb}{0.000000,0.000000,0.000000}%
\pgfsetstrokecolor{currentstroke}%
\pgfsetdash{}{0pt}%
\pgfpathmoveto{\pgfqpoint{3.901839in}{1.403149in}}%
\pgfpathlineto{\pgfqpoint{4.088065in}{1.403149in}}%
\pgfpathlineto{\pgfqpoint{4.088065in}{1.484877in}}%
\pgfpathlineto{\pgfqpoint{3.901839in}{1.484877in}}%
\pgfpathlineto{\pgfqpoint{3.901839in}{1.403149in}}%
\pgfusepath{}%
\end{pgfscope}%
\begin{pgfscope}%
\pgfpathrectangle{\pgfqpoint{0.549740in}{0.463273in}}{\pgfqpoint{9.320225in}{4.495057in}}%
\pgfusepath{clip}%
\pgfsetbuttcap%
\pgfsetroundjoin%
\pgfsetlinewidth{0.000000pt}%
\definecolor{currentstroke}{rgb}{0.000000,0.000000,0.000000}%
\pgfsetstrokecolor{currentstroke}%
\pgfsetdash{}{0pt}%
\pgfpathmoveto{\pgfqpoint{4.088065in}{1.403149in}}%
\pgfpathlineto{\pgfqpoint{4.274292in}{1.403149in}}%
\pgfpathlineto{\pgfqpoint{4.274292in}{1.484877in}}%
\pgfpathlineto{\pgfqpoint{4.088065in}{1.484877in}}%
\pgfpathlineto{\pgfqpoint{4.088065in}{1.403149in}}%
\pgfusepath{}%
\end{pgfscope}%
\begin{pgfscope}%
\pgfpathrectangle{\pgfqpoint{0.549740in}{0.463273in}}{\pgfqpoint{9.320225in}{4.495057in}}%
\pgfusepath{clip}%
\pgfsetbuttcap%
\pgfsetroundjoin%
\pgfsetlinewidth{0.000000pt}%
\definecolor{currentstroke}{rgb}{0.000000,0.000000,0.000000}%
\pgfsetstrokecolor{currentstroke}%
\pgfsetdash{}{0pt}%
\pgfpathmoveto{\pgfqpoint{4.274292in}{1.403149in}}%
\pgfpathlineto{\pgfqpoint{4.460519in}{1.403149in}}%
\pgfpathlineto{\pgfqpoint{4.460519in}{1.484877in}}%
\pgfpathlineto{\pgfqpoint{4.274292in}{1.484877in}}%
\pgfpathlineto{\pgfqpoint{4.274292in}{1.403149in}}%
\pgfusepath{}%
\end{pgfscope}%
\begin{pgfscope}%
\pgfpathrectangle{\pgfqpoint{0.549740in}{0.463273in}}{\pgfqpoint{9.320225in}{4.495057in}}%
\pgfusepath{clip}%
\pgfsetbuttcap%
\pgfsetroundjoin%
\pgfsetlinewidth{0.000000pt}%
\definecolor{currentstroke}{rgb}{0.000000,0.000000,0.000000}%
\pgfsetstrokecolor{currentstroke}%
\pgfsetdash{}{0pt}%
\pgfpathmoveto{\pgfqpoint{4.460519in}{1.403149in}}%
\pgfpathlineto{\pgfqpoint{4.646745in}{1.403149in}}%
\pgfpathlineto{\pgfqpoint{4.646745in}{1.484877in}}%
\pgfpathlineto{\pgfqpoint{4.460519in}{1.484877in}}%
\pgfpathlineto{\pgfqpoint{4.460519in}{1.403149in}}%
\pgfusepath{}%
\end{pgfscope}%
\begin{pgfscope}%
\pgfpathrectangle{\pgfqpoint{0.549740in}{0.463273in}}{\pgfqpoint{9.320225in}{4.495057in}}%
\pgfusepath{clip}%
\pgfsetbuttcap%
\pgfsetroundjoin%
\pgfsetlinewidth{0.000000pt}%
\definecolor{currentstroke}{rgb}{0.000000,0.000000,0.000000}%
\pgfsetstrokecolor{currentstroke}%
\pgfsetdash{}{0pt}%
\pgfpathmoveto{\pgfqpoint{4.646745in}{1.403149in}}%
\pgfpathlineto{\pgfqpoint{4.832972in}{1.403149in}}%
\pgfpathlineto{\pgfqpoint{4.832972in}{1.484877in}}%
\pgfpathlineto{\pgfqpoint{4.646745in}{1.484877in}}%
\pgfpathlineto{\pgfqpoint{4.646745in}{1.403149in}}%
\pgfusepath{}%
\end{pgfscope}%
\begin{pgfscope}%
\pgfpathrectangle{\pgfqpoint{0.549740in}{0.463273in}}{\pgfqpoint{9.320225in}{4.495057in}}%
\pgfusepath{clip}%
\pgfsetbuttcap%
\pgfsetroundjoin%
\pgfsetlinewidth{0.000000pt}%
\definecolor{currentstroke}{rgb}{0.000000,0.000000,0.000000}%
\pgfsetstrokecolor{currentstroke}%
\pgfsetdash{}{0pt}%
\pgfpathmoveto{\pgfqpoint{4.832972in}{1.403149in}}%
\pgfpathlineto{\pgfqpoint{5.019198in}{1.403149in}}%
\pgfpathlineto{\pgfqpoint{5.019198in}{1.484877in}}%
\pgfpathlineto{\pgfqpoint{4.832972in}{1.484877in}}%
\pgfpathlineto{\pgfqpoint{4.832972in}{1.403149in}}%
\pgfusepath{}%
\end{pgfscope}%
\begin{pgfscope}%
\pgfpathrectangle{\pgfqpoint{0.549740in}{0.463273in}}{\pgfqpoint{9.320225in}{4.495057in}}%
\pgfusepath{clip}%
\pgfsetbuttcap%
\pgfsetroundjoin%
\pgfsetlinewidth{0.000000pt}%
\definecolor{currentstroke}{rgb}{0.000000,0.000000,0.000000}%
\pgfsetstrokecolor{currentstroke}%
\pgfsetdash{}{0pt}%
\pgfpathmoveto{\pgfqpoint{5.019198in}{1.403149in}}%
\pgfpathlineto{\pgfqpoint{5.205425in}{1.403149in}}%
\pgfpathlineto{\pgfqpoint{5.205425in}{1.484877in}}%
\pgfpathlineto{\pgfqpoint{5.019198in}{1.484877in}}%
\pgfpathlineto{\pgfqpoint{5.019198in}{1.403149in}}%
\pgfusepath{}%
\end{pgfscope}%
\begin{pgfscope}%
\pgfpathrectangle{\pgfqpoint{0.549740in}{0.463273in}}{\pgfqpoint{9.320225in}{4.495057in}}%
\pgfusepath{clip}%
\pgfsetbuttcap%
\pgfsetroundjoin%
\pgfsetlinewidth{0.000000pt}%
\definecolor{currentstroke}{rgb}{0.000000,0.000000,0.000000}%
\pgfsetstrokecolor{currentstroke}%
\pgfsetdash{}{0pt}%
\pgfpathmoveto{\pgfqpoint{5.205425in}{1.403149in}}%
\pgfpathlineto{\pgfqpoint{5.391651in}{1.403149in}}%
\pgfpathlineto{\pgfqpoint{5.391651in}{1.484877in}}%
\pgfpathlineto{\pgfqpoint{5.205425in}{1.484877in}}%
\pgfpathlineto{\pgfqpoint{5.205425in}{1.403149in}}%
\pgfusepath{}%
\end{pgfscope}%
\begin{pgfscope}%
\pgfpathrectangle{\pgfqpoint{0.549740in}{0.463273in}}{\pgfqpoint{9.320225in}{4.495057in}}%
\pgfusepath{clip}%
\pgfsetbuttcap%
\pgfsetroundjoin%
\pgfsetlinewidth{0.000000pt}%
\definecolor{currentstroke}{rgb}{0.000000,0.000000,0.000000}%
\pgfsetstrokecolor{currentstroke}%
\pgfsetdash{}{0pt}%
\pgfpathmoveto{\pgfqpoint{5.391651in}{1.403149in}}%
\pgfpathlineto{\pgfqpoint{5.577878in}{1.403149in}}%
\pgfpathlineto{\pgfqpoint{5.577878in}{1.484877in}}%
\pgfpathlineto{\pgfqpoint{5.391651in}{1.484877in}}%
\pgfpathlineto{\pgfqpoint{5.391651in}{1.403149in}}%
\pgfusepath{}%
\end{pgfscope}%
\begin{pgfscope}%
\pgfpathrectangle{\pgfqpoint{0.549740in}{0.463273in}}{\pgfqpoint{9.320225in}{4.495057in}}%
\pgfusepath{clip}%
\pgfsetbuttcap%
\pgfsetroundjoin%
\pgfsetlinewidth{0.000000pt}%
\definecolor{currentstroke}{rgb}{0.000000,0.000000,0.000000}%
\pgfsetstrokecolor{currentstroke}%
\pgfsetdash{}{0pt}%
\pgfpathmoveto{\pgfqpoint{5.577878in}{1.403149in}}%
\pgfpathlineto{\pgfqpoint{5.764104in}{1.403149in}}%
\pgfpathlineto{\pgfqpoint{5.764104in}{1.484877in}}%
\pgfpathlineto{\pgfqpoint{5.577878in}{1.484877in}}%
\pgfpathlineto{\pgfqpoint{5.577878in}{1.403149in}}%
\pgfusepath{}%
\end{pgfscope}%
\begin{pgfscope}%
\pgfpathrectangle{\pgfqpoint{0.549740in}{0.463273in}}{\pgfqpoint{9.320225in}{4.495057in}}%
\pgfusepath{clip}%
\pgfsetbuttcap%
\pgfsetroundjoin%
\definecolor{currentfill}{rgb}{0.273225,0.662144,0.968515}%
\pgfsetfillcolor{currentfill}%
\pgfsetlinewidth{0.000000pt}%
\definecolor{currentstroke}{rgb}{0.000000,0.000000,0.000000}%
\pgfsetstrokecolor{currentstroke}%
\pgfsetdash{}{0pt}%
\pgfpathmoveto{\pgfqpoint{5.764104in}{1.403149in}}%
\pgfpathlineto{\pgfqpoint{5.950331in}{1.403149in}}%
\pgfpathlineto{\pgfqpoint{5.950331in}{1.484877in}}%
\pgfpathlineto{\pgfqpoint{5.764104in}{1.484877in}}%
\pgfpathlineto{\pgfqpoint{5.764104in}{1.403149in}}%
\pgfusepath{fill}%
\end{pgfscope}%
\begin{pgfscope}%
\pgfpathrectangle{\pgfqpoint{0.549740in}{0.463273in}}{\pgfqpoint{9.320225in}{4.495057in}}%
\pgfusepath{clip}%
\pgfsetbuttcap%
\pgfsetroundjoin%
\pgfsetlinewidth{0.000000pt}%
\definecolor{currentstroke}{rgb}{0.000000,0.000000,0.000000}%
\pgfsetstrokecolor{currentstroke}%
\pgfsetdash{}{0pt}%
\pgfpathmoveto{\pgfqpoint{5.950331in}{1.403149in}}%
\pgfpathlineto{\pgfqpoint{6.136557in}{1.403149in}}%
\pgfpathlineto{\pgfqpoint{6.136557in}{1.484877in}}%
\pgfpathlineto{\pgfqpoint{5.950331in}{1.484877in}}%
\pgfpathlineto{\pgfqpoint{5.950331in}{1.403149in}}%
\pgfusepath{}%
\end{pgfscope}%
\begin{pgfscope}%
\pgfpathrectangle{\pgfqpoint{0.549740in}{0.463273in}}{\pgfqpoint{9.320225in}{4.495057in}}%
\pgfusepath{clip}%
\pgfsetbuttcap%
\pgfsetroundjoin%
\pgfsetlinewidth{0.000000pt}%
\definecolor{currentstroke}{rgb}{0.000000,0.000000,0.000000}%
\pgfsetstrokecolor{currentstroke}%
\pgfsetdash{}{0pt}%
\pgfpathmoveto{\pgfqpoint{6.136557in}{1.403149in}}%
\pgfpathlineto{\pgfqpoint{6.322784in}{1.403149in}}%
\pgfpathlineto{\pgfqpoint{6.322784in}{1.484877in}}%
\pgfpathlineto{\pgfqpoint{6.136557in}{1.484877in}}%
\pgfpathlineto{\pgfqpoint{6.136557in}{1.403149in}}%
\pgfusepath{}%
\end{pgfscope}%
\begin{pgfscope}%
\pgfpathrectangle{\pgfqpoint{0.549740in}{0.463273in}}{\pgfqpoint{9.320225in}{4.495057in}}%
\pgfusepath{clip}%
\pgfsetbuttcap%
\pgfsetroundjoin%
\pgfsetlinewidth{0.000000pt}%
\definecolor{currentstroke}{rgb}{0.000000,0.000000,0.000000}%
\pgfsetstrokecolor{currentstroke}%
\pgfsetdash{}{0pt}%
\pgfpathmoveto{\pgfqpoint{6.322784in}{1.403149in}}%
\pgfpathlineto{\pgfqpoint{6.509011in}{1.403149in}}%
\pgfpathlineto{\pgfqpoint{6.509011in}{1.484877in}}%
\pgfpathlineto{\pgfqpoint{6.322784in}{1.484877in}}%
\pgfpathlineto{\pgfqpoint{6.322784in}{1.403149in}}%
\pgfusepath{}%
\end{pgfscope}%
\begin{pgfscope}%
\pgfpathrectangle{\pgfqpoint{0.549740in}{0.463273in}}{\pgfqpoint{9.320225in}{4.495057in}}%
\pgfusepath{clip}%
\pgfsetbuttcap%
\pgfsetroundjoin%
\definecolor{currentfill}{rgb}{0.547810,0.736432,0.947518}%
\pgfsetfillcolor{currentfill}%
\pgfsetlinewidth{0.000000pt}%
\definecolor{currentstroke}{rgb}{0.000000,0.000000,0.000000}%
\pgfsetstrokecolor{currentstroke}%
\pgfsetdash{}{0pt}%
\pgfpathmoveto{\pgfqpoint{6.509011in}{1.403149in}}%
\pgfpathlineto{\pgfqpoint{6.695237in}{1.403149in}}%
\pgfpathlineto{\pgfqpoint{6.695237in}{1.484877in}}%
\pgfpathlineto{\pgfqpoint{6.509011in}{1.484877in}}%
\pgfpathlineto{\pgfqpoint{6.509011in}{1.403149in}}%
\pgfusepath{fill}%
\end{pgfscope}%
\begin{pgfscope}%
\pgfpathrectangle{\pgfqpoint{0.549740in}{0.463273in}}{\pgfqpoint{9.320225in}{4.495057in}}%
\pgfusepath{clip}%
\pgfsetbuttcap%
\pgfsetroundjoin%
\definecolor{currentfill}{rgb}{0.547810,0.736432,0.947518}%
\pgfsetfillcolor{currentfill}%
\pgfsetlinewidth{0.000000pt}%
\definecolor{currentstroke}{rgb}{0.000000,0.000000,0.000000}%
\pgfsetstrokecolor{currentstroke}%
\pgfsetdash{}{0pt}%
\pgfpathmoveto{\pgfqpoint{6.695237in}{1.403149in}}%
\pgfpathlineto{\pgfqpoint{6.881464in}{1.403149in}}%
\pgfpathlineto{\pgfqpoint{6.881464in}{1.484877in}}%
\pgfpathlineto{\pgfqpoint{6.695237in}{1.484877in}}%
\pgfpathlineto{\pgfqpoint{6.695237in}{1.403149in}}%
\pgfusepath{fill}%
\end{pgfscope}%
\begin{pgfscope}%
\pgfpathrectangle{\pgfqpoint{0.549740in}{0.463273in}}{\pgfqpoint{9.320225in}{4.495057in}}%
\pgfusepath{clip}%
\pgfsetbuttcap%
\pgfsetroundjoin%
\pgfsetlinewidth{0.000000pt}%
\definecolor{currentstroke}{rgb}{0.000000,0.000000,0.000000}%
\pgfsetstrokecolor{currentstroke}%
\pgfsetdash{}{0pt}%
\pgfpathmoveto{\pgfqpoint{6.881464in}{1.403149in}}%
\pgfpathlineto{\pgfqpoint{7.067690in}{1.403149in}}%
\pgfpathlineto{\pgfqpoint{7.067690in}{1.484877in}}%
\pgfpathlineto{\pgfqpoint{6.881464in}{1.484877in}}%
\pgfpathlineto{\pgfqpoint{6.881464in}{1.403149in}}%
\pgfusepath{}%
\end{pgfscope}%
\begin{pgfscope}%
\pgfpathrectangle{\pgfqpoint{0.549740in}{0.463273in}}{\pgfqpoint{9.320225in}{4.495057in}}%
\pgfusepath{clip}%
\pgfsetbuttcap%
\pgfsetroundjoin%
\pgfsetlinewidth{0.000000pt}%
\definecolor{currentstroke}{rgb}{0.000000,0.000000,0.000000}%
\pgfsetstrokecolor{currentstroke}%
\pgfsetdash{}{0pt}%
\pgfpathmoveto{\pgfqpoint{7.067690in}{1.403149in}}%
\pgfpathlineto{\pgfqpoint{7.253917in}{1.403149in}}%
\pgfpathlineto{\pgfqpoint{7.253917in}{1.484877in}}%
\pgfpathlineto{\pgfqpoint{7.067690in}{1.484877in}}%
\pgfpathlineto{\pgfqpoint{7.067690in}{1.403149in}}%
\pgfusepath{}%
\end{pgfscope}%
\begin{pgfscope}%
\pgfpathrectangle{\pgfqpoint{0.549740in}{0.463273in}}{\pgfqpoint{9.320225in}{4.495057in}}%
\pgfusepath{clip}%
\pgfsetbuttcap%
\pgfsetroundjoin%
\pgfsetlinewidth{0.000000pt}%
\definecolor{currentstroke}{rgb}{0.000000,0.000000,0.000000}%
\pgfsetstrokecolor{currentstroke}%
\pgfsetdash{}{0pt}%
\pgfpathmoveto{\pgfqpoint{7.253917in}{1.403149in}}%
\pgfpathlineto{\pgfqpoint{7.440143in}{1.403149in}}%
\pgfpathlineto{\pgfqpoint{7.440143in}{1.484877in}}%
\pgfpathlineto{\pgfqpoint{7.253917in}{1.484877in}}%
\pgfpathlineto{\pgfqpoint{7.253917in}{1.403149in}}%
\pgfusepath{}%
\end{pgfscope}%
\begin{pgfscope}%
\pgfpathrectangle{\pgfqpoint{0.549740in}{0.463273in}}{\pgfqpoint{9.320225in}{4.495057in}}%
\pgfusepath{clip}%
\pgfsetbuttcap%
\pgfsetroundjoin%
\pgfsetlinewidth{0.000000pt}%
\definecolor{currentstroke}{rgb}{0.000000,0.000000,0.000000}%
\pgfsetstrokecolor{currentstroke}%
\pgfsetdash{}{0pt}%
\pgfpathmoveto{\pgfqpoint{7.440143in}{1.403149in}}%
\pgfpathlineto{\pgfqpoint{7.626370in}{1.403149in}}%
\pgfpathlineto{\pgfqpoint{7.626370in}{1.484877in}}%
\pgfpathlineto{\pgfqpoint{7.440143in}{1.484877in}}%
\pgfpathlineto{\pgfqpoint{7.440143in}{1.403149in}}%
\pgfusepath{}%
\end{pgfscope}%
\begin{pgfscope}%
\pgfpathrectangle{\pgfqpoint{0.549740in}{0.463273in}}{\pgfqpoint{9.320225in}{4.495057in}}%
\pgfusepath{clip}%
\pgfsetbuttcap%
\pgfsetroundjoin%
\pgfsetlinewidth{0.000000pt}%
\definecolor{currentstroke}{rgb}{0.000000,0.000000,0.000000}%
\pgfsetstrokecolor{currentstroke}%
\pgfsetdash{}{0pt}%
\pgfpathmoveto{\pgfqpoint{7.626370in}{1.403149in}}%
\pgfpathlineto{\pgfqpoint{7.812596in}{1.403149in}}%
\pgfpathlineto{\pgfqpoint{7.812596in}{1.484877in}}%
\pgfpathlineto{\pgfqpoint{7.626370in}{1.484877in}}%
\pgfpathlineto{\pgfqpoint{7.626370in}{1.403149in}}%
\pgfusepath{}%
\end{pgfscope}%
\begin{pgfscope}%
\pgfpathrectangle{\pgfqpoint{0.549740in}{0.463273in}}{\pgfqpoint{9.320225in}{4.495057in}}%
\pgfusepath{clip}%
\pgfsetbuttcap%
\pgfsetroundjoin%
\definecolor{currentfill}{rgb}{0.472869,0.711325,0.955316}%
\pgfsetfillcolor{currentfill}%
\pgfsetlinewidth{0.000000pt}%
\definecolor{currentstroke}{rgb}{0.000000,0.000000,0.000000}%
\pgfsetstrokecolor{currentstroke}%
\pgfsetdash{}{0pt}%
\pgfpathmoveto{\pgfqpoint{7.812596in}{1.403149in}}%
\pgfpathlineto{\pgfqpoint{7.998823in}{1.403149in}}%
\pgfpathlineto{\pgfqpoint{7.998823in}{1.484877in}}%
\pgfpathlineto{\pgfqpoint{7.812596in}{1.484877in}}%
\pgfpathlineto{\pgfqpoint{7.812596in}{1.403149in}}%
\pgfusepath{fill}%
\end{pgfscope}%
\begin{pgfscope}%
\pgfpathrectangle{\pgfqpoint{0.549740in}{0.463273in}}{\pgfqpoint{9.320225in}{4.495057in}}%
\pgfusepath{clip}%
\pgfsetbuttcap%
\pgfsetroundjoin%
\pgfsetlinewidth{0.000000pt}%
\definecolor{currentstroke}{rgb}{0.000000,0.000000,0.000000}%
\pgfsetstrokecolor{currentstroke}%
\pgfsetdash{}{0pt}%
\pgfpathmoveto{\pgfqpoint{7.998823in}{1.403149in}}%
\pgfpathlineto{\pgfqpoint{8.185049in}{1.403149in}}%
\pgfpathlineto{\pgfqpoint{8.185049in}{1.484877in}}%
\pgfpathlineto{\pgfqpoint{7.998823in}{1.484877in}}%
\pgfpathlineto{\pgfqpoint{7.998823in}{1.403149in}}%
\pgfusepath{}%
\end{pgfscope}%
\begin{pgfscope}%
\pgfpathrectangle{\pgfqpoint{0.549740in}{0.463273in}}{\pgfqpoint{9.320225in}{4.495057in}}%
\pgfusepath{clip}%
\pgfsetbuttcap%
\pgfsetroundjoin%
\pgfsetlinewidth{0.000000pt}%
\definecolor{currentstroke}{rgb}{0.000000,0.000000,0.000000}%
\pgfsetstrokecolor{currentstroke}%
\pgfsetdash{}{0pt}%
\pgfpathmoveto{\pgfqpoint{8.185049in}{1.403149in}}%
\pgfpathlineto{\pgfqpoint{8.371276in}{1.403149in}}%
\pgfpathlineto{\pgfqpoint{8.371276in}{1.484877in}}%
\pgfpathlineto{\pgfqpoint{8.185049in}{1.484877in}}%
\pgfpathlineto{\pgfqpoint{8.185049in}{1.403149in}}%
\pgfusepath{}%
\end{pgfscope}%
\begin{pgfscope}%
\pgfpathrectangle{\pgfqpoint{0.549740in}{0.463273in}}{\pgfqpoint{9.320225in}{4.495057in}}%
\pgfusepath{clip}%
\pgfsetbuttcap%
\pgfsetroundjoin%
\pgfsetlinewidth{0.000000pt}%
\definecolor{currentstroke}{rgb}{0.000000,0.000000,0.000000}%
\pgfsetstrokecolor{currentstroke}%
\pgfsetdash{}{0pt}%
\pgfpathmoveto{\pgfqpoint{8.371276in}{1.403149in}}%
\pgfpathlineto{\pgfqpoint{8.557503in}{1.403149in}}%
\pgfpathlineto{\pgfqpoint{8.557503in}{1.484877in}}%
\pgfpathlineto{\pgfqpoint{8.371276in}{1.484877in}}%
\pgfpathlineto{\pgfqpoint{8.371276in}{1.403149in}}%
\pgfusepath{}%
\end{pgfscope}%
\begin{pgfscope}%
\pgfpathrectangle{\pgfqpoint{0.549740in}{0.463273in}}{\pgfqpoint{9.320225in}{4.495057in}}%
\pgfusepath{clip}%
\pgfsetbuttcap%
\pgfsetroundjoin%
\pgfsetlinewidth{0.000000pt}%
\definecolor{currentstroke}{rgb}{0.000000,0.000000,0.000000}%
\pgfsetstrokecolor{currentstroke}%
\pgfsetdash{}{0pt}%
\pgfpathmoveto{\pgfqpoint{8.557503in}{1.403149in}}%
\pgfpathlineto{\pgfqpoint{8.743729in}{1.403149in}}%
\pgfpathlineto{\pgfqpoint{8.743729in}{1.484877in}}%
\pgfpathlineto{\pgfqpoint{8.557503in}{1.484877in}}%
\pgfpathlineto{\pgfqpoint{8.557503in}{1.403149in}}%
\pgfusepath{}%
\end{pgfscope}%
\begin{pgfscope}%
\pgfpathrectangle{\pgfqpoint{0.549740in}{0.463273in}}{\pgfqpoint{9.320225in}{4.495057in}}%
\pgfusepath{clip}%
\pgfsetbuttcap%
\pgfsetroundjoin%
\definecolor{currentfill}{rgb}{0.273225,0.662144,0.968515}%
\pgfsetfillcolor{currentfill}%
\pgfsetlinewidth{0.000000pt}%
\definecolor{currentstroke}{rgb}{0.000000,0.000000,0.000000}%
\pgfsetstrokecolor{currentstroke}%
\pgfsetdash{}{0pt}%
\pgfpathmoveto{\pgfqpoint{8.743729in}{1.403149in}}%
\pgfpathlineto{\pgfqpoint{8.929956in}{1.403149in}}%
\pgfpathlineto{\pgfqpoint{8.929956in}{1.484877in}}%
\pgfpathlineto{\pgfqpoint{8.743729in}{1.484877in}}%
\pgfpathlineto{\pgfqpoint{8.743729in}{1.403149in}}%
\pgfusepath{fill}%
\end{pgfscope}%
\begin{pgfscope}%
\pgfpathrectangle{\pgfqpoint{0.549740in}{0.463273in}}{\pgfqpoint{9.320225in}{4.495057in}}%
\pgfusepath{clip}%
\pgfsetbuttcap%
\pgfsetroundjoin%
\definecolor{currentfill}{rgb}{0.614330,0.761948,0.940009}%
\pgfsetfillcolor{currentfill}%
\pgfsetlinewidth{0.000000pt}%
\definecolor{currentstroke}{rgb}{0.000000,0.000000,0.000000}%
\pgfsetstrokecolor{currentstroke}%
\pgfsetdash{}{0pt}%
\pgfpathmoveto{\pgfqpoint{8.929956in}{1.403149in}}%
\pgfpathlineto{\pgfqpoint{9.116182in}{1.403149in}}%
\pgfpathlineto{\pgfqpoint{9.116182in}{1.484877in}}%
\pgfpathlineto{\pgfqpoint{8.929956in}{1.484877in}}%
\pgfpathlineto{\pgfqpoint{8.929956in}{1.403149in}}%
\pgfusepath{fill}%
\end{pgfscope}%
\begin{pgfscope}%
\pgfpathrectangle{\pgfqpoint{0.549740in}{0.463273in}}{\pgfqpoint{9.320225in}{4.495057in}}%
\pgfusepath{clip}%
\pgfsetbuttcap%
\pgfsetroundjoin%
\pgfsetlinewidth{0.000000pt}%
\definecolor{currentstroke}{rgb}{0.000000,0.000000,0.000000}%
\pgfsetstrokecolor{currentstroke}%
\pgfsetdash{}{0pt}%
\pgfpathmoveto{\pgfqpoint{9.116182in}{1.403149in}}%
\pgfpathlineto{\pgfqpoint{9.302409in}{1.403149in}}%
\pgfpathlineto{\pgfqpoint{9.302409in}{1.484877in}}%
\pgfpathlineto{\pgfqpoint{9.116182in}{1.484877in}}%
\pgfpathlineto{\pgfqpoint{9.116182in}{1.403149in}}%
\pgfusepath{}%
\end{pgfscope}%
\begin{pgfscope}%
\pgfpathrectangle{\pgfqpoint{0.549740in}{0.463273in}}{\pgfqpoint{9.320225in}{4.495057in}}%
\pgfusepath{clip}%
\pgfsetbuttcap%
\pgfsetroundjoin%
\pgfsetlinewidth{0.000000pt}%
\definecolor{currentstroke}{rgb}{0.000000,0.000000,0.000000}%
\pgfsetstrokecolor{currentstroke}%
\pgfsetdash{}{0pt}%
\pgfpathmoveto{\pgfqpoint{9.302409in}{1.403149in}}%
\pgfpathlineto{\pgfqpoint{9.488635in}{1.403149in}}%
\pgfpathlineto{\pgfqpoint{9.488635in}{1.484877in}}%
\pgfpathlineto{\pgfqpoint{9.302409in}{1.484877in}}%
\pgfpathlineto{\pgfqpoint{9.302409in}{1.403149in}}%
\pgfusepath{}%
\end{pgfscope}%
\begin{pgfscope}%
\pgfpathrectangle{\pgfqpoint{0.549740in}{0.463273in}}{\pgfqpoint{9.320225in}{4.495057in}}%
\pgfusepath{clip}%
\pgfsetbuttcap%
\pgfsetroundjoin%
\pgfsetlinewidth{0.000000pt}%
\definecolor{currentstroke}{rgb}{0.000000,0.000000,0.000000}%
\pgfsetstrokecolor{currentstroke}%
\pgfsetdash{}{0pt}%
\pgfpathmoveto{\pgfqpoint{9.488635in}{1.403149in}}%
\pgfpathlineto{\pgfqpoint{9.674862in}{1.403149in}}%
\pgfpathlineto{\pgfqpoint{9.674862in}{1.484877in}}%
\pgfpathlineto{\pgfqpoint{9.488635in}{1.484877in}}%
\pgfpathlineto{\pgfqpoint{9.488635in}{1.403149in}}%
\pgfusepath{}%
\end{pgfscope}%
\begin{pgfscope}%
\pgfpathrectangle{\pgfqpoint{0.549740in}{0.463273in}}{\pgfqpoint{9.320225in}{4.495057in}}%
\pgfusepath{clip}%
\pgfsetbuttcap%
\pgfsetroundjoin%
\pgfsetlinewidth{0.000000pt}%
\definecolor{currentstroke}{rgb}{0.000000,0.000000,0.000000}%
\pgfsetstrokecolor{currentstroke}%
\pgfsetdash{}{0pt}%
\pgfpathmoveto{\pgfqpoint{9.674862in}{1.403149in}}%
\pgfpathlineto{\pgfqpoint{9.861088in}{1.403149in}}%
\pgfpathlineto{\pgfqpoint{9.861088in}{1.484877in}}%
\pgfpathlineto{\pgfqpoint{9.674862in}{1.484877in}}%
\pgfpathlineto{\pgfqpoint{9.674862in}{1.403149in}}%
\pgfusepath{}%
\end{pgfscope}%
\begin{pgfscope}%
\pgfpathrectangle{\pgfqpoint{0.549740in}{0.463273in}}{\pgfqpoint{9.320225in}{4.495057in}}%
\pgfusepath{clip}%
\pgfsetbuttcap%
\pgfsetroundjoin%
\pgfsetlinewidth{0.000000pt}%
\definecolor{currentstroke}{rgb}{0.000000,0.000000,0.000000}%
\pgfsetstrokecolor{currentstroke}%
\pgfsetdash{}{0pt}%
\pgfpathmoveto{\pgfqpoint{0.549761in}{1.484877in}}%
\pgfpathlineto{\pgfqpoint{0.735988in}{1.484877in}}%
\pgfpathlineto{\pgfqpoint{0.735988in}{1.566605in}}%
\pgfpathlineto{\pgfqpoint{0.549761in}{1.566605in}}%
\pgfpathlineto{\pgfqpoint{0.549761in}{1.484877in}}%
\pgfusepath{}%
\end{pgfscope}%
\begin{pgfscope}%
\pgfpathrectangle{\pgfqpoint{0.549740in}{0.463273in}}{\pgfqpoint{9.320225in}{4.495057in}}%
\pgfusepath{clip}%
\pgfsetbuttcap%
\pgfsetroundjoin%
\pgfsetlinewidth{0.000000pt}%
\definecolor{currentstroke}{rgb}{0.000000,0.000000,0.000000}%
\pgfsetstrokecolor{currentstroke}%
\pgfsetdash{}{0pt}%
\pgfpathmoveto{\pgfqpoint{0.735988in}{1.484877in}}%
\pgfpathlineto{\pgfqpoint{0.922214in}{1.484877in}}%
\pgfpathlineto{\pgfqpoint{0.922214in}{1.566605in}}%
\pgfpathlineto{\pgfqpoint{0.735988in}{1.566605in}}%
\pgfpathlineto{\pgfqpoint{0.735988in}{1.484877in}}%
\pgfusepath{}%
\end{pgfscope}%
\begin{pgfscope}%
\pgfpathrectangle{\pgfqpoint{0.549740in}{0.463273in}}{\pgfqpoint{9.320225in}{4.495057in}}%
\pgfusepath{clip}%
\pgfsetbuttcap%
\pgfsetroundjoin%
\pgfsetlinewidth{0.000000pt}%
\definecolor{currentstroke}{rgb}{0.000000,0.000000,0.000000}%
\pgfsetstrokecolor{currentstroke}%
\pgfsetdash{}{0pt}%
\pgfpathmoveto{\pgfqpoint{0.922214in}{1.484877in}}%
\pgfpathlineto{\pgfqpoint{1.108441in}{1.484877in}}%
\pgfpathlineto{\pgfqpoint{1.108441in}{1.566605in}}%
\pgfpathlineto{\pgfqpoint{0.922214in}{1.566605in}}%
\pgfpathlineto{\pgfqpoint{0.922214in}{1.484877in}}%
\pgfusepath{}%
\end{pgfscope}%
\begin{pgfscope}%
\pgfpathrectangle{\pgfqpoint{0.549740in}{0.463273in}}{\pgfqpoint{9.320225in}{4.495057in}}%
\pgfusepath{clip}%
\pgfsetbuttcap%
\pgfsetroundjoin%
\pgfsetlinewidth{0.000000pt}%
\definecolor{currentstroke}{rgb}{0.000000,0.000000,0.000000}%
\pgfsetstrokecolor{currentstroke}%
\pgfsetdash{}{0pt}%
\pgfpathmoveto{\pgfqpoint{1.108441in}{1.484877in}}%
\pgfpathlineto{\pgfqpoint{1.294667in}{1.484877in}}%
\pgfpathlineto{\pgfqpoint{1.294667in}{1.566605in}}%
\pgfpathlineto{\pgfqpoint{1.108441in}{1.566605in}}%
\pgfpathlineto{\pgfqpoint{1.108441in}{1.484877in}}%
\pgfusepath{}%
\end{pgfscope}%
\begin{pgfscope}%
\pgfpathrectangle{\pgfqpoint{0.549740in}{0.463273in}}{\pgfqpoint{9.320225in}{4.495057in}}%
\pgfusepath{clip}%
\pgfsetbuttcap%
\pgfsetroundjoin%
\pgfsetlinewidth{0.000000pt}%
\definecolor{currentstroke}{rgb}{0.000000,0.000000,0.000000}%
\pgfsetstrokecolor{currentstroke}%
\pgfsetdash{}{0pt}%
\pgfpathmoveto{\pgfqpoint{1.294667in}{1.484877in}}%
\pgfpathlineto{\pgfqpoint{1.480894in}{1.484877in}}%
\pgfpathlineto{\pgfqpoint{1.480894in}{1.566605in}}%
\pgfpathlineto{\pgfqpoint{1.294667in}{1.566605in}}%
\pgfpathlineto{\pgfqpoint{1.294667in}{1.484877in}}%
\pgfusepath{}%
\end{pgfscope}%
\begin{pgfscope}%
\pgfpathrectangle{\pgfqpoint{0.549740in}{0.463273in}}{\pgfqpoint{9.320225in}{4.495057in}}%
\pgfusepath{clip}%
\pgfsetbuttcap%
\pgfsetroundjoin%
\pgfsetlinewidth{0.000000pt}%
\definecolor{currentstroke}{rgb}{0.000000,0.000000,0.000000}%
\pgfsetstrokecolor{currentstroke}%
\pgfsetdash{}{0pt}%
\pgfpathmoveto{\pgfqpoint{1.480894in}{1.484877in}}%
\pgfpathlineto{\pgfqpoint{1.667120in}{1.484877in}}%
\pgfpathlineto{\pgfqpoint{1.667120in}{1.566605in}}%
\pgfpathlineto{\pgfqpoint{1.480894in}{1.566605in}}%
\pgfpathlineto{\pgfqpoint{1.480894in}{1.484877in}}%
\pgfusepath{}%
\end{pgfscope}%
\begin{pgfscope}%
\pgfpathrectangle{\pgfqpoint{0.549740in}{0.463273in}}{\pgfqpoint{9.320225in}{4.495057in}}%
\pgfusepath{clip}%
\pgfsetbuttcap%
\pgfsetroundjoin%
\pgfsetlinewidth{0.000000pt}%
\definecolor{currentstroke}{rgb}{0.000000,0.000000,0.000000}%
\pgfsetstrokecolor{currentstroke}%
\pgfsetdash{}{0pt}%
\pgfpathmoveto{\pgfqpoint{1.667120in}{1.484877in}}%
\pgfpathlineto{\pgfqpoint{1.853347in}{1.484877in}}%
\pgfpathlineto{\pgfqpoint{1.853347in}{1.566605in}}%
\pgfpathlineto{\pgfqpoint{1.667120in}{1.566605in}}%
\pgfpathlineto{\pgfqpoint{1.667120in}{1.484877in}}%
\pgfusepath{}%
\end{pgfscope}%
\begin{pgfscope}%
\pgfpathrectangle{\pgfqpoint{0.549740in}{0.463273in}}{\pgfqpoint{9.320225in}{4.495057in}}%
\pgfusepath{clip}%
\pgfsetbuttcap%
\pgfsetroundjoin%
\pgfsetlinewidth{0.000000pt}%
\definecolor{currentstroke}{rgb}{0.000000,0.000000,0.000000}%
\pgfsetstrokecolor{currentstroke}%
\pgfsetdash{}{0pt}%
\pgfpathmoveto{\pgfqpoint{1.853347in}{1.484877in}}%
\pgfpathlineto{\pgfqpoint{2.039573in}{1.484877in}}%
\pgfpathlineto{\pgfqpoint{2.039573in}{1.566605in}}%
\pgfpathlineto{\pgfqpoint{1.853347in}{1.566605in}}%
\pgfpathlineto{\pgfqpoint{1.853347in}{1.484877in}}%
\pgfusepath{}%
\end{pgfscope}%
\begin{pgfscope}%
\pgfpathrectangle{\pgfqpoint{0.549740in}{0.463273in}}{\pgfqpoint{9.320225in}{4.495057in}}%
\pgfusepath{clip}%
\pgfsetbuttcap%
\pgfsetroundjoin%
\pgfsetlinewidth{0.000000pt}%
\definecolor{currentstroke}{rgb}{0.000000,0.000000,0.000000}%
\pgfsetstrokecolor{currentstroke}%
\pgfsetdash{}{0pt}%
\pgfpathmoveto{\pgfqpoint{2.039573in}{1.484877in}}%
\pgfpathlineto{\pgfqpoint{2.225800in}{1.484877in}}%
\pgfpathlineto{\pgfqpoint{2.225800in}{1.566605in}}%
\pgfpathlineto{\pgfqpoint{2.039573in}{1.566605in}}%
\pgfpathlineto{\pgfqpoint{2.039573in}{1.484877in}}%
\pgfusepath{}%
\end{pgfscope}%
\begin{pgfscope}%
\pgfpathrectangle{\pgfqpoint{0.549740in}{0.463273in}}{\pgfqpoint{9.320225in}{4.495057in}}%
\pgfusepath{clip}%
\pgfsetbuttcap%
\pgfsetroundjoin%
\pgfsetlinewidth{0.000000pt}%
\definecolor{currentstroke}{rgb}{0.000000,0.000000,0.000000}%
\pgfsetstrokecolor{currentstroke}%
\pgfsetdash{}{0pt}%
\pgfpathmoveto{\pgfqpoint{2.225800in}{1.484877in}}%
\pgfpathlineto{\pgfqpoint{2.412027in}{1.484877in}}%
\pgfpathlineto{\pgfqpoint{2.412027in}{1.566605in}}%
\pgfpathlineto{\pgfqpoint{2.225800in}{1.566605in}}%
\pgfpathlineto{\pgfqpoint{2.225800in}{1.484877in}}%
\pgfusepath{}%
\end{pgfscope}%
\begin{pgfscope}%
\pgfpathrectangle{\pgfqpoint{0.549740in}{0.463273in}}{\pgfqpoint{9.320225in}{4.495057in}}%
\pgfusepath{clip}%
\pgfsetbuttcap%
\pgfsetroundjoin%
\pgfsetlinewidth{0.000000pt}%
\definecolor{currentstroke}{rgb}{0.000000,0.000000,0.000000}%
\pgfsetstrokecolor{currentstroke}%
\pgfsetdash{}{0pt}%
\pgfpathmoveto{\pgfqpoint{2.412027in}{1.484877in}}%
\pgfpathlineto{\pgfqpoint{2.598253in}{1.484877in}}%
\pgfpathlineto{\pgfqpoint{2.598253in}{1.566605in}}%
\pgfpathlineto{\pgfqpoint{2.412027in}{1.566605in}}%
\pgfpathlineto{\pgfqpoint{2.412027in}{1.484877in}}%
\pgfusepath{}%
\end{pgfscope}%
\begin{pgfscope}%
\pgfpathrectangle{\pgfqpoint{0.549740in}{0.463273in}}{\pgfqpoint{9.320225in}{4.495057in}}%
\pgfusepath{clip}%
\pgfsetbuttcap%
\pgfsetroundjoin%
\pgfsetlinewidth{0.000000pt}%
\definecolor{currentstroke}{rgb}{0.000000,0.000000,0.000000}%
\pgfsetstrokecolor{currentstroke}%
\pgfsetdash{}{0pt}%
\pgfpathmoveto{\pgfqpoint{2.598253in}{1.484877in}}%
\pgfpathlineto{\pgfqpoint{2.784480in}{1.484877in}}%
\pgfpathlineto{\pgfqpoint{2.784480in}{1.566605in}}%
\pgfpathlineto{\pgfqpoint{2.598253in}{1.566605in}}%
\pgfpathlineto{\pgfqpoint{2.598253in}{1.484877in}}%
\pgfusepath{}%
\end{pgfscope}%
\begin{pgfscope}%
\pgfpathrectangle{\pgfqpoint{0.549740in}{0.463273in}}{\pgfqpoint{9.320225in}{4.495057in}}%
\pgfusepath{clip}%
\pgfsetbuttcap%
\pgfsetroundjoin%
\pgfsetlinewidth{0.000000pt}%
\definecolor{currentstroke}{rgb}{0.000000,0.000000,0.000000}%
\pgfsetstrokecolor{currentstroke}%
\pgfsetdash{}{0pt}%
\pgfpathmoveto{\pgfqpoint{2.784480in}{1.484877in}}%
\pgfpathlineto{\pgfqpoint{2.970706in}{1.484877in}}%
\pgfpathlineto{\pgfqpoint{2.970706in}{1.566605in}}%
\pgfpathlineto{\pgfqpoint{2.784480in}{1.566605in}}%
\pgfpathlineto{\pgfqpoint{2.784480in}{1.484877in}}%
\pgfusepath{}%
\end{pgfscope}%
\begin{pgfscope}%
\pgfpathrectangle{\pgfqpoint{0.549740in}{0.463273in}}{\pgfqpoint{9.320225in}{4.495057in}}%
\pgfusepath{clip}%
\pgfsetbuttcap%
\pgfsetroundjoin%
\pgfsetlinewidth{0.000000pt}%
\definecolor{currentstroke}{rgb}{0.000000,0.000000,0.000000}%
\pgfsetstrokecolor{currentstroke}%
\pgfsetdash{}{0pt}%
\pgfpathmoveto{\pgfqpoint{2.970706in}{1.484877in}}%
\pgfpathlineto{\pgfqpoint{3.156933in}{1.484877in}}%
\pgfpathlineto{\pgfqpoint{3.156933in}{1.566605in}}%
\pgfpathlineto{\pgfqpoint{2.970706in}{1.566605in}}%
\pgfpathlineto{\pgfqpoint{2.970706in}{1.484877in}}%
\pgfusepath{}%
\end{pgfscope}%
\begin{pgfscope}%
\pgfpathrectangle{\pgfqpoint{0.549740in}{0.463273in}}{\pgfqpoint{9.320225in}{4.495057in}}%
\pgfusepath{clip}%
\pgfsetbuttcap%
\pgfsetroundjoin%
\pgfsetlinewidth{0.000000pt}%
\definecolor{currentstroke}{rgb}{0.000000,0.000000,0.000000}%
\pgfsetstrokecolor{currentstroke}%
\pgfsetdash{}{0pt}%
\pgfpathmoveto{\pgfqpoint{3.156933in}{1.484877in}}%
\pgfpathlineto{\pgfqpoint{3.343159in}{1.484877in}}%
\pgfpathlineto{\pgfqpoint{3.343159in}{1.566605in}}%
\pgfpathlineto{\pgfqpoint{3.156933in}{1.566605in}}%
\pgfpathlineto{\pgfqpoint{3.156933in}{1.484877in}}%
\pgfusepath{}%
\end{pgfscope}%
\begin{pgfscope}%
\pgfpathrectangle{\pgfqpoint{0.549740in}{0.463273in}}{\pgfqpoint{9.320225in}{4.495057in}}%
\pgfusepath{clip}%
\pgfsetbuttcap%
\pgfsetroundjoin%
\pgfsetlinewidth{0.000000pt}%
\definecolor{currentstroke}{rgb}{0.000000,0.000000,0.000000}%
\pgfsetstrokecolor{currentstroke}%
\pgfsetdash{}{0pt}%
\pgfpathmoveto{\pgfqpoint{3.343159in}{1.484877in}}%
\pgfpathlineto{\pgfqpoint{3.529386in}{1.484877in}}%
\pgfpathlineto{\pgfqpoint{3.529386in}{1.566605in}}%
\pgfpathlineto{\pgfqpoint{3.343159in}{1.566605in}}%
\pgfpathlineto{\pgfqpoint{3.343159in}{1.484877in}}%
\pgfusepath{}%
\end{pgfscope}%
\begin{pgfscope}%
\pgfpathrectangle{\pgfqpoint{0.549740in}{0.463273in}}{\pgfqpoint{9.320225in}{4.495057in}}%
\pgfusepath{clip}%
\pgfsetbuttcap%
\pgfsetroundjoin%
\pgfsetlinewidth{0.000000pt}%
\definecolor{currentstroke}{rgb}{0.000000,0.000000,0.000000}%
\pgfsetstrokecolor{currentstroke}%
\pgfsetdash{}{0pt}%
\pgfpathmoveto{\pgfqpoint{3.529386in}{1.484877in}}%
\pgfpathlineto{\pgfqpoint{3.715612in}{1.484877in}}%
\pgfpathlineto{\pgfqpoint{3.715612in}{1.566605in}}%
\pgfpathlineto{\pgfqpoint{3.529386in}{1.566605in}}%
\pgfpathlineto{\pgfqpoint{3.529386in}{1.484877in}}%
\pgfusepath{}%
\end{pgfscope}%
\begin{pgfscope}%
\pgfpathrectangle{\pgfqpoint{0.549740in}{0.463273in}}{\pgfqpoint{9.320225in}{4.495057in}}%
\pgfusepath{clip}%
\pgfsetbuttcap%
\pgfsetroundjoin%
\pgfsetlinewidth{0.000000pt}%
\definecolor{currentstroke}{rgb}{0.000000,0.000000,0.000000}%
\pgfsetstrokecolor{currentstroke}%
\pgfsetdash{}{0pt}%
\pgfpathmoveto{\pgfqpoint{3.715612in}{1.484877in}}%
\pgfpathlineto{\pgfqpoint{3.901839in}{1.484877in}}%
\pgfpathlineto{\pgfqpoint{3.901839in}{1.566605in}}%
\pgfpathlineto{\pgfqpoint{3.715612in}{1.566605in}}%
\pgfpathlineto{\pgfqpoint{3.715612in}{1.484877in}}%
\pgfusepath{}%
\end{pgfscope}%
\begin{pgfscope}%
\pgfpathrectangle{\pgfqpoint{0.549740in}{0.463273in}}{\pgfqpoint{9.320225in}{4.495057in}}%
\pgfusepath{clip}%
\pgfsetbuttcap%
\pgfsetroundjoin%
\pgfsetlinewidth{0.000000pt}%
\definecolor{currentstroke}{rgb}{0.000000,0.000000,0.000000}%
\pgfsetstrokecolor{currentstroke}%
\pgfsetdash{}{0pt}%
\pgfpathmoveto{\pgfqpoint{3.901839in}{1.484877in}}%
\pgfpathlineto{\pgfqpoint{4.088065in}{1.484877in}}%
\pgfpathlineto{\pgfqpoint{4.088065in}{1.566605in}}%
\pgfpathlineto{\pgfqpoint{3.901839in}{1.566605in}}%
\pgfpathlineto{\pgfqpoint{3.901839in}{1.484877in}}%
\pgfusepath{}%
\end{pgfscope}%
\begin{pgfscope}%
\pgfpathrectangle{\pgfqpoint{0.549740in}{0.463273in}}{\pgfqpoint{9.320225in}{4.495057in}}%
\pgfusepath{clip}%
\pgfsetbuttcap%
\pgfsetroundjoin%
\pgfsetlinewidth{0.000000pt}%
\definecolor{currentstroke}{rgb}{0.000000,0.000000,0.000000}%
\pgfsetstrokecolor{currentstroke}%
\pgfsetdash{}{0pt}%
\pgfpathmoveto{\pgfqpoint{4.088065in}{1.484877in}}%
\pgfpathlineto{\pgfqpoint{4.274292in}{1.484877in}}%
\pgfpathlineto{\pgfqpoint{4.274292in}{1.566605in}}%
\pgfpathlineto{\pgfqpoint{4.088065in}{1.566605in}}%
\pgfpathlineto{\pgfqpoint{4.088065in}{1.484877in}}%
\pgfusepath{}%
\end{pgfscope}%
\begin{pgfscope}%
\pgfpathrectangle{\pgfqpoint{0.549740in}{0.463273in}}{\pgfqpoint{9.320225in}{4.495057in}}%
\pgfusepath{clip}%
\pgfsetbuttcap%
\pgfsetroundjoin%
\pgfsetlinewidth{0.000000pt}%
\definecolor{currentstroke}{rgb}{0.000000,0.000000,0.000000}%
\pgfsetstrokecolor{currentstroke}%
\pgfsetdash{}{0pt}%
\pgfpathmoveto{\pgfqpoint{4.274292in}{1.484877in}}%
\pgfpathlineto{\pgfqpoint{4.460519in}{1.484877in}}%
\pgfpathlineto{\pgfqpoint{4.460519in}{1.566605in}}%
\pgfpathlineto{\pgfqpoint{4.274292in}{1.566605in}}%
\pgfpathlineto{\pgfqpoint{4.274292in}{1.484877in}}%
\pgfusepath{}%
\end{pgfscope}%
\begin{pgfscope}%
\pgfpathrectangle{\pgfqpoint{0.549740in}{0.463273in}}{\pgfqpoint{9.320225in}{4.495057in}}%
\pgfusepath{clip}%
\pgfsetbuttcap%
\pgfsetroundjoin%
\pgfsetlinewidth{0.000000pt}%
\definecolor{currentstroke}{rgb}{0.000000,0.000000,0.000000}%
\pgfsetstrokecolor{currentstroke}%
\pgfsetdash{}{0pt}%
\pgfpathmoveto{\pgfqpoint{4.460519in}{1.484877in}}%
\pgfpathlineto{\pgfqpoint{4.646745in}{1.484877in}}%
\pgfpathlineto{\pgfqpoint{4.646745in}{1.566605in}}%
\pgfpathlineto{\pgfqpoint{4.460519in}{1.566605in}}%
\pgfpathlineto{\pgfqpoint{4.460519in}{1.484877in}}%
\pgfusepath{}%
\end{pgfscope}%
\begin{pgfscope}%
\pgfpathrectangle{\pgfqpoint{0.549740in}{0.463273in}}{\pgfqpoint{9.320225in}{4.495057in}}%
\pgfusepath{clip}%
\pgfsetbuttcap%
\pgfsetroundjoin%
\pgfsetlinewidth{0.000000pt}%
\definecolor{currentstroke}{rgb}{0.000000,0.000000,0.000000}%
\pgfsetstrokecolor{currentstroke}%
\pgfsetdash{}{0pt}%
\pgfpathmoveto{\pgfqpoint{4.646745in}{1.484877in}}%
\pgfpathlineto{\pgfqpoint{4.832972in}{1.484877in}}%
\pgfpathlineto{\pgfqpoint{4.832972in}{1.566605in}}%
\pgfpathlineto{\pgfqpoint{4.646745in}{1.566605in}}%
\pgfpathlineto{\pgfqpoint{4.646745in}{1.484877in}}%
\pgfusepath{}%
\end{pgfscope}%
\begin{pgfscope}%
\pgfpathrectangle{\pgfqpoint{0.549740in}{0.463273in}}{\pgfqpoint{9.320225in}{4.495057in}}%
\pgfusepath{clip}%
\pgfsetbuttcap%
\pgfsetroundjoin%
\pgfsetlinewidth{0.000000pt}%
\definecolor{currentstroke}{rgb}{0.000000,0.000000,0.000000}%
\pgfsetstrokecolor{currentstroke}%
\pgfsetdash{}{0pt}%
\pgfpathmoveto{\pgfqpoint{4.832972in}{1.484877in}}%
\pgfpathlineto{\pgfqpoint{5.019198in}{1.484877in}}%
\pgfpathlineto{\pgfqpoint{5.019198in}{1.566605in}}%
\pgfpathlineto{\pgfqpoint{4.832972in}{1.566605in}}%
\pgfpathlineto{\pgfqpoint{4.832972in}{1.484877in}}%
\pgfusepath{}%
\end{pgfscope}%
\begin{pgfscope}%
\pgfpathrectangle{\pgfqpoint{0.549740in}{0.463273in}}{\pgfqpoint{9.320225in}{4.495057in}}%
\pgfusepath{clip}%
\pgfsetbuttcap%
\pgfsetroundjoin%
\pgfsetlinewidth{0.000000pt}%
\definecolor{currentstroke}{rgb}{0.000000,0.000000,0.000000}%
\pgfsetstrokecolor{currentstroke}%
\pgfsetdash{}{0pt}%
\pgfpathmoveto{\pgfqpoint{5.019198in}{1.484877in}}%
\pgfpathlineto{\pgfqpoint{5.205425in}{1.484877in}}%
\pgfpathlineto{\pgfqpoint{5.205425in}{1.566605in}}%
\pgfpathlineto{\pgfqpoint{5.019198in}{1.566605in}}%
\pgfpathlineto{\pgfqpoint{5.019198in}{1.484877in}}%
\pgfusepath{}%
\end{pgfscope}%
\begin{pgfscope}%
\pgfpathrectangle{\pgfqpoint{0.549740in}{0.463273in}}{\pgfqpoint{9.320225in}{4.495057in}}%
\pgfusepath{clip}%
\pgfsetbuttcap%
\pgfsetroundjoin%
\pgfsetlinewidth{0.000000pt}%
\definecolor{currentstroke}{rgb}{0.000000,0.000000,0.000000}%
\pgfsetstrokecolor{currentstroke}%
\pgfsetdash{}{0pt}%
\pgfpathmoveto{\pgfqpoint{5.205425in}{1.484877in}}%
\pgfpathlineto{\pgfqpoint{5.391651in}{1.484877in}}%
\pgfpathlineto{\pgfqpoint{5.391651in}{1.566605in}}%
\pgfpathlineto{\pgfqpoint{5.205425in}{1.566605in}}%
\pgfpathlineto{\pgfqpoint{5.205425in}{1.484877in}}%
\pgfusepath{}%
\end{pgfscope}%
\begin{pgfscope}%
\pgfpathrectangle{\pgfqpoint{0.549740in}{0.463273in}}{\pgfqpoint{9.320225in}{4.495057in}}%
\pgfusepath{clip}%
\pgfsetbuttcap%
\pgfsetroundjoin%
\pgfsetlinewidth{0.000000pt}%
\definecolor{currentstroke}{rgb}{0.000000,0.000000,0.000000}%
\pgfsetstrokecolor{currentstroke}%
\pgfsetdash{}{0pt}%
\pgfpathmoveto{\pgfqpoint{5.391651in}{1.484877in}}%
\pgfpathlineto{\pgfqpoint{5.577878in}{1.484877in}}%
\pgfpathlineto{\pgfqpoint{5.577878in}{1.566605in}}%
\pgfpathlineto{\pgfqpoint{5.391651in}{1.566605in}}%
\pgfpathlineto{\pgfqpoint{5.391651in}{1.484877in}}%
\pgfusepath{}%
\end{pgfscope}%
\begin{pgfscope}%
\pgfpathrectangle{\pgfqpoint{0.549740in}{0.463273in}}{\pgfqpoint{9.320225in}{4.495057in}}%
\pgfusepath{clip}%
\pgfsetbuttcap%
\pgfsetroundjoin%
\definecolor{currentfill}{rgb}{0.472869,0.711325,0.955316}%
\pgfsetfillcolor{currentfill}%
\pgfsetlinewidth{0.000000pt}%
\definecolor{currentstroke}{rgb}{0.000000,0.000000,0.000000}%
\pgfsetstrokecolor{currentstroke}%
\pgfsetdash{}{0pt}%
\pgfpathmoveto{\pgfqpoint{5.577878in}{1.484877in}}%
\pgfpathlineto{\pgfqpoint{5.764104in}{1.484877in}}%
\pgfpathlineto{\pgfqpoint{5.764104in}{1.566605in}}%
\pgfpathlineto{\pgfqpoint{5.577878in}{1.566605in}}%
\pgfpathlineto{\pgfqpoint{5.577878in}{1.484877in}}%
\pgfusepath{fill}%
\end{pgfscope}%
\begin{pgfscope}%
\pgfpathrectangle{\pgfqpoint{0.549740in}{0.463273in}}{\pgfqpoint{9.320225in}{4.495057in}}%
\pgfusepath{clip}%
\pgfsetbuttcap%
\pgfsetroundjoin%
\pgfsetlinewidth{0.000000pt}%
\definecolor{currentstroke}{rgb}{0.000000,0.000000,0.000000}%
\pgfsetstrokecolor{currentstroke}%
\pgfsetdash{}{0pt}%
\pgfpathmoveto{\pgfqpoint{5.764104in}{1.484877in}}%
\pgfpathlineto{\pgfqpoint{5.950331in}{1.484877in}}%
\pgfpathlineto{\pgfqpoint{5.950331in}{1.566605in}}%
\pgfpathlineto{\pgfqpoint{5.764104in}{1.566605in}}%
\pgfpathlineto{\pgfqpoint{5.764104in}{1.484877in}}%
\pgfusepath{}%
\end{pgfscope}%
\begin{pgfscope}%
\pgfpathrectangle{\pgfqpoint{0.549740in}{0.463273in}}{\pgfqpoint{9.320225in}{4.495057in}}%
\pgfusepath{clip}%
\pgfsetbuttcap%
\pgfsetroundjoin%
\pgfsetlinewidth{0.000000pt}%
\definecolor{currentstroke}{rgb}{0.000000,0.000000,0.000000}%
\pgfsetstrokecolor{currentstroke}%
\pgfsetdash{}{0pt}%
\pgfpathmoveto{\pgfqpoint{5.950331in}{1.484877in}}%
\pgfpathlineto{\pgfqpoint{6.136557in}{1.484877in}}%
\pgfpathlineto{\pgfqpoint{6.136557in}{1.566605in}}%
\pgfpathlineto{\pgfqpoint{5.950331in}{1.566605in}}%
\pgfpathlineto{\pgfqpoint{5.950331in}{1.484877in}}%
\pgfusepath{}%
\end{pgfscope}%
\begin{pgfscope}%
\pgfpathrectangle{\pgfqpoint{0.549740in}{0.463273in}}{\pgfqpoint{9.320225in}{4.495057in}}%
\pgfusepath{clip}%
\pgfsetbuttcap%
\pgfsetroundjoin%
\pgfsetlinewidth{0.000000pt}%
\definecolor{currentstroke}{rgb}{0.000000,0.000000,0.000000}%
\pgfsetstrokecolor{currentstroke}%
\pgfsetdash{}{0pt}%
\pgfpathmoveto{\pgfqpoint{6.136557in}{1.484877in}}%
\pgfpathlineto{\pgfqpoint{6.322784in}{1.484877in}}%
\pgfpathlineto{\pgfqpoint{6.322784in}{1.566605in}}%
\pgfpathlineto{\pgfqpoint{6.136557in}{1.566605in}}%
\pgfpathlineto{\pgfqpoint{6.136557in}{1.484877in}}%
\pgfusepath{}%
\end{pgfscope}%
\begin{pgfscope}%
\pgfpathrectangle{\pgfqpoint{0.549740in}{0.463273in}}{\pgfqpoint{9.320225in}{4.495057in}}%
\pgfusepath{clip}%
\pgfsetbuttcap%
\pgfsetroundjoin%
\pgfsetlinewidth{0.000000pt}%
\definecolor{currentstroke}{rgb}{0.000000,0.000000,0.000000}%
\pgfsetstrokecolor{currentstroke}%
\pgfsetdash{}{0pt}%
\pgfpathmoveto{\pgfqpoint{6.322784in}{1.484877in}}%
\pgfpathlineto{\pgfqpoint{6.509011in}{1.484877in}}%
\pgfpathlineto{\pgfqpoint{6.509011in}{1.566605in}}%
\pgfpathlineto{\pgfqpoint{6.322784in}{1.566605in}}%
\pgfpathlineto{\pgfqpoint{6.322784in}{1.484877in}}%
\pgfusepath{}%
\end{pgfscope}%
\begin{pgfscope}%
\pgfpathrectangle{\pgfqpoint{0.549740in}{0.463273in}}{\pgfqpoint{9.320225in}{4.495057in}}%
\pgfusepath{clip}%
\pgfsetbuttcap%
\pgfsetroundjoin%
\definecolor{currentfill}{rgb}{0.472869,0.711325,0.955316}%
\pgfsetfillcolor{currentfill}%
\pgfsetlinewidth{0.000000pt}%
\definecolor{currentstroke}{rgb}{0.000000,0.000000,0.000000}%
\pgfsetstrokecolor{currentstroke}%
\pgfsetdash{}{0pt}%
\pgfpathmoveto{\pgfqpoint{6.509011in}{1.484877in}}%
\pgfpathlineto{\pgfqpoint{6.695237in}{1.484877in}}%
\pgfpathlineto{\pgfqpoint{6.695237in}{1.566605in}}%
\pgfpathlineto{\pgfqpoint{6.509011in}{1.566605in}}%
\pgfpathlineto{\pgfqpoint{6.509011in}{1.484877in}}%
\pgfusepath{fill}%
\end{pgfscope}%
\begin{pgfscope}%
\pgfpathrectangle{\pgfqpoint{0.549740in}{0.463273in}}{\pgfqpoint{9.320225in}{4.495057in}}%
\pgfusepath{clip}%
\pgfsetbuttcap%
\pgfsetroundjoin%
\pgfsetlinewidth{0.000000pt}%
\definecolor{currentstroke}{rgb}{0.000000,0.000000,0.000000}%
\pgfsetstrokecolor{currentstroke}%
\pgfsetdash{}{0pt}%
\pgfpathmoveto{\pgfqpoint{6.695237in}{1.484877in}}%
\pgfpathlineto{\pgfqpoint{6.881464in}{1.484877in}}%
\pgfpathlineto{\pgfqpoint{6.881464in}{1.566605in}}%
\pgfpathlineto{\pgfqpoint{6.695237in}{1.566605in}}%
\pgfpathlineto{\pgfqpoint{6.695237in}{1.484877in}}%
\pgfusepath{}%
\end{pgfscope}%
\begin{pgfscope}%
\pgfpathrectangle{\pgfqpoint{0.549740in}{0.463273in}}{\pgfqpoint{9.320225in}{4.495057in}}%
\pgfusepath{clip}%
\pgfsetbuttcap%
\pgfsetroundjoin%
\pgfsetlinewidth{0.000000pt}%
\definecolor{currentstroke}{rgb}{0.000000,0.000000,0.000000}%
\pgfsetstrokecolor{currentstroke}%
\pgfsetdash{}{0pt}%
\pgfpathmoveto{\pgfqpoint{6.881464in}{1.484877in}}%
\pgfpathlineto{\pgfqpoint{7.067690in}{1.484877in}}%
\pgfpathlineto{\pgfqpoint{7.067690in}{1.566605in}}%
\pgfpathlineto{\pgfqpoint{6.881464in}{1.566605in}}%
\pgfpathlineto{\pgfqpoint{6.881464in}{1.484877in}}%
\pgfusepath{}%
\end{pgfscope}%
\begin{pgfscope}%
\pgfpathrectangle{\pgfqpoint{0.549740in}{0.463273in}}{\pgfqpoint{9.320225in}{4.495057in}}%
\pgfusepath{clip}%
\pgfsetbuttcap%
\pgfsetroundjoin%
\pgfsetlinewidth{0.000000pt}%
\definecolor{currentstroke}{rgb}{0.000000,0.000000,0.000000}%
\pgfsetstrokecolor{currentstroke}%
\pgfsetdash{}{0pt}%
\pgfpathmoveto{\pgfqpoint{7.067690in}{1.484877in}}%
\pgfpathlineto{\pgfqpoint{7.253917in}{1.484877in}}%
\pgfpathlineto{\pgfqpoint{7.253917in}{1.566605in}}%
\pgfpathlineto{\pgfqpoint{7.067690in}{1.566605in}}%
\pgfpathlineto{\pgfqpoint{7.067690in}{1.484877in}}%
\pgfusepath{}%
\end{pgfscope}%
\begin{pgfscope}%
\pgfpathrectangle{\pgfqpoint{0.549740in}{0.463273in}}{\pgfqpoint{9.320225in}{4.495057in}}%
\pgfusepath{clip}%
\pgfsetbuttcap%
\pgfsetroundjoin%
\pgfsetlinewidth{0.000000pt}%
\definecolor{currentstroke}{rgb}{0.000000,0.000000,0.000000}%
\pgfsetstrokecolor{currentstroke}%
\pgfsetdash{}{0pt}%
\pgfpathmoveto{\pgfqpoint{7.253917in}{1.484877in}}%
\pgfpathlineto{\pgfqpoint{7.440143in}{1.484877in}}%
\pgfpathlineto{\pgfqpoint{7.440143in}{1.566605in}}%
\pgfpathlineto{\pgfqpoint{7.253917in}{1.566605in}}%
\pgfpathlineto{\pgfqpoint{7.253917in}{1.484877in}}%
\pgfusepath{}%
\end{pgfscope}%
\begin{pgfscope}%
\pgfpathrectangle{\pgfqpoint{0.549740in}{0.463273in}}{\pgfqpoint{9.320225in}{4.495057in}}%
\pgfusepath{clip}%
\pgfsetbuttcap%
\pgfsetroundjoin%
\pgfsetlinewidth{0.000000pt}%
\definecolor{currentstroke}{rgb}{0.000000,0.000000,0.000000}%
\pgfsetstrokecolor{currentstroke}%
\pgfsetdash{}{0pt}%
\pgfpathmoveto{\pgfqpoint{7.440143in}{1.484877in}}%
\pgfpathlineto{\pgfqpoint{7.626370in}{1.484877in}}%
\pgfpathlineto{\pgfqpoint{7.626370in}{1.566605in}}%
\pgfpathlineto{\pgfqpoint{7.440143in}{1.566605in}}%
\pgfpathlineto{\pgfqpoint{7.440143in}{1.484877in}}%
\pgfusepath{}%
\end{pgfscope}%
\begin{pgfscope}%
\pgfpathrectangle{\pgfqpoint{0.549740in}{0.463273in}}{\pgfqpoint{9.320225in}{4.495057in}}%
\pgfusepath{clip}%
\pgfsetbuttcap%
\pgfsetroundjoin%
\definecolor{currentfill}{rgb}{0.385185,0.686583,0.962589}%
\pgfsetfillcolor{currentfill}%
\pgfsetlinewidth{0.000000pt}%
\definecolor{currentstroke}{rgb}{0.000000,0.000000,0.000000}%
\pgfsetstrokecolor{currentstroke}%
\pgfsetdash{}{0pt}%
\pgfpathmoveto{\pgfqpoint{7.626370in}{1.484877in}}%
\pgfpathlineto{\pgfqpoint{7.812596in}{1.484877in}}%
\pgfpathlineto{\pgfqpoint{7.812596in}{1.566605in}}%
\pgfpathlineto{\pgfqpoint{7.626370in}{1.566605in}}%
\pgfpathlineto{\pgfqpoint{7.626370in}{1.484877in}}%
\pgfusepath{fill}%
\end{pgfscope}%
\begin{pgfscope}%
\pgfpathrectangle{\pgfqpoint{0.549740in}{0.463273in}}{\pgfqpoint{9.320225in}{4.495057in}}%
\pgfusepath{clip}%
\pgfsetbuttcap%
\pgfsetroundjoin%
\definecolor{currentfill}{rgb}{0.614330,0.761948,0.940009}%
\pgfsetfillcolor{currentfill}%
\pgfsetlinewidth{0.000000pt}%
\definecolor{currentstroke}{rgb}{0.000000,0.000000,0.000000}%
\pgfsetstrokecolor{currentstroke}%
\pgfsetdash{}{0pt}%
\pgfpathmoveto{\pgfqpoint{7.812596in}{1.484877in}}%
\pgfpathlineto{\pgfqpoint{7.998823in}{1.484877in}}%
\pgfpathlineto{\pgfqpoint{7.998823in}{1.566605in}}%
\pgfpathlineto{\pgfqpoint{7.812596in}{1.566605in}}%
\pgfpathlineto{\pgfqpoint{7.812596in}{1.484877in}}%
\pgfusepath{fill}%
\end{pgfscope}%
\begin{pgfscope}%
\pgfpathrectangle{\pgfqpoint{0.549740in}{0.463273in}}{\pgfqpoint{9.320225in}{4.495057in}}%
\pgfusepath{clip}%
\pgfsetbuttcap%
\pgfsetroundjoin%
\pgfsetlinewidth{0.000000pt}%
\definecolor{currentstroke}{rgb}{0.000000,0.000000,0.000000}%
\pgfsetstrokecolor{currentstroke}%
\pgfsetdash{}{0pt}%
\pgfpathmoveto{\pgfqpoint{7.998823in}{1.484877in}}%
\pgfpathlineto{\pgfqpoint{8.185049in}{1.484877in}}%
\pgfpathlineto{\pgfqpoint{8.185049in}{1.566605in}}%
\pgfpathlineto{\pgfqpoint{7.998823in}{1.566605in}}%
\pgfpathlineto{\pgfqpoint{7.998823in}{1.484877in}}%
\pgfusepath{}%
\end{pgfscope}%
\begin{pgfscope}%
\pgfpathrectangle{\pgfqpoint{0.549740in}{0.463273in}}{\pgfqpoint{9.320225in}{4.495057in}}%
\pgfusepath{clip}%
\pgfsetbuttcap%
\pgfsetroundjoin%
\pgfsetlinewidth{0.000000pt}%
\definecolor{currentstroke}{rgb}{0.000000,0.000000,0.000000}%
\pgfsetstrokecolor{currentstroke}%
\pgfsetdash{}{0pt}%
\pgfpathmoveto{\pgfqpoint{8.185049in}{1.484877in}}%
\pgfpathlineto{\pgfqpoint{8.371276in}{1.484877in}}%
\pgfpathlineto{\pgfqpoint{8.371276in}{1.566605in}}%
\pgfpathlineto{\pgfqpoint{8.185049in}{1.566605in}}%
\pgfpathlineto{\pgfqpoint{8.185049in}{1.484877in}}%
\pgfusepath{}%
\end{pgfscope}%
\begin{pgfscope}%
\pgfpathrectangle{\pgfqpoint{0.549740in}{0.463273in}}{\pgfqpoint{9.320225in}{4.495057in}}%
\pgfusepath{clip}%
\pgfsetbuttcap%
\pgfsetroundjoin%
\pgfsetlinewidth{0.000000pt}%
\definecolor{currentstroke}{rgb}{0.000000,0.000000,0.000000}%
\pgfsetstrokecolor{currentstroke}%
\pgfsetdash{}{0pt}%
\pgfpathmoveto{\pgfqpoint{8.371276in}{1.484877in}}%
\pgfpathlineto{\pgfqpoint{8.557503in}{1.484877in}}%
\pgfpathlineto{\pgfqpoint{8.557503in}{1.566605in}}%
\pgfpathlineto{\pgfqpoint{8.371276in}{1.566605in}}%
\pgfpathlineto{\pgfqpoint{8.371276in}{1.484877in}}%
\pgfusepath{}%
\end{pgfscope}%
\begin{pgfscope}%
\pgfpathrectangle{\pgfqpoint{0.549740in}{0.463273in}}{\pgfqpoint{9.320225in}{4.495057in}}%
\pgfusepath{clip}%
\pgfsetbuttcap%
\pgfsetroundjoin%
\pgfsetlinewidth{0.000000pt}%
\definecolor{currentstroke}{rgb}{0.000000,0.000000,0.000000}%
\pgfsetstrokecolor{currentstroke}%
\pgfsetdash{}{0pt}%
\pgfpathmoveto{\pgfqpoint{8.557503in}{1.484877in}}%
\pgfpathlineto{\pgfqpoint{8.743729in}{1.484877in}}%
\pgfpathlineto{\pgfqpoint{8.743729in}{1.566605in}}%
\pgfpathlineto{\pgfqpoint{8.557503in}{1.566605in}}%
\pgfpathlineto{\pgfqpoint{8.557503in}{1.484877in}}%
\pgfusepath{}%
\end{pgfscope}%
\begin{pgfscope}%
\pgfpathrectangle{\pgfqpoint{0.549740in}{0.463273in}}{\pgfqpoint{9.320225in}{4.495057in}}%
\pgfusepath{clip}%
\pgfsetbuttcap%
\pgfsetroundjoin%
\definecolor{currentfill}{rgb}{0.385185,0.686583,0.962589}%
\pgfsetfillcolor{currentfill}%
\pgfsetlinewidth{0.000000pt}%
\definecolor{currentstroke}{rgb}{0.000000,0.000000,0.000000}%
\pgfsetstrokecolor{currentstroke}%
\pgfsetdash{}{0pt}%
\pgfpathmoveto{\pgfqpoint{8.743729in}{1.484877in}}%
\pgfpathlineto{\pgfqpoint{8.929956in}{1.484877in}}%
\pgfpathlineto{\pgfqpoint{8.929956in}{1.566605in}}%
\pgfpathlineto{\pgfqpoint{8.743729in}{1.566605in}}%
\pgfpathlineto{\pgfqpoint{8.743729in}{1.484877in}}%
\pgfusepath{fill}%
\end{pgfscope}%
\begin{pgfscope}%
\pgfpathrectangle{\pgfqpoint{0.549740in}{0.463273in}}{\pgfqpoint{9.320225in}{4.495057in}}%
\pgfusepath{clip}%
\pgfsetbuttcap%
\pgfsetroundjoin%
\pgfsetlinewidth{0.000000pt}%
\definecolor{currentstroke}{rgb}{0.000000,0.000000,0.000000}%
\pgfsetstrokecolor{currentstroke}%
\pgfsetdash{}{0pt}%
\pgfpathmoveto{\pgfqpoint{8.929956in}{1.484877in}}%
\pgfpathlineto{\pgfqpoint{9.116182in}{1.484877in}}%
\pgfpathlineto{\pgfqpoint{9.116182in}{1.566605in}}%
\pgfpathlineto{\pgfqpoint{8.929956in}{1.566605in}}%
\pgfpathlineto{\pgfqpoint{8.929956in}{1.484877in}}%
\pgfusepath{}%
\end{pgfscope}%
\begin{pgfscope}%
\pgfpathrectangle{\pgfqpoint{0.549740in}{0.463273in}}{\pgfqpoint{9.320225in}{4.495057in}}%
\pgfusepath{clip}%
\pgfsetbuttcap%
\pgfsetroundjoin%
\pgfsetlinewidth{0.000000pt}%
\definecolor{currentstroke}{rgb}{0.000000,0.000000,0.000000}%
\pgfsetstrokecolor{currentstroke}%
\pgfsetdash{}{0pt}%
\pgfpathmoveto{\pgfqpoint{9.116182in}{1.484877in}}%
\pgfpathlineto{\pgfqpoint{9.302409in}{1.484877in}}%
\pgfpathlineto{\pgfqpoint{9.302409in}{1.566605in}}%
\pgfpathlineto{\pgfqpoint{9.116182in}{1.566605in}}%
\pgfpathlineto{\pgfqpoint{9.116182in}{1.484877in}}%
\pgfusepath{}%
\end{pgfscope}%
\begin{pgfscope}%
\pgfpathrectangle{\pgfqpoint{0.549740in}{0.463273in}}{\pgfqpoint{9.320225in}{4.495057in}}%
\pgfusepath{clip}%
\pgfsetbuttcap%
\pgfsetroundjoin%
\pgfsetlinewidth{0.000000pt}%
\definecolor{currentstroke}{rgb}{0.000000,0.000000,0.000000}%
\pgfsetstrokecolor{currentstroke}%
\pgfsetdash{}{0pt}%
\pgfpathmoveto{\pgfqpoint{9.302409in}{1.484877in}}%
\pgfpathlineto{\pgfqpoint{9.488635in}{1.484877in}}%
\pgfpathlineto{\pgfqpoint{9.488635in}{1.566605in}}%
\pgfpathlineto{\pgfqpoint{9.302409in}{1.566605in}}%
\pgfpathlineto{\pgfqpoint{9.302409in}{1.484877in}}%
\pgfusepath{}%
\end{pgfscope}%
\begin{pgfscope}%
\pgfpathrectangle{\pgfqpoint{0.549740in}{0.463273in}}{\pgfqpoint{9.320225in}{4.495057in}}%
\pgfusepath{clip}%
\pgfsetbuttcap%
\pgfsetroundjoin%
\pgfsetlinewidth{0.000000pt}%
\definecolor{currentstroke}{rgb}{0.000000,0.000000,0.000000}%
\pgfsetstrokecolor{currentstroke}%
\pgfsetdash{}{0pt}%
\pgfpathmoveto{\pgfqpoint{9.488635in}{1.484877in}}%
\pgfpathlineto{\pgfqpoint{9.674862in}{1.484877in}}%
\pgfpathlineto{\pgfqpoint{9.674862in}{1.566605in}}%
\pgfpathlineto{\pgfqpoint{9.488635in}{1.566605in}}%
\pgfpathlineto{\pgfqpoint{9.488635in}{1.484877in}}%
\pgfusepath{}%
\end{pgfscope}%
\begin{pgfscope}%
\pgfpathrectangle{\pgfqpoint{0.549740in}{0.463273in}}{\pgfqpoint{9.320225in}{4.495057in}}%
\pgfusepath{clip}%
\pgfsetbuttcap%
\pgfsetroundjoin%
\pgfsetlinewidth{0.000000pt}%
\definecolor{currentstroke}{rgb}{0.000000,0.000000,0.000000}%
\pgfsetstrokecolor{currentstroke}%
\pgfsetdash{}{0pt}%
\pgfpathmoveto{\pgfqpoint{9.674862in}{1.484877in}}%
\pgfpathlineto{\pgfqpoint{9.861088in}{1.484877in}}%
\pgfpathlineto{\pgfqpoint{9.861088in}{1.566605in}}%
\pgfpathlineto{\pgfqpoint{9.674862in}{1.566605in}}%
\pgfpathlineto{\pgfqpoint{9.674862in}{1.484877in}}%
\pgfusepath{}%
\end{pgfscope}%
\begin{pgfscope}%
\pgfpathrectangle{\pgfqpoint{0.549740in}{0.463273in}}{\pgfqpoint{9.320225in}{4.495057in}}%
\pgfusepath{clip}%
\pgfsetbuttcap%
\pgfsetroundjoin%
\pgfsetlinewidth{0.000000pt}%
\definecolor{currentstroke}{rgb}{0.000000,0.000000,0.000000}%
\pgfsetstrokecolor{currentstroke}%
\pgfsetdash{}{0pt}%
\pgfpathmoveto{\pgfqpoint{0.549761in}{1.566605in}}%
\pgfpathlineto{\pgfqpoint{0.735988in}{1.566605in}}%
\pgfpathlineto{\pgfqpoint{0.735988in}{1.648334in}}%
\pgfpathlineto{\pgfqpoint{0.549761in}{1.648334in}}%
\pgfpathlineto{\pgfqpoint{0.549761in}{1.566605in}}%
\pgfusepath{}%
\end{pgfscope}%
\begin{pgfscope}%
\pgfpathrectangle{\pgfqpoint{0.549740in}{0.463273in}}{\pgfqpoint{9.320225in}{4.495057in}}%
\pgfusepath{clip}%
\pgfsetbuttcap%
\pgfsetroundjoin%
\pgfsetlinewidth{0.000000pt}%
\definecolor{currentstroke}{rgb}{0.000000,0.000000,0.000000}%
\pgfsetstrokecolor{currentstroke}%
\pgfsetdash{}{0pt}%
\pgfpathmoveto{\pgfqpoint{0.735988in}{1.566605in}}%
\pgfpathlineto{\pgfqpoint{0.922214in}{1.566605in}}%
\pgfpathlineto{\pgfqpoint{0.922214in}{1.648334in}}%
\pgfpathlineto{\pgfqpoint{0.735988in}{1.648334in}}%
\pgfpathlineto{\pgfqpoint{0.735988in}{1.566605in}}%
\pgfusepath{}%
\end{pgfscope}%
\begin{pgfscope}%
\pgfpathrectangle{\pgfqpoint{0.549740in}{0.463273in}}{\pgfqpoint{9.320225in}{4.495057in}}%
\pgfusepath{clip}%
\pgfsetbuttcap%
\pgfsetroundjoin%
\pgfsetlinewidth{0.000000pt}%
\definecolor{currentstroke}{rgb}{0.000000,0.000000,0.000000}%
\pgfsetstrokecolor{currentstroke}%
\pgfsetdash{}{0pt}%
\pgfpathmoveto{\pgfqpoint{0.922214in}{1.566605in}}%
\pgfpathlineto{\pgfqpoint{1.108441in}{1.566605in}}%
\pgfpathlineto{\pgfqpoint{1.108441in}{1.648334in}}%
\pgfpathlineto{\pgfqpoint{0.922214in}{1.648334in}}%
\pgfpathlineto{\pgfqpoint{0.922214in}{1.566605in}}%
\pgfusepath{}%
\end{pgfscope}%
\begin{pgfscope}%
\pgfpathrectangle{\pgfqpoint{0.549740in}{0.463273in}}{\pgfqpoint{9.320225in}{4.495057in}}%
\pgfusepath{clip}%
\pgfsetbuttcap%
\pgfsetroundjoin%
\pgfsetlinewidth{0.000000pt}%
\definecolor{currentstroke}{rgb}{0.000000,0.000000,0.000000}%
\pgfsetstrokecolor{currentstroke}%
\pgfsetdash{}{0pt}%
\pgfpathmoveto{\pgfqpoint{1.108441in}{1.566605in}}%
\pgfpathlineto{\pgfqpoint{1.294667in}{1.566605in}}%
\pgfpathlineto{\pgfqpoint{1.294667in}{1.648334in}}%
\pgfpathlineto{\pgfqpoint{1.108441in}{1.648334in}}%
\pgfpathlineto{\pgfqpoint{1.108441in}{1.566605in}}%
\pgfusepath{}%
\end{pgfscope}%
\begin{pgfscope}%
\pgfpathrectangle{\pgfqpoint{0.549740in}{0.463273in}}{\pgfqpoint{9.320225in}{4.495057in}}%
\pgfusepath{clip}%
\pgfsetbuttcap%
\pgfsetroundjoin%
\pgfsetlinewidth{0.000000pt}%
\definecolor{currentstroke}{rgb}{0.000000,0.000000,0.000000}%
\pgfsetstrokecolor{currentstroke}%
\pgfsetdash{}{0pt}%
\pgfpathmoveto{\pgfqpoint{1.294667in}{1.566605in}}%
\pgfpathlineto{\pgfqpoint{1.480894in}{1.566605in}}%
\pgfpathlineto{\pgfqpoint{1.480894in}{1.648334in}}%
\pgfpathlineto{\pgfqpoint{1.294667in}{1.648334in}}%
\pgfpathlineto{\pgfqpoint{1.294667in}{1.566605in}}%
\pgfusepath{}%
\end{pgfscope}%
\begin{pgfscope}%
\pgfpathrectangle{\pgfqpoint{0.549740in}{0.463273in}}{\pgfqpoint{9.320225in}{4.495057in}}%
\pgfusepath{clip}%
\pgfsetbuttcap%
\pgfsetroundjoin%
\pgfsetlinewidth{0.000000pt}%
\definecolor{currentstroke}{rgb}{0.000000,0.000000,0.000000}%
\pgfsetstrokecolor{currentstroke}%
\pgfsetdash{}{0pt}%
\pgfpathmoveto{\pgfqpoint{1.480894in}{1.566605in}}%
\pgfpathlineto{\pgfqpoint{1.667120in}{1.566605in}}%
\pgfpathlineto{\pgfqpoint{1.667120in}{1.648334in}}%
\pgfpathlineto{\pgfqpoint{1.480894in}{1.648334in}}%
\pgfpathlineto{\pgfqpoint{1.480894in}{1.566605in}}%
\pgfusepath{}%
\end{pgfscope}%
\begin{pgfscope}%
\pgfpathrectangle{\pgfqpoint{0.549740in}{0.463273in}}{\pgfqpoint{9.320225in}{4.495057in}}%
\pgfusepath{clip}%
\pgfsetbuttcap%
\pgfsetroundjoin%
\pgfsetlinewidth{0.000000pt}%
\definecolor{currentstroke}{rgb}{0.000000,0.000000,0.000000}%
\pgfsetstrokecolor{currentstroke}%
\pgfsetdash{}{0pt}%
\pgfpathmoveto{\pgfqpoint{1.667120in}{1.566605in}}%
\pgfpathlineto{\pgfqpoint{1.853347in}{1.566605in}}%
\pgfpathlineto{\pgfqpoint{1.853347in}{1.648334in}}%
\pgfpathlineto{\pgfqpoint{1.667120in}{1.648334in}}%
\pgfpathlineto{\pgfqpoint{1.667120in}{1.566605in}}%
\pgfusepath{}%
\end{pgfscope}%
\begin{pgfscope}%
\pgfpathrectangle{\pgfqpoint{0.549740in}{0.463273in}}{\pgfqpoint{9.320225in}{4.495057in}}%
\pgfusepath{clip}%
\pgfsetbuttcap%
\pgfsetroundjoin%
\pgfsetlinewidth{0.000000pt}%
\definecolor{currentstroke}{rgb}{0.000000,0.000000,0.000000}%
\pgfsetstrokecolor{currentstroke}%
\pgfsetdash{}{0pt}%
\pgfpathmoveto{\pgfqpoint{1.853347in}{1.566605in}}%
\pgfpathlineto{\pgfqpoint{2.039573in}{1.566605in}}%
\pgfpathlineto{\pgfqpoint{2.039573in}{1.648334in}}%
\pgfpathlineto{\pgfqpoint{1.853347in}{1.648334in}}%
\pgfpathlineto{\pgfqpoint{1.853347in}{1.566605in}}%
\pgfusepath{}%
\end{pgfscope}%
\begin{pgfscope}%
\pgfpathrectangle{\pgfqpoint{0.549740in}{0.463273in}}{\pgfqpoint{9.320225in}{4.495057in}}%
\pgfusepath{clip}%
\pgfsetbuttcap%
\pgfsetroundjoin%
\pgfsetlinewidth{0.000000pt}%
\definecolor{currentstroke}{rgb}{0.000000,0.000000,0.000000}%
\pgfsetstrokecolor{currentstroke}%
\pgfsetdash{}{0pt}%
\pgfpathmoveto{\pgfqpoint{2.039573in}{1.566605in}}%
\pgfpathlineto{\pgfqpoint{2.225800in}{1.566605in}}%
\pgfpathlineto{\pgfqpoint{2.225800in}{1.648334in}}%
\pgfpathlineto{\pgfqpoint{2.039573in}{1.648334in}}%
\pgfpathlineto{\pgfqpoint{2.039573in}{1.566605in}}%
\pgfusepath{}%
\end{pgfscope}%
\begin{pgfscope}%
\pgfpathrectangle{\pgfqpoint{0.549740in}{0.463273in}}{\pgfqpoint{9.320225in}{4.495057in}}%
\pgfusepath{clip}%
\pgfsetbuttcap%
\pgfsetroundjoin%
\pgfsetlinewidth{0.000000pt}%
\definecolor{currentstroke}{rgb}{0.000000,0.000000,0.000000}%
\pgfsetstrokecolor{currentstroke}%
\pgfsetdash{}{0pt}%
\pgfpathmoveto{\pgfqpoint{2.225800in}{1.566605in}}%
\pgfpathlineto{\pgfqpoint{2.412027in}{1.566605in}}%
\pgfpathlineto{\pgfqpoint{2.412027in}{1.648334in}}%
\pgfpathlineto{\pgfqpoint{2.225800in}{1.648334in}}%
\pgfpathlineto{\pgfqpoint{2.225800in}{1.566605in}}%
\pgfusepath{}%
\end{pgfscope}%
\begin{pgfscope}%
\pgfpathrectangle{\pgfqpoint{0.549740in}{0.463273in}}{\pgfqpoint{9.320225in}{4.495057in}}%
\pgfusepath{clip}%
\pgfsetbuttcap%
\pgfsetroundjoin%
\pgfsetlinewidth{0.000000pt}%
\definecolor{currentstroke}{rgb}{0.000000,0.000000,0.000000}%
\pgfsetstrokecolor{currentstroke}%
\pgfsetdash{}{0pt}%
\pgfpathmoveto{\pgfqpoint{2.412027in}{1.566605in}}%
\pgfpathlineto{\pgfqpoint{2.598253in}{1.566605in}}%
\pgfpathlineto{\pgfqpoint{2.598253in}{1.648334in}}%
\pgfpathlineto{\pgfqpoint{2.412027in}{1.648334in}}%
\pgfpathlineto{\pgfqpoint{2.412027in}{1.566605in}}%
\pgfusepath{}%
\end{pgfscope}%
\begin{pgfscope}%
\pgfpathrectangle{\pgfqpoint{0.549740in}{0.463273in}}{\pgfqpoint{9.320225in}{4.495057in}}%
\pgfusepath{clip}%
\pgfsetbuttcap%
\pgfsetroundjoin%
\pgfsetlinewidth{0.000000pt}%
\definecolor{currentstroke}{rgb}{0.000000,0.000000,0.000000}%
\pgfsetstrokecolor{currentstroke}%
\pgfsetdash{}{0pt}%
\pgfpathmoveto{\pgfqpoint{2.598253in}{1.566605in}}%
\pgfpathlineto{\pgfqpoint{2.784480in}{1.566605in}}%
\pgfpathlineto{\pgfqpoint{2.784480in}{1.648334in}}%
\pgfpathlineto{\pgfqpoint{2.598253in}{1.648334in}}%
\pgfpathlineto{\pgfqpoint{2.598253in}{1.566605in}}%
\pgfusepath{}%
\end{pgfscope}%
\begin{pgfscope}%
\pgfpathrectangle{\pgfqpoint{0.549740in}{0.463273in}}{\pgfqpoint{9.320225in}{4.495057in}}%
\pgfusepath{clip}%
\pgfsetbuttcap%
\pgfsetroundjoin%
\pgfsetlinewidth{0.000000pt}%
\definecolor{currentstroke}{rgb}{0.000000,0.000000,0.000000}%
\pgfsetstrokecolor{currentstroke}%
\pgfsetdash{}{0pt}%
\pgfpathmoveto{\pgfqpoint{2.784480in}{1.566605in}}%
\pgfpathlineto{\pgfqpoint{2.970706in}{1.566605in}}%
\pgfpathlineto{\pgfqpoint{2.970706in}{1.648334in}}%
\pgfpathlineto{\pgfqpoint{2.784480in}{1.648334in}}%
\pgfpathlineto{\pgfqpoint{2.784480in}{1.566605in}}%
\pgfusepath{}%
\end{pgfscope}%
\begin{pgfscope}%
\pgfpathrectangle{\pgfqpoint{0.549740in}{0.463273in}}{\pgfqpoint{9.320225in}{4.495057in}}%
\pgfusepath{clip}%
\pgfsetbuttcap%
\pgfsetroundjoin%
\pgfsetlinewidth{0.000000pt}%
\definecolor{currentstroke}{rgb}{0.000000,0.000000,0.000000}%
\pgfsetstrokecolor{currentstroke}%
\pgfsetdash{}{0pt}%
\pgfpathmoveto{\pgfqpoint{2.970706in}{1.566605in}}%
\pgfpathlineto{\pgfqpoint{3.156933in}{1.566605in}}%
\pgfpathlineto{\pgfqpoint{3.156933in}{1.648334in}}%
\pgfpathlineto{\pgfqpoint{2.970706in}{1.648334in}}%
\pgfpathlineto{\pgfqpoint{2.970706in}{1.566605in}}%
\pgfusepath{}%
\end{pgfscope}%
\begin{pgfscope}%
\pgfpathrectangle{\pgfqpoint{0.549740in}{0.463273in}}{\pgfqpoint{9.320225in}{4.495057in}}%
\pgfusepath{clip}%
\pgfsetbuttcap%
\pgfsetroundjoin%
\pgfsetlinewidth{0.000000pt}%
\definecolor{currentstroke}{rgb}{0.000000,0.000000,0.000000}%
\pgfsetstrokecolor{currentstroke}%
\pgfsetdash{}{0pt}%
\pgfpathmoveto{\pgfqpoint{3.156933in}{1.566605in}}%
\pgfpathlineto{\pgfqpoint{3.343159in}{1.566605in}}%
\pgfpathlineto{\pgfqpoint{3.343159in}{1.648334in}}%
\pgfpathlineto{\pgfqpoint{3.156933in}{1.648334in}}%
\pgfpathlineto{\pgfqpoint{3.156933in}{1.566605in}}%
\pgfusepath{}%
\end{pgfscope}%
\begin{pgfscope}%
\pgfpathrectangle{\pgfqpoint{0.549740in}{0.463273in}}{\pgfqpoint{9.320225in}{4.495057in}}%
\pgfusepath{clip}%
\pgfsetbuttcap%
\pgfsetroundjoin%
\pgfsetlinewidth{0.000000pt}%
\definecolor{currentstroke}{rgb}{0.000000,0.000000,0.000000}%
\pgfsetstrokecolor{currentstroke}%
\pgfsetdash{}{0pt}%
\pgfpathmoveto{\pgfqpoint{3.343159in}{1.566605in}}%
\pgfpathlineto{\pgfqpoint{3.529386in}{1.566605in}}%
\pgfpathlineto{\pgfqpoint{3.529386in}{1.648334in}}%
\pgfpathlineto{\pgfqpoint{3.343159in}{1.648334in}}%
\pgfpathlineto{\pgfqpoint{3.343159in}{1.566605in}}%
\pgfusepath{}%
\end{pgfscope}%
\begin{pgfscope}%
\pgfpathrectangle{\pgfqpoint{0.549740in}{0.463273in}}{\pgfqpoint{9.320225in}{4.495057in}}%
\pgfusepath{clip}%
\pgfsetbuttcap%
\pgfsetroundjoin%
\pgfsetlinewidth{0.000000pt}%
\definecolor{currentstroke}{rgb}{0.000000,0.000000,0.000000}%
\pgfsetstrokecolor{currentstroke}%
\pgfsetdash{}{0pt}%
\pgfpathmoveto{\pgfqpoint{3.529386in}{1.566605in}}%
\pgfpathlineto{\pgfqpoint{3.715612in}{1.566605in}}%
\pgfpathlineto{\pgfqpoint{3.715612in}{1.648334in}}%
\pgfpathlineto{\pgfqpoint{3.529386in}{1.648334in}}%
\pgfpathlineto{\pgfqpoint{3.529386in}{1.566605in}}%
\pgfusepath{}%
\end{pgfscope}%
\begin{pgfscope}%
\pgfpathrectangle{\pgfqpoint{0.549740in}{0.463273in}}{\pgfqpoint{9.320225in}{4.495057in}}%
\pgfusepath{clip}%
\pgfsetbuttcap%
\pgfsetroundjoin%
\pgfsetlinewidth{0.000000pt}%
\definecolor{currentstroke}{rgb}{0.000000,0.000000,0.000000}%
\pgfsetstrokecolor{currentstroke}%
\pgfsetdash{}{0pt}%
\pgfpathmoveto{\pgfqpoint{3.715612in}{1.566605in}}%
\pgfpathlineto{\pgfqpoint{3.901839in}{1.566605in}}%
\pgfpathlineto{\pgfqpoint{3.901839in}{1.648334in}}%
\pgfpathlineto{\pgfqpoint{3.715612in}{1.648334in}}%
\pgfpathlineto{\pgfqpoint{3.715612in}{1.566605in}}%
\pgfusepath{}%
\end{pgfscope}%
\begin{pgfscope}%
\pgfpathrectangle{\pgfqpoint{0.549740in}{0.463273in}}{\pgfqpoint{9.320225in}{4.495057in}}%
\pgfusepath{clip}%
\pgfsetbuttcap%
\pgfsetroundjoin%
\pgfsetlinewidth{0.000000pt}%
\definecolor{currentstroke}{rgb}{0.000000,0.000000,0.000000}%
\pgfsetstrokecolor{currentstroke}%
\pgfsetdash{}{0pt}%
\pgfpathmoveto{\pgfqpoint{3.901839in}{1.566605in}}%
\pgfpathlineto{\pgfqpoint{4.088065in}{1.566605in}}%
\pgfpathlineto{\pgfqpoint{4.088065in}{1.648334in}}%
\pgfpathlineto{\pgfqpoint{3.901839in}{1.648334in}}%
\pgfpathlineto{\pgfqpoint{3.901839in}{1.566605in}}%
\pgfusepath{}%
\end{pgfscope}%
\begin{pgfscope}%
\pgfpathrectangle{\pgfqpoint{0.549740in}{0.463273in}}{\pgfqpoint{9.320225in}{4.495057in}}%
\pgfusepath{clip}%
\pgfsetbuttcap%
\pgfsetroundjoin%
\pgfsetlinewidth{0.000000pt}%
\definecolor{currentstroke}{rgb}{0.000000,0.000000,0.000000}%
\pgfsetstrokecolor{currentstroke}%
\pgfsetdash{}{0pt}%
\pgfpathmoveto{\pgfqpoint{4.088065in}{1.566605in}}%
\pgfpathlineto{\pgfqpoint{4.274292in}{1.566605in}}%
\pgfpathlineto{\pgfqpoint{4.274292in}{1.648334in}}%
\pgfpathlineto{\pgfqpoint{4.088065in}{1.648334in}}%
\pgfpathlineto{\pgfqpoint{4.088065in}{1.566605in}}%
\pgfusepath{}%
\end{pgfscope}%
\begin{pgfscope}%
\pgfpathrectangle{\pgfqpoint{0.549740in}{0.463273in}}{\pgfqpoint{9.320225in}{4.495057in}}%
\pgfusepath{clip}%
\pgfsetbuttcap%
\pgfsetroundjoin%
\pgfsetlinewidth{0.000000pt}%
\definecolor{currentstroke}{rgb}{0.000000,0.000000,0.000000}%
\pgfsetstrokecolor{currentstroke}%
\pgfsetdash{}{0pt}%
\pgfpathmoveto{\pgfqpoint{4.274292in}{1.566605in}}%
\pgfpathlineto{\pgfqpoint{4.460519in}{1.566605in}}%
\pgfpathlineto{\pgfqpoint{4.460519in}{1.648334in}}%
\pgfpathlineto{\pgfqpoint{4.274292in}{1.648334in}}%
\pgfpathlineto{\pgfqpoint{4.274292in}{1.566605in}}%
\pgfusepath{}%
\end{pgfscope}%
\begin{pgfscope}%
\pgfpathrectangle{\pgfqpoint{0.549740in}{0.463273in}}{\pgfqpoint{9.320225in}{4.495057in}}%
\pgfusepath{clip}%
\pgfsetbuttcap%
\pgfsetroundjoin%
\pgfsetlinewidth{0.000000pt}%
\definecolor{currentstroke}{rgb}{0.000000,0.000000,0.000000}%
\pgfsetstrokecolor{currentstroke}%
\pgfsetdash{}{0pt}%
\pgfpathmoveto{\pgfqpoint{4.460519in}{1.566605in}}%
\pgfpathlineto{\pgfqpoint{4.646745in}{1.566605in}}%
\pgfpathlineto{\pgfqpoint{4.646745in}{1.648334in}}%
\pgfpathlineto{\pgfqpoint{4.460519in}{1.648334in}}%
\pgfpathlineto{\pgfqpoint{4.460519in}{1.566605in}}%
\pgfusepath{}%
\end{pgfscope}%
\begin{pgfscope}%
\pgfpathrectangle{\pgfqpoint{0.549740in}{0.463273in}}{\pgfqpoint{9.320225in}{4.495057in}}%
\pgfusepath{clip}%
\pgfsetbuttcap%
\pgfsetroundjoin%
\pgfsetlinewidth{0.000000pt}%
\definecolor{currentstroke}{rgb}{0.000000,0.000000,0.000000}%
\pgfsetstrokecolor{currentstroke}%
\pgfsetdash{}{0pt}%
\pgfpathmoveto{\pgfqpoint{4.646745in}{1.566605in}}%
\pgfpathlineto{\pgfqpoint{4.832972in}{1.566605in}}%
\pgfpathlineto{\pgfqpoint{4.832972in}{1.648334in}}%
\pgfpathlineto{\pgfqpoint{4.646745in}{1.648334in}}%
\pgfpathlineto{\pgfqpoint{4.646745in}{1.566605in}}%
\pgfusepath{}%
\end{pgfscope}%
\begin{pgfscope}%
\pgfpathrectangle{\pgfqpoint{0.549740in}{0.463273in}}{\pgfqpoint{9.320225in}{4.495057in}}%
\pgfusepath{clip}%
\pgfsetbuttcap%
\pgfsetroundjoin%
\pgfsetlinewidth{0.000000pt}%
\definecolor{currentstroke}{rgb}{0.000000,0.000000,0.000000}%
\pgfsetstrokecolor{currentstroke}%
\pgfsetdash{}{0pt}%
\pgfpathmoveto{\pgfqpoint{4.832972in}{1.566605in}}%
\pgfpathlineto{\pgfqpoint{5.019198in}{1.566605in}}%
\pgfpathlineto{\pgfqpoint{5.019198in}{1.648334in}}%
\pgfpathlineto{\pgfqpoint{4.832972in}{1.648334in}}%
\pgfpathlineto{\pgfqpoint{4.832972in}{1.566605in}}%
\pgfusepath{}%
\end{pgfscope}%
\begin{pgfscope}%
\pgfpathrectangle{\pgfqpoint{0.549740in}{0.463273in}}{\pgfqpoint{9.320225in}{4.495057in}}%
\pgfusepath{clip}%
\pgfsetbuttcap%
\pgfsetroundjoin%
\pgfsetlinewidth{0.000000pt}%
\definecolor{currentstroke}{rgb}{0.000000,0.000000,0.000000}%
\pgfsetstrokecolor{currentstroke}%
\pgfsetdash{}{0pt}%
\pgfpathmoveto{\pgfqpoint{5.019198in}{1.566605in}}%
\pgfpathlineto{\pgfqpoint{5.205425in}{1.566605in}}%
\pgfpathlineto{\pgfqpoint{5.205425in}{1.648334in}}%
\pgfpathlineto{\pgfqpoint{5.019198in}{1.648334in}}%
\pgfpathlineto{\pgfqpoint{5.019198in}{1.566605in}}%
\pgfusepath{}%
\end{pgfscope}%
\begin{pgfscope}%
\pgfpathrectangle{\pgfqpoint{0.549740in}{0.463273in}}{\pgfqpoint{9.320225in}{4.495057in}}%
\pgfusepath{clip}%
\pgfsetbuttcap%
\pgfsetroundjoin%
\pgfsetlinewidth{0.000000pt}%
\definecolor{currentstroke}{rgb}{0.000000,0.000000,0.000000}%
\pgfsetstrokecolor{currentstroke}%
\pgfsetdash{}{0pt}%
\pgfpathmoveto{\pgfqpoint{5.205425in}{1.566605in}}%
\pgfpathlineto{\pgfqpoint{5.391651in}{1.566605in}}%
\pgfpathlineto{\pgfqpoint{5.391651in}{1.648334in}}%
\pgfpathlineto{\pgfqpoint{5.205425in}{1.648334in}}%
\pgfpathlineto{\pgfqpoint{5.205425in}{1.566605in}}%
\pgfusepath{}%
\end{pgfscope}%
\begin{pgfscope}%
\pgfpathrectangle{\pgfqpoint{0.549740in}{0.463273in}}{\pgfqpoint{9.320225in}{4.495057in}}%
\pgfusepath{clip}%
\pgfsetbuttcap%
\pgfsetroundjoin%
\pgfsetlinewidth{0.000000pt}%
\definecolor{currentstroke}{rgb}{0.000000,0.000000,0.000000}%
\pgfsetstrokecolor{currentstroke}%
\pgfsetdash{}{0pt}%
\pgfpathmoveto{\pgfqpoint{5.391651in}{1.566605in}}%
\pgfpathlineto{\pgfqpoint{5.577878in}{1.566605in}}%
\pgfpathlineto{\pgfqpoint{5.577878in}{1.648334in}}%
\pgfpathlineto{\pgfqpoint{5.391651in}{1.648334in}}%
\pgfpathlineto{\pgfqpoint{5.391651in}{1.566605in}}%
\pgfusepath{}%
\end{pgfscope}%
\begin{pgfscope}%
\pgfpathrectangle{\pgfqpoint{0.549740in}{0.463273in}}{\pgfqpoint{9.320225in}{4.495057in}}%
\pgfusepath{clip}%
\pgfsetbuttcap%
\pgfsetroundjoin%
\definecolor{currentfill}{rgb}{0.472869,0.711325,0.955316}%
\pgfsetfillcolor{currentfill}%
\pgfsetlinewidth{0.000000pt}%
\definecolor{currentstroke}{rgb}{0.000000,0.000000,0.000000}%
\pgfsetstrokecolor{currentstroke}%
\pgfsetdash{}{0pt}%
\pgfpathmoveto{\pgfqpoint{5.577878in}{1.566605in}}%
\pgfpathlineto{\pgfqpoint{5.764104in}{1.566605in}}%
\pgfpathlineto{\pgfqpoint{5.764104in}{1.648334in}}%
\pgfpathlineto{\pgfqpoint{5.577878in}{1.648334in}}%
\pgfpathlineto{\pgfqpoint{5.577878in}{1.566605in}}%
\pgfusepath{fill}%
\end{pgfscope}%
\begin{pgfscope}%
\pgfpathrectangle{\pgfqpoint{0.549740in}{0.463273in}}{\pgfqpoint{9.320225in}{4.495057in}}%
\pgfusepath{clip}%
\pgfsetbuttcap%
\pgfsetroundjoin%
\pgfsetlinewidth{0.000000pt}%
\definecolor{currentstroke}{rgb}{0.000000,0.000000,0.000000}%
\pgfsetstrokecolor{currentstroke}%
\pgfsetdash{}{0pt}%
\pgfpathmoveto{\pgfqpoint{5.764104in}{1.566605in}}%
\pgfpathlineto{\pgfqpoint{5.950331in}{1.566605in}}%
\pgfpathlineto{\pgfqpoint{5.950331in}{1.648334in}}%
\pgfpathlineto{\pgfqpoint{5.764104in}{1.648334in}}%
\pgfpathlineto{\pgfqpoint{5.764104in}{1.566605in}}%
\pgfusepath{}%
\end{pgfscope}%
\begin{pgfscope}%
\pgfpathrectangle{\pgfqpoint{0.549740in}{0.463273in}}{\pgfqpoint{9.320225in}{4.495057in}}%
\pgfusepath{clip}%
\pgfsetbuttcap%
\pgfsetroundjoin%
\pgfsetlinewidth{0.000000pt}%
\definecolor{currentstroke}{rgb}{0.000000,0.000000,0.000000}%
\pgfsetstrokecolor{currentstroke}%
\pgfsetdash{}{0pt}%
\pgfpathmoveto{\pgfqpoint{5.950331in}{1.566605in}}%
\pgfpathlineto{\pgfqpoint{6.136557in}{1.566605in}}%
\pgfpathlineto{\pgfqpoint{6.136557in}{1.648334in}}%
\pgfpathlineto{\pgfqpoint{5.950331in}{1.648334in}}%
\pgfpathlineto{\pgfqpoint{5.950331in}{1.566605in}}%
\pgfusepath{}%
\end{pgfscope}%
\begin{pgfscope}%
\pgfpathrectangle{\pgfqpoint{0.549740in}{0.463273in}}{\pgfqpoint{9.320225in}{4.495057in}}%
\pgfusepath{clip}%
\pgfsetbuttcap%
\pgfsetroundjoin%
\pgfsetlinewidth{0.000000pt}%
\definecolor{currentstroke}{rgb}{0.000000,0.000000,0.000000}%
\pgfsetstrokecolor{currentstroke}%
\pgfsetdash{}{0pt}%
\pgfpathmoveto{\pgfqpoint{6.136557in}{1.566605in}}%
\pgfpathlineto{\pgfqpoint{6.322784in}{1.566605in}}%
\pgfpathlineto{\pgfqpoint{6.322784in}{1.648334in}}%
\pgfpathlineto{\pgfqpoint{6.136557in}{1.648334in}}%
\pgfpathlineto{\pgfqpoint{6.136557in}{1.566605in}}%
\pgfusepath{}%
\end{pgfscope}%
\begin{pgfscope}%
\pgfpathrectangle{\pgfqpoint{0.549740in}{0.463273in}}{\pgfqpoint{9.320225in}{4.495057in}}%
\pgfusepath{clip}%
\pgfsetbuttcap%
\pgfsetroundjoin%
\pgfsetlinewidth{0.000000pt}%
\definecolor{currentstroke}{rgb}{0.000000,0.000000,0.000000}%
\pgfsetstrokecolor{currentstroke}%
\pgfsetdash{}{0pt}%
\pgfpathmoveto{\pgfqpoint{6.322784in}{1.566605in}}%
\pgfpathlineto{\pgfqpoint{6.509011in}{1.566605in}}%
\pgfpathlineto{\pgfqpoint{6.509011in}{1.648334in}}%
\pgfpathlineto{\pgfqpoint{6.322784in}{1.648334in}}%
\pgfpathlineto{\pgfqpoint{6.322784in}{1.566605in}}%
\pgfusepath{}%
\end{pgfscope}%
\begin{pgfscope}%
\pgfpathrectangle{\pgfqpoint{0.549740in}{0.463273in}}{\pgfqpoint{9.320225in}{4.495057in}}%
\pgfusepath{clip}%
\pgfsetbuttcap%
\pgfsetroundjoin%
\definecolor{currentfill}{rgb}{0.472869,0.711325,0.955316}%
\pgfsetfillcolor{currentfill}%
\pgfsetlinewidth{0.000000pt}%
\definecolor{currentstroke}{rgb}{0.000000,0.000000,0.000000}%
\pgfsetstrokecolor{currentstroke}%
\pgfsetdash{}{0pt}%
\pgfpathmoveto{\pgfqpoint{6.509011in}{1.566605in}}%
\pgfpathlineto{\pgfqpoint{6.695237in}{1.566605in}}%
\pgfpathlineto{\pgfqpoint{6.695237in}{1.648334in}}%
\pgfpathlineto{\pgfqpoint{6.509011in}{1.648334in}}%
\pgfpathlineto{\pgfqpoint{6.509011in}{1.566605in}}%
\pgfusepath{fill}%
\end{pgfscope}%
\begin{pgfscope}%
\pgfpathrectangle{\pgfqpoint{0.549740in}{0.463273in}}{\pgfqpoint{9.320225in}{4.495057in}}%
\pgfusepath{clip}%
\pgfsetbuttcap%
\pgfsetroundjoin%
\pgfsetlinewidth{0.000000pt}%
\definecolor{currentstroke}{rgb}{0.000000,0.000000,0.000000}%
\pgfsetstrokecolor{currentstroke}%
\pgfsetdash{}{0pt}%
\pgfpathmoveto{\pgfqpoint{6.695237in}{1.566605in}}%
\pgfpathlineto{\pgfqpoint{6.881464in}{1.566605in}}%
\pgfpathlineto{\pgfqpoint{6.881464in}{1.648334in}}%
\pgfpathlineto{\pgfqpoint{6.695237in}{1.648334in}}%
\pgfpathlineto{\pgfqpoint{6.695237in}{1.566605in}}%
\pgfusepath{}%
\end{pgfscope}%
\begin{pgfscope}%
\pgfpathrectangle{\pgfqpoint{0.549740in}{0.463273in}}{\pgfqpoint{9.320225in}{4.495057in}}%
\pgfusepath{clip}%
\pgfsetbuttcap%
\pgfsetroundjoin%
\pgfsetlinewidth{0.000000pt}%
\definecolor{currentstroke}{rgb}{0.000000,0.000000,0.000000}%
\pgfsetstrokecolor{currentstroke}%
\pgfsetdash{}{0pt}%
\pgfpathmoveto{\pgfqpoint{6.881464in}{1.566605in}}%
\pgfpathlineto{\pgfqpoint{7.067690in}{1.566605in}}%
\pgfpathlineto{\pgfqpoint{7.067690in}{1.648334in}}%
\pgfpathlineto{\pgfqpoint{6.881464in}{1.648334in}}%
\pgfpathlineto{\pgfqpoint{6.881464in}{1.566605in}}%
\pgfusepath{}%
\end{pgfscope}%
\begin{pgfscope}%
\pgfpathrectangle{\pgfqpoint{0.549740in}{0.463273in}}{\pgfqpoint{9.320225in}{4.495057in}}%
\pgfusepath{clip}%
\pgfsetbuttcap%
\pgfsetroundjoin%
\pgfsetlinewidth{0.000000pt}%
\definecolor{currentstroke}{rgb}{0.000000,0.000000,0.000000}%
\pgfsetstrokecolor{currentstroke}%
\pgfsetdash{}{0pt}%
\pgfpathmoveto{\pgfqpoint{7.067690in}{1.566605in}}%
\pgfpathlineto{\pgfqpoint{7.253917in}{1.566605in}}%
\pgfpathlineto{\pgfqpoint{7.253917in}{1.648334in}}%
\pgfpathlineto{\pgfqpoint{7.067690in}{1.648334in}}%
\pgfpathlineto{\pgfqpoint{7.067690in}{1.566605in}}%
\pgfusepath{}%
\end{pgfscope}%
\begin{pgfscope}%
\pgfpathrectangle{\pgfqpoint{0.549740in}{0.463273in}}{\pgfqpoint{9.320225in}{4.495057in}}%
\pgfusepath{clip}%
\pgfsetbuttcap%
\pgfsetroundjoin%
\pgfsetlinewidth{0.000000pt}%
\definecolor{currentstroke}{rgb}{0.000000,0.000000,0.000000}%
\pgfsetstrokecolor{currentstroke}%
\pgfsetdash{}{0pt}%
\pgfpathmoveto{\pgfqpoint{7.253917in}{1.566605in}}%
\pgfpathlineto{\pgfqpoint{7.440143in}{1.566605in}}%
\pgfpathlineto{\pgfqpoint{7.440143in}{1.648334in}}%
\pgfpathlineto{\pgfqpoint{7.253917in}{1.648334in}}%
\pgfpathlineto{\pgfqpoint{7.253917in}{1.566605in}}%
\pgfusepath{}%
\end{pgfscope}%
\begin{pgfscope}%
\pgfpathrectangle{\pgfqpoint{0.549740in}{0.463273in}}{\pgfqpoint{9.320225in}{4.495057in}}%
\pgfusepath{clip}%
\pgfsetbuttcap%
\pgfsetroundjoin%
\pgfsetlinewidth{0.000000pt}%
\definecolor{currentstroke}{rgb}{0.000000,0.000000,0.000000}%
\pgfsetstrokecolor{currentstroke}%
\pgfsetdash{}{0pt}%
\pgfpathmoveto{\pgfqpoint{7.440143in}{1.566605in}}%
\pgfpathlineto{\pgfqpoint{7.626370in}{1.566605in}}%
\pgfpathlineto{\pgfqpoint{7.626370in}{1.648334in}}%
\pgfpathlineto{\pgfqpoint{7.440143in}{1.648334in}}%
\pgfpathlineto{\pgfqpoint{7.440143in}{1.566605in}}%
\pgfusepath{}%
\end{pgfscope}%
\begin{pgfscope}%
\pgfpathrectangle{\pgfqpoint{0.549740in}{0.463273in}}{\pgfqpoint{9.320225in}{4.495057in}}%
\pgfusepath{clip}%
\pgfsetbuttcap%
\pgfsetroundjoin%
\definecolor{currentfill}{rgb}{0.189527,0.635753,0.950228}%
\pgfsetfillcolor{currentfill}%
\pgfsetlinewidth{0.000000pt}%
\definecolor{currentstroke}{rgb}{0.000000,0.000000,0.000000}%
\pgfsetstrokecolor{currentstroke}%
\pgfsetdash{}{0pt}%
\pgfpathmoveto{\pgfqpoint{7.626370in}{1.566605in}}%
\pgfpathlineto{\pgfqpoint{7.812596in}{1.566605in}}%
\pgfpathlineto{\pgfqpoint{7.812596in}{1.648334in}}%
\pgfpathlineto{\pgfqpoint{7.626370in}{1.648334in}}%
\pgfpathlineto{\pgfqpoint{7.626370in}{1.566605in}}%
\pgfusepath{fill}%
\end{pgfscope}%
\begin{pgfscope}%
\pgfpathrectangle{\pgfqpoint{0.549740in}{0.463273in}}{\pgfqpoint{9.320225in}{4.495057in}}%
\pgfusepath{clip}%
\pgfsetbuttcap%
\pgfsetroundjoin%
\pgfsetlinewidth{0.000000pt}%
\definecolor{currentstroke}{rgb}{0.000000,0.000000,0.000000}%
\pgfsetstrokecolor{currentstroke}%
\pgfsetdash{}{0pt}%
\pgfpathmoveto{\pgfqpoint{7.812596in}{1.566605in}}%
\pgfpathlineto{\pgfqpoint{7.998823in}{1.566605in}}%
\pgfpathlineto{\pgfqpoint{7.998823in}{1.648334in}}%
\pgfpathlineto{\pgfqpoint{7.812596in}{1.648334in}}%
\pgfpathlineto{\pgfqpoint{7.812596in}{1.566605in}}%
\pgfusepath{}%
\end{pgfscope}%
\begin{pgfscope}%
\pgfpathrectangle{\pgfqpoint{0.549740in}{0.463273in}}{\pgfqpoint{9.320225in}{4.495057in}}%
\pgfusepath{clip}%
\pgfsetbuttcap%
\pgfsetroundjoin%
\pgfsetlinewidth{0.000000pt}%
\definecolor{currentstroke}{rgb}{0.000000,0.000000,0.000000}%
\pgfsetstrokecolor{currentstroke}%
\pgfsetdash{}{0pt}%
\pgfpathmoveto{\pgfqpoint{7.998823in}{1.566605in}}%
\pgfpathlineto{\pgfqpoint{8.185049in}{1.566605in}}%
\pgfpathlineto{\pgfqpoint{8.185049in}{1.648334in}}%
\pgfpathlineto{\pgfqpoint{7.998823in}{1.648334in}}%
\pgfpathlineto{\pgfqpoint{7.998823in}{1.566605in}}%
\pgfusepath{}%
\end{pgfscope}%
\begin{pgfscope}%
\pgfpathrectangle{\pgfqpoint{0.549740in}{0.463273in}}{\pgfqpoint{9.320225in}{4.495057in}}%
\pgfusepath{clip}%
\pgfsetbuttcap%
\pgfsetroundjoin%
\pgfsetlinewidth{0.000000pt}%
\definecolor{currentstroke}{rgb}{0.000000,0.000000,0.000000}%
\pgfsetstrokecolor{currentstroke}%
\pgfsetdash{}{0pt}%
\pgfpathmoveto{\pgfqpoint{8.185049in}{1.566605in}}%
\pgfpathlineto{\pgfqpoint{8.371276in}{1.566605in}}%
\pgfpathlineto{\pgfqpoint{8.371276in}{1.648334in}}%
\pgfpathlineto{\pgfqpoint{8.185049in}{1.648334in}}%
\pgfpathlineto{\pgfqpoint{8.185049in}{1.566605in}}%
\pgfusepath{}%
\end{pgfscope}%
\begin{pgfscope}%
\pgfpathrectangle{\pgfqpoint{0.549740in}{0.463273in}}{\pgfqpoint{9.320225in}{4.495057in}}%
\pgfusepath{clip}%
\pgfsetbuttcap%
\pgfsetroundjoin%
\pgfsetlinewidth{0.000000pt}%
\definecolor{currentstroke}{rgb}{0.000000,0.000000,0.000000}%
\pgfsetstrokecolor{currentstroke}%
\pgfsetdash{}{0pt}%
\pgfpathmoveto{\pgfqpoint{8.371276in}{1.566605in}}%
\pgfpathlineto{\pgfqpoint{8.557503in}{1.566605in}}%
\pgfpathlineto{\pgfqpoint{8.557503in}{1.648334in}}%
\pgfpathlineto{\pgfqpoint{8.371276in}{1.648334in}}%
\pgfpathlineto{\pgfqpoint{8.371276in}{1.566605in}}%
\pgfusepath{}%
\end{pgfscope}%
\begin{pgfscope}%
\pgfpathrectangle{\pgfqpoint{0.549740in}{0.463273in}}{\pgfqpoint{9.320225in}{4.495057in}}%
\pgfusepath{clip}%
\pgfsetbuttcap%
\pgfsetroundjoin%
\definecolor{currentfill}{rgb}{0.614330,0.761948,0.940009}%
\pgfsetfillcolor{currentfill}%
\pgfsetlinewidth{0.000000pt}%
\definecolor{currentstroke}{rgb}{0.000000,0.000000,0.000000}%
\pgfsetstrokecolor{currentstroke}%
\pgfsetdash{}{0pt}%
\pgfpathmoveto{\pgfqpoint{8.557503in}{1.566605in}}%
\pgfpathlineto{\pgfqpoint{8.743729in}{1.566605in}}%
\pgfpathlineto{\pgfqpoint{8.743729in}{1.648334in}}%
\pgfpathlineto{\pgfqpoint{8.557503in}{1.648334in}}%
\pgfpathlineto{\pgfqpoint{8.557503in}{1.566605in}}%
\pgfusepath{fill}%
\end{pgfscope}%
\begin{pgfscope}%
\pgfpathrectangle{\pgfqpoint{0.549740in}{0.463273in}}{\pgfqpoint{9.320225in}{4.495057in}}%
\pgfusepath{clip}%
\pgfsetbuttcap%
\pgfsetroundjoin%
\definecolor{currentfill}{rgb}{0.472869,0.711325,0.955316}%
\pgfsetfillcolor{currentfill}%
\pgfsetlinewidth{0.000000pt}%
\definecolor{currentstroke}{rgb}{0.000000,0.000000,0.000000}%
\pgfsetstrokecolor{currentstroke}%
\pgfsetdash{}{0pt}%
\pgfpathmoveto{\pgfqpoint{8.743729in}{1.566605in}}%
\pgfpathlineto{\pgfqpoint{8.929956in}{1.566605in}}%
\pgfpathlineto{\pgfqpoint{8.929956in}{1.648334in}}%
\pgfpathlineto{\pgfqpoint{8.743729in}{1.648334in}}%
\pgfpathlineto{\pgfqpoint{8.743729in}{1.566605in}}%
\pgfusepath{fill}%
\end{pgfscope}%
\begin{pgfscope}%
\pgfpathrectangle{\pgfqpoint{0.549740in}{0.463273in}}{\pgfqpoint{9.320225in}{4.495057in}}%
\pgfusepath{clip}%
\pgfsetbuttcap%
\pgfsetroundjoin%
\pgfsetlinewidth{0.000000pt}%
\definecolor{currentstroke}{rgb}{0.000000,0.000000,0.000000}%
\pgfsetstrokecolor{currentstroke}%
\pgfsetdash{}{0pt}%
\pgfpathmoveto{\pgfqpoint{8.929956in}{1.566605in}}%
\pgfpathlineto{\pgfqpoint{9.116182in}{1.566605in}}%
\pgfpathlineto{\pgfqpoint{9.116182in}{1.648334in}}%
\pgfpathlineto{\pgfqpoint{8.929956in}{1.648334in}}%
\pgfpathlineto{\pgfqpoint{8.929956in}{1.566605in}}%
\pgfusepath{}%
\end{pgfscope}%
\begin{pgfscope}%
\pgfpathrectangle{\pgfqpoint{0.549740in}{0.463273in}}{\pgfqpoint{9.320225in}{4.495057in}}%
\pgfusepath{clip}%
\pgfsetbuttcap%
\pgfsetroundjoin%
\pgfsetlinewidth{0.000000pt}%
\definecolor{currentstroke}{rgb}{0.000000,0.000000,0.000000}%
\pgfsetstrokecolor{currentstroke}%
\pgfsetdash{}{0pt}%
\pgfpathmoveto{\pgfqpoint{9.116182in}{1.566605in}}%
\pgfpathlineto{\pgfqpoint{9.302409in}{1.566605in}}%
\pgfpathlineto{\pgfqpoint{9.302409in}{1.648334in}}%
\pgfpathlineto{\pgfqpoint{9.116182in}{1.648334in}}%
\pgfpathlineto{\pgfqpoint{9.116182in}{1.566605in}}%
\pgfusepath{}%
\end{pgfscope}%
\begin{pgfscope}%
\pgfpathrectangle{\pgfqpoint{0.549740in}{0.463273in}}{\pgfqpoint{9.320225in}{4.495057in}}%
\pgfusepath{clip}%
\pgfsetbuttcap%
\pgfsetroundjoin%
\pgfsetlinewidth{0.000000pt}%
\definecolor{currentstroke}{rgb}{0.000000,0.000000,0.000000}%
\pgfsetstrokecolor{currentstroke}%
\pgfsetdash{}{0pt}%
\pgfpathmoveto{\pgfqpoint{9.302409in}{1.566605in}}%
\pgfpathlineto{\pgfqpoint{9.488635in}{1.566605in}}%
\pgfpathlineto{\pgfqpoint{9.488635in}{1.648334in}}%
\pgfpathlineto{\pgfqpoint{9.302409in}{1.648334in}}%
\pgfpathlineto{\pgfqpoint{9.302409in}{1.566605in}}%
\pgfusepath{}%
\end{pgfscope}%
\begin{pgfscope}%
\pgfpathrectangle{\pgfqpoint{0.549740in}{0.463273in}}{\pgfqpoint{9.320225in}{4.495057in}}%
\pgfusepath{clip}%
\pgfsetbuttcap%
\pgfsetroundjoin%
\pgfsetlinewidth{0.000000pt}%
\definecolor{currentstroke}{rgb}{0.000000,0.000000,0.000000}%
\pgfsetstrokecolor{currentstroke}%
\pgfsetdash{}{0pt}%
\pgfpathmoveto{\pgfqpoint{9.488635in}{1.566605in}}%
\pgfpathlineto{\pgfqpoint{9.674862in}{1.566605in}}%
\pgfpathlineto{\pgfqpoint{9.674862in}{1.648334in}}%
\pgfpathlineto{\pgfqpoint{9.488635in}{1.648334in}}%
\pgfpathlineto{\pgfqpoint{9.488635in}{1.566605in}}%
\pgfusepath{}%
\end{pgfscope}%
\begin{pgfscope}%
\pgfpathrectangle{\pgfqpoint{0.549740in}{0.463273in}}{\pgfqpoint{9.320225in}{4.495057in}}%
\pgfusepath{clip}%
\pgfsetbuttcap%
\pgfsetroundjoin%
\pgfsetlinewidth{0.000000pt}%
\definecolor{currentstroke}{rgb}{0.000000,0.000000,0.000000}%
\pgfsetstrokecolor{currentstroke}%
\pgfsetdash{}{0pt}%
\pgfpathmoveto{\pgfqpoint{9.674862in}{1.566605in}}%
\pgfpathlineto{\pgfqpoint{9.861088in}{1.566605in}}%
\pgfpathlineto{\pgfqpoint{9.861088in}{1.648334in}}%
\pgfpathlineto{\pgfqpoint{9.674862in}{1.648334in}}%
\pgfpathlineto{\pgfqpoint{9.674862in}{1.566605in}}%
\pgfusepath{}%
\end{pgfscope}%
\begin{pgfscope}%
\pgfpathrectangle{\pgfqpoint{0.549740in}{0.463273in}}{\pgfqpoint{9.320225in}{4.495057in}}%
\pgfusepath{clip}%
\pgfsetbuttcap%
\pgfsetroundjoin%
\pgfsetlinewidth{0.000000pt}%
\definecolor{currentstroke}{rgb}{0.000000,0.000000,0.000000}%
\pgfsetstrokecolor{currentstroke}%
\pgfsetdash{}{0pt}%
\pgfpathmoveto{\pgfqpoint{0.549761in}{1.648334in}}%
\pgfpathlineto{\pgfqpoint{0.735988in}{1.648334in}}%
\pgfpathlineto{\pgfqpoint{0.735988in}{1.730062in}}%
\pgfpathlineto{\pgfqpoint{0.549761in}{1.730062in}}%
\pgfpathlineto{\pgfqpoint{0.549761in}{1.648334in}}%
\pgfusepath{}%
\end{pgfscope}%
\begin{pgfscope}%
\pgfpathrectangle{\pgfqpoint{0.549740in}{0.463273in}}{\pgfqpoint{9.320225in}{4.495057in}}%
\pgfusepath{clip}%
\pgfsetbuttcap%
\pgfsetroundjoin%
\pgfsetlinewidth{0.000000pt}%
\definecolor{currentstroke}{rgb}{0.000000,0.000000,0.000000}%
\pgfsetstrokecolor{currentstroke}%
\pgfsetdash{}{0pt}%
\pgfpathmoveto{\pgfqpoint{0.735988in}{1.648334in}}%
\pgfpathlineto{\pgfqpoint{0.922214in}{1.648334in}}%
\pgfpathlineto{\pgfqpoint{0.922214in}{1.730062in}}%
\pgfpathlineto{\pgfqpoint{0.735988in}{1.730062in}}%
\pgfpathlineto{\pgfqpoint{0.735988in}{1.648334in}}%
\pgfusepath{}%
\end{pgfscope}%
\begin{pgfscope}%
\pgfpathrectangle{\pgfqpoint{0.549740in}{0.463273in}}{\pgfqpoint{9.320225in}{4.495057in}}%
\pgfusepath{clip}%
\pgfsetbuttcap%
\pgfsetroundjoin%
\pgfsetlinewidth{0.000000pt}%
\definecolor{currentstroke}{rgb}{0.000000,0.000000,0.000000}%
\pgfsetstrokecolor{currentstroke}%
\pgfsetdash{}{0pt}%
\pgfpathmoveto{\pgfqpoint{0.922214in}{1.648334in}}%
\pgfpathlineto{\pgfqpoint{1.108441in}{1.648334in}}%
\pgfpathlineto{\pgfqpoint{1.108441in}{1.730062in}}%
\pgfpathlineto{\pgfqpoint{0.922214in}{1.730062in}}%
\pgfpathlineto{\pgfqpoint{0.922214in}{1.648334in}}%
\pgfusepath{}%
\end{pgfscope}%
\begin{pgfscope}%
\pgfpathrectangle{\pgfqpoint{0.549740in}{0.463273in}}{\pgfqpoint{9.320225in}{4.495057in}}%
\pgfusepath{clip}%
\pgfsetbuttcap%
\pgfsetroundjoin%
\pgfsetlinewidth{0.000000pt}%
\definecolor{currentstroke}{rgb}{0.000000,0.000000,0.000000}%
\pgfsetstrokecolor{currentstroke}%
\pgfsetdash{}{0pt}%
\pgfpathmoveto{\pgfqpoint{1.108441in}{1.648334in}}%
\pgfpathlineto{\pgfqpoint{1.294667in}{1.648334in}}%
\pgfpathlineto{\pgfqpoint{1.294667in}{1.730062in}}%
\pgfpathlineto{\pgfqpoint{1.108441in}{1.730062in}}%
\pgfpathlineto{\pgfqpoint{1.108441in}{1.648334in}}%
\pgfusepath{}%
\end{pgfscope}%
\begin{pgfscope}%
\pgfpathrectangle{\pgfqpoint{0.549740in}{0.463273in}}{\pgfqpoint{9.320225in}{4.495057in}}%
\pgfusepath{clip}%
\pgfsetbuttcap%
\pgfsetroundjoin%
\pgfsetlinewidth{0.000000pt}%
\definecolor{currentstroke}{rgb}{0.000000,0.000000,0.000000}%
\pgfsetstrokecolor{currentstroke}%
\pgfsetdash{}{0pt}%
\pgfpathmoveto{\pgfqpoint{1.294667in}{1.648334in}}%
\pgfpathlineto{\pgfqpoint{1.480894in}{1.648334in}}%
\pgfpathlineto{\pgfqpoint{1.480894in}{1.730062in}}%
\pgfpathlineto{\pgfqpoint{1.294667in}{1.730062in}}%
\pgfpathlineto{\pgfqpoint{1.294667in}{1.648334in}}%
\pgfusepath{}%
\end{pgfscope}%
\begin{pgfscope}%
\pgfpathrectangle{\pgfqpoint{0.549740in}{0.463273in}}{\pgfqpoint{9.320225in}{4.495057in}}%
\pgfusepath{clip}%
\pgfsetbuttcap%
\pgfsetroundjoin%
\pgfsetlinewidth{0.000000pt}%
\definecolor{currentstroke}{rgb}{0.000000,0.000000,0.000000}%
\pgfsetstrokecolor{currentstroke}%
\pgfsetdash{}{0pt}%
\pgfpathmoveto{\pgfqpoint{1.480894in}{1.648334in}}%
\pgfpathlineto{\pgfqpoint{1.667120in}{1.648334in}}%
\pgfpathlineto{\pgfqpoint{1.667120in}{1.730062in}}%
\pgfpathlineto{\pgfqpoint{1.480894in}{1.730062in}}%
\pgfpathlineto{\pgfqpoint{1.480894in}{1.648334in}}%
\pgfusepath{}%
\end{pgfscope}%
\begin{pgfscope}%
\pgfpathrectangle{\pgfqpoint{0.549740in}{0.463273in}}{\pgfqpoint{9.320225in}{4.495057in}}%
\pgfusepath{clip}%
\pgfsetbuttcap%
\pgfsetroundjoin%
\pgfsetlinewidth{0.000000pt}%
\definecolor{currentstroke}{rgb}{0.000000,0.000000,0.000000}%
\pgfsetstrokecolor{currentstroke}%
\pgfsetdash{}{0pt}%
\pgfpathmoveto{\pgfqpoint{1.667120in}{1.648334in}}%
\pgfpathlineto{\pgfqpoint{1.853347in}{1.648334in}}%
\pgfpathlineto{\pgfqpoint{1.853347in}{1.730062in}}%
\pgfpathlineto{\pgfqpoint{1.667120in}{1.730062in}}%
\pgfpathlineto{\pgfqpoint{1.667120in}{1.648334in}}%
\pgfusepath{}%
\end{pgfscope}%
\begin{pgfscope}%
\pgfpathrectangle{\pgfqpoint{0.549740in}{0.463273in}}{\pgfqpoint{9.320225in}{4.495057in}}%
\pgfusepath{clip}%
\pgfsetbuttcap%
\pgfsetroundjoin%
\pgfsetlinewidth{0.000000pt}%
\definecolor{currentstroke}{rgb}{0.000000,0.000000,0.000000}%
\pgfsetstrokecolor{currentstroke}%
\pgfsetdash{}{0pt}%
\pgfpathmoveto{\pgfqpoint{1.853347in}{1.648334in}}%
\pgfpathlineto{\pgfqpoint{2.039573in}{1.648334in}}%
\pgfpathlineto{\pgfqpoint{2.039573in}{1.730062in}}%
\pgfpathlineto{\pgfqpoint{1.853347in}{1.730062in}}%
\pgfpathlineto{\pgfqpoint{1.853347in}{1.648334in}}%
\pgfusepath{}%
\end{pgfscope}%
\begin{pgfscope}%
\pgfpathrectangle{\pgfqpoint{0.549740in}{0.463273in}}{\pgfqpoint{9.320225in}{4.495057in}}%
\pgfusepath{clip}%
\pgfsetbuttcap%
\pgfsetroundjoin%
\pgfsetlinewidth{0.000000pt}%
\definecolor{currentstroke}{rgb}{0.000000,0.000000,0.000000}%
\pgfsetstrokecolor{currentstroke}%
\pgfsetdash{}{0pt}%
\pgfpathmoveto{\pgfqpoint{2.039573in}{1.648334in}}%
\pgfpathlineto{\pgfqpoint{2.225800in}{1.648334in}}%
\pgfpathlineto{\pgfqpoint{2.225800in}{1.730062in}}%
\pgfpathlineto{\pgfqpoint{2.039573in}{1.730062in}}%
\pgfpathlineto{\pgfqpoint{2.039573in}{1.648334in}}%
\pgfusepath{}%
\end{pgfscope}%
\begin{pgfscope}%
\pgfpathrectangle{\pgfqpoint{0.549740in}{0.463273in}}{\pgfqpoint{9.320225in}{4.495057in}}%
\pgfusepath{clip}%
\pgfsetbuttcap%
\pgfsetroundjoin%
\pgfsetlinewidth{0.000000pt}%
\definecolor{currentstroke}{rgb}{0.000000,0.000000,0.000000}%
\pgfsetstrokecolor{currentstroke}%
\pgfsetdash{}{0pt}%
\pgfpathmoveto{\pgfqpoint{2.225800in}{1.648334in}}%
\pgfpathlineto{\pgfqpoint{2.412027in}{1.648334in}}%
\pgfpathlineto{\pgfqpoint{2.412027in}{1.730062in}}%
\pgfpathlineto{\pgfqpoint{2.225800in}{1.730062in}}%
\pgfpathlineto{\pgfqpoint{2.225800in}{1.648334in}}%
\pgfusepath{}%
\end{pgfscope}%
\begin{pgfscope}%
\pgfpathrectangle{\pgfqpoint{0.549740in}{0.463273in}}{\pgfqpoint{9.320225in}{4.495057in}}%
\pgfusepath{clip}%
\pgfsetbuttcap%
\pgfsetroundjoin%
\pgfsetlinewidth{0.000000pt}%
\definecolor{currentstroke}{rgb}{0.000000,0.000000,0.000000}%
\pgfsetstrokecolor{currentstroke}%
\pgfsetdash{}{0pt}%
\pgfpathmoveto{\pgfqpoint{2.412027in}{1.648334in}}%
\pgfpathlineto{\pgfqpoint{2.598253in}{1.648334in}}%
\pgfpathlineto{\pgfqpoint{2.598253in}{1.730062in}}%
\pgfpathlineto{\pgfqpoint{2.412027in}{1.730062in}}%
\pgfpathlineto{\pgfqpoint{2.412027in}{1.648334in}}%
\pgfusepath{}%
\end{pgfscope}%
\begin{pgfscope}%
\pgfpathrectangle{\pgfqpoint{0.549740in}{0.463273in}}{\pgfqpoint{9.320225in}{4.495057in}}%
\pgfusepath{clip}%
\pgfsetbuttcap%
\pgfsetroundjoin%
\pgfsetlinewidth{0.000000pt}%
\definecolor{currentstroke}{rgb}{0.000000,0.000000,0.000000}%
\pgfsetstrokecolor{currentstroke}%
\pgfsetdash{}{0pt}%
\pgfpathmoveto{\pgfqpoint{2.598253in}{1.648334in}}%
\pgfpathlineto{\pgfqpoint{2.784480in}{1.648334in}}%
\pgfpathlineto{\pgfqpoint{2.784480in}{1.730062in}}%
\pgfpathlineto{\pgfqpoint{2.598253in}{1.730062in}}%
\pgfpathlineto{\pgfqpoint{2.598253in}{1.648334in}}%
\pgfusepath{}%
\end{pgfscope}%
\begin{pgfscope}%
\pgfpathrectangle{\pgfqpoint{0.549740in}{0.463273in}}{\pgfqpoint{9.320225in}{4.495057in}}%
\pgfusepath{clip}%
\pgfsetbuttcap%
\pgfsetroundjoin%
\pgfsetlinewidth{0.000000pt}%
\definecolor{currentstroke}{rgb}{0.000000,0.000000,0.000000}%
\pgfsetstrokecolor{currentstroke}%
\pgfsetdash{}{0pt}%
\pgfpathmoveto{\pgfqpoint{2.784480in}{1.648334in}}%
\pgfpathlineto{\pgfqpoint{2.970706in}{1.648334in}}%
\pgfpathlineto{\pgfqpoint{2.970706in}{1.730062in}}%
\pgfpathlineto{\pgfqpoint{2.784480in}{1.730062in}}%
\pgfpathlineto{\pgfqpoint{2.784480in}{1.648334in}}%
\pgfusepath{}%
\end{pgfscope}%
\begin{pgfscope}%
\pgfpathrectangle{\pgfqpoint{0.549740in}{0.463273in}}{\pgfqpoint{9.320225in}{4.495057in}}%
\pgfusepath{clip}%
\pgfsetbuttcap%
\pgfsetroundjoin%
\pgfsetlinewidth{0.000000pt}%
\definecolor{currentstroke}{rgb}{0.000000,0.000000,0.000000}%
\pgfsetstrokecolor{currentstroke}%
\pgfsetdash{}{0pt}%
\pgfpathmoveto{\pgfqpoint{2.970706in}{1.648334in}}%
\pgfpathlineto{\pgfqpoint{3.156933in}{1.648334in}}%
\pgfpathlineto{\pgfqpoint{3.156933in}{1.730062in}}%
\pgfpathlineto{\pgfqpoint{2.970706in}{1.730062in}}%
\pgfpathlineto{\pgfqpoint{2.970706in}{1.648334in}}%
\pgfusepath{}%
\end{pgfscope}%
\begin{pgfscope}%
\pgfpathrectangle{\pgfqpoint{0.549740in}{0.463273in}}{\pgfqpoint{9.320225in}{4.495057in}}%
\pgfusepath{clip}%
\pgfsetbuttcap%
\pgfsetroundjoin%
\pgfsetlinewidth{0.000000pt}%
\definecolor{currentstroke}{rgb}{0.000000,0.000000,0.000000}%
\pgfsetstrokecolor{currentstroke}%
\pgfsetdash{}{0pt}%
\pgfpathmoveto{\pgfqpoint{3.156933in}{1.648334in}}%
\pgfpathlineto{\pgfqpoint{3.343159in}{1.648334in}}%
\pgfpathlineto{\pgfqpoint{3.343159in}{1.730062in}}%
\pgfpathlineto{\pgfqpoint{3.156933in}{1.730062in}}%
\pgfpathlineto{\pgfqpoint{3.156933in}{1.648334in}}%
\pgfusepath{}%
\end{pgfscope}%
\begin{pgfscope}%
\pgfpathrectangle{\pgfqpoint{0.549740in}{0.463273in}}{\pgfqpoint{9.320225in}{4.495057in}}%
\pgfusepath{clip}%
\pgfsetbuttcap%
\pgfsetroundjoin%
\pgfsetlinewidth{0.000000pt}%
\definecolor{currentstroke}{rgb}{0.000000,0.000000,0.000000}%
\pgfsetstrokecolor{currentstroke}%
\pgfsetdash{}{0pt}%
\pgfpathmoveto{\pgfqpoint{3.343159in}{1.648334in}}%
\pgfpathlineto{\pgfqpoint{3.529386in}{1.648334in}}%
\pgfpathlineto{\pgfqpoint{3.529386in}{1.730062in}}%
\pgfpathlineto{\pgfqpoint{3.343159in}{1.730062in}}%
\pgfpathlineto{\pgfqpoint{3.343159in}{1.648334in}}%
\pgfusepath{}%
\end{pgfscope}%
\begin{pgfscope}%
\pgfpathrectangle{\pgfqpoint{0.549740in}{0.463273in}}{\pgfqpoint{9.320225in}{4.495057in}}%
\pgfusepath{clip}%
\pgfsetbuttcap%
\pgfsetroundjoin%
\pgfsetlinewidth{0.000000pt}%
\definecolor{currentstroke}{rgb}{0.000000,0.000000,0.000000}%
\pgfsetstrokecolor{currentstroke}%
\pgfsetdash{}{0pt}%
\pgfpathmoveto{\pgfqpoint{3.529386in}{1.648334in}}%
\pgfpathlineto{\pgfqpoint{3.715612in}{1.648334in}}%
\pgfpathlineto{\pgfqpoint{3.715612in}{1.730062in}}%
\pgfpathlineto{\pgfqpoint{3.529386in}{1.730062in}}%
\pgfpathlineto{\pgfqpoint{3.529386in}{1.648334in}}%
\pgfusepath{}%
\end{pgfscope}%
\begin{pgfscope}%
\pgfpathrectangle{\pgfqpoint{0.549740in}{0.463273in}}{\pgfqpoint{9.320225in}{4.495057in}}%
\pgfusepath{clip}%
\pgfsetbuttcap%
\pgfsetroundjoin%
\pgfsetlinewidth{0.000000pt}%
\definecolor{currentstroke}{rgb}{0.000000,0.000000,0.000000}%
\pgfsetstrokecolor{currentstroke}%
\pgfsetdash{}{0pt}%
\pgfpathmoveto{\pgfqpoint{3.715612in}{1.648334in}}%
\pgfpathlineto{\pgfqpoint{3.901839in}{1.648334in}}%
\pgfpathlineto{\pgfqpoint{3.901839in}{1.730062in}}%
\pgfpathlineto{\pgfqpoint{3.715612in}{1.730062in}}%
\pgfpathlineto{\pgfqpoint{3.715612in}{1.648334in}}%
\pgfusepath{}%
\end{pgfscope}%
\begin{pgfscope}%
\pgfpathrectangle{\pgfqpoint{0.549740in}{0.463273in}}{\pgfqpoint{9.320225in}{4.495057in}}%
\pgfusepath{clip}%
\pgfsetbuttcap%
\pgfsetroundjoin%
\pgfsetlinewidth{0.000000pt}%
\definecolor{currentstroke}{rgb}{0.000000,0.000000,0.000000}%
\pgfsetstrokecolor{currentstroke}%
\pgfsetdash{}{0pt}%
\pgfpathmoveto{\pgfqpoint{3.901839in}{1.648334in}}%
\pgfpathlineto{\pgfqpoint{4.088065in}{1.648334in}}%
\pgfpathlineto{\pgfqpoint{4.088065in}{1.730062in}}%
\pgfpathlineto{\pgfqpoint{3.901839in}{1.730062in}}%
\pgfpathlineto{\pgfqpoint{3.901839in}{1.648334in}}%
\pgfusepath{}%
\end{pgfscope}%
\begin{pgfscope}%
\pgfpathrectangle{\pgfqpoint{0.549740in}{0.463273in}}{\pgfqpoint{9.320225in}{4.495057in}}%
\pgfusepath{clip}%
\pgfsetbuttcap%
\pgfsetroundjoin%
\pgfsetlinewidth{0.000000pt}%
\definecolor{currentstroke}{rgb}{0.000000,0.000000,0.000000}%
\pgfsetstrokecolor{currentstroke}%
\pgfsetdash{}{0pt}%
\pgfpathmoveto{\pgfqpoint{4.088065in}{1.648334in}}%
\pgfpathlineto{\pgfqpoint{4.274292in}{1.648334in}}%
\pgfpathlineto{\pgfqpoint{4.274292in}{1.730062in}}%
\pgfpathlineto{\pgfqpoint{4.088065in}{1.730062in}}%
\pgfpathlineto{\pgfqpoint{4.088065in}{1.648334in}}%
\pgfusepath{}%
\end{pgfscope}%
\begin{pgfscope}%
\pgfpathrectangle{\pgfqpoint{0.549740in}{0.463273in}}{\pgfqpoint{9.320225in}{4.495057in}}%
\pgfusepath{clip}%
\pgfsetbuttcap%
\pgfsetroundjoin%
\pgfsetlinewidth{0.000000pt}%
\definecolor{currentstroke}{rgb}{0.000000,0.000000,0.000000}%
\pgfsetstrokecolor{currentstroke}%
\pgfsetdash{}{0pt}%
\pgfpathmoveto{\pgfqpoint{4.274292in}{1.648334in}}%
\pgfpathlineto{\pgfqpoint{4.460519in}{1.648334in}}%
\pgfpathlineto{\pgfqpoint{4.460519in}{1.730062in}}%
\pgfpathlineto{\pgfqpoint{4.274292in}{1.730062in}}%
\pgfpathlineto{\pgfqpoint{4.274292in}{1.648334in}}%
\pgfusepath{}%
\end{pgfscope}%
\begin{pgfscope}%
\pgfpathrectangle{\pgfqpoint{0.549740in}{0.463273in}}{\pgfqpoint{9.320225in}{4.495057in}}%
\pgfusepath{clip}%
\pgfsetbuttcap%
\pgfsetroundjoin%
\pgfsetlinewidth{0.000000pt}%
\definecolor{currentstroke}{rgb}{0.000000,0.000000,0.000000}%
\pgfsetstrokecolor{currentstroke}%
\pgfsetdash{}{0pt}%
\pgfpathmoveto{\pgfqpoint{4.460519in}{1.648334in}}%
\pgfpathlineto{\pgfqpoint{4.646745in}{1.648334in}}%
\pgfpathlineto{\pgfqpoint{4.646745in}{1.730062in}}%
\pgfpathlineto{\pgfqpoint{4.460519in}{1.730062in}}%
\pgfpathlineto{\pgfqpoint{4.460519in}{1.648334in}}%
\pgfusepath{}%
\end{pgfscope}%
\begin{pgfscope}%
\pgfpathrectangle{\pgfqpoint{0.549740in}{0.463273in}}{\pgfqpoint{9.320225in}{4.495057in}}%
\pgfusepath{clip}%
\pgfsetbuttcap%
\pgfsetroundjoin%
\pgfsetlinewidth{0.000000pt}%
\definecolor{currentstroke}{rgb}{0.000000,0.000000,0.000000}%
\pgfsetstrokecolor{currentstroke}%
\pgfsetdash{}{0pt}%
\pgfpathmoveto{\pgfqpoint{4.646745in}{1.648334in}}%
\pgfpathlineto{\pgfqpoint{4.832972in}{1.648334in}}%
\pgfpathlineto{\pgfqpoint{4.832972in}{1.730062in}}%
\pgfpathlineto{\pgfqpoint{4.646745in}{1.730062in}}%
\pgfpathlineto{\pgfqpoint{4.646745in}{1.648334in}}%
\pgfusepath{}%
\end{pgfscope}%
\begin{pgfscope}%
\pgfpathrectangle{\pgfqpoint{0.549740in}{0.463273in}}{\pgfqpoint{9.320225in}{4.495057in}}%
\pgfusepath{clip}%
\pgfsetbuttcap%
\pgfsetroundjoin%
\pgfsetlinewidth{0.000000pt}%
\definecolor{currentstroke}{rgb}{0.000000,0.000000,0.000000}%
\pgfsetstrokecolor{currentstroke}%
\pgfsetdash{}{0pt}%
\pgfpathmoveto{\pgfqpoint{4.832972in}{1.648334in}}%
\pgfpathlineto{\pgfqpoint{5.019198in}{1.648334in}}%
\pgfpathlineto{\pgfqpoint{5.019198in}{1.730062in}}%
\pgfpathlineto{\pgfqpoint{4.832972in}{1.730062in}}%
\pgfpathlineto{\pgfqpoint{4.832972in}{1.648334in}}%
\pgfusepath{}%
\end{pgfscope}%
\begin{pgfscope}%
\pgfpathrectangle{\pgfqpoint{0.549740in}{0.463273in}}{\pgfqpoint{9.320225in}{4.495057in}}%
\pgfusepath{clip}%
\pgfsetbuttcap%
\pgfsetroundjoin%
\pgfsetlinewidth{0.000000pt}%
\definecolor{currentstroke}{rgb}{0.000000,0.000000,0.000000}%
\pgfsetstrokecolor{currentstroke}%
\pgfsetdash{}{0pt}%
\pgfpathmoveto{\pgfqpoint{5.019198in}{1.648334in}}%
\pgfpathlineto{\pgfqpoint{5.205425in}{1.648334in}}%
\pgfpathlineto{\pgfqpoint{5.205425in}{1.730062in}}%
\pgfpathlineto{\pgfqpoint{5.019198in}{1.730062in}}%
\pgfpathlineto{\pgfqpoint{5.019198in}{1.648334in}}%
\pgfusepath{}%
\end{pgfscope}%
\begin{pgfscope}%
\pgfpathrectangle{\pgfqpoint{0.549740in}{0.463273in}}{\pgfqpoint{9.320225in}{4.495057in}}%
\pgfusepath{clip}%
\pgfsetbuttcap%
\pgfsetroundjoin%
\pgfsetlinewidth{0.000000pt}%
\definecolor{currentstroke}{rgb}{0.000000,0.000000,0.000000}%
\pgfsetstrokecolor{currentstroke}%
\pgfsetdash{}{0pt}%
\pgfpathmoveto{\pgfqpoint{5.205425in}{1.648334in}}%
\pgfpathlineto{\pgfqpoint{5.391651in}{1.648334in}}%
\pgfpathlineto{\pgfqpoint{5.391651in}{1.730062in}}%
\pgfpathlineto{\pgfqpoint{5.205425in}{1.730062in}}%
\pgfpathlineto{\pgfqpoint{5.205425in}{1.648334in}}%
\pgfusepath{}%
\end{pgfscope}%
\begin{pgfscope}%
\pgfpathrectangle{\pgfqpoint{0.549740in}{0.463273in}}{\pgfqpoint{9.320225in}{4.495057in}}%
\pgfusepath{clip}%
\pgfsetbuttcap%
\pgfsetroundjoin%
\pgfsetlinewidth{0.000000pt}%
\definecolor{currentstroke}{rgb}{0.000000,0.000000,0.000000}%
\pgfsetstrokecolor{currentstroke}%
\pgfsetdash{}{0pt}%
\pgfpathmoveto{\pgfqpoint{5.391651in}{1.648334in}}%
\pgfpathlineto{\pgfqpoint{5.577878in}{1.648334in}}%
\pgfpathlineto{\pgfqpoint{5.577878in}{1.730062in}}%
\pgfpathlineto{\pgfqpoint{5.391651in}{1.730062in}}%
\pgfpathlineto{\pgfqpoint{5.391651in}{1.648334in}}%
\pgfusepath{}%
\end{pgfscope}%
\begin{pgfscope}%
\pgfpathrectangle{\pgfqpoint{0.549740in}{0.463273in}}{\pgfqpoint{9.320225in}{4.495057in}}%
\pgfusepath{clip}%
\pgfsetbuttcap%
\pgfsetroundjoin%
\definecolor{currentfill}{rgb}{0.547810,0.736432,0.947518}%
\pgfsetfillcolor{currentfill}%
\pgfsetlinewidth{0.000000pt}%
\definecolor{currentstroke}{rgb}{0.000000,0.000000,0.000000}%
\pgfsetstrokecolor{currentstroke}%
\pgfsetdash{}{0pt}%
\pgfpathmoveto{\pgfqpoint{5.577878in}{1.648334in}}%
\pgfpathlineto{\pgfqpoint{5.764104in}{1.648334in}}%
\pgfpathlineto{\pgfqpoint{5.764104in}{1.730062in}}%
\pgfpathlineto{\pgfqpoint{5.577878in}{1.730062in}}%
\pgfpathlineto{\pgfqpoint{5.577878in}{1.648334in}}%
\pgfusepath{fill}%
\end{pgfscope}%
\begin{pgfscope}%
\pgfpathrectangle{\pgfqpoint{0.549740in}{0.463273in}}{\pgfqpoint{9.320225in}{4.495057in}}%
\pgfusepath{clip}%
\pgfsetbuttcap%
\pgfsetroundjoin%
\pgfsetlinewidth{0.000000pt}%
\definecolor{currentstroke}{rgb}{0.000000,0.000000,0.000000}%
\pgfsetstrokecolor{currentstroke}%
\pgfsetdash{}{0pt}%
\pgfpathmoveto{\pgfqpoint{5.764104in}{1.648334in}}%
\pgfpathlineto{\pgfqpoint{5.950331in}{1.648334in}}%
\pgfpathlineto{\pgfqpoint{5.950331in}{1.730062in}}%
\pgfpathlineto{\pgfqpoint{5.764104in}{1.730062in}}%
\pgfpathlineto{\pgfqpoint{5.764104in}{1.648334in}}%
\pgfusepath{}%
\end{pgfscope}%
\begin{pgfscope}%
\pgfpathrectangle{\pgfqpoint{0.549740in}{0.463273in}}{\pgfqpoint{9.320225in}{4.495057in}}%
\pgfusepath{clip}%
\pgfsetbuttcap%
\pgfsetroundjoin%
\pgfsetlinewidth{0.000000pt}%
\definecolor{currentstroke}{rgb}{0.000000,0.000000,0.000000}%
\pgfsetstrokecolor{currentstroke}%
\pgfsetdash{}{0pt}%
\pgfpathmoveto{\pgfqpoint{5.950331in}{1.648334in}}%
\pgfpathlineto{\pgfqpoint{6.136557in}{1.648334in}}%
\pgfpathlineto{\pgfqpoint{6.136557in}{1.730062in}}%
\pgfpathlineto{\pgfqpoint{5.950331in}{1.730062in}}%
\pgfpathlineto{\pgfqpoint{5.950331in}{1.648334in}}%
\pgfusepath{}%
\end{pgfscope}%
\begin{pgfscope}%
\pgfpathrectangle{\pgfqpoint{0.549740in}{0.463273in}}{\pgfqpoint{9.320225in}{4.495057in}}%
\pgfusepath{clip}%
\pgfsetbuttcap%
\pgfsetroundjoin%
\pgfsetlinewidth{0.000000pt}%
\definecolor{currentstroke}{rgb}{0.000000,0.000000,0.000000}%
\pgfsetstrokecolor{currentstroke}%
\pgfsetdash{}{0pt}%
\pgfpathmoveto{\pgfqpoint{6.136557in}{1.648334in}}%
\pgfpathlineto{\pgfqpoint{6.322784in}{1.648334in}}%
\pgfpathlineto{\pgfqpoint{6.322784in}{1.730062in}}%
\pgfpathlineto{\pgfqpoint{6.136557in}{1.730062in}}%
\pgfpathlineto{\pgfqpoint{6.136557in}{1.648334in}}%
\pgfusepath{}%
\end{pgfscope}%
\begin{pgfscope}%
\pgfpathrectangle{\pgfqpoint{0.549740in}{0.463273in}}{\pgfqpoint{9.320225in}{4.495057in}}%
\pgfusepath{clip}%
\pgfsetbuttcap%
\pgfsetroundjoin%
\pgfsetlinewidth{0.000000pt}%
\definecolor{currentstroke}{rgb}{0.000000,0.000000,0.000000}%
\pgfsetstrokecolor{currentstroke}%
\pgfsetdash{}{0pt}%
\pgfpathmoveto{\pgfqpoint{6.322784in}{1.648334in}}%
\pgfpathlineto{\pgfqpoint{6.509011in}{1.648334in}}%
\pgfpathlineto{\pgfqpoint{6.509011in}{1.730062in}}%
\pgfpathlineto{\pgfqpoint{6.322784in}{1.730062in}}%
\pgfpathlineto{\pgfqpoint{6.322784in}{1.648334in}}%
\pgfusepath{}%
\end{pgfscope}%
\begin{pgfscope}%
\pgfpathrectangle{\pgfqpoint{0.549740in}{0.463273in}}{\pgfqpoint{9.320225in}{4.495057in}}%
\pgfusepath{clip}%
\pgfsetbuttcap%
\pgfsetroundjoin%
\definecolor{currentfill}{rgb}{0.472869,0.711325,0.955316}%
\pgfsetfillcolor{currentfill}%
\pgfsetlinewidth{0.000000pt}%
\definecolor{currentstroke}{rgb}{0.000000,0.000000,0.000000}%
\pgfsetstrokecolor{currentstroke}%
\pgfsetdash{}{0pt}%
\pgfpathmoveto{\pgfqpoint{6.509011in}{1.648334in}}%
\pgfpathlineto{\pgfqpoint{6.695237in}{1.648334in}}%
\pgfpathlineto{\pgfqpoint{6.695237in}{1.730062in}}%
\pgfpathlineto{\pgfqpoint{6.509011in}{1.730062in}}%
\pgfpathlineto{\pgfqpoint{6.509011in}{1.648334in}}%
\pgfusepath{fill}%
\end{pgfscope}%
\begin{pgfscope}%
\pgfpathrectangle{\pgfqpoint{0.549740in}{0.463273in}}{\pgfqpoint{9.320225in}{4.495057in}}%
\pgfusepath{clip}%
\pgfsetbuttcap%
\pgfsetroundjoin%
\pgfsetlinewidth{0.000000pt}%
\definecolor{currentstroke}{rgb}{0.000000,0.000000,0.000000}%
\pgfsetstrokecolor{currentstroke}%
\pgfsetdash{}{0pt}%
\pgfpathmoveto{\pgfqpoint{6.695237in}{1.648334in}}%
\pgfpathlineto{\pgfqpoint{6.881464in}{1.648334in}}%
\pgfpathlineto{\pgfqpoint{6.881464in}{1.730062in}}%
\pgfpathlineto{\pgfqpoint{6.695237in}{1.730062in}}%
\pgfpathlineto{\pgfqpoint{6.695237in}{1.648334in}}%
\pgfusepath{}%
\end{pgfscope}%
\begin{pgfscope}%
\pgfpathrectangle{\pgfqpoint{0.549740in}{0.463273in}}{\pgfqpoint{9.320225in}{4.495057in}}%
\pgfusepath{clip}%
\pgfsetbuttcap%
\pgfsetroundjoin%
\pgfsetlinewidth{0.000000pt}%
\definecolor{currentstroke}{rgb}{0.000000,0.000000,0.000000}%
\pgfsetstrokecolor{currentstroke}%
\pgfsetdash{}{0pt}%
\pgfpathmoveto{\pgfqpoint{6.881464in}{1.648334in}}%
\pgfpathlineto{\pgfqpoint{7.067690in}{1.648334in}}%
\pgfpathlineto{\pgfqpoint{7.067690in}{1.730062in}}%
\pgfpathlineto{\pgfqpoint{6.881464in}{1.730062in}}%
\pgfpathlineto{\pgfqpoint{6.881464in}{1.648334in}}%
\pgfusepath{}%
\end{pgfscope}%
\begin{pgfscope}%
\pgfpathrectangle{\pgfqpoint{0.549740in}{0.463273in}}{\pgfqpoint{9.320225in}{4.495057in}}%
\pgfusepath{clip}%
\pgfsetbuttcap%
\pgfsetroundjoin%
\pgfsetlinewidth{0.000000pt}%
\definecolor{currentstroke}{rgb}{0.000000,0.000000,0.000000}%
\pgfsetstrokecolor{currentstroke}%
\pgfsetdash{}{0pt}%
\pgfpathmoveto{\pgfqpoint{7.067690in}{1.648334in}}%
\pgfpathlineto{\pgfqpoint{7.253917in}{1.648334in}}%
\pgfpathlineto{\pgfqpoint{7.253917in}{1.730062in}}%
\pgfpathlineto{\pgfqpoint{7.067690in}{1.730062in}}%
\pgfpathlineto{\pgfqpoint{7.067690in}{1.648334in}}%
\pgfusepath{}%
\end{pgfscope}%
\begin{pgfscope}%
\pgfpathrectangle{\pgfqpoint{0.549740in}{0.463273in}}{\pgfqpoint{9.320225in}{4.495057in}}%
\pgfusepath{clip}%
\pgfsetbuttcap%
\pgfsetroundjoin%
\pgfsetlinewidth{0.000000pt}%
\definecolor{currentstroke}{rgb}{0.000000,0.000000,0.000000}%
\pgfsetstrokecolor{currentstroke}%
\pgfsetdash{}{0pt}%
\pgfpathmoveto{\pgfqpoint{7.253917in}{1.648334in}}%
\pgfpathlineto{\pgfqpoint{7.440143in}{1.648334in}}%
\pgfpathlineto{\pgfqpoint{7.440143in}{1.730062in}}%
\pgfpathlineto{\pgfqpoint{7.253917in}{1.730062in}}%
\pgfpathlineto{\pgfqpoint{7.253917in}{1.648334in}}%
\pgfusepath{}%
\end{pgfscope}%
\begin{pgfscope}%
\pgfpathrectangle{\pgfqpoint{0.549740in}{0.463273in}}{\pgfqpoint{9.320225in}{4.495057in}}%
\pgfusepath{clip}%
\pgfsetbuttcap%
\pgfsetroundjoin%
\pgfsetlinewidth{0.000000pt}%
\definecolor{currentstroke}{rgb}{0.000000,0.000000,0.000000}%
\pgfsetstrokecolor{currentstroke}%
\pgfsetdash{}{0pt}%
\pgfpathmoveto{\pgfqpoint{7.440143in}{1.648334in}}%
\pgfpathlineto{\pgfqpoint{7.626370in}{1.648334in}}%
\pgfpathlineto{\pgfqpoint{7.626370in}{1.730062in}}%
\pgfpathlineto{\pgfqpoint{7.440143in}{1.730062in}}%
\pgfpathlineto{\pgfqpoint{7.440143in}{1.648334in}}%
\pgfusepath{}%
\end{pgfscope}%
\begin{pgfscope}%
\pgfpathrectangle{\pgfqpoint{0.549740in}{0.463273in}}{\pgfqpoint{9.320225in}{4.495057in}}%
\pgfusepath{clip}%
\pgfsetbuttcap%
\pgfsetroundjoin%
\definecolor{currentfill}{rgb}{0.273225,0.662144,0.968515}%
\pgfsetfillcolor{currentfill}%
\pgfsetlinewidth{0.000000pt}%
\definecolor{currentstroke}{rgb}{0.000000,0.000000,0.000000}%
\pgfsetstrokecolor{currentstroke}%
\pgfsetdash{}{0pt}%
\pgfpathmoveto{\pgfqpoint{7.626370in}{1.648334in}}%
\pgfpathlineto{\pgfqpoint{7.812596in}{1.648334in}}%
\pgfpathlineto{\pgfqpoint{7.812596in}{1.730062in}}%
\pgfpathlineto{\pgfqpoint{7.626370in}{1.730062in}}%
\pgfpathlineto{\pgfqpoint{7.626370in}{1.648334in}}%
\pgfusepath{fill}%
\end{pgfscope}%
\begin{pgfscope}%
\pgfpathrectangle{\pgfqpoint{0.549740in}{0.463273in}}{\pgfqpoint{9.320225in}{4.495057in}}%
\pgfusepath{clip}%
\pgfsetbuttcap%
\pgfsetroundjoin%
\pgfsetlinewidth{0.000000pt}%
\definecolor{currentstroke}{rgb}{0.000000,0.000000,0.000000}%
\pgfsetstrokecolor{currentstroke}%
\pgfsetdash{}{0pt}%
\pgfpathmoveto{\pgfqpoint{7.812596in}{1.648334in}}%
\pgfpathlineto{\pgfqpoint{7.998823in}{1.648334in}}%
\pgfpathlineto{\pgfqpoint{7.998823in}{1.730062in}}%
\pgfpathlineto{\pgfqpoint{7.812596in}{1.730062in}}%
\pgfpathlineto{\pgfqpoint{7.812596in}{1.648334in}}%
\pgfusepath{}%
\end{pgfscope}%
\begin{pgfscope}%
\pgfpathrectangle{\pgfqpoint{0.549740in}{0.463273in}}{\pgfqpoint{9.320225in}{4.495057in}}%
\pgfusepath{clip}%
\pgfsetbuttcap%
\pgfsetroundjoin%
\pgfsetlinewidth{0.000000pt}%
\definecolor{currentstroke}{rgb}{0.000000,0.000000,0.000000}%
\pgfsetstrokecolor{currentstroke}%
\pgfsetdash{}{0pt}%
\pgfpathmoveto{\pgfqpoint{7.998823in}{1.648334in}}%
\pgfpathlineto{\pgfqpoint{8.185049in}{1.648334in}}%
\pgfpathlineto{\pgfqpoint{8.185049in}{1.730062in}}%
\pgfpathlineto{\pgfqpoint{7.998823in}{1.730062in}}%
\pgfpathlineto{\pgfqpoint{7.998823in}{1.648334in}}%
\pgfusepath{}%
\end{pgfscope}%
\begin{pgfscope}%
\pgfpathrectangle{\pgfqpoint{0.549740in}{0.463273in}}{\pgfqpoint{9.320225in}{4.495057in}}%
\pgfusepath{clip}%
\pgfsetbuttcap%
\pgfsetroundjoin%
\pgfsetlinewidth{0.000000pt}%
\definecolor{currentstroke}{rgb}{0.000000,0.000000,0.000000}%
\pgfsetstrokecolor{currentstroke}%
\pgfsetdash{}{0pt}%
\pgfpathmoveto{\pgfqpoint{8.185049in}{1.648334in}}%
\pgfpathlineto{\pgfqpoint{8.371276in}{1.648334in}}%
\pgfpathlineto{\pgfqpoint{8.371276in}{1.730062in}}%
\pgfpathlineto{\pgfqpoint{8.185049in}{1.730062in}}%
\pgfpathlineto{\pgfqpoint{8.185049in}{1.648334in}}%
\pgfusepath{}%
\end{pgfscope}%
\begin{pgfscope}%
\pgfpathrectangle{\pgfqpoint{0.549740in}{0.463273in}}{\pgfqpoint{9.320225in}{4.495057in}}%
\pgfusepath{clip}%
\pgfsetbuttcap%
\pgfsetroundjoin%
\pgfsetlinewidth{0.000000pt}%
\definecolor{currentstroke}{rgb}{0.000000,0.000000,0.000000}%
\pgfsetstrokecolor{currentstroke}%
\pgfsetdash{}{0pt}%
\pgfpathmoveto{\pgfqpoint{8.371276in}{1.648334in}}%
\pgfpathlineto{\pgfqpoint{8.557503in}{1.648334in}}%
\pgfpathlineto{\pgfqpoint{8.557503in}{1.730062in}}%
\pgfpathlineto{\pgfqpoint{8.371276in}{1.730062in}}%
\pgfpathlineto{\pgfqpoint{8.371276in}{1.648334in}}%
\pgfusepath{}%
\end{pgfscope}%
\begin{pgfscope}%
\pgfpathrectangle{\pgfqpoint{0.549740in}{0.463273in}}{\pgfqpoint{9.320225in}{4.495057in}}%
\pgfusepath{clip}%
\pgfsetbuttcap%
\pgfsetroundjoin%
\definecolor{currentfill}{rgb}{0.472869,0.711325,0.955316}%
\pgfsetfillcolor{currentfill}%
\pgfsetlinewidth{0.000000pt}%
\definecolor{currentstroke}{rgb}{0.000000,0.000000,0.000000}%
\pgfsetstrokecolor{currentstroke}%
\pgfsetdash{}{0pt}%
\pgfpathmoveto{\pgfqpoint{8.557503in}{1.648334in}}%
\pgfpathlineto{\pgfqpoint{8.743729in}{1.648334in}}%
\pgfpathlineto{\pgfqpoint{8.743729in}{1.730062in}}%
\pgfpathlineto{\pgfqpoint{8.557503in}{1.730062in}}%
\pgfpathlineto{\pgfqpoint{8.557503in}{1.648334in}}%
\pgfusepath{fill}%
\end{pgfscope}%
\begin{pgfscope}%
\pgfpathrectangle{\pgfqpoint{0.549740in}{0.463273in}}{\pgfqpoint{9.320225in}{4.495057in}}%
\pgfusepath{clip}%
\pgfsetbuttcap%
\pgfsetroundjoin%
\definecolor{currentfill}{rgb}{0.472869,0.711325,0.955316}%
\pgfsetfillcolor{currentfill}%
\pgfsetlinewidth{0.000000pt}%
\definecolor{currentstroke}{rgb}{0.000000,0.000000,0.000000}%
\pgfsetstrokecolor{currentstroke}%
\pgfsetdash{}{0pt}%
\pgfpathmoveto{\pgfqpoint{8.743729in}{1.648334in}}%
\pgfpathlineto{\pgfqpoint{8.929956in}{1.648334in}}%
\pgfpathlineto{\pgfqpoint{8.929956in}{1.730062in}}%
\pgfpathlineto{\pgfqpoint{8.743729in}{1.730062in}}%
\pgfpathlineto{\pgfqpoint{8.743729in}{1.648334in}}%
\pgfusepath{fill}%
\end{pgfscope}%
\begin{pgfscope}%
\pgfpathrectangle{\pgfqpoint{0.549740in}{0.463273in}}{\pgfqpoint{9.320225in}{4.495057in}}%
\pgfusepath{clip}%
\pgfsetbuttcap%
\pgfsetroundjoin%
\pgfsetlinewidth{0.000000pt}%
\definecolor{currentstroke}{rgb}{0.000000,0.000000,0.000000}%
\pgfsetstrokecolor{currentstroke}%
\pgfsetdash{}{0pt}%
\pgfpathmoveto{\pgfqpoint{8.929956in}{1.648334in}}%
\pgfpathlineto{\pgfqpoint{9.116182in}{1.648334in}}%
\pgfpathlineto{\pgfqpoint{9.116182in}{1.730062in}}%
\pgfpathlineto{\pgfqpoint{8.929956in}{1.730062in}}%
\pgfpathlineto{\pgfqpoint{8.929956in}{1.648334in}}%
\pgfusepath{}%
\end{pgfscope}%
\begin{pgfscope}%
\pgfpathrectangle{\pgfqpoint{0.549740in}{0.463273in}}{\pgfqpoint{9.320225in}{4.495057in}}%
\pgfusepath{clip}%
\pgfsetbuttcap%
\pgfsetroundjoin%
\pgfsetlinewidth{0.000000pt}%
\definecolor{currentstroke}{rgb}{0.000000,0.000000,0.000000}%
\pgfsetstrokecolor{currentstroke}%
\pgfsetdash{}{0pt}%
\pgfpathmoveto{\pgfqpoint{9.116182in}{1.648334in}}%
\pgfpathlineto{\pgfqpoint{9.302409in}{1.648334in}}%
\pgfpathlineto{\pgfqpoint{9.302409in}{1.730062in}}%
\pgfpathlineto{\pgfqpoint{9.116182in}{1.730062in}}%
\pgfpathlineto{\pgfqpoint{9.116182in}{1.648334in}}%
\pgfusepath{}%
\end{pgfscope}%
\begin{pgfscope}%
\pgfpathrectangle{\pgfqpoint{0.549740in}{0.463273in}}{\pgfqpoint{9.320225in}{4.495057in}}%
\pgfusepath{clip}%
\pgfsetbuttcap%
\pgfsetroundjoin%
\pgfsetlinewidth{0.000000pt}%
\definecolor{currentstroke}{rgb}{0.000000,0.000000,0.000000}%
\pgfsetstrokecolor{currentstroke}%
\pgfsetdash{}{0pt}%
\pgfpathmoveto{\pgfqpoint{9.302409in}{1.648334in}}%
\pgfpathlineto{\pgfqpoint{9.488635in}{1.648334in}}%
\pgfpathlineto{\pgfqpoint{9.488635in}{1.730062in}}%
\pgfpathlineto{\pgfqpoint{9.302409in}{1.730062in}}%
\pgfpathlineto{\pgfqpoint{9.302409in}{1.648334in}}%
\pgfusepath{}%
\end{pgfscope}%
\begin{pgfscope}%
\pgfpathrectangle{\pgfqpoint{0.549740in}{0.463273in}}{\pgfqpoint{9.320225in}{4.495057in}}%
\pgfusepath{clip}%
\pgfsetbuttcap%
\pgfsetroundjoin%
\pgfsetlinewidth{0.000000pt}%
\definecolor{currentstroke}{rgb}{0.000000,0.000000,0.000000}%
\pgfsetstrokecolor{currentstroke}%
\pgfsetdash{}{0pt}%
\pgfpathmoveto{\pgfqpoint{9.488635in}{1.648334in}}%
\pgfpathlineto{\pgfqpoint{9.674862in}{1.648334in}}%
\pgfpathlineto{\pgfqpoint{9.674862in}{1.730062in}}%
\pgfpathlineto{\pgfqpoint{9.488635in}{1.730062in}}%
\pgfpathlineto{\pgfqpoint{9.488635in}{1.648334in}}%
\pgfusepath{}%
\end{pgfscope}%
\begin{pgfscope}%
\pgfpathrectangle{\pgfqpoint{0.549740in}{0.463273in}}{\pgfqpoint{9.320225in}{4.495057in}}%
\pgfusepath{clip}%
\pgfsetbuttcap%
\pgfsetroundjoin%
\pgfsetlinewidth{0.000000pt}%
\definecolor{currentstroke}{rgb}{0.000000,0.000000,0.000000}%
\pgfsetstrokecolor{currentstroke}%
\pgfsetdash{}{0pt}%
\pgfpathmoveto{\pgfqpoint{9.674862in}{1.648334in}}%
\pgfpathlineto{\pgfqpoint{9.861088in}{1.648334in}}%
\pgfpathlineto{\pgfqpoint{9.861088in}{1.730062in}}%
\pgfpathlineto{\pgfqpoint{9.674862in}{1.730062in}}%
\pgfpathlineto{\pgfqpoint{9.674862in}{1.648334in}}%
\pgfusepath{}%
\end{pgfscope}%
\begin{pgfscope}%
\pgfpathrectangle{\pgfqpoint{0.549740in}{0.463273in}}{\pgfqpoint{9.320225in}{4.495057in}}%
\pgfusepath{clip}%
\pgfsetbuttcap%
\pgfsetroundjoin%
\pgfsetlinewidth{0.000000pt}%
\definecolor{currentstroke}{rgb}{0.000000,0.000000,0.000000}%
\pgfsetstrokecolor{currentstroke}%
\pgfsetdash{}{0pt}%
\pgfpathmoveto{\pgfqpoint{0.549761in}{1.730062in}}%
\pgfpathlineto{\pgfqpoint{0.735988in}{1.730062in}}%
\pgfpathlineto{\pgfqpoint{0.735988in}{1.811790in}}%
\pgfpathlineto{\pgfqpoint{0.549761in}{1.811790in}}%
\pgfpathlineto{\pgfqpoint{0.549761in}{1.730062in}}%
\pgfusepath{}%
\end{pgfscope}%
\begin{pgfscope}%
\pgfpathrectangle{\pgfqpoint{0.549740in}{0.463273in}}{\pgfqpoint{9.320225in}{4.495057in}}%
\pgfusepath{clip}%
\pgfsetbuttcap%
\pgfsetroundjoin%
\pgfsetlinewidth{0.000000pt}%
\definecolor{currentstroke}{rgb}{0.000000,0.000000,0.000000}%
\pgfsetstrokecolor{currentstroke}%
\pgfsetdash{}{0pt}%
\pgfpathmoveto{\pgfqpoint{0.735988in}{1.730062in}}%
\pgfpathlineto{\pgfqpoint{0.922214in}{1.730062in}}%
\pgfpathlineto{\pgfqpoint{0.922214in}{1.811790in}}%
\pgfpathlineto{\pgfqpoint{0.735988in}{1.811790in}}%
\pgfpathlineto{\pgfqpoint{0.735988in}{1.730062in}}%
\pgfusepath{}%
\end{pgfscope}%
\begin{pgfscope}%
\pgfpathrectangle{\pgfqpoint{0.549740in}{0.463273in}}{\pgfqpoint{9.320225in}{4.495057in}}%
\pgfusepath{clip}%
\pgfsetbuttcap%
\pgfsetroundjoin%
\pgfsetlinewidth{0.000000pt}%
\definecolor{currentstroke}{rgb}{0.000000,0.000000,0.000000}%
\pgfsetstrokecolor{currentstroke}%
\pgfsetdash{}{0pt}%
\pgfpathmoveto{\pgfqpoint{0.922214in}{1.730062in}}%
\pgfpathlineto{\pgfqpoint{1.108441in}{1.730062in}}%
\pgfpathlineto{\pgfqpoint{1.108441in}{1.811790in}}%
\pgfpathlineto{\pgfqpoint{0.922214in}{1.811790in}}%
\pgfpathlineto{\pgfqpoint{0.922214in}{1.730062in}}%
\pgfusepath{}%
\end{pgfscope}%
\begin{pgfscope}%
\pgfpathrectangle{\pgfqpoint{0.549740in}{0.463273in}}{\pgfqpoint{9.320225in}{4.495057in}}%
\pgfusepath{clip}%
\pgfsetbuttcap%
\pgfsetroundjoin%
\pgfsetlinewidth{0.000000pt}%
\definecolor{currentstroke}{rgb}{0.000000,0.000000,0.000000}%
\pgfsetstrokecolor{currentstroke}%
\pgfsetdash{}{0pt}%
\pgfpathmoveto{\pgfqpoint{1.108441in}{1.730062in}}%
\pgfpathlineto{\pgfqpoint{1.294667in}{1.730062in}}%
\pgfpathlineto{\pgfqpoint{1.294667in}{1.811790in}}%
\pgfpathlineto{\pgfqpoint{1.108441in}{1.811790in}}%
\pgfpathlineto{\pgfqpoint{1.108441in}{1.730062in}}%
\pgfusepath{}%
\end{pgfscope}%
\begin{pgfscope}%
\pgfpathrectangle{\pgfqpoint{0.549740in}{0.463273in}}{\pgfqpoint{9.320225in}{4.495057in}}%
\pgfusepath{clip}%
\pgfsetbuttcap%
\pgfsetroundjoin%
\pgfsetlinewidth{0.000000pt}%
\definecolor{currentstroke}{rgb}{0.000000,0.000000,0.000000}%
\pgfsetstrokecolor{currentstroke}%
\pgfsetdash{}{0pt}%
\pgfpathmoveto{\pgfqpoint{1.294667in}{1.730062in}}%
\pgfpathlineto{\pgfqpoint{1.480894in}{1.730062in}}%
\pgfpathlineto{\pgfqpoint{1.480894in}{1.811790in}}%
\pgfpathlineto{\pgfqpoint{1.294667in}{1.811790in}}%
\pgfpathlineto{\pgfqpoint{1.294667in}{1.730062in}}%
\pgfusepath{}%
\end{pgfscope}%
\begin{pgfscope}%
\pgfpathrectangle{\pgfqpoint{0.549740in}{0.463273in}}{\pgfqpoint{9.320225in}{4.495057in}}%
\pgfusepath{clip}%
\pgfsetbuttcap%
\pgfsetroundjoin%
\pgfsetlinewidth{0.000000pt}%
\definecolor{currentstroke}{rgb}{0.000000,0.000000,0.000000}%
\pgfsetstrokecolor{currentstroke}%
\pgfsetdash{}{0pt}%
\pgfpathmoveto{\pgfqpoint{1.480894in}{1.730062in}}%
\pgfpathlineto{\pgfqpoint{1.667120in}{1.730062in}}%
\pgfpathlineto{\pgfqpoint{1.667120in}{1.811790in}}%
\pgfpathlineto{\pgfqpoint{1.480894in}{1.811790in}}%
\pgfpathlineto{\pgfqpoint{1.480894in}{1.730062in}}%
\pgfusepath{}%
\end{pgfscope}%
\begin{pgfscope}%
\pgfpathrectangle{\pgfqpoint{0.549740in}{0.463273in}}{\pgfqpoint{9.320225in}{4.495057in}}%
\pgfusepath{clip}%
\pgfsetbuttcap%
\pgfsetroundjoin%
\pgfsetlinewidth{0.000000pt}%
\definecolor{currentstroke}{rgb}{0.000000,0.000000,0.000000}%
\pgfsetstrokecolor{currentstroke}%
\pgfsetdash{}{0pt}%
\pgfpathmoveto{\pgfqpoint{1.667120in}{1.730062in}}%
\pgfpathlineto{\pgfqpoint{1.853347in}{1.730062in}}%
\pgfpathlineto{\pgfqpoint{1.853347in}{1.811790in}}%
\pgfpathlineto{\pgfqpoint{1.667120in}{1.811790in}}%
\pgfpathlineto{\pgfqpoint{1.667120in}{1.730062in}}%
\pgfusepath{}%
\end{pgfscope}%
\begin{pgfscope}%
\pgfpathrectangle{\pgfqpoint{0.549740in}{0.463273in}}{\pgfqpoint{9.320225in}{4.495057in}}%
\pgfusepath{clip}%
\pgfsetbuttcap%
\pgfsetroundjoin%
\pgfsetlinewidth{0.000000pt}%
\definecolor{currentstroke}{rgb}{0.000000,0.000000,0.000000}%
\pgfsetstrokecolor{currentstroke}%
\pgfsetdash{}{0pt}%
\pgfpathmoveto{\pgfqpoint{1.853347in}{1.730062in}}%
\pgfpathlineto{\pgfqpoint{2.039573in}{1.730062in}}%
\pgfpathlineto{\pgfqpoint{2.039573in}{1.811790in}}%
\pgfpathlineto{\pgfqpoint{1.853347in}{1.811790in}}%
\pgfpathlineto{\pgfqpoint{1.853347in}{1.730062in}}%
\pgfusepath{}%
\end{pgfscope}%
\begin{pgfscope}%
\pgfpathrectangle{\pgfqpoint{0.549740in}{0.463273in}}{\pgfqpoint{9.320225in}{4.495057in}}%
\pgfusepath{clip}%
\pgfsetbuttcap%
\pgfsetroundjoin%
\pgfsetlinewidth{0.000000pt}%
\definecolor{currentstroke}{rgb}{0.000000,0.000000,0.000000}%
\pgfsetstrokecolor{currentstroke}%
\pgfsetdash{}{0pt}%
\pgfpathmoveto{\pgfqpoint{2.039573in}{1.730062in}}%
\pgfpathlineto{\pgfqpoint{2.225800in}{1.730062in}}%
\pgfpathlineto{\pgfqpoint{2.225800in}{1.811790in}}%
\pgfpathlineto{\pgfqpoint{2.039573in}{1.811790in}}%
\pgfpathlineto{\pgfqpoint{2.039573in}{1.730062in}}%
\pgfusepath{}%
\end{pgfscope}%
\begin{pgfscope}%
\pgfpathrectangle{\pgfqpoint{0.549740in}{0.463273in}}{\pgfqpoint{9.320225in}{4.495057in}}%
\pgfusepath{clip}%
\pgfsetbuttcap%
\pgfsetroundjoin%
\pgfsetlinewidth{0.000000pt}%
\definecolor{currentstroke}{rgb}{0.000000,0.000000,0.000000}%
\pgfsetstrokecolor{currentstroke}%
\pgfsetdash{}{0pt}%
\pgfpathmoveto{\pgfqpoint{2.225800in}{1.730062in}}%
\pgfpathlineto{\pgfqpoint{2.412027in}{1.730062in}}%
\pgfpathlineto{\pgfqpoint{2.412027in}{1.811790in}}%
\pgfpathlineto{\pgfqpoint{2.225800in}{1.811790in}}%
\pgfpathlineto{\pgfqpoint{2.225800in}{1.730062in}}%
\pgfusepath{}%
\end{pgfscope}%
\begin{pgfscope}%
\pgfpathrectangle{\pgfqpoint{0.549740in}{0.463273in}}{\pgfqpoint{9.320225in}{4.495057in}}%
\pgfusepath{clip}%
\pgfsetbuttcap%
\pgfsetroundjoin%
\pgfsetlinewidth{0.000000pt}%
\definecolor{currentstroke}{rgb}{0.000000,0.000000,0.000000}%
\pgfsetstrokecolor{currentstroke}%
\pgfsetdash{}{0pt}%
\pgfpathmoveto{\pgfqpoint{2.412027in}{1.730062in}}%
\pgfpathlineto{\pgfqpoint{2.598253in}{1.730062in}}%
\pgfpathlineto{\pgfqpoint{2.598253in}{1.811790in}}%
\pgfpathlineto{\pgfqpoint{2.412027in}{1.811790in}}%
\pgfpathlineto{\pgfqpoint{2.412027in}{1.730062in}}%
\pgfusepath{}%
\end{pgfscope}%
\begin{pgfscope}%
\pgfpathrectangle{\pgfqpoint{0.549740in}{0.463273in}}{\pgfqpoint{9.320225in}{4.495057in}}%
\pgfusepath{clip}%
\pgfsetbuttcap%
\pgfsetroundjoin%
\pgfsetlinewidth{0.000000pt}%
\definecolor{currentstroke}{rgb}{0.000000,0.000000,0.000000}%
\pgfsetstrokecolor{currentstroke}%
\pgfsetdash{}{0pt}%
\pgfpathmoveto{\pgfqpoint{2.598253in}{1.730062in}}%
\pgfpathlineto{\pgfqpoint{2.784480in}{1.730062in}}%
\pgfpathlineto{\pgfqpoint{2.784480in}{1.811790in}}%
\pgfpathlineto{\pgfqpoint{2.598253in}{1.811790in}}%
\pgfpathlineto{\pgfqpoint{2.598253in}{1.730062in}}%
\pgfusepath{}%
\end{pgfscope}%
\begin{pgfscope}%
\pgfpathrectangle{\pgfqpoint{0.549740in}{0.463273in}}{\pgfqpoint{9.320225in}{4.495057in}}%
\pgfusepath{clip}%
\pgfsetbuttcap%
\pgfsetroundjoin%
\pgfsetlinewidth{0.000000pt}%
\definecolor{currentstroke}{rgb}{0.000000,0.000000,0.000000}%
\pgfsetstrokecolor{currentstroke}%
\pgfsetdash{}{0pt}%
\pgfpathmoveto{\pgfqpoint{2.784480in}{1.730062in}}%
\pgfpathlineto{\pgfqpoint{2.970706in}{1.730062in}}%
\pgfpathlineto{\pgfqpoint{2.970706in}{1.811790in}}%
\pgfpathlineto{\pgfqpoint{2.784480in}{1.811790in}}%
\pgfpathlineto{\pgfqpoint{2.784480in}{1.730062in}}%
\pgfusepath{}%
\end{pgfscope}%
\begin{pgfscope}%
\pgfpathrectangle{\pgfqpoint{0.549740in}{0.463273in}}{\pgfqpoint{9.320225in}{4.495057in}}%
\pgfusepath{clip}%
\pgfsetbuttcap%
\pgfsetroundjoin%
\pgfsetlinewidth{0.000000pt}%
\definecolor{currentstroke}{rgb}{0.000000,0.000000,0.000000}%
\pgfsetstrokecolor{currentstroke}%
\pgfsetdash{}{0pt}%
\pgfpathmoveto{\pgfqpoint{2.970706in}{1.730062in}}%
\pgfpathlineto{\pgfqpoint{3.156933in}{1.730062in}}%
\pgfpathlineto{\pgfqpoint{3.156933in}{1.811790in}}%
\pgfpathlineto{\pgfqpoint{2.970706in}{1.811790in}}%
\pgfpathlineto{\pgfqpoint{2.970706in}{1.730062in}}%
\pgfusepath{}%
\end{pgfscope}%
\begin{pgfscope}%
\pgfpathrectangle{\pgfqpoint{0.549740in}{0.463273in}}{\pgfqpoint{9.320225in}{4.495057in}}%
\pgfusepath{clip}%
\pgfsetbuttcap%
\pgfsetroundjoin%
\pgfsetlinewidth{0.000000pt}%
\definecolor{currentstroke}{rgb}{0.000000,0.000000,0.000000}%
\pgfsetstrokecolor{currentstroke}%
\pgfsetdash{}{0pt}%
\pgfpathmoveto{\pgfqpoint{3.156933in}{1.730062in}}%
\pgfpathlineto{\pgfqpoint{3.343159in}{1.730062in}}%
\pgfpathlineto{\pgfqpoint{3.343159in}{1.811790in}}%
\pgfpathlineto{\pgfqpoint{3.156933in}{1.811790in}}%
\pgfpathlineto{\pgfqpoint{3.156933in}{1.730062in}}%
\pgfusepath{}%
\end{pgfscope}%
\begin{pgfscope}%
\pgfpathrectangle{\pgfqpoint{0.549740in}{0.463273in}}{\pgfqpoint{9.320225in}{4.495057in}}%
\pgfusepath{clip}%
\pgfsetbuttcap%
\pgfsetroundjoin%
\pgfsetlinewidth{0.000000pt}%
\definecolor{currentstroke}{rgb}{0.000000,0.000000,0.000000}%
\pgfsetstrokecolor{currentstroke}%
\pgfsetdash{}{0pt}%
\pgfpathmoveto{\pgfqpoint{3.343159in}{1.730062in}}%
\pgfpathlineto{\pgfqpoint{3.529386in}{1.730062in}}%
\pgfpathlineto{\pgfqpoint{3.529386in}{1.811790in}}%
\pgfpathlineto{\pgfqpoint{3.343159in}{1.811790in}}%
\pgfpathlineto{\pgfqpoint{3.343159in}{1.730062in}}%
\pgfusepath{}%
\end{pgfscope}%
\begin{pgfscope}%
\pgfpathrectangle{\pgfqpoint{0.549740in}{0.463273in}}{\pgfqpoint{9.320225in}{4.495057in}}%
\pgfusepath{clip}%
\pgfsetbuttcap%
\pgfsetroundjoin%
\pgfsetlinewidth{0.000000pt}%
\definecolor{currentstroke}{rgb}{0.000000,0.000000,0.000000}%
\pgfsetstrokecolor{currentstroke}%
\pgfsetdash{}{0pt}%
\pgfpathmoveto{\pgfqpoint{3.529386in}{1.730062in}}%
\pgfpathlineto{\pgfqpoint{3.715612in}{1.730062in}}%
\pgfpathlineto{\pgfqpoint{3.715612in}{1.811790in}}%
\pgfpathlineto{\pgfqpoint{3.529386in}{1.811790in}}%
\pgfpathlineto{\pgfqpoint{3.529386in}{1.730062in}}%
\pgfusepath{}%
\end{pgfscope}%
\begin{pgfscope}%
\pgfpathrectangle{\pgfqpoint{0.549740in}{0.463273in}}{\pgfqpoint{9.320225in}{4.495057in}}%
\pgfusepath{clip}%
\pgfsetbuttcap%
\pgfsetroundjoin%
\pgfsetlinewidth{0.000000pt}%
\definecolor{currentstroke}{rgb}{0.000000,0.000000,0.000000}%
\pgfsetstrokecolor{currentstroke}%
\pgfsetdash{}{0pt}%
\pgfpathmoveto{\pgfqpoint{3.715612in}{1.730062in}}%
\pgfpathlineto{\pgfqpoint{3.901839in}{1.730062in}}%
\pgfpathlineto{\pgfqpoint{3.901839in}{1.811790in}}%
\pgfpathlineto{\pgfqpoint{3.715612in}{1.811790in}}%
\pgfpathlineto{\pgfqpoint{3.715612in}{1.730062in}}%
\pgfusepath{}%
\end{pgfscope}%
\begin{pgfscope}%
\pgfpathrectangle{\pgfqpoint{0.549740in}{0.463273in}}{\pgfqpoint{9.320225in}{4.495057in}}%
\pgfusepath{clip}%
\pgfsetbuttcap%
\pgfsetroundjoin%
\pgfsetlinewidth{0.000000pt}%
\definecolor{currentstroke}{rgb}{0.000000,0.000000,0.000000}%
\pgfsetstrokecolor{currentstroke}%
\pgfsetdash{}{0pt}%
\pgfpathmoveto{\pgfqpoint{3.901839in}{1.730062in}}%
\pgfpathlineto{\pgfqpoint{4.088065in}{1.730062in}}%
\pgfpathlineto{\pgfqpoint{4.088065in}{1.811790in}}%
\pgfpathlineto{\pgfqpoint{3.901839in}{1.811790in}}%
\pgfpathlineto{\pgfqpoint{3.901839in}{1.730062in}}%
\pgfusepath{}%
\end{pgfscope}%
\begin{pgfscope}%
\pgfpathrectangle{\pgfqpoint{0.549740in}{0.463273in}}{\pgfqpoint{9.320225in}{4.495057in}}%
\pgfusepath{clip}%
\pgfsetbuttcap%
\pgfsetroundjoin%
\pgfsetlinewidth{0.000000pt}%
\definecolor{currentstroke}{rgb}{0.000000,0.000000,0.000000}%
\pgfsetstrokecolor{currentstroke}%
\pgfsetdash{}{0pt}%
\pgfpathmoveto{\pgfqpoint{4.088065in}{1.730062in}}%
\pgfpathlineto{\pgfqpoint{4.274292in}{1.730062in}}%
\pgfpathlineto{\pgfqpoint{4.274292in}{1.811790in}}%
\pgfpathlineto{\pgfqpoint{4.088065in}{1.811790in}}%
\pgfpathlineto{\pgfqpoint{4.088065in}{1.730062in}}%
\pgfusepath{}%
\end{pgfscope}%
\begin{pgfscope}%
\pgfpathrectangle{\pgfqpoint{0.549740in}{0.463273in}}{\pgfqpoint{9.320225in}{4.495057in}}%
\pgfusepath{clip}%
\pgfsetbuttcap%
\pgfsetroundjoin%
\pgfsetlinewidth{0.000000pt}%
\definecolor{currentstroke}{rgb}{0.000000,0.000000,0.000000}%
\pgfsetstrokecolor{currentstroke}%
\pgfsetdash{}{0pt}%
\pgfpathmoveto{\pgfqpoint{4.274292in}{1.730062in}}%
\pgfpathlineto{\pgfqpoint{4.460519in}{1.730062in}}%
\pgfpathlineto{\pgfqpoint{4.460519in}{1.811790in}}%
\pgfpathlineto{\pgfqpoint{4.274292in}{1.811790in}}%
\pgfpathlineto{\pgfqpoint{4.274292in}{1.730062in}}%
\pgfusepath{}%
\end{pgfscope}%
\begin{pgfscope}%
\pgfpathrectangle{\pgfqpoint{0.549740in}{0.463273in}}{\pgfqpoint{9.320225in}{4.495057in}}%
\pgfusepath{clip}%
\pgfsetbuttcap%
\pgfsetroundjoin%
\pgfsetlinewidth{0.000000pt}%
\definecolor{currentstroke}{rgb}{0.000000,0.000000,0.000000}%
\pgfsetstrokecolor{currentstroke}%
\pgfsetdash{}{0pt}%
\pgfpathmoveto{\pgfqpoint{4.460519in}{1.730062in}}%
\pgfpathlineto{\pgfqpoint{4.646745in}{1.730062in}}%
\pgfpathlineto{\pgfqpoint{4.646745in}{1.811790in}}%
\pgfpathlineto{\pgfqpoint{4.460519in}{1.811790in}}%
\pgfpathlineto{\pgfqpoint{4.460519in}{1.730062in}}%
\pgfusepath{}%
\end{pgfscope}%
\begin{pgfscope}%
\pgfpathrectangle{\pgfqpoint{0.549740in}{0.463273in}}{\pgfqpoint{9.320225in}{4.495057in}}%
\pgfusepath{clip}%
\pgfsetbuttcap%
\pgfsetroundjoin%
\pgfsetlinewidth{0.000000pt}%
\definecolor{currentstroke}{rgb}{0.000000,0.000000,0.000000}%
\pgfsetstrokecolor{currentstroke}%
\pgfsetdash{}{0pt}%
\pgfpathmoveto{\pgfqpoint{4.646745in}{1.730062in}}%
\pgfpathlineto{\pgfqpoint{4.832972in}{1.730062in}}%
\pgfpathlineto{\pgfqpoint{4.832972in}{1.811790in}}%
\pgfpathlineto{\pgfqpoint{4.646745in}{1.811790in}}%
\pgfpathlineto{\pgfqpoint{4.646745in}{1.730062in}}%
\pgfusepath{}%
\end{pgfscope}%
\begin{pgfscope}%
\pgfpathrectangle{\pgfqpoint{0.549740in}{0.463273in}}{\pgfqpoint{9.320225in}{4.495057in}}%
\pgfusepath{clip}%
\pgfsetbuttcap%
\pgfsetroundjoin%
\pgfsetlinewidth{0.000000pt}%
\definecolor{currentstroke}{rgb}{0.000000,0.000000,0.000000}%
\pgfsetstrokecolor{currentstroke}%
\pgfsetdash{}{0pt}%
\pgfpathmoveto{\pgfqpoint{4.832972in}{1.730062in}}%
\pgfpathlineto{\pgfqpoint{5.019198in}{1.730062in}}%
\pgfpathlineto{\pgfqpoint{5.019198in}{1.811790in}}%
\pgfpathlineto{\pgfqpoint{4.832972in}{1.811790in}}%
\pgfpathlineto{\pgfqpoint{4.832972in}{1.730062in}}%
\pgfusepath{}%
\end{pgfscope}%
\begin{pgfscope}%
\pgfpathrectangle{\pgfqpoint{0.549740in}{0.463273in}}{\pgfqpoint{9.320225in}{4.495057in}}%
\pgfusepath{clip}%
\pgfsetbuttcap%
\pgfsetroundjoin%
\pgfsetlinewidth{0.000000pt}%
\definecolor{currentstroke}{rgb}{0.000000,0.000000,0.000000}%
\pgfsetstrokecolor{currentstroke}%
\pgfsetdash{}{0pt}%
\pgfpathmoveto{\pgfqpoint{5.019198in}{1.730062in}}%
\pgfpathlineto{\pgfqpoint{5.205425in}{1.730062in}}%
\pgfpathlineto{\pgfqpoint{5.205425in}{1.811790in}}%
\pgfpathlineto{\pgfqpoint{5.019198in}{1.811790in}}%
\pgfpathlineto{\pgfqpoint{5.019198in}{1.730062in}}%
\pgfusepath{}%
\end{pgfscope}%
\begin{pgfscope}%
\pgfpathrectangle{\pgfqpoint{0.549740in}{0.463273in}}{\pgfqpoint{9.320225in}{4.495057in}}%
\pgfusepath{clip}%
\pgfsetbuttcap%
\pgfsetroundjoin%
\pgfsetlinewidth{0.000000pt}%
\definecolor{currentstroke}{rgb}{0.000000,0.000000,0.000000}%
\pgfsetstrokecolor{currentstroke}%
\pgfsetdash{}{0pt}%
\pgfpathmoveto{\pgfqpoint{5.205425in}{1.730062in}}%
\pgfpathlineto{\pgfqpoint{5.391651in}{1.730062in}}%
\pgfpathlineto{\pgfqpoint{5.391651in}{1.811790in}}%
\pgfpathlineto{\pgfqpoint{5.205425in}{1.811790in}}%
\pgfpathlineto{\pgfqpoint{5.205425in}{1.730062in}}%
\pgfusepath{}%
\end{pgfscope}%
\begin{pgfscope}%
\pgfpathrectangle{\pgfqpoint{0.549740in}{0.463273in}}{\pgfqpoint{9.320225in}{4.495057in}}%
\pgfusepath{clip}%
\pgfsetbuttcap%
\pgfsetroundjoin%
\pgfsetlinewidth{0.000000pt}%
\definecolor{currentstroke}{rgb}{0.000000,0.000000,0.000000}%
\pgfsetstrokecolor{currentstroke}%
\pgfsetdash{}{0pt}%
\pgfpathmoveto{\pgfqpoint{5.391651in}{1.730062in}}%
\pgfpathlineto{\pgfqpoint{5.577878in}{1.730062in}}%
\pgfpathlineto{\pgfqpoint{5.577878in}{1.811790in}}%
\pgfpathlineto{\pgfqpoint{5.391651in}{1.811790in}}%
\pgfpathlineto{\pgfqpoint{5.391651in}{1.730062in}}%
\pgfusepath{}%
\end{pgfscope}%
\begin{pgfscope}%
\pgfpathrectangle{\pgfqpoint{0.549740in}{0.463273in}}{\pgfqpoint{9.320225in}{4.495057in}}%
\pgfusepath{clip}%
\pgfsetbuttcap%
\pgfsetroundjoin%
\definecolor{currentfill}{rgb}{0.472869,0.711325,0.955316}%
\pgfsetfillcolor{currentfill}%
\pgfsetlinewidth{0.000000pt}%
\definecolor{currentstroke}{rgb}{0.000000,0.000000,0.000000}%
\pgfsetstrokecolor{currentstroke}%
\pgfsetdash{}{0pt}%
\pgfpathmoveto{\pgfqpoint{5.577878in}{1.730062in}}%
\pgfpathlineto{\pgfqpoint{5.764104in}{1.730062in}}%
\pgfpathlineto{\pgfqpoint{5.764104in}{1.811790in}}%
\pgfpathlineto{\pgfqpoint{5.577878in}{1.811790in}}%
\pgfpathlineto{\pgfqpoint{5.577878in}{1.730062in}}%
\pgfusepath{fill}%
\end{pgfscope}%
\begin{pgfscope}%
\pgfpathrectangle{\pgfqpoint{0.549740in}{0.463273in}}{\pgfqpoint{9.320225in}{4.495057in}}%
\pgfusepath{clip}%
\pgfsetbuttcap%
\pgfsetroundjoin%
\pgfsetlinewidth{0.000000pt}%
\definecolor{currentstroke}{rgb}{0.000000,0.000000,0.000000}%
\pgfsetstrokecolor{currentstroke}%
\pgfsetdash{}{0pt}%
\pgfpathmoveto{\pgfqpoint{5.764104in}{1.730062in}}%
\pgfpathlineto{\pgfqpoint{5.950331in}{1.730062in}}%
\pgfpathlineto{\pgfqpoint{5.950331in}{1.811790in}}%
\pgfpathlineto{\pgfqpoint{5.764104in}{1.811790in}}%
\pgfpathlineto{\pgfqpoint{5.764104in}{1.730062in}}%
\pgfusepath{}%
\end{pgfscope}%
\begin{pgfscope}%
\pgfpathrectangle{\pgfqpoint{0.549740in}{0.463273in}}{\pgfqpoint{9.320225in}{4.495057in}}%
\pgfusepath{clip}%
\pgfsetbuttcap%
\pgfsetroundjoin%
\pgfsetlinewidth{0.000000pt}%
\definecolor{currentstroke}{rgb}{0.000000,0.000000,0.000000}%
\pgfsetstrokecolor{currentstroke}%
\pgfsetdash{}{0pt}%
\pgfpathmoveto{\pgfqpoint{5.950331in}{1.730062in}}%
\pgfpathlineto{\pgfqpoint{6.136557in}{1.730062in}}%
\pgfpathlineto{\pgfqpoint{6.136557in}{1.811790in}}%
\pgfpathlineto{\pgfqpoint{5.950331in}{1.811790in}}%
\pgfpathlineto{\pgfqpoint{5.950331in}{1.730062in}}%
\pgfusepath{}%
\end{pgfscope}%
\begin{pgfscope}%
\pgfpathrectangle{\pgfqpoint{0.549740in}{0.463273in}}{\pgfqpoint{9.320225in}{4.495057in}}%
\pgfusepath{clip}%
\pgfsetbuttcap%
\pgfsetroundjoin%
\pgfsetlinewidth{0.000000pt}%
\definecolor{currentstroke}{rgb}{0.000000,0.000000,0.000000}%
\pgfsetstrokecolor{currentstroke}%
\pgfsetdash{}{0pt}%
\pgfpathmoveto{\pgfqpoint{6.136557in}{1.730062in}}%
\pgfpathlineto{\pgfqpoint{6.322784in}{1.730062in}}%
\pgfpathlineto{\pgfqpoint{6.322784in}{1.811790in}}%
\pgfpathlineto{\pgfqpoint{6.136557in}{1.811790in}}%
\pgfpathlineto{\pgfqpoint{6.136557in}{1.730062in}}%
\pgfusepath{}%
\end{pgfscope}%
\begin{pgfscope}%
\pgfpathrectangle{\pgfqpoint{0.549740in}{0.463273in}}{\pgfqpoint{9.320225in}{4.495057in}}%
\pgfusepath{clip}%
\pgfsetbuttcap%
\pgfsetroundjoin%
\pgfsetlinewidth{0.000000pt}%
\definecolor{currentstroke}{rgb}{0.000000,0.000000,0.000000}%
\pgfsetstrokecolor{currentstroke}%
\pgfsetdash{}{0pt}%
\pgfpathmoveto{\pgfqpoint{6.322784in}{1.730062in}}%
\pgfpathlineto{\pgfqpoint{6.509011in}{1.730062in}}%
\pgfpathlineto{\pgfqpoint{6.509011in}{1.811790in}}%
\pgfpathlineto{\pgfqpoint{6.322784in}{1.811790in}}%
\pgfpathlineto{\pgfqpoint{6.322784in}{1.730062in}}%
\pgfusepath{}%
\end{pgfscope}%
\begin{pgfscope}%
\pgfpathrectangle{\pgfqpoint{0.549740in}{0.463273in}}{\pgfqpoint{9.320225in}{4.495057in}}%
\pgfusepath{clip}%
\pgfsetbuttcap%
\pgfsetroundjoin%
\definecolor{currentfill}{rgb}{0.472869,0.711325,0.955316}%
\pgfsetfillcolor{currentfill}%
\pgfsetlinewidth{0.000000pt}%
\definecolor{currentstroke}{rgb}{0.000000,0.000000,0.000000}%
\pgfsetstrokecolor{currentstroke}%
\pgfsetdash{}{0pt}%
\pgfpathmoveto{\pgfqpoint{6.509011in}{1.730062in}}%
\pgfpathlineto{\pgfqpoint{6.695237in}{1.730062in}}%
\pgfpathlineto{\pgfqpoint{6.695237in}{1.811790in}}%
\pgfpathlineto{\pgfqpoint{6.509011in}{1.811790in}}%
\pgfpathlineto{\pgfqpoint{6.509011in}{1.730062in}}%
\pgfusepath{fill}%
\end{pgfscope}%
\begin{pgfscope}%
\pgfpathrectangle{\pgfqpoint{0.549740in}{0.463273in}}{\pgfqpoint{9.320225in}{4.495057in}}%
\pgfusepath{clip}%
\pgfsetbuttcap%
\pgfsetroundjoin%
\pgfsetlinewidth{0.000000pt}%
\definecolor{currentstroke}{rgb}{0.000000,0.000000,0.000000}%
\pgfsetstrokecolor{currentstroke}%
\pgfsetdash{}{0pt}%
\pgfpathmoveto{\pgfqpoint{6.695237in}{1.730062in}}%
\pgfpathlineto{\pgfqpoint{6.881464in}{1.730062in}}%
\pgfpathlineto{\pgfqpoint{6.881464in}{1.811790in}}%
\pgfpathlineto{\pgfqpoint{6.695237in}{1.811790in}}%
\pgfpathlineto{\pgfqpoint{6.695237in}{1.730062in}}%
\pgfusepath{}%
\end{pgfscope}%
\begin{pgfscope}%
\pgfpathrectangle{\pgfqpoint{0.549740in}{0.463273in}}{\pgfqpoint{9.320225in}{4.495057in}}%
\pgfusepath{clip}%
\pgfsetbuttcap%
\pgfsetroundjoin%
\pgfsetlinewidth{0.000000pt}%
\definecolor{currentstroke}{rgb}{0.000000,0.000000,0.000000}%
\pgfsetstrokecolor{currentstroke}%
\pgfsetdash{}{0pt}%
\pgfpathmoveto{\pgfqpoint{6.881464in}{1.730062in}}%
\pgfpathlineto{\pgfqpoint{7.067690in}{1.730062in}}%
\pgfpathlineto{\pgfqpoint{7.067690in}{1.811790in}}%
\pgfpathlineto{\pgfqpoint{6.881464in}{1.811790in}}%
\pgfpathlineto{\pgfqpoint{6.881464in}{1.730062in}}%
\pgfusepath{}%
\end{pgfscope}%
\begin{pgfscope}%
\pgfpathrectangle{\pgfqpoint{0.549740in}{0.463273in}}{\pgfqpoint{9.320225in}{4.495057in}}%
\pgfusepath{clip}%
\pgfsetbuttcap%
\pgfsetroundjoin%
\pgfsetlinewidth{0.000000pt}%
\definecolor{currentstroke}{rgb}{0.000000,0.000000,0.000000}%
\pgfsetstrokecolor{currentstroke}%
\pgfsetdash{}{0pt}%
\pgfpathmoveto{\pgfqpoint{7.067690in}{1.730062in}}%
\pgfpathlineto{\pgfqpoint{7.253917in}{1.730062in}}%
\pgfpathlineto{\pgfqpoint{7.253917in}{1.811790in}}%
\pgfpathlineto{\pgfqpoint{7.067690in}{1.811790in}}%
\pgfpathlineto{\pgfqpoint{7.067690in}{1.730062in}}%
\pgfusepath{}%
\end{pgfscope}%
\begin{pgfscope}%
\pgfpathrectangle{\pgfqpoint{0.549740in}{0.463273in}}{\pgfqpoint{9.320225in}{4.495057in}}%
\pgfusepath{clip}%
\pgfsetbuttcap%
\pgfsetroundjoin%
\pgfsetlinewidth{0.000000pt}%
\definecolor{currentstroke}{rgb}{0.000000,0.000000,0.000000}%
\pgfsetstrokecolor{currentstroke}%
\pgfsetdash{}{0pt}%
\pgfpathmoveto{\pgfqpoint{7.253917in}{1.730062in}}%
\pgfpathlineto{\pgfqpoint{7.440143in}{1.730062in}}%
\pgfpathlineto{\pgfqpoint{7.440143in}{1.811790in}}%
\pgfpathlineto{\pgfqpoint{7.253917in}{1.811790in}}%
\pgfpathlineto{\pgfqpoint{7.253917in}{1.730062in}}%
\pgfusepath{}%
\end{pgfscope}%
\begin{pgfscope}%
\pgfpathrectangle{\pgfqpoint{0.549740in}{0.463273in}}{\pgfqpoint{9.320225in}{4.495057in}}%
\pgfusepath{clip}%
\pgfsetbuttcap%
\pgfsetroundjoin%
\definecolor{currentfill}{rgb}{0.472869,0.711325,0.955316}%
\pgfsetfillcolor{currentfill}%
\pgfsetlinewidth{0.000000pt}%
\definecolor{currentstroke}{rgb}{0.000000,0.000000,0.000000}%
\pgfsetstrokecolor{currentstroke}%
\pgfsetdash{}{0pt}%
\pgfpathmoveto{\pgfqpoint{7.440143in}{1.730062in}}%
\pgfpathlineto{\pgfqpoint{7.626370in}{1.730062in}}%
\pgfpathlineto{\pgfqpoint{7.626370in}{1.811790in}}%
\pgfpathlineto{\pgfqpoint{7.440143in}{1.811790in}}%
\pgfpathlineto{\pgfqpoint{7.440143in}{1.730062in}}%
\pgfusepath{fill}%
\end{pgfscope}%
\begin{pgfscope}%
\pgfpathrectangle{\pgfqpoint{0.549740in}{0.463273in}}{\pgfqpoint{9.320225in}{4.495057in}}%
\pgfusepath{clip}%
\pgfsetbuttcap%
\pgfsetroundjoin%
\definecolor{currentfill}{rgb}{0.472869,0.711325,0.955316}%
\pgfsetfillcolor{currentfill}%
\pgfsetlinewidth{0.000000pt}%
\definecolor{currentstroke}{rgb}{0.000000,0.000000,0.000000}%
\pgfsetstrokecolor{currentstroke}%
\pgfsetdash{}{0pt}%
\pgfpathmoveto{\pgfqpoint{7.626370in}{1.730062in}}%
\pgfpathlineto{\pgfqpoint{7.812596in}{1.730062in}}%
\pgfpathlineto{\pgfqpoint{7.812596in}{1.811790in}}%
\pgfpathlineto{\pgfqpoint{7.626370in}{1.811790in}}%
\pgfpathlineto{\pgfqpoint{7.626370in}{1.730062in}}%
\pgfusepath{fill}%
\end{pgfscope}%
\begin{pgfscope}%
\pgfpathrectangle{\pgfqpoint{0.549740in}{0.463273in}}{\pgfqpoint{9.320225in}{4.495057in}}%
\pgfusepath{clip}%
\pgfsetbuttcap%
\pgfsetroundjoin%
\pgfsetlinewidth{0.000000pt}%
\definecolor{currentstroke}{rgb}{0.000000,0.000000,0.000000}%
\pgfsetstrokecolor{currentstroke}%
\pgfsetdash{}{0pt}%
\pgfpathmoveto{\pgfqpoint{7.812596in}{1.730062in}}%
\pgfpathlineto{\pgfqpoint{7.998823in}{1.730062in}}%
\pgfpathlineto{\pgfqpoint{7.998823in}{1.811790in}}%
\pgfpathlineto{\pgfqpoint{7.812596in}{1.811790in}}%
\pgfpathlineto{\pgfqpoint{7.812596in}{1.730062in}}%
\pgfusepath{}%
\end{pgfscope}%
\begin{pgfscope}%
\pgfpathrectangle{\pgfqpoint{0.549740in}{0.463273in}}{\pgfqpoint{9.320225in}{4.495057in}}%
\pgfusepath{clip}%
\pgfsetbuttcap%
\pgfsetroundjoin%
\pgfsetlinewidth{0.000000pt}%
\definecolor{currentstroke}{rgb}{0.000000,0.000000,0.000000}%
\pgfsetstrokecolor{currentstroke}%
\pgfsetdash{}{0pt}%
\pgfpathmoveto{\pgfqpoint{7.998823in}{1.730062in}}%
\pgfpathlineto{\pgfqpoint{8.185049in}{1.730062in}}%
\pgfpathlineto{\pgfqpoint{8.185049in}{1.811790in}}%
\pgfpathlineto{\pgfqpoint{7.998823in}{1.811790in}}%
\pgfpathlineto{\pgfqpoint{7.998823in}{1.730062in}}%
\pgfusepath{}%
\end{pgfscope}%
\begin{pgfscope}%
\pgfpathrectangle{\pgfqpoint{0.549740in}{0.463273in}}{\pgfqpoint{9.320225in}{4.495057in}}%
\pgfusepath{clip}%
\pgfsetbuttcap%
\pgfsetroundjoin%
\pgfsetlinewidth{0.000000pt}%
\definecolor{currentstroke}{rgb}{0.000000,0.000000,0.000000}%
\pgfsetstrokecolor{currentstroke}%
\pgfsetdash{}{0pt}%
\pgfpathmoveto{\pgfqpoint{8.185049in}{1.730062in}}%
\pgfpathlineto{\pgfqpoint{8.371276in}{1.730062in}}%
\pgfpathlineto{\pgfqpoint{8.371276in}{1.811790in}}%
\pgfpathlineto{\pgfqpoint{8.185049in}{1.811790in}}%
\pgfpathlineto{\pgfqpoint{8.185049in}{1.730062in}}%
\pgfusepath{}%
\end{pgfscope}%
\begin{pgfscope}%
\pgfpathrectangle{\pgfqpoint{0.549740in}{0.463273in}}{\pgfqpoint{9.320225in}{4.495057in}}%
\pgfusepath{clip}%
\pgfsetbuttcap%
\pgfsetroundjoin%
\pgfsetlinewidth{0.000000pt}%
\definecolor{currentstroke}{rgb}{0.000000,0.000000,0.000000}%
\pgfsetstrokecolor{currentstroke}%
\pgfsetdash{}{0pt}%
\pgfpathmoveto{\pgfqpoint{8.371276in}{1.730062in}}%
\pgfpathlineto{\pgfqpoint{8.557503in}{1.730062in}}%
\pgfpathlineto{\pgfqpoint{8.557503in}{1.811790in}}%
\pgfpathlineto{\pgfqpoint{8.371276in}{1.811790in}}%
\pgfpathlineto{\pgfqpoint{8.371276in}{1.730062in}}%
\pgfusepath{}%
\end{pgfscope}%
\begin{pgfscope}%
\pgfpathrectangle{\pgfqpoint{0.549740in}{0.463273in}}{\pgfqpoint{9.320225in}{4.495057in}}%
\pgfusepath{clip}%
\pgfsetbuttcap%
\pgfsetroundjoin%
\definecolor{currentfill}{rgb}{0.273225,0.662144,0.968515}%
\pgfsetfillcolor{currentfill}%
\pgfsetlinewidth{0.000000pt}%
\definecolor{currentstroke}{rgb}{0.000000,0.000000,0.000000}%
\pgfsetstrokecolor{currentstroke}%
\pgfsetdash{}{0pt}%
\pgfpathmoveto{\pgfqpoint{8.557503in}{1.730062in}}%
\pgfpathlineto{\pgfqpoint{8.743729in}{1.730062in}}%
\pgfpathlineto{\pgfqpoint{8.743729in}{1.811790in}}%
\pgfpathlineto{\pgfqpoint{8.557503in}{1.811790in}}%
\pgfpathlineto{\pgfqpoint{8.557503in}{1.730062in}}%
\pgfusepath{fill}%
\end{pgfscope}%
\begin{pgfscope}%
\pgfpathrectangle{\pgfqpoint{0.549740in}{0.463273in}}{\pgfqpoint{9.320225in}{4.495057in}}%
\pgfusepath{clip}%
\pgfsetbuttcap%
\pgfsetroundjoin%
\definecolor{currentfill}{rgb}{0.614330,0.761948,0.940009}%
\pgfsetfillcolor{currentfill}%
\pgfsetlinewidth{0.000000pt}%
\definecolor{currentstroke}{rgb}{0.000000,0.000000,0.000000}%
\pgfsetstrokecolor{currentstroke}%
\pgfsetdash{}{0pt}%
\pgfpathmoveto{\pgfqpoint{8.743729in}{1.730062in}}%
\pgfpathlineto{\pgfqpoint{8.929956in}{1.730062in}}%
\pgfpathlineto{\pgfqpoint{8.929956in}{1.811790in}}%
\pgfpathlineto{\pgfqpoint{8.743729in}{1.811790in}}%
\pgfpathlineto{\pgfqpoint{8.743729in}{1.730062in}}%
\pgfusepath{fill}%
\end{pgfscope}%
\begin{pgfscope}%
\pgfpathrectangle{\pgfqpoint{0.549740in}{0.463273in}}{\pgfqpoint{9.320225in}{4.495057in}}%
\pgfusepath{clip}%
\pgfsetbuttcap%
\pgfsetroundjoin%
\pgfsetlinewidth{0.000000pt}%
\definecolor{currentstroke}{rgb}{0.000000,0.000000,0.000000}%
\pgfsetstrokecolor{currentstroke}%
\pgfsetdash{}{0pt}%
\pgfpathmoveto{\pgfqpoint{8.929956in}{1.730062in}}%
\pgfpathlineto{\pgfqpoint{9.116182in}{1.730062in}}%
\pgfpathlineto{\pgfqpoint{9.116182in}{1.811790in}}%
\pgfpathlineto{\pgfqpoint{8.929956in}{1.811790in}}%
\pgfpathlineto{\pgfqpoint{8.929956in}{1.730062in}}%
\pgfusepath{}%
\end{pgfscope}%
\begin{pgfscope}%
\pgfpathrectangle{\pgfqpoint{0.549740in}{0.463273in}}{\pgfqpoint{9.320225in}{4.495057in}}%
\pgfusepath{clip}%
\pgfsetbuttcap%
\pgfsetroundjoin%
\pgfsetlinewidth{0.000000pt}%
\definecolor{currentstroke}{rgb}{0.000000,0.000000,0.000000}%
\pgfsetstrokecolor{currentstroke}%
\pgfsetdash{}{0pt}%
\pgfpathmoveto{\pgfqpoint{9.116182in}{1.730062in}}%
\pgfpathlineto{\pgfqpoint{9.302409in}{1.730062in}}%
\pgfpathlineto{\pgfqpoint{9.302409in}{1.811790in}}%
\pgfpathlineto{\pgfqpoint{9.116182in}{1.811790in}}%
\pgfpathlineto{\pgfqpoint{9.116182in}{1.730062in}}%
\pgfusepath{}%
\end{pgfscope}%
\begin{pgfscope}%
\pgfpathrectangle{\pgfqpoint{0.549740in}{0.463273in}}{\pgfqpoint{9.320225in}{4.495057in}}%
\pgfusepath{clip}%
\pgfsetbuttcap%
\pgfsetroundjoin%
\pgfsetlinewidth{0.000000pt}%
\definecolor{currentstroke}{rgb}{0.000000,0.000000,0.000000}%
\pgfsetstrokecolor{currentstroke}%
\pgfsetdash{}{0pt}%
\pgfpathmoveto{\pgfqpoint{9.302409in}{1.730062in}}%
\pgfpathlineto{\pgfqpoint{9.488635in}{1.730062in}}%
\pgfpathlineto{\pgfqpoint{9.488635in}{1.811790in}}%
\pgfpathlineto{\pgfqpoint{9.302409in}{1.811790in}}%
\pgfpathlineto{\pgfqpoint{9.302409in}{1.730062in}}%
\pgfusepath{}%
\end{pgfscope}%
\begin{pgfscope}%
\pgfpathrectangle{\pgfqpoint{0.549740in}{0.463273in}}{\pgfqpoint{9.320225in}{4.495057in}}%
\pgfusepath{clip}%
\pgfsetbuttcap%
\pgfsetroundjoin%
\pgfsetlinewidth{0.000000pt}%
\definecolor{currentstroke}{rgb}{0.000000,0.000000,0.000000}%
\pgfsetstrokecolor{currentstroke}%
\pgfsetdash{}{0pt}%
\pgfpathmoveto{\pgfqpoint{9.488635in}{1.730062in}}%
\pgfpathlineto{\pgfqpoint{9.674862in}{1.730062in}}%
\pgfpathlineto{\pgfqpoint{9.674862in}{1.811790in}}%
\pgfpathlineto{\pgfqpoint{9.488635in}{1.811790in}}%
\pgfpathlineto{\pgfqpoint{9.488635in}{1.730062in}}%
\pgfusepath{}%
\end{pgfscope}%
\begin{pgfscope}%
\pgfpathrectangle{\pgfqpoint{0.549740in}{0.463273in}}{\pgfqpoint{9.320225in}{4.495057in}}%
\pgfusepath{clip}%
\pgfsetbuttcap%
\pgfsetroundjoin%
\pgfsetlinewidth{0.000000pt}%
\definecolor{currentstroke}{rgb}{0.000000,0.000000,0.000000}%
\pgfsetstrokecolor{currentstroke}%
\pgfsetdash{}{0pt}%
\pgfpathmoveto{\pgfqpoint{9.674862in}{1.730062in}}%
\pgfpathlineto{\pgfqpoint{9.861088in}{1.730062in}}%
\pgfpathlineto{\pgfqpoint{9.861088in}{1.811790in}}%
\pgfpathlineto{\pgfqpoint{9.674862in}{1.811790in}}%
\pgfpathlineto{\pgfqpoint{9.674862in}{1.730062in}}%
\pgfusepath{}%
\end{pgfscope}%
\begin{pgfscope}%
\pgfpathrectangle{\pgfqpoint{0.549740in}{0.463273in}}{\pgfqpoint{9.320225in}{4.495057in}}%
\pgfusepath{clip}%
\pgfsetbuttcap%
\pgfsetroundjoin%
\pgfsetlinewidth{0.000000pt}%
\definecolor{currentstroke}{rgb}{0.000000,0.000000,0.000000}%
\pgfsetstrokecolor{currentstroke}%
\pgfsetdash{}{0pt}%
\pgfpathmoveto{\pgfqpoint{0.549761in}{1.811790in}}%
\pgfpathlineto{\pgfqpoint{0.735988in}{1.811790in}}%
\pgfpathlineto{\pgfqpoint{0.735988in}{1.893519in}}%
\pgfpathlineto{\pgfqpoint{0.549761in}{1.893519in}}%
\pgfpathlineto{\pgfqpoint{0.549761in}{1.811790in}}%
\pgfusepath{}%
\end{pgfscope}%
\begin{pgfscope}%
\pgfpathrectangle{\pgfqpoint{0.549740in}{0.463273in}}{\pgfqpoint{9.320225in}{4.495057in}}%
\pgfusepath{clip}%
\pgfsetbuttcap%
\pgfsetroundjoin%
\pgfsetlinewidth{0.000000pt}%
\definecolor{currentstroke}{rgb}{0.000000,0.000000,0.000000}%
\pgfsetstrokecolor{currentstroke}%
\pgfsetdash{}{0pt}%
\pgfpathmoveto{\pgfqpoint{0.735988in}{1.811790in}}%
\pgfpathlineto{\pgfqpoint{0.922214in}{1.811790in}}%
\pgfpathlineto{\pgfqpoint{0.922214in}{1.893519in}}%
\pgfpathlineto{\pgfqpoint{0.735988in}{1.893519in}}%
\pgfpathlineto{\pgfqpoint{0.735988in}{1.811790in}}%
\pgfusepath{}%
\end{pgfscope}%
\begin{pgfscope}%
\pgfpathrectangle{\pgfqpoint{0.549740in}{0.463273in}}{\pgfqpoint{9.320225in}{4.495057in}}%
\pgfusepath{clip}%
\pgfsetbuttcap%
\pgfsetroundjoin%
\pgfsetlinewidth{0.000000pt}%
\definecolor{currentstroke}{rgb}{0.000000,0.000000,0.000000}%
\pgfsetstrokecolor{currentstroke}%
\pgfsetdash{}{0pt}%
\pgfpathmoveto{\pgfqpoint{0.922214in}{1.811790in}}%
\pgfpathlineto{\pgfqpoint{1.108441in}{1.811790in}}%
\pgfpathlineto{\pgfqpoint{1.108441in}{1.893519in}}%
\pgfpathlineto{\pgfqpoint{0.922214in}{1.893519in}}%
\pgfpathlineto{\pgfqpoint{0.922214in}{1.811790in}}%
\pgfusepath{}%
\end{pgfscope}%
\begin{pgfscope}%
\pgfpathrectangle{\pgfqpoint{0.549740in}{0.463273in}}{\pgfqpoint{9.320225in}{4.495057in}}%
\pgfusepath{clip}%
\pgfsetbuttcap%
\pgfsetroundjoin%
\pgfsetlinewidth{0.000000pt}%
\definecolor{currentstroke}{rgb}{0.000000,0.000000,0.000000}%
\pgfsetstrokecolor{currentstroke}%
\pgfsetdash{}{0pt}%
\pgfpathmoveto{\pgfqpoint{1.108441in}{1.811790in}}%
\pgfpathlineto{\pgfqpoint{1.294667in}{1.811790in}}%
\pgfpathlineto{\pgfqpoint{1.294667in}{1.893519in}}%
\pgfpathlineto{\pgfqpoint{1.108441in}{1.893519in}}%
\pgfpathlineto{\pgfqpoint{1.108441in}{1.811790in}}%
\pgfusepath{}%
\end{pgfscope}%
\begin{pgfscope}%
\pgfpathrectangle{\pgfqpoint{0.549740in}{0.463273in}}{\pgfqpoint{9.320225in}{4.495057in}}%
\pgfusepath{clip}%
\pgfsetbuttcap%
\pgfsetroundjoin%
\pgfsetlinewidth{0.000000pt}%
\definecolor{currentstroke}{rgb}{0.000000,0.000000,0.000000}%
\pgfsetstrokecolor{currentstroke}%
\pgfsetdash{}{0pt}%
\pgfpathmoveto{\pgfqpoint{1.294667in}{1.811790in}}%
\pgfpathlineto{\pgfqpoint{1.480894in}{1.811790in}}%
\pgfpathlineto{\pgfqpoint{1.480894in}{1.893519in}}%
\pgfpathlineto{\pgfqpoint{1.294667in}{1.893519in}}%
\pgfpathlineto{\pgfqpoint{1.294667in}{1.811790in}}%
\pgfusepath{}%
\end{pgfscope}%
\begin{pgfscope}%
\pgfpathrectangle{\pgfqpoint{0.549740in}{0.463273in}}{\pgfqpoint{9.320225in}{4.495057in}}%
\pgfusepath{clip}%
\pgfsetbuttcap%
\pgfsetroundjoin%
\pgfsetlinewidth{0.000000pt}%
\definecolor{currentstroke}{rgb}{0.000000,0.000000,0.000000}%
\pgfsetstrokecolor{currentstroke}%
\pgfsetdash{}{0pt}%
\pgfpathmoveto{\pgfqpoint{1.480894in}{1.811790in}}%
\pgfpathlineto{\pgfqpoint{1.667120in}{1.811790in}}%
\pgfpathlineto{\pgfqpoint{1.667120in}{1.893519in}}%
\pgfpathlineto{\pgfqpoint{1.480894in}{1.893519in}}%
\pgfpathlineto{\pgfqpoint{1.480894in}{1.811790in}}%
\pgfusepath{}%
\end{pgfscope}%
\begin{pgfscope}%
\pgfpathrectangle{\pgfqpoint{0.549740in}{0.463273in}}{\pgfqpoint{9.320225in}{4.495057in}}%
\pgfusepath{clip}%
\pgfsetbuttcap%
\pgfsetroundjoin%
\pgfsetlinewidth{0.000000pt}%
\definecolor{currentstroke}{rgb}{0.000000,0.000000,0.000000}%
\pgfsetstrokecolor{currentstroke}%
\pgfsetdash{}{0pt}%
\pgfpathmoveto{\pgfqpoint{1.667120in}{1.811790in}}%
\pgfpathlineto{\pgfqpoint{1.853347in}{1.811790in}}%
\pgfpathlineto{\pgfqpoint{1.853347in}{1.893519in}}%
\pgfpathlineto{\pgfqpoint{1.667120in}{1.893519in}}%
\pgfpathlineto{\pgfqpoint{1.667120in}{1.811790in}}%
\pgfusepath{}%
\end{pgfscope}%
\begin{pgfscope}%
\pgfpathrectangle{\pgfqpoint{0.549740in}{0.463273in}}{\pgfqpoint{9.320225in}{4.495057in}}%
\pgfusepath{clip}%
\pgfsetbuttcap%
\pgfsetroundjoin%
\pgfsetlinewidth{0.000000pt}%
\definecolor{currentstroke}{rgb}{0.000000,0.000000,0.000000}%
\pgfsetstrokecolor{currentstroke}%
\pgfsetdash{}{0pt}%
\pgfpathmoveto{\pgfqpoint{1.853347in}{1.811790in}}%
\pgfpathlineto{\pgfqpoint{2.039573in}{1.811790in}}%
\pgfpathlineto{\pgfqpoint{2.039573in}{1.893519in}}%
\pgfpathlineto{\pgfqpoint{1.853347in}{1.893519in}}%
\pgfpathlineto{\pgfqpoint{1.853347in}{1.811790in}}%
\pgfusepath{}%
\end{pgfscope}%
\begin{pgfscope}%
\pgfpathrectangle{\pgfqpoint{0.549740in}{0.463273in}}{\pgfqpoint{9.320225in}{4.495057in}}%
\pgfusepath{clip}%
\pgfsetbuttcap%
\pgfsetroundjoin%
\pgfsetlinewidth{0.000000pt}%
\definecolor{currentstroke}{rgb}{0.000000,0.000000,0.000000}%
\pgfsetstrokecolor{currentstroke}%
\pgfsetdash{}{0pt}%
\pgfpathmoveto{\pgfqpoint{2.039573in}{1.811790in}}%
\pgfpathlineto{\pgfqpoint{2.225800in}{1.811790in}}%
\pgfpathlineto{\pgfqpoint{2.225800in}{1.893519in}}%
\pgfpathlineto{\pgfqpoint{2.039573in}{1.893519in}}%
\pgfpathlineto{\pgfqpoint{2.039573in}{1.811790in}}%
\pgfusepath{}%
\end{pgfscope}%
\begin{pgfscope}%
\pgfpathrectangle{\pgfqpoint{0.549740in}{0.463273in}}{\pgfqpoint{9.320225in}{4.495057in}}%
\pgfusepath{clip}%
\pgfsetbuttcap%
\pgfsetroundjoin%
\pgfsetlinewidth{0.000000pt}%
\definecolor{currentstroke}{rgb}{0.000000,0.000000,0.000000}%
\pgfsetstrokecolor{currentstroke}%
\pgfsetdash{}{0pt}%
\pgfpathmoveto{\pgfqpoint{2.225800in}{1.811790in}}%
\pgfpathlineto{\pgfqpoint{2.412027in}{1.811790in}}%
\pgfpathlineto{\pgfqpoint{2.412027in}{1.893519in}}%
\pgfpathlineto{\pgfqpoint{2.225800in}{1.893519in}}%
\pgfpathlineto{\pgfqpoint{2.225800in}{1.811790in}}%
\pgfusepath{}%
\end{pgfscope}%
\begin{pgfscope}%
\pgfpathrectangle{\pgfqpoint{0.549740in}{0.463273in}}{\pgfqpoint{9.320225in}{4.495057in}}%
\pgfusepath{clip}%
\pgfsetbuttcap%
\pgfsetroundjoin%
\pgfsetlinewidth{0.000000pt}%
\definecolor{currentstroke}{rgb}{0.000000,0.000000,0.000000}%
\pgfsetstrokecolor{currentstroke}%
\pgfsetdash{}{0pt}%
\pgfpathmoveto{\pgfqpoint{2.412027in}{1.811790in}}%
\pgfpathlineto{\pgfqpoint{2.598253in}{1.811790in}}%
\pgfpathlineto{\pgfqpoint{2.598253in}{1.893519in}}%
\pgfpathlineto{\pgfqpoint{2.412027in}{1.893519in}}%
\pgfpathlineto{\pgfqpoint{2.412027in}{1.811790in}}%
\pgfusepath{}%
\end{pgfscope}%
\begin{pgfscope}%
\pgfpathrectangle{\pgfqpoint{0.549740in}{0.463273in}}{\pgfqpoint{9.320225in}{4.495057in}}%
\pgfusepath{clip}%
\pgfsetbuttcap%
\pgfsetroundjoin%
\pgfsetlinewidth{0.000000pt}%
\definecolor{currentstroke}{rgb}{0.000000,0.000000,0.000000}%
\pgfsetstrokecolor{currentstroke}%
\pgfsetdash{}{0pt}%
\pgfpathmoveto{\pgfqpoint{2.598253in}{1.811790in}}%
\pgfpathlineto{\pgfqpoint{2.784480in}{1.811790in}}%
\pgfpathlineto{\pgfqpoint{2.784480in}{1.893519in}}%
\pgfpathlineto{\pgfqpoint{2.598253in}{1.893519in}}%
\pgfpathlineto{\pgfqpoint{2.598253in}{1.811790in}}%
\pgfusepath{}%
\end{pgfscope}%
\begin{pgfscope}%
\pgfpathrectangle{\pgfqpoint{0.549740in}{0.463273in}}{\pgfqpoint{9.320225in}{4.495057in}}%
\pgfusepath{clip}%
\pgfsetbuttcap%
\pgfsetroundjoin%
\pgfsetlinewidth{0.000000pt}%
\definecolor{currentstroke}{rgb}{0.000000,0.000000,0.000000}%
\pgfsetstrokecolor{currentstroke}%
\pgfsetdash{}{0pt}%
\pgfpathmoveto{\pgfqpoint{2.784480in}{1.811790in}}%
\pgfpathlineto{\pgfqpoint{2.970706in}{1.811790in}}%
\pgfpathlineto{\pgfqpoint{2.970706in}{1.893519in}}%
\pgfpathlineto{\pgfqpoint{2.784480in}{1.893519in}}%
\pgfpathlineto{\pgfqpoint{2.784480in}{1.811790in}}%
\pgfusepath{}%
\end{pgfscope}%
\begin{pgfscope}%
\pgfpathrectangle{\pgfqpoint{0.549740in}{0.463273in}}{\pgfqpoint{9.320225in}{4.495057in}}%
\pgfusepath{clip}%
\pgfsetbuttcap%
\pgfsetroundjoin%
\pgfsetlinewidth{0.000000pt}%
\definecolor{currentstroke}{rgb}{0.000000,0.000000,0.000000}%
\pgfsetstrokecolor{currentstroke}%
\pgfsetdash{}{0pt}%
\pgfpathmoveto{\pgfqpoint{2.970706in}{1.811790in}}%
\pgfpathlineto{\pgfqpoint{3.156933in}{1.811790in}}%
\pgfpathlineto{\pgfqpoint{3.156933in}{1.893519in}}%
\pgfpathlineto{\pgfqpoint{2.970706in}{1.893519in}}%
\pgfpathlineto{\pgfqpoint{2.970706in}{1.811790in}}%
\pgfusepath{}%
\end{pgfscope}%
\begin{pgfscope}%
\pgfpathrectangle{\pgfqpoint{0.549740in}{0.463273in}}{\pgfqpoint{9.320225in}{4.495057in}}%
\pgfusepath{clip}%
\pgfsetbuttcap%
\pgfsetroundjoin%
\pgfsetlinewidth{0.000000pt}%
\definecolor{currentstroke}{rgb}{0.000000,0.000000,0.000000}%
\pgfsetstrokecolor{currentstroke}%
\pgfsetdash{}{0pt}%
\pgfpathmoveto{\pgfqpoint{3.156933in}{1.811790in}}%
\pgfpathlineto{\pgfqpoint{3.343159in}{1.811790in}}%
\pgfpathlineto{\pgfqpoint{3.343159in}{1.893519in}}%
\pgfpathlineto{\pgfqpoint{3.156933in}{1.893519in}}%
\pgfpathlineto{\pgfqpoint{3.156933in}{1.811790in}}%
\pgfusepath{}%
\end{pgfscope}%
\begin{pgfscope}%
\pgfpathrectangle{\pgfqpoint{0.549740in}{0.463273in}}{\pgfqpoint{9.320225in}{4.495057in}}%
\pgfusepath{clip}%
\pgfsetbuttcap%
\pgfsetroundjoin%
\pgfsetlinewidth{0.000000pt}%
\definecolor{currentstroke}{rgb}{0.000000,0.000000,0.000000}%
\pgfsetstrokecolor{currentstroke}%
\pgfsetdash{}{0pt}%
\pgfpathmoveto{\pgfqpoint{3.343159in}{1.811790in}}%
\pgfpathlineto{\pgfqpoint{3.529386in}{1.811790in}}%
\pgfpathlineto{\pgfqpoint{3.529386in}{1.893519in}}%
\pgfpathlineto{\pgfqpoint{3.343159in}{1.893519in}}%
\pgfpathlineto{\pgfqpoint{3.343159in}{1.811790in}}%
\pgfusepath{}%
\end{pgfscope}%
\begin{pgfscope}%
\pgfpathrectangle{\pgfqpoint{0.549740in}{0.463273in}}{\pgfqpoint{9.320225in}{4.495057in}}%
\pgfusepath{clip}%
\pgfsetbuttcap%
\pgfsetroundjoin%
\pgfsetlinewidth{0.000000pt}%
\definecolor{currentstroke}{rgb}{0.000000,0.000000,0.000000}%
\pgfsetstrokecolor{currentstroke}%
\pgfsetdash{}{0pt}%
\pgfpathmoveto{\pgfqpoint{3.529386in}{1.811790in}}%
\pgfpathlineto{\pgfqpoint{3.715612in}{1.811790in}}%
\pgfpathlineto{\pgfqpoint{3.715612in}{1.893519in}}%
\pgfpathlineto{\pgfqpoint{3.529386in}{1.893519in}}%
\pgfpathlineto{\pgfqpoint{3.529386in}{1.811790in}}%
\pgfusepath{}%
\end{pgfscope}%
\begin{pgfscope}%
\pgfpathrectangle{\pgfqpoint{0.549740in}{0.463273in}}{\pgfqpoint{9.320225in}{4.495057in}}%
\pgfusepath{clip}%
\pgfsetbuttcap%
\pgfsetroundjoin%
\pgfsetlinewidth{0.000000pt}%
\definecolor{currentstroke}{rgb}{0.000000,0.000000,0.000000}%
\pgfsetstrokecolor{currentstroke}%
\pgfsetdash{}{0pt}%
\pgfpathmoveto{\pgfqpoint{3.715612in}{1.811790in}}%
\pgfpathlineto{\pgfqpoint{3.901839in}{1.811790in}}%
\pgfpathlineto{\pgfqpoint{3.901839in}{1.893519in}}%
\pgfpathlineto{\pgfqpoint{3.715612in}{1.893519in}}%
\pgfpathlineto{\pgfqpoint{3.715612in}{1.811790in}}%
\pgfusepath{}%
\end{pgfscope}%
\begin{pgfscope}%
\pgfpathrectangle{\pgfqpoint{0.549740in}{0.463273in}}{\pgfqpoint{9.320225in}{4.495057in}}%
\pgfusepath{clip}%
\pgfsetbuttcap%
\pgfsetroundjoin%
\pgfsetlinewidth{0.000000pt}%
\definecolor{currentstroke}{rgb}{0.000000,0.000000,0.000000}%
\pgfsetstrokecolor{currentstroke}%
\pgfsetdash{}{0pt}%
\pgfpathmoveto{\pgfqpoint{3.901839in}{1.811790in}}%
\pgfpathlineto{\pgfqpoint{4.088065in}{1.811790in}}%
\pgfpathlineto{\pgfqpoint{4.088065in}{1.893519in}}%
\pgfpathlineto{\pgfqpoint{3.901839in}{1.893519in}}%
\pgfpathlineto{\pgfqpoint{3.901839in}{1.811790in}}%
\pgfusepath{}%
\end{pgfscope}%
\begin{pgfscope}%
\pgfpathrectangle{\pgfqpoint{0.549740in}{0.463273in}}{\pgfqpoint{9.320225in}{4.495057in}}%
\pgfusepath{clip}%
\pgfsetbuttcap%
\pgfsetroundjoin%
\pgfsetlinewidth{0.000000pt}%
\definecolor{currentstroke}{rgb}{0.000000,0.000000,0.000000}%
\pgfsetstrokecolor{currentstroke}%
\pgfsetdash{}{0pt}%
\pgfpathmoveto{\pgfqpoint{4.088065in}{1.811790in}}%
\pgfpathlineto{\pgfqpoint{4.274292in}{1.811790in}}%
\pgfpathlineto{\pgfqpoint{4.274292in}{1.893519in}}%
\pgfpathlineto{\pgfqpoint{4.088065in}{1.893519in}}%
\pgfpathlineto{\pgfqpoint{4.088065in}{1.811790in}}%
\pgfusepath{}%
\end{pgfscope}%
\begin{pgfscope}%
\pgfpathrectangle{\pgfqpoint{0.549740in}{0.463273in}}{\pgfqpoint{9.320225in}{4.495057in}}%
\pgfusepath{clip}%
\pgfsetbuttcap%
\pgfsetroundjoin%
\pgfsetlinewidth{0.000000pt}%
\definecolor{currentstroke}{rgb}{0.000000,0.000000,0.000000}%
\pgfsetstrokecolor{currentstroke}%
\pgfsetdash{}{0pt}%
\pgfpathmoveto{\pgfqpoint{4.274292in}{1.811790in}}%
\pgfpathlineto{\pgfqpoint{4.460519in}{1.811790in}}%
\pgfpathlineto{\pgfqpoint{4.460519in}{1.893519in}}%
\pgfpathlineto{\pgfqpoint{4.274292in}{1.893519in}}%
\pgfpathlineto{\pgfqpoint{4.274292in}{1.811790in}}%
\pgfusepath{}%
\end{pgfscope}%
\begin{pgfscope}%
\pgfpathrectangle{\pgfqpoint{0.549740in}{0.463273in}}{\pgfqpoint{9.320225in}{4.495057in}}%
\pgfusepath{clip}%
\pgfsetbuttcap%
\pgfsetroundjoin%
\pgfsetlinewidth{0.000000pt}%
\definecolor{currentstroke}{rgb}{0.000000,0.000000,0.000000}%
\pgfsetstrokecolor{currentstroke}%
\pgfsetdash{}{0pt}%
\pgfpathmoveto{\pgfqpoint{4.460519in}{1.811790in}}%
\pgfpathlineto{\pgfqpoint{4.646745in}{1.811790in}}%
\pgfpathlineto{\pgfqpoint{4.646745in}{1.893519in}}%
\pgfpathlineto{\pgfqpoint{4.460519in}{1.893519in}}%
\pgfpathlineto{\pgfqpoint{4.460519in}{1.811790in}}%
\pgfusepath{}%
\end{pgfscope}%
\begin{pgfscope}%
\pgfpathrectangle{\pgfqpoint{0.549740in}{0.463273in}}{\pgfqpoint{9.320225in}{4.495057in}}%
\pgfusepath{clip}%
\pgfsetbuttcap%
\pgfsetroundjoin%
\pgfsetlinewidth{0.000000pt}%
\definecolor{currentstroke}{rgb}{0.000000,0.000000,0.000000}%
\pgfsetstrokecolor{currentstroke}%
\pgfsetdash{}{0pt}%
\pgfpathmoveto{\pgfqpoint{4.646745in}{1.811790in}}%
\pgfpathlineto{\pgfqpoint{4.832972in}{1.811790in}}%
\pgfpathlineto{\pgfqpoint{4.832972in}{1.893519in}}%
\pgfpathlineto{\pgfqpoint{4.646745in}{1.893519in}}%
\pgfpathlineto{\pgfqpoint{4.646745in}{1.811790in}}%
\pgfusepath{}%
\end{pgfscope}%
\begin{pgfscope}%
\pgfpathrectangle{\pgfqpoint{0.549740in}{0.463273in}}{\pgfqpoint{9.320225in}{4.495057in}}%
\pgfusepath{clip}%
\pgfsetbuttcap%
\pgfsetroundjoin%
\pgfsetlinewidth{0.000000pt}%
\definecolor{currentstroke}{rgb}{0.000000,0.000000,0.000000}%
\pgfsetstrokecolor{currentstroke}%
\pgfsetdash{}{0pt}%
\pgfpathmoveto{\pgfqpoint{4.832972in}{1.811790in}}%
\pgfpathlineto{\pgfqpoint{5.019198in}{1.811790in}}%
\pgfpathlineto{\pgfqpoint{5.019198in}{1.893519in}}%
\pgfpathlineto{\pgfqpoint{4.832972in}{1.893519in}}%
\pgfpathlineto{\pgfqpoint{4.832972in}{1.811790in}}%
\pgfusepath{}%
\end{pgfscope}%
\begin{pgfscope}%
\pgfpathrectangle{\pgfqpoint{0.549740in}{0.463273in}}{\pgfqpoint{9.320225in}{4.495057in}}%
\pgfusepath{clip}%
\pgfsetbuttcap%
\pgfsetroundjoin%
\pgfsetlinewidth{0.000000pt}%
\definecolor{currentstroke}{rgb}{0.000000,0.000000,0.000000}%
\pgfsetstrokecolor{currentstroke}%
\pgfsetdash{}{0pt}%
\pgfpathmoveto{\pgfqpoint{5.019198in}{1.811790in}}%
\pgfpathlineto{\pgfqpoint{5.205425in}{1.811790in}}%
\pgfpathlineto{\pgfqpoint{5.205425in}{1.893519in}}%
\pgfpathlineto{\pgfqpoint{5.019198in}{1.893519in}}%
\pgfpathlineto{\pgfqpoint{5.019198in}{1.811790in}}%
\pgfusepath{}%
\end{pgfscope}%
\begin{pgfscope}%
\pgfpathrectangle{\pgfqpoint{0.549740in}{0.463273in}}{\pgfqpoint{9.320225in}{4.495057in}}%
\pgfusepath{clip}%
\pgfsetbuttcap%
\pgfsetroundjoin%
\pgfsetlinewidth{0.000000pt}%
\definecolor{currentstroke}{rgb}{0.000000,0.000000,0.000000}%
\pgfsetstrokecolor{currentstroke}%
\pgfsetdash{}{0pt}%
\pgfpathmoveto{\pgfqpoint{5.205425in}{1.811790in}}%
\pgfpathlineto{\pgfqpoint{5.391651in}{1.811790in}}%
\pgfpathlineto{\pgfqpoint{5.391651in}{1.893519in}}%
\pgfpathlineto{\pgfqpoint{5.205425in}{1.893519in}}%
\pgfpathlineto{\pgfqpoint{5.205425in}{1.811790in}}%
\pgfusepath{}%
\end{pgfscope}%
\begin{pgfscope}%
\pgfpathrectangle{\pgfqpoint{0.549740in}{0.463273in}}{\pgfqpoint{9.320225in}{4.495057in}}%
\pgfusepath{clip}%
\pgfsetbuttcap%
\pgfsetroundjoin%
\pgfsetlinewidth{0.000000pt}%
\definecolor{currentstroke}{rgb}{0.000000,0.000000,0.000000}%
\pgfsetstrokecolor{currentstroke}%
\pgfsetdash{}{0pt}%
\pgfpathmoveto{\pgfqpoint{5.391651in}{1.811790in}}%
\pgfpathlineto{\pgfqpoint{5.577878in}{1.811790in}}%
\pgfpathlineto{\pgfqpoint{5.577878in}{1.893519in}}%
\pgfpathlineto{\pgfqpoint{5.391651in}{1.893519in}}%
\pgfpathlineto{\pgfqpoint{5.391651in}{1.811790in}}%
\pgfusepath{}%
\end{pgfscope}%
\begin{pgfscope}%
\pgfpathrectangle{\pgfqpoint{0.549740in}{0.463273in}}{\pgfqpoint{9.320225in}{4.495057in}}%
\pgfusepath{clip}%
\pgfsetbuttcap%
\pgfsetroundjoin%
\definecolor{currentfill}{rgb}{0.472869,0.711325,0.955316}%
\pgfsetfillcolor{currentfill}%
\pgfsetlinewidth{0.000000pt}%
\definecolor{currentstroke}{rgb}{0.000000,0.000000,0.000000}%
\pgfsetstrokecolor{currentstroke}%
\pgfsetdash{}{0pt}%
\pgfpathmoveto{\pgfqpoint{5.577878in}{1.811790in}}%
\pgfpathlineto{\pgfqpoint{5.764104in}{1.811790in}}%
\pgfpathlineto{\pgfqpoint{5.764104in}{1.893519in}}%
\pgfpathlineto{\pgfqpoint{5.577878in}{1.893519in}}%
\pgfpathlineto{\pgfqpoint{5.577878in}{1.811790in}}%
\pgfusepath{fill}%
\end{pgfscope}%
\begin{pgfscope}%
\pgfpathrectangle{\pgfqpoint{0.549740in}{0.463273in}}{\pgfqpoint{9.320225in}{4.495057in}}%
\pgfusepath{clip}%
\pgfsetbuttcap%
\pgfsetroundjoin%
\pgfsetlinewidth{0.000000pt}%
\definecolor{currentstroke}{rgb}{0.000000,0.000000,0.000000}%
\pgfsetstrokecolor{currentstroke}%
\pgfsetdash{}{0pt}%
\pgfpathmoveto{\pgfqpoint{5.764104in}{1.811790in}}%
\pgfpathlineto{\pgfqpoint{5.950331in}{1.811790in}}%
\pgfpathlineto{\pgfqpoint{5.950331in}{1.893519in}}%
\pgfpathlineto{\pgfqpoint{5.764104in}{1.893519in}}%
\pgfpathlineto{\pgfqpoint{5.764104in}{1.811790in}}%
\pgfusepath{}%
\end{pgfscope}%
\begin{pgfscope}%
\pgfpathrectangle{\pgfqpoint{0.549740in}{0.463273in}}{\pgfqpoint{9.320225in}{4.495057in}}%
\pgfusepath{clip}%
\pgfsetbuttcap%
\pgfsetroundjoin%
\pgfsetlinewidth{0.000000pt}%
\definecolor{currentstroke}{rgb}{0.000000,0.000000,0.000000}%
\pgfsetstrokecolor{currentstroke}%
\pgfsetdash{}{0pt}%
\pgfpathmoveto{\pgfqpoint{5.950331in}{1.811790in}}%
\pgfpathlineto{\pgfqpoint{6.136557in}{1.811790in}}%
\pgfpathlineto{\pgfqpoint{6.136557in}{1.893519in}}%
\pgfpathlineto{\pgfqpoint{5.950331in}{1.893519in}}%
\pgfpathlineto{\pgfqpoint{5.950331in}{1.811790in}}%
\pgfusepath{}%
\end{pgfscope}%
\begin{pgfscope}%
\pgfpathrectangle{\pgfqpoint{0.549740in}{0.463273in}}{\pgfqpoint{9.320225in}{4.495057in}}%
\pgfusepath{clip}%
\pgfsetbuttcap%
\pgfsetroundjoin%
\pgfsetlinewidth{0.000000pt}%
\definecolor{currentstroke}{rgb}{0.000000,0.000000,0.000000}%
\pgfsetstrokecolor{currentstroke}%
\pgfsetdash{}{0pt}%
\pgfpathmoveto{\pgfqpoint{6.136557in}{1.811790in}}%
\pgfpathlineto{\pgfqpoint{6.322784in}{1.811790in}}%
\pgfpathlineto{\pgfqpoint{6.322784in}{1.893519in}}%
\pgfpathlineto{\pgfqpoint{6.136557in}{1.893519in}}%
\pgfpathlineto{\pgfqpoint{6.136557in}{1.811790in}}%
\pgfusepath{}%
\end{pgfscope}%
\begin{pgfscope}%
\pgfpathrectangle{\pgfqpoint{0.549740in}{0.463273in}}{\pgfqpoint{9.320225in}{4.495057in}}%
\pgfusepath{clip}%
\pgfsetbuttcap%
\pgfsetroundjoin%
\pgfsetlinewidth{0.000000pt}%
\definecolor{currentstroke}{rgb}{0.000000,0.000000,0.000000}%
\pgfsetstrokecolor{currentstroke}%
\pgfsetdash{}{0pt}%
\pgfpathmoveto{\pgfqpoint{6.322784in}{1.811790in}}%
\pgfpathlineto{\pgfqpoint{6.509011in}{1.811790in}}%
\pgfpathlineto{\pgfqpoint{6.509011in}{1.893519in}}%
\pgfpathlineto{\pgfqpoint{6.322784in}{1.893519in}}%
\pgfpathlineto{\pgfqpoint{6.322784in}{1.811790in}}%
\pgfusepath{}%
\end{pgfscope}%
\begin{pgfscope}%
\pgfpathrectangle{\pgfqpoint{0.549740in}{0.463273in}}{\pgfqpoint{9.320225in}{4.495057in}}%
\pgfusepath{clip}%
\pgfsetbuttcap%
\pgfsetroundjoin%
\definecolor{currentfill}{rgb}{0.472869,0.711325,0.955316}%
\pgfsetfillcolor{currentfill}%
\pgfsetlinewidth{0.000000pt}%
\definecolor{currentstroke}{rgb}{0.000000,0.000000,0.000000}%
\pgfsetstrokecolor{currentstroke}%
\pgfsetdash{}{0pt}%
\pgfpathmoveto{\pgfqpoint{6.509011in}{1.811790in}}%
\pgfpathlineto{\pgfqpoint{6.695237in}{1.811790in}}%
\pgfpathlineto{\pgfqpoint{6.695237in}{1.893519in}}%
\pgfpathlineto{\pgfqpoint{6.509011in}{1.893519in}}%
\pgfpathlineto{\pgfqpoint{6.509011in}{1.811790in}}%
\pgfusepath{fill}%
\end{pgfscope}%
\begin{pgfscope}%
\pgfpathrectangle{\pgfqpoint{0.549740in}{0.463273in}}{\pgfqpoint{9.320225in}{4.495057in}}%
\pgfusepath{clip}%
\pgfsetbuttcap%
\pgfsetroundjoin%
\pgfsetlinewidth{0.000000pt}%
\definecolor{currentstroke}{rgb}{0.000000,0.000000,0.000000}%
\pgfsetstrokecolor{currentstroke}%
\pgfsetdash{}{0pt}%
\pgfpathmoveto{\pgfqpoint{6.695237in}{1.811790in}}%
\pgfpathlineto{\pgfqpoint{6.881464in}{1.811790in}}%
\pgfpathlineto{\pgfqpoint{6.881464in}{1.893519in}}%
\pgfpathlineto{\pgfqpoint{6.695237in}{1.893519in}}%
\pgfpathlineto{\pgfqpoint{6.695237in}{1.811790in}}%
\pgfusepath{}%
\end{pgfscope}%
\begin{pgfscope}%
\pgfpathrectangle{\pgfqpoint{0.549740in}{0.463273in}}{\pgfqpoint{9.320225in}{4.495057in}}%
\pgfusepath{clip}%
\pgfsetbuttcap%
\pgfsetroundjoin%
\pgfsetlinewidth{0.000000pt}%
\definecolor{currentstroke}{rgb}{0.000000,0.000000,0.000000}%
\pgfsetstrokecolor{currentstroke}%
\pgfsetdash{}{0pt}%
\pgfpathmoveto{\pgfqpoint{6.881464in}{1.811790in}}%
\pgfpathlineto{\pgfqpoint{7.067690in}{1.811790in}}%
\pgfpathlineto{\pgfqpoint{7.067690in}{1.893519in}}%
\pgfpathlineto{\pgfqpoint{6.881464in}{1.893519in}}%
\pgfpathlineto{\pgfqpoint{6.881464in}{1.811790in}}%
\pgfusepath{}%
\end{pgfscope}%
\begin{pgfscope}%
\pgfpathrectangle{\pgfqpoint{0.549740in}{0.463273in}}{\pgfqpoint{9.320225in}{4.495057in}}%
\pgfusepath{clip}%
\pgfsetbuttcap%
\pgfsetroundjoin%
\pgfsetlinewidth{0.000000pt}%
\definecolor{currentstroke}{rgb}{0.000000,0.000000,0.000000}%
\pgfsetstrokecolor{currentstroke}%
\pgfsetdash{}{0pt}%
\pgfpathmoveto{\pgfqpoint{7.067690in}{1.811790in}}%
\pgfpathlineto{\pgfqpoint{7.253917in}{1.811790in}}%
\pgfpathlineto{\pgfqpoint{7.253917in}{1.893519in}}%
\pgfpathlineto{\pgfqpoint{7.067690in}{1.893519in}}%
\pgfpathlineto{\pgfqpoint{7.067690in}{1.811790in}}%
\pgfusepath{}%
\end{pgfscope}%
\begin{pgfscope}%
\pgfpathrectangle{\pgfqpoint{0.549740in}{0.463273in}}{\pgfqpoint{9.320225in}{4.495057in}}%
\pgfusepath{clip}%
\pgfsetbuttcap%
\pgfsetroundjoin%
\pgfsetlinewidth{0.000000pt}%
\definecolor{currentstroke}{rgb}{0.000000,0.000000,0.000000}%
\pgfsetstrokecolor{currentstroke}%
\pgfsetdash{}{0pt}%
\pgfpathmoveto{\pgfqpoint{7.253917in}{1.811790in}}%
\pgfpathlineto{\pgfqpoint{7.440143in}{1.811790in}}%
\pgfpathlineto{\pgfqpoint{7.440143in}{1.893519in}}%
\pgfpathlineto{\pgfqpoint{7.253917in}{1.893519in}}%
\pgfpathlineto{\pgfqpoint{7.253917in}{1.811790in}}%
\pgfusepath{}%
\end{pgfscope}%
\begin{pgfscope}%
\pgfpathrectangle{\pgfqpoint{0.549740in}{0.463273in}}{\pgfqpoint{9.320225in}{4.495057in}}%
\pgfusepath{clip}%
\pgfsetbuttcap%
\pgfsetroundjoin%
\definecolor{currentfill}{rgb}{0.472869,0.711325,0.955316}%
\pgfsetfillcolor{currentfill}%
\pgfsetlinewidth{0.000000pt}%
\definecolor{currentstroke}{rgb}{0.000000,0.000000,0.000000}%
\pgfsetstrokecolor{currentstroke}%
\pgfsetdash{}{0pt}%
\pgfpathmoveto{\pgfqpoint{7.440143in}{1.811790in}}%
\pgfpathlineto{\pgfqpoint{7.626370in}{1.811790in}}%
\pgfpathlineto{\pgfqpoint{7.626370in}{1.893519in}}%
\pgfpathlineto{\pgfqpoint{7.440143in}{1.893519in}}%
\pgfpathlineto{\pgfqpoint{7.440143in}{1.811790in}}%
\pgfusepath{fill}%
\end{pgfscope}%
\begin{pgfscope}%
\pgfpathrectangle{\pgfqpoint{0.549740in}{0.463273in}}{\pgfqpoint{9.320225in}{4.495057in}}%
\pgfusepath{clip}%
\pgfsetbuttcap%
\pgfsetroundjoin%
\definecolor{currentfill}{rgb}{0.547810,0.736432,0.947518}%
\pgfsetfillcolor{currentfill}%
\pgfsetlinewidth{0.000000pt}%
\definecolor{currentstroke}{rgb}{0.000000,0.000000,0.000000}%
\pgfsetstrokecolor{currentstroke}%
\pgfsetdash{}{0pt}%
\pgfpathmoveto{\pgfqpoint{7.626370in}{1.811790in}}%
\pgfpathlineto{\pgfqpoint{7.812596in}{1.811790in}}%
\pgfpathlineto{\pgfqpoint{7.812596in}{1.893519in}}%
\pgfpathlineto{\pgfqpoint{7.626370in}{1.893519in}}%
\pgfpathlineto{\pgfqpoint{7.626370in}{1.811790in}}%
\pgfusepath{fill}%
\end{pgfscope}%
\begin{pgfscope}%
\pgfpathrectangle{\pgfqpoint{0.549740in}{0.463273in}}{\pgfqpoint{9.320225in}{4.495057in}}%
\pgfusepath{clip}%
\pgfsetbuttcap%
\pgfsetroundjoin%
\pgfsetlinewidth{0.000000pt}%
\definecolor{currentstroke}{rgb}{0.000000,0.000000,0.000000}%
\pgfsetstrokecolor{currentstroke}%
\pgfsetdash{}{0pt}%
\pgfpathmoveto{\pgfqpoint{7.812596in}{1.811790in}}%
\pgfpathlineto{\pgfqpoint{7.998823in}{1.811790in}}%
\pgfpathlineto{\pgfqpoint{7.998823in}{1.893519in}}%
\pgfpathlineto{\pgfqpoint{7.812596in}{1.893519in}}%
\pgfpathlineto{\pgfqpoint{7.812596in}{1.811790in}}%
\pgfusepath{}%
\end{pgfscope}%
\begin{pgfscope}%
\pgfpathrectangle{\pgfqpoint{0.549740in}{0.463273in}}{\pgfqpoint{9.320225in}{4.495057in}}%
\pgfusepath{clip}%
\pgfsetbuttcap%
\pgfsetroundjoin%
\pgfsetlinewidth{0.000000pt}%
\definecolor{currentstroke}{rgb}{0.000000,0.000000,0.000000}%
\pgfsetstrokecolor{currentstroke}%
\pgfsetdash{}{0pt}%
\pgfpathmoveto{\pgfqpoint{7.998823in}{1.811790in}}%
\pgfpathlineto{\pgfqpoint{8.185049in}{1.811790in}}%
\pgfpathlineto{\pgfqpoint{8.185049in}{1.893519in}}%
\pgfpathlineto{\pgfqpoint{7.998823in}{1.893519in}}%
\pgfpathlineto{\pgfqpoint{7.998823in}{1.811790in}}%
\pgfusepath{}%
\end{pgfscope}%
\begin{pgfscope}%
\pgfpathrectangle{\pgfqpoint{0.549740in}{0.463273in}}{\pgfqpoint{9.320225in}{4.495057in}}%
\pgfusepath{clip}%
\pgfsetbuttcap%
\pgfsetroundjoin%
\pgfsetlinewidth{0.000000pt}%
\definecolor{currentstroke}{rgb}{0.000000,0.000000,0.000000}%
\pgfsetstrokecolor{currentstroke}%
\pgfsetdash{}{0pt}%
\pgfpathmoveto{\pgfqpoint{8.185049in}{1.811790in}}%
\pgfpathlineto{\pgfqpoint{8.371276in}{1.811790in}}%
\pgfpathlineto{\pgfqpoint{8.371276in}{1.893519in}}%
\pgfpathlineto{\pgfqpoint{8.185049in}{1.893519in}}%
\pgfpathlineto{\pgfqpoint{8.185049in}{1.811790in}}%
\pgfusepath{}%
\end{pgfscope}%
\begin{pgfscope}%
\pgfpathrectangle{\pgfqpoint{0.549740in}{0.463273in}}{\pgfqpoint{9.320225in}{4.495057in}}%
\pgfusepath{clip}%
\pgfsetbuttcap%
\pgfsetroundjoin%
\pgfsetlinewidth{0.000000pt}%
\definecolor{currentstroke}{rgb}{0.000000,0.000000,0.000000}%
\pgfsetstrokecolor{currentstroke}%
\pgfsetdash{}{0pt}%
\pgfpathmoveto{\pgfqpoint{8.371276in}{1.811790in}}%
\pgfpathlineto{\pgfqpoint{8.557503in}{1.811790in}}%
\pgfpathlineto{\pgfqpoint{8.557503in}{1.893519in}}%
\pgfpathlineto{\pgfqpoint{8.371276in}{1.893519in}}%
\pgfpathlineto{\pgfqpoint{8.371276in}{1.811790in}}%
\pgfusepath{}%
\end{pgfscope}%
\begin{pgfscope}%
\pgfpathrectangle{\pgfqpoint{0.549740in}{0.463273in}}{\pgfqpoint{9.320225in}{4.495057in}}%
\pgfusepath{clip}%
\pgfsetbuttcap%
\pgfsetroundjoin%
\definecolor{currentfill}{rgb}{0.472869,0.711325,0.955316}%
\pgfsetfillcolor{currentfill}%
\pgfsetlinewidth{0.000000pt}%
\definecolor{currentstroke}{rgb}{0.000000,0.000000,0.000000}%
\pgfsetstrokecolor{currentstroke}%
\pgfsetdash{}{0pt}%
\pgfpathmoveto{\pgfqpoint{8.557503in}{1.811790in}}%
\pgfpathlineto{\pgfqpoint{8.743729in}{1.811790in}}%
\pgfpathlineto{\pgfqpoint{8.743729in}{1.893519in}}%
\pgfpathlineto{\pgfqpoint{8.557503in}{1.893519in}}%
\pgfpathlineto{\pgfqpoint{8.557503in}{1.811790in}}%
\pgfusepath{fill}%
\end{pgfscope}%
\begin{pgfscope}%
\pgfpathrectangle{\pgfqpoint{0.549740in}{0.463273in}}{\pgfqpoint{9.320225in}{4.495057in}}%
\pgfusepath{clip}%
\pgfsetbuttcap%
\pgfsetroundjoin%
\pgfsetlinewidth{0.000000pt}%
\definecolor{currentstroke}{rgb}{0.000000,0.000000,0.000000}%
\pgfsetstrokecolor{currentstroke}%
\pgfsetdash{}{0pt}%
\pgfpathmoveto{\pgfqpoint{8.743729in}{1.811790in}}%
\pgfpathlineto{\pgfqpoint{8.929956in}{1.811790in}}%
\pgfpathlineto{\pgfqpoint{8.929956in}{1.893519in}}%
\pgfpathlineto{\pgfqpoint{8.743729in}{1.893519in}}%
\pgfpathlineto{\pgfqpoint{8.743729in}{1.811790in}}%
\pgfusepath{}%
\end{pgfscope}%
\begin{pgfscope}%
\pgfpathrectangle{\pgfqpoint{0.549740in}{0.463273in}}{\pgfqpoint{9.320225in}{4.495057in}}%
\pgfusepath{clip}%
\pgfsetbuttcap%
\pgfsetroundjoin%
\pgfsetlinewidth{0.000000pt}%
\definecolor{currentstroke}{rgb}{0.000000,0.000000,0.000000}%
\pgfsetstrokecolor{currentstroke}%
\pgfsetdash{}{0pt}%
\pgfpathmoveto{\pgfqpoint{8.929956in}{1.811790in}}%
\pgfpathlineto{\pgfqpoint{9.116182in}{1.811790in}}%
\pgfpathlineto{\pgfqpoint{9.116182in}{1.893519in}}%
\pgfpathlineto{\pgfqpoint{8.929956in}{1.893519in}}%
\pgfpathlineto{\pgfqpoint{8.929956in}{1.811790in}}%
\pgfusepath{}%
\end{pgfscope}%
\begin{pgfscope}%
\pgfpathrectangle{\pgfqpoint{0.549740in}{0.463273in}}{\pgfqpoint{9.320225in}{4.495057in}}%
\pgfusepath{clip}%
\pgfsetbuttcap%
\pgfsetroundjoin%
\pgfsetlinewidth{0.000000pt}%
\definecolor{currentstroke}{rgb}{0.000000,0.000000,0.000000}%
\pgfsetstrokecolor{currentstroke}%
\pgfsetdash{}{0pt}%
\pgfpathmoveto{\pgfqpoint{9.116182in}{1.811790in}}%
\pgfpathlineto{\pgfqpoint{9.302409in}{1.811790in}}%
\pgfpathlineto{\pgfqpoint{9.302409in}{1.893519in}}%
\pgfpathlineto{\pgfqpoint{9.116182in}{1.893519in}}%
\pgfpathlineto{\pgfqpoint{9.116182in}{1.811790in}}%
\pgfusepath{}%
\end{pgfscope}%
\begin{pgfscope}%
\pgfpathrectangle{\pgfqpoint{0.549740in}{0.463273in}}{\pgfqpoint{9.320225in}{4.495057in}}%
\pgfusepath{clip}%
\pgfsetbuttcap%
\pgfsetroundjoin%
\pgfsetlinewidth{0.000000pt}%
\definecolor{currentstroke}{rgb}{0.000000,0.000000,0.000000}%
\pgfsetstrokecolor{currentstroke}%
\pgfsetdash{}{0pt}%
\pgfpathmoveto{\pgfqpoint{9.302409in}{1.811790in}}%
\pgfpathlineto{\pgfqpoint{9.488635in}{1.811790in}}%
\pgfpathlineto{\pgfqpoint{9.488635in}{1.893519in}}%
\pgfpathlineto{\pgfqpoint{9.302409in}{1.893519in}}%
\pgfpathlineto{\pgfqpoint{9.302409in}{1.811790in}}%
\pgfusepath{}%
\end{pgfscope}%
\begin{pgfscope}%
\pgfpathrectangle{\pgfqpoint{0.549740in}{0.463273in}}{\pgfqpoint{9.320225in}{4.495057in}}%
\pgfusepath{clip}%
\pgfsetbuttcap%
\pgfsetroundjoin%
\pgfsetlinewidth{0.000000pt}%
\definecolor{currentstroke}{rgb}{0.000000,0.000000,0.000000}%
\pgfsetstrokecolor{currentstroke}%
\pgfsetdash{}{0pt}%
\pgfpathmoveto{\pgfqpoint{9.488635in}{1.811790in}}%
\pgfpathlineto{\pgfqpoint{9.674862in}{1.811790in}}%
\pgfpathlineto{\pgfqpoint{9.674862in}{1.893519in}}%
\pgfpathlineto{\pgfqpoint{9.488635in}{1.893519in}}%
\pgfpathlineto{\pgfqpoint{9.488635in}{1.811790in}}%
\pgfusepath{}%
\end{pgfscope}%
\begin{pgfscope}%
\pgfpathrectangle{\pgfqpoint{0.549740in}{0.463273in}}{\pgfqpoint{9.320225in}{4.495057in}}%
\pgfusepath{clip}%
\pgfsetbuttcap%
\pgfsetroundjoin%
\pgfsetlinewidth{0.000000pt}%
\definecolor{currentstroke}{rgb}{0.000000,0.000000,0.000000}%
\pgfsetstrokecolor{currentstroke}%
\pgfsetdash{}{0pt}%
\pgfpathmoveto{\pgfqpoint{9.674862in}{1.811790in}}%
\pgfpathlineto{\pgfqpoint{9.861088in}{1.811790in}}%
\pgfpathlineto{\pgfqpoint{9.861088in}{1.893519in}}%
\pgfpathlineto{\pgfqpoint{9.674862in}{1.893519in}}%
\pgfpathlineto{\pgfqpoint{9.674862in}{1.811790in}}%
\pgfusepath{}%
\end{pgfscope}%
\begin{pgfscope}%
\pgfpathrectangle{\pgfqpoint{0.549740in}{0.463273in}}{\pgfqpoint{9.320225in}{4.495057in}}%
\pgfusepath{clip}%
\pgfsetbuttcap%
\pgfsetroundjoin%
\pgfsetlinewidth{0.000000pt}%
\definecolor{currentstroke}{rgb}{0.000000,0.000000,0.000000}%
\pgfsetstrokecolor{currentstroke}%
\pgfsetdash{}{0pt}%
\pgfpathmoveto{\pgfqpoint{0.549761in}{1.893519in}}%
\pgfpathlineto{\pgfqpoint{0.735988in}{1.893519in}}%
\pgfpathlineto{\pgfqpoint{0.735988in}{1.975247in}}%
\pgfpathlineto{\pgfqpoint{0.549761in}{1.975247in}}%
\pgfpathlineto{\pgfqpoint{0.549761in}{1.893519in}}%
\pgfusepath{}%
\end{pgfscope}%
\begin{pgfscope}%
\pgfpathrectangle{\pgfqpoint{0.549740in}{0.463273in}}{\pgfqpoint{9.320225in}{4.495057in}}%
\pgfusepath{clip}%
\pgfsetbuttcap%
\pgfsetroundjoin%
\pgfsetlinewidth{0.000000pt}%
\definecolor{currentstroke}{rgb}{0.000000,0.000000,0.000000}%
\pgfsetstrokecolor{currentstroke}%
\pgfsetdash{}{0pt}%
\pgfpathmoveto{\pgfqpoint{0.735988in}{1.893519in}}%
\pgfpathlineto{\pgfqpoint{0.922214in}{1.893519in}}%
\pgfpathlineto{\pgfqpoint{0.922214in}{1.975247in}}%
\pgfpathlineto{\pgfqpoint{0.735988in}{1.975247in}}%
\pgfpathlineto{\pgfqpoint{0.735988in}{1.893519in}}%
\pgfusepath{}%
\end{pgfscope}%
\begin{pgfscope}%
\pgfpathrectangle{\pgfqpoint{0.549740in}{0.463273in}}{\pgfqpoint{9.320225in}{4.495057in}}%
\pgfusepath{clip}%
\pgfsetbuttcap%
\pgfsetroundjoin%
\pgfsetlinewidth{0.000000pt}%
\definecolor{currentstroke}{rgb}{0.000000,0.000000,0.000000}%
\pgfsetstrokecolor{currentstroke}%
\pgfsetdash{}{0pt}%
\pgfpathmoveto{\pgfqpoint{0.922214in}{1.893519in}}%
\pgfpathlineto{\pgfqpoint{1.108441in}{1.893519in}}%
\pgfpathlineto{\pgfqpoint{1.108441in}{1.975247in}}%
\pgfpathlineto{\pgfqpoint{0.922214in}{1.975247in}}%
\pgfpathlineto{\pgfqpoint{0.922214in}{1.893519in}}%
\pgfusepath{}%
\end{pgfscope}%
\begin{pgfscope}%
\pgfpathrectangle{\pgfqpoint{0.549740in}{0.463273in}}{\pgfqpoint{9.320225in}{4.495057in}}%
\pgfusepath{clip}%
\pgfsetbuttcap%
\pgfsetroundjoin%
\pgfsetlinewidth{0.000000pt}%
\definecolor{currentstroke}{rgb}{0.000000,0.000000,0.000000}%
\pgfsetstrokecolor{currentstroke}%
\pgfsetdash{}{0pt}%
\pgfpathmoveto{\pgfqpoint{1.108441in}{1.893519in}}%
\pgfpathlineto{\pgfqpoint{1.294667in}{1.893519in}}%
\pgfpathlineto{\pgfqpoint{1.294667in}{1.975247in}}%
\pgfpathlineto{\pgfqpoint{1.108441in}{1.975247in}}%
\pgfpathlineto{\pgfqpoint{1.108441in}{1.893519in}}%
\pgfusepath{}%
\end{pgfscope}%
\begin{pgfscope}%
\pgfpathrectangle{\pgfqpoint{0.549740in}{0.463273in}}{\pgfqpoint{9.320225in}{4.495057in}}%
\pgfusepath{clip}%
\pgfsetbuttcap%
\pgfsetroundjoin%
\pgfsetlinewidth{0.000000pt}%
\definecolor{currentstroke}{rgb}{0.000000,0.000000,0.000000}%
\pgfsetstrokecolor{currentstroke}%
\pgfsetdash{}{0pt}%
\pgfpathmoveto{\pgfqpoint{1.294667in}{1.893519in}}%
\pgfpathlineto{\pgfqpoint{1.480894in}{1.893519in}}%
\pgfpathlineto{\pgfqpoint{1.480894in}{1.975247in}}%
\pgfpathlineto{\pgfqpoint{1.294667in}{1.975247in}}%
\pgfpathlineto{\pgfqpoint{1.294667in}{1.893519in}}%
\pgfusepath{}%
\end{pgfscope}%
\begin{pgfscope}%
\pgfpathrectangle{\pgfqpoint{0.549740in}{0.463273in}}{\pgfqpoint{9.320225in}{4.495057in}}%
\pgfusepath{clip}%
\pgfsetbuttcap%
\pgfsetroundjoin%
\pgfsetlinewidth{0.000000pt}%
\definecolor{currentstroke}{rgb}{0.000000,0.000000,0.000000}%
\pgfsetstrokecolor{currentstroke}%
\pgfsetdash{}{0pt}%
\pgfpathmoveto{\pgfqpoint{1.480894in}{1.893519in}}%
\pgfpathlineto{\pgfqpoint{1.667120in}{1.893519in}}%
\pgfpathlineto{\pgfqpoint{1.667120in}{1.975247in}}%
\pgfpathlineto{\pgfqpoint{1.480894in}{1.975247in}}%
\pgfpathlineto{\pgfqpoint{1.480894in}{1.893519in}}%
\pgfusepath{}%
\end{pgfscope}%
\begin{pgfscope}%
\pgfpathrectangle{\pgfqpoint{0.549740in}{0.463273in}}{\pgfqpoint{9.320225in}{4.495057in}}%
\pgfusepath{clip}%
\pgfsetbuttcap%
\pgfsetroundjoin%
\pgfsetlinewidth{0.000000pt}%
\definecolor{currentstroke}{rgb}{0.000000,0.000000,0.000000}%
\pgfsetstrokecolor{currentstroke}%
\pgfsetdash{}{0pt}%
\pgfpathmoveto{\pgfqpoint{1.667120in}{1.893519in}}%
\pgfpathlineto{\pgfqpoint{1.853347in}{1.893519in}}%
\pgfpathlineto{\pgfqpoint{1.853347in}{1.975247in}}%
\pgfpathlineto{\pgfqpoint{1.667120in}{1.975247in}}%
\pgfpathlineto{\pgfqpoint{1.667120in}{1.893519in}}%
\pgfusepath{}%
\end{pgfscope}%
\begin{pgfscope}%
\pgfpathrectangle{\pgfqpoint{0.549740in}{0.463273in}}{\pgfqpoint{9.320225in}{4.495057in}}%
\pgfusepath{clip}%
\pgfsetbuttcap%
\pgfsetroundjoin%
\pgfsetlinewidth{0.000000pt}%
\definecolor{currentstroke}{rgb}{0.000000,0.000000,0.000000}%
\pgfsetstrokecolor{currentstroke}%
\pgfsetdash{}{0pt}%
\pgfpathmoveto{\pgfqpoint{1.853347in}{1.893519in}}%
\pgfpathlineto{\pgfqpoint{2.039573in}{1.893519in}}%
\pgfpathlineto{\pgfqpoint{2.039573in}{1.975247in}}%
\pgfpathlineto{\pgfqpoint{1.853347in}{1.975247in}}%
\pgfpathlineto{\pgfqpoint{1.853347in}{1.893519in}}%
\pgfusepath{}%
\end{pgfscope}%
\begin{pgfscope}%
\pgfpathrectangle{\pgfqpoint{0.549740in}{0.463273in}}{\pgfqpoint{9.320225in}{4.495057in}}%
\pgfusepath{clip}%
\pgfsetbuttcap%
\pgfsetroundjoin%
\pgfsetlinewidth{0.000000pt}%
\definecolor{currentstroke}{rgb}{0.000000,0.000000,0.000000}%
\pgfsetstrokecolor{currentstroke}%
\pgfsetdash{}{0pt}%
\pgfpathmoveto{\pgfqpoint{2.039573in}{1.893519in}}%
\pgfpathlineto{\pgfqpoint{2.225800in}{1.893519in}}%
\pgfpathlineto{\pgfqpoint{2.225800in}{1.975247in}}%
\pgfpathlineto{\pgfqpoint{2.039573in}{1.975247in}}%
\pgfpathlineto{\pgfqpoint{2.039573in}{1.893519in}}%
\pgfusepath{}%
\end{pgfscope}%
\begin{pgfscope}%
\pgfpathrectangle{\pgfqpoint{0.549740in}{0.463273in}}{\pgfqpoint{9.320225in}{4.495057in}}%
\pgfusepath{clip}%
\pgfsetbuttcap%
\pgfsetroundjoin%
\pgfsetlinewidth{0.000000pt}%
\definecolor{currentstroke}{rgb}{0.000000,0.000000,0.000000}%
\pgfsetstrokecolor{currentstroke}%
\pgfsetdash{}{0pt}%
\pgfpathmoveto{\pgfqpoint{2.225800in}{1.893519in}}%
\pgfpathlineto{\pgfqpoint{2.412027in}{1.893519in}}%
\pgfpathlineto{\pgfqpoint{2.412027in}{1.975247in}}%
\pgfpathlineto{\pgfqpoint{2.225800in}{1.975247in}}%
\pgfpathlineto{\pgfqpoint{2.225800in}{1.893519in}}%
\pgfusepath{}%
\end{pgfscope}%
\begin{pgfscope}%
\pgfpathrectangle{\pgfqpoint{0.549740in}{0.463273in}}{\pgfqpoint{9.320225in}{4.495057in}}%
\pgfusepath{clip}%
\pgfsetbuttcap%
\pgfsetroundjoin%
\pgfsetlinewidth{0.000000pt}%
\definecolor{currentstroke}{rgb}{0.000000,0.000000,0.000000}%
\pgfsetstrokecolor{currentstroke}%
\pgfsetdash{}{0pt}%
\pgfpathmoveto{\pgfqpoint{2.412027in}{1.893519in}}%
\pgfpathlineto{\pgfqpoint{2.598253in}{1.893519in}}%
\pgfpathlineto{\pgfqpoint{2.598253in}{1.975247in}}%
\pgfpathlineto{\pgfqpoint{2.412027in}{1.975247in}}%
\pgfpathlineto{\pgfqpoint{2.412027in}{1.893519in}}%
\pgfusepath{}%
\end{pgfscope}%
\begin{pgfscope}%
\pgfpathrectangle{\pgfqpoint{0.549740in}{0.463273in}}{\pgfqpoint{9.320225in}{4.495057in}}%
\pgfusepath{clip}%
\pgfsetbuttcap%
\pgfsetroundjoin%
\pgfsetlinewidth{0.000000pt}%
\definecolor{currentstroke}{rgb}{0.000000,0.000000,0.000000}%
\pgfsetstrokecolor{currentstroke}%
\pgfsetdash{}{0pt}%
\pgfpathmoveto{\pgfqpoint{2.598253in}{1.893519in}}%
\pgfpathlineto{\pgfqpoint{2.784480in}{1.893519in}}%
\pgfpathlineto{\pgfqpoint{2.784480in}{1.975247in}}%
\pgfpathlineto{\pgfqpoint{2.598253in}{1.975247in}}%
\pgfpathlineto{\pgfqpoint{2.598253in}{1.893519in}}%
\pgfusepath{}%
\end{pgfscope}%
\begin{pgfscope}%
\pgfpathrectangle{\pgfqpoint{0.549740in}{0.463273in}}{\pgfqpoint{9.320225in}{4.495057in}}%
\pgfusepath{clip}%
\pgfsetbuttcap%
\pgfsetroundjoin%
\pgfsetlinewidth{0.000000pt}%
\definecolor{currentstroke}{rgb}{0.000000,0.000000,0.000000}%
\pgfsetstrokecolor{currentstroke}%
\pgfsetdash{}{0pt}%
\pgfpathmoveto{\pgfqpoint{2.784480in}{1.893519in}}%
\pgfpathlineto{\pgfqpoint{2.970706in}{1.893519in}}%
\pgfpathlineto{\pgfqpoint{2.970706in}{1.975247in}}%
\pgfpathlineto{\pgfqpoint{2.784480in}{1.975247in}}%
\pgfpathlineto{\pgfqpoint{2.784480in}{1.893519in}}%
\pgfusepath{}%
\end{pgfscope}%
\begin{pgfscope}%
\pgfpathrectangle{\pgfqpoint{0.549740in}{0.463273in}}{\pgfqpoint{9.320225in}{4.495057in}}%
\pgfusepath{clip}%
\pgfsetbuttcap%
\pgfsetroundjoin%
\pgfsetlinewidth{0.000000pt}%
\definecolor{currentstroke}{rgb}{0.000000,0.000000,0.000000}%
\pgfsetstrokecolor{currentstroke}%
\pgfsetdash{}{0pt}%
\pgfpathmoveto{\pgfqpoint{2.970706in}{1.893519in}}%
\pgfpathlineto{\pgfqpoint{3.156933in}{1.893519in}}%
\pgfpathlineto{\pgfqpoint{3.156933in}{1.975247in}}%
\pgfpathlineto{\pgfqpoint{2.970706in}{1.975247in}}%
\pgfpathlineto{\pgfqpoint{2.970706in}{1.893519in}}%
\pgfusepath{}%
\end{pgfscope}%
\begin{pgfscope}%
\pgfpathrectangle{\pgfqpoint{0.549740in}{0.463273in}}{\pgfqpoint{9.320225in}{4.495057in}}%
\pgfusepath{clip}%
\pgfsetbuttcap%
\pgfsetroundjoin%
\pgfsetlinewidth{0.000000pt}%
\definecolor{currentstroke}{rgb}{0.000000,0.000000,0.000000}%
\pgfsetstrokecolor{currentstroke}%
\pgfsetdash{}{0pt}%
\pgfpathmoveto{\pgfqpoint{3.156933in}{1.893519in}}%
\pgfpathlineto{\pgfqpoint{3.343159in}{1.893519in}}%
\pgfpathlineto{\pgfqpoint{3.343159in}{1.975247in}}%
\pgfpathlineto{\pgfqpoint{3.156933in}{1.975247in}}%
\pgfpathlineto{\pgfqpoint{3.156933in}{1.893519in}}%
\pgfusepath{}%
\end{pgfscope}%
\begin{pgfscope}%
\pgfpathrectangle{\pgfqpoint{0.549740in}{0.463273in}}{\pgfqpoint{9.320225in}{4.495057in}}%
\pgfusepath{clip}%
\pgfsetbuttcap%
\pgfsetroundjoin%
\pgfsetlinewidth{0.000000pt}%
\definecolor{currentstroke}{rgb}{0.000000,0.000000,0.000000}%
\pgfsetstrokecolor{currentstroke}%
\pgfsetdash{}{0pt}%
\pgfpathmoveto{\pgfqpoint{3.343159in}{1.893519in}}%
\pgfpathlineto{\pgfqpoint{3.529386in}{1.893519in}}%
\pgfpathlineto{\pgfqpoint{3.529386in}{1.975247in}}%
\pgfpathlineto{\pgfqpoint{3.343159in}{1.975247in}}%
\pgfpathlineto{\pgfqpoint{3.343159in}{1.893519in}}%
\pgfusepath{}%
\end{pgfscope}%
\begin{pgfscope}%
\pgfpathrectangle{\pgfqpoint{0.549740in}{0.463273in}}{\pgfqpoint{9.320225in}{4.495057in}}%
\pgfusepath{clip}%
\pgfsetbuttcap%
\pgfsetroundjoin%
\pgfsetlinewidth{0.000000pt}%
\definecolor{currentstroke}{rgb}{0.000000,0.000000,0.000000}%
\pgfsetstrokecolor{currentstroke}%
\pgfsetdash{}{0pt}%
\pgfpathmoveto{\pgfqpoint{3.529386in}{1.893519in}}%
\pgfpathlineto{\pgfqpoint{3.715612in}{1.893519in}}%
\pgfpathlineto{\pgfqpoint{3.715612in}{1.975247in}}%
\pgfpathlineto{\pgfqpoint{3.529386in}{1.975247in}}%
\pgfpathlineto{\pgfqpoint{3.529386in}{1.893519in}}%
\pgfusepath{}%
\end{pgfscope}%
\begin{pgfscope}%
\pgfpathrectangle{\pgfqpoint{0.549740in}{0.463273in}}{\pgfqpoint{9.320225in}{4.495057in}}%
\pgfusepath{clip}%
\pgfsetbuttcap%
\pgfsetroundjoin%
\pgfsetlinewidth{0.000000pt}%
\definecolor{currentstroke}{rgb}{0.000000,0.000000,0.000000}%
\pgfsetstrokecolor{currentstroke}%
\pgfsetdash{}{0pt}%
\pgfpathmoveto{\pgfqpoint{3.715612in}{1.893519in}}%
\pgfpathlineto{\pgfqpoint{3.901839in}{1.893519in}}%
\pgfpathlineto{\pgfqpoint{3.901839in}{1.975247in}}%
\pgfpathlineto{\pgfqpoint{3.715612in}{1.975247in}}%
\pgfpathlineto{\pgfqpoint{3.715612in}{1.893519in}}%
\pgfusepath{}%
\end{pgfscope}%
\begin{pgfscope}%
\pgfpathrectangle{\pgfqpoint{0.549740in}{0.463273in}}{\pgfqpoint{9.320225in}{4.495057in}}%
\pgfusepath{clip}%
\pgfsetbuttcap%
\pgfsetroundjoin%
\pgfsetlinewidth{0.000000pt}%
\definecolor{currentstroke}{rgb}{0.000000,0.000000,0.000000}%
\pgfsetstrokecolor{currentstroke}%
\pgfsetdash{}{0pt}%
\pgfpathmoveto{\pgfqpoint{3.901839in}{1.893519in}}%
\pgfpathlineto{\pgfqpoint{4.088065in}{1.893519in}}%
\pgfpathlineto{\pgfqpoint{4.088065in}{1.975247in}}%
\pgfpathlineto{\pgfqpoint{3.901839in}{1.975247in}}%
\pgfpathlineto{\pgfqpoint{3.901839in}{1.893519in}}%
\pgfusepath{}%
\end{pgfscope}%
\begin{pgfscope}%
\pgfpathrectangle{\pgfqpoint{0.549740in}{0.463273in}}{\pgfqpoint{9.320225in}{4.495057in}}%
\pgfusepath{clip}%
\pgfsetbuttcap%
\pgfsetroundjoin%
\pgfsetlinewidth{0.000000pt}%
\definecolor{currentstroke}{rgb}{0.000000,0.000000,0.000000}%
\pgfsetstrokecolor{currentstroke}%
\pgfsetdash{}{0pt}%
\pgfpathmoveto{\pgfqpoint{4.088065in}{1.893519in}}%
\pgfpathlineto{\pgfqpoint{4.274292in}{1.893519in}}%
\pgfpathlineto{\pgfqpoint{4.274292in}{1.975247in}}%
\pgfpathlineto{\pgfqpoint{4.088065in}{1.975247in}}%
\pgfpathlineto{\pgfqpoint{4.088065in}{1.893519in}}%
\pgfusepath{}%
\end{pgfscope}%
\begin{pgfscope}%
\pgfpathrectangle{\pgfqpoint{0.549740in}{0.463273in}}{\pgfqpoint{9.320225in}{4.495057in}}%
\pgfusepath{clip}%
\pgfsetbuttcap%
\pgfsetroundjoin%
\pgfsetlinewidth{0.000000pt}%
\definecolor{currentstroke}{rgb}{0.000000,0.000000,0.000000}%
\pgfsetstrokecolor{currentstroke}%
\pgfsetdash{}{0pt}%
\pgfpathmoveto{\pgfqpoint{4.274292in}{1.893519in}}%
\pgfpathlineto{\pgfqpoint{4.460519in}{1.893519in}}%
\pgfpathlineto{\pgfqpoint{4.460519in}{1.975247in}}%
\pgfpathlineto{\pgfqpoint{4.274292in}{1.975247in}}%
\pgfpathlineto{\pgfqpoint{4.274292in}{1.893519in}}%
\pgfusepath{}%
\end{pgfscope}%
\begin{pgfscope}%
\pgfpathrectangle{\pgfqpoint{0.549740in}{0.463273in}}{\pgfqpoint{9.320225in}{4.495057in}}%
\pgfusepath{clip}%
\pgfsetbuttcap%
\pgfsetroundjoin%
\pgfsetlinewidth{0.000000pt}%
\definecolor{currentstroke}{rgb}{0.000000,0.000000,0.000000}%
\pgfsetstrokecolor{currentstroke}%
\pgfsetdash{}{0pt}%
\pgfpathmoveto{\pgfqpoint{4.460519in}{1.893519in}}%
\pgfpathlineto{\pgfqpoint{4.646745in}{1.893519in}}%
\pgfpathlineto{\pgfqpoint{4.646745in}{1.975247in}}%
\pgfpathlineto{\pgfqpoint{4.460519in}{1.975247in}}%
\pgfpathlineto{\pgfqpoint{4.460519in}{1.893519in}}%
\pgfusepath{}%
\end{pgfscope}%
\begin{pgfscope}%
\pgfpathrectangle{\pgfqpoint{0.549740in}{0.463273in}}{\pgfqpoint{9.320225in}{4.495057in}}%
\pgfusepath{clip}%
\pgfsetbuttcap%
\pgfsetroundjoin%
\pgfsetlinewidth{0.000000pt}%
\definecolor{currentstroke}{rgb}{0.000000,0.000000,0.000000}%
\pgfsetstrokecolor{currentstroke}%
\pgfsetdash{}{0pt}%
\pgfpathmoveto{\pgfqpoint{4.646745in}{1.893519in}}%
\pgfpathlineto{\pgfqpoint{4.832972in}{1.893519in}}%
\pgfpathlineto{\pgfqpoint{4.832972in}{1.975247in}}%
\pgfpathlineto{\pgfqpoint{4.646745in}{1.975247in}}%
\pgfpathlineto{\pgfqpoint{4.646745in}{1.893519in}}%
\pgfusepath{}%
\end{pgfscope}%
\begin{pgfscope}%
\pgfpathrectangle{\pgfqpoint{0.549740in}{0.463273in}}{\pgfqpoint{9.320225in}{4.495057in}}%
\pgfusepath{clip}%
\pgfsetbuttcap%
\pgfsetroundjoin%
\pgfsetlinewidth{0.000000pt}%
\definecolor{currentstroke}{rgb}{0.000000,0.000000,0.000000}%
\pgfsetstrokecolor{currentstroke}%
\pgfsetdash{}{0pt}%
\pgfpathmoveto{\pgfqpoint{4.832972in}{1.893519in}}%
\pgfpathlineto{\pgfqpoint{5.019198in}{1.893519in}}%
\pgfpathlineto{\pgfqpoint{5.019198in}{1.975247in}}%
\pgfpathlineto{\pgfqpoint{4.832972in}{1.975247in}}%
\pgfpathlineto{\pgfqpoint{4.832972in}{1.893519in}}%
\pgfusepath{}%
\end{pgfscope}%
\begin{pgfscope}%
\pgfpathrectangle{\pgfqpoint{0.549740in}{0.463273in}}{\pgfqpoint{9.320225in}{4.495057in}}%
\pgfusepath{clip}%
\pgfsetbuttcap%
\pgfsetroundjoin%
\pgfsetlinewidth{0.000000pt}%
\definecolor{currentstroke}{rgb}{0.000000,0.000000,0.000000}%
\pgfsetstrokecolor{currentstroke}%
\pgfsetdash{}{0pt}%
\pgfpathmoveto{\pgfqpoint{5.019198in}{1.893519in}}%
\pgfpathlineto{\pgfqpoint{5.205425in}{1.893519in}}%
\pgfpathlineto{\pgfqpoint{5.205425in}{1.975247in}}%
\pgfpathlineto{\pgfqpoint{5.019198in}{1.975247in}}%
\pgfpathlineto{\pgfqpoint{5.019198in}{1.893519in}}%
\pgfusepath{}%
\end{pgfscope}%
\begin{pgfscope}%
\pgfpathrectangle{\pgfqpoint{0.549740in}{0.463273in}}{\pgfqpoint{9.320225in}{4.495057in}}%
\pgfusepath{clip}%
\pgfsetbuttcap%
\pgfsetroundjoin%
\pgfsetlinewidth{0.000000pt}%
\definecolor{currentstroke}{rgb}{0.000000,0.000000,0.000000}%
\pgfsetstrokecolor{currentstroke}%
\pgfsetdash{}{0pt}%
\pgfpathmoveto{\pgfqpoint{5.205425in}{1.893519in}}%
\pgfpathlineto{\pgfqpoint{5.391651in}{1.893519in}}%
\pgfpathlineto{\pgfqpoint{5.391651in}{1.975247in}}%
\pgfpathlineto{\pgfqpoint{5.205425in}{1.975247in}}%
\pgfpathlineto{\pgfqpoint{5.205425in}{1.893519in}}%
\pgfusepath{}%
\end{pgfscope}%
\begin{pgfscope}%
\pgfpathrectangle{\pgfqpoint{0.549740in}{0.463273in}}{\pgfqpoint{9.320225in}{4.495057in}}%
\pgfusepath{clip}%
\pgfsetbuttcap%
\pgfsetroundjoin%
\pgfsetlinewidth{0.000000pt}%
\definecolor{currentstroke}{rgb}{0.000000,0.000000,0.000000}%
\pgfsetstrokecolor{currentstroke}%
\pgfsetdash{}{0pt}%
\pgfpathmoveto{\pgfqpoint{5.391651in}{1.893519in}}%
\pgfpathlineto{\pgfqpoint{5.577878in}{1.893519in}}%
\pgfpathlineto{\pgfqpoint{5.577878in}{1.975247in}}%
\pgfpathlineto{\pgfqpoint{5.391651in}{1.975247in}}%
\pgfpathlineto{\pgfqpoint{5.391651in}{1.893519in}}%
\pgfusepath{}%
\end{pgfscope}%
\begin{pgfscope}%
\pgfpathrectangle{\pgfqpoint{0.549740in}{0.463273in}}{\pgfqpoint{9.320225in}{4.495057in}}%
\pgfusepath{clip}%
\pgfsetbuttcap%
\pgfsetroundjoin%
\definecolor{currentfill}{rgb}{0.472869,0.711325,0.955316}%
\pgfsetfillcolor{currentfill}%
\pgfsetlinewidth{0.000000pt}%
\definecolor{currentstroke}{rgb}{0.000000,0.000000,0.000000}%
\pgfsetstrokecolor{currentstroke}%
\pgfsetdash{}{0pt}%
\pgfpathmoveto{\pgfqpoint{5.577878in}{1.893519in}}%
\pgfpathlineto{\pgfqpoint{5.764104in}{1.893519in}}%
\pgfpathlineto{\pgfqpoint{5.764104in}{1.975247in}}%
\pgfpathlineto{\pgfqpoint{5.577878in}{1.975247in}}%
\pgfpathlineto{\pgfqpoint{5.577878in}{1.893519in}}%
\pgfusepath{fill}%
\end{pgfscope}%
\begin{pgfscope}%
\pgfpathrectangle{\pgfqpoint{0.549740in}{0.463273in}}{\pgfqpoint{9.320225in}{4.495057in}}%
\pgfusepath{clip}%
\pgfsetbuttcap%
\pgfsetroundjoin%
\pgfsetlinewidth{0.000000pt}%
\definecolor{currentstroke}{rgb}{0.000000,0.000000,0.000000}%
\pgfsetstrokecolor{currentstroke}%
\pgfsetdash{}{0pt}%
\pgfpathmoveto{\pgfqpoint{5.764104in}{1.893519in}}%
\pgfpathlineto{\pgfqpoint{5.950331in}{1.893519in}}%
\pgfpathlineto{\pgfqpoint{5.950331in}{1.975247in}}%
\pgfpathlineto{\pgfqpoint{5.764104in}{1.975247in}}%
\pgfpathlineto{\pgfqpoint{5.764104in}{1.893519in}}%
\pgfusepath{}%
\end{pgfscope}%
\begin{pgfscope}%
\pgfpathrectangle{\pgfqpoint{0.549740in}{0.463273in}}{\pgfqpoint{9.320225in}{4.495057in}}%
\pgfusepath{clip}%
\pgfsetbuttcap%
\pgfsetroundjoin%
\pgfsetlinewidth{0.000000pt}%
\definecolor{currentstroke}{rgb}{0.000000,0.000000,0.000000}%
\pgfsetstrokecolor{currentstroke}%
\pgfsetdash{}{0pt}%
\pgfpathmoveto{\pgfqpoint{5.950331in}{1.893519in}}%
\pgfpathlineto{\pgfqpoint{6.136557in}{1.893519in}}%
\pgfpathlineto{\pgfqpoint{6.136557in}{1.975247in}}%
\pgfpathlineto{\pgfqpoint{5.950331in}{1.975247in}}%
\pgfpathlineto{\pgfqpoint{5.950331in}{1.893519in}}%
\pgfusepath{}%
\end{pgfscope}%
\begin{pgfscope}%
\pgfpathrectangle{\pgfqpoint{0.549740in}{0.463273in}}{\pgfqpoint{9.320225in}{4.495057in}}%
\pgfusepath{clip}%
\pgfsetbuttcap%
\pgfsetroundjoin%
\pgfsetlinewidth{0.000000pt}%
\definecolor{currentstroke}{rgb}{0.000000,0.000000,0.000000}%
\pgfsetstrokecolor{currentstroke}%
\pgfsetdash{}{0pt}%
\pgfpathmoveto{\pgfqpoint{6.136557in}{1.893519in}}%
\pgfpathlineto{\pgfqpoint{6.322784in}{1.893519in}}%
\pgfpathlineto{\pgfqpoint{6.322784in}{1.975247in}}%
\pgfpathlineto{\pgfqpoint{6.136557in}{1.975247in}}%
\pgfpathlineto{\pgfqpoint{6.136557in}{1.893519in}}%
\pgfusepath{}%
\end{pgfscope}%
\begin{pgfscope}%
\pgfpathrectangle{\pgfqpoint{0.549740in}{0.463273in}}{\pgfqpoint{9.320225in}{4.495057in}}%
\pgfusepath{clip}%
\pgfsetbuttcap%
\pgfsetroundjoin%
\definecolor{currentfill}{rgb}{0.472869,0.711325,0.955316}%
\pgfsetfillcolor{currentfill}%
\pgfsetlinewidth{0.000000pt}%
\definecolor{currentstroke}{rgb}{0.000000,0.000000,0.000000}%
\pgfsetstrokecolor{currentstroke}%
\pgfsetdash{}{0pt}%
\pgfpathmoveto{\pgfqpoint{6.322784in}{1.893519in}}%
\pgfpathlineto{\pgfqpoint{6.509011in}{1.893519in}}%
\pgfpathlineto{\pgfqpoint{6.509011in}{1.975247in}}%
\pgfpathlineto{\pgfqpoint{6.322784in}{1.975247in}}%
\pgfpathlineto{\pgfqpoint{6.322784in}{1.893519in}}%
\pgfusepath{fill}%
\end{pgfscope}%
\begin{pgfscope}%
\pgfpathrectangle{\pgfqpoint{0.549740in}{0.463273in}}{\pgfqpoint{9.320225in}{4.495057in}}%
\pgfusepath{clip}%
\pgfsetbuttcap%
\pgfsetroundjoin%
\pgfsetlinewidth{0.000000pt}%
\definecolor{currentstroke}{rgb}{0.000000,0.000000,0.000000}%
\pgfsetstrokecolor{currentstroke}%
\pgfsetdash{}{0pt}%
\pgfpathmoveto{\pgfqpoint{6.509011in}{1.893519in}}%
\pgfpathlineto{\pgfqpoint{6.695237in}{1.893519in}}%
\pgfpathlineto{\pgfqpoint{6.695237in}{1.975247in}}%
\pgfpathlineto{\pgfqpoint{6.509011in}{1.975247in}}%
\pgfpathlineto{\pgfqpoint{6.509011in}{1.893519in}}%
\pgfusepath{}%
\end{pgfscope}%
\begin{pgfscope}%
\pgfpathrectangle{\pgfqpoint{0.549740in}{0.463273in}}{\pgfqpoint{9.320225in}{4.495057in}}%
\pgfusepath{clip}%
\pgfsetbuttcap%
\pgfsetroundjoin%
\pgfsetlinewidth{0.000000pt}%
\definecolor{currentstroke}{rgb}{0.000000,0.000000,0.000000}%
\pgfsetstrokecolor{currentstroke}%
\pgfsetdash{}{0pt}%
\pgfpathmoveto{\pgfqpoint{6.695237in}{1.893519in}}%
\pgfpathlineto{\pgfqpoint{6.881464in}{1.893519in}}%
\pgfpathlineto{\pgfqpoint{6.881464in}{1.975247in}}%
\pgfpathlineto{\pgfqpoint{6.695237in}{1.975247in}}%
\pgfpathlineto{\pgfqpoint{6.695237in}{1.893519in}}%
\pgfusepath{}%
\end{pgfscope}%
\begin{pgfscope}%
\pgfpathrectangle{\pgfqpoint{0.549740in}{0.463273in}}{\pgfqpoint{9.320225in}{4.495057in}}%
\pgfusepath{clip}%
\pgfsetbuttcap%
\pgfsetroundjoin%
\pgfsetlinewidth{0.000000pt}%
\definecolor{currentstroke}{rgb}{0.000000,0.000000,0.000000}%
\pgfsetstrokecolor{currentstroke}%
\pgfsetdash{}{0pt}%
\pgfpathmoveto{\pgfqpoint{6.881464in}{1.893519in}}%
\pgfpathlineto{\pgfqpoint{7.067690in}{1.893519in}}%
\pgfpathlineto{\pgfqpoint{7.067690in}{1.975247in}}%
\pgfpathlineto{\pgfqpoint{6.881464in}{1.975247in}}%
\pgfpathlineto{\pgfqpoint{6.881464in}{1.893519in}}%
\pgfusepath{}%
\end{pgfscope}%
\begin{pgfscope}%
\pgfpathrectangle{\pgfqpoint{0.549740in}{0.463273in}}{\pgfqpoint{9.320225in}{4.495057in}}%
\pgfusepath{clip}%
\pgfsetbuttcap%
\pgfsetroundjoin%
\pgfsetlinewidth{0.000000pt}%
\definecolor{currentstroke}{rgb}{0.000000,0.000000,0.000000}%
\pgfsetstrokecolor{currentstroke}%
\pgfsetdash{}{0pt}%
\pgfpathmoveto{\pgfqpoint{7.067690in}{1.893519in}}%
\pgfpathlineto{\pgfqpoint{7.253917in}{1.893519in}}%
\pgfpathlineto{\pgfqpoint{7.253917in}{1.975247in}}%
\pgfpathlineto{\pgfqpoint{7.067690in}{1.975247in}}%
\pgfpathlineto{\pgfqpoint{7.067690in}{1.893519in}}%
\pgfusepath{}%
\end{pgfscope}%
\begin{pgfscope}%
\pgfpathrectangle{\pgfqpoint{0.549740in}{0.463273in}}{\pgfqpoint{9.320225in}{4.495057in}}%
\pgfusepath{clip}%
\pgfsetbuttcap%
\pgfsetroundjoin%
\pgfsetlinewidth{0.000000pt}%
\definecolor{currentstroke}{rgb}{0.000000,0.000000,0.000000}%
\pgfsetstrokecolor{currentstroke}%
\pgfsetdash{}{0pt}%
\pgfpathmoveto{\pgfqpoint{7.253917in}{1.893519in}}%
\pgfpathlineto{\pgfqpoint{7.440143in}{1.893519in}}%
\pgfpathlineto{\pgfqpoint{7.440143in}{1.975247in}}%
\pgfpathlineto{\pgfqpoint{7.253917in}{1.975247in}}%
\pgfpathlineto{\pgfqpoint{7.253917in}{1.893519in}}%
\pgfusepath{}%
\end{pgfscope}%
\begin{pgfscope}%
\pgfpathrectangle{\pgfqpoint{0.549740in}{0.463273in}}{\pgfqpoint{9.320225in}{4.495057in}}%
\pgfusepath{clip}%
\pgfsetbuttcap%
\pgfsetroundjoin%
\definecolor{currentfill}{rgb}{0.472869,0.711325,0.955316}%
\pgfsetfillcolor{currentfill}%
\pgfsetlinewidth{0.000000pt}%
\definecolor{currentstroke}{rgb}{0.000000,0.000000,0.000000}%
\pgfsetstrokecolor{currentstroke}%
\pgfsetdash{}{0pt}%
\pgfpathmoveto{\pgfqpoint{7.440143in}{1.893519in}}%
\pgfpathlineto{\pgfqpoint{7.626370in}{1.893519in}}%
\pgfpathlineto{\pgfqpoint{7.626370in}{1.975247in}}%
\pgfpathlineto{\pgfqpoint{7.440143in}{1.975247in}}%
\pgfpathlineto{\pgfqpoint{7.440143in}{1.893519in}}%
\pgfusepath{fill}%
\end{pgfscope}%
\begin{pgfscope}%
\pgfpathrectangle{\pgfqpoint{0.549740in}{0.463273in}}{\pgfqpoint{9.320225in}{4.495057in}}%
\pgfusepath{clip}%
\pgfsetbuttcap%
\pgfsetroundjoin%
\pgfsetlinewidth{0.000000pt}%
\definecolor{currentstroke}{rgb}{0.000000,0.000000,0.000000}%
\pgfsetstrokecolor{currentstroke}%
\pgfsetdash{}{0pt}%
\pgfpathmoveto{\pgfqpoint{7.626370in}{1.893519in}}%
\pgfpathlineto{\pgfqpoint{7.812596in}{1.893519in}}%
\pgfpathlineto{\pgfqpoint{7.812596in}{1.975247in}}%
\pgfpathlineto{\pgfqpoint{7.626370in}{1.975247in}}%
\pgfpathlineto{\pgfqpoint{7.626370in}{1.893519in}}%
\pgfusepath{}%
\end{pgfscope}%
\begin{pgfscope}%
\pgfpathrectangle{\pgfqpoint{0.549740in}{0.463273in}}{\pgfqpoint{9.320225in}{4.495057in}}%
\pgfusepath{clip}%
\pgfsetbuttcap%
\pgfsetroundjoin%
\pgfsetlinewidth{0.000000pt}%
\definecolor{currentstroke}{rgb}{0.000000,0.000000,0.000000}%
\pgfsetstrokecolor{currentstroke}%
\pgfsetdash{}{0pt}%
\pgfpathmoveto{\pgfqpoint{7.812596in}{1.893519in}}%
\pgfpathlineto{\pgfqpoint{7.998823in}{1.893519in}}%
\pgfpathlineto{\pgfqpoint{7.998823in}{1.975247in}}%
\pgfpathlineto{\pgfqpoint{7.812596in}{1.975247in}}%
\pgfpathlineto{\pgfqpoint{7.812596in}{1.893519in}}%
\pgfusepath{}%
\end{pgfscope}%
\begin{pgfscope}%
\pgfpathrectangle{\pgfqpoint{0.549740in}{0.463273in}}{\pgfqpoint{9.320225in}{4.495057in}}%
\pgfusepath{clip}%
\pgfsetbuttcap%
\pgfsetroundjoin%
\pgfsetlinewidth{0.000000pt}%
\definecolor{currentstroke}{rgb}{0.000000,0.000000,0.000000}%
\pgfsetstrokecolor{currentstroke}%
\pgfsetdash{}{0pt}%
\pgfpathmoveto{\pgfqpoint{7.998823in}{1.893519in}}%
\pgfpathlineto{\pgfqpoint{8.185049in}{1.893519in}}%
\pgfpathlineto{\pgfqpoint{8.185049in}{1.975247in}}%
\pgfpathlineto{\pgfqpoint{7.998823in}{1.975247in}}%
\pgfpathlineto{\pgfqpoint{7.998823in}{1.893519in}}%
\pgfusepath{}%
\end{pgfscope}%
\begin{pgfscope}%
\pgfpathrectangle{\pgfqpoint{0.549740in}{0.463273in}}{\pgfqpoint{9.320225in}{4.495057in}}%
\pgfusepath{clip}%
\pgfsetbuttcap%
\pgfsetroundjoin%
\pgfsetlinewidth{0.000000pt}%
\definecolor{currentstroke}{rgb}{0.000000,0.000000,0.000000}%
\pgfsetstrokecolor{currentstroke}%
\pgfsetdash{}{0pt}%
\pgfpathmoveto{\pgfqpoint{8.185049in}{1.893519in}}%
\pgfpathlineto{\pgfqpoint{8.371276in}{1.893519in}}%
\pgfpathlineto{\pgfqpoint{8.371276in}{1.975247in}}%
\pgfpathlineto{\pgfqpoint{8.185049in}{1.975247in}}%
\pgfpathlineto{\pgfqpoint{8.185049in}{1.893519in}}%
\pgfusepath{}%
\end{pgfscope}%
\begin{pgfscope}%
\pgfpathrectangle{\pgfqpoint{0.549740in}{0.463273in}}{\pgfqpoint{9.320225in}{4.495057in}}%
\pgfusepath{clip}%
\pgfsetbuttcap%
\pgfsetroundjoin%
\pgfsetlinewidth{0.000000pt}%
\definecolor{currentstroke}{rgb}{0.000000,0.000000,0.000000}%
\pgfsetstrokecolor{currentstroke}%
\pgfsetdash{}{0pt}%
\pgfpathmoveto{\pgfqpoint{8.371276in}{1.893519in}}%
\pgfpathlineto{\pgfqpoint{8.557503in}{1.893519in}}%
\pgfpathlineto{\pgfqpoint{8.557503in}{1.975247in}}%
\pgfpathlineto{\pgfqpoint{8.371276in}{1.975247in}}%
\pgfpathlineto{\pgfqpoint{8.371276in}{1.893519in}}%
\pgfusepath{}%
\end{pgfscope}%
\begin{pgfscope}%
\pgfpathrectangle{\pgfqpoint{0.549740in}{0.463273in}}{\pgfqpoint{9.320225in}{4.495057in}}%
\pgfusepath{clip}%
\pgfsetbuttcap%
\pgfsetroundjoin%
\definecolor{currentfill}{rgb}{0.472869,0.711325,0.955316}%
\pgfsetfillcolor{currentfill}%
\pgfsetlinewidth{0.000000pt}%
\definecolor{currentstroke}{rgb}{0.000000,0.000000,0.000000}%
\pgfsetstrokecolor{currentstroke}%
\pgfsetdash{}{0pt}%
\pgfpathmoveto{\pgfqpoint{8.557503in}{1.893519in}}%
\pgfpathlineto{\pgfqpoint{8.743729in}{1.893519in}}%
\pgfpathlineto{\pgfqpoint{8.743729in}{1.975247in}}%
\pgfpathlineto{\pgfqpoint{8.557503in}{1.975247in}}%
\pgfpathlineto{\pgfqpoint{8.557503in}{1.893519in}}%
\pgfusepath{fill}%
\end{pgfscope}%
\begin{pgfscope}%
\pgfpathrectangle{\pgfqpoint{0.549740in}{0.463273in}}{\pgfqpoint{9.320225in}{4.495057in}}%
\pgfusepath{clip}%
\pgfsetbuttcap%
\pgfsetroundjoin%
\pgfsetlinewidth{0.000000pt}%
\definecolor{currentstroke}{rgb}{0.000000,0.000000,0.000000}%
\pgfsetstrokecolor{currentstroke}%
\pgfsetdash{}{0pt}%
\pgfpathmoveto{\pgfqpoint{8.743729in}{1.893519in}}%
\pgfpathlineto{\pgfqpoint{8.929956in}{1.893519in}}%
\pgfpathlineto{\pgfqpoint{8.929956in}{1.975247in}}%
\pgfpathlineto{\pgfqpoint{8.743729in}{1.975247in}}%
\pgfpathlineto{\pgfqpoint{8.743729in}{1.893519in}}%
\pgfusepath{}%
\end{pgfscope}%
\begin{pgfscope}%
\pgfpathrectangle{\pgfqpoint{0.549740in}{0.463273in}}{\pgfqpoint{9.320225in}{4.495057in}}%
\pgfusepath{clip}%
\pgfsetbuttcap%
\pgfsetroundjoin%
\pgfsetlinewidth{0.000000pt}%
\definecolor{currentstroke}{rgb}{0.000000,0.000000,0.000000}%
\pgfsetstrokecolor{currentstroke}%
\pgfsetdash{}{0pt}%
\pgfpathmoveto{\pgfqpoint{8.929956in}{1.893519in}}%
\pgfpathlineto{\pgfqpoint{9.116182in}{1.893519in}}%
\pgfpathlineto{\pgfqpoint{9.116182in}{1.975247in}}%
\pgfpathlineto{\pgfqpoint{8.929956in}{1.975247in}}%
\pgfpathlineto{\pgfqpoint{8.929956in}{1.893519in}}%
\pgfusepath{}%
\end{pgfscope}%
\begin{pgfscope}%
\pgfpathrectangle{\pgfqpoint{0.549740in}{0.463273in}}{\pgfqpoint{9.320225in}{4.495057in}}%
\pgfusepath{clip}%
\pgfsetbuttcap%
\pgfsetroundjoin%
\pgfsetlinewidth{0.000000pt}%
\definecolor{currentstroke}{rgb}{0.000000,0.000000,0.000000}%
\pgfsetstrokecolor{currentstroke}%
\pgfsetdash{}{0pt}%
\pgfpathmoveto{\pgfqpoint{9.116182in}{1.893519in}}%
\pgfpathlineto{\pgfqpoint{9.302409in}{1.893519in}}%
\pgfpathlineto{\pgfqpoint{9.302409in}{1.975247in}}%
\pgfpathlineto{\pgfqpoint{9.116182in}{1.975247in}}%
\pgfpathlineto{\pgfqpoint{9.116182in}{1.893519in}}%
\pgfusepath{}%
\end{pgfscope}%
\begin{pgfscope}%
\pgfpathrectangle{\pgfqpoint{0.549740in}{0.463273in}}{\pgfqpoint{9.320225in}{4.495057in}}%
\pgfusepath{clip}%
\pgfsetbuttcap%
\pgfsetroundjoin%
\pgfsetlinewidth{0.000000pt}%
\definecolor{currentstroke}{rgb}{0.000000,0.000000,0.000000}%
\pgfsetstrokecolor{currentstroke}%
\pgfsetdash{}{0pt}%
\pgfpathmoveto{\pgfqpoint{9.302409in}{1.893519in}}%
\pgfpathlineto{\pgfqpoint{9.488635in}{1.893519in}}%
\pgfpathlineto{\pgfqpoint{9.488635in}{1.975247in}}%
\pgfpathlineto{\pgfqpoint{9.302409in}{1.975247in}}%
\pgfpathlineto{\pgfqpoint{9.302409in}{1.893519in}}%
\pgfusepath{}%
\end{pgfscope}%
\begin{pgfscope}%
\pgfpathrectangle{\pgfqpoint{0.549740in}{0.463273in}}{\pgfqpoint{9.320225in}{4.495057in}}%
\pgfusepath{clip}%
\pgfsetbuttcap%
\pgfsetroundjoin%
\pgfsetlinewidth{0.000000pt}%
\definecolor{currentstroke}{rgb}{0.000000,0.000000,0.000000}%
\pgfsetstrokecolor{currentstroke}%
\pgfsetdash{}{0pt}%
\pgfpathmoveto{\pgfqpoint{9.488635in}{1.893519in}}%
\pgfpathlineto{\pgfqpoint{9.674862in}{1.893519in}}%
\pgfpathlineto{\pgfqpoint{9.674862in}{1.975247in}}%
\pgfpathlineto{\pgfqpoint{9.488635in}{1.975247in}}%
\pgfpathlineto{\pgfqpoint{9.488635in}{1.893519in}}%
\pgfusepath{}%
\end{pgfscope}%
\begin{pgfscope}%
\pgfpathrectangle{\pgfqpoint{0.549740in}{0.463273in}}{\pgfqpoint{9.320225in}{4.495057in}}%
\pgfusepath{clip}%
\pgfsetbuttcap%
\pgfsetroundjoin%
\pgfsetlinewidth{0.000000pt}%
\definecolor{currentstroke}{rgb}{0.000000,0.000000,0.000000}%
\pgfsetstrokecolor{currentstroke}%
\pgfsetdash{}{0pt}%
\pgfpathmoveto{\pgfqpoint{9.674862in}{1.893519in}}%
\pgfpathlineto{\pgfqpoint{9.861088in}{1.893519in}}%
\pgfpathlineto{\pgfqpoint{9.861088in}{1.975247in}}%
\pgfpathlineto{\pgfqpoint{9.674862in}{1.975247in}}%
\pgfpathlineto{\pgfqpoint{9.674862in}{1.893519in}}%
\pgfusepath{}%
\end{pgfscope}%
\begin{pgfscope}%
\pgfpathrectangle{\pgfqpoint{0.549740in}{0.463273in}}{\pgfqpoint{9.320225in}{4.495057in}}%
\pgfusepath{clip}%
\pgfsetbuttcap%
\pgfsetroundjoin%
\pgfsetlinewidth{0.000000pt}%
\definecolor{currentstroke}{rgb}{0.000000,0.000000,0.000000}%
\pgfsetstrokecolor{currentstroke}%
\pgfsetdash{}{0pt}%
\pgfpathmoveto{\pgfqpoint{0.549761in}{1.975247in}}%
\pgfpathlineto{\pgfqpoint{0.735988in}{1.975247in}}%
\pgfpathlineto{\pgfqpoint{0.735988in}{2.056975in}}%
\pgfpathlineto{\pgfqpoint{0.549761in}{2.056975in}}%
\pgfpathlineto{\pgfqpoint{0.549761in}{1.975247in}}%
\pgfusepath{}%
\end{pgfscope}%
\begin{pgfscope}%
\pgfpathrectangle{\pgfqpoint{0.549740in}{0.463273in}}{\pgfqpoint{9.320225in}{4.495057in}}%
\pgfusepath{clip}%
\pgfsetbuttcap%
\pgfsetroundjoin%
\pgfsetlinewidth{0.000000pt}%
\definecolor{currentstroke}{rgb}{0.000000,0.000000,0.000000}%
\pgfsetstrokecolor{currentstroke}%
\pgfsetdash{}{0pt}%
\pgfpathmoveto{\pgfqpoint{0.735988in}{1.975247in}}%
\pgfpathlineto{\pgfqpoint{0.922214in}{1.975247in}}%
\pgfpathlineto{\pgfqpoint{0.922214in}{2.056975in}}%
\pgfpathlineto{\pgfqpoint{0.735988in}{2.056975in}}%
\pgfpathlineto{\pgfqpoint{0.735988in}{1.975247in}}%
\pgfusepath{}%
\end{pgfscope}%
\begin{pgfscope}%
\pgfpathrectangle{\pgfqpoint{0.549740in}{0.463273in}}{\pgfqpoint{9.320225in}{4.495057in}}%
\pgfusepath{clip}%
\pgfsetbuttcap%
\pgfsetroundjoin%
\pgfsetlinewidth{0.000000pt}%
\definecolor{currentstroke}{rgb}{0.000000,0.000000,0.000000}%
\pgfsetstrokecolor{currentstroke}%
\pgfsetdash{}{0pt}%
\pgfpathmoveto{\pgfqpoint{0.922214in}{1.975247in}}%
\pgfpathlineto{\pgfqpoint{1.108441in}{1.975247in}}%
\pgfpathlineto{\pgfqpoint{1.108441in}{2.056975in}}%
\pgfpathlineto{\pgfqpoint{0.922214in}{2.056975in}}%
\pgfpathlineto{\pgfqpoint{0.922214in}{1.975247in}}%
\pgfusepath{}%
\end{pgfscope}%
\begin{pgfscope}%
\pgfpathrectangle{\pgfqpoint{0.549740in}{0.463273in}}{\pgfqpoint{9.320225in}{4.495057in}}%
\pgfusepath{clip}%
\pgfsetbuttcap%
\pgfsetroundjoin%
\pgfsetlinewidth{0.000000pt}%
\definecolor{currentstroke}{rgb}{0.000000,0.000000,0.000000}%
\pgfsetstrokecolor{currentstroke}%
\pgfsetdash{}{0pt}%
\pgfpathmoveto{\pgfqpoint{1.108441in}{1.975247in}}%
\pgfpathlineto{\pgfqpoint{1.294667in}{1.975247in}}%
\pgfpathlineto{\pgfqpoint{1.294667in}{2.056975in}}%
\pgfpathlineto{\pgfqpoint{1.108441in}{2.056975in}}%
\pgfpathlineto{\pgfqpoint{1.108441in}{1.975247in}}%
\pgfusepath{}%
\end{pgfscope}%
\begin{pgfscope}%
\pgfpathrectangle{\pgfqpoint{0.549740in}{0.463273in}}{\pgfqpoint{9.320225in}{4.495057in}}%
\pgfusepath{clip}%
\pgfsetbuttcap%
\pgfsetroundjoin%
\pgfsetlinewidth{0.000000pt}%
\definecolor{currentstroke}{rgb}{0.000000,0.000000,0.000000}%
\pgfsetstrokecolor{currentstroke}%
\pgfsetdash{}{0pt}%
\pgfpathmoveto{\pgfqpoint{1.294667in}{1.975247in}}%
\pgfpathlineto{\pgfqpoint{1.480894in}{1.975247in}}%
\pgfpathlineto{\pgfqpoint{1.480894in}{2.056975in}}%
\pgfpathlineto{\pgfqpoint{1.294667in}{2.056975in}}%
\pgfpathlineto{\pgfqpoint{1.294667in}{1.975247in}}%
\pgfusepath{}%
\end{pgfscope}%
\begin{pgfscope}%
\pgfpathrectangle{\pgfqpoint{0.549740in}{0.463273in}}{\pgfqpoint{9.320225in}{4.495057in}}%
\pgfusepath{clip}%
\pgfsetbuttcap%
\pgfsetroundjoin%
\pgfsetlinewidth{0.000000pt}%
\definecolor{currentstroke}{rgb}{0.000000,0.000000,0.000000}%
\pgfsetstrokecolor{currentstroke}%
\pgfsetdash{}{0pt}%
\pgfpathmoveto{\pgfqpoint{1.480894in}{1.975247in}}%
\pgfpathlineto{\pgfqpoint{1.667120in}{1.975247in}}%
\pgfpathlineto{\pgfqpoint{1.667120in}{2.056975in}}%
\pgfpathlineto{\pgfqpoint{1.480894in}{2.056975in}}%
\pgfpathlineto{\pgfqpoint{1.480894in}{1.975247in}}%
\pgfusepath{}%
\end{pgfscope}%
\begin{pgfscope}%
\pgfpathrectangle{\pgfqpoint{0.549740in}{0.463273in}}{\pgfqpoint{9.320225in}{4.495057in}}%
\pgfusepath{clip}%
\pgfsetbuttcap%
\pgfsetroundjoin%
\pgfsetlinewidth{0.000000pt}%
\definecolor{currentstroke}{rgb}{0.000000,0.000000,0.000000}%
\pgfsetstrokecolor{currentstroke}%
\pgfsetdash{}{0pt}%
\pgfpathmoveto{\pgfqpoint{1.667120in}{1.975247in}}%
\pgfpathlineto{\pgfqpoint{1.853347in}{1.975247in}}%
\pgfpathlineto{\pgfqpoint{1.853347in}{2.056975in}}%
\pgfpathlineto{\pgfqpoint{1.667120in}{2.056975in}}%
\pgfpathlineto{\pgfqpoint{1.667120in}{1.975247in}}%
\pgfusepath{}%
\end{pgfscope}%
\begin{pgfscope}%
\pgfpathrectangle{\pgfqpoint{0.549740in}{0.463273in}}{\pgfqpoint{9.320225in}{4.495057in}}%
\pgfusepath{clip}%
\pgfsetbuttcap%
\pgfsetroundjoin%
\pgfsetlinewidth{0.000000pt}%
\definecolor{currentstroke}{rgb}{0.000000,0.000000,0.000000}%
\pgfsetstrokecolor{currentstroke}%
\pgfsetdash{}{0pt}%
\pgfpathmoveto{\pgfqpoint{1.853347in}{1.975247in}}%
\pgfpathlineto{\pgfqpoint{2.039573in}{1.975247in}}%
\pgfpathlineto{\pgfqpoint{2.039573in}{2.056975in}}%
\pgfpathlineto{\pgfqpoint{1.853347in}{2.056975in}}%
\pgfpathlineto{\pgfqpoint{1.853347in}{1.975247in}}%
\pgfusepath{}%
\end{pgfscope}%
\begin{pgfscope}%
\pgfpathrectangle{\pgfqpoint{0.549740in}{0.463273in}}{\pgfqpoint{9.320225in}{4.495057in}}%
\pgfusepath{clip}%
\pgfsetbuttcap%
\pgfsetroundjoin%
\pgfsetlinewidth{0.000000pt}%
\definecolor{currentstroke}{rgb}{0.000000,0.000000,0.000000}%
\pgfsetstrokecolor{currentstroke}%
\pgfsetdash{}{0pt}%
\pgfpathmoveto{\pgfqpoint{2.039573in}{1.975247in}}%
\pgfpathlineto{\pgfqpoint{2.225800in}{1.975247in}}%
\pgfpathlineto{\pgfqpoint{2.225800in}{2.056975in}}%
\pgfpathlineto{\pgfqpoint{2.039573in}{2.056975in}}%
\pgfpathlineto{\pgfqpoint{2.039573in}{1.975247in}}%
\pgfusepath{}%
\end{pgfscope}%
\begin{pgfscope}%
\pgfpathrectangle{\pgfqpoint{0.549740in}{0.463273in}}{\pgfqpoint{9.320225in}{4.495057in}}%
\pgfusepath{clip}%
\pgfsetbuttcap%
\pgfsetroundjoin%
\pgfsetlinewidth{0.000000pt}%
\definecolor{currentstroke}{rgb}{0.000000,0.000000,0.000000}%
\pgfsetstrokecolor{currentstroke}%
\pgfsetdash{}{0pt}%
\pgfpathmoveto{\pgfqpoint{2.225800in}{1.975247in}}%
\pgfpathlineto{\pgfqpoint{2.412027in}{1.975247in}}%
\pgfpathlineto{\pgfqpoint{2.412027in}{2.056975in}}%
\pgfpathlineto{\pgfqpoint{2.225800in}{2.056975in}}%
\pgfpathlineto{\pgfqpoint{2.225800in}{1.975247in}}%
\pgfusepath{}%
\end{pgfscope}%
\begin{pgfscope}%
\pgfpathrectangle{\pgfqpoint{0.549740in}{0.463273in}}{\pgfqpoint{9.320225in}{4.495057in}}%
\pgfusepath{clip}%
\pgfsetbuttcap%
\pgfsetroundjoin%
\pgfsetlinewidth{0.000000pt}%
\definecolor{currentstroke}{rgb}{0.000000,0.000000,0.000000}%
\pgfsetstrokecolor{currentstroke}%
\pgfsetdash{}{0pt}%
\pgfpathmoveto{\pgfqpoint{2.412027in}{1.975247in}}%
\pgfpathlineto{\pgfqpoint{2.598253in}{1.975247in}}%
\pgfpathlineto{\pgfqpoint{2.598253in}{2.056975in}}%
\pgfpathlineto{\pgfqpoint{2.412027in}{2.056975in}}%
\pgfpathlineto{\pgfqpoint{2.412027in}{1.975247in}}%
\pgfusepath{}%
\end{pgfscope}%
\begin{pgfscope}%
\pgfpathrectangle{\pgfqpoint{0.549740in}{0.463273in}}{\pgfqpoint{9.320225in}{4.495057in}}%
\pgfusepath{clip}%
\pgfsetbuttcap%
\pgfsetroundjoin%
\pgfsetlinewidth{0.000000pt}%
\definecolor{currentstroke}{rgb}{0.000000,0.000000,0.000000}%
\pgfsetstrokecolor{currentstroke}%
\pgfsetdash{}{0pt}%
\pgfpathmoveto{\pgfqpoint{2.598253in}{1.975247in}}%
\pgfpathlineto{\pgfqpoint{2.784480in}{1.975247in}}%
\pgfpathlineto{\pgfqpoint{2.784480in}{2.056975in}}%
\pgfpathlineto{\pgfqpoint{2.598253in}{2.056975in}}%
\pgfpathlineto{\pgfqpoint{2.598253in}{1.975247in}}%
\pgfusepath{}%
\end{pgfscope}%
\begin{pgfscope}%
\pgfpathrectangle{\pgfqpoint{0.549740in}{0.463273in}}{\pgfqpoint{9.320225in}{4.495057in}}%
\pgfusepath{clip}%
\pgfsetbuttcap%
\pgfsetroundjoin%
\pgfsetlinewidth{0.000000pt}%
\definecolor{currentstroke}{rgb}{0.000000,0.000000,0.000000}%
\pgfsetstrokecolor{currentstroke}%
\pgfsetdash{}{0pt}%
\pgfpathmoveto{\pgfqpoint{2.784480in}{1.975247in}}%
\pgfpathlineto{\pgfqpoint{2.970706in}{1.975247in}}%
\pgfpathlineto{\pgfqpoint{2.970706in}{2.056975in}}%
\pgfpathlineto{\pgfqpoint{2.784480in}{2.056975in}}%
\pgfpathlineto{\pgfqpoint{2.784480in}{1.975247in}}%
\pgfusepath{}%
\end{pgfscope}%
\begin{pgfscope}%
\pgfpathrectangle{\pgfqpoint{0.549740in}{0.463273in}}{\pgfqpoint{9.320225in}{4.495057in}}%
\pgfusepath{clip}%
\pgfsetbuttcap%
\pgfsetroundjoin%
\pgfsetlinewidth{0.000000pt}%
\definecolor{currentstroke}{rgb}{0.000000,0.000000,0.000000}%
\pgfsetstrokecolor{currentstroke}%
\pgfsetdash{}{0pt}%
\pgfpathmoveto{\pgfqpoint{2.970706in}{1.975247in}}%
\pgfpathlineto{\pgfqpoint{3.156933in}{1.975247in}}%
\pgfpathlineto{\pgfqpoint{3.156933in}{2.056975in}}%
\pgfpathlineto{\pgfqpoint{2.970706in}{2.056975in}}%
\pgfpathlineto{\pgfqpoint{2.970706in}{1.975247in}}%
\pgfusepath{}%
\end{pgfscope}%
\begin{pgfscope}%
\pgfpathrectangle{\pgfqpoint{0.549740in}{0.463273in}}{\pgfqpoint{9.320225in}{4.495057in}}%
\pgfusepath{clip}%
\pgfsetbuttcap%
\pgfsetroundjoin%
\pgfsetlinewidth{0.000000pt}%
\definecolor{currentstroke}{rgb}{0.000000,0.000000,0.000000}%
\pgfsetstrokecolor{currentstroke}%
\pgfsetdash{}{0pt}%
\pgfpathmoveto{\pgfqpoint{3.156933in}{1.975247in}}%
\pgfpathlineto{\pgfqpoint{3.343159in}{1.975247in}}%
\pgfpathlineto{\pgfqpoint{3.343159in}{2.056975in}}%
\pgfpathlineto{\pgfqpoint{3.156933in}{2.056975in}}%
\pgfpathlineto{\pgfqpoint{3.156933in}{1.975247in}}%
\pgfusepath{}%
\end{pgfscope}%
\begin{pgfscope}%
\pgfpathrectangle{\pgfqpoint{0.549740in}{0.463273in}}{\pgfqpoint{9.320225in}{4.495057in}}%
\pgfusepath{clip}%
\pgfsetbuttcap%
\pgfsetroundjoin%
\pgfsetlinewidth{0.000000pt}%
\definecolor{currentstroke}{rgb}{0.000000,0.000000,0.000000}%
\pgfsetstrokecolor{currentstroke}%
\pgfsetdash{}{0pt}%
\pgfpathmoveto{\pgfqpoint{3.343159in}{1.975247in}}%
\pgfpathlineto{\pgfqpoint{3.529386in}{1.975247in}}%
\pgfpathlineto{\pgfqpoint{3.529386in}{2.056975in}}%
\pgfpathlineto{\pgfqpoint{3.343159in}{2.056975in}}%
\pgfpathlineto{\pgfqpoint{3.343159in}{1.975247in}}%
\pgfusepath{}%
\end{pgfscope}%
\begin{pgfscope}%
\pgfpathrectangle{\pgfqpoint{0.549740in}{0.463273in}}{\pgfqpoint{9.320225in}{4.495057in}}%
\pgfusepath{clip}%
\pgfsetbuttcap%
\pgfsetroundjoin%
\pgfsetlinewidth{0.000000pt}%
\definecolor{currentstroke}{rgb}{0.000000,0.000000,0.000000}%
\pgfsetstrokecolor{currentstroke}%
\pgfsetdash{}{0pt}%
\pgfpathmoveto{\pgfqpoint{3.529386in}{1.975247in}}%
\pgfpathlineto{\pgfqpoint{3.715612in}{1.975247in}}%
\pgfpathlineto{\pgfqpoint{3.715612in}{2.056975in}}%
\pgfpathlineto{\pgfqpoint{3.529386in}{2.056975in}}%
\pgfpathlineto{\pgfqpoint{3.529386in}{1.975247in}}%
\pgfusepath{}%
\end{pgfscope}%
\begin{pgfscope}%
\pgfpathrectangle{\pgfqpoint{0.549740in}{0.463273in}}{\pgfqpoint{9.320225in}{4.495057in}}%
\pgfusepath{clip}%
\pgfsetbuttcap%
\pgfsetroundjoin%
\pgfsetlinewidth{0.000000pt}%
\definecolor{currentstroke}{rgb}{0.000000,0.000000,0.000000}%
\pgfsetstrokecolor{currentstroke}%
\pgfsetdash{}{0pt}%
\pgfpathmoveto{\pgfqpoint{3.715612in}{1.975247in}}%
\pgfpathlineto{\pgfqpoint{3.901839in}{1.975247in}}%
\pgfpathlineto{\pgfqpoint{3.901839in}{2.056975in}}%
\pgfpathlineto{\pgfqpoint{3.715612in}{2.056975in}}%
\pgfpathlineto{\pgfqpoint{3.715612in}{1.975247in}}%
\pgfusepath{}%
\end{pgfscope}%
\begin{pgfscope}%
\pgfpathrectangle{\pgfqpoint{0.549740in}{0.463273in}}{\pgfqpoint{9.320225in}{4.495057in}}%
\pgfusepath{clip}%
\pgfsetbuttcap%
\pgfsetroundjoin%
\pgfsetlinewidth{0.000000pt}%
\definecolor{currentstroke}{rgb}{0.000000,0.000000,0.000000}%
\pgfsetstrokecolor{currentstroke}%
\pgfsetdash{}{0pt}%
\pgfpathmoveto{\pgfqpoint{3.901839in}{1.975247in}}%
\pgfpathlineto{\pgfqpoint{4.088065in}{1.975247in}}%
\pgfpathlineto{\pgfqpoint{4.088065in}{2.056975in}}%
\pgfpathlineto{\pgfqpoint{3.901839in}{2.056975in}}%
\pgfpathlineto{\pgfqpoint{3.901839in}{1.975247in}}%
\pgfusepath{}%
\end{pgfscope}%
\begin{pgfscope}%
\pgfpathrectangle{\pgfqpoint{0.549740in}{0.463273in}}{\pgfqpoint{9.320225in}{4.495057in}}%
\pgfusepath{clip}%
\pgfsetbuttcap%
\pgfsetroundjoin%
\pgfsetlinewidth{0.000000pt}%
\definecolor{currentstroke}{rgb}{0.000000,0.000000,0.000000}%
\pgfsetstrokecolor{currentstroke}%
\pgfsetdash{}{0pt}%
\pgfpathmoveto{\pgfqpoint{4.088065in}{1.975247in}}%
\pgfpathlineto{\pgfqpoint{4.274292in}{1.975247in}}%
\pgfpathlineto{\pgfqpoint{4.274292in}{2.056975in}}%
\pgfpathlineto{\pgfqpoint{4.088065in}{2.056975in}}%
\pgfpathlineto{\pgfqpoint{4.088065in}{1.975247in}}%
\pgfusepath{}%
\end{pgfscope}%
\begin{pgfscope}%
\pgfpathrectangle{\pgfqpoint{0.549740in}{0.463273in}}{\pgfqpoint{9.320225in}{4.495057in}}%
\pgfusepath{clip}%
\pgfsetbuttcap%
\pgfsetroundjoin%
\pgfsetlinewidth{0.000000pt}%
\definecolor{currentstroke}{rgb}{0.000000,0.000000,0.000000}%
\pgfsetstrokecolor{currentstroke}%
\pgfsetdash{}{0pt}%
\pgfpathmoveto{\pgfqpoint{4.274292in}{1.975247in}}%
\pgfpathlineto{\pgfqpoint{4.460519in}{1.975247in}}%
\pgfpathlineto{\pgfqpoint{4.460519in}{2.056975in}}%
\pgfpathlineto{\pgfqpoint{4.274292in}{2.056975in}}%
\pgfpathlineto{\pgfqpoint{4.274292in}{1.975247in}}%
\pgfusepath{}%
\end{pgfscope}%
\begin{pgfscope}%
\pgfpathrectangle{\pgfqpoint{0.549740in}{0.463273in}}{\pgfqpoint{9.320225in}{4.495057in}}%
\pgfusepath{clip}%
\pgfsetbuttcap%
\pgfsetroundjoin%
\pgfsetlinewidth{0.000000pt}%
\definecolor{currentstroke}{rgb}{0.000000,0.000000,0.000000}%
\pgfsetstrokecolor{currentstroke}%
\pgfsetdash{}{0pt}%
\pgfpathmoveto{\pgfqpoint{4.460519in}{1.975247in}}%
\pgfpathlineto{\pgfqpoint{4.646745in}{1.975247in}}%
\pgfpathlineto{\pgfqpoint{4.646745in}{2.056975in}}%
\pgfpathlineto{\pgfqpoint{4.460519in}{2.056975in}}%
\pgfpathlineto{\pgfqpoint{4.460519in}{1.975247in}}%
\pgfusepath{}%
\end{pgfscope}%
\begin{pgfscope}%
\pgfpathrectangle{\pgfqpoint{0.549740in}{0.463273in}}{\pgfqpoint{9.320225in}{4.495057in}}%
\pgfusepath{clip}%
\pgfsetbuttcap%
\pgfsetroundjoin%
\pgfsetlinewidth{0.000000pt}%
\definecolor{currentstroke}{rgb}{0.000000,0.000000,0.000000}%
\pgfsetstrokecolor{currentstroke}%
\pgfsetdash{}{0pt}%
\pgfpathmoveto{\pgfqpoint{4.646745in}{1.975247in}}%
\pgfpathlineto{\pgfqpoint{4.832972in}{1.975247in}}%
\pgfpathlineto{\pgfqpoint{4.832972in}{2.056975in}}%
\pgfpathlineto{\pgfqpoint{4.646745in}{2.056975in}}%
\pgfpathlineto{\pgfqpoint{4.646745in}{1.975247in}}%
\pgfusepath{}%
\end{pgfscope}%
\begin{pgfscope}%
\pgfpathrectangle{\pgfqpoint{0.549740in}{0.463273in}}{\pgfqpoint{9.320225in}{4.495057in}}%
\pgfusepath{clip}%
\pgfsetbuttcap%
\pgfsetroundjoin%
\pgfsetlinewidth{0.000000pt}%
\definecolor{currentstroke}{rgb}{0.000000,0.000000,0.000000}%
\pgfsetstrokecolor{currentstroke}%
\pgfsetdash{}{0pt}%
\pgfpathmoveto{\pgfqpoint{4.832972in}{1.975247in}}%
\pgfpathlineto{\pgfqpoint{5.019198in}{1.975247in}}%
\pgfpathlineto{\pgfqpoint{5.019198in}{2.056975in}}%
\pgfpathlineto{\pgfqpoint{4.832972in}{2.056975in}}%
\pgfpathlineto{\pgfqpoint{4.832972in}{1.975247in}}%
\pgfusepath{}%
\end{pgfscope}%
\begin{pgfscope}%
\pgfpathrectangle{\pgfqpoint{0.549740in}{0.463273in}}{\pgfqpoint{9.320225in}{4.495057in}}%
\pgfusepath{clip}%
\pgfsetbuttcap%
\pgfsetroundjoin%
\pgfsetlinewidth{0.000000pt}%
\definecolor{currentstroke}{rgb}{0.000000,0.000000,0.000000}%
\pgfsetstrokecolor{currentstroke}%
\pgfsetdash{}{0pt}%
\pgfpathmoveto{\pgfqpoint{5.019198in}{1.975247in}}%
\pgfpathlineto{\pgfqpoint{5.205425in}{1.975247in}}%
\pgfpathlineto{\pgfqpoint{5.205425in}{2.056975in}}%
\pgfpathlineto{\pgfqpoint{5.019198in}{2.056975in}}%
\pgfpathlineto{\pgfqpoint{5.019198in}{1.975247in}}%
\pgfusepath{}%
\end{pgfscope}%
\begin{pgfscope}%
\pgfpathrectangle{\pgfqpoint{0.549740in}{0.463273in}}{\pgfqpoint{9.320225in}{4.495057in}}%
\pgfusepath{clip}%
\pgfsetbuttcap%
\pgfsetroundjoin%
\pgfsetlinewidth{0.000000pt}%
\definecolor{currentstroke}{rgb}{0.000000,0.000000,0.000000}%
\pgfsetstrokecolor{currentstroke}%
\pgfsetdash{}{0pt}%
\pgfpathmoveto{\pgfqpoint{5.205425in}{1.975247in}}%
\pgfpathlineto{\pgfqpoint{5.391651in}{1.975247in}}%
\pgfpathlineto{\pgfqpoint{5.391651in}{2.056975in}}%
\pgfpathlineto{\pgfqpoint{5.205425in}{2.056975in}}%
\pgfpathlineto{\pgfqpoint{5.205425in}{1.975247in}}%
\pgfusepath{}%
\end{pgfscope}%
\begin{pgfscope}%
\pgfpathrectangle{\pgfqpoint{0.549740in}{0.463273in}}{\pgfqpoint{9.320225in}{4.495057in}}%
\pgfusepath{clip}%
\pgfsetbuttcap%
\pgfsetroundjoin%
\pgfsetlinewidth{0.000000pt}%
\definecolor{currentstroke}{rgb}{0.000000,0.000000,0.000000}%
\pgfsetstrokecolor{currentstroke}%
\pgfsetdash{}{0pt}%
\pgfpathmoveto{\pgfqpoint{5.391651in}{1.975247in}}%
\pgfpathlineto{\pgfqpoint{5.577878in}{1.975247in}}%
\pgfpathlineto{\pgfqpoint{5.577878in}{2.056975in}}%
\pgfpathlineto{\pgfqpoint{5.391651in}{2.056975in}}%
\pgfpathlineto{\pgfqpoint{5.391651in}{1.975247in}}%
\pgfusepath{}%
\end{pgfscope}%
\begin{pgfscope}%
\pgfpathrectangle{\pgfqpoint{0.549740in}{0.463273in}}{\pgfqpoint{9.320225in}{4.495057in}}%
\pgfusepath{clip}%
\pgfsetbuttcap%
\pgfsetroundjoin%
\definecolor{currentfill}{rgb}{0.472869,0.711325,0.955316}%
\pgfsetfillcolor{currentfill}%
\pgfsetlinewidth{0.000000pt}%
\definecolor{currentstroke}{rgb}{0.000000,0.000000,0.000000}%
\pgfsetstrokecolor{currentstroke}%
\pgfsetdash{}{0pt}%
\pgfpathmoveto{\pgfqpoint{5.577878in}{1.975247in}}%
\pgfpathlineto{\pgfqpoint{5.764104in}{1.975247in}}%
\pgfpathlineto{\pgfqpoint{5.764104in}{2.056975in}}%
\pgfpathlineto{\pgfqpoint{5.577878in}{2.056975in}}%
\pgfpathlineto{\pgfqpoint{5.577878in}{1.975247in}}%
\pgfusepath{fill}%
\end{pgfscope}%
\begin{pgfscope}%
\pgfpathrectangle{\pgfqpoint{0.549740in}{0.463273in}}{\pgfqpoint{9.320225in}{4.495057in}}%
\pgfusepath{clip}%
\pgfsetbuttcap%
\pgfsetroundjoin%
\pgfsetlinewidth{0.000000pt}%
\definecolor{currentstroke}{rgb}{0.000000,0.000000,0.000000}%
\pgfsetstrokecolor{currentstroke}%
\pgfsetdash{}{0pt}%
\pgfpathmoveto{\pgfqpoint{5.764104in}{1.975247in}}%
\pgfpathlineto{\pgfqpoint{5.950331in}{1.975247in}}%
\pgfpathlineto{\pgfqpoint{5.950331in}{2.056975in}}%
\pgfpathlineto{\pgfqpoint{5.764104in}{2.056975in}}%
\pgfpathlineto{\pgfqpoint{5.764104in}{1.975247in}}%
\pgfusepath{}%
\end{pgfscope}%
\begin{pgfscope}%
\pgfpathrectangle{\pgfqpoint{0.549740in}{0.463273in}}{\pgfqpoint{9.320225in}{4.495057in}}%
\pgfusepath{clip}%
\pgfsetbuttcap%
\pgfsetroundjoin%
\pgfsetlinewidth{0.000000pt}%
\definecolor{currentstroke}{rgb}{0.000000,0.000000,0.000000}%
\pgfsetstrokecolor{currentstroke}%
\pgfsetdash{}{0pt}%
\pgfpathmoveto{\pgfqpoint{5.950331in}{1.975247in}}%
\pgfpathlineto{\pgfqpoint{6.136557in}{1.975247in}}%
\pgfpathlineto{\pgfqpoint{6.136557in}{2.056975in}}%
\pgfpathlineto{\pgfqpoint{5.950331in}{2.056975in}}%
\pgfpathlineto{\pgfqpoint{5.950331in}{1.975247in}}%
\pgfusepath{}%
\end{pgfscope}%
\begin{pgfscope}%
\pgfpathrectangle{\pgfqpoint{0.549740in}{0.463273in}}{\pgfqpoint{9.320225in}{4.495057in}}%
\pgfusepath{clip}%
\pgfsetbuttcap%
\pgfsetroundjoin%
\pgfsetlinewidth{0.000000pt}%
\definecolor{currentstroke}{rgb}{0.000000,0.000000,0.000000}%
\pgfsetstrokecolor{currentstroke}%
\pgfsetdash{}{0pt}%
\pgfpathmoveto{\pgfqpoint{6.136557in}{1.975247in}}%
\pgfpathlineto{\pgfqpoint{6.322784in}{1.975247in}}%
\pgfpathlineto{\pgfqpoint{6.322784in}{2.056975in}}%
\pgfpathlineto{\pgfqpoint{6.136557in}{2.056975in}}%
\pgfpathlineto{\pgfqpoint{6.136557in}{1.975247in}}%
\pgfusepath{}%
\end{pgfscope}%
\begin{pgfscope}%
\pgfpathrectangle{\pgfqpoint{0.549740in}{0.463273in}}{\pgfqpoint{9.320225in}{4.495057in}}%
\pgfusepath{clip}%
\pgfsetbuttcap%
\pgfsetroundjoin%
\definecolor{currentfill}{rgb}{0.472869,0.711325,0.955316}%
\pgfsetfillcolor{currentfill}%
\pgfsetlinewidth{0.000000pt}%
\definecolor{currentstroke}{rgb}{0.000000,0.000000,0.000000}%
\pgfsetstrokecolor{currentstroke}%
\pgfsetdash{}{0pt}%
\pgfpathmoveto{\pgfqpoint{6.322784in}{1.975247in}}%
\pgfpathlineto{\pgfqpoint{6.509011in}{1.975247in}}%
\pgfpathlineto{\pgfqpoint{6.509011in}{2.056975in}}%
\pgfpathlineto{\pgfqpoint{6.322784in}{2.056975in}}%
\pgfpathlineto{\pgfqpoint{6.322784in}{1.975247in}}%
\pgfusepath{fill}%
\end{pgfscope}%
\begin{pgfscope}%
\pgfpathrectangle{\pgfqpoint{0.549740in}{0.463273in}}{\pgfqpoint{9.320225in}{4.495057in}}%
\pgfusepath{clip}%
\pgfsetbuttcap%
\pgfsetroundjoin%
\pgfsetlinewidth{0.000000pt}%
\definecolor{currentstroke}{rgb}{0.000000,0.000000,0.000000}%
\pgfsetstrokecolor{currentstroke}%
\pgfsetdash{}{0pt}%
\pgfpathmoveto{\pgfqpoint{6.509011in}{1.975247in}}%
\pgfpathlineto{\pgfqpoint{6.695237in}{1.975247in}}%
\pgfpathlineto{\pgfqpoint{6.695237in}{2.056975in}}%
\pgfpathlineto{\pgfqpoint{6.509011in}{2.056975in}}%
\pgfpathlineto{\pgfqpoint{6.509011in}{1.975247in}}%
\pgfusepath{}%
\end{pgfscope}%
\begin{pgfscope}%
\pgfpathrectangle{\pgfqpoint{0.549740in}{0.463273in}}{\pgfqpoint{9.320225in}{4.495057in}}%
\pgfusepath{clip}%
\pgfsetbuttcap%
\pgfsetroundjoin%
\pgfsetlinewidth{0.000000pt}%
\definecolor{currentstroke}{rgb}{0.000000,0.000000,0.000000}%
\pgfsetstrokecolor{currentstroke}%
\pgfsetdash{}{0pt}%
\pgfpathmoveto{\pgfqpoint{6.695237in}{1.975247in}}%
\pgfpathlineto{\pgfqpoint{6.881464in}{1.975247in}}%
\pgfpathlineto{\pgfqpoint{6.881464in}{2.056975in}}%
\pgfpathlineto{\pgfqpoint{6.695237in}{2.056975in}}%
\pgfpathlineto{\pgfqpoint{6.695237in}{1.975247in}}%
\pgfusepath{}%
\end{pgfscope}%
\begin{pgfscope}%
\pgfpathrectangle{\pgfqpoint{0.549740in}{0.463273in}}{\pgfqpoint{9.320225in}{4.495057in}}%
\pgfusepath{clip}%
\pgfsetbuttcap%
\pgfsetroundjoin%
\pgfsetlinewidth{0.000000pt}%
\definecolor{currentstroke}{rgb}{0.000000,0.000000,0.000000}%
\pgfsetstrokecolor{currentstroke}%
\pgfsetdash{}{0pt}%
\pgfpathmoveto{\pgfqpoint{6.881464in}{1.975247in}}%
\pgfpathlineto{\pgfqpoint{7.067690in}{1.975247in}}%
\pgfpathlineto{\pgfqpoint{7.067690in}{2.056975in}}%
\pgfpathlineto{\pgfqpoint{6.881464in}{2.056975in}}%
\pgfpathlineto{\pgfqpoint{6.881464in}{1.975247in}}%
\pgfusepath{}%
\end{pgfscope}%
\begin{pgfscope}%
\pgfpathrectangle{\pgfqpoint{0.549740in}{0.463273in}}{\pgfqpoint{9.320225in}{4.495057in}}%
\pgfusepath{clip}%
\pgfsetbuttcap%
\pgfsetroundjoin%
\pgfsetlinewidth{0.000000pt}%
\definecolor{currentstroke}{rgb}{0.000000,0.000000,0.000000}%
\pgfsetstrokecolor{currentstroke}%
\pgfsetdash{}{0pt}%
\pgfpathmoveto{\pgfqpoint{7.067690in}{1.975247in}}%
\pgfpathlineto{\pgfqpoint{7.253917in}{1.975247in}}%
\pgfpathlineto{\pgfqpoint{7.253917in}{2.056975in}}%
\pgfpathlineto{\pgfqpoint{7.067690in}{2.056975in}}%
\pgfpathlineto{\pgfqpoint{7.067690in}{1.975247in}}%
\pgfusepath{}%
\end{pgfscope}%
\begin{pgfscope}%
\pgfpathrectangle{\pgfqpoint{0.549740in}{0.463273in}}{\pgfqpoint{9.320225in}{4.495057in}}%
\pgfusepath{clip}%
\pgfsetbuttcap%
\pgfsetroundjoin%
\pgfsetlinewidth{0.000000pt}%
\definecolor{currentstroke}{rgb}{0.000000,0.000000,0.000000}%
\pgfsetstrokecolor{currentstroke}%
\pgfsetdash{}{0pt}%
\pgfpathmoveto{\pgfqpoint{7.253917in}{1.975247in}}%
\pgfpathlineto{\pgfqpoint{7.440143in}{1.975247in}}%
\pgfpathlineto{\pgfqpoint{7.440143in}{2.056975in}}%
\pgfpathlineto{\pgfqpoint{7.253917in}{2.056975in}}%
\pgfpathlineto{\pgfqpoint{7.253917in}{1.975247in}}%
\pgfusepath{}%
\end{pgfscope}%
\begin{pgfscope}%
\pgfpathrectangle{\pgfqpoint{0.549740in}{0.463273in}}{\pgfqpoint{9.320225in}{4.495057in}}%
\pgfusepath{clip}%
\pgfsetbuttcap%
\pgfsetroundjoin%
\definecolor{currentfill}{rgb}{0.472869,0.711325,0.955316}%
\pgfsetfillcolor{currentfill}%
\pgfsetlinewidth{0.000000pt}%
\definecolor{currentstroke}{rgb}{0.000000,0.000000,0.000000}%
\pgfsetstrokecolor{currentstroke}%
\pgfsetdash{}{0pt}%
\pgfpathmoveto{\pgfqpoint{7.440143in}{1.975247in}}%
\pgfpathlineto{\pgfqpoint{7.626370in}{1.975247in}}%
\pgfpathlineto{\pgfqpoint{7.626370in}{2.056975in}}%
\pgfpathlineto{\pgfqpoint{7.440143in}{2.056975in}}%
\pgfpathlineto{\pgfqpoint{7.440143in}{1.975247in}}%
\pgfusepath{fill}%
\end{pgfscope}%
\begin{pgfscope}%
\pgfpathrectangle{\pgfqpoint{0.549740in}{0.463273in}}{\pgfqpoint{9.320225in}{4.495057in}}%
\pgfusepath{clip}%
\pgfsetbuttcap%
\pgfsetroundjoin%
\pgfsetlinewidth{0.000000pt}%
\definecolor{currentstroke}{rgb}{0.000000,0.000000,0.000000}%
\pgfsetstrokecolor{currentstroke}%
\pgfsetdash{}{0pt}%
\pgfpathmoveto{\pgfqpoint{7.626370in}{1.975247in}}%
\pgfpathlineto{\pgfqpoint{7.812596in}{1.975247in}}%
\pgfpathlineto{\pgfqpoint{7.812596in}{2.056975in}}%
\pgfpathlineto{\pgfqpoint{7.626370in}{2.056975in}}%
\pgfpathlineto{\pgfqpoint{7.626370in}{1.975247in}}%
\pgfusepath{}%
\end{pgfscope}%
\begin{pgfscope}%
\pgfpathrectangle{\pgfqpoint{0.549740in}{0.463273in}}{\pgfqpoint{9.320225in}{4.495057in}}%
\pgfusepath{clip}%
\pgfsetbuttcap%
\pgfsetroundjoin%
\pgfsetlinewidth{0.000000pt}%
\definecolor{currentstroke}{rgb}{0.000000,0.000000,0.000000}%
\pgfsetstrokecolor{currentstroke}%
\pgfsetdash{}{0pt}%
\pgfpathmoveto{\pgfqpoint{7.812596in}{1.975247in}}%
\pgfpathlineto{\pgfqpoint{7.998823in}{1.975247in}}%
\pgfpathlineto{\pgfqpoint{7.998823in}{2.056975in}}%
\pgfpathlineto{\pgfqpoint{7.812596in}{2.056975in}}%
\pgfpathlineto{\pgfqpoint{7.812596in}{1.975247in}}%
\pgfusepath{}%
\end{pgfscope}%
\begin{pgfscope}%
\pgfpathrectangle{\pgfqpoint{0.549740in}{0.463273in}}{\pgfqpoint{9.320225in}{4.495057in}}%
\pgfusepath{clip}%
\pgfsetbuttcap%
\pgfsetroundjoin%
\pgfsetlinewidth{0.000000pt}%
\definecolor{currentstroke}{rgb}{0.000000,0.000000,0.000000}%
\pgfsetstrokecolor{currentstroke}%
\pgfsetdash{}{0pt}%
\pgfpathmoveto{\pgfqpoint{7.998823in}{1.975247in}}%
\pgfpathlineto{\pgfqpoint{8.185049in}{1.975247in}}%
\pgfpathlineto{\pgfqpoint{8.185049in}{2.056975in}}%
\pgfpathlineto{\pgfqpoint{7.998823in}{2.056975in}}%
\pgfpathlineto{\pgfqpoint{7.998823in}{1.975247in}}%
\pgfusepath{}%
\end{pgfscope}%
\begin{pgfscope}%
\pgfpathrectangle{\pgfqpoint{0.549740in}{0.463273in}}{\pgfqpoint{9.320225in}{4.495057in}}%
\pgfusepath{clip}%
\pgfsetbuttcap%
\pgfsetroundjoin%
\pgfsetlinewidth{0.000000pt}%
\definecolor{currentstroke}{rgb}{0.000000,0.000000,0.000000}%
\pgfsetstrokecolor{currentstroke}%
\pgfsetdash{}{0pt}%
\pgfpathmoveto{\pgfqpoint{8.185049in}{1.975247in}}%
\pgfpathlineto{\pgfqpoint{8.371276in}{1.975247in}}%
\pgfpathlineto{\pgfqpoint{8.371276in}{2.056975in}}%
\pgfpathlineto{\pgfqpoint{8.185049in}{2.056975in}}%
\pgfpathlineto{\pgfqpoint{8.185049in}{1.975247in}}%
\pgfusepath{}%
\end{pgfscope}%
\begin{pgfscope}%
\pgfpathrectangle{\pgfqpoint{0.549740in}{0.463273in}}{\pgfqpoint{9.320225in}{4.495057in}}%
\pgfusepath{clip}%
\pgfsetbuttcap%
\pgfsetroundjoin%
\pgfsetlinewidth{0.000000pt}%
\definecolor{currentstroke}{rgb}{0.000000,0.000000,0.000000}%
\pgfsetstrokecolor{currentstroke}%
\pgfsetdash{}{0pt}%
\pgfpathmoveto{\pgfqpoint{8.371276in}{1.975247in}}%
\pgfpathlineto{\pgfqpoint{8.557503in}{1.975247in}}%
\pgfpathlineto{\pgfqpoint{8.557503in}{2.056975in}}%
\pgfpathlineto{\pgfqpoint{8.371276in}{2.056975in}}%
\pgfpathlineto{\pgfqpoint{8.371276in}{1.975247in}}%
\pgfusepath{}%
\end{pgfscope}%
\begin{pgfscope}%
\pgfpathrectangle{\pgfqpoint{0.549740in}{0.463273in}}{\pgfqpoint{9.320225in}{4.495057in}}%
\pgfusepath{clip}%
\pgfsetbuttcap%
\pgfsetroundjoin%
\definecolor{currentfill}{rgb}{0.472869,0.711325,0.955316}%
\pgfsetfillcolor{currentfill}%
\pgfsetlinewidth{0.000000pt}%
\definecolor{currentstroke}{rgb}{0.000000,0.000000,0.000000}%
\pgfsetstrokecolor{currentstroke}%
\pgfsetdash{}{0pt}%
\pgfpathmoveto{\pgfqpoint{8.557503in}{1.975247in}}%
\pgfpathlineto{\pgfqpoint{8.743729in}{1.975247in}}%
\pgfpathlineto{\pgfqpoint{8.743729in}{2.056975in}}%
\pgfpathlineto{\pgfqpoint{8.557503in}{2.056975in}}%
\pgfpathlineto{\pgfqpoint{8.557503in}{1.975247in}}%
\pgfusepath{fill}%
\end{pgfscope}%
\begin{pgfscope}%
\pgfpathrectangle{\pgfqpoint{0.549740in}{0.463273in}}{\pgfqpoint{9.320225in}{4.495057in}}%
\pgfusepath{clip}%
\pgfsetbuttcap%
\pgfsetroundjoin%
\pgfsetlinewidth{0.000000pt}%
\definecolor{currentstroke}{rgb}{0.000000,0.000000,0.000000}%
\pgfsetstrokecolor{currentstroke}%
\pgfsetdash{}{0pt}%
\pgfpathmoveto{\pgfqpoint{8.743729in}{1.975247in}}%
\pgfpathlineto{\pgfqpoint{8.929956in}{1.975247in}}%
\pgfpathlineto{\pgfqpoint{8.929956in}{2.056975in}}%
\pgfpathlineto{\pgfqpoint{8.743729in}{2.056975in}}%
\pgfpathlineto{\pgfqpoint{8.743729in}{1.975247in}}%
\pgfusepath{}%
\end{pgfscope}%
\begin{pgfscope}%
\pgfpathrectangle{\pgfqpoint{0.549740in}{0.463273in}}{\pgfqpoint{9.320225in}{4.495057in}}%
\pgfusepath{clip}%
\pgfsetbuttcap%
\pgfsetroundjoin%
\pgfsetlinewidth{0.000000pt}%
\definecolor{currentstroke}{rgb}{0.000000,0.000000,0.000000}%
\pgfsetstrokecolor{currentstroke}%
\pgfsetdash{}{0pt}%
\pgfpathmoveto{\pgfqpoint{8.929956in}{1.975247in}}%
\pgfpathlineto{\pgfqpoint{9.116182in}{1.975247in}}%
\pgfpathlineto{\pgfqpoint{9.116182in}{2.056975in}}%
\pgfpathlineto{\pgfqpoint{8.929956in}{2.056975in}}%
\pgfpathlineto{\pgfqpoint{8.929956in}{1.975247in}}%
\pgfusepath{}%
\end{pgfscope}%
\begin{pgfscope}%
\pgfpathrectangle{\pgfqpoint{0.549740in}{0.463273in}}{\pgfqpoint{9.320225in}{4.495057in}}%
\pgfusepath{clip}%
\pgfsetbuttcap%
\pgfsetroundjoin%
\pgfsetlinewidth{0.000000pt}%
\definecolor{currentstroke}{rgb}{0.000000,0.000000,0.000000}%
\pgfsetstrokecolor{currentstroke}%
\pgfsetdash{}{0pt}%
\pgfpathmoveto{\pgfqpoint{9.116182in}{1.975247in}}%
\pgfpathlineto{\pgfqpoint{9.302409in}{1.975247in}}%
\pgfpathlineto{\pgfqpoint{9.302409in}{2.056975in}}%
\pgfpathlineto{\pgfqpoint{9.116182in}{2.056975in}}%
\pgfpathlineto{\pgfqpoint{9.116182in}{1.975247in}}%
\pgfusepath{}%
\end{pgfscope}%
\begin{pgfscope}%
\pgfpathrectangle{\pgfqpoint{0.549740in}{0.463273in}}{\pgfqpoint{9.320225in}{4.495057in}}%
\pgfusepath{clip}%
\pgfsetbuttcap%
\pgfsetroundjoin%
\pgfsetlinewidth{0.000000pt}%
\definecolor{currentstroke}{rgb}{0.000000,0.000000,0.000000}%
\pgfsetstrokecolor{currentstroke}%
\pgfsetdash{}{0pt}%
\pgfpathmoveto{\pgfqpoint{9.302409in}{1.975247in}}%
\pgfpathlineto{\pgfqpoint{9.488635in}{1.975247in}}%
\pgfpathlineto{\pgfqpoint{9.488635in}{2.056975in}}%
\pgfpathlineto{\pgfqpoint{9.302409in}{2.056975in}}%
\pgfpathlineto{\pgfqpoint{9.302409in}{1.975247in}}%
\pgfusepath{}%
\end{pgfscope}%
\begin{pgfscope}%
\pgfpathrectangle{\pgfqpoint{0.549740in}{0.463273in}}{\pgfqpoint{9.320225in}{4.495057in}}%
\pgfusepath{clip}%
\pgfsetbuttcap%
\pgfsetroundjoin%
\pgfsetlinewidth{0.000000pt}%
\definecolor{currentstroke}{rgb}{0.000000,0.000000,0.000000}%
\pgfsetstrokecolor{currentstroke}%
\pgfsetdash{}{0pt}%
\pgfpathmoveto{\pgfqpoint{9.488635in}{1.975247in}}%
\pgfpathlineto{\pgfqpoint{9.674862in}{1.975247in}}%
\pgfpathlineto{\pgfqpoint{9.674862in}{2.056975in}}%
\pgfpathlineto{\pgfqpoint{9.488635in}{2.056975in}}%
\pgfpathlineto{\pgfqpoint{9.488635in}{1.975247in}}%
\pgfusepath{}%
\end{pgfscope}%
\begin{pgfscope}%
\pgfpathrectangle{\pgfqpoint{0.549740in}{0.463273in}}{\pgfqpoint{9.320225in}{4.495057in}}%
\pgfusepath{clip}%
\pgfsetbuttcap%
\pgfsetroundjoin%
\pgfsetlinewidth{0.000000pt}%
\definecolor{currentstroke}{rgb}{0.000000,0.000000,0.000000}%
\pgfsetstrokecolor{currentstroke}%
\pgfsetdash{}{0pt}%
\pgfpathmoveto{\pgfqpoint{9.674862in}{1.975247in}}%
\pgfpathlineto{\pgfqpoint{9.861088in}{1.975247in}}%
\pgfpathlineto{\pgfqpoint{9.861088in}{2.056975in}}%
\pgfpathlineto{\pgfqpoint{9.674862in}{2.056975in}}%
\pgfpathlineto{\pgfqpoint{9.674862in}{1.975247in}}%
\pgfusepath{}%
\end{pgfscope}%
\begin{pgfscope}%
\pgfpathrectangle{\pgfqpoint{0.549740in}{0.463273in}}{\pgfqpoint{9.320225in}{4.495057in}}%
\pgfusepath{clip}%
\pgfsetbuttcap%
\pgfsetroundjoin%
\pgfsetlinewidth{0.000000pt}%
\definecolor{currentstroke}{rgb}{0.000000,0.000000,0.000000}%
\pgfsetstrokecolor{currentstroke}%
\pgfsetdash{}{0pt}%
\pgfpathmoveto{\pgfqpoint{0.549761in}{2.056975in}}%
\pgfpathlineto{\pgfqpoint{0.735988in}{2.056975in}}%
\pgfpathlineto{\pgfqpoint{0.735988in}{2.138704in}}%
\pgfpathlineto{\pgfqpoint{0.549761in}{2.138704in}}%
\pgfpathlineto{\pgfqpoint{0.549761in}{2.056975in}}%
\pgfusepath{}%
\end{pgfscope}%
\begin{pgfscope}%
\pgfpathrectangle{\pgfqpoint{0.549740in}{0.463273in}}{\pgfqpoint{9.320225in}{4.495057in}}%
\pgfusepath{clip}%
\pgfsetbuttcap%
\pgfsetroundjoin%
\pgfsetlinewidth{0.000000pt}%
\definecolor{currentstroke}{rgb}{0.000000,0.000000,0.000000}%
\pgfsetstrokecolor{currentstroke}%
\pgfsetdash{}{0pt}%
\pgfpathmoveto{\pgfqpoint{0.735988in}{2.056975in}}%
\pgfpathlineto{\pgfqpoint{0.922214in}{2.056975in}}%
\pgfpathlineto{\pgfqpoint{0.922214in}{2.138704in}}%
\pgfpathlineto{\pgfqpoint{0.735988in}{2.138704in}}%
\pgfpathlineto{\pgfqpoint{0.735988in}{2.056975in}}%
\pgfusepath{}%
\end{pgfscope}%
\begin{pgfscope}%
\pgfpathrectangle{\pgfqpoint{0.549740in}{0.463273in}}{\pgfqpoint{9.320225in}{4.495057in}}%
\pgfusepath{clip}%
\pgfsetbuttcap%
\pgfsetroundjoin%
\pgfsetlinewidth{0.000000pt}%
\definecolor{currentstroke}{rgb}{0.000000,0.000000,0.000000}%
\pgfsetstrokecolor{currentstroke}%
\pgfsetdash{}{0pt}%
\pgfpathmoveto{\pgfqpoint{0.922214in}{2.056975in}}%
\pgfpathlineto{\pgfqpoint{1.108441in}{2.056975in}}%
\pgfpathlineto{\pgfqpoint{1.108441in}{2.138704in}}%
\pgfpathlineto{\pgfqpoint{0.922214in}{2.138704in}}%
\pgfpathlineto{\pgfqpoint{0.922214in}{2.056975in}}%
\pgfusepath{}%
\end{pgfscope}%
\begin{pgfscope}%
\pgfpathrectangle{\pgfqpoint{0.549740in}{0.463273in}}{\pgfqpoint{9.320225in}{4.495057in}}%
\pgfusepath{clip}%
\pgfsetbuttcap%
\pgfsetroundjoin%
\pgfsetlinewidth{0.000000pt}%
\definecolor{currentstroke}{rgb}{0.000000,0.000000,0.000000}%
\pgfsetstrokecolor{currentstroke}%
\pgfsetdash{}{0pt}%
\pgfpathmoveto{\pgfqpoint{1.108441in}{2.056975in}}%
\pgfpathlineto{\pgfqpoint{1.294667in}{2.056975in}}%
\pgfpathlineto{\pgfqpoint{1.294667in}{2.138704in}}%
\pgfpathlineto{\pgfqpoint{1.108441in}{2.138704in}}%
\pgfpathlineto{\pgfqpoint{1.108441in}{2.056975in}}%
\pgfusepath{}%
\end{pgfscope}%
\begin{pgfscope}%
\pgfpathrectangle{\pgfqpoint{0.549740in}{0.463273in}}{\pgfqpoint{9.320225in}{4.495057in}}%
\pgfusepath{clip}%
\pgfsetbuttcap%
\pgfsetroundjoin%
\pgfsetlinewidth{0.000000pt}%
\definecolor{currentstroke}{rgb}{0.000000,0.000000,0.000000}%
\pgfsetstrokecolor{currentstroke}%
\pgfsetdash{}{0pt}%
\pgfpathmoveto{\pgfqpoint{1.294667in}{2.056975in}}%
\pgfpathlineto{\pgfqpoint{1.480894in}{2.056975in}}%
\pgfpathlineto{\pgfqpoint{1.480894in}{2.138704in}}%
\pgfpathlineto{\pgfqpoint{1.294667in}{2.138704in}}%
\pgfpathlineto{\pgfqpoint{1.294667in}{2.056975in}}%
\pgfusepath{}%
\end{pgfscope}%
\begin{pgfscope}%
\pgfpathrectangle{\pgfqpoint{0.549740in}{0.463273in}}{\pgfqpoint{9.320225in}{4.495057in}}%
\pgfusepath{clip}%
\pgfsetbuttcap%
\pgfsetroundjoin%
\pgfsetlinewidth{0.000000pt}%
\definecolor{currentstroke}{rgb}{0.000000,0.000000,0.000000}%
\pgfsetstrokecolor{currentstroke}%
\pgfsetdash{}{0pt}%
\pgfpathmoveto{\pgfqpoint{1.480894in}{2.056975in}}%
\pgfpathlineto{\pgfqpoint{1.667120in}{2.056975in}}%
\pgfpathlineto{\pgfqpoint{1.667120in}{2.138704in}}%
\pgfpathlineto{\pgfqpoint{1.480894in}{2.138704in}}%
\pgfpathlineto{\pgfqpoint{1.480894in}{2.056975in}}%
\pgfusepath{}%
\end{pgfscope}%
\begin{pgfscope}%
\pgfpathrectangle{\pgfqpoint{0.549740in}{0.463273in}}{\pgfqpoint{9.320225in}{4.495057in}}%
\pgfusepath{clip}%
\pgfsetbuttcap%
\pgfsetroundjoin%
\pgfsetlinewidth{0.000000pt}%
\definecolor{currentstroke}{rgb}{0.000000,0.000000,0.000000}%
\pgfsetstrokecolor{currentstroke}%
\pgfsetdash{}{0pt}%
\pgfpathmoveto{\pgfqpoint{1.667120in}{2.056975in}}%
\pgfpathlineto{\pgfqpoint{1.853347in}{2.056975in}}%
\pgfpathlineto{\pgfqpoint{1.853347in}{2.138704in}}%
\pgfpathlineto{\pgfqpoint{1.667120in}{2.138704in}}%
\pgfpathlineto{\pgfqpoint{1.667120in}{2.056975in}}%
\pgfusepath{}%
\end{pgfscope}%
\begin{pgfscope}%
\pgfpathrectangle{\pgfqpoint{0.549740in}{0.463273in}}{\pgfqpoint{9.320225in}{4.495057in}}%
\pgfusepath{clip}%
\pgfsetbuttcap%
\pgfsetroundjoin%
\pgfsetlinewidth{0.000000pt}%
\definecolor{currentstroke}{rgb}{0.000000,0.000000,0.000000}%
\pgfsetstrokecolor{currentstroke}%
\pgfsetdash{}{0pt}%
\pgfpathmoveto{\pgfqpoint{1.853347in}{2.056975in}}%
\pgfpathlineto{\pgfqpoint{2.039573in}{2.056975in}}%
\pgfpathlineto{\pgfqpoint{2.039573in}{2.138704in}}%
\pgfpathlineto{\pgfqpoint{1.853347in}{2.138704in}}%
\pgfpathlineto{\pgfqpoint{1.853347in}{2.056975in}}%
\pgfusepath{}%
\end{pgfscope}%
\begin{pgfscope}%
\pgfpathrectangle{\pgfqpoint{0.549740in}{0.463273in}}{\pgfqpoint{9.320225in}{4.495057in}}%
\pgfusepath{clip}%
\pgfsetbuttcap%
\pgfsetroundjoin%
\pgfsetlinewidth{0.000000pt}%
\definecolor{currentstroke}{rgb}{0.000000,0.000000,0.000000}%
\pgfsetstrokecolor{currentstroke}%
\pgfsetdash{}{0pt}%
\pgfpathmoveto{\pgfqpoint{2.039573in}{2.056975in}}%
\pgfpathlineto{\pgfqpoint{2.225800in}{2.056975in}}%
\pgfpathlineto{\pgfqpoint{2.225800in}{2.138704in}}%
\pgfpathlineto{\pgfqpoint{2.039573in}{2.138704in}}%
\pgfpathlineto{\pgfqpoint{2.039573in}{2.056975in}}%
\pgfusepath{}%
\end{pgfscope}%
\begin{pgfscope}%
\pgfpathrectangle{\pgfqpoint{0.549740in}{0.463273in}}{\pgfqpoint{9.320225in}{4.495057in}}%
\pgfusepath{clip}%
\pgfsetbuttcap%
\pgfsetroundjoin%
\pgfsetlinewidth{0.000000pt}%
\definecolor{currentstroke}{rgb}{0.000000,0.000000,0.000000}%
\pgfsetstrokecolor{currentstroke}%
\pgfsetdash{}{0pt}%
\pgfpathmoveto{\pgfqpoint{2.225800in}{2.056975in}}%
\pgfpathlineto{\pgfqpoint{2.412027in}{2.056975in}}%
\pgfpathlineto{\pgfqpoint{2.412027in}{2.138704in}}%
\pgfpathlineto{\pgfqpoint{2.225800in}{2.138704in}}%
\pgfpathlineto{\pgfqpoint{2.225800in}{2.056975in}}%
\pgfusepath{}%
\end{pgfscope}%
\begin{pgfscope}%
\pgfpathrectangle{\pgfqpoint{0.549740in}{0.463273in}}{\pgfqpoint{9.320225in}{4.495057in}}%
\pgfusepath{clip}%
\pgfsetbuttcap%
\pgfsetroundjoin%
\pgfsetlinewidth{0.000000pt}%
\definecolor{currentstroke}{rgb}{0.000000,0.000000,0.000000}%
\pgfsetstrokecolor{currentstroke}%
\pgfsetdash{}{0pt}%
\pgfpathmoveto{\pgfqpoint{2.412027in}{2.056975in}}%
\pgfpathlineto{\pgfqpoint{2.598253in}{2.056975in}}%
\pgfpathlineto{\pgfqpoint{2.598253in}{2.138704in}}%
\pgfpathlineto{\pgfqpoint{2.412027in}{2.138704in}}%
\pgfpathlineto{\pgfqpoint{2.412027in}{2.056975in}}%
\pgfusepath{}%
\end{pgfscope}%
\begin{pgfscope}%
\pgfpathrectangle{\pgfqpoint{0.549740in}{0.463273in}}{\pgfqpoint{9.320225in}{4.495057in}}%
\pgfusepath{clip}%
\pgfsetbuttcap%
\pgfsetroundjoin%
\pgfsetlinewidth{0.000000pt}%
\definecolor{currentstroke}{rgb}{0.000000,0.000000,0.000000}%
\pgfsetstrokecolor{currentstroke}%
\pgfsetdash{}{0pt}%
\pgfpathmoveto{\pgfqpoint{2.598253in}{2.056975in}}%
\pgfpathlineto{\pgfqpoint{2.784480in}{2.056975in}}%
\pgfpathlineto{\pgfqpoint{2.784480in}{2.138704in}}%
\pgfpathlineto{\pgfqpoint{2.598253in}{2.138704in}}%
\pgfpathlineto{\pgfqpoint{2.598253in}{2.056975in}}%
\pgfusepath{}%
\end{pgfscope}%
\begin{pgfscope}%
\pgfpathrectangle{\pgfqpoint{0.549740in}{0.463273in}}{\pgfqpoint{9.320225in}{4.495057in}}%
\pgfusepath{clip}%
\pgfsetbuttcap%
\pgfsetroundjoin%
\pgfsetlinewidth{0.000000pt}%
\definecolor{currentstroke}{rgb}{0.000000,0.000000,0.000000}%
\pgfsetstrokecolor{currentstroke}%
\pgfsetdash{}{0pt}%
\pgfpathmoveto{\pgfqpoint{2.784480in}{2.056975in}}%
\pgfpathlineto{\pgfqpoint{2.970706in}{2.056975in}}%
\pgfpathlineto{\pgfqpoint{2.970706in}{2.138704in}}%
\pgfpathlineto{\pgfqpoint{2.784480in}{2.138704in}}%
\pgfpathlineto{\pgfqpoint{2.784480in}{2.056975in}}%
\pgfusepath{}%
\end{pgfscope}%
\begin{pgfscope}%
\pgfpathrectangle{\pgfqpoint{0.549740in}{0.463273in}}{\pgfqpoint{9.320225in}{4.495057in}}%
\pgfusepath{clip}%
\pgfsetbuttcap%
\pgfsetroundjoin%
\pgfsetlinewidth{0.000000pt}%
\definecolor{currentstroke}{rgb}{0.000000,0.000000,0.000000}%
\pgfsetstrokecolor{currentstroke}%
\pgfsetdash{}{0pt}%
\pgfpathmoveto{\pgfqpoint{2.970706in}{2.056975in}}%
\pgfpathlineto{\pgfqpoint{3.156933in}{2.056975in}}%
\pgfpathlineto{\pgfqpoint{3.156933in}{2.138704in}}%
\pgfpathlineto{\pgfqpoint{2.970706in}{2.138704in}}%
\pgfpathlineto{\pgfqpoint{2.970706in}{2.056975in}}%
\pgfusepath{}%
\end{pgfscope}%
\begin{pgfscope}%
\pgfpathrectangle{\pgfqpoint{0.549740in}{0.463273in}}{\pgfqpoint{9.320225in}{4.495057in}}%
\pgfusepath{clip}%
\pgfsetbuttcap%
\pgfsetroundjoin%
\pgfsetlinewidth{0.000000pt}%
\definecolor{currentstroke}{rgb}{0.000000,0.000000,0.000000}%
\pgfsetstrokecolor{currentstroke}%
\pgfsetdash{}{0pt}%
\pgfpathmoveto{\pgfqpoint{3.156933in}{2.056975in}}%
\pgfpathlineto{\pgfqpoint{3.343159in}{2.056975in}}%
\pgfpathlineto{\pgfqpoint{3.343159in}{2.138704in}}%
\pgfpathlineto{\pgfqpoint{3.156933in}{2.138704in}}%
\pgfpathlineto{\pgfqpoint{3.156933in}{2.056975in}}%
\pgfusepath{}%
\end{pgfscope}%
\begin{pgfscope}%
\pgfpathrectangle{\pgfqpoint{0.549740in}{0.463273in}}{\pgfqpoint{9.320225in}{4.495057in}}%
\pgfusepath{clip}%
\pgfsetbuttcap%
\pgfsetroundjoin%
\pgfsetlinewidth{0.000000pt}%
\definecolor{currentstroke}{rgb}{0.000000,0.000000,0.000000}%
\pgfsetstrokecolor{currentstroke}%
\pgfsetdash{}{0pt}%
\pgfpathmoveto{\pgfqpoint{3.343159in}{2.056975in}}%
\pgfpathlineto{\pgfqpoint{3.529386in}{2.056975in}}%
\pgfpathlineto{\pgfqpoint{3.529386in}{2.138704in}}%
\pgfpathlineto{\pgfqpoint{3.343159in}{2.138704in}}%
\pgfpathlineto{\pgfqpoint{3.343159in}{2.056975in}}%
\pgfusepath{}%
\end{pgfscope}%
\begin{pgfscope}%
\pgfpathrectangle{\pgfqpoint{0.549740in}{0.463273in}}{\pgfqpoint{9.320225in}{4.495057in}}%
\pgfusepath{clip}%
\pgfsetbuttcap%
\pgfsetroundjoin%
\pgfsetlinewidth{0.000000pt}%
\definecolor{currentstroke}{rgb}{0.000000,0.000000,0.000000}%
\pgfsetstrokecolor{currentstroke}%
\pgfsetdash{}{0pt}%
\pgfpathmoveto{\pgfqpoint{3.529386in}{2.056975in}}%
\pgfpathlineto{\pgfqpoint{3.715612in}{2.056975in}}%
\pgfpathlineto{\pgfqpoint{3.715612in}{2.138704in}}%
\pgfpathlineto{\pgfqpoint{3.529386in}{2.138704in}}%
\pgfpathlineto{\pgfqpoint{3.529386in}{2.056975in}}%
\pgfusepath{}%
\end{pgfscope}%
\begin{pgfscope}%
\pgfpathrectangle{\pgfqpoint{0.549740in}{0.463273in}}{\pgfqpoint{9.320225in}{4.495057in}}%
\pgfusepath{clip}%
\pgfsetbuttcap%
\pgfsetroundjoin%
\pgfsetlinewidth{0.000000pt}%
\definecolor{currentstroke}{rgb}{0.000000,0.000000,0.000000}%
\pgfsetstrokecolor{currentstroke}%
\pgfsetdash{}{0pt}%
\pgfpathmoveto{\pgfqpoint{3.715612in}{2.056975in}}%
\pgfpathlineto{\pgfqpoint{3.901839in}{2.056975in}}%
\pgfpathlineto{\pgfqpoint{3.901839in}{2.138704in}}%
\pgfpathlineto{\pgfqpoint{3.715612in}{2.138704in}}%
\pgfpathlineto{\pgfqpoint{3.715612in}{2.056975in}}%
\pgfusepath{}%
\end{pgfscope}%
\begin{pgfscope}%
\pgfpathrectangle{\pgfqpoint{0.549740in}{0.463273in}}{\pgfqpoint{9.320225in}{4.495057in}}%
\pgfusepath{clip}%
\pgfsetbuttcap%
\pgfsetroundjoin%
\pgfsetlinewidth{0.000000pt}%
\definecolor{currentstroke}{rgb}{0.000000,0.000000,0.000000}%
\pgfsetstrokecolor{currentstroke}%
\pgfsetdash{}{0pt}%
\pgfpathmoveto{\pgfqpoint{3.901839in}{2.056975in}}%
\pgfpathlineto{\pgfqpoint{4.088065in}{2.056975in}}%
\pgfpathlineto{\pgfqpoint{4.088065in}{2.138704in}}%
\pgfpathlineto{\pgfqpoint{3.901839in}{2.138704in}}%
\pgfpathlineto{\pgfqpoint{3.901839in}{2.056975in}}%
\pgfusepath{}%
\end{pgfscope}%
\begin{pgfscope}%
\pgfpathrectangle{\pgfqpoint{0.549740in}{0.463273in}}{\pgfqpoint{9.320225in}{4.495057in}}%
\pgfusepath{clip}%
\pgfsetbuttcap%
\pgfsetroundjoin%
\pgfsetlinewidth{0.000000pt}%
\definecolor{currentstroke}{rgb}{0.000000,0.000000,0.000000}%
\pgfsetstrokecolor{currentstroke}%
\pgfsetdash{}{0pt}%
\pgfpathmoveto{\pgfqpoint{4.088065in}{2.056975in}}%
\pgfpathlineto{\pgfqpoint{4.274292in}{2.056975in}}%
\pgfpathlineto{\pgfqpoint{4.274292in}{2.138704in}}%
\pgfpathlineto{\pgfqpoint{4.088065in}{2.138704in}}%
\pgfpathlineto{\pgfqpoint{4.088065in}{2.056975in}}%
\pgfusepath{}%
\end{pgfscope}%
\begin{pgfscope}%
\pgfpathrectangle{\pgfqpoint{0.549740in}{0.463273in}}{\pgfqpoint{9.320225in}{4.495057in}}%
\pgfusepath{clip}%
\pgfsetbuttcap%
\pgfsetroundjoin%
\pgfsetlinewidth{0.000000pt}%
\definecolor{currentstroke}{rgb}{0.000000,0.000000,0.000000}%
\pgfsetstrokecolor{currentstroke}%
\pgfsetdash{}{0pt}%
\pgfpathmoveto{\pgfqpoint{4.274292in}{2.056975in}}%
\pgfpathlineto{\pgfqpoint{4.460519in}{2.056975in}}%
\pgfpathlineto{\pgfqpoint{4.460519in}{2.138704in}}%
\pgfpathlineto{\pgfqpoint{4.274292in}{2.138704in}}%
\pgfpathlineto{\pgfqpoint{4.274292in}{2.056975in}}%
\pgfusepath{}%
\end{pgfscope}%
\begin{pgfscope}%
\pgfpathrectangle{\pgfqpoint{0.549740in}{0.463273in}}{\pgfqpoint{9.320225in}{4.495057in}}%
\pgfusepath{clip}%
\pgfsetbuttcap%
\pgfsetroundjoin%
\pgfsetlinewidth{0.000000pt}%
\definecolor{currentstroke}{rgb}{0.000000,0.000000,0.000000}%
\pgfsetstrokecolor{currentstroke}%
\pgfsetdash{}{0pt}%
\pgfpathmoveto{\pgfqpoint{4.460519in}{2.056975in}}%
\pgfpathlineto{\pgfqpoint{4.646745in}{2.056975in}}%
\pgfpathlineto{\pgfqpoint{4.646745in}{2.138704in}}%
\pgfpathlineto{\pgfqpoint{4.460519in}{2.138704in}}%
\pgfpathlineto{\pgfqpoint{4.460519in}{2.056975in}}%
\pgfusepath{}%
\end{pgfscope}%
\begin{pgfscope}%
\pgfpathrectangle{\pgfqpoint{0.549740in}{0.463273in}}{\pgfqpoint{9.320225in}{4.495057in}}%
\pgfusepath{clip}%
\pgfsetbuttcap%
\pgfsetroundjoin%
\pgfsetlinewidth{0.000000pt}%
\definecolor{currentstroke}{rgb}{0.000000,0.000000,0.000000}%
\pgfsetstrokecolor{currentstroke}%
\pgfsetdash{}{0pt}%
\pgfpathmoveto{\pgfqpoint{4.646745in}{2.056975in}}%
\pgfpathlineto{\pgfqpoint{4.832972in}{2.056975in}}%
\pgfpathlineto{\pgfqpoint{4.832972in}{2.138704in}}%
\pgfpathlineto{\pgfqpoint{4.646745in}{2.138704in}}%
\pgfpathlineto{\pgfqpoint{4.646745in}{2.056975in}}%
\pgfusepath{}%
\end{pgfscope}%
\begin{pgfscope}%
\pgfpathrectangle{\pgfqpoint{0.549740in}{0.463273in}}{\pgfqpoint{9.320225in}{4.495057in}}%
\pgfusepath{clip}%
\pgfsetbuttcap%
\pgfsetroundjoin%
\pgfsetlinewidth{0.000000pt}%
\definecolor{currentstroke}{rgb}{0.000000,0.000000,0.000000}%
\pgfsetstrokecolor{currentstroke}%
\pgfsetdash{}{0pt}%
\pgfpathmoveto{\pgfqpoint{4.832972in}{2.056975in}}%
\pgfpathlineto{\pgfqpoint{5.019198in}{2.056975in}}%
\pgfpathlineto{\pgfqpoint{5.019198in}{2.138704in}}%
\pgfpathlineto{\pgfqpoint{4.832972in}{2.138704in}}%
\pgfpathlineto{\pgfqpoint{4.832972in}{2.056975in}}%
\pgfusepath{}%
\end{pgfscope}%
\begin{pgfscope}%
\pgfpathrectangle{\pgfqpoint{0.549740in}{0.463273in}}{\pgfqpoint{9.320225in}{4.495057in}}%
\pgfusepath{clip}%
\pgfsetbuttcap%
\pgfsetroundjoin%
\pgfsetlinewidth{0.000000pt}%
\definecolor{currentstroke}{rgb}{0.000000,0.000000,0.000000}%
\pgfsetstrokecolor{currentstroke}%
\pgfsetdash{}{0pt}%
\pgfpathmoveto{\pgfqpoint{5.019198in}{2.056975in}}%
\pgfpathlineto{\pgfqpoint{5.205425in}{2.056975in}}%
\pgfpathlineto{\pgfqpoint{5.205425in}{2.138704in}}%
\pgfpathlineto{\pgfqpoint{5.019198in}{2.138704in}}%
\pgfpathlineto{\pgfqpoint{5.019198in}{2.056975in}}%
\pgfusepath{}%
\end{pgfscope}%
\begin{pgfscope}%
\pgfpathrectangle{\pgfqpoint{0.549740in}{0.463273in}}{\pgfqpoint{9.320225in}{4.495057in}}%
\pgfusepath{clip}%
\pgfsetbuttcap%
\pgfsetroundjoin%
\pgfsetlinewidth{0.000000pt}%
\definecolor{currentstroke}{rgb}{0.000000,0.000000,0.000000}%
\pgfsetstrokecolor{currentstroke}%
\pgfsetdash{}{0pt}%
\pgfpathmoveto{\pgfqpoint{5.205425in}{2.056975in}}%
\pgfpathlineto{\pgfqpoint{5.391651in}{2.056975in}}%
\pgfpathlineto{\pgfqpoint{5.391651in}{2.138704in}}%
\pgfpathlineto{\pgfqpoint{5.205425in}{2.138704in}}%
\pgfpathlineto{\pgfqpoint{5.205425in}{2.056975in}}%
\pgfusepath{}%
\end{pgfscope}%
\begin{pgfscope}%
\pgfpathrectangle{\pgfqpoint{0.549740in}{0.463273in}}{\pgfqpoint{9.320225in}{4.495057in}}%
\pgfusepath{clip}%
\pgfsetbuttcap%
\pgfsetroundjoin%
\definecolor{currentfill}{rgb}{0.472869,0.711325,0.955316}%
\pgfsetfillcolor{currentfill}%
\pgfsetlinewidth{0.000000pt}%
\definecolor{currentstroke}{rgb}{0.000000,0.000000,0.000000}%
\pgfsetstrokecolor{currentstroke}%
\pgfsetdash{}{0pt}%
\pgfpathmoveto{\pgfqpoint{5.391651in}{2.056975in}}%
\pgfpathlineto{\pgfqpoint{5.577878in}{2.056975in}}%
\pgfpathlineto{\pgfqpoint{5.577878in}{2.138704in}}%
\pgfpathlineto{\pgfqpoint{5.391651in}{2.138704in}}%
\pgfpathlineto{\pgfqpoint{5.391651in}{2.056975in}}%
\pgfusepath{fill}%
\end{pgfscope}%
\begin{pgfscope}%
\pgfpathrectangle{\pgfqpoint{0.549740in}{0.463273in}}{\pgfqpoint{9.320225in}{4.495057in}}%
\pgfusepath{clip}%
\pgfsetbuttcap%
\pgfsetroundjoin%
\pgfsetlinewidth{0.000000pt}%
\definecolor{currentstroke}{rgb}{0.000000,0.000000,0.000000}%
\pgfsetstrokecolor{currentstroke}%
\pgfsetdash{}{0pt}%
\pgfpathmoveto{\pgfqpoint{5.577878in}{2.056975in}}%
\pgfpathlineto{\pgfqpoint{5.764104in}{2.056975in}}%
\pgfpathlineto{\pgfqpoint{5.764104in}{2.138704in}}%
\pgfpathlineto{\pgfqpoint{5.577878in}{2.138704in}}%
\pgfpathlineto{\pgfqpoint{5.577878in}{2.056975in}}%
\pgfusepath{}%
\end{pgfscope}%
\begin{pgfscope}%
\pgfpathrectangle{\pgfqpoint{0.549740in}{0.463273in}}{\pgfqpoint{9.320225in}{4.495057in}}%
\pgfusepath{clip}%
\pgfsetbuttcap%
\pgfsetroundjoin%
\pgfsetlinewidth{0.000000pt}%
\definecolor{currentstroke}{rgb}{0.000000,0.000000,0.000000}%
\pgfsetstrokecolor{currentstroke}%
\pgfsetdash{}{0pt}%
\pgfpathmoveto{\pgfqpoint{5.764104in}{2.056975in}}%
\pgfpathlineto{\pgfqpoint{5.950331in}{2.056975in}}%
\pgfpathlineto{\pgfqpoint{5.950331in}{2.138704in}}%
\pgfpathlineto{\pgfqpoint{5.764104in}{2.138704in}}%
\pgfpathlineto{\pgfqpoint{5.764104in}{2.056975in}}%
\pgfusepath{}%
\end{pgfscope}%
\begin{pgfscope}%
\pgfpathrectangle{\pgfqpoint{0.549740in}{0.463273in}}{\pgfqpoint{9.320225in}{4.495057in}}%
\pgfusepath{clip}%
\pgfsetbuttcap%
\pgfsetroundjoin%
\pgfsetlinewidth{0.000000pt}%
\definecolor{currentstroke}{rgb}{0.000000,0.000000,0.000000}%
\pgfsetstrokecolor{currentstroke}%
\pgfsetdash{}{0pt}%
\pgfpathmoveto{\pgfqpoint{5.950331in}{2.056975in}}%
\pgfpathlineto{\pgfqpoint{6.136557in}{2.056975in}}%
\pgfpathlineto{\pgfqpoint{6.136557in}{2.138704in}}%
\pgfpathlineto{\pgfqpoint{5.950331in}{2.138704in}}%
\pgfpathlineto{\pgfqpoint{5.950331in}{2.056975in}}%
\pgfusepath{}%
\end{pgfscope}%
\begin{pgfscope}%
\pgfpathrectangle{\pgfqpoint{0.549740in}{0.463273in}}{\pgfqpoint{9.320225in}{4.495057in}}%
\pgfusepath{clip}%
\pgfsetbuttcap%
\pgfsetroundjoin%
\pgfsetlinewidth{0.000000pt}%
\definecolor{currentstroke}{rgb}{0.000000,0.000000,0.000000}%
\pgfsetstrokecolor{currentstroke}%
\pgfsetdash{}{0pt}%
\pgfpathmoveto{\pgfqpoint{6.136557in}{2.056975in}}%
\pgfpathlineto{\pgfqpoint{6.322784in}{2.056975in}}%
\pgfpathlineto{\pgfqpoint{6.322784in}{2.138704in}}%
\pgfpathlineto{\pgfqpoint{6.136557in}{2.138704in}}%
\pgfpathlineto{\pgfqpoint{6.136557in}{2.056975in}}%
\pgfusepath{}%
\end{pgfscope}%
\begin{pgfscope}%
\pgfpathrectangle{\pgfqpoint{0.549740in}{0.463273in}}{\pgfqpoint{9.320225in}{4.495057in}}%
\pgfusepath{clip}%
\pgfsetbuttcap%
\pgfsetroundjoin%
\definecolor{currentfill}{rgb}{0.472869,0.711325,0.955316}%
\pgfsetfillcolor{currentfill}%
\pgfsetlinewidth{0.000000pt}%
\definecolor{currentstroke}{rgb}{0.000000,0.000000,0.000000}%
\pgfsetstrokecolor{currentstroke}%
\pgfsetdash{}{0pt}%
\pgfpathmoveto{\pgfqpoint{6.322784in}{2.056975in}}%
\pgfpathlineto{\pgfqpoint{6.509011in}{2.056975in}}%
\pgfpathlineto{\pgfqpoint{6.509011in}{2.138704in}}%
\pgfpathlineto{\pgfqpoint{6.322784in}{2.138704in}}%
\pgfpathlineto{\pgfqpoint{6.322784in}{2.056975in}}%
\pgfusepath{fill}%
\end{pgfscope}%
\begin{pgfscope}%
\pgfpathrectangle{\pgfqpoint{0.549740in}{0.463273in}}{\pgfqpoint{9.320225in}{4.495057in}}%
\pgfusepath{clip}%
\pgfsetbuttcap%
\pgfsetroundjoin%
\pgfsetlinewidth{0.000000pt}%
\definecolor{currentstroke}{rgb}{0.000000,0.000000,0.000000}%
\pgfsetstrokecolor{currentstroke}%
\pgfsetdash{}{0pt}%
\pgfpathmoveto{\pgfqpoint{6.509011in}{2.056975in}}%
\pgfpathlineto{\pgfqpoint{6.695237in}{2.056975in}}%
\pgfpathlineto{\pgfqpoint{6.695237in}{2.138704in}}%
\pgfpathlineto{\pgfqpoint{6.509011in}{2.138704in}}%
\pgfpathlineto{\pgfqpoint{6.509011in}{2.056975in}}%
\pgfusepath{}%
\end{pgfscope}%
\begin{pgfscope}%
\pgfpathrectangle{\pgfqpoint{0.549740in}{0.463273in}}{\pgfqpoint{9.320225in}{4.495057in}}%
\pgfusepath{clip}%
\pgfsetbuttcap%
\pgfsetroundjoin%
\pgfsetlinewidth{0.000000pt}%
\definecolor{currentstroke}{rgb}{0.000000,0.000000,0.000000}%
\pgfsetstrokecolor{currentstroke}%
\pgfsetdash{}{0pt}%
\pgfpathmoveto{\pgfqpoint{6.695237in}{2.056975in}}%
\pgfpathlineto{\pgfqpoint{6.881464in}{2.056975in}}%
\pgfpathlineto{\pgfqpoint{6.881464in}{2.138704in}}%
\pgfpathlineto{\pgfqpoint{6.695237in}{2.138704in}}%
\pgfpathlineto{\pgfqpoint{6.695237in}{2.056975in}}%
\pgfusepath{}%
\end{pgfscope}%
\begin{pgfscope}%
\pgfpathrectangle{\pgfqpoint{0.549740in}{0.463273in}}{\pgfqpoint{9.320225in}{4.495057in}}%
\pgfusepath{clip}%
\pgfsetbuttcap%
\pgfsetroundjoin%
\pgfsetlinewidth{0.000000pt}%
\definecolor{currentstroke}{rgb}{0.000000,0.000000,0.000000}%
\pgfsetstrokecolor{currentstroke}%
\pgfsetdash{}{0pt}%
\pgfpathmoveto{\pgfqpoint{6.881464in}{2.056975in}}%
\pgfpathlineto{\pgfqpoint{7.067690in}{2.056975in}}%
\pgfpathlineto{\pgfqpoint{7.067690in}{2.138704in}}%
\pgfpathlineto{\pgfqpoint{6.881464in}{2.138704in}}%
\pgfpathlineto{\pgfqpoint{6.881464in}{2.056975in}}%
\pgfusepath{}%
\end{pgfscope}%
\begin{pgfscope}%
\pgfpathrectangle{\pgfqpoint{0.549740in}{0.463273in}}{\pgfqpoint{9.320225in}{4.495057in}}%
\pgfusepath{clip}%
\pgfsetbuttcap%
\pgfsetroundjoin%
\pgfsetlinewidth{0.000000pt}%
\definecolor{currentstroke}{rgb}{0.000000,0.000000,0.000000}%
\pgfsetstrokecolor{currentstroke}%
\pgfsetdash{}{0pt}%
\pgfpathmoveto{\pgfqpoint{7.067690in}{2.056975in}}%
\pgfpathlineto{\pgfqpoint{7.253917in}{2.056975in}}%
\pgfpathlineto{\pgfqpoint{7.253917in}{2.138704in}}%
\pgfpathlineto{\pgfqpoint{7.067690in}{2.138704in}}%
\pgfpathlineto{\pgfqpoint{7.067690in}{2.056975in}}%
\pgfusepath{}%
\end{pgfscope}%
\begin{pgfscope}%
\pgfpathrectangle{\pgfqpoint{0.549740in}{0.463273in}}{\pgfqpoint{9.320225in}{4.495057in}}%
\pgfusepath{clip}%
\pgfsetbuttcap%
\pgfsetroundjoin%
\pgfsetlinewidth{0.000000pt}%
\definecolor{currentstroke}{rgb}{0.000000,0.000000,0.000000}%
\pgfsetstrokecolor{currentstroke}%
\pgfsetdash{}{0pt}%
\pgfpathmoveto{\pgfqpoint{7.253917in}{2.056975in}}%
\pgfpathlineto{\pgfqpoint{7.440143in}{2.056975in}}%
\pgfpathlineto{\pgfqpoint{7.440143in}{2.138704in}}%
\pgfpathlineto{\pgfqpoint{7.253917in}{2.138704in}}%
\pgfpathlineto{\pgfqpoint{7.253917in}{2.056975in}}%
\pgfusepath{}%
\end{pgfscope}%
\begin{pgfscope}%
\pgfpathrectangle{\pgfqpoint{0.549740in}{0.463273in}}{\pgfqpoint{9.320225in}{4.495057in}}%
\pgfusepath{clip}%
\pgfsetbuttcap%
\pgfsetroundjoin%
\definecolor{currentfill}{rgb}{0.472869,0.711325,0.955316}%
\pgfsetfillcolor{currentfill}%
\pgfsetlinewidth{0.000000pt}%
\definecolor{currentstroke}{rgb}{0.000000,0.000000,0.000000}%
\pgfsetstrokecolor{currentstroke}%
\pgfsetdash{}{0pt}%
\pgfpathmoveto{\pgfqpoint{7.440143in}{2.056975in}}%
\pgfpathlineto{\pgfqpoint{7.626370in}{2.056975in}}%
\pgfpathlineto{\pgfqpoint{7.626370in}{2.138704in}}%
\pgfpathlineto{\pgfqpoint{7.440143in}{2.138704in}}%
\pgfpathlineto{\pgfqpoint{7.440143in}{2.056975in}}%
\pgfusepath{fill}%
\end{pgfscope}%
\begin{pgfscope}%
\pgfpathrectangle{\pgfqpoint{0.549740in}{0.463273in}}{\pgfqpoint{9.320225in}{4.495057in}}%
\pgfusepath{clip}%
\pgfsetbuttcap%
\pgfsetroundjoin%
\pgfsetlinewidth{0.000000pt}%
\definecolor{currentstroke}{rgb}{0.000000,0.000000,0.000000}%
\pgfsetstrokecolor{currentstroke}%
\pgfsetdash{}{0pt}%
\pgfpathmoveto{\pgfqpoint{7.626370in}{2.056975in}}%
\pgfpathlineto{\pgfqpoint{7.812596in}{2.056975in}}%
\pgfpathlineto{\pgfqpoint{7.812596in}{2.138704in}}%
\pgfpathlineto{\pgfqpoint{7.626370in}{2.138704in}}%
\pgfpathlineto{\pgfqpoint{7.626370in}{2.056975in}}%
\pgfusepath{}%
\end{pgfscope}%
\begin{pgfscope}%
\pgfpathrectangle{\pgfqpoint{0.549740in}{0.463273in}}{\pgfqpoint{9.320225in}{4.495057in}}%
\pgfusepath{clip}%
\pgfsetbuttcap%
\pgfsetroundjoin%
\pgfsetlinewidth{0.000000pt}%
\definecolor{currentstroke}{rgb}{0.000000,0.000000,0.000000}%
\pgfsetstrokecolor{currentstroke}%
\pgfsetdash{}{0pt}%
\pgfpathmoveto{\pgfqpoint{7.812596in}{2.056975in}}%
\pgfpathlineto{\pgfqpoint{7.998823in}{2.056975in}}%
\pgfpathlineto{\pgfqpoint{7.998823in}{2.138704in}}%
\pgfpathlineto{\pgfqpoint{7.812596in}{2.138704in}}%
\pgfpathlineto{\pgfqpoint{7.812596in}{2.056975in}}%
\pgfusepath{}%
\end{pgfscope}%
\begin{pgfscope}%
\pgfpathrectangle{\pgfqpoint{0.549740in}{0.463273in}}{\pgfqpoint{9.320225in}{4.495057in}}%
\pgfusepath{clip}%
\pgfsetbuttcap%
\pgfsetroundjoin%
\pgfsetlinewidth{0.000000pt}%
\definecolor{currentstroke}{rgb}{0.000000,0.000000,0.000000}%
\pgfsetstrokecolor{currentstroke}%
\pgfsetdash{}{0pt}%
\pgfpathmoveto{\pgfqpoint{7.998823in}{2.056975in}}%
\pgfpathlineto{\pgfqpoint{8.185049in}{2.056975in}}%
\pgfpathlineto{\pgfqpoint{8.185049in}{2.138704in}}%
\pgfpathlineto{\pgfqpoint{7.998823in}{2.138704in}}%
\pgfpathlineto{\pgfqpoint{7.998823in}{2.056975in}}%
\pgfusepath{}%
\end{pgfscope}%
\begin{pgfscope}%
\pgfpathrectangle{\pgfqpoint{0.549740in}{0.463273in}}{\pgfqpoint{9.320225in}{4.495057in}}%
\pgfusepath{clip}%
\pgfsetbuttcap%
\pgfsetroundjoin%
\pgfsetlinewidth{0.000000pt}%
\definecolor{currentstroke}{rgb}{0.000000,0.000000,0.000000}%
\pgfsetstrokecolor{currentstroke}%
\pgfsetdash{}{0pt}%
\pgfpathmoveto{\pgfqpoint{8.185049in}{2.056975in}}%
\pgfpathlineto{\pgfqpoint{8.371276in}{2.056975in}}%
\pgfpathlineto{\pgfqpoint{8.371276in}{2.138704in}}%
\pgfpathlineto{\pgfqpoint{8.185049in}{2.138704in}}%
\pgfpathlineto{\pgfqpoint{8.185049in}{2.056975in}}%
\pgfusepath{}%
\end{pgfscope}%
\begin{pgfscope}%
\pgfpathrectangle{\pgfqpoint{0.549740in}{0.463273in}}{\pgfqpoint{9.320225in}{4.495057in}}%
\pgfusepath{clip}%
\pgfsetbuttcap%
\pgfsetroundjoin%
\definecolor{currentfill}{rgb}{0.614330,0.761948,0.940009}%
\pgfsetfillcolor{currentfill}%
\pgfsetlinewidth{0.000000pt}%
\definecolor{currentstroke}{rgb}{0.000000,0.000000,0.000000}%
\pgfsetstrokecolor{currentstroke}%
\pgfsetdash{}{0pt}%
\pgfpathmoveto{\pgfqpoint{8.371276in}{2.056975in}}%
\pgfpathlineto{\pgfqpoint{8.557503in}{2.056975in}}%
\pgfpathlineto{\pgfqpoint{8.557503in}{2.138704in}}%
\pgfpathlineto{\pgfqpoint{8.371276in}{2.138704in}}%
\pgfpathlineto{\pgfqpoint{8.371276in}{2.056975in}}%
\pgfusepath{fill}%
\end{pgfscope}%
\begin{pgfscope}%
\pgfpathrectangle{\pgfqpoint{0.549740in}{0.463273in}}{\pgfqpoint{9.320225in}{4.495057in}}%
\pgfusepath{clip}%
\pgfsetbuttcap%
\pgfsetroundjoin%
\definecolor{currentfill}{rgb}{0.547810,0.736432,0.947518}%
\pgfsetfillcolor{currentfill}%
\pgfsetlinewidth{0.000000pt}%
\definecolor{currentstroke}{rgb}{0.000000,0.000000,0.000000}%
\pgfsetstrokecolor{currentstroke}%
\pgfsetdash{}{0pt}%
\pgfpathmoveto{\pgfqpoint{8.557503in}{2.056975in}}%
\pgfpathlineto{\pgfqpoint{8.743729in}{2.056975in}}%
\pgfpathlineto{\pgfqpoint{8.743729in}{2.138704in}}%
\pgfpathlineto{\pgfqpoint{8.557503in}{2.138704in}}%
\pgfpathlineto{\pgfqpoint{8.557503in}{2.056975in}}%
\pgfusepath{fill}%
\end{pgfscope}%
\begin{pgfscope}%
\pgfpathrectangle{\pgfqpoint{0.549740in}{0.463273in}}{\pgfqpoint{9.320225in}{4.495057in}}%
\pgfusepath{clip}%
\pgfsetbuttcap%
\pgfsetroundjoin%
\pgfsetlinewidth{0.000000pt}%
\definecolor{currentstroke}{rgb}{0.000000,0.000000,0.000000}%
\pgfsetstrokecolor{currentstroke}%
\pgfsetdash{}{0pt}%
\pgfpathmoveto{\pgfqpoint{8.743729in}{2.056975in}}%
\pgfpathlineto{\pgfqpoint{8.929956in}{2.056975in}}%
\pgfpathlineto{\pgfqpoint{8.929956in}{2.138704in}}%
\pgfpathlineto{\pgfqpoint{8.743729in}{2.138704in}}%
\pgfpathlineto{\pgfqpoint{8.743729in}{2.056975in}}%
\pgfusepath{}%
\end{pgfscope}%
\begin{pgfscope}%
\pgfpathrectangle{\pgfqpoint{0.549740in}{0.463273in}}{\pgfqpoint{9.320225in}{4.495057in}}%
\pgfusepath{clip}%
\pgfsetbuttcap%
\pgfsetroundjoin%
\pgfsetlinewidth{0.000000pt}%
\definecolor{currentstroke}{rgb}{0.000000,0.000000,0.000000}%
\pgfsetstrokecolor{currentstroke}%
\pgfsetdash{}{0pt}%
\pgfpathmoveto{\pgfqpoint{8.929956in}{2.056975in}}%
\pgfpathlineto{\pgfqpoint{9.116182in}{2.056975in}}%
\pgfpathlineto{\pgfqpoint{9.116182in}{2.138704in}}%
\pgfpathlineto{\pgfqpoint{8.929956in}{2.138704in}}%
\pgfpathlineto{\pgfqpoint{8.929956in}{2.056975in}}%
\pgfusepath{}%
\end{pgfscope}%
\begin{pgfscope}%
\pgfpathrectangle{\pgfqpoint{0.549740in}{0.463273in}}{\pgfqpoint{9.320225in}{4.495057in}}%
\pgfusepath{clip}%
\pgfsetbuttcap%
\pgfsetroundjoin%
\pgfsetlinewidth{0.000000pt}%
\definecolor{currentstroke}{rgb}{0.000000,0.000000,0.000000}%
\pgfsetstrokecolor{currentstroke}%
\pgfsetdash{}{0pt}%
\pgfpathmoveto{\pgfqpoint{9.116182in}{2.056975in}}%
\pgfpathlineto{\pgfqpoint{9.302409in}{2.056975in}}%
\pgfpathlineto{\pgfqpoint{9.302409in}{2.138704in}}%
\pgfpathlineto{\pgfqpoint{9.116182in}{2.138704in}}%
\pgfpathlineto{\pgfqpoint{9.116182in}{2.056975in}}%
\pgfusepath{}%
\end{pgfscope}%
\begin{pgfscope}%
\pgfpathrectangle{\pgfqpoint{0.549740in}{0.463273in}}{\pgfqpoint{9.320225in}{4.495057in}}%
\pgfusepath{clip}%
\pgfsetbuttcap%
\pgfsetroundjoin%
\pgfsetlinewidth{0.000000pt}%
\definecolor{currentstroke}{rgb}{0.000000,0.000000,0.000000}%
\pgfsetstrokecolor{currentstroke}%
\pgfsetdash{}{0pt}%
\pgfpathmoveto{\pgfqpoint{9.302409in}{2.056975in}}%
\pgfpathlineto{\pgfqpoint{9.488635in}{2.056975in}}%
\pgfpathlineto{\pgfqpoint{9.488635in}{2.138704in}}%
\pgfpathlineto{\pgfqpoint{9.302409in}{2.138704in}}%
\pgfpathlineto{\pgfqpoint{9.302409in}{2.056975in}}%
\pgfusepath{}%
\end{pgfscope}%
\begin{pgfscope}%
\pgfpathrectangle{\pgfqpoint{0.549740in}{0.463273in}}{\pgfqpoint{9.320225in}{4.495057in}}%
\pgfusepath{clip}%
\pgfsetbuttcap%
\pgfsetroundjoin%
\pgfsetlinewidth{0.000000pt}%
\definecolor{currentstroke}{rgb}{0.000000,0.000000,0.000000}%
\pgfsetstrokecolor{currentstroke}%
\pgfsetdash{}{0pt}%
\pgfpathmoveto{\pgfqpoint{9.488635in}{2.056975in}}%
\pgfpathlineto{\pgfqpoint{9.674862in}{2.056975in}}%
\pgfpathlineto{\pgfqpoint{9.674862in}{2.138704in}}%
\pgfpathlineto{\pgfqpoint{9.488635in}{2.138704in}}%
\pgfpathlineto{\pgfqpoint{9.488635in}{2.056975in}}%
\pgfusepath{}%
\end{pgfscope}%
\begin{pgfscope}%
\pgfpathrectangle{\pgfqpoint{0.549740in}{0.463273in}}{\pgfqpoint{9.320225in}{4.495057in}}%
\pgfusepath{clip}%
\pgfsetbuttcap%
\pgfsetroundjoin%
\pgfsetlinewidth{0.000000pt}%
\definecolor{currentstroke}{rgb}{0.000000,0.000000,0.000000}%
\pgfsetstrokecolor{currentstroke}%
\pgfsetdash{}{0pt}%
\pgfpathmoveto{\pgfqpoint{9.674862in}{2.056975in}}%
\pgfpathlineto{\pgfqpoint{9.861088in}{2.056975in}}%
\pgfpathlineto{\pgfqpoint{9.861088in}{2.138704in}}%
\pgfpathlineto{\pgfqpoint{9.674862in}{2.138704in}}%
\pgfpathlineto{\pgfqpoint{9.674862in}{2.056975in}}%
\pgfusepath{}%
\end{pgfscope}%
\begin{pgfscope}%
\pgfpathrectangle{\pgfqpoint{0.549740in}{0.463273in}}{\pgfqpoint{9.320225in}{4.495057in}}%
\pgfusepath{clip}%
\pgfsetbuttcap%
\pgfsetroundjoin%
\pgfsetlinewidth{0.000000pt}%
\definecolor{currentstroke}{rgb}{0.000000,0.000000,0.000000}%
\pgfsetstrokecolor{currentstroke}%
\pgfsetdash{}{0pt}%
\pgfpathmoveto{\pgfqpoint{0.549761in}{2.138704in}}%
\pgfpathlineto{\pgfqpoint{0.735988in}{2.138704in}}%
\pgfpathlineto{\pgfqpoint{0.735988in}{2.220432in}}%
\pgfpathlineto{\pgfqpoint{0.549761in}{2.220432in}}%
\pgfpathlineto{\pgfqpoint{0.549761in}{2.138704in}}%
\pgfusepath{}%
\end{pgfscope}%
\begin{pgfscope}%
\pgfpathrectangle{\pgfqpoint{0.549740in}{0.463273in}}{\pgfqpoint{9.320225in}{4.495057in}}%
\pgfusepath{clip}%
\pgfsetbuttcap%
\pgfsetroundjoin%
\pgfsetlinewidth{0.000000pt}%
\definecolor{currentstroke}{rgb}{0.000000,0.000000,0.000000}%
\pgfsetstrokecolor{currentstroke}%
\pgfsetdash{}{0pt}%
\pgfpathmoveto{\pgfqpoint{0.735988in}{2.138704in}}%
\pgfpathlineto{\pgfqpoint{0.922214in}{2.138704in}}%
\pgfpathlineto{\pgfqpoint{0.922214in}{2.220432in}}%
\pgfpathlineto{\pgfqpoint{0.735988in}{2.220432in}}%
\pgfpathlineto{\pgfqpoint{0.735988in}{2.138704in}}%
\pgfusepath{}%
\end{pgfscope}%
\begin{pgfscope}%
\pgfpathrectangle{\pgfqpoint{0.549740in}{0.463273in}}{\pgfqpoint{9.320225in}{4.495057in}}%
\pgfusepath{clip}%
\pgfsetbuttcap%
\pgfsetroundjoin%
\pgfsetlinewidth{0.000000pt}%
\definecolor{currentstroke}{rgb}{0.000000,0.000000,0.000000}%
\pgfsetstrokecolor{currentstroke}%
\pgfsetdash{}{0pt}%
\pgfpathmoveto{\pgfqpoint{0.922214in}{2.138704in}}%
\pgfpathlineto{\pgfqpoint{1.108441in}{2.138704in}}%
\pgfpathlineto{\pgfqpoint{1.108441in}{2.220432in}}%
\pgfpathlineto{\pgfqpoint{0.922214in}{2.220432in}}%
\pgfpathlineto{\pgfqpoint{0.922214in}{2.138704in}}%
\pgfusepath{}%
\end{pgfscope}%
\begin{pgfscope}%
\pgfpathrectangle{\pgfqpoint{0.549740in}{0.463273in}}{\pgfqpoint{9.320225in}{4.495057in}}%
\pgfusepath{clip}%
\pgfsetbuttcap%
\pgfsetroundjoin%
\pgfsetlinewidth{0.000000pt}%
\definecolor{currentstroke}{rgb}{0.000000,0.000000,0.000000}%
\pgfsetstrokecolor{currentstroke}%
\pgfsetdash{}{0pt}%
\pgfpathmoveto{\pgfqpoint{1.108441in}{2.138704in}}%
\pgfpathlineto{\pgfqpoint{1.294667in}{2.138704in}}%
\pgfpathlineto{\pgfqpoint{1.294667in}{2.220432in}}%
\pgfpathlineto{\pgfqpoint{1.108441in}{2.220432in}}%
\pgfpathlineto{\pgfqpoint{1.108441in}{2.138704in}}%
\pgfusepath{}%
\end{pgfscope}%
\begin{pgfscope}%
\pgfpathrectangle{\pgfqpoint{0.549740in}{0.463273in}}{\pgfqpoint{9.320225in}{4.495057in}}%
\pgfusepath{clip}%
\pgfsetbuttcap%
\pgfsetroundjoin%
\pgfsetlinewidth{0.000000pt}%
\definecolor{currentstroke}{rgb}{0.000000,0.000000,0.000000}%
\pgfsetstrokecolor{currentstroke}%
\pgfsetdash{}{0pt}%
\pgfpathmoveto{\pgfqpoint{1.294667in}{2.138704in}}%
\pgfpathlineto{\pgfqpoint{1.480894in}{2.138704in}}%
\pgfpathlineto{\pgfqpoint{1.480894in}{2.220432in}}%
\pgfpathlineto{\pgfqpoint{1.294667in}{2.220432in}}%
\pgfpathlineto{\pgfqpoint{1.294667in}{2.138704in}}%
\pgfusepath{}%
\end{pgfscope}%
\begin{pgfscope}%
\pgfpathrectangle{\pgfqpoint{0.549740in}{0.463273in}}{\pgfqpoint{9.320225in}{4.495057in}}%
\pgfusepath{clip}%
\pgfsetbuttcap%
\pgfsetroundjoin%
\pgfsetlinewidth{0.000000pt}%
\definecolor{currentstroke}{rgb}{0.000000,0.000000,0.000000}%
\pgfsetstrokecolor{currentstroke}%
\pgfsetdash{}{0pt}%
\pgfpathmoveto{\pgfqpoint{1.480894in}{2.138704in}}%
\pgfpathlineto{\pgfqpoint{1.667120in}{2.138704in}}%
\pgfpathlineto{\pgfqpoint{1.667120in}{2.220432in}}%
\pgfpathlineto{\pgfqpoint{1.480894in}{2.220432in}}%
\pgfpathlineto{\pgfqpoint{1.480894in}{2.138704in}}%
\pgfusepath{}%
\end{pgfscope}%
\begin{pgfscope}%
\pgfpathrectangle{\pgfqpoint{0.549740in}{0.463273in}}{\pgfqpoint{9.320225in}{4.495057in}}%
\pgfusepath{clip}%
\pgfsetbuttcap%
\pgfsetroundjoin%
\pgfsetlinewidth{0.000000pt}%
\definecolor{currentstroke}{rgb}{0.000000,0.000000,0.000000}%
\pgfsetstrokecolor{currentstroke}%
\pgfsetdash{}{0pt}%
\pgfpathmoveto{\pgfqpoint{1.667120in}{2.138704in}}%
\pgfpathlineto{\pgfqpoint{1.853347in}{2.138704in}}%
\pgfpathlineto{\pgfqpoint{1.853347in}{2.220432in}}%
\pgfpathlineto{\pgfqpoint{1.667120in}{2.220432in}}%
\pgfpathlineto{\pgfqpoint{1.667120in}{2.138704in}}%
\pgfusepath{}%
\end{pgfscope}%
\begin{pgfscope}%
\pgfpathrectangle{\pgfqpoint{0.549740in}{0.463273in}}{\pgfqpoint{9.320225in}{4.495057in}}%
\pgfusepath{clip}%
\pgfsetbuttcap%
\pgfsetroundjoin%
\pgfsetlinewidth{0.000000pt}%
\definecolor{currentstroke}{rgb}{0.000000,0.000000,0.000000}%
\pgfsetstrokecolor{currentstroke}%
\pgfsetdash{}{0pt}%
\pgfpathmoveto{\pgfqpoint{1.853347in}{2.138704in}}%
\pgfpathlineto{\pgfqpoint{2.039573in}{2.138704in}}%
\pgfpathlineto{\pgfqpoint{2.039573in}{2.220432in}}%
\pgfpathlineto{\pgfqpoint{1.853347in}{2.220432in}}%
\pgfpathlineto{\pgfqpoint{1.853347in}{2.138704in}}%
\pgfusepath{}%
\end{pgfscope}%
\begin{pgfscope}%
\pgfpathrectangle{\pgfqpoint{0.549740in}{0.463273in}}{\pgfqpoint{9.320225in}{4.495057in}}%
\pgfusepath{clip}%
\pgfsetbuttcap%
\pgfsetroundjoin%
\pgfsetlinewidth{0.000000pt}%
\definecolor{currentstroke}{rgb}{0.000000,0.000000,0.000000}%
\pgfsetstrokecolor{currentstroke}%
\pgfsetdash{}{0pt}%
\pgfpathmoveto{\pgfqpoint{2.039573in}{2.138704in}}%
\pgfpathlineto{\pgfqpoint{2.225800in}{2.138704in}}%
\pgfpathlineto{\pgfqpoint{2.225800in}{2.220432in}}%
\pgfpathlineto{\pgfqpoint{2.039573in}{2.220432in}}%
\pgfpathlineto{\pgfqpoint{2.039573in}{2.138704in}}%
\pgfusepath{}%
\end{pgfscope}%
\begin{pgfscope}%
\pgfpathrectangle{\pgfqpoint{0.549740in}{0.463273in}}{\pgfqpoint{9.320225in}{4.495057in}}%
\pgfusepath{clip}%
\pgfsetbuttcap%
\pgfsetroundjoin%
\pgfsetlinewidth{0.000000pt}%
\definecolor{currentstroke}{rgb}{0.000000,0.000000,0.000000}%
\pgfsetstrokecolor{currentstroke}%
\pgfsetdash{}{0pt}%
\pgfpathmoveto{\pgfqpoint{2.225800in}{2.138704in}}%
\pgfpathlineto{\pgfqpoint{2.412027in}{2.138704in}}%
\pgfpathlineto{\pgfqpoint{2.412027in}{2.220432in}}%
\pgfpathlineto{\pgfqpoint{2.225800in}{2.220432in}}%
\pgfpathlineto{\pgfqpoint{2.225800in}{2.138704in}}%
\pgfusepath{}%
\end{pgfscope}%
\begin{pgfscope}%
\pgfpathrectangle{\pgfqpoint{0.549740in}{0.463273in}}{\pgfqpoint{9.320225in}{4.495057in}}%
\pgfusepath{clip}%
\pgfsetbuttcap%
\pgfsetroundjoin%
\pgfsetlinewidth{0.000000pt}%
\definecolor{currentstroke}{rgb}{0.000000,0.000000,0.000000}%
\pgfsetstrokecolor{currentstroke}%
\pgfsetdash{}{0pt}%
\pgfpathmoveto{\pgfqpoint{2.412027in}{2.138704in}}%
\pgfpathlineto{\pgfqpoint{2.598253in}{2.138704in}}%
\pgfpathlineto{\pgfqpoint{2.598253in}{2.220432in}}%
\pgfpathlineto{\pgfqpoint{2.412027in}{2.220432in}}%
\pgfpathlineto{\pgfqpoint{2.412027in}{2.138704in}}%
\pgfusepath{}%
\end{pgfscope}%
\begin{pgfscope}%
\pgfpathrectangle{\pgfqpoint{0.549740in}{0.463273in}}{\pgfqpoint{9.320225in}{4.495057in}}%
\pgfusepath{clip}%
\pgfsetbuttcap%
\pgfsetroundjoin%
\pgfsetlinewidth{0.000000pt}%
\definecolor{currentstroke}{rgb}{0.000000,0.000000,0.000000}%
\pgfsetstrokecolor{currentstroke}%
\pgfsetdash{}{0pt}%
\pgfpathmoveto{\pgfqpoint{2.598253in}{2.138704in}}%
\pgfpathlineto{\pgfqpoint{2.784480in}{2.138704in}}%
\pgfpathlineto{\pgfqpoint{2.784480in}{2.220432in}}%
\pgfpathlineto{\pgfqpoint{2.598253in}{2.220432in}}%
\pgfpathlineto{\pgfqpoint{2.598253in}{2.138704in}}%
\pgfusepath{}%
\end{pgfscope}%
\begin{pgfscope}%
\pgfpathrectangle{\pgfqpoint{0.549740in}{0.463273in}}{\pgfqpoint{9.320225in}{4.495057in}}%
\pgfusepath{clip}%
\pgfsetbuttcap%
\pgfsetroundjoin%
\pgfsetlinewidth{0.000000pt}%
\definecolor{currentstroke}{rgb}{0.000000,0.000000,0.000000}%
\pgfsetstrokecolor{currentstroke}%
\pgfsetdash{}{0pt}%
\pgfpathmoveto{\pgfqpoint{2.784480in}{2.138704in}}%
\pgfpathlineto{\pgfqpoint{2.970706in}{2.138704in}}%
\pgfpathlineto{\pgfqpoint{2.970706in}{2.220432in}}%
\pgfpathlineto{\pgfqpoint{2.784480in}{2.220432in}}%
\pgfpathlineto{\pgfqpoint{2.784480in}{2.138704in}}%
\pgfusepath{}%
\end{pgfscope}%
\begin{pgfscope}%
\pgfpathrectangle{\pgfqpoint{0.549740in}{0.463273in}}{\pgfqpoint{9.320225in}{4.495057in}}%
\pgfusepath{clip}%
\pgfsetbuttcap%
\pgfsetroundjoin%
\pgfsetlinewidth{0.000000pt}%
\definecolor{currentstroke}{rgb}{0.000000,0.000000,0.000000}%
\pgfsetstrokecolor{currentstroke}%
\pgfsetdash{}{0pt}%
\pgfpathmoveto{\pgfqpoint{2.970706in}{2.138704in}}%
\pgfpathlineto{\pgfqpoint{3.156933in}{2.138704in}}%
\pgfpathlineto{\pgfqpoint{3.156933in}{2.220432in}}%
\pgfpathlineto{\pgfqpoint{2.970706in}{2.220432in}}%
\pgfpathlineto{\pgfqpoint{2.970706in}{2.138704in}}%
\pgfusepath{}%
\end{pgfscope}%
\begin{pgfscope}%
\pgfpathrectangle{\pgfqpoint{0.549740in}{0.463273in}}{\pgfqpoint{9.320225in}{4.495057in}}%
\pgfusepath{clip}%
\pgfsetbuttcap%
\pgfsetroundjoin%
\pgfsetlinewidth{0.000000pt}%
\definecolor{currentstroke}{rgb}{0.000000,0.000000,0.000000}%
\pgfsetstrokecolor{currentstroke}%
\pgfsetdash{}{0pt}%
\pgfpathmoveto{\pgfqpoint{3.156933in}{2.138704in}}%
\pgfpathlineto{\pgfqpoint{3.343159in}{2.138704in}}%
\pgfpathlineto{\pgfqpoint{3.343159in}{2.220432in}}%
\pgfpathlineto{\pgfqpoint{3.156933in}{2.220432in}}%
\pgfpathlineto{\pgfqpoint{3.156933in}{2.138704in}}%
\pgfusepath{}%
\end{pgfscope}%
\begin{pgfscope}%
\pgfpathrectangle{\pgfqpoint{0.549740in}{0.463273in}}{\pgfqpoint{9.320225in}{4.495057in}}%
\pgfusepath{clip}%
\pgfsetbuttcap%
\pgfsetroundjoin%
\pgfsetlinewidth{0.000000pt}%
\definecolor{currentstroke}{rgb}{0.000000,0.000000,0.000000}%
\pgfsetstrokecolor{currentstroke}%
\pgfsetdash{}{0pt}%
\pgfpathmoveto{\pgfqpoint{3.343159in}{2.138704in}}%
\pgfpathlineto{\pgfqpoint{3.529386in}{2.138704in}}%
\pgfpathlineto{\pgfqpoint{3.529386in}{2.220432in}}%
\pgfpathlineto{\pgfqpoint{3.343159in}{2.220432in}}%
\pgfpathlineto{\pgfqpoint{3.343159in}{2.138704in}}%
\pgfusepath{}%
\end{pgfscope}%
\begin{pgfscope}%
\pgfpathrectangle{\pgfqpoint{0.549740in}{0.463273in}}{\pgfqpoint{9.320225in}{4.495057in}}%
\pgfusepath{clip}%
\pgfsetbuttcap%
\pgfsetroundjoin%
\pgfsetlinewidth{0.000000pt}%
\definecolor{currentstroke}{rgb}{0.000000,0.000000,0.000000}%
\pgfsetstrokecolor{currentstroke}%
\pgfsetdash{}{0pt}%
\pgfpathmoveto{\pgfqpoint{3.529386in}{2.138704in}}%
\pgfpathlineto{\pgfqpoint{3.715612in}{2.138704in}}%
\pgfpathlineto{\pgfqpoint{3.715612in}{2.220432in}}%
\pgfpathlineto{\pgfqpoint{3.529386in}{2.220432in}}%
\pgfpathlineto{\pgfqpoint{3.529386in}{2.138704in}}%
\pgfusepath{}%
\end{pgfscope}%
\begin{pgfscope}%
\pgfpathrectangle{\pgfqpoint{0.549740in}{0.463273in}}{\pgfqpoint{9.320225in}{4.495057in}}%
\pgfusepath{clip}%
\pgfsetbuttcap%
\pgfsetroundjoin%
\pgfsetlinewidth{0.000000pt}%
\definecolor{currentstroke}{rgb}{0.000000,0.000000,0.000000}%
\pgfsetstrokecolor{currentstroke}%
\pgfsetdash{}{0pt}%
\pgfpathmoveto{\pgfqpoint{3.715612in}{2.138704in}}%
\pgfpathlineto{\pgfqpoint{3.901839in}{2.138704in}}%
\pgfpathlineto{\pgfqpoint{3.901839in}{2.220432in}}%
\pgfpathlineto{\pgfqpoint{3.715612in}{2.220432in}}%
\pgfpathlineto{\pgfqpoint{3.715612in}{2.138704in}}%
\pgfusepath{}%
\end{pgfscope}%
\begin{pgfscope}%
\pgfpathrectangle{\pgfqpoint{0.549740in}{0.463273in}}{\pgfqpoint{9.320225in}{4.495057in}}%
\pgfusepath{clip}%
\pgfsetbuttcap%
\pgfsetroundjoin%
\pgfsetlinewidth{0.000000pt}%
\definecolor{currentstroke}{rgb}{0.000000,0.000000,0.000000}%
\pgfsetstrokecolor{currentstroke}%
\pgfsetdash{}{0pt}%
\pgfpathmoveto{\pgfqpoint{3.901839in}{2.138704in}}%
\pgfpathlineto{\pgfqpoint{4.088065in}{2.138704in}}%
\pgfpathlineto{\pgfqpoint{4.088065in}{2.220432in}}%
\pgfpathlineto{\pgfqpoint{3.901839in}{2.220432in}}%
\pgfpathlineto{\pgfqpoint{3.901839in}{2.138704in}}%
\pgfusepath{}%
\end{pgfscope}%
\begin{pgfscope}%
\pgfpathrectangle{\pgfqpoint{0.549740in}{0.463273in}}{\pgfqpoint{9.320225in}{4.495057in}}%
\pgfusepath{clip}%
\pgfsetbuttcap%
\pgfsetroundjoin%
\pgfsetlinewidth{0.000000pt}%
\definecolor{currentstroke}{rgb}{0.000000,0.000000,0.000000}%
\pgfsetstrokecolor{currentstroke}%
\pgfsetdash{}{0pt}%
\pgfpathmoveto{\pgfqpoint{4.088065in}{2.138704in}}%
\pgfpathlineto{\pgfqpoint{4.274292in}{2.138704in}}%
\pgfpathlineto{\pgfqpoint{4.274292in}{2.220432in}}%
\pgfpathlineto{\pgfqpoint{4.088065in}{2.220432in}}%
\pgfpathlineto{\pgfqpoint{4.088065in}{2.138704in}}%
\pgfusepath{}%
\end{pgfscope}%
\begin{pgfscope}%
\pgfpathrectangle{\pgfqpoint{0.549740in}{0.463273in}}{\pgfqpoint{9.320225in}{4.495057in}}%
\pgfusepath{clip}%
\pgfsetbuttcap%
\pgfsetroundjoin%
\pgfsetlinewidth{0.000000pt}%
\definecolor{currentstroke}{rgb}{0.000000,0.000000,0.000000}%
\pgfsetstrokecolor{currentstroke}%
\pgfsetdash{}{0pt}%
\pgfpathmoveto{\pgfqpoint{4.274292in}{2.138704in}}%
\pgfpathlineto{\pgfqpoint{4.460519in}{2.138704in}}%
\pgfpathlineto{\pgfqpoint{4.460519in}{2.220432in}}%
\pgfpathlineto{\pgfqpoint{4.274292in}{2.220432in}}%
\pgfpathlineto{\pgfqpoint{4.274292in}{2.138704in}}%
\pgfusepath{}%
\end{pgfscope}%
\begin{pgfscope}%
\pgfpathrectangle{\pgfqpoint{0.549740in}{0.463273in}}{\pgfqpoint{9.320225in}{4.495057in}}%
\pgfusepath{clip}%
\pgfsetbuttcap%
\pgfsetroundjoin%
\pgfsetlinewidth{0.000000pt}%
\definecolor{currentstroke}{rgb}{0.000000,0.000000,0.000000}%
\pgfsetstrokecolor{currentstroke}%
\pgfsetdash{}{0pt}%
\pgfpathmoveto{\pgfqpoint{4.460519in}{2.138704in}}%
\pgfpathlineto{\pgfqpoint{4.646745in}{2.138704in}}%
\pgfpathlineto{\pgfqpoint{4.646745in}{2.220432in}}%
\pgfpathlineto{\pgfqpoint{4.460519in}{2.220432in}}%
\pgfpathlineto{\pgfqpoint{4.460519in}{2.138704in}}%
\pgfusepath{}%
\end{pgfscope}%
\begin{pgfscope}%
\pgfpathrectangle{\pgfqpoint{0.549740in}{0.463273in}}{\pgfqpoint{9.320225in}{4.495057in}}%
\pgfusepath{clip}%
\pgfsetbuttcap%
\pgfsetroundjoin%
\pgfsetlinewidth{0.000000pt}%
\definecolor{currentstroke}{rgb}{0.000000,0.000000,0.000000}%
\pgfsetstrokecolor{currentstroke}%
\pgfsetdash{}{0pt}%
\pgfpathmoveto{\pgfqpoint{4.646745in}{2.138704in}}%
\pgfpathlineto{\pgfqpoint{4.832972in}{2.138704in}}%
\pgfpathlineto{\pgfqpoint{4.832972in}{2.220432in}}%
\pgfpathlineto{\pgfqpoint{4.646745in}{2.220432in}}%
\pgfpathlineto{\pgfqpoint{4.646745in}{2.138704in}}%
\pgfusepath{}%
\end{pgfscope}%
\begin{pgfscope}%
\pgfpathrectangle{\pgfqpoint{0.549740in}{0.463273in}}{\pgfqpoint{9.320225in}{4.495057in}}%
\pgfusepath{clip}%
\pgfsetbuttcap%
\pgfsetroundjoin%
\pgfsetlinewidth{0.000000pt}%
\definecolor{currentstroke}{rgb}{0.000000,0.000000,0.000000}%
\pgfsetstrokecolor{currentstroke}%
\pgfsetdash{}{0pt}%
\pgfpathmoveto{\pgfqpoint{4.832972in}{2.138704in}}%
\pgfpathlineto{\pgfqpoint{5.019198in}{2.138704in}}%
\pgfpathlineto{\pgfqpoint{5.019198in}{2.220432in}}%
\pgfpathlineto{\pgfqpoint{4.832972in}{2.220432in}}%
\pgfpathlineto{\pgfqpoint{4.832972in}{2.138704in}}%
\pgfusepath{}%
\end{pgfscope}%
\begin{pgfscope}%
\pgfpathrectangle{\pgfqpoint{0.549740in}{0.463273in}}{\pgfqpoint{9.320225in}{4.495057in}}%
\pgfusepath{clip}%
\pgfsetbuttcap%
\pgfsetroundjoin%
\pgfsetlinewidth{0.000000pt}%
\definecolor{currentstroke}{rgb}{0.000000,0.000000,0.000000}%
\pgfsetstrokecolor{currentstroke}%
\pgfsetdash{}{0pt}%
\pgfpathmoveto{\pgfqpoint{5.019198in}{2.138704in}}%
\pgfpathlineto{\pgfqpoint{5.205425in}{2.138704in}}%
\pgfpathlineto{\pgfqpoint{5.205425in}{2.220432in}}%
\pgfpathlineto{\pgfqpoint{5.019198in}{2.220432in}}%
\pgfpathlineto{\pgfqpoint{5.019198in}{2.138704in}}%
\pgfusepath{}%
\end{pgfscope}%
\begin{pgfscope}%
\pgfpathrectangle{\pgfqpoint{0.549740in}{0.463273in}}{\pgfqpoint{9.320225in}{4.495057in}}%
\pgfusepath{clip}%
\pgfsetbuttcap%
\pgfsetroundjoin%
\pgfsetlinewidth{0.000000pt}%
\definecolor{currentstroke}{rgb}{0.000000,0.000000,0.000000}%
\pgfsetstrokecolor{currentstroke}%
\pgfsetdash{}{0pt}%
\pgfpathmoveto{\pgfqpoint{5.205425in}{2.138704in}}%
\pgfpathlineto{\pgfqpoint{5.391651in}{2.138704in}}%
\pgfpathlineto{\pgfqpoint{5.391651in}{2.220432in}}%
\pgfpathlineto{\pgfqpoint{5.205425in}{2.220432in}}%
\pgfpathlineto{\pgfqpoint{5.205425in}{2.138704in}}%
\pgfusepath{}%
\end{pgfscope}%
\begin{pgfscope}%
\pgfpathrectangle{\pgfqpoint{0.549740in}{0.463273in}}{\pgfqpoint{9.320225in}{4.495057in}}%
\pgfusepath{clip}%
\pgfsetbuttcap%
\pgfsetroundjoin%
\definecolor{currentfill}{rgb}{0.472869,0.711325,0.955316}%
\pgfsetfillcolor{currentfill}%
\pgfsetlinewidth{0.000000pt}%
\definecolor{currentstroke}{rgb}{0.000000,0.000000,0.000000}%
\pgfsetstrokecolor{currentstroke}%
\pgfsetdash{}{0pt}%
\pgfpathmoveto{\pgfqpoint{5.391651in}{2.138704in}}%
\pgfpathlineto{\pgfqpoint{5.577878in}{2.138704in}}%
\pgfpathlineto{\pgfqpoint{5.577878in}{2.220432in}}%
\pgfpathlineto{\pgfqpoint{5.391651in}{2.220432in}}%
\pgfpathlineto{\pgfqpoint{5.391651in}{2.138704in}}%
\pgfusepath{fill}%
\end{pgfscope}%
\begin{pgfscope}%
\pgfpathrectangle{\pgfqpoint{0.549740in}{0.463273in}}{\pgfqpoint{9.320225in}{4.495057in}}%
\pgfusepath{clip}%
\pgfsetbuttcap%
\pgfsetroundjoin%
\pgfsetlinewidth{0.000000pt}%
\definecolor{currentstroke}{rgb}{0.000000,0.000000,0.000000}%
\pgfsetstrokecolor{currentstroke}%
\pgfsetdash{}{0pt}%
\pgfpathmoveto{\pgfqpoint{5.577878in}{2.138704in}}%
\pgfpathlineto{\pgfqpoint{5.764104in}{2.138704in}}%
\pgfpathlineto{\pgfqpoint{5.764104in}{2.220432in}}%
\pgfpathlineto{\pgfqpoint{5.577878in}{2.220432in}}%
\pgfpathlineto{\pgfqpoint{5.577878in}{2.138704in}}%
\pgfusepath{}%
\end{pgfscope}%
\begin{pgfscope}%
\pgfpathrectangle{\pgfqpoint{0.549740in}{0.463273in}}{\pgfqpoint{9.320225in}{4.495057in}}%
\pgfusepath{clip}%
\pgfsetbuttcap%
\pgfsetroundjoin%
\pgfsetlinewidth{0.000000pt}%
\definecolor{currentstroke}{rgb}{0.000000,0.000000,0.000000}%
\pgfsetstrokecolor{currentstroke}%
\pgfsetdash{}{0pt}%
\pgfpathmoveto{\pgfqpoint{5.764104in}{2.138704in}}%
\pgfpathlineto{\pgfqpoint{5.950331in}{2.138704in}}%
\pgfpathlineto{\pgfqpoint{5.950331in}{2.220432in}}%
\pgfpathlineto{\pgfqpoint{5.764104in}{2.220432in}}%
\pgfpathlineto{\pgfqpoint{5.764104in}{2.138704in}}%
\pgfusepath{}%
\end{pgfscope}%
\begin{pgfscope}%
\pgfpathrectangle{\pgfqpoint{0.549740in}{0.463273in}}{\pgfqpoint{9.320225in}{4.495057in}}%
\pgfusepath{clip}%
\pgfsetbuttcap%
\pgfsetroundjoin%
\pgfsetlinewidth{0.000000pt}%
\definecolor{currentstroke}{rgb}{0.000000,0.000000,0.000000}%
\pgfsetstrokecolor{currentstroke}%
\pgfsetdash{}{0pt}%
\pgfpathmoveto{\pgfqpoint{5.950331in}{2.138704in}}%
\pgfpathlineto{\pgfqpoint{6.136557in}{2.138704in}}%
\pgfpathlineto{\pgfqpoint{6.136557in}{2.220432in}}%
\pgfpathlineto{\pgfqpoint{5.950331in}{2.220432in}}%
\pgfpathlineto{\pgfqpoint{5.950331in}{2.138704in}}%
\pgfusepath{}%
\end{pgfscope}%
\begin{pgfscope}%
\pgfpathrectangle{\pgfqpoint{0.549740in}{0.463273in}}{\pgfqpoint{9.320225in}{4.495057in}}%
\pgfusepath{clip}%
\pgfsetbuttcap%
\pgfsetroundjoin%
\pgfsetlinewidth{0.000000pt}%
\definecolor{currentstroke}{rgb}{0.000000,0.000000,0.000000}%
\pgfsetstrokecolor{currentstroke}%
\pgfsetdash{}{0pt}%
\pgfpathmoveto{\pgfqpoint{6.136557in}{2.138704in}}%
\pgfpathlineto{\pgfqpoint{6.322784in}{2.138704in}}%
\pgfpathlineto{\pgfqpoint{6.322784in}{2.220432in}}%
\pgfpathlineto{\pgfqpoint{6.136557in}{2.220432in}}%
\pgfpathlineto{\pgfqpoint{6.136557in}{2.138704in}}%
\pgfusepath{}%
\end{pgfscope}%
\begin{pgfscope}%
\pgfpathrectangle{\pgfqpoint{0.549740in}{0.463273in}}{\pgfqpoint{9.320225in}{4.495057in}}%
\pgfusepath{clip}%
\pgfsetbuttcap%
\pgfsetroundjoin%
\definecolor{currentfill}{rgb}{0.472869,0.711325,0.955316}%
\pgfsetfillcolor{currentfill}%
\pgfsetlinewidth{0.000000pt}%
\definecolor{currentstroke}{rgb}{0.000000,0.000000,0.000000}%
\pgfsetstrokecolor{currentstroke}%
\pgfsetdash{}{0pt}%
\pgfpathmoveto{\pgfqpoint{6.322784in}{2.138704in}}%
\pgfpathlineto{\pgfqpoint{6.509011in}{2.138704in}}%
\pgfpathlineto{\pgfqpoint{6.509011in}{2.220432in}}%
\pgfpathlineto{\pgfqpoint{6.322784in}{2.220432in}}%
\pgfpathlineto{\pgfqpoint{6.322784in}{2.138704in}}%
\pgfusepath{fill}%
\end{pgfscope}%
\begin{pgfscope}%
\pgfpathrectangle{\pgfqpoint{0.549740in}{0.463273in}}{\pgfqpoint{9.320225in}{4.495057in}}%
\pgfusepath{clip}%
\pgfsetbuttcap%
\pgfsetroundjoin%
\pgfsetlinewidth{0.000000pt}%
\definecolor{currentstroke}{rgb}{0.000000,0.000000,0.000000}%
\pgfsetstrokecolor{currentstroke}%
\pgfsetdash{}{0pt}%
\pgfpathmoveto{\pgfqpoint{6.509011in}{2.138704in}}%
\pgfpathlineto{\pgfqpoint{6.695237in}{2.138704in}}%
\pgfpathlineto{\pgfqpoint{6.695237in}{2.220432in}}%
\pgfpathlineto{\pgfqpoint{6.509011in}{2.220432in}}%
\pgfpathlineto{\pgfqpoint{6.509011in}{2.138704in}}%
\pgfusepath{}%
\end{pgfscope}%
\begin{pgfscope}%
\pgfpathrectangle{\pgfqpoint{0.549740in}{0.463273in}}{\pgfqpoint{9.320225in}{4.495057in}}%
\pgfusepath{clip}%
\pgfsetbuttcap%
\pgfsetroundjoin%
\pgfsetlinewidth{0.000000pt}%
\definecolor{currentstroke}{rgb}{0.000000,0.000000,0.000000}%
\pgfsetstrokecolor{currentstroke}%
\pgfsetdash{}{0pt}%
\pgfpathmoveto{\pgfqpoint{6.695237in}{2.138704in}}%
\pgfpathlineto{\pgfqpoint{6.881464in}{2.138704in}}%
\pgfpathlineto{\pgfqpoint{6.881464in}{2.220432in}}%
\pgfpathlineto{\pgfqpoint{6.695237in}{2.220432in}}%
\pgfpathlineto{\pgfqpoint{6.695237in}{2.138704in}}%
\pgfusepath{}%
\end{pgfscope}%
\begin{pgfscope}%
\pgfpathrectangle{\pgfqpoint{0.549740in}{0.463273in}}{\pgfqpoint{9.320225in}{4.495057in}}%
\pgfusepath{clip}%
\pgfsetbuttcap%
\pgfsetroundjoin%
\pgfsetlinewidth{0.000000pt}%
\definecolor{currentstroke}{rgb}{0.000000,0.000000,0.000000}%
\pgfsetstrokecolor{currentstroke}%
\pgfsetdash{}{0pt}%
\pgfpathmoveto{\pgfqpoint{6.881464in}{2.138704in}}%
\pgfpathlineto{\pgfqpoint{7.067690in}{2.138704in}}%
\pgfpathlineto{\pgfqpoint{7.067690in}{2.220432in}}%
\pgfpathlineto{\pgfqpoint{6.881464in}{2.220432in}}%
\pgfpathlineto{\pgfqpoint{6.881464in}{2.138704in}}%
\pgfusepath{}%
\end{pgfscope}%
\begin{pgfscope}%
\pgfpathrectangle{\pgfqpoint{0.549740in}{0.463273in}}{\pgfqpoint{9.320225in}{4.495057in}}%
\pgfusepath{clip}%
\pgfsetbuttcap%
\pgfsetroundjoin%
\pgfsetlinewidth{0.000000pt}%
\definecolor{currentstroke}{rgb}{0.000000,0.000000,0.000000}%
\pgfsetstrokecolor{currentstroke}%
\pgfsetdash{}{0pt}%
\pgfpathmoveto{\pgfqpoint{7.067690in}{2.138704in}}%
\pgfpathlineto{\pgfqpoint{7.253917in}{2.138704in}}%
\pgfpathlineto{\pgfqpoint{7.253917in}{2.220432in}}%
\pgfpathlineto{\pgfqpoint{7.067690in}{2.220432in}}%
\pgfpathlineto{\pgfqpoint{7.067690in}{2.138704in}}%
\pgfusepath{}%
\end{pgfscope}%
\begin{pgfscope}%
\pgfpathrectangle{\pgfqpoint{0.549740in}{0.463273in}}{\pgfqpoint{9.320225in}{4.495057in}}%
\pgfusepath{clip}%
\pgfsetbuttcap%
\pgfsetroundjoin%
\pgfsetlinewidth{0.000000pt}%
\definecolor{currentstroke}{rgb}{0.000000,0.000000,0.000000}%
\pgfsetstrokecolor{currentstroke}%
\pgfsetdash{}{0pt}%
\pgfpathmoveto{\pgfqpoint{7.253917in}{2.138704in}}%
\pgfpathlineto{\pgfqpoint{7.440143in}{2.138704in}}%
\pgfpathlineto{\pgfqpoint{7.440143in}{2.220432in}}%
\pgfpathlineto{\pgfqpoint{7.253917in}{2.220432in}}%
\pgfpathlineto{\pgfqpoint{7.253917in}{2.138704in}}%
\pgfusepath{}%
\end{pgfscope}%
\begin{pgfscope}%
\pgfpathrectangle{\pgfqpoint{0.549740in}{0.463273in}}{\pgfqpoint{9.320225in}{4.495057in}}%
\pgfusepath{clip}%
\pgfsetbuttcap%
\pgfsetroundjoin%
\definecolor{currentfill}{rgb}{0.472869,0.711325,0.955316}%
\pgfsetfillcolor{currentfill}%
\pgfsetlinewidth{0.000000pt}%
\definecolor{currentstroke}{rgb}{0.000000,0.000000,0.000000}%
\pgfsetstrokecolor{currentstroke}%
\pgfsetdash{}{0pt}%
\pgfpathmoveto{\pgfqpoint{7.440143in}{2.138704in}}%
\pgfpathlineto{\pgfqpoint{7.626370in}{2.138704in}}%
\pgfpathlineto{\pgfqpoint{7.626370in}{2.220432in}}%
\pgfpathlineto{\pgfqpoint{7.440143in}{2.220432in}}%
\pgfpathlineto{\pgfqpoint{7.440143in}{2.138704in}}%
\pgfusepath{fill}%
\end{pgfscope}%
\begin{pgfscope}%
\pgfpathrectangle{\pgfqpoint{0.549740in}{0.463273in}}{\pgfqpoint{9.320225in}{4.495057in}}%
\pgfusepath{clip}%
\pgfsetbuttcap%
\pgfsetroundjoin%
\pgfsetlinewidth{0.000000pt}%
\definecolor{currentstroke}{rgb}{0.000000,0.000000,0.000000}%
\pgfsetstrokecolor{currentstroke}%
\pgfsetdash{}{0pt}%
\pgfpathmoveto{\pgfqpoint{7.626370in}{2.138704in}}%
\pgfpathlineto{\pgfqpoint{7.812596in}{2.138704in}}%
\pgfpathlineto{\pgfqpoint{7.812596in}{2.220432in}}%
\pgfpathlineto{\pgfqpoint{7.626370in}{2.220432in}}%
\pgfpathlineto{\pgfqpoint{7.626370in}{2.138704in}}%
\pgfusepath{}%
\end{pgfscope}%
\begin{pgfscope}%
\pgfpathrectangle{\pgfqpoint{0.549740in}{0.463273in}}{\pgfqpoint{9.320225in}{4.495057in}}%
\pgfusepath{clip}%
\pgfsetbuttcap%
\pgfsetroundjoin%
\pgfsetlinewidth{0.000000pt}%
\definecolor{currentstroke}{rgb}{0.000000,0.000000,0.000000}%
\pgfsetstrokecolor{currentstroke}%
\pgfsetdash{}{0pt}%
\pgfpathmoveto{\pgfqpoint{7.812596in}{2.138704in}}%
\pgfpathlineto{\pgfqpoint{7.998823in}{2.138704in}}%
\pgfpathlineto{\pgfqpoint{7.998823in}{2.220432in}}%
\pgfpathlineto{\pgfqpoint{7.812596in}{2.220432in}}%
\pgfpathlineto{\pgfqpoint{7.812596in}{2.138704in}}%
\pgfusepath{}%
\end{pgfscope}%
\begin{pgfscope}%
\pgfpathrectangle{\pgfqpoint{0.549740in}{0.463273in}}{\pgfqpoint{9.320225in}{4.495057in}}%
\pgfusepath{clip}%
\pgfsetbuttcap%
\pgfsetroundjoin%
\pgfsetlinewidth{0.000000pt}%
\definecolor{currentstroke}{rgb}{0.000000,0.000000,0.000000}%
\pgfsetstrokecolor{currentstroke}%
\pgfsetdash{}{0pt}%
\pgfpathmoveto{\pgfqpoint{7.998823in}{2.138704in}}%
\pgfpathlineto{\pgfqpoint{8.185049in}{2.138704in}}%
\pgfpathlineto{\pgfqpoint{8.185049in}{2.220432in}}%
\pgfpathlineto{\pgfqpoint{7.998823in}{2.220432in}}%
\pgfpathlineto{\pgfqpoint{7.998823in}{2.138704in}}%
\pgfusepath{}%
\end{pgfscope}%
\begin{pgfscope}%
\pgfpathrectangle{\pgfqpoint{0.549740in}{0.463273in}}{\pgfqpoint{9.320225in}{4.495057in}}%
\pgfusepath{clip}%
\pgfsetbuttcap%
\pgfsetroundjoin%
\pgfsetlinewidth{0.000000pt}%
\definecolor{currentstroke}{rgb}{0.000000,0.000000,0.000000}%
\pgfsetstrokecolor{currentstroke}%
\pgfsetdash{}{0pt}%
\pgfpathmoveto{\pgfqpoint{8.185049in}{2.138704in}}%
\pgfpathlineto{\pgfqpoint{8.371276in}{2.138704in}}%
\pgfpathlineto{\pgfqpoint{8.371276in}{2.220432in}}%
\pgfpathlineto{\pgfqpoint{8.185049in}{2.220432in}}%
\pgfpathlineto{\pgfqpoint{8.185049in}{2.138704in}}%
\pgfusepath{}%
\end{pgfscope}%
\begin{pgfscope}%
\pgfpathrectangle{\pgfqpoint{0.549740in}{0.463273in}}{\pgfqpoint{9.320225in}{4.495057in}}%
\pgfusepath{clip}%
\pgfsetbuttcap%
\pgfsetroundjoin%
\definecolor{currentfill}{rgb}{0.547810,0.736432,0.947518}%
\pgfsetfillcolor{currentfill}%
\pgfsetlinewidth{0.000000pt}%
\definecolor{currentstroke}{rgb}{0.000000,0.000000,0.000000}%
\pgfsetstrokecolor{currentstroke}%
\pgfsetdash{}{0pt}%
\pgfpathmoveto{\pgfqpoint{8.371276in}{2.138704in}}%
\pgfpathlineto{\pgfqpoint{8.557503in}{2.138704in}}%
\pgfpathlineto{\pgfqpoint{8.557503in}{2.220432in}}%
\pgfpathlineto{\pgfqpoint{8.371276in}{2.220432in}}%
\pgfpathlineto{\pgfqpoint{8.371276in}{2.138704in}}%
\pgfusepath{fill}%
\end{pgfscope}%
\begin{pgfscope}%
\pgfpathrectangle{\pgfqpoint{0.549740in}{0.463273in}}{\pgfqpoint{9.320225in}{4.495057in}}%
\pgfusepath{clip}%
\pgfsetbuttcap%
\pgfsetroundjoin%
\pgfsetlinewidth{0.000000pt}%
\definecolor{currentstroke}{rgb}{0.000000,0.000000,0.000000}%
\pgfsetstrokecolor{currentstroke}%
\pgfsetdash{}{0pt}%
\pgfpathmoveto{\pgfqpoint{8.557503in}{2.138704in}}%
\pgfpathlineto{\pgfqpoint{8.743729in}{2.138704in}}%
\pgfpathlineto{\pgfqpoint{8.743729in}{2.220432in}}%
\pgfpathlineto{\pgfqpoint{8.557503in}{2.220432in}}%
\pgfpathlineto{\pgfqpoint{8.557503in}{2.138704in}}%
\pgfusepath{}%
\end{pgfscope}%
\begin{pgfscope}%
\pgfpathrectangle{\pgfqpoint{0.549740in}{0.463273in}}{\pgfqpoint{9.320225in}{4.495057in}}%
\pgfusepath{clip}%
\pgfsetbuttcap%
\pgfsetroundjoin%
\pgfsetlinewidth{0.000000pt}%
\definecolor{currentstroke}{rgb}{0.000000,0.000000,0.000000}%
\pgfsetstrokecolor{currentstroke}%
\pgfsetdash{}{0pt}%
\pgfpathmoveto{\pgfqpoint{8.743729in}{2.138704in}}%
\pgfpathlineto{\pgfqpoint{8.929956in}{2.138704in}}%
\pgfpathlineto{\pgfqpoint{8.929956in}{2.220432in}}%
\pgfpathlineto{\pgfqpoint{8.743729in}{2.220432in}}%
\pgfpathlineto{\pgfqpoint{8.743729in}{2.138704in}}%
\pgfusepath{}%
\end{pgfscope}%
\begin{pgfscope}%
\pgfpathrectangle{\pgfqpoint{0.549740in}{0.463273in}}{\pgfqpoint{9.320225in}{4.495057in}}%
\pgfusepath{clip}%
\pgfsetbuttcap%
\pgfsetroundjoin%
\pgfsetlinewidth{0.000000pt}%
\definecolor{currentstroke}{rgb}{0.000000,0.000000,0.000000}%
\pgfsetstrokecolor{currentstroke}%
\pgfsetdash{}{0pt}%
\pgfpathmoveto{\pgfqpoint{8.929956in}{2.138704in}}%
\pgfpathlineto{\pgfqpoint{9.116182in}{2.138704in}}%
\pgfpathlineto{\pgfqpoint{9.116182in}{2.220432in}}%
\pgfpathlineto{\pgfqpoint{8.929956in}{2.220432in}}%
\pgfpathlineto{\pgfqpoint{8.929956in}{2.138704in}}%
\pgfusepath{}%
\end{pgfscope}%
\begin{pgfscope}%
\pgfpathrectangle{\pgfqpoint{0.549740in}{0.463273in}}{\pgfqpoint{9.320225in}{4.495057in}}%
\pgfusepath{clip}%
\pgfsetbuttcap%
\pgfsetroundjoin%
\pgfsetlinewidth{0.000000pt}%
\definecolor{currentstroke}{rgb}{0.000000,0.000000,0.000000}%
\pgfsetstrokecolor{currentstroke}%
\pgfsetdash{}{0pt}%
\pgfpathmoveto{\pgfqpoint{9.116182in}{2.138704in}}%
\pgfpathlineto{\pgfqpoint{9.302409in}{2.138704in}}%
\pgfpathlineto{\pgfqpoint{9.302409in}{2.220432in}}%
\pgfpathlineto{\pgfqpoint{9.116182in}{2.220432in}}%
\pgfpathlineto{\pgfqpoint{9.116182in}{2.138704in}}%
\pgfusepath{}%
\end{pgfscope}%
\begin{pgfscope}%
\pgfpathrectangle{\pgfqpoint{0.549740in}{0.463273in}}{\pgfqpoint{9.320225in}{4.495057in}}%
\pgfusepath{clip}%
\pgfsetbuttcap%
\pgfsetroundjoin%
\pgfsetlinewidth{0.000000pt}%
\definecolor{currentstroke}{rgb}{0.000000,0.000000,0.000000}%
\pgfsetstrokecolor{currentstroke}%
\pgfsetdash{}{0pt}%
\pgfpathmoveto{\pgfqpoint{9.302409in}{2.138704in}}%
\pgfpathlineto{\pgfqpoint{9.488635in}{2.138704in}}%
\pgfpathlineto{\pgfqpoint{9.488635in}{2.220432in}}%
\pgfpathlineto{\pgfqpoint{9.302409in}{2.220432in}}%
\pgfpathlineto{\pgfqpoint{9.302409in}{2.138704in}}%
\pgfusepath{}%
\end{pgfscope}%
\begin{pgfscope}%
\pgfpathrectangle{\pgfqpoint{0.549740in}{0.463273in}}{\pgfqpoint{9.320225in}{4.495057in}}%
\pgfusepath{clip}%
\pgfsetbuttcap%
\pgfsetroundjoin%
\pgfsetlinewidth{0.000000pt}%
\definecolor{currentstroke}{rgb}{0.000000,0.000000,0.000000}%
\pgfsetstrokecolor{currentstroke}%
\pgfsetdash{}{0pt}%
\pgfpathmoveto{\pgfqpoint{9.488635in}{2.138704in}}%
\pgfpathlineto{\pgfqpoint{9.674862in}{2.138704in}}%
\pgfpathlineto{\pgfqpoint{9.674862in}{2.220432in}}%
\pgfpathlineto{\pgfqpoint{9.488635in}{2.220432in}}%
\pgfpathlineto{\pgfqpoint{9.488635in}{2.138704in}}%
\pgfusepath{}%
\end{pgfscope}%
\begin{pgfscope}%
\pgfpathrectangle{\pgfqpoint{0.549740in}{0.463273in}}{\pgfqpoint{9.320225in}{4.495057in}}%
\pgfusepath{clip}%
\pgfsetbuttcap%
\pgfsetroundjoin%
\definecolor{currentfill}{rgb}{0.614330,0.761948,0.940009}%
\pgfsetfillcolor{currentfill}%
\pgfsetlinewidth{0.000000pt}%
\definecolor{currentstroke}{rgb}{0.000000,0.000000,0.000000}%
\pgfsetstrokecolor{currentstroke}%
\pgfsetdash{}{0pt}%
\pgfpathmoveto{\pgfqpoint{9.674862in}{2.138704in}}%
\pgfpathlineto{\pgfqpoint{9.861088in}{2.138704in}}%
\pgfpathlineto{\pgfqpoint{9.861088in}{2.220432in}}%
\pgfpathlineto{\pgfqpoint{9.674862in}{2.220432in}}%
\pgfpathlineto{\pgfqpoint{9.674862in}{2.138704in}}%
\pgfusepath{fill}%
\end{pgfscope}%
\begin{pgfscope}%
\pgfpathrectangle{\pgfqpoint{0.549740in}{0.463273in}}{\pgfqpoint{9.320225in}{4.495057in}}%
\pgfusepath{clip}%
\pgfsetbuttcap%
\pgfsetroundjoin%
\pgfsetlinewidth{0.000000pt}%
\definecolor{currentstroke}{rgb}{0.000000,0.000000,0.000000}%
\pgfsetstrokecolor{currentstroke}%
\pgfsetdash{}{0pt}%
\pgfpathmoveto{\pgfqpoint{0.549761in}{2.220432in}}%
\pgfpathlineto{\pgfqpoint{0.735988in}{2.220432in}}%
\pgfpathlineto{\pgfqpoint{0.735988in}{2.302160in}}%
\pgfpathlineto{\pgfqpoint{0.549761in}{2.302160in}}%
\pgfpathlineto{\pgfqpoint{0.549761in}{2.220432in}}%
\pgfusepath{}%
\end{pgfscope}%
\begin{pgfscope}%
\pgfpathrectangle{\pgfqpoint{0.549740in}{0.463273in}}{\pgfqpoint{9.320225in}{4.495057in}}%
\pgfusepath{clip}%
\pgfsetbuttcap%
\pgfsetroundjoin%
\pgfsetlinewidth{0.000000pt}%
\definecolor{currentstroke}{rgb}{0.000000,0.000000,0.000000}%
\pgfsetstrokecolor{currentstroke}%
\pgfsetdash{}{0pt}%
\pgfpathmoveto{\pgfqpoint{0.735988in}{2.220432in}}%
\pgfpathlineto{\pgfqpoint{0.922214in}{2.220432in}}%
\pgfpathlineto{\pgfqpoint{0.922214in}{2.302160in}}%
\pgfpathlineto{\pgfqpoint{0.735988in}{2.302160in}}%
\pgfpathlineto{\pgfqpoint{0.735988in}{2.220432in}}%
\pgfusepath{}%
\end{pgfscope}%
\begin{pgfscope}%
\pgfpathrectangle{\pgfqpoint{0.549740in}{0.463273in}}{\pgfqpoint{9.320225in}{4.495057in}}%
\pgfusepath{clip}%
\pgfsetbuttcap%
\pgfsetroundjoin%
\pgfsetlinewidth{0.000000pt}%
\definecolor{currentstroke}{rgb}{0.000000,0.000000,0.000000}%
\pgfsetstrokecolor{currentstroke}%
\pgfsetdash{}{0pt}%
\pgfpathmoveto{\pgfqpoint{0.922214in}{2.220432in}}%
\pgfpathlineto{\pgfqpoint{1.108441in}{2.220432in}}%
\pgfpathlineto{\pgfqpoint{1.108441in}{2.302160in}}%
\pgfpathlineto{\pgfqpoint{0.922214in}{2.302160in}}%
\pgfpathlineto{\pgfqpoint{0.922214in}{2.220432in}}%
\pgfusepath{}%
\end{pgfscope}%
\begin{pgfscope}%
\pgfpathrectangle{\pgfqpoint{0.549740in}{0.463273in}}{\pgfqpoint{9.320225in}{4.495057in}}%
\pgfusepath{clip}%
\pgfsetbuttcap%
\pgfsetroundjoin%
\pgfsetlinewidth{0.000000pt}%
\definecolor{currentstroke}{rgb}{0.000000,0.000000,0.000000}%
\pgfsetstrokecolor{currentstroke}%
\pgfsetdash{}{0pt}%
\pgfpathmoveto{\pgfqpoint{1.108441in}{2.220432in}}%
\pgfpathlineto{\pgfqpoint{1.294667in}{2.220432in}}%
\pgfpathlineto{\pgfqpoint{1.294667in}{2.302160in}}%
\pgfpathlineto{\pgfqpoint{1.108441in}{2.302160in}}%
\pgfpathlineto{\pgfqpoint{1.108441in}{2.220432in}}%
\pgfusepath{}%
\end{pgfscope}%
\begin{pgfscope}%
\pgfpathrectangle{\pgfqpoint{0.549740in}{0.463273in}}{\pgfqpoint{9.320225in}{4.495057in}}%
\pgfusepath{clip}%
\pgfsetbuttcap%
\pgfsetroundjoin%
\pgfsetlinewidth{0.000000pt}%
\definecolor{currentstroke}{rgb}{0.000000,0.000000,0.000000}%
\pgfsetstrokecolor{currentstroke}%
\pgfsetdash{}{0pt}%
\pgfpathmoveto{\pgfqpoint{1.294667in}{2.220432in}}%
\pgfpathlineto{\pgfqpoint{1.480894in}{2.220432in}}%
\pgfpathlineto{\pgfqpoint{1.480894in}{2.302160in}}%
\pgfpathlineto{\pgfqpoint{1.294667in}{2.302160in}}%
\pgfpathlineto{\pgfqpoint{1.294667in}{2.220432in}}%
\pgfusepath{}%
\end{pgfscope}%
\begin{pgfscope}%
\pgfpathrectangle{\pgfqpoint{0.549740in}{0.463273in}}{\pgfqpoint{9.320225in}{4.495057in}}%
\pgfusepath{clip}%
\pgfsetbuttcap%
\pgfsetroundjoin%
\pgfsetlinewidth{0.000000pt}%
\definecolor{currentstroke}{rgb}{0.000000,0.000000,0.000000}%
\pgfsetstrokecolor{currentstroke}%
\pgfsetdash{}{0pt}%
\pgfpathmoveto{\pgfqpoint{1.480894in}{2.220432in}}%
\pgfpathlineto{\pgfqpoint{1.667120in}{2.220432in}}%
\pgfpathlineto{\pgfqpoint{1.667120in}{2.302160in}}%
\pgfpathlineto{\pgfqpoint{1.480894in}{2.302160in}}%
\pgfpathlineto{\pgfqpoint{1.480894in}{2.220432in}}%
\pgfusepath{}%
\end{pgfscope}%
\begin{pgfscope}%
\pgfpathrectangle{\pgfqpoint{0.549740in}{0.463273in}}{\pgfqpoint{9.320225in}{4.495057in}}%
\pgfusepath{clip}%
\pgfsetbuttcap%
\pgfsetroundjoin%
\pgfsetlinewidth{0.000000pt}%
\definecolor{currentstroke}{rgb}{0.000000,0.000000,0.000000}%
\pgfsetstrokecolor{currentstroke}%
\pgfsetdash{}{0pt}%
\pgfpathmoveto{\pgfqpoint{1.667120in}{2.220432in}}%
\pgfpathlineto{\pgfqpoint{1.853347in}{2.220432in}}%
\pgfpathlineto{\pgfqpoint{1.853347in}{2.302160in}}%
\pgfpathlineto{\pgfqpoint{1.667120in}{2.302160in}}%
\pgfpathlineto{\pgfqpoint{1.667120in}{2.220432in}}%
\pgfusepath{}%
\end{pgfscope}%
\begin{pgfscope}%
\pgfpathrectangle{\pgfqpoint{0.549740in}{0.463273in}}{\pgfqpoint{9.320225in}{4.495057in}}%
\pgfusepath{clip}%
\pgfsetbuttcap%
\pgfsetroundjoin%
\pgfsetlinewidth{0.000000pt}%
\definecolor{currentstroke}{rgb}{0.000000,0.000000,0.000000}%
\pgfsetstrokecolor{currentstroke}%
\pgfsetdash{}{0pt}%
\pgfpathmoveto{\pgfqpoint{1.853347in}{2.220432in}}%
\pgfpathlineto{\pgfqpoint{2.039573in}{2.220432in}}%
\pgfpathlineto{\pgfqpoint{2.039573in}{2.302160in}}%
\pgfpathlineto{\pgfqpoint{1.853347in}{2.302160in}}%
\pgfpathlineto{\pgfqpoint{1.853347in}{2.220432in}}%
\pgfusepath{}%
\end{pgfscope}%
\begin{pgfscope}%
\pgfpathrectangle{\pgfqpoint{0.549740in}{0.463273in}}{\pgfqpoint{9.320225in}{4.495057in}}%
\pgfusepath{clip}%
\pgfsetbuttcap%
\pgfsetroundjoin%
\pgfsetlinewidth{0.000000pt}%
\definecolor{currentstroke}{rgb}{0.000000,0.000000,0.000000}%
\pgfsetstrokecolor{currentstroke}%
\pgfsetdash{}{0pt}%
\pgfpathmoveto{\pgfqpoint{2.039573in}{2.220432in}}%
\pgfpathlineto{\pgfqpoint{2.225800in}{2.220432in}}%
\pgfpathlineto{\pgfqpoint{2.225800in}{2.302160in}}%
\pgfpathlineto{\pgfqpoint{2.039573in}{2.302160in}}%
\pgfpathlineto{\pgfqpoint{2.039573in}{2.220432in}}%
\pgfusepath{}%
\end{pgfscope}%
\begin{pgfscope}%
\pgfpathrectangle{\pgfqpoint{0.549740in}{0.463273in}}{\pgfqpoint{9.320225in}{4.495057in}}%
\pgfusepath{clip}%
\pgfsetbuttcap%
\pgfsetroundjoin%
\pgfsetlinewidth{0.000000pt}%
\definecolor{currentstroke}{rgb}{0.000000,0.000000,0.000000}%
\pgfsetstrokecolor{currentstroke}%
\pgfsetdash{}{0pt}%
\pgfpathmoveto{\pgfqpoint{2.225800in}{2.220432in}}%
\pgfpathlineto{\pgfqpoint{2.412027in}{2.220432in}}%
\pgfpathlineto{\pgfqpoint{2.412027in}{2.302160in}}%
\pgfpathlineto{\pgfqpoint{2.225800in}{2.302160in}}%
\pgfpathlineto{\pgfqpoint{2.225800in}{2.220432in}}%
\pgfusepath{}%
\end{pgfscope}%
\begin{pgfscope}%
\pgfpathrectangle{\pgfqpoint{0.549740in}{0.463273in}}{\pgfqpoint{9.320225in}{4.495057in}}%
\pgfusepath{clip}%
\pgfsetbuttcap%
\pgfsetroundjoin%
\pgfsetlinewidth{0.000000pt}%
\definecolor{currentstroke}{rgb}{0.000000,0.000000,0.000000}%
\pgfsetstrokecolor{currentstroke}%
\pgfsetdash{}{0pt}%
\pgfpathmoveto{\pgfqpoint{2.412027in}{2.220432in}}%
\pgfpathlineto{\pgfqpoint{2.598253in}{2.220432in}}%
\pgfpathlineto{\pgfqpoint{2.598253in}{2.302160in}}%
\pgfpathlineto{\pgfqpoint{2.412027in}{2.302160in}}%
\pgfpathlineto{\pgfqpoint{2.412027in}{2.220432in}}%
\pgfusepath{}%
\end{pgfscope}%
\begin{pgfscope}%
\pgfpathrectangle{\pgfqpoint{0.549740in}{0.463273in}}{\pgfqpoint{9.320225in}{4.495057in}}%
\pgfusepath{clip}%
\pgfsetbuttcap%
\pgfsetroundjoin%
\pgfsetlinewidth{0.000000pt}%
\definecolor{currentstroke}{rgb}{0.000000,0.000000,0.000000}%
\pgfsetstrokecolor{currentstroke}%
\pgfsetdash{}{0pt}%
\pgfpathmoveto{\pgfqpoint{2.598253in}{2.220432in}}%
\pgfpathlineto{\pgfqpoint{2.784480in}{2.220432in}}%
\pgfpathlineto{\pgfqpoint{2.784480in}{2.302160in}}%
\pgfpathlineto{\pgfqpoint{2.598253in}{2.302160in}}%
\pgfpathlineto{\pgfqpoint{2.598253in}{2.220432in}}%
\pgfusepath{}%
\end{pgfscope}%
\begin{pgfscope}%
\pgfpathrectangle{\pgfqpoint{0.549740in}{0.463273in}}{\pgfqpoint{9.320225in}{4.495057in}}%
\pgfusepath{clip}%
\pgfsetbuttcap%
\pgfsetroundjoin%
\pgfsetlinewidth{0.000000pt}%
\definecolor{currentstroke}{rgb}{0.000000,0.000000,0.000000}%
\pgfsetstrokecolor{currentstroke}%
\pgfsetdash{}{0pt}%
\pgfpathmoveto{\pgfqpoint{2.784480in}{2.220432in}}%
\pgfpathlineto{\pgfqpoint{2.970706in}{2.220432in}}%
\pgfpathlineto{\pgfqpoint{2.970706in}{2.302160in}}%
\pgfpathlineto{\pgfqpoint{2.784480in}{2.302160in}}%
\pgfpathlineto{\pgfqpoint{2.784480in}{2.220432in}}%
\pgfusepath{}%
\end{pgfscope}%
\begin{pgfscope}%
\pgfpathrectangle{\pgfqpoint{0.549740in}{0.463273in}}{\pgfqpoint{9.320225in}{4.495057in}}%
\pgfusepath{clip}%
\pgfsetbuttcap%
\pgfsetroundjoin%
\pgfsetlinewidth{0.000000pt}%
\definecolor{currentstroke}{rgb}{0.000000,0.000000,0.000000}%
\pgfsetstrokecolor{currentstroke}%
\pgfsetdash{}{0pt}%
\pgfpathmoveto{\pgfqpoint{2.970706in}{2.220432in}}%
\pgfpathlineto{\pgfqpoint{3.156933in}{2.220432in}}%
\pgfpathlineto{\pgfqpoint{3.156933in}{2.302160in}}%
\pgfpathlineto{\pgfqpoint{2.970706in}{2.302160in}}%
\pgfpathlineto{\pgfqpoint{2.970706in}{2.220432in}}%
\pgfusepath{}%
\end{pgfscope}%
\begin{pgfscope}%
\pgfpathrectangle{\pgfqpoint{0.549740in}{0.463273in}}{\pgfqpoint{9.320225in}{4.495057in}}%
\pgfusepath{clip}%
\pgfsetbuttcap%
\pgfsetroundjoin%
\pgfsetlinewidth{0.000000pt}%
\definecolor{currentstroke}{rgb}{0.000000,0.000000,0.000000}%
\pgfsetstrokecolor{currentstroke}%
\pgfsetdash{}{0pt}%
\pgfpathmoveto{\pgfqpoint{3.156933in}{2.220432in}}%
\pgfpathlineto{\pgfqpoint{3.343159in}{2.220432in}}%
\pgfpathlineto{\pgfqpoint{3.343159in}{2.302160in}}%
\pgfpathlineto{\pgfqpoint{3.156933in}{2.302160in}}%
\pgfpathlineto{\pgfqpoint{3.156933in}{2.220432in}}%
\pgfusepath{}%
\end{pgfscope}%
\begin{pgfscope}%
\pgfpathrectangle{\pgfqpoint{0.549740in}{0.463273in}}{\pgfqpoint{9.320225in}{4.495057in}}%
\pgfusepath{clip}%
\pgfsetbuttcap%
\pgfsetroundjoin%
\pgfsetlinewidth{0.000000pt}%
\definecolor{currentstroke}{rgb}{0.000000,0.000000,0.000000}%
\pgfsetstrokecolor{currentstroke}%
\pgfsetdash{}{0pt}%
\pgfpathmoveto{\pgfqpoint{3.343159in}{2.220432in}}%
\pgfpathlineto{\pgfqpoint{3.529386in}{2.220432in}}%
\pgfpathlineto{\pgfqpoint{3.529386in}{2.302160in}}%
\pgfpathlineto{\pgfqpoint{3.343159in}{2.302160in}}%
\pgfpathlineto{\pgfqpoint{3.343159in}{2.220432in}}%
\pgfusepath{}%
\end{pgfscope}%
\begin{pgfscope}%
\pgfpathrectangle{\pgfqpoint{0.549740in}{0.463273in}}{\pgfqpoint{9.320225in}{4.495057in}}%
\pgfusepath{clip}%
\pgfsetbuttcap%
\pgfsetroundjoin%
\pgfsetlinewidth{0.000000pt}%
\definecolor{currentstroke}{rgb}{0.000000,0.000000,0.000000}%
\pgfsetstrokecolor{currentstroke}%
\pgfsetdash{}{0pt}%
\pgfpathmoveto{\pgfqpoint{3.529386in}{2.220432in}}%
\pgfpathlineto{\pgfqpoint{3.715612in}{2.220432in}}%
\pgfpathlineto{\pgfqpoint{3.715612in}{2.302160in}}%
\pgfpathlineto{\pgfqpoint{3.529386in}{2.302160in}}%
\pgfpathlineto{\pgfqpoint{3.529386in}{2.220432in}}%
\pgfusepath{}%
\end{pgfscope}%
\begin{pgfscope}%
\pgfpathrectangle{\pgfqpoint{0.549740in}{0.463273in}}{\pgfqpoint{9.320225in}{4.495057in}}%
\pgfusepath{clip}%
\pgfsetbuttcap%
\pgfsetroundjoin%
\pgfsetlinewidth{0.000000pt}%
\definecolor{currentstroke}{rgb}{0.000000,0.000000,0.000000}%
\pgfsetstrokecolor{currentstroke}%
\pgfsetdash{}{0pt}%
\pgfpathmoveto{\pgfqpoint{3.715612in}{2.220432in}}%
\pgfpathlineto{\pgfqpoint{3.901839in}{2.220432in}}%
\pgfpathlineto{\pgfqpoint{3.901839in}{2.302160in}}%
\pgfpathlineto{\pgfqpoint{3.715612in}{2.302160in}}%
\pgfpathlineto{\pgfqpoint{3.715612in}{2.220432in}}%
\pgfusepath{}%
\end{pgfscope}%
\begin{pgfscope}%
\pgfpathrectangle{\pgfqpoint{0.549740in}{0.463273in}}{\pgfqpoint{9.320225in}{4.495057in}}%
\pgfusepath{clip}%
\pgfsetbuttcap%
\pgfsetroundjoin%
\pgfsetlinewidth{0.000000pt}%
\definecolor{currentstroke}{rgb}{0.000000,0.000000,0.000000}%
\pgfsetstrokecolor{currentstroke}%
\pgfsetdash{}{0pt}%
\pgfpathmoveto{\pgfqpoint{3.901839in}{2.220432in}}%
\pgfpathlineto{\pgfqpoint{4.088065in}{2.220432in}}%
\pgfpathlineto{\pgfqpoint{4.088065in}{2.302160in}}%
\pgfpathlineto{\pgfqpoint{3.901839in}{2.302160in}}%
\pgfpathlineto{\pgfqpoint{3.901839in}{2.220432in}}%
\pgfusepath{}%
\end{pgfscope}%
\begin{pgfscope}%
\pgfpathrectangle{\pgfqpoint{0.549740in}{0.463273in}}{\pgfqpoint{9.320225in}{4.495057in}}%
\pgfusepath{clip}%
\pgfsetbuttcap%
\pgfsetroundjoin%
\pgfsetlinewidth{0.000000pt}%
\definecolor{currentstroke}{rgb}{0.000000,0.000000,0.000000}%
\pgfsetstrokecolor{currentstroke}%
\pgfsetdash{}{0pt}%
\pgfpathmoveto{\pgfqpoint{4.088065in}{2.220432in}}%
\pgfpathlineto{\pgfqpoint{4.274292in}{2.220432in}}%
\pgfpathlineto{\pgfqpoint{4.274292in}{2.302160in}}%
\pgfpathlineto{\pgfqpoint{4.088065in}{2.302160in}}%
\pgfpathlineto{\pgfqpoint{4.088065in}{2.220432in}}%
\pgfusepath{}%
\end{pgfscope}%
\begin{pgfscope}%
\pgfpathrectangle{\pgfqpoint{0.549740in}{0.463273in}}{\pgfqpoint{9.320225in}{4.495057in}}%
\pgfusepath{clip}%
\pgfsetbuttcap%
\pgfsetroundjoin%
\pgfsetlinewidth{0.000000pt}%
\definecolor{currentstroke}{rgb}{0.000000,0.000000,0.000000}%
\pgfsetstrokecolor{currentstroke}%
\pgfsetdash{}{0pt}%
\pgfpathmoveto{\pgfqpoint{4.274292in}{2.220432in}}%
\pgfpathlineto{\pgfqpoint{4.460519in}{2.220432in}}%
\pgfpathlineto{\pgfqpoint{4.460519in}{2.302160in}}%
\pgfpathlineto{\pgfqpoint{4.274292in}{2.302160in}}%
\pgfpathlineto{\pgfqpoint{4.274292in}{2.220432in}}%
\pgfusepath{}%
\end{pgfscope}%
\begin{pgfscope}%
\pgfpathrectangle{\pgfqpoint{0.549740in}{0.463273in}}{\pgfqpoint{9.320225in}{4.495057in}}%
\pgfusepath{clip}%
\pgfsetbuttcap%
\pgfsetroundjoin%
\pgfsetlinewidth{0.000000pt}%
\definecolor{currentstroke}{rgb}{0.000000,0.000000,0.000000}%
\pgfsetstrokecolor{currentstroke}%
\pgfsetdash{}{0pt}%
\pgfpathmoveto{\pgfqpoint{4.460519in}{2.220432in}}%
\pgfpathlineto{\pgfqpoint{4.646745in}{2.220432in}}%
\pgfpathlineto{\pgfqpoint{4.646745in}{2.302160in}}%
\pgfpathlineto{\pgfqpoint{4.460519in}{2.302160in}}%
\pgfpathlineto{\pgfqpoint{4.460519in}{2.220432in}}%
\pgfusepath{}%
\end{pgfscope}%
\begin{pgfscope}%
\pgfpathrectangle{\pgfqpoint{0.549740in}{0.463273in}}{\pgfqpoint{9.320225in}{4.495057in}}%
\pgfusepath{clip}%
\pgfsetbuttcap%
\pgfsetroundjoin%
\pgfsetlinewidth{0.000000pt}%
\definecolor{currentstroke}{rgb}{0.000000,0.000000,0.000000}%
\pgfsetstrokecolor{currentstroke}%
\pgfsetdash{}{0pt}%
\pgfpathmoveto{\pgfqpoint{4.646745in}{2.220432in}}%
\pgfpathlineto{\pgfqpoint{4.832972in}{2.220432in}}%
\pgfpathlineto{\pgfqpoint{4.832972in}{2.302160in}}%
\pgfpathlineto{\pgfqpoint{4.646745in}{2.302160in}}%
\pgfpathlineto{\pgfqpoint{4.646745in}{2.220432in}}%
\pgfusepath{}%
\end{pgfscope}%
\begin{pgfscope}%
\pgfpathrectangle{\pgfqpoint{0.549740in}{0.463273in}}{\pgfqpoint{9.320225in}{4.495057in}}%
\pgfusepath{clip}%
\pgfsetbuttcap%
\pgfsetroundjoin%
\pgfsetlinewidth{0.000000pt}%
\definecolor{currentstroke}{rgb}{0.000000,0.000000,0.000000}%
\pgfsetstrokecolor{currentstroke}%
\pgfsetdash{}{0pt}%
\pgfpathmoveto{\pgfqpoint{4.832972in}{2.220432in}}%
\pgfpathlineto{\pgfqpoint{5.019198in}{2.220432in}}%
\pgfpathlineto{\pgfqpoint{5.019198in}{2.302160in}}%
\pgfpathlineto{\pgfqpoint{4.832972in}{2.302160in}}%
\pgfpathlineto{\pgfqpoint{4.832972in}{2.220432in}}%
\pgfusepath{}%
\end{pgfscope}%
\begin{pgfscope}%
\pgfpathrectangle{\pgfqpoint{0.549740in}{0.463273in}}{\pgfqpoint{9.320225in}{4.495057in}}%
\pgfusepath{clip}%
\pgfsetbuttcap%
\pgfsetroundjoin%
\pgfsetlinewidth{0.000000pt}%
\definecolor{currentstroke}{rgb}{0.000000,0.000000,0.000000}%
\pgfsetstrokecolor{currentstroke}%
\pgfsetdash{}{0pt}%
\pgfpathmoveto{\pgfqpoint{5.019198in}{2.220432in}}%
\pgfpathlineto{\pgfqpoint{5.205425in}{2.220432in}}%
\pgfpathlineto{\pgfqpoint{5.205425in}{2.302160in}}%
\pgfpathlineto{\pgfqpoint{5.019198in}{2.302160in}}%
\pgfpathlineto{\pgfqpoint{5.019198in}{2.220432in}}%
\pgfusepath{}%
\end{pgfscope}%
\begin{pgfscope}%
\pgfpathrectangle{\pgfqpoint{0.549740in}{0.463273in}}{\pgfqpoint{9.320225in}{4.495057in}}%
\pgfusepath{clip}%
\pgfsetbuttcap%
\pgfsetroundjoin%
\pgfsetlinewidth{0.000000pt}%
\definecolor{currentstroke}{rgb}{0.000000,0.000000,0.000000}%
\pgfsetstrokecolor{currentstroke}%
\pgfsetdash{}{0pt}%
\pgfpathmoveto{\pgfqpoint{5.205425in}{2.220432in}}%
\pgfpathlineto{\pgfqpoint{5.391651in}{2.220432in}}%
\pgfpathlineto{\pgfqpoint{5.391651in}{2.302160in}}%
\pgfpathlineto{\pgfqpoint{5.205425in}{2.302160in}}%
\pgfpathlineto{\pgfqpoint{5.205425in}{2.220432in}}%
\pgfusepath{}%
\end{pgfscope}%
\begin{pgfscope}%
\pgfpathrectangle{\pgfqpoint{0.549740in}{0.463273in}}{\pgfqpoint{9.320225in}{4.495057in}}%
\pgfusepath{clip}%
\pgfsetbuttcap%
\pgfsetroundjoin%
\definecolor{currentfill}{rgb}{0.472869,0.711325,0.955316}%
\pgfsetfillcolor{currentfill}%
\pgfsetlinewidth{0.000000pt}%
\definecolor{currentstroke}{rgb}{0.000000,0.000000,0.000000}%
\pgfsetstrokecolor{currentstroke}%
\pgfsetdash{}{0pt}%
\pgfpathmoveto{\pgfqpoint{5.391651in}{2.220432in}}%
\pgfpathlineto{\pgfqpoint{5.577878in}{2.220432in}}%
\pgfpathlineto{\pgfqpoint{5.577878in}{2.302160in}}%
\pgfpathlineto{\pgfqpoint{5.391651in}{2.302160in}}%
\pgfpathlineto{\pgfqpoint{5.391651in}{2.220432in}}%
\pgfusepath{fill}%
\end{pgfscope}%
\begin{pgfscope}%
\pgfpathrectangle{\pgfqpoint{0.549740in}{0.463273in}}{\pgfqpoint{9.320225in}{4.495057in}}%
\pgfusepath{clip}%
\pgfsetbuttcap%
\pgfsetroundjoin%
\pgfsetlinewidth{0.000000pt}%
\definecolor{currentstroke}{rgb}{0.000000,0.000000,0.000000}%
\pgfsetstrokecolor{currentstroke}%
\pgfsetdash{}{0pt}%
\pgfpathmoveto{\pgfqpoint{5.577878in}{2.220432in}}%
\pgfpathlineto{\pgfqpoint{5.764104in}{2.220432in}}%
\pgfpathlineto{\pgfqpoint{5.764104in}{2.302160in}}%
\pgfpathlineto{\pgfqpoint{5.577878in}{2.302160in}}%
\pgfpathlineto{\pgfqpoint{5.577878in}{2.220432in}}%
\pgfusepath{}%
\end{pgfscope}%
\begin{pgfscope}%
\pgfpathrectangle{\pgfqpoint{0.549740in}{0.463273in}}{\pgfqpoint{9.320225in}{4.495057in}}%
\pgfusepath{clip}%
\pgfsetbuttcap%
\pgfsetroundjoin%
\pgfsetlinewidth{0.000000pt}%
\definecolor{currentstroke}{rgb}{0.000000,0.000000,0.000000}%
\pgfsetstrokecolor{currentstroke}%
\pgfsetdash{}{0pt}%
\pgfpathmoveto{\pgfqpoint{5.764104in}{2.220432in}}%
\pgfpathlineto{\pgfqpoint{5.950331in}{2.220432in}}%
\pgfpathlineto{\pgfqpoint{5.950331in}{2.302160in}}%
\pgfpathlineto{\pgfqpoint{5.764104in}{2.302160in}}%
\pgfpathlineto{\pgfqpoint{5.764104in}{2.220432in}}%
\pgfusepath{}%
\end{pgfscope}%
\begin{pgfscope}%
\pgfpathrectangle{\pgfqpoint{0.549740in}{0.463273in}}{\pgfqpoint{9.320225in}{4.495057in}}%
\pgfusepath{clip}%
\pgfsetbuttcap%
\pgfsetroundjoin%
\pgfsetlinewidth{0.000000pt}%
\definecolor{currentstroke}{rgb}{0.000000,0.000000,0.000000}%
\pgfsetstrokecolor{currentstroke}%
\pgfsetdash{}{0pt}%
\pgfpathmoveto{\pgfqpoint{5.950331in}{2.220432in}}%
\pgfpathlineto{\pgfqpoint{6.136557in}{2.220432in}}%
\pgfpathlineto{\pgfqpoint{6.136557in}{2.302160in}}%
\pgfpathlineto{\pgfqpoint{5.950331in}{2.302160in}}%
\pgfpathlineto{\pgfqpoint{5.950331in}{2.220432in}}%
\pgfusepath{}%
\end{pgfscope}%
\begin{pgfscope}%
\pgfpathrectangle{\pgfqpoint{0.549740in}{0.463273in}}{\pgfqpoint{9.320225in}{4.495057in}}%
\pgfusepath{clip}%
\pgfsetbuttcap%
\pgfsetroundjoin%
\pgfsetlinewidth{0.000000pt}%
\definecolor{currentstroke}{rgb}{0.000000,0.000000,0.000000}%
\pgfsetstrokecolor{currentstroke}%
\pgfsetdash{}{0pt}%
\pgfpathmoveto{\pgfqpoint{6.136557in}{2.220432in}}%
\pgfpathlineto{\pgfqpoint{6.322784in}{2.220432in}}%
\pgfpathlineto{\pgfqpoint{6.322784in}{2.302160in}}%
\pgfpathlineto{\pgfqpoint{6.136557in}{2.302160in}}%
\pgfpathlineto{\pgfqpoint{6.136557in}{2.220432in}}%
\pgfusepath{}%
\end{pgfscope}%
\begin{pgfscope}%
\pgfpathrectangle{\pgfqpoint{0.549740in}{0.463273in}}{\pgfqpoint{9.320225in}{4.495057in}}%
\pgfusepath{clip}%
\pgfsetbuttcap%
\pgfsetroundjoin%
\definecolor{currentfill}{rgb}{0.472869,0.711325,0.955316}%
\pgfsetfillcolor{currentfill}%
\pgfsetlinewidth{0.000000pt}%
\definecolor{currentstroke}{rgb}{0.000000,0.000000,0.000000}%
\pgfsetstrokecolor{currentstroke}%
\pgfsetdash{}{0pt}%
\pgfpathmoveto{\pgfqpoint{6.322784in}{2.220432in}}%
\pgfpathlineto{\pgfqpoint{6.509011in}{2.220432in}}%
\pgfpathlineto{\pgfqpoint{6.509011in}{2.302160in}}%
\pgfpathlineto{\pgfqpoint{6.322784in}{2.302160in}}%
\pgfpathlineto{\pgfqpoint{6.322784in}{2.220432in}}%
\pgfusepath{fill}%
\end{pgfscope}%
\begin{pgfscope}%
\pgfpathrectangle{\pgfqpoint{0.549740in}{0.463273in}}{\pgfqpoint{9.320225in}{4.495057in}}%
\pgfusepath{clip}%
\pgfsetbuttcap%
\pgfsetroundjoin%
\pgfsetlinewidth{0.000000pt}%
\definecolor{currentstroke}{rgb}{0.000000,0.000000,0.000000}%
\pgfsetstrokecolor{currentstroke}%
\pgfsetdash{}{0pt}%
\pgfpathmoveto{\pgfqpoint{6.509011in}{2.220432in}}%
\pgfpathlineto{\pgfqpoint{6.695237in}{2.220432in}}%
\pgfpathlineto{\pgfqpoint{6.695237in}{2.302160in}}%
\pgfpathlineto{\pgfqpoint{6.509011in}{2.302160in}}%
\pgfpathlineto{\pgfqpoint{6.509011in}{2.220432in}}%
\pgfusepath{}%
\end{pgfscope}%
\begin{pgfscope}%
\pgfpathrectangle{\pgfqpoint{0.549740in}{0.463273in}}{\pgfqpoint{9.320225in}{4.495057in}}%
\pgfusepath{clip}%
\pgfsetbuttcap%
\pgfsetroundjoin%
\pgfsetlinewidth{0.000000pt}%
\definecolor{currentstroke}{rgb}{0.000000,0.000000,0.000000}%
\pgfsetstrokecolor{currentstroke}%
\pgfsetdash{}{0pt}%
\pgfpathmoveto{\pgfqpoint{6.695237in}{2.220432in}}%
\pgfpathlineto{\pgfqpoint{6.881464in}{2.220432in}}%
\pgfpathlineto{\pgfqpoint{6.881464in}{2.302160in}}%
\pgfpathlineto{\pgfqpoint{6.695237in}{2.302160in}}%
\pgfpathlineto{\pgfqpoint{6.695237in}{2.220432in}}%
\pgfusepath{}%
\end{pgfscope}%
\begin{pgfscope}%
\pgfpathrectangle{\pgfqpoint{0.549740in}{0.463273in}}{\pgfqpoint{9.320225in}{4.495057in}}%
\pgfusepath{clip}%
\pgfsetbuttcap%
\pgfsetroundjoin%
\pgfsetlinewidth{0.000000pt}%
\definecolor{currentstroke}{rgb}{0.000000,0.000000,0.000000}%
\pgfsetstrokecolor{currentstroke}%
\pgfsetdash{}{0pt}%
\pgfpathmoveto{\pgfqpoint{6.881464in}{2.220432in}}%
\pgfpathlineto{\pgfqpoint{7.067690in}{2.220432in}}%
\pgfpathlineto{\pgfqpoint{7.067690in}{2.302160in}}%
\pgfpathlineto{\pgfqpoint{6.881464in}{2.302160in}}%
\pgfpathlineto{\pgfqpoint{6.881464in}{2.220432in}}%
\pgfusepath{}%
\end{pgfscope}%
\begin{pgfscope}%
\pgfpathrectangle{\pgfqpoint{0.549740in}{0.463273in}}{\pgfqpoint{9.320225in}{4.495057in}}%
\pgfusepath{clip}%
\pgfsetbuttcap%
\pgfsetroundjoin%
\pgfsetlinewidth{0.000000pt}%
\definecolor{currentstroke}{rgb}{0.000000,0.000000,0.000000}%
\pgfsetstrokecolor{currentstroke}%
\pgfsetdash{}{0pt}%
\pgfpathmoveto{\pgfqpoint{7.067690in}{2.220432in}}%
\pgfpathlineto{\pgfqpoint{7.253917in}{2.220432in}}%
\pgfpathlineto{\pgfqpoint{7.253917in}{2.302160in}}%
\pgfpathlineto{\pgfqpoint{7.067690in}{2.302160in}}%
\pgfpathlineto{\pgfqpoint{7.067690in}{2.220432in}}%
\pgfusepath{}%
\end{pgfscope}%
\begin{pgfscope}%
\pgfpathrectangle{\pgfqpoint{0.549740in}{0.463273in}}{\pgfqpoint{9.320225in}{4.495057in}}%
\pgfusepath{clip}%
\pgfsetbuttcap%
\pgfsetroundjoin%
\definecolor{currentfill}{rgb}{0.614330,0.761948,0.940009}%
\pgfsetfillcolor{currentfill}%
\pgfsetlinewidth{0.000000pt}%
\definecolor{currentstroke}{rgb}{0.000000,0.000000,0.000000}%
\pgfsetstrokecolor{currentstroke}%
\pgfsetdash{}{0pt}%
\pgfpathmoveto{\pgfqpoint{7.253917in}{2.220432in}}%
\pgfpathlineto{\pgfqpoint{7.440143in}{2.220432in}}%
\pgfpathlineto{\pgfqpoint{7.440143in}{2.302160in}}%
\pgfpathlineto{\pgfqpoint{7.253917in}{2.302160in}}%
\pgfpathlineto{\pgfqpoint{7.253917in}{2.220432in}}%
\pgfusepath{fill}%
\end{pgfscope}%
\begin{pgfscope}%
\pgfpathrectangle{\pgfqpoint{0.549740in}{0.463273in}}{\pgfqpoint{9.320225in}{4.495057in}}%
\pgfusepath{clip}%
\pgfsetbuttcap%
\pgfsetroundjoin%
\definecolor{currentfill}{rgb}{0.547810,0.736432,0.947518}%
\pgfsetfillcolor{currentfill}%
\pgfsetlinewidth{0.000000pt}%
\definecolor{currentstroke}{rgb}{0.000000,0.000000,0.000000}%
\pgfsetstrokecolor{currentstroke}%
\pgfsetdash{}{0pt}%
\pgfpathmoveto{\pgfqpoint{7.440143in}{2.220432in}}%
\pgfpathlineto{\pgfqpoint{7.626370in}{2.220432in}}%
\pgfpathlineto{\pgfqpoint{7.626370in}{2.302160in}}%
\pgfpathlineto{\pgfqpoint{7.440143in}{2.302160in}}%
\pgfpathlineto{\pgfqpoint{7.440143in}{2.220432in}}%
\pgfusepath{fill}%
\end{pgfscope}%
\begin{pgfscope}%
\pgfpathrectangle{\pgfqpoint{0.549740in}{0.463273in}}{\pgfqpoint{9.320225in}{4.495057in}}%
\pgfusepath{clip}%
\pgfsetbuttcap%
\pgfsetroundjoin%
\pgfsetlinewidth{0.000000pt}%
\definecolor{currentstroke}{rgb}{0.000000,0.000000,0.000000}%
\pgfsetstrokecolor{currentstroke}%
\pgfsetdash{}{0pt}%
\pgfpathmoveto{\pgfqpoint{7.626370in}{2.220432in}}%
\pgfpathlineto{\pgfqpoint{7.812596in}{2.220432in}}%
\pgfpathlineto{\pgfqpoint{7.812596in}{2.302160in}}%
\pgfpathlineto{\pgfqpoint{7.626370in}{2.302160in}}%
\pgfpathlineto{\pgfqpoint{7.626370in}{2.220432in}}%
\pgfusepath{}%
\end{pgfscope}%
\begin{pgfscope}%
\pgfpathrectangle{\pgfqpoint{0.549740in}{0.463273in}}{\pgfqpoint{9.320225in}{4.495057in}}%
\pgfusepath{clip}%
\pgfsetbuttcap%
\pgfsetroundjoin%
\pgfsetlinewidth{0.000000pt}%
\definecolor{currentstroke}{rgb}{0.000000,0.000000,0.000000}%
\pgfsetstrokecolor{currentstroke}%
\pgfsetdash{}{0pt}%
\pgfpathmoveto{\pgfqpoint{7.812596in}{2.220432in}}%
\pgfpathlineto{\pgfqpoint{7.998823in}{2.220432in}}%
\pgfpathlineto{\pgfqpoint{7.998823in}{2.302160in}}%
\pgfpathlineto{\pgfqpoint{7.812596in}{2.302160in}}%
\pgfpathlineto{\pgfqpoint{7.812596in}{2.220432in}}%
\pgfusepath{}%
\end{pgfscope}%
\begin{pgfscope}%
\pgfpathrectangle{\pgfqpoint{0.549740in}{0.463273in}}{\pgfqpoint{9.320225in}{4.495057in}}%
\pgfusepath{clip}%
\pgfsetbuttcap%
\pgfsetroundjoin%
\pgfsetlinewidth{0.000000pt}%
\definecolor{currentstroke}{rgb}{0.000000,0.000000,0.000000}%
\pgfsetstrokecolor{currentstroke}%
\pgfsetdash{}{0pt}%
\pgfpathmoveto{\pgfqpoint{7.998823in}{2.220432in}}%
\pgfpathlineto{\pgfqpoint{8.185049in}{2.220432in}}%
\pgfpathlineto{\pgfqpoint{8.185049in}{2.302160in}}%
\pgfpathlineto{\pgfqpoint{7.998823in}{2.302160in}}%
\pgfpathlineto{\pgfqpoint{7.998823in}{2.220432in}}%
\pgfusepath{}%
\end{pgfscope}%
\begin{pgfscope}%
\pgfpathrectangle{\pgfqpoint{0.549740in}{0.463273in}}{\pgfqpoint{9.320225in}{4.495057in}}%
\pgfusepath{clip}%
\pgfsetbuttcap%
\pgfsetroundjoin%
\pgfsetlinewidth{0.000000pt}%
\definecolor{currentstroke}{rgb}{0.000000,0.000000,0.000000}%
\pgfsetstrokecolor{currentstroke}%
\pgfsetdash{}{0pt}%
\pgfpathmoveto{\pgfqpoint{8.185049in}{2.220432in}}%
\pgfpathlineto{\pgfqpoint{8.371276in}{2.220432in}}%
\pgfpathlineto{\pgfqpoint{8.371276in}{2.302160in}}%
\pgfpathlineto{\pgfqpoint{8.185049in}{2.302160in}}%
\pgfpathlineto{\pgfqpoint{8.185049in}{2.220432in}}%
\pgfusepath{}%
\end{pgfscope}%
\begin{pgfscope}%
\pgfpathrectangle{\pgfqpoint{0.549740in}{0.463273in}}{\pgfqpoint{9.320225in}{4.495057in}}%
\pgfusepath{clip}%
\pgfsetbuttcap%
\pgfsetroundjoin%
\definecolor{currentfill}{rgb}{0.472869,0.711325,0.955316}%
\pgfsetfillcolor{currentfill}%
\pgfsetlinewidth{0.000000pt}%
\definecolor{currentstroke}{rgb}{0.000000,0.000000,0.000000}%
\pgfsetstrokecolor{currentstroke}%
\pgfsetdash{}{0pt}%
\pgfpathmoveto{\pgfqpoint{8.371276in}{2.220432in}}%
\pgfpathlineto{\pgfqpoint{8.557503in}{2.220432in}}%
\pgfpathlineto{\pgfqpoint{8.557503in}{2.302160in}}%
\pgfpathlineto{\pgfqpoint{8.371276in}{2.302160in}}%
\pgfpathlineto{\pgfqpoint{8.371276in}{2.220432in}}%
\pgfusepath{fill}%
\end{pgfscope}%
\begin{pgfscope}%
\pgfpathrectangle{\pgfqpoint{0.549740in}{0.463273in}}{\pgfqpoint{9.320225in}{4.495057in}}%
\pgfusepath{clip}%
\pgfsetbuttcap%
\pgfsetroundjoin%
\pgfsetlinewidth{0.000000pt}%
\definecolor{currentstroke}{rgb}{0.000000,0.000000,0.000000}%
\pgfsetstrokecolor{currentstroke}%
\pgfsetdash{}{0pt}%
\pgfpathmoveto{\pgfqpoint{8.557503in}{2.220432in}}%
\pgfpathlineto{\pgfqpoint{8.743729in}{2.220432in}}%
\pgfpathlineto{\pgfqpoint{8.743729in}{2.302160in}}%
\pgfpathlineto{\pgfqpoint{8.557503in}{2.302160in}}%
\pgfpathlineto{\pgfqpoint{8.557503in}{2.220432in}}%
\pgfusepath{}%
\end{pgfscope}%
\begin{pgfscope}%
\pgfpathrectangle{\pgfqpoint{0.549740in}{0.463273in}}{\pgfqpoint{9.320225in}{4.495057in}}%
\pgfusepath{clip}%
\pgfsetbuttcap%
\pgfsetroundjoin%
\pgfsetlinewidth{0.000000pt}%
\definecolor{currentstroke}{rgb}{0.000000,0.000000,0.000000}%
\pgfsetstrokecolor{currentstroke}%
\pgfsetdash{}{0pt}%
\pgfpathmoveto{\pgfqpoint{8.743729in}{2.220432in}}%
\pgfpathlineto{\pgfqpoint{8.929956in}{2.220432in}}%
\pgfpathlineto{\pgfqpoint{8.929956in}{2.302160in}}%
\pgfpathlineto{\pgfqpoint{8.743729in}{2.302160in}}%
\pgfpathlineto{\pgfqpoint{8.743729in}{2.220432in}}%
\pgfusepath{}%
\end{pgfscope}%
\begin{pgfscope}%
\pgfpathrectangle{\pgfqpoint{0.549740in}{0.463273in}}{\pgfqpoint{9.320225in}{4.495057in}}%
\pgfusepath{clip}%
\pgfsetbuttcap%
\pgfsetroundjoin%
\pgfsetlinewidth{0.000000pt}%
\definecolor{currentstroke}{rgb}{0.000000,0.000000,0.000000}%
\pgfsetstrokecolor{currentstroke}%
\pgfsetdash{}{0pt}%
\pgfpathmoveto{\pgfqpoint{8.929956in}{2.220432in}}%
\pgfpathlineto{\pgfqpoint{9.116182in}{2.220432in}}%
\pgfpathlineto{\pgfqpoint{9.116182in}{2.302160in}}%
\pgfpathlineto{\pgfqpoint{8.929956in}{2.302160in}}%
\pgfpathlineto{\pgfqpoint{8.929956in}{2.220432in}}%
\pgfusepath{}%
\end{pgfscope}%
\begin{pgfscope}%
\pgfpathrectangle{\pgfqpoint{0.549740in}{0.463273in}}{\pgfqpoint{9.320225in}{4.495057in}}%
\pgfusepath{clip}%
\pgfsetbuttcap%
\pgfsetroundjoin%
\pgfsetlinewidth{0.000000pt}%
\definecolor{currentstroke}{rgb}{0.000000,0.000000,0.000000}%
\pgfsetstrokecolor{currentstroke}%
\pgfsetdash{}{0pt}%
\pgfpathmoveto{\pgfqpoint{9.116182in}{2.220432in}}%
\pgfpathlineto{\pgfqpoint{9.302409in}{2.220432in}}%
\pgfpathlineto{\pgfqpoint{9.302409in}{2.302160in}}%
\pgfpathlineto{\pgfqpoint{9.116182in}{2.302160in}}%
\pgfpathlineto{\pgfqpoint{9.116182in}{2.220432in}}%
\pgfusepath{}%
\end{pgfscope}%
\begin{pgfscope}%
\pgfpathrectangle{\pgfqpoint{0.549740in}{0.463273in}}{\pgfqpoint{9.320225in}{4.495057in}}%
\pgfusepath{clip}%
\pgfsetbuttcap%
\pgfsetroundjoin%
\pgfsetlinewidth{0.000000pt}%
\definecolor{currentstroke}{rgb}{0.000000,0.000000,0.000000}%
\pgfsetstrokecolor{currentstroke}%
\pgfsetdash{}{0pt}%
\pgfpathmoveto{\pgfqpoint{9.302409in}{2.220432in}}%
\pgfpathlineto{\pgfqpoint{9.488635in}{2.220432in}}%
\pgfpathlineto{\pgfqpoint{9.488635in}{2.302160in}}%
\pgfpathlineto{\pgfqpoint{9.302409in}{2.302160in}}%
\pgfpathlineto{\pgfqpoint{9.302409in}{2.220432in}}%
\pgfusepath{}%
\end{pgfscope}%
\begin{pgfscope}%
\pgfpathrectangle{\pgfqpoint{0.549740in}{0.463273in}}{\pgfqpoint{9.320225in}{4.495057in}}%
\pgfusepath{clip}%
\pgfsetbuttcap%
\pgfsetroundjoin%
\pgfsetlinewidth{0.000000pt}%
\definecolor{currentstroke}{rgb}{0.000000,0.000000,0.000000}%
\pgfsetstrokecolor{currentstroke}%
\pgfsetdash{}{0pt}%
\pgfpathmoveto{\pgfqpoint{9.488635in}{2.220432in}}%
\pgfpathlineto{\pgfqpoint{9.674862in}{2.220432in}}%
\pgfpathlineto{\pgfqpoint{9.674862in}{2.302160in}}%
\pgfpathlineto{\pgfqpoint{9.488635in}{2.302160in}}%
\pgfpathlineto{\pgfqpoint{9.488635in}{2.220432in}}%
\pgfusepath{}%
\end{pgfscope}%
\begin{pgfscope}%
\pgfpathrectangle{\pgfqpoint{0.549740in}{0.463273in}}{\pgfqpoint{9.320225in}{4.495057in}}%
\pgfusepath{clip}%
\pgfsetbuttcap%
\pgfsetroundjoin%
\definecolor{currentfill}{rgb}{0.472869,0.711325,0.955316}%
\pgfsetfillcolor{currentfill}%
\pgfsetlinewidth{0.000000pt}%
\definecolor{currentstroke}{rgb}{0.000000,0.000000,0.000000}%
\pgfsetstrokecolor{currentstroke}%
\pgfsetdash{}{0pt}%
\pgfpathmoveto{\pgfqpoint{9.674862in}{2.220432in}}%
\pgfpathlineto{\pgfqpoint{9.861088in}{2.220432in}}%
\pgfpathlineto{\pgfqpoint{9.861088in}{2.302160in}}%
\pgfpathlineto{\pgfqpoint{9.674862in}{2.302160in}}%
\pgfpathlineto{\pgfqpoint{9.674862in}{2.220432in}}%
\pgfusepath{fill}%
\end{pgfscope}%
\begin{pgfscope}%
\pgfpathrectangle{\pgfqpoint{0.549740in}{0.463273in}}{\pgfqpoint{9.320225in}{4.495057in}}%
\pgfusepath{clip}%
\pgfsetbuttcap%
\pgfsetroundjoin%
\pgfsetlinewidth{0.000000pt}%
\definecolor{currentstroke}{rgb}{0.000000,0.000000,0.000000}%
\pgfsetstrokecolor{currentstroke}%
\pgfsetdash{}{0pt}%
\pgfpathmoveto{\pgfqpoint{0.549761in}{2.302160in}}%
\pgfpathlineto{\pgfqpoint{0.735988in}{2.302160in}}%
\pgfpathlineto{\pgfqpoint{0.735988in}{2.383888in}}%
\pgfpathlineto{\pgfqpoint{0.549761in}{2.383888in}}%
\pgfpathlineto{\pgfqpoint{0.549761in}{2.302160in}}%
\pgfusepath{}%
\end{pgfscope}%
\begin{pgfscope}%
\pgfpathrectangle{\pgfqpoint{0.549740in}{0.463273in}}{\pgfqpoint{9.320225in}{4.495057in}}%
\pgfusepath{clip}%
\pgfsetbuttcap%
\pgfsetroundjoin%
\pgfsetlinewidth{0.000000pt}%
\definecolor{currentstroke}{rgb}{0.000000,0.000000,0.000000}%
\pgfsetstrokecolor{currentstroke}%
\pgfsetdash{}{0pt}%
\pgfpathmoveto{\pgfqpoint{0.735988in}{2.302160in}}%
\pgfpathlineto{\pgfqpoint{0.922214in}{2.302160in}}%
\pgfpathlineto{\pgfqpoint{0.922214in}{2.383888in}}%
\pgfpathlineto{\pgfqpoint{0.735988in}{2.383888in}}%
\pgfpathlineto{\pgfqpoint{0.735988in}{2.302160in}}%
\pgfusepath{}%
\end{pgfscope}%
\begin{pgfscope}%
\pgfpathrectangle{\pgfqpoint{0.549740in}{0.463273in}}{\pgfqpoint{9.320225in}{4.495057in}}%
\pgfusepath{clip}%
\pgfsetbuttcap%
\pgfsetroundjoin%
\pgfsetlinewidth{0.000000pt}%
\definecolor{currentstroke}{rgb}{0.000000,0.000000,0.000000}%
\pgfsetstrokecolor{currentstroke}%
\pgfsetdash{}{0pt}%
\pgfpathmoveto{\pgfqpoint{0.922214in}{2.302160in}}%
\pgfpathlineto{\pgfqpoint{1.108441in}{2.302160in}}%
\pgfpathlineto{\pgfqpoint{1.108441in}{2.383888in}}%
\pgfpathlineto{\pgfqpoint{0.922214in}{2.383888in}}%
\pgfpathlineto{\pgfqpoint{0.922214in}{2.302160in}}%
\pgfusepath{}%
\end{pgfscope}%
\begin{pgfscope}%
\pgfpathrectangle{\pgfqpoint{0.549740in}{0.463273in}}{\pgfqpoint{9.320225in}{4.495057in}}%
\pgfusepath{clip}%
\pgfsetbuttcap%
\pgfsetroundjoin%
\pgfsetlinewidth{0.000000pt}%
\definecolor{currentstroke}{rgb}{0.000000,0.000000,0.000000}%
\pgfsetstrokecolor{currentstroke}%
\pgfsetdash{}{0pt}%
\pgfpathmoveto{\pgfqpoint{1.108441in}{2.302160in}}%
\pgfpathlineto{\pgfqpoint{1.294667in}{2.302160in}}%
\pgfpathlineto{\pgfqpoint{1.294667in}{2.383888in}}%
\pgfpathlineto{\pgfqpoint{1.108441in}{2.383888in}}%
\pgfpathlineto{\pgfqpoint{1.108441in}{2.302160in}}%
\pgfusepath{}%
\end{pgfscope}%
\begin{pgfscope}%
\pgfpathrectangle{\pgfqpoint{0.549740in}{0.463273in}}{\pgfqpoint{9.320225in}{4.495057in}}%
\pgfusepath{clip}%
\pgfsetbuttcap%
\pgfsetroundjoin%
\pgfsetlinewidth{0.000000pt}%
\definecolor{currentstroke}{rgb}{0.000000,0.000000,0.000000}%
\pgfsetstrokecolor{currentstroke}%
\pgfsetdash{}{0pt}%
\pgfpathmoveto{\pgfqpoint{1.294667in}{2.302160in}}%
\pgfpathlineto{\pgfqpoint{1.480894in}{2.302160in}}%
\pgfpathlineto{\pgfqpoint{1.480894in}{2.383888in}}%
\pgfpathlineto{\pgfqpoint{1.294667in}{2.383888in}}%
\pgfpathlineto{\pgfqpoint{1.294667in}{2.302160in}}%
\pgfusepath{}%
\end{pgfscope}%
\begin{pgfscope}%
\pgfpathrectangle{\pgfqpoint{0.549740in}{0.463273in}}{\pgfqpoint{9.320225in}{4.495057in}}%
\pgfusepath{clip}%
\pgfsetbuttcap%
\pgfsetroundjoin%
\pgfsetlinewidth{0.000000pt}%
\definecolor{currentstroke}{rgb}{0.000000,0.000000,0.000000}%
\pgfsetstrokecolor{currentstroke}%
\pgfsetdash{}{0pt}%
\pgfpathmoveto{\pgfqpoint{1.480894in}{2.302160in}}%
\pgfpathlineto{\pgfqpoint{1.667120in}{2.302160in}}%
\pgfpathlineto{\pgfqpoint{1.667120in}{2.383888in}}%
\pgfpathlineto{\pgfqpoint{1.480894in}{2.383888in}}%
\pgfpathlineto{\pgfqpoint{1.480894in}{2.302160in}}%
\pgfusepath{}%
\end{pgfscope}%
\begin{pgfscope}%
\pgfpathrectangle{\pgfqpoint{0.549740in}{0.463273in}}{\pgfqpoint{9.320225in}{4.495057in}}%
\pgfusepath{clip}%
\pgfsetbuttcap%
\pgfsetroundjoin%
\pgfsetlinewidth{0.000000pt}%
\definecolor{currentstroke}{rgb}{0.000000,0.000000,0.000000}%
\pgfsetstrokecolor{currentstroke}%
\pgfsetdash{}{0pt}%
\pgfpathmoveto{\pgfqpoint{1.667120in}{2.302160in}}%
\pgfpathlineto{\pgfqpoint{1.853347in}{2.302160in}}%
\pgfpathlineto{\pgfqpoint{1.853347in}{2.383888in}}%
\pgfpathlineto{\pgfqpoint{1.667120in}{2.383888in}}%
\pgfpathlineto{\pgfqpoint{1.667120in}{2.302160in}}%
\pgfusepath{}%
\end{pgfscope}%
\begin{pgfscope}%
\pgfpathrectangle{\pgfqpoint{0.549740in}{0.463273in}}{\pgfqpoint{9.320225in}{4.495057in}}%
\pgfusepath{clip}%
\pgfsetbuttcap%
\pgfsetroundjoin%
\pgfsetlinewidth{0.000000pt}%
\definecolor{currentstroke}{rgb}{0.000000,0.000000,0.000000}%
\pgfsetstrokecolor{currentstroke}%
\pgfsetdash{}{0pt}%
\pgfpathmoveto{\pgfqpoint{1.853347in}{2.302160in}}%
\pgfpathlineto{\pgfqpoint{2.039573in}{2.302160in}}%
\pgfpathlineto{\pgfqpoint{2.039573in}{2.383888in}}%
\pgfpathlineto{\pgfqpoint{1.853347in}{2.383888in}}%
\pgfpathlineto{\pgfqpoint{1.853347in}{2.302160in}}%
\pgfusepath{}%
\end{pgfscope}%
\begin{pgfscope}%
\pgfpathrectangle{\pgfqpoint{0.549740in}{0.463273in}}{\pgfqpoint{9.320225in}{4.495057in}}%
\pgfusepath{clip}%
\pgfsetbuttcap%
\pgfsetroundjoin%
\pgfsetlinewidth{0.000000pt}%
\definecolor{currentstroke}{rgb}{0.000000,0.000000,0.000000}%
\pgfsetstrokecolor{currentstroke}%
\pgfsetdash{}{0pt}%
\pgfpathmoveto{\pgfqpoint{2.039573in}{2.302160in}}%
\pgfpathlineto{\pgfqpoint{2.225800in}{2.302160in}}%
\pgfpathlineto{\pgfqpoint{2.225800in}{2.383888in}}%
\pgfpathlineto{\pgfqpoint{2.039573in}{2.383888in}}%
\pgfpathlineto{\pgfqpoint{2.039573in}{2.302160in}}%
\pgfusepath{}%
\end{pgfscope}%
\begin{pgfscope}%
\pgfpathrectangle{\pgfqpoint{0.549740in}{0.463273in}}{\pgfqpoint{9.320225in}{4.495057in}}%
\pgfusepath{clip}%
\pgfsetbuttcap%
\pgfsetroundjoin%
\pgfsetlinewidth{0.000000pt}%
\definecolor{currentstroke}{rgb}{0.000000,0.000000,0.000000}%
\pgfsetstrokecolor{currentstroke}%
\pgfsetdash{}{0pt}%
\pgfpathmoveto{\pgfqpoint{2.225800in}{2.302160in}}%
\pgfpathlineto{\pgfqpoint{2.412027in}{2.302160in}}%
\pgfpathlineto{\pgfqpoint{2.412027in}{2.383888in}}%
\pgfpathlineto{\pgfqpoint{2.225800in}{2.383888in}}%
\pgfpathlineto{\pgfqpoint{2.225800in}{2.302160in}}%
\pgfusepath{}%
\end{pgfscope}%
\begin{pgfscope}%
\pgfpathrectangle{\pgfqpoint{0.549740in}{0.463273in}}{\pgfqpoint{9.320225in}{4.495057in}}%
\pgfusepath{clip}%
\pgfsetbuttcap%
\pgfsetroundjoin%
\pgfsetlinewidth{0.000000pt}%
\definecolor{currentstroke}{rgb}{0.000000,0.000000,0.000000}%
\pgfsetstrokecolor{currentstroke}%
\pgfsetdash{}{0pt}%
\pgfpathmoveto{\pgfqpoint{2.412027in}{2.302160in}}%
\pgfpathlineto{\pgfqpoint{2.598253in}{2.302160in}}%
\pgfpathlineto{\pgfqpoint{2.598253in}{2.383888in}}%
\pgfpathlineto{\pgfqpoint{2.412027in}{2.383888in}}%
\pgfpathlineto{\pgfqpoint{2.412027in}{2.302160in}}%
\pgfusepath{}%
\end{pgfscope}%
\begin{pgfscope}%
\pgfpathrectangle{\pgfqpoint{0.549740in}{0.463273in}}{\pgfqpoint{9.320225in}{4.495057in}}%
\pgfusepath{clip}%
\pgfsetbuttcap%
\pgfsetroundjoin%
\pgfsetlinewidth{0.000000pt}%
\definecolor{currentstroke}{rgb}{0.000000,0.000000,0.000000}%
\pgfsetstrokecolor{currentstroke}%
\pgfsetdash{}{0pt}%
\pgfpathmoveto{\pgfqpoint{2.598253in}{2.302160in}}%
\pgfpathlineto{\pgfqpoint{2.784480in}{2.302160in}}%
\pgfpathlineto{\pgfqpoint{2.784480in}{2.383888in}}%
\pgfpathlineto{\pgfqpoint{2.598253in}{2.383888in}}%
\pgfpathlineto{\pgfqpoint{2.598253in}{2.302160in}}%
\pgfusepath{}%
\end{pgfscope}%
\begin{pgfscope}%
\pgfpathrectangle{\pgfqpoint{0.549740in}{0.463273in}}{\pgfqpoint{9.320225in}{4.495057in}}%
\pgfusepath{clip}%
\pgfsetbuttcap%
\pgfsetroundjoin%
\pgfsetlinewidth{0.000000pt}%
\definecolor{currentstroke}{rgb}{0.000000,0.000000,0.000000}%
\pgfsetstrokecolor{currentstroke}%
\pgfsetdash{}{0pt}%
\pgfpathmoveto{\pgfqpoint{2.784480in}{2.302160in}}%
\pgfpathlineto{\pgfqpoint{2.970706in}{2.302160in}}%
\pgfpathlineto{\pgfqpoint{2.970706in}{2.383888in}}%
\pgfpathlineto{\pgfqpoint{2.784480in}{2.383888in}}%
\pgfpathlineto{\pgfqpoint{2.784480in}{2.302160in}}%
\pgfusepath{}%
\end{pgfscope}%
\begin{pgfscope}%
\pgfpathrectangle{\pgfqpoint{0.549740in}{0.463273in}}{\pgfqpoint{9.320225in}{4.495057in}}%
\pgfusepath{clip}%
\pgfsetbuttcap%
\pgfsetroundjoin%
\pgfsetlinewidth{0.000000pt}%
\definecolor{currentstroke}{rgb}{0.000000,0.000000,0.000000}%
\pgfsetstrokecolor{currentstroke}%
\pgfsetdash{}{0pt}%
\pgfpathmoveto{\pgfqpoint{2.970706in}{2.302160in}}%
\pgfpathlineto{\pgfqpoint{3.156933in}{2.302160in}}%
\pgfpathlineto{\pgfqpoint{3.156933in}{2.383888in}}%
\pgfpathlineto{\pgfqpoint{2.970706in}{2.383888in}}%
\pgfpathlineto{\pgfqpoint{2.970706in}{2.302160in}}%
\pgfusepath{}%
\end{pgfscope}%
\begin{pgfscope}%
\pgfpathrectangle{\pgfqpoint{0.549740in}{0.463273in}}{\pgfqpoint{9.320225in}{4.495057in}}%
\pgfusepath{clip}%
\pgfsetbuttcap%
\pgfsetroundjoin%
\pgfsetlinewidth{0.000000pt}%
\definecolor{currentstroke}{rgb}{0.000000,0.000000,0.000000}%
\pgfsetstrokecolor{currentstroke}%
\pgfsetdash{}{0pt}%
\pgfpathmoveto{\pgfqpoint{3.156933in}{2.302160in}}%
\pgfpathlineto{\pgfqpoint{3.343159in}{2.302160in}}%
\pgfpathlineto{\pgfqpoint{3.343159in}{2.383888in}}%
\pgfpathlineto{\pgfqpoint{3.156933in}{2.383888in}}%
\pgfpathlineto{\pgfqpoint{3.156933in}{2.302160in}}%
\pgfusepath{}%
\end{pgfscope}%
\begin{pgfscope}%
\pgfpathrectangle{\pgfqpoint{0.549740in}{0.463273in}}{\pgfqpoint{9.320225in}{4.495057in}}%
\pgfusepath{clip}%
\pgfsetbuttcap%
\pgfsetroundjoin%
\pgfsetlinewidth{0.000000pt}%
\definecolor{currentstroke}{rgb}{0.000000,0.000000,0.000000}%
\pgfsetstrokecolor{currentstroke}%
\pgfsetdash{}{0pt}%
\pgfpathmoveto{\pgfqpoint{3.343159in}{2.302160in}}%
\pgfpathlineto{\pgfqpoint{3.529386in}{2.302160in}}%
\pgfpathlineto{\pgfqpoint{3.529386in}{2.383888in}}%
\pgfpathlineto{\pgfqpoint{3.343159in}{2.383888in}}%
\pgfpathlineto{\pgfqpoint{3.343159in}{2.302160in}}%
\pgfusepath{}%
\end{pgfscope}%
\begin{pgfscope}%
\pgfpathrectangle{\pgfqpoint{0.549740in}{0.463273in}}{\pgfqpoint{9.320225in}{4.495057in}}%
\pgfusepath{clip}%
\pgfsetbuttcap%
\pgfsetroundjoin%
\pgfsetlinewidth{0.000000pt}%
\definecolor{currentstroke}{rgb}{0.000000,0.000000,0.000000}%
\pgfsetstrokecolor{currentstroke}%
\pgfsetdash{}{0pt}%
\pgfpathmoveto{\pgfqpoint{3.529386in}{2.302160in}}%
\pgfpathlineto{\pgfqpoint{3.715612in}{2.302160in}}%
\pgfpathlineto{\pgfqpoint{3.715612in}{2.383888in}}%
\pgfpathlineto{\pgfqpoint{3.529386in}{2.383888in}}%
\pgfpathlineto{\pgfqpoint{3.529386in}{2.302160in}}%
\pgfusepath{}%
\end{pgfscope}%
\begin{pgfscope}%
\pgfpathrectangle{\pgfqpoint{0.549740in}{0.463273in}}{\pgfqpoint{9.320225in}{4.495057in}}%
\pgfusepath{clip}%
\pgfsetbuttcap%
\pgfsetroundjoin%
\pgfsetlinewidth{0.000000pt}%
\definecolor{currentstroke}{rgb}{0.000000,0.000000,0.000000}%
\pgfsetstrokecolor{currentstroke}%
\pgfsetdash{}{0pt}%
\pgfpathmoveto{\pgfqpoint{3.715612in}{2.302160in}}%
\pgfpathlineto{\pgfqpoint{3.901839in}{2.302160in}}%
\pgfpathlineto{\pgfqpoint{3.901839in}{2.383888in}}%
\pgfpathlineto{\pgfqpoint{3.715612in}{2.383888in}}%
\pgfpathlineto{\pgfqpoint{3.715612in}{2.302160in}}%
\pgfusepath{}%
\end{pgfscope}%
\begin{pgfscope}%
\pgfpathrectangle{\pgfqpoint{0.549740in}{0.463273in}}{\pgfqpoint{9.320225in}{4.495057in}}%
\pgfusepath{clip}%
\pgfsetbuttcap%
\pgfsetroundjoin%
\pgfsetlinewidth{0.000000pt}%
\definecolor{currentstroke}{rgb}{0.000000,0.000000,0.000000}%
\pgfsetstrokecolor{currentstroke}%
\pgfsetdash{}{0pt}%
\pgfpathmoveto{\pgfqpoint{3.901839in}{2.302160in}}%
\pgfpathlineto{\pgfqpoint{4.088065in}{2.302160in}}%
\pgfpathlineto{\pgfqpoint{4.088065in}{2.383888in}}%
\pgfpathlineto{\pgfqpoint{3.901839in}{2.383888in}}%
\pgfpathlineto{\pgfqpoint{3.901839in}{2.302160in}}%
\pgfusepath{}%
\end{pgfscope}%
\begin{pgfscope}%
\pgfpathrectangle{\pgfqpoint{0.549740in}{0.463273in}}{\pgfqpoint{9.320225in}{4.495057in}}%
\pgfusepath{clip}%
\pgfsetbuttcap%
\pgfsetroundjoin%
\pgfsetlinewidth{0.000000pt}%
\definecolor{currentstroke}{rgb}{0.000000,0.000000,0.000000}%
\pgfsetstrokecolor{currentstroke}%
\pgfsetdash{}{0pt}%
\pgfpathmoveto{\pgfqpoint{4.088065in}{2.302160in}}%
\pgfpathlineto{\pgfqpoint{4.274292in}{2.302160in}}%
\pgfpathlineto{\pgfqpoint{4.274292in}{2.383888in}}%
\pgfpathlineto{\pgfqpoint{4.088065in}{2.383888in}}%
\pgfpathlineto{\pgfqpoint{4.088065in}{2.302160in}}%
\pgfusepath{}%
\end{pgfscope}%
\begin{pgfscope}%
\pgfpathrectangle{\pgfqpoint{0.549740in}{0.463273in}}{\pgfqpoint{9.320225in}{4.495057in}}%
\pgfusepath{clip}%
\pgfsetbuttcap%
\pgfsetroundjoin%
\pgfsetlinewidth{0.000000pt}%
\definecolor{currentstroke}{rgb}{0.000000,0.000000,0.000000}%
\pgfsetstrokecolor{currentstroke}%
\pgfsetdash{}{0pt}%
\pgfpathmoveto{\pgfqpoint{4.274292in}{2.302160in}}%
\pgfpathlineto{\pgfqpoint{4.460519in}{2.302160in}}%
\pgfpathlineto{\pgfqpoint{4.460519in}{2.383888in}}%
\pgfpathlineto{\pgfqpoint{4.274292in}{2.383888in}}%
\pgfpathlineto{\pgfqpoint{4.274292in}{2.302160in}}%
\pgfusepath{}%
\end{pgfscope}%
\begin{pgfscope}%
\pgfpathrectangle{\pgfqpoint{0.549740in}{0.463273in}}{\pgfqpoint{9.320225in}{4.495057in}}%
\pgfusepath{clip}%
\pgfsetbuttcap%
\pgfsetroundjoin%
\pgfsetlinewidth{0.000000pt}%
\definecolor{currentstroke}{rgb}{0.000000,0.000000,0.000000}%
\pgfsetstrokecolor{currentstroke}%
\pgfsetdash{}{0pt}%
\pgfpathmoveto{\pgfqpoint{4.460519in}{2.302160in}}%
\pgfpathlineto{\pgfqpoint{4.646745in}{2.302160in}}%
\pgfpathlineto{\pgfqpoint{4.646745in}{2.383888in}}%
\pgfpathlineto{\pgfqpoint{4.460519in}{2.383888in}}%
\pgfpathlineto{\pgfqpoint{4.460519in}{2.302160in}}%
\pgfusepath{}%
\end{pgfscope}%
\begin{pgfscope}%
\pgfpathrectangle{\pgfqpoint{0.549740in}{0.463273in}}{\pgfqpoint{9.320225in}{4.495057in}}%
\pgfusepath{clip}%
\pgfsetbuttcap%
\pgfsetroundjoin%
\pgfsetlinewidth{0.000000pt}%
\definecolor{currentstroke}{rgb}{0.000000,0.000000,0.000000}%
\pgfsetstrokecolor{currentstroke}%
\pgfsetdash{}{0pt}%
\pgfpathmoveto{\pgfqpoint{4.646745in}{2.302160in}}%
\pgfpathlineto{\pgfqpoint{4.832972in}{2.302160in}}%
\pgfpathlineto{\pgfqpoint{4.832972in}{2.383888in}}%
\pgfpathlineto{\pgfqpoint{4.646745in}{2.383888in}}%
\pgfpathlineto{\pgfqpoint{4.646745in}{2.302160in}}%
\pgfusepath{}%
\end{pgfscope}%
\begin{pgfscope}%
\pgfpathrectangle{\pgfqpoint{0.549740in}{0.463273in}}{\pgfqpoint{9.320225in}{4.495057in}}%
\pgfusepath{clip}%
\pgfsetbuttcap%
\pgfsetroundjoin%
\pgfsetlinewidth{0.000000pt}%
\definecolor{currentstroke}{rgb}{0.000000,0.000000,0.000000}%
\pgfsetstrokecolor{currentstroke}%
\pgfsetdash{}{0pt}%
\pgfpathmoveto{\pgfqpoint{4.832972in}{2.302160in}}%
\pgfpathlineto{\pgfqpoint{5.019198in}{2.302160in}}%
\pgfpathlineto{\pgfqpoint{5.019198in}{2.383888in}}%
\pgfpathlineto{\pgfqpoint{4.832972in}{2.383888in}}%
\pgfpathlineto{\pgfqpoint{4.832972in}{2.302160in}}%
\pgfusepath{}%
\end{pgfscope}%
\begin{pgfscope}%
\pgfpathrectangle{\pgfqpoint{0.549740in}{0.463273in}}{\pgfqpoint{9.320225in}{4.495057in}}%
\pgfusepath{clip}%
\pgfsetbuttcap%
\pgfsetroundjoin%
\pgfsetlinewidth{0.000000pt}%
\definecolor{currentstroke}{rgb}{0.000000,0.000000,0.000000}%
\pgfsetstrokecolor{currentstroke}%
\pgfsetdash{}{0pt}%
\pgfpathmoveto{\pgfqpoint{5.019198in}{2.302160in}}%
\pgfpathlineto{\pgfqpoint{5.205425in}{2.302160in}}%
\pgfpathlineto{\pgfqpoint{5.205425in}{2.383888in}}%
\pgfpathlineto{\pgfqpoint{5.019198in}{2.383888in}}%
\pgfpathlineto{\pgfqpoint{5.019198in}{2.302160in}}%
\pgfusepath{}%
\end{pgfscope}%
\begin{pgfscope}%
\pgfpathrectangle{\pgfqpoint{0.549740in}{0.463273in}}{\pgfqpoint{9.320225in}{4.495057in}}%
\pgfusepath{clip}%
\pgfsetbuttcap%
\pgfsetroundjoin%
\pgfsetlinewidth{0.000000pt}%
\definecolor{currentstroke}{rgb}{0.000000,0.000000,0.000000}%
\pgfsetstrokecolor{currentstroke}%
\pgfsetdash{}{0pt}%
\pgfpathmoveto{\pgfqpoint{5.205425in}{2.302160in}}%
\pgfpathlineto{\pgfqpoint{5.391651in}{2.302160in}}%
\pgfpathlineto{\pgfqpoint{5.391651in}{2.383888in}}%
\pgfpathlineto{\pgfqpoint{5.205425in}{2.383888in}}%
\pgfpathlineto{\pgfqpoint{5.205425in}{2.302160in}}%
\pgfusepath{}%
\end{pgfscope}%
\begin{pgfscope}%
\pgfpathrectangle{\pgfqpoint{0.549740in}{0.463273in}}{\pgfqpoint{9.320225in}{4.495057in}}%
\pgfusepath{clip}%
\pgfsetbuttcap%
\pgfsetroundjoin%
\definecolor{currentfill}{rgb}{0.472869,0.711325,0.955316}%
\pgfsetfillcolor{currentfill}%
\pgfsetlinewidth{0.000000pt}%
\definecolor{currentstroke}{rgb}{0.000000,0.000000,0.000000}%
\pgfsetstrokecolor{currentstroke}%
\pgfsetdash{}{0pt}%
\pgfpathmoveto{\pgfqpoint{5.391651in}{2.302160in}}%
\pgfpathlineto{\pgfqpoint{5.577878in}{2.302160in}}%
\pgfpathlineto{\pgfqpoint{5.577878in}{2.383888in}}%
\pgfpathlineto{\pgfqpoint{5.391651in}{2.383888in}}%
\pgfpathlineto{\pgfqpoint{5.391651in}{2.302160in}}%
\pgfusepath{fill}%
\end{pgfscope}%
\begin{pgfscope}%
\pgfpathrectangle{\pgfqpoint{0.549740in}{0.463273in}}{\pgfqpoint{9.320225in}{4.495057in}}%
\pgfusepath{clip}%
\pgfsetbuttcap%
\pgfsetroundjoin%
\pgfsetlinewidth{0.000000pt}%
\definecolor{currentstroke}{rgb}{0.000000,0.000000,0.000000}%
\pgfsetstrokecolor{currentstroke}%
\pgfsetdash{}{0pt}%
\pgfpathmoveto{\pgfqpoint{5.577878in}{2.302160in}}%
\pgfpathlineto{\pgfqpoint{5.764104in}{2.302160in}}%
\pgfpathlineto{\pgfqpoint{5.764104in}{2.383888in}}%
\pgfpathlineto{\pgfqpoint{5.577878in}{2.383888in}}%
\pgfpathlineto{\pgfqpoint{5.577878in}{2.302160in}}%
\pgfusepath{}%
\end{pgfscope}%
\begin{pgfscope}%
\pgfpathrectangle{\pgfqpoint{0.549740in}{0.463273in}}{\pgfqpoint{9.320225in}{4.495057in}}%
\pgfusepath{clip}%
\pgfsetbuttcap%
\pgfsetroundjoin%
\pgfsetlinewidth{0.000000pt}%
\definecolor{currentstroke}{rgb}{0.000000,0.000000,0.000000}%
\pgfsetstrokecolor{currentstroke}%
\pgfsetdash{}{0pt}%
\pgfpathmoveto{\pgfqpoint{5.764104in}{2.302160in}}%
\pgfpathlineto{\pgfqpoint{5.950331in}{2.302160in}}%
\pgfpathlineto{\pgfqpoint{5.950331in}{2.383888in}}%
\pgfpathlineto{\pgfqpoint{5.764104in}{2.383888in}}%
\pgfpathlineto{\pgfqpoint{5.764104in}{2.302160in}}%
\pgfusepath{}%
\end{pgfscope}%
\begin{pgfscope}%
\pgfpathrectangle{\pgfqpoint{0.549740in}{0.463273in}}{\pgfqpoint{9.320225in}{4.495057in}}%
\pgfusepath{clip}%
\pgfsetbuttcap%
\pgfsetroundjoin%
\pgfsetlinewidth{0.000000pt}%
\definecolor{currentstroke}{rgb}{0.000000,0.000000,0.000000}%
\pgfsetstrokecolor{currentstroke}%
\pgfsetdash{}{0pt}%
\pgfpathmoveto{\pgfqpoint{5.950331in}{2.302160in}}%
\pgfpathlineto{\pgfqpoint{6.136557in}{2.302160in}}%
\pgfpathlineto{\pgfqpoint{6.136557in}{2.383888in}}%
\pgfpathlineto{\pgfqpoint{5.950331in}{2.383888in}}%
\pgfpathlineto{\pgfqpoint{5.950331in}{2.302160in}}%
\pgfusepath{}%
\end{pgfscope}%
\begin{pgfscope}%
\pgfpathrectangle{\pgfqpoint{0.549740in}{0.463273in}}{\pgfqpoint{9.320225in}{4.495057in}}%
\pgfusepath{clip}%
\pgfsetbuttcap%
\pgfsetroundjoin%
\pgfsetlinewidth{0.000000pt}%
\definecolor{currentstroke}{rgb}{0.000000,0.000000,0.000000}%
\pgfsetstrokecolor{currentstroke}%
\pgfsetdash{}{0pt}%
\pgfpathmoveto{\pgfqpoint{6.136557in}{2.302160in}}%
\pgfpathlineto{\pgfqpoint{6.322784in}{2.302160in}}%
\pgfpathlineto{\pgfqpoint{6.322784in}{2.383888in}}%
\pgfpathlineto{\pgfqpoint{6.136557in}{2.383888in}}%
\pgfpathlineto{\pgfqpoint{6.136557in}{2.302160in}}%
\pgfusepath{}%
\end{pgfscope}%
\begin{pgfscope}%
\pgfpathrectangle{\pgfqpoint{0.549740in}{0.463273in}}{\pgfqpoint{9.320225in}{4.495057in}}%
\pgfusepath{clip}%
\pgfsetbuttcap%
\pgfsetroundjoin%
\definecolor{currentfill}{rgb}{0.472869,0.711325,0.955316}%
\pgfsetfillcolor{currentfill}%
\pgfsetlinewidth{0.000000pt}%
\definecolor{currentstroke}{rgb}{0.000000,0.000000,0.000000}%
\pgfsetstrokecolor{currentstroke}%
\pgfsetdash{}{0pt}%
\pgfpathmoveto{\pgfqpoint{6.322784in}{2.302160in}}%
\pgfpathlineto{\pgfqpoint{6.509011in}{2.302160in}}%
\pgfpathlineto{\pgfqpoint{6.509011in}{2.383888in}}%
\pgfpathlineto{\pgfqpoint{6.322784in}{2.383888in}}%
\pgfpathlineto{\pgfqpoint{6.322784in}{2.302160in}}%
\pgfusepath{fill}%
\end{pgfscope}%
\begin{pgfscope}%
\pgfpathrectangle{\pgfqpoint{0.549740in}{0.463273in}}{\pgfqpoint{9.320225in}{4.495057in}}%
\pgfusepath{clip}%
\pgfsetbuttcap%
\pgfsetroundjoin%
\pgfsetlinewidth{0.000000pt}%
\definecolor{currentstroke}{rgb}{0.000000,0.000000,0.000000}%
\pgfsetstrokecolor{currentstroke}%
\pgfsetdash{}{0pt}%
\pgfpathmoveto{\pgfqpoint{6.509011in}{2.302160in}}%
\pgfpathlineto{\pgfqpoint{6.695237in}{2.302160in}}%
\pgfpathlineto{\pgfqpoint{6.695237in}{2.383888in}}%
\pgfpathlineto{\pgfqpoint{6.509011in}{2.383888in}}%
\pgfpathlineto{\pgfqpoint{6.509011in}{2.302160in}}%
\pgfusepath{}%
\end{pgfscope}%
\begin{pgfscope}%
\pgfpathrectangle{\pgfqpoint{0.549740in}{0.463273in}}{\pgfqpoint{9.320225in}{4.495057in}}%
\pgfusepath{clip}%
\pgfsetbuttcap%
\pgfsetroundjoin%
\pgfsetlinewidth{0.000000pt}%
\definecolor{currentstroke}{rgb}{0.000000,0.000000,0.000000}%
\pgfsetstrokecolor{currentstroke}%
\pgfsetdash{}{0pt}%
\pgfpathmoveto{\pgfqpoint{6.695237in}{2.302160in}}%
\pgfpathlineto{\pgfqpoint{6.881464in}{2.302160in}}%
\pgfpathlineto{\pgfqpoint{6.881464in}{2.383888in}}%
\pgfpathlineto{\pgfqpoint{6.695237in}{2.383888in}}%
\pgfpathlineto{\pgfqpoint{6.695237in}{2.302160in}}%
\pgfusepath{}%
\end{pgfscope}%
\begin{pgfscope}%
\pgfpathrectangle{\pgfqpoint{0.549740in}{0.463273in}}{\pgfqpoint{9.320225in}{4.495057in}}%
\pgfusepath{clip}%
\pgfsetbuttcap%
\pgfsetroundjoin%
\pgfsetlinewidth{0.000000pt}%
\definecolor{currentstroke}{rgb}{0.000000,0.000000,0.000000}%
\pgfsetstrokecolor{currentstroke}%
\pgfsetdash{}{0pt}%
\pgfpathmoveto{\pgfqpoint{6.881464in}{2.302160in}}%
\pgfpathlineto{\pgfqpoint{7.067690in}{2.302160in}}%
\pgfpathlineto{\pgfqpoint{7.067690in}{2.383888in}}%
\pgfpathlineto{\pgfqpoint{6.881464in}{2.383888in}}%
\pgfpathlineto{\pgfqpoint{6.881464in}{2.302160in}}%
\pgfusepath{}%
\end{pgfscope}%
\begin{pgfscope}%
\pgfpathrectangle{\pgfqpoint{0.549740in}{0.463273in}}{\pgfqpoint{9.320225in}{4.495057in}}%
\pgfusepath{clip}%
\pgfsetbuttcap%
\pgfsetroundjoin%
\pgfsetlinewidth{0.000000pt}%
\definecolor{currentstroke}{rgb}{0.000000,0.000000,0.000000}%
\pgfsetstrokecolor{currentstroke}%
\pgfsetdash{}{0pt}%
\pgfpathmoveto{\pgfqpoint{7.067690in}{2.302160in}}%
\pgfpathlineto{\pgfqpoint{7.253917in}{2.302160in}}%
\pgfpathlineto{\pgfqpoint{7.253917in}{2.383888in}}%
\pgfpathlineto{\pgfqpoint{7.067690in}{2.383888in}}%
\pgfpathlineto{\pgfqpoint{7.067690in}{2.302160in}}%
\pgfusepath{}%
\end{pgfscope}%
\begin{pgfscope}%
\pgfpathrectangle{\pgfqpoint{0.549740in}{0.463273in}}{\pgfqpoint{9.320225in}{4.495057in}}%
\pgfusepath{clip}%
\pgfsetbuttcap%
\pgfsetroundjoin%
\definecolor{currentfill}{rgb}{0.472869,0.711325,0.955316}%
\pgfsetfillcolor{currentfill}%
\pgfsetlinewidth{0.000000pt}%
\definecolor{currentstroke}{rgb}{0.000000,0.000000,0.000000}%
\pgfsetstrokecolor{currentstroke}%
\pgfsetdash{}{0pt}%
\pgfpathmoveto{\pgfqpoint{7.253917in}{2.302160in}}%
\pgfpathlineto{\pgfqpoint{7.440143in}{2.302160in}}%
\pgfpathlineto{\pgfqpoint{7.440143in}{2.383888in}}%
\pgfpathlineto{\pgfqpoint{7.253917in}{2.383888in}}%
\pgfpathlineto{\pgfqpoint{7.253917in}{2.302160in}}%
\pgfusepath{fill}%
\end{pgfscope}%
\begin{pgfscope}%
\pgfpathrectangle{\pgfqpoint{0.549740in}{0.463273in}}{\pgfqpoint{9.320225in}{4.495057in}}%
\pgfusepath{clip}%
\pgfsetbuttcap%
\pgfsetroundjoin%
\pgfsetlinewidth{0.000000pt}%
\definecolor{currentstroke}{rgb}{0.000000,0.000000,0.000000}%
\pgfsetstrokecolor{currentstroke}%
\pgfsetdash{}{0pt}%
\pgfpathmoveto{\pgfqpoint{7.440143in}{2.302160in}}%
\pgfpathlineto{\pgfqpoint{7.626370in}{2.302160in}}%
\pgfpathlineto{\pgfqpoint{7.626370in}{2.383888in}}%
\pgfpathlineto{\pgfqpoint{7.440143in}{2.383888in}}%
\pgfpathlineto{\pgfqpoint{7.440143in}{2.302160in}}%
\pgfusepath{}%
\end{pgfscope}%
\begin{pgfscope}%
\pgfpathrectangle{\pgfqpoint{0.549740in}{0.463273in}}{\pgfqpoint{9.320225in}{4.495057in}}%
\pgfusepath{clip}%
\pgfsetbuttcap%
\pgfsetroundjoin%
\pgfsetlinewidth{0.000000pt}%
\definecolor{currentstroke}{rgb}{0.000000,0.000000,0.000000}%
\pgfsetstrokecolor{currentstroke}%
\pgfsetdash{}{0pt}%
\pgfpathmoveto{\pgfqpoint{7.626370in}{2.302160in}}%
\pgfpathlineto{\pgfqpoint{7.812596in}{2.302160in}}%
\pgfpathlineto{\pgfqpoint{7.812596in}{2.383888in}}%
\pgfpathlineto{\pgfqpoint{7.626370in}{2.383888in}}%
\pgfpathlineto{\pgfqpoint{7.626370in}{2.302160in}}%
\pgfusepath{}%
\end{pgfscope}%
\begin{pgfscope}%
\pgfpathrectangle{\pgfqpoint{0.549740in}{0.463273in}}{\pgfqpoint{9.320225in}{4.495057in}}%
\pgfusepath{clip}%
\pgfsetbuttcap%
\pgfsetroundjoin%
\pgfsetlinewidth{0.000000pt}%
\definecolor{currentstroke}{rgb}{0.000000,0.000000,0.000000}%
\pgfsetstrokecolor{currentstroke}%
\pgfsetdash{}{0pt}%
\pgfpathmoveto{\pgfqpoint{7.812596in}{2.302160in}}%
\pgfpathlineto{\pgfqpoint{7.998823in}{2.302160in}}%
\pgfpathlineto{\pgfqpoint{7.998823in}{2.383888in}}%
\pgfpathlineto{\pgfqpoint{7.812596in}{2.383888in}}%
\pgfpathlineto{\pgfqpoint{7.812596in}{2.302160in}}%
\pgfusepath{}%
\end{pgfscope}%
\begin{pgfscope}%
\pgfpathrectangle{\pgfqpoint{0.549740in}{0.463273in}}{\pgfqpoint{9.320225in}{4.495057in}}%
\pgfusepath{clip}%
\pgfsetbuttcap%
\pgfsetroundjoin%
\pgfsetlinewidth{0.000000pt}%
\definecolor{currentstroke}{rgb}{0.000000,0.000000,0.000000}%
\pgfsetstrokecolor{currentstroke}%
\pgfsetdash{}{0pt}%
\pgfpathmoveto{\pgfqpoint{7.998823in}{2.302160in}}%
\pgfpathlineto{\pgfqpoint{8.185049in}{2.302160in}}%
\pgfpathlineto{\pgfqpoint{8.185049in}{2.383888in}}%
\pgfpathlineto{\pgfqpoint{7.998823in}{2.383888in}}%
\pgfpathlineto{\pgfqpoint{7.998823in}{2.302160in}}%
\pgfusepath{}%
\end{pgfscope}%
\begin{pgfscope}%
\pgfpathrectangle{\pgfqpoint{0.549740in}{0.463273in}}{\pgfqpoint{9.320225in}{4.495057in}}%
\pgfusepath{clip}%
\pgfsetbuttcap%
\pgfsetroundjoin%
\pgfsetlinewidth{0.000000pt}%
\definecolor{currentstroke}{rgb}{0.000000,0.000000,0.000000}%
\pgfsetstrokecolor{currentstroke}%
\pgfsetdash{}{0pt}%
\pgfpathmoveto{\pgfqpoint{8.185049in}{2.302160in}}%
\pgfpathlineto{\pgfqpoint{8.371276in}{2.302160in}}%
\pgfpathlineto{\pgfqpoint{8.371276in}{2.383888in}}%
\pgfpathlineto{\pgfqpoint{8.185049in}{2.383888in}}%
\pgfpathlineto{\pgfqpoint{8.185049in}{2.302160in}}%
\pgfusepath{}%
\end{pgfscope}%
\begin{pgfscope}%
\pgfpathrectangle{\pgfqpoint{0.549740in}{0.463273in}}{\pgfqpoint{9.320225in}{4.495057in}}%
\pgfusepath{clip}%
\pgfsetbuttcap%
\pgfsetroundjoin%
\definecolor{currentfill}{rgb}{0.472869,0.711325,0.955316}%
\pgfsetfillcolor{currentfill}%
\pgfsetlinewidth{0.000000pt}%
\definecolor{currentstroke}{rgb}{0.000000,0.000000,0.000000}%
\pgfsetstrokecolor{currentstroke}%
\pgfsetdash{}{0pt}%
\pgfpathmoveto{\pgfqpoint{8.371276in}{2.302160in}}%
\pgfpathlineto{\pgfqpoint{8.557503in}{2.302160in}}%
\pgfpathlineto{\pgfqpoint{8.557503in}{2.383888in}}%
\pgfpathlineto{\pgfqpoint{8.371276in}{2.383888in}}%
\pgfpathlineto{\pgfqpoint{8.371276in}{2.302160in}}%
\pgfusepath{fill}%
\end{pgfscope}%
\begin{pgfscope}%
\pgfpathrectangle{\pgfqpoint{0.549740in}{0.463273in}}{\pgfqpoint{9.320225in}{4.495057in}}%
\pgfusepath{clip}%
\pgfsetbuttcap%
\pgfsetroundjoin%
\pgfsetlinewidth{0.000000pt}%
\definecolor{currentstroke}{rgb}{0.000000,0.000000,0.000000}%
\pgfsetstrokecolor{currentstroke}%
\pgfsetdash{}{0pt}%
\pgfpathmoveto{\pgfqpoint{8.557503in}{2.302160in}}%
\pgfpathlineto{\pgfqpoint{8.743729in}{2.302160in}}%
\pgfpathlineto{\pgfqpoint{8.743729in}{2.383888in}}%
\pgfpathlineto{\pgfqpoint{8.557503in}{2.383888in}}%
\pgfpathlineto{\pgfqpoint{8.557503in}{2.302160in}}%
\pgfusepath{}%
\end{pgfscope}%
\begin{pgfscope}%
\pgfpathrectangle{\pgfqpoint{0.549740in}{0.463273in}}{\pgfqpoint{9.320225in}{4.495057in}}%
\pgfusepath{clip}%
\pgfsetbuttcap%
\pgfsetroundjoin%
\pgfsetlinewidth{0.000000pt}%
\definecolor{currentstroke}{rgb}{0.000000,0.000000,0.000000}%
\pgfsetstrokecolor{currentstroke}%
\pgfsetdash{}{0pt}%
\pgfpathmoveto{\pgfqpoint{8.743729in}{2.302160in}}%
\pgfpathlineto{\pgfqpoint{8.929956in}{2.302160in}}%
\pgfpathlineto{\pgfqpoint{8.929956in}{2.383888in}}%
\pgfpathlineto{\pgfqpoint{8.743729in}{2.383888in}}%
\pgfpathlineto{\pgfqpoint{8.743729in}{2.302160in}}%
\pgfusepath{}%
\end{pgfscope}%
\begin{pgfscope}%
\pgfpathrectangle{\pgfqpoint{0.549740in}{0.463273in}}{\pgfqpoint{9.320225in}{4.495057in}}%
\pgfusepath{clip}%
\pgfsetbuttcap%
\pgfsetroundjoin%
\pgfsetlinewidth{0.000000pt}%
\definecolor{currentstroke}{rgb}{0.000000,0.000000,0.000000}%
\pgfsetstrokecolor{currentstroke}%
\pgfsetdash{}{0pt}%
\pgfpathmoveto{\pgfqpoint{8.929956in}{2.302160in}}%
\pgfpathlineto{\pgfqpoint{9.116182in}{2.302160in}}%
\pgfpathlineto{\pgfqpoint{9.116182in}{2.383888in}}%
\pgfpathlineto{\pgfqpoint{8.929956in}{2.383888in}}%
\pgfpathlineto{\pgfqpoint{8.929956in}{2.302160in}}%
\pgfusepath{}%
\end{pgfscope}%
\begin{pgfscope}%
\pgfpathrectangle{\pgfqpoint{0.549740in}{0.463273in}}{\pgfqpoint{9.320225in}{4.495057in}}%
\pgfusepath{clip}%
\pgfsetbuttcap%
\pgfsetroundjoin%
\pgfsetlinewidth{0.000000pt}%
\definecolor{currentstroke}{rgb}{0.000000,0.000000,0.000000}%
\pgfsetstrokecolor{currentstroke}%
\pgfsetdash{}{0pt}%
\pgfpathmoveto{\pgfqpoint{9.116182in}{2.302160in}}%
\pgfpathlineto{\pgfqpoint{9.302409in}{2.302160in}}%
\pgfpathlineto{\pgfqpoint{9.302409in}{2.383888in}}%
\pgfpathlineto{\pgfqpoint{9.116182in}{2.383888in}}%
\pgfpathlineto{\pgfqpoint{9.116182in}{2.302160in}}%
\pgfusepath{}%
\end{pgfscope}%
\begin{pgfscope}%
\pgfpathrectangle{\pgfqpoint{0.549740in}{0.463273in}}{\pgfqpoint{9.320225in}{4.495057in}}%
\pgfusepath{clip}%
\pgfsetbuttcap%
\pgfsetroundjoin%
\pgfsetlinewidth{0.000000pt}%
\definecolor{currentstroke}{rgb}{0.000000,0.000000,0.000000}%
\pgfsetstrokecolor{currentstroke}%
\pgfsetdash{}{0pt}%
\pgfpathmoveto{\pgfqpoint{9.302409in}{2.302160in}}%
\pgfpathlineto{\pgfqpoint{9.488635in}{2.302160in}}%
\pgfpathlineto{\pgfqpoint{9.488635in}{2.383888in}}%
\pgfpathlineto{\pgfqpoint{9.302409in}{2.383888in}}%
\pgfpathlineto{\pgfqpoint{9.302409in}{2.302160in}}%
\pgfusepath{}%
\end{pgfscope}%
\begin{pgfscope}%
\pgfpathrectangle{\pgfqpoint{0.549740in}{0.463273in}}{\pgfqpoint{9.320225in}{4.495057in}}%
\pgfusepath{clip}%
\pgfsetbuttcap%
\pgfsetroundjoin%
\pgfsetlinewidth{0.000000pt}%
\definecolor{currentstroke}{rgb}{0.000000,0.000000,0.000000}%
\pgfsetstrokecolor{currentstroke}%
\pgfsetdash{}{0pt}%
\pgfpathmoveto{\pgfqpoint{9.488635in}{2.302160in}}%
\pgfpathlineto{\pgfqpoint{9.674862in}{2.302160in}}%
\pgfpathlineto{\pgfqpoint{9.674862in}{2.383888in}}%
\pgfpathlineto{\pgfqpoint{9.488635in}{2.383888in}}%
\pgfpathlineto{\pgfqpoint{9.488635in}{2.302160in}}%
\pgfusepath{}%
\end{pgfscope}%
\begin{pgfscope}%
\pgfpathrectangle{\pgfqpoint{0.549740in}{0.463273in}}{\pgfqpoint{9.320225in}{4.495057in}}%
\pgfusepath{clip}%
\pgfsetbuttcap%
\pgfsetroundjoin%
\definecolor{currentfill}{rgb}{0.472869,0.711325,0.955316}%
\pgfsetfillcolor{currentfill}%
\pgfsetlinewidth{0.000000pt}%
\definecolor{currentstroke}{rgb}{0.000000,0.000000,0.000000}%
\pgfsetstrokecolor{currentstroke}%
\pgfsetdash{}{0pt}%
\pgfpathmoveto{\pgfqpoint{9.674862in}{2.302160in}}%
\pgfpathlineto{\pgfqpoint{9.861088in}{2.302160in}}%
\pgfpathlineto{\pgfqpoint{9.861088in}{2.383888in}}%
\pgfpathlineto{\pgfqpoint{9.674862in}{2.383888in}}%
\pgfpathlineto{\pgfqpoint{9.674862in}{2.302160in}}%
\pgfusepath{fill}%
\end{pgfscope}%
\begin{pgfscope}%
\pgfpathrectangle{\pgfqpoint{0.549740in}{0.463273in}}{\pgfqpoint{9.320225in}{4.495057in}}%
\pgfusepath{clip}%
\pgfsetbuttcap%
\pgfsetroundjoin%
\pgfsetlinewidth{0.000000pt}%
\definecolor{currentstroke}{rgb}{0.000000,0.000000,0.000000}%
\pgfsetstrokecolor{currentstroke}%
\pgfsetdash{}{0pt}%
\pgfpathmoveto{\pgfqpoint{0.549761in}{2.383888in}}%
\pgfpathlineto{\pgfqpoint{0.735988in}{2.383888in}}%
\pgfpathlineto{\pgfqpoint{0.735988in}{2.465617in}}%
\pgfpathlineto{\pgfqpoint{0.549761in}{2.465617in}}%
\pgfpathlineto{\pgfqpoint{0.549761in}{2.383888in}}%
\pgfusepath{}%
\end{pgfscope}%
\begin{pgfscope}%
\pgfpathrectangle{\pgfqpoint{0.549740in}{0.463273in}}{\pgfqpoint{9.320225in}{4.495057in}}%
\pgfusepath{clip}%
\pgfsetbuttcap%
\pgfsetroundjoin%
\pgfsetlinewidth{0.000000pt}%
\definecolor{currentstroke}{rgb}{0.000000,0.000000,0.000000}%
\pgfsetstrokecolor{currentstroke}%
\pgfsetdash{}{0pt}%
\pgfpathmoveto{\pgfqpoint{0.735988in}{2.383888in}}%
\pgfpathlineto{\pgfqpoint{0.922214in}{2.383888in}}%
\pgfpathlineto{\pgfqpoint{0.922214in}{2.465617in}}%
\pgfpathlineto{\pgfqpoint{0.735988in}{2.465617in}}%
\pgfpathlineto{\pgfqpoint{0.735988in}{2.383888in}}%
\pgfusepath{}%
\end{pgfscope}%
\begin{pgfscope}%
\pgfpathrectangle{\pgfqpoint{0.549740in}{0.463273in}}{\pgfqpoint{9.320225in}{4.495057in}}%
\pgfusepath{clip}%
\pgfsetbuttcap%
\pgfsetroundjoin%
\pgfsetlinewidth{0.000000pt}%
\definecolor{currentstroke}{rgb}{0.000000,0.000000,0.000000}%
\pgfsetstrokecolor{currentstroke}%
\pgfsetdash{}{0pt}%
\pgfpathmoveto{\pgfqpoint{0.922214in}{2.383888in}}%
\pgfpathlineto{\pgfqpoint{1.108441in}{2.383888in}}%
\pgfpathlineto{\pgfqpoint{1.108441in}{2.465617in}}%
\pgfpathlineto{\pgfqpoint{0.922214in}{2.465617in}}%
\pgfpathlineto{\pgfqpoint{0.922214in}{2.383888in}}%
\pgfusepath{}%
\end{pgfscope}%
\begin{pgfscope}%
\pgfpathrectangle{\pgfqpoint{0.549740in}{0.463273in}}{\pgfqpoint{9.320225in}{4.495057in}}%
\pgfusepath{clip}%
\pgfsetbuttcap%
\pgfsetroundjoin%
\pgfsetlinewidth{0.000000pt}%
\definecolor{currentstroke}{rgb}{0.000000,0.000000,0.000000}%
\pgfsetstrokecolor{currentstroke}%
\pgfsetdash{}{0pt}%
\pgfpathmoveto{\pgfqpoint{1.108441in}{2.383888in}}%
\pgfpathlineto{\pgfqpoint{1.294667in}{2.383888in}}%
\pgfpathlineto{\pgfqpoint{1.294667in}{2.465617in}}%
\pgfpathlineto{\pgfqpoint{1.108441in}{2.465617in}}%
\pgfpathlineto{\pgfqpoint{1.108441in}{2.383888in}}%
\pgfusepath{}%
\end{pgfscope}%
\begin{pgfscope}%
\pgfpathrectangle{\pgfqpoint{0.549740in}{0.463273in}}{\pgfqpoint{9.320225in}{4.495057in}}%
\pgfusepath{clip}%
\pgfsetbuttcap%
\pgfsetroundjoin%
\pgfsetlinewidth{0.000000pt}%
\definecolor{currentstroke}{rgb}{0.000000,0.000000,0.000000}%
\pgfsetstrokecolor{currentstroke}%
\pgfsetdash{}{0pt}%
\pgfpathmoveto{\pgfqpoint{1.294667in}{2.383888in}}%
\pgfpathlineto{\pgfqpoint{1.480894in}{2.383888in}}%
\pgfpathlineto{\pgfqpoint{1.480894in}{2.465617in}}%
\pgfpathlineto{\pgfqpoint{1.294667in}{2.465617in}}%
\pgfpathlineto{\pgfqpoint{1.294667in}{2.383888in}}%
\pgfusepath{}%
\end{pgfscope}%
\begin{pgfscope}%
\pgfpathrectangle{\pgfqpoint{0.549740in}{0.463273in}}{\pgfqpoint{9.320225in}{4.495057in}}%
\pgfusepath{clip}%
\pgfsetbuttcap%
\pgfsetroundjoin%
\pgfsetlinewidth{0.000000pt}%
\definecolor{currentstroke}{rgb}{0.000000,0.000000,0.000000}%
\pgfsetstrokecolor{currentstroke}%
\pgfsetdash{}{0pt}%
\pgfpathmoveto{\pgfqpoint{1.480894in}{2.383888in}}%
\pgfpathlineto{\pgfqpoint{1.667120in}{2.383888in}}%
\pgfpathlineto{\pgfqpoint{1.667120in}{2.465617in}}%
\pgfpathlineto{\pgfqpoint{1.480894in}{2.465617in}}%
\pgfpathlineto{\pgfqpoint{1.480894in}{2.383888in}}%
\pgfusepath{}%
\end{pgfscope}%
\begin{pgfscope}%
\pgfpathrectangle{\pgfqpoint{0.549740in}{0.463273in}}{\pgfqpoint{9.320225in}{4.495057in}}%
\pgfusepath{clip}%
\pgfsetbuttcap%
\pgfsetroundjoin%
\pgfsetlinewidth{0.000000pt}%
\definecolor{currentstroke}{rgb}{0.000000,0.000000,0.000000}%
\pgfsetstrokecolor{currentstroke}%
\pgfsetdash{}{0pt}%
\pgfpathmoveto{\pgfqpoint{1.667120in}{2.383888in}}%
\pgfpathlineto{\pgfqpoint{1.853347in}{2.383888in}}%
\pgfpathlineto{\pgfqpoint{1.853347in}{2.465617in}}%
\pgfpathlineto{\pgfqpoint{1.667120in}{2.465617in}}%
\pgfpathlineto{\pgfqpoint{1.667120in}{2.383888in}}%
\pgfusepath{}%
\end{pgfscope}%
\begin{pgfscope}%
\pgfpathrectangle{\pgfqpoint{0.549740in}{0.463273in}}{\pgfqpoint{9.320225in}{4.495057in}}%
\pgfusepath{clip}%
\pgfsetbuttcap%
\pgfsetroundjoin%
\pgfsetlinewidth{0.000000pt}%
\definecolor{currentstroke}{rgb}{0.000000,0.000000,0.000000}%
\pgfsetstrokecolor{currentstroke}%
\pgfsetdash{}{0pt}%
\pgfpathmoveto{\pgfqpoint{1.853347in}{2.383888in}}%
\pgfpathlineto{\pgfqpoint{2.039573in}{2.383888in}}%
\pgfpathlineto{\pgfqpoint{2.039573in}{2.465617in}}%
\pgfpathlineto{\pgfqpoint{1.853347in}{2.465617in}}%
\pgfpathlineto{\pgfqpoint{1.853347in}{2.383888in}}%
\pgfusepath{}%
\end{pgfscope}%
\begin{pgfscope}%
\pgfpathrectangle{\pgfqpoint{0.549740in}{0.463273in}}{\pgfqpoint{9.320225in}{4.495057in}}%
\pgfusepath{clip}%
\pgfsetbuttcap%
\pgfsetroundjoin%
\pgfsetlinewidth{0.000000pt}%
\definecolor{currentstroke}{rgb}{0.000000,0.000000,0.000000}%
\pgfsetstrokecolor{currentstroke}%
\pgfsetdash{}{0pt}%
\pgfpathmoveto{\pgfqpoint{2.039573in}{2.383888in}}%
\pgfpathlineto{\pgfqpoint{2.225800in}{2.383888in}}%
\pgfpathlineto{\pgfqpoint{2.225800in}{2.465617in}}%
\pgfpathlineto{\pgfqpoint{2.039573in}{2.465617in}}%
\pgfpathlineto{\pgfqpoint{2.039573in}{2.383888in}}%
\pgfusepath{}%
\end{pgfscope}%
\begin{pgfscope}%
\pgfpathrectangle{\pgfqpoint{0.549740in}{0.463273in}}{\pgfqpoint{9.320225in}{4.495057in}}%
\pgfusepath{clip}%
\pgfsetbuttcap%
\pgfsetroundjoin%
\pgfsetlinewidth{0.000000pt}%
\definecolor{currentstroke}{rgb}{0.000000,0.000000,0.000000}%
\pgfsetstrokecolor{currentstroke}%
\pgfsetdash{}{0pt}%
\pgfpathmoveto{\pgfqpoint{2.225800in}{2.383888in}}%
\pgfpathlineto{\pgfqpoint{2.412027in}{2.383888in}}%
\pgfpathlineto{\pgfqpoint{2.412027in}{2.465617in}}%
\pgfpathlineto{\pgfqpoint{2.225800in}{2.465617in}}%
\pgfpathlineto{\pgfqpoint{2.225800in}{2.383888in}}%
\pgfusepath{}%
\end{pgfscope}%
\begin{pgfscope}%
\pgfpathrectangle{\pgfqpoint{0.549740in}{0.463273in}}{\pgfqpoint{9.320225in}{4.495057in}}%
\pgfusepath{clip}%
\pgfsetbuttcap%
\pgfsetroundjoin%
\pgfsetlinewidth{0.000000pt}%
\definecolor{currentstroke}{rgb}{0.000000,0.000000,0.000000}%
\pgfsetstrokecolor{currentstroke}%
\pgfsetdash{}{0pt}%
\pgfpathmoveto{\pgfqpoint{2.412027in}{2.383888in}}%
\pgfpathlineto{\pgfqpoint{2.598253in}{2.383888in}}%
\pgfpathlineto{\pgfqpoint{2.598253in}{2.465617in}}%
\pgfpathlineto{\pgfqpoint{2.412027in}{2.465617in}}%
\pgfpathlineto{\pgfqpoint{2.412027in}{2.383888in}}%
\pgfusepath{}%
\end{pgfscope}%
\begin{pgfscope}%
\pgfpathrectangle{\pgfqpoint{0.549740in}{0.463273in}}{\pgfqpoint{9.320225in}{4.495057in}}%
\pgfusepath{clip}%
\pgfsetbuttcap%
\pgfsetroundjoin%
\pgfsetlinewidth{0.000000pt}%
\definecolor{currentstroke}{rgb}{0.000000,0.000000,0.000000}%
\pgfsetstrokecolor{currentstroke}%
\pgfsetdash{}{0pt}%
\pgfpathmoveto{\pgfqpoint{2.598253in}{2.383888in}}%
\pgfpathlineto{\pgfqpoint{2.784480in}{2.383888in}}%
\pgfpathlineto{\pgfqpoint{2.784480in}{2.465617in}}%
\pgfpathlineto{\pgfqpoint{2.598253in}{2.465617in}}%
\pgfpathlineto{\pgfqpoint{2.598253in}{2.383888in}}%
\pgfusepath{}%
\end{pgfscope}%
\begin{pgfscope}%
\pgfpathrectangle{\pgfqpoint{0.549740in}{0.463273in}}{\pgfqpoint{9.320225in}{4.495057in}}%
\pgfusepath{clip}%
\pgfsetbuttcap%
\pgfsetroundjoin%
\pgfsetlinewidth{0.000000pt}%
\definecolor{currentstroke}{rgb}{0.000000,0.000000,0.000000}%
\pgfsetstrokecolor{currentstroke}%
\pgfsetdash{}{0pt}%
\pgfpathmoveto{\pgfqpoint{2.784480in}{2.383888in}}%
\pgfpathlineto{\pgfqpoint{2.970706in}{2.383888in}}%
\pgfpathlineto{\pgfqpoint{2.970706in}{2.465617in}}%
\pgfpathlineto{\pgfqpoint{2.784480in}{2.465617in}}%
\pgfpathlineto{\pgfqpoint{2.784480in}{2.383888in}}%
\pgfusepath{}%
\end{pgfscope}%
\begin{pgfscope}%
\pgfpathrectangle{\pgfqpoint{0.549740in}{0.463273in}}{\pgfqpoint{9.320225in}{4.495057in}}%
\pgfusepath{clip}%
\pgfsetbuttcap%
\pgfsetroundjoin%
\pgfsetlinewidth{0.000000pt}%
\definecolor{currentstroke}{rgb}{0.000000,0.000000,0.000000}%
\pgfsetstrokecolor{currentstroke}%
\pgfsetdash{}{0pt}%
\pgfpathmoveto{\pgfqpoint{2.970706in}{2.383888in}}%
\pgfpathlineto{\pgfqpoint{3.156933in}{2.383888in}}%
\pgfpathlineto{\pgfqpoint{3.156933in}{2.465617in}}%
\pgfpathlineto{\pgfqpoint{2.970706in}{2.465617in}}%
\pgfpathlineto{\pgfqpoint{2.970706in}{2.383888in}}%
\pgfusepath{}%
\end{pgfscope}%
\begin{pgfscope}%
\pgfpathrectangle{\pgfqpoint{0.549740in}{0.463273in}}{\pgfqpoint{9.320225in}{4.495057in}}%
\pgfusepath{clip}%
\pgfsetbuttcap%
\pgfsetroundjoin%
\pgfsetlinewidth{0.000000pt}%
\definecolor{currentstroke}{rgb}{0.000000,0.000000,0.000000}%
\pgfsetstrokecolor{currentstroke}%
\pgfsetdash{}{0pt}%
\pgfpathmoveto{\pgfqpoint{3.156933in}{2.383888in}}%
\pgfpathlineto{\pgfqpoint{3.343159in}{2.383888in}}%
\pgfpathlineto{\pgfqpoint{3.343159in}{2.465617in}}%
\pgfpathlineto{\pgfqpoint{3.156933in}{2.465617in}}%
\pgfpathlineto{\pgfqpoint{3.156933in}{2.383888in}}%
\pgfusepath{}%
\end{pgfscope}%
\begin{pgfscope}%
\pgfpathrectangle{\pgfqpoint{0.549740in}{0.463273in}}{\pgfqpoint{9.320225in}{4.495057in}}%
\pgfusepath{clip}%
\pgfsetbuttcap%
\pgfsetroundjoin%
\pgfsetlinewidth{0.000000pt}%
\definecolor{currentstroke}{rgb}{0.000000,0.000000,0.000000}%
\pgfsetstrokecolor{currentstroke}%
\pgfsetdash{}{0pt}%
\pgfpathmoveto{\pgfqpoint{3.343159in}{2.383888in}}%
\pgfpathlineto{\pgfqpoint{3.529386in}{2.383888in}}%
\pgfpathlineto{\pgfqpoint{3.529386in}{2.465617in}}%
\pgfpathlineto{\pgfqpoint{3.343159in}{2.465617in}}%
\pgfpathlineto{\pgfqpoint{3.343159in}{2.383888in}}%
\pgfusepath{}%
\end{pgfscope}%
\begin{pgfscope}%
\pgfpathrectangle{\pgfqpoint{0.549740in}{0.463273in}}{\pgfqpoint{9.320225in}{4.495057in}}%
\pgfusepath{clip}%
\pgfsetbuttcap%
\pgfsetroundjoin%
\pgfsetlinewidth{0.000000pt}%
\definecolor{currentstroke}{rgb}{0.000000,0.000000,0.000000}%
\pgfsetstrokecolor{currentstroke}%
\pgfsetdash{}{0pt}%
\pgfpathmoveto{\pgfqpoint{3.529386in}{2.383888in}}%
\pgfpathlineto{\pgfqpoint{3.715612in}{2.383888in}}%
\pgfpathlineto{\pgfqpoint{3.715612in}{2.465617in}}%
\pgfpathlineto{\pgfqpoint{3.529386in}{2.465617in}}%
\pgfpathlineto{\pgfqpoint{3.529386in}{2.383888in}}%
\pgfusepath{}%
\end{pgfscope}%
\begin{pgfscope}%
\pgfpathrectangle{\pgfqpoint{0.549740in}{0.463273in}}{\pgfqpoint{9.320225in}{4.495057in}}%
\pgfusepath{clip}%
\pgfsetbuttcap%
\pgfsetroundjoin%
\pgfsetlinewidth{0.000000pt}%
\definecolor{currentstroke}{rgb}{0.000000,0.000000,0.000000}%
\pgfsetstrokecolor{currentstroke}%
\pgfsetdash{}{0pt}%
\pgfpathmoveto{\pgfqpoint{3.715612in}{2.383888in}}%
\pgfpathlineto{\pgfqpoint{3.901839in}{2.383888in}}%
\pgfpathlineto{\pgfqpoint{3.901839in}{2.465617in}}%
\pgfpathlineto{\pgfqpoint{3.715612in}{2.465617in}}%
\pgfpathlineto{\pgfqpoint{3.715612in}{2.383888in}}%
\pgfusepath{}%
\end{pgfscope}%
\begin{pgfscope}%
\pgfpathrectangle{\pgfqpoint{0.549740in}{0.463273in}}{\pgfqpoint{9.320225in}{4.495057in}}%
\pgfusepath{clip}%
\pgfsetbuttcap%
\pgfsetroundjoin%
\pgfsetlinewidth{0.000000pt}%
\definecolor{currentstroke}{rgb}{0.000000,0.000000,0.000000}%
\pgfsetstrokecolor{currentstroke}%
\pgfsetdash{}{0pt}%
\pgfpathmoveto{\pgfqpoint{3.901839in}{2.383888in}}%
\pgfpathlineto{\pgfqpoint{4.088065in}{2.383888in}}%
\pgfpathlineto{\pgfqpoint{4.088065in}{2.465617in}}%
\pgfpathlineto{\pgfqpoint{3.901839in}{2.465617in}}%
\pgfpathlineto{\pgfqpoint{3.901839in}{2.383888in}}%
\pgfusepath{}%
\end{pgfscope}%
\begin{pgfscope}%
\pgfpathrectangle{\pgfqpoint{0.549740in}{0.463273in}}{\pgfqpoint{9.320225in}{4.495057in}}%
\pgfusepath{clip}%
\pgfsetbuttcap%
\pgfsetroundjoin%
\pgfsetlinewidth{0.000000pt}%
\definecolor{currentstroke}{rgb}{0.000000,0.000000,0.000000}%
\pgfsetstrokecolor{currentstroke}%
\pgfsetdash{}{0pt}%
\pgfpathmoveto{\pgfqpoint{4.088065in}{2.383888in}}%
\pgfpathlineto{\pgfqpoint{4.274292in}{2.383888in}}%
\pgfpathlineto{\pgfqpoint{4.274292in}{2.465617in}}%
\pgfpathlineto{\pgfqpoint{4.088065in}{2.465617in}}%
\pgfpathlineto{\pgfqpoint{4.088065in}{2.383888in}}%
\pgfusepath{}%
\end{pgfscope}%
\begin{pgfscope}%
\pgfpathrectangle{\pgfqpoint{0.549740in}{0.463273in}}{\pgfqpoint{9.320225in}{4.495057in}}%
\pgfusepath{clip}%
\pgfsetbuttcap%
\pgfsetroundjoin%
\pgfsetlinewidth{0.000000pt}%
\definecolor{currentstroke}{rgb}{0.000000,0.000000,0.000000}%
\pgfsetstrokecolor{currentstroke}%
\pgfsetdash{}{0pt}%
\pgfpathmoveto{\pgfqpoint{4.274292in}{2.383888in}}%
\pgfpathlineto{\pgfqpoint{4.460519in}{2.383888in}}%
\pgfpathlineto{\pgfqpoint{4.460519in}{2.465617in}}%
\pgfpathlineto{\pgfqpoint{4.274292in}{2.465617in}}%
\pgfpathlineto{\pgfqpoint{4.274292in}{2.383888in}}%
\pgfusepath{}%
\end{pgfscope}%
\begin{pgfscope}%
\pgfpathrectangle{\pgfqpoint{0.549740in}{0.463273in}}{\pgfqpoint{9.320225in}{4.495057in}}%
\pgfusepath{clip}%
\pgfsetbuttcap%
\pgfsetroundjoin%
\pgfsetlinewidth{0.000000pt}%
\definecolor{currentstroke}{rgb}{0.000000,0.000000,0.000000}%
\pgfsetstrokecolor{currentstroke}%
\pgfsetdash{}{0pt}%
\pgfpathmoveto{\pgfqpoint{4.460519in}{2.383888in}}%
\pgfpathlineto{\pgfqpoint{4.646745in}{2.383888in}}%
\pgfpathlineto{\pgfqpoint{4.646745in}{2.465617in}}%
\pgfpathlineto{\pgfqpoint{4.460519in}{2.465617in}}%
\pgfpathlineto{\pgfqpoint{4.460519in}{2.383888in}}%
\pgfusepath{}%
\end{pgfscope}%
\begin{pgfscope}%
\pgfpathrectangle{\pgfqpoint{0.549740in}{0.463273in}}{\pgfqpoint{9.320225in}{4.495057in}}%
\pgfusepath{clip}%
\pgfsetbuttcap%
\pgfsetroundjoin%
\pgfsetlinewidth{0.000000pt}%
\definecolor{currentstroke}{rgb}{0.000000,0.000000,0.000000}%
\pgfsetstrokecolor{currentstroke}%
\pgfsetdash{}{0pt}%
\pgfpathmoveto{\pgfqpoint{4.646745in}{2.383888in}}%
\pgfpathlineto{\pgfqpoint{4.832972in}{2.383888in}}%
\pgfpathlineto{\pgfqpoint{4.832972in}{2.465617in}}%
\pgfpathlineto{\pgfqpoint{4.646745in}{2.465617in}}%
\pgfpathlineto{\pgfqpoint{4.646745in}{2.383888in}}%
\pgfusepath{}%
\end{pgfscope}%
\begin{pgfscope}%
\pgfpathrectangle{\pgfqpoint{0.549740in}{0.463273in}}{\pgfqpoint{9.320225in}{4.495057in}}%
\pgfusepath{clip}%
\pgfsetbuttcap%
\pgfsetroundjoin%
\pgfsetlinewidth{0.000000pt}%
\definecolor{currentstroke}{rgb}{0.000000,0.000000,0.000000}%
\pgfsetstrokecolor{currentstroke}%
\pgfsetdash{}{0pt}%
\pgfpathmoveto{\pgfqpoint{4.832972in}{2.383888in}}%
\pgfpathlineto{\pgfqpoint{5.019198in}{2.383888in}}%
\pgfpathlineto{\pgfqpoint{5.019198in}{2.465617in}}%
\pgfpathlineto{\pgfqpoint{4.832972in}{2.465617in}}%
\pgfpathlineto{\pgfqpoint{4.832972in}{2.383888in}}%
\pgfusepath{}%
\end{pgfscope}%
\begin{pgfscope}%
\pgfpathrectangle{\pgfqpoint{0.549740in}{0.463273in}}{\pgfqpoint{9.320225in}{4.495057in}}%
\pgfusepath{clip}%
\pgfsetbuttcap%
\pgfsetroundjoin%
\pgfsetlinewidth{0.000000pt}%
\definecolor{currentstroke}{rgb}{0.000000,0.000000,0.000000}%
\pgfsetstrokecolor{currentstroke}%
\pgfsetdash{}{0pt}%
\pgfpathmoveto{\pgfqpoint{5.019198in}{2.383888in}}%
\pgfpathlineto{\pgfqpoint{5.205425in}{2.383888in}}%
\pgfpathlineto{\pgfqpoint{5.205425in}{2.465617in}}%
\pgfpathlineto{\pgfqpoint{5.019198in}{2.465617in}}%
\pgfpathlineto{\pgfqpoint{5.019198in}{2.383888in}}%
\pgfusepath{}%
\end{pgfscope}%
\begin{pgfscope}%
\pgfpathrectangle{\pgfqpoint{0.549740in}{0.463273in}}{\pgfqpoint{9.320225in}{4.495057in}}%
\pgfusepath{clip}%
\pgfsetbuttcap%
\pgfsetroundjoin%
\pgfsetlinewidth{0.000000pt}%
\definecolor{currentstroke}{rgb}{0.000000,0.000000,0.000000}%
\pgfsetstrokecolor{currentstroke}%
\pgfsetdash{}{0pt}%
\pgfpathmoveto{\pgfqpoint{5.205425in}{2.383888in}}%
\pgfpathlineto{\pgfqpoint{5.391651in}{2.383888in}}%
\pgfpathlineto{\pgfqpoint{5.391651in}{2.465617in}}%
\pgfpathlineto{\pgfqpoint{5.205425in}{2.465617in}}%
\pgfpathlineto{\pgfqpoint{5.205425in}{2.383888in}}%
\pgfusepath{}%
\end{pgfscope}%
\begin{pgfscope}%
\pgfpathrectangle{\pgfqpoint{0.549740in}{0.463273in}}{\pgfqpoint{9.320225in}{4.495057in}}%
\pgfusepath{clip}%
\pgfsetbuttcap%
\pgfsetroundjoin%
\definecolor{currentfill}{rgb}{0.472869,0.711325,0.955316}%
\pgfsetfillcolor{currentfill}%
\pgfsetlinewidth{0.000000pt}%
\definecolor{currentstroke}{rgb}{0.000000,0.000000,0.000000}%
\pgfsetstrokecolor{currentstroke}%
\pgfsetdash{}{0pt}%
\pgfpathmoveto{\pgfqpoint{5.391651in}{2.383888in}}%
\pgfpathlineto{\pgfqpoint{5.577878in}{2.383888in}}%
\pgfpathlineto{\pgfqpoint{5.577878in}{2.465617in}}%
\pgfpathlineto{\pgfqpoint{5.391651in}{2.465617in}}%
\pgfpathlineto{\pgfqpoint{5.391651in}{2.383888in}}%
\pgfusepath{fill}%
\end{pgfscope}%
\begin{pgfscope}%
\pgfpathrectangle{\pgfqpoint{0.549740in}{0.463273in}}{\pgfqpoint{9.320225in}{4.495057in}}%
\pgfusepath{clip}%
\pgfsetbuttcap%
\pgfsetroundjoin%
\pgfsetlinewidth{0.000000pt}%
\definecolor{currentstroke}{rgb}{0.000000,0.000000,0.000000}%
\pgfsetstrokecolor{currentstroke}%
\pgfsetdash{}{0pt}%
\pgfpathmoveto{\pgfqpoint{5.577878in}{2.383888in}}%
\pgfpathlineto{\pgfqpoint{5.764104in}{2.383888in}}%
\pgfpathlineto{\pgfqpoint{5.764104in}{2.465617in}}%
\pgfpathlineto{\pgfqpoint{5.577878in}{2.465617in}}%
\pgfpathlineto{\pgfqpoint{5.577878in}{2.383888in}}%
\pgfusepath{}%
\end{pgfscope}%
\begin{pgfscope}%
\pgfpathrectangle{\pgfqpoint{0.549740in}{0.463273in}}{\pgfqpoint{9.320225in}{4.495057in}}%
\pgfusepath{clip}%
\pgfsetbuttcap%
\pgfsetroundjoin%
\pgfsetlinewidth{0.000000pt}%
\definecolor{currentstroke}{rgb}{0.000000,0.000000,0.000000}%
\pgfsetstrokecolor{currentstroke}%
\pgfsetdash{}{0pt}%
\pgfpathmoveto{\pgfqpoint{5.764104in}{2.383888in}}%
\pgfpathlineto{\pgfqpoint{5.950331in}{2.383888in}}%
\pgfpathlineto{\pgfqpoint{5.950331in}{2.465617in}}%
\pgfpathlineto{\pgfqpoint{5.764104in}{2.465617in}}%
\pgfpathlineto{\pgfqpoint{5.764104in}{2.383888in}}%
\pgfusepath{}%
\end{pgfscope}%
\begin{pgfscope}%
\pgfpathrectangle{\pgfqpoint{0.549740in}{0.463273in}}{\pgfqpoint{9.320225in}{4.495057in}}%
\pgfusepath{clip}%
\pgfsetbuttcap%
\pgfsetroundjoin%
\pgfsetlinewidth{0.000000pt}%
\definecolor{currentstroke}{rgb}{0.000000,0.000000,0.000000}%
\pgfsetstrokecolor{currentstroke}%
\pgfsetdash{}{0pt}%
\pgfpathmoveto{\pgfqpoint{5.950331in}{2.383888in}}%
\pgfpathlineto{\pgfqpoint{6.136557in}{2.383888in}}%
\pgfpathlineto{\pgfqpoint{6.136557in}{2.465617in}}%
\pgfpathlineto{\pgfqpoint{5.950331in}{2.465617in}}%
\pgfpathlineto{\pgfqpoint{5.950331in}{2.383888in}}%
\pgfusepath{}%
\end{pgfscope}%
\begin{pgfscope}%
\pgfpathrectangle{\pgfqpoint{0.549740in}{0.463273in}}{\pgfqpoint{9.320225in}{4.495057in}}%
\pgfusepath{clip}%
\pgfsetbuttcap%
\pgfsetroundjoin%
\definecolor{currentfill}{rgb}{0.614330,0.761948,0.940009}%
\pgfsetfillcolor{currentfill}%
\pgfsetlinewidth{0.000000pt}%
\definecolor{currentstroke}{rgb}{0.000000,0.000000,0.000000}%
\pgfsetstrokecolor{currentstroke}%
\pgfsetdash{}{0pt}%
\pgfpathmoveto{\pgfqpoint{6.136557in}{2.383888in}}%
\pgfpathlineto{\pgfqpoint{6.322784in}{2.383888in}}%
\pgfpathlineto{\pgfqpoint{6.322784in}{2.465617in}}%
\pgfpathlineto{\pgfqpoint{6.136557in}{2.465617in}}%
\pgfpathlineto{\pgfqpoint{6.136557in}{2.383888in}}%
\pgfusepath{fill}%
\end{pgfscope}%
\begin{pgfscope}%
\pgfpathrectangle{\pgfqpoint{0.549740in}{0.463273in}}{\pgfqpoint{9.320225in}{4.495057in}}%
\pgfusepath{clip}%
\pgfsetbuttcap%
\pgfsetroundjoin%
\definecolor{currentfill}{rgb}{0.547810,0.736432,0.947518}%
\pgfsetfillcolor{currentfill}%
\pgfsetlinewidth{0.000000pt}%
\definecolor{currentstroke}{rgb}{0.000000,0.000000,0.000000}%
\pgfsetstrokecolor{currentstroke}%
\pgfsetdash{}{0pt}%
\pgfpathmoveto{\pgfqpoint{6.322784in}{2.383888in}}%
\pgfpathlineto{\pgfqpoint{6.509011in}{2.383888in}}%
\pgfpathlineto{\pgfqpoint{6.509011in}{2.465617in}}%
\pgfpathlineto{\pgfqpoint{6.322784in}{2.465617in}}%
\pgfpathlineto{\pgfqpoint{6.322784in}{2.383888in}}%
\pgfusepath{fill}%
\end{pgfscope}%
\begin{pgfscope}%
\pgfpathrectangle{\pgfqpoint{0.549740in}{0.463273in}}{\pgfqpoint{9.320225in}{4.495057in}}%
\pgfusepath{clip}%
\pgfsetbuttcap%
\pgfsetroundjoin%
\pgfsetlinewidth{0.000000pt}%
\definecolor{currentstroke}{rgb}{0.000000,0.000000,0.000000}%
\pgfsetstrokecolor{currentstroke}%
\pgfsetdash{}{0pt}%
\pgfpathmoveto{\pgfqpoint{6.509011in}{2.383888in}}%
\pgfpathlineto{\pgfqpoint{6.695237in}{2.383888in}}%
\pgfpathlineto{\pgfqpoint{6.695237in}{2.465617in}}%
\pgfpathlineto{\pgfqpoint{6.509011in}{2.465617in}}%
\pgfpathlineto{\pgfqpoint{6.509011in}{2.383888in}}%
\pgfusepath{}%
\end{pgfscope}%
\begin{pgfscope}%
\pgfpathrectangle{\pgfqpoint{0.549740in}{0.463273in}}{\pgfqpoint{9.320225in}{4.495057in}}%
\pgfusepath{clip}%
\pgfsetbuttcap%
\pgfsetroundjoin%
\pgfsetlinewidth{0.000000pt}%
\definecolor{currentstroke}{rgb}{0.000000,0.000000,0.000000}%
\pgfsetstrokecolor{currentstroke}%
\pgfsetdash{}{0pt}%
\pgfpathmoveto{\pgfqpoint{6.695237in}{2.383888in}}%
\pgfpathlineto{\pgfqpoint{6.881464in}{2.383888in}}%
\pgfpathlineto{\pgfqpoint{6.881464in}{2.465617in}}%
\pgfpathlineto{\pgfqpoint{6.695237in}{2.465617in}}%
\pgfpathlineto{\pgfqpoint{6.695237in}{2.383888in}}%
\pgfusepath{}%
\end{pgfscope}%
\begin{pgfscope}%
\pgfpathrectangle{\pgfqpoint{0.549740in}{0.463273in}}{\pgfqpoint{9.320225in}{4.495057in}}%
\pgfusepath{clip}%
\pgfsetbuttcap%
\pgfsetroundjoin%
\pgfsetlinewidth{0.000000pt}%
\definecolor{currentstroke}{rgb}{0.000000,0.000000,0.000000}%
\pgfsetstrokecolor{currentstroke}%
\pgfsetdash{}{0pt}%
\pgfpathmoveto{\pgfqpoint{6.881464in}{2.383888in}}%
\pgfpathlineto{\pgfqpoint{7.067690in}{2.383888in}}%
\pgfpathlineto{\pgfqpoint{7.067690in}{2.465617in}}%
\pgfpathlineto{\pgfqpoint{6.881464in}{2.465617in}}%
\pgfpathlineto{\pgfqpoint{6.881464in}{2.383888in}}%
\pgfusepath{}%
\end{pgfscope}%
\begin{pgfscope}%
\pgfpathrectangle{\pgfqpoint{0.549740in}{0.463273in}}{\pgfqpoint{9.320225in}{4.495057in}}%
\pgfusepath{clip}%
\pgfsetbuttcap%
\pgfsetroundjoin%
\pgfsetlinewidth{0.000000pt}%
\definecolor{currentstroke}{rgb}{0.000000,0.000000,0.000000}%
\pgfsetstrokecolor{currentstroke}%
\pgfsetdash{}{0pt}%
\pgfpathmoveto{\pgfqpoint{7.067690in}{2.383888in}}%
\pgfpathlineto{\pgfqpoint{7.253917in}{2.383888in}}%
\pgfpathlineto{\pgfqpoint{7.253917in}{2.465617in}}%
\pgfpathlineto{\pgfqpoint{7.067690in}{2.465617in}}%
\pgfpathlineto{\pgfqpoint{7.067690in}{2.383888in}}%
\pgfusepath{}%
\end{pgfscope}%
\begin{pgfscope}%
\pgfpathrectangle{\pgfqpoint{0.549740in}{0.463273in}}{\pgfqpoint{9.320225in}{4.495057in}}%
\pgfusepath{clip}%
\pgfsetbuttcap%
\pgfsetroundjoin%
\definecolor{currentfill}{rgb}{0.472869,0.711325,0.955316}%
\pgfsetfillcolor{currentfill}%
\pgfsetlinewidth{0.000000pt}%
\definecolor{currentstroke}{rgb}{0.000000,0.000000,0.000000}%
\pgfsetstrokecolor{currentstroke}%
\pgfsetdash{}{0pt}%
\pgfpathmoveto{\pgfqpoint{7.253917in}{2.383888in}}%
\pgfpathlineto{\pgfqpoint{7.440143in}{2.383888in}}%
\pgfpathlineto{\pgfqpoint{7.440143in}{2.465617in}}%
\pgfpathlineto{\pgfqpoint{7.253917in}{2.465617in}}%
\pgfpathlineto{\pgfqpoint{7.253917in}{2.383888in}}%
\pgfusepath{fill}%
\end{pgfscope}%
\begin{pgfscope}%
\pgfpathrectangle{\pgfqpoint{0.549740in}{0.463273in}}{\pgfqpoint{9.320225in}{4.495057in}}%
\pgfusepath{clip}%
\pgfsetbuttcap%
\pgfsetroundjoin%
\pgfsetlinewidth{0.000000pt}%
\definecolor{currentstroke}{rgb}{0.000000,0.000000,0.000000}%
\pgfsetstrokecolor{currentstroke}%
\pgfsetdash{}{0pt}%
\pgfpathmoveto{\pgfqpoint{7.440143in}{2.383888in}}%
\pgfpathlineto{\pgfqpoint{7.626370in}{2.383888in}}%
\pgfpathlineto{\pgfqpoint{7.626370in}{2.465617in}}%
\pgfpathlineto{\pgfqpoint{7.440143in}{2.465617in}}%
\pgfpathlineto{\pgfqpoint{7.440143in}{2.383888in}}%
\pgfusepath{}%
\end{pgfscope}%
\begin{pgfscope}%
\pgfpathrectangle{\pgfqpoint{0.549740in}{0.463273in}}{\pgfqpoint{9.320225in}{4.495057in}}%
\pgfusepath{clip}%
\pgfsetbuttcap%
\pgfsetroundjoin%
\pgfsetlinewidth{0.000000pt}%
\definecolor{currentstroke}{rgb}{0.000000,0.000000,0.000000}%
\pgfsetstrokecolor{currentstroke}%
\pgfsetdash{}{0pt}%
\pgfpathmoveto{\pgfqpoint{7.626370in}{2.383888in}}%
\pgfpathlineto{\pgfqpoint{7.812596in}{2.383888in}}%
\pgfpathlineto{\pgfqpoint{7.812596in}{2.465617in}}%
\pgfpathlineto{\pgfqpoint{7.626370in}{2.465617in}}%
\pgfpathlineto{\pgfqpoint{7.626370in}{2.383888in}}%
\pgfusepath{}%
\end{pgfscope}%
\begin{pgfscope}%
\pgfpathrectangle{\pgfqpoint{0.549740in}{0.463273in}}{\pgfqpoint{9.320225in}{4.495057in}}%
\pgfusepath{clip}%
\pgfsetbuttcap%
\pgfsetroundjoin%
\pgfsetlinewidth{0.000000pt}%
\definecolor{currentstroke}{rgb}{0.000000,0.000000,0.000000}%
\pgfsetstrokecolor{currentstroke}%
\pgfsetdash{}{0pt}%
\pgfpathmoveto{\pgfqpoint{7.812596in}{2.383888in}}%
\pgfpathlineto{\pgfqpoint{7.998823in}{2.383888in}}%
\pgfpathlineto{\pgfqpoint{7.998823in}{2.465617in}}%
\pgfpathlineto{\pgfqpoint{7.812596in}{2.465617in}}%
\pgfpathlineto{\pgfqpoint{7.812596in}{2.383888in}}%
\pgfusepath{}%
\end{pgfscope}%
\begin{pgfscope}%
\pgfpathrectangle{\pgfqpoint{0.549740in}{0.463273in}}{\pgfqpoint{9.320225in}{4.495057in}}%
\pgfusepath{clip}%
\pgfsetbuttcap%
\pgfsetroundjoin%
\pgfsetlinewidth{0.000000pt}%
\definecolor{currentstroke}{rgb}{0.000000,0.000000,0.000000}%
\pgfsetstrokecolor{currentstroke}%
\pgfsetdash{}{0pt}%
\pgfpathmoveto{\pgfqpoint{7.998823in}{2.383888in}}%
\pgfpathlineto{\pgfqpoint{8.185049in}{2.383888in}}%
\pgfpathlineto{\pgfqpoint{8.185049in}{2.465617in}}%
\pgfpathlineto{\pgfqpoint{7.998823in}{2.465617in}}%
\pgfpathlineto{\pgfqpoint{7.998823in}{2.383888in}}%
\pgfusepath{}%
\end{pgfscope}%
\begin{pgfscope}%
\pgfpathrectangle{\pgfqpoint{0.549740in}{0.463273in}}{\pgfqpoint{9.320225in}{4.495057in}}%
\pgfusepath{clip}%
\pgfsetbuttcap%
\pgfsetroundjoin%
\pgfsetlinewidth{0.000000pt}%
\definecolor{currentstroke}{rgb}{0.000000,0.000000,0.000000}%
\pgfsetstrokecolor{currentstroke}%
\pgfsetdash{}{0pt}%
\pgfpathmoveto{\pgfqpoint{8.185049in}{2.383888in}}%
\pgfpathlineto{\pgfqpoint{8.371276in}{2.383888in}}%
\pgfpathlineto{\pgfqpoint{8.371276in}{2.465617in}}%
\pgfpathlineto{\pgfqpoint{8.185049in}{2.465617in}}%
\pgfpathlineto{\pgfqpoint{8.185049in}{2.383888in}}%
\pgfusepath{}%
\end{pgfscope}%
\begin{pgfscope}%
\pgfpathrectangle{\pgfqpoint{0.549740in}{0.463273in}}{\pgfqpoint{9.320225in}{4.495057in}}%
\pgfusepath{clip}%
\pgfsetbuttcap%
\pgfsetroundjoin%
\definecolor{currentfill}{rgb}{0.472869,0.711325,0.955316}%
\pgfsetfillcolor{currentfill}%
\pgfsetlinewidth{0.000000pt}%
\definecolor{currentstroke}{rgb}{0.000000,0.000000,0.000000}%
\pgfsetstrokecolor{currentstroke}%
\pgfsetdash{}{0pt}%
\pgfpathmoveto{\pgfqpoint{8.371276in}{2.383888in}}%
\pgfpathlineto{\pgfqpoint{8.557503in}{2.383888in}}%
\pgfpathlineto{\pgfqpoint{8.557503in}{2.465617in}}%
\pgfpathlineto{\pgfqpoint{8.371276in}{2.465617in}}%
\pgfpathlineto{\pgfqpoint{8.371276in}{2.383888in}}%
\pgfusepath{fill}%
\end{pgfscope}%
\begin{pgfscope}%
\pgfpathrectangle{\pgfqpoint{0.549740in}{0.463273in}}{\pgfqpoint{9.320225in}{4.495057in}}%
\pgfusepath{clip}%
\pgfsetbuttcap%
\pgfsetroundjoin%
\pgfsetlinewidth{0.000000pt}%
\definecolor{currentstroke}{rgb}{0.000000,0.000000,0.000000}%
\pgfsetstrokecolor{currentstroke}%
\pgfsetdash{}{0pt}%
\pgfpathmoveto{\pgfqpoint{8.557503in}{2.383888in}}%
\pgfpathlineto{\pgfqpoint{8.743729in}{2.383888in}}%
\pgfpathlineto{\pgfqpoint{8.743729in}{2.465617in}}%
\pgfpathlineto{\pgfqpoint{8.557503in}{2.465617in}}%
\pgfpathlineto{\pgfqpoint{8.557503in}{2.383888in}}%
\pgfusepath{}%
\end{pgfscope}%
\begin{pgfscope}%
\pgfpathrectangle{\pgfqpoint{0.549740in}{0.463273in}}{\pgfqpoint{9.320225in}{4.495057in}}%
\pgfusepath{clip}%
\pgfsetbuttcap%
\pgfsetroundjoin%
\pgfsetlinewidth{0.000000pt}%
\definecolor{currentstroke}{rgb}{0.000000,0.000000,0.000000}%
\pgfsetstrokecolor{currentstroke}%
\pgfsetdash{}{0pt}%
\pgfpathmoveto{\pgfqpoint{8.743729in}{2.383888in}}%
\pgfpathlineto{\pgfqpoint{8.929956in}{2.383888in}}%
\pgfpathlineto{\pgfqpoint{8.929956in}{2.465617in}}%
\pgfpathlineto{\pgfqpoint{8.743729in}{2.465617in}}%
\pgfpathlineto{\pgfqpoint{8.743729in}{2.383888in}}%
\pgfusepath{}%
\end{pgfscope}%
\begin{pgfscope}%
\pgfpathrectangle{\pgfqpoint{0.549740in}{0.463273in}}{\pgfqpoint{9.320225in}{4.495057in}}%
\pgfusepath{clip}%
\pgfsetbuttcap%
\pgfsetroundjoin%
\pgfsetlinewidth{0.000000pt}%
\definecolor{currentstroke}{rgb}{0.000000,0.000000,0.000000}%
\pgfsetstrokecolor{currentstroke}%
\pgfsetdash{}{0pt}%
\pgfpathmoveto{\pgfqpoint{8.929956in}{2.383888in}}%
\pgfpathlineto{\pgfqpoint{9.116182in}{2.383888in}}%
\pgfpathlineto{\pgfqpoint{9.116182in}{2.465617in}}%
\pgfpathlineto{\pgfqpoint{8.929956in}{2.465617in}}%
\pgfpathlineto{\pgfqpoint{8.929956in}{2.383888in}}%
\pgfusepath{}%
\end{pgfscope}%
\begin{pgfscope}%
\pgfpathrectangle{\pgfqpoint{0.549740in}{0.463273in}}{\pgfqpoint{9.320225in}{4.495057in}}%
\pgfusepath{clip}%
\pgfsetbuttcap%
\pgfsetroundjoin%
\pgfsetlinewidth{0.000000pt}%
\definecolor{currentstroke}{rgb}{0.000000,0.000000,0.000000}%
\pgfsetstrokecolor{currentstroke}%
\pgfsetdash{}{0pt}%
\pgfpathmoveto{\pgfqpoint{9.116182in}{2.383888in}}%
\pgfpathlineto{\pgfqpoint{9.302409in}{2.383888in}}%
\pgfpathlineto{\pgfqpoint{9.302409in}{2.465617in}}%
\pgfpathlineto{\pgfqpoint{9.116182in}{2.465617in}}%
\pgfpathlineto{\pgfqpoint{9.116182in}{2.383888in}}%
\pgfusepath{}%
\end{pgfscope}%
\begin{pgfscope}%
\pgfpathrectangle{\pgfqpoint{0.549740in}{0.463273in}}{\pgfqpoint{9.320225in}{4.495057in}}%
\pgfusepath{clip}%
\pgfsetbuttcap%
\pgfsetroundjoin%
\pgfsetlinewidth{0.000000pt}%
\definecolor{currentstroke}{rgb}{0.000000,0.000000,0.000000}%
\pgfsetstrokecolor{currentstroke}%
\pgfsetdash{}{0pt}%
\pgfpathmoveto{\pgfqpoint{9.302409in}{2.383888in}}%
\pgfpathlineto{\pgfqpoint{9.488635in}{2.383888in}}%
\pgfpathlineto{\pgfqpoint{9.488635in}{2.465617in}}%
\pgfpathlineto{\pgfqpoint{9.302409in}{2.465617in}}%
\pgfpathlineto{\pgfqpoint{9.302409in}{2.383888in}}%
\pgfusepath{}%
\end{pgfscope}%
\begin{pgfscope}%
\pgfpathrectangle{\pgfqpoint{0.549740in}{0.463273in}}{\pgfqpoint{9.320225in}{4.495057in}}%
\pgfusepath{clip}%
\pgfsetbuttcap%
\pgfsetroundjoin%
\pgfsetlinewidth{0.000000pt}%
\definecolor{currentstroke}{rgb}{0.000000,0.000000,0.000000}%
\pgfsetstrokecolor{currentstroke}%
\pgfsetdash{}{0pt}%
\pgfpathmoveto{\pgfqpoint{9.488635in}{2.383888in}}%
\pgfpathlineto{\pgfqpoint{9.674862in}{2.383888in}}%
\pgfpathlineto{\pgfqpoint{9.674862in}{2.465617in}}%
\pgfpathlineto{\pgfqpoint{9.488635in}{2.465617in}}%
\pgfpathlineto{\pgfqpoint{9.488635in}{2.383888in}}%
\pgfusepath{}%
\end{pgfscope}%
\begin{pgfscope}%
\pgfpathrectangle{\pgfqpoint{0.549740in}{0.463273in}}{\pgfqpoint{9.320225in}{4.495057in}}%
\pgfusepath{clip}%
\pgfsetbuttcap%
\pgfsetroundjoin%
\definecolor{currentfill}{rgb}{0.472869,0.711325,0.955316}%
\pgfsetfillcolor{currentfill}%
\pgfsetlinewidth{0.000000pt}%
\definecolor{currentstroke}{rgb}{0.000000,0.000000,0.000000}%
\pgfsetstrokecolor{currentstroke}%
\pgfsetdash{}{0pt}%
\pgfpathmoveto{\pgfqpoint{9.674862in}{2.383888in}}%
\pgfpathlineto{\pgfqpoint{9.861088in}{2.383888in}}%
\pgfpathlineto{\pgfqpoint{9.861088in}{2.465617in}}%
\pgfpathlineto{\pgfqpoint{9.674862in}{2.465617in}}%
\pgfpathlineto{\pgfqpoint{9.674862in}{2.383888in}}%
\pgfusepath{fill}%
\end{pgfscope}%
\begin{pgfscope}%
\pgfpathrectangle{\pgfqpoint{0.549740in}{0.463273in}}{\pgfqpoint{9.320225in}{4.495057in}}%
\pgfusepath{clip}%
\pgfsetbuttcap%
\pgfsetroundjoin%
\pgfsetlinewidth{0.000000pt}%
\definecolor{currentstroke}{rgb}{0.000000,0.000000,0.000000}%
\pgfsetstrokecolor{currentstroke}%
\pgfsetdash{}{0pt}%
\pgfpathmoveto{\pgfqpoint{0.549761in}{2.465617in}}%
\pgfpathlineto{\pgfqpoint{0.735988in}{2.465617in}}%
\pgfpathlineto{\pgfqpoint{0.735988in}{2.547345in}}%
\pgfpathlineto{\pgfqpoint{0.549761in}{2.547345in}}%
\pgfpathlineto{\pgfqpoint{0.549761in}{2.465617in}}%
\pgfusepath{}%
\end{pgfscope}%
\begin{pgfscope}%
\pgfpathrectangle{\pgfqpoint{0.549740in}{0.463273in}}{\pgfqpoint{9.320225in}{4.495057in}}%
\pgfusepath{clip}%
\pgfsetbuttcap%
\pgfsetroundjoin%
\pgfsetlinewidth{0.000000pt}%
\definecolor{currentstroke}{rgb}{0.000000,0.000000,0.000000}%
\pgfsetstrokecolor{currentstroke}%
\pgfsetdash{}{0pt}%
\pgfpathmoveto{\pgfqpoint{0.735988in}{2.465617in}}%
\pgfpathlineto{\pgfqpoint{0.922214in}{2.465617in}}%
\pgfpathlineto{\pgfqpoint{0.922214in}{2.547345in}}%
\pgfpathlineto{\pgfqpoint{0.735988in}{2.547345in}}%
\pgfpathlineto{\pgfqpoint{0.735988in}{2.465617in}}%
\pgfusepath{}%
\end{pgfscope}%
\begin{pgfscope}%
\pgfpathrectangle{\pgfqpoint{0.549740in}{0.463273in}}{\pgfqpoint{9.320225in}{4.495057in}}%
\pgfusepath{clip}%
\pgfsetbuttcap%
\pgfsetroundjoin%
\pgfsetlinewidth{0.000000pt}%
\definecolor{currentstroke}{rgb}{0.000000,0.000000,0.000000}%
\pgfsetstrokecolor{currentstroke}%
\pgfsetdash{}{0pt}%
\pgfpathmoveto{\pgfqpoint{0.922214in}{2.465617in}}%
\pgfpathlineto{\pgfqpoint{1.108441in}{2.465617in}}%
\pgfpathlineto{\pgfqpoint{1.108441in}{2.547345in}}%
\pgfpathlineto{\pgfqpoint{0.922214in}{2.547345in}}%
\pgfpathlineto{\pgfqpoint{0.922214in}{2.465617in}}%
\pgfusepath{}%
\end{pgfscope}%
\begin{pgfscope}%
\pgfpathrectangle{\pgfqpoint{0.549740in}{0.463273in}}{\pgfqpoint{9.320225in}{4.495057in}}%
\pgfusepath{clip}%
\pgfsetbuttcap%
\pgfsetroundjoin%
\pgfsetlinewidth{0.000000pt}%
\definecolor{currentstroke}{rgb}{0.000000,0.000000,0.000000}%
\pgfsetstrokecolor{currentstroke}%
\pgfsetdash{}{0pt}%
\pgfpathmoveto{\pgfqpoint{1.108441in}{2.465617in}}%
\pgfpathlineto{\pgfqpoint{1.294667in}{2.465617in}}%
\pgfpathlineto{\pgfqpoint{1.294667in}{2.547345in}}%
\pgfpathlineto{\pgfqpoint{1.108441in}{2.547345in}}%
\pgfpathlineto{\pgfqpoint{1.108441in}{2.465617in}}%
\pgfusepath{}%
\end{pgfscope}%
\begin{pgfscope}%
\pgfpathrectangle{\pgfqpoint{0.549740in}{0.463273in}}{\pgfqpoint{9.320225in}{4.495057in}}%
\pgfusepath{clip}%
\pgfsetbuttcap%
\pgfsetroundjoin%
\pgfsetlinewidth{0.000000pt}%
\definecolor{currentstroke}{rgb}{0.000000,0.000000,0.000000}%
\pgfsetstrokecolor{currentstroke}%
\pgfsetdash{}{0pt}%
\pgfpathmoveto{\pgfqpoint{1.294667in}{2.465617in}}%
\pgfpathlineto{\pgfqpoint{1.480894in}{2.465617in}}%
\pgfpathlineto{\pgfqpoint{1.480894in}{2.547345in}}%
\pgfpathlineto{\pgfqpoint{1.294667in}{2.547345in}}%
\pgfpathlineto{\pgfqpoint{1.294667in}{2.465617in}}%
\pgfusepath{}%
\end{pgfscope}%
\begin{pgfscope}%
\pgfpathrectangle{\pgfqpoint{0.549740in}{0.463273in}}{\pgfqpoint{9.320225in}{4.495057in}}%
\pgfusepath{clip}%
\pgfsetbuttcap%
\pgfsetroundjoin%
\pgfsetlinewidth{0.000000pt}%
\definecolor{currentstroke}{rgb}{0.000000,0.000000,0.000000}%
\pgfsetstrokecolor{currentstroke}%
\pgfsetdash{}{0pt}%
\pgfpathmoveto{\pgfqpoint{1.480894in}{2.465617in}}%
\pgfpathlineto{\pgfqpoint{1.667120in}{2.465617in}}%
\pgfpathlineto{\pgfqpoint{1.667120in}{2.547345in}}%
\pgfpathlineto{\pgfqpoint{1.480894in}{2.547345in}}%
\pgfpathlineto{\pgfqpoint{1.480894in}{2.465617in}}%
\pgfusepath{}%
\end{pgfscope}%
\begin{pgfscope}%
\pgfpathrectangle{\pgfqpoint{0.549740in}{0.463273in}}{\pgfqpoint{9.320225in}{4.495057in}}%
\pgfusepath{clip}%
\pgfsetbuttcap%
\pgfsetroundjoin%
\pgfsetlinewidth{0.000000pt}%
\definecolor{currentstroke}{rgb}{0.000000,0.000000,0.000000}%
\pgfsetstrokecolor{currentstroke}%
\pgfsetdash{}{0pt}%
\pgfpathmoveto{\pgfqpoint{1.667120in}{2.465617in}}%
\pgfpathlineto{\pgfqpoint{1.853347in}{2.465617in}}%
\pgfpathlineto{\pgfqpoint{1.853347in}{2.547345in}}%
\pgfpathlineto{\pgfqpoint{1.667120in}{2.547345in}}%
\pgfpathlineto{\pgfqpoint{1.667120in}{2.465617in}}%
\pgfusepath{}%
\end{pgfscope}%
\begin{pgfscope}%
\pgfpathrectangle{\pgfqpoint{0.549740in}{0.463273in}}{\pgfqpoint{9.320225in}{4.495057in}}%
\pgfusepath{clip}%
\pgfsetbuttcap%
\pgfsetroundjoin%
\pgfsetlinewidth{0.000000pt}%
\definecolor{currentstroke}{rgb}{0.000000,0.000000,0.000000}%
\pgfsetstrokecolor{currentstroke}%
\pgfsetdash{}{0pt}%
\pgfpathmoveto{\pgfqpoint{1.853347in}{2.465617in}}%
\pgfpathlineto{\pgfqpoint{2.039573in}{2.465617in}}%
\pgfpathlineto{\pgfqpoint{2.039573in}{2.547345in}}%
\pgfpathlineto{\pgfqpoint{1.853347in}{2.547345in}}%
\pgfpathlineto{\pgfqpoint{1.853347in}{2.465617in}}%
\pgfusepath{}%
\end{pgfscope}%
\begin{pgfscope}%
\pgfpathrectangle{\pgfqpoint{0.549740in}{0.463273in}}{\pgfqpoint{9.320225in}{4.495057in}}%
\pgfusepath{clip}%
\pgfsetbuttcap%
\pgfsetroundjoin%
\pgfsetlinewidth{0.000000pt}%
\definecolor{currentstroke}{rgb}{0.000000,0.000000,0.000000}%
\pgfsetstrokecolor{currentstroke}%
\pgfsetdash{}{0pt}%
\pgfpathmoveto{\pgfqpoint{2.039573in}{2.465617in}}%
\pgfpathlineto{\pgfqpoint{2.225800in}{2.465617in}}%
\pgfpathlineto{\pgfqpoint{2.225800in}{2.547345in}}%
\pgfpathlineto{\pgfqpoint{2.039573in}{2.547345in}}%
\pgfpathlineto{\pgfqpoint{2.039573in}{2.465617in}}%
\pgfusepath{}%
\end{pgfscope}%
\begin{pgfscope}%
\pgfpathrectangle{\pgfqpoint{0.549740in}{0.463273in}}{\pgfqpoint{9.320225in}{4.495057in}}%
\pgfusepath{clip}%
\pgfsetbuttcap%
\pgfsetroundjoin%
\pgfsetlinewidth{0.000000pt}%
\definecolor{currentstroke}{rgb}{0.000000,0.000000,0.000000}%
\pgfsetstrokecolor{currentstroke}%
\pgfsetdash{}{0pt}%
\pgfpathmoveto{\pgfqpoint{2.225800in}{2.465617in}}%
\pgfpathlineto{\pgfqpoint{2.412027in}{2.465617in}}%
\pgfpathlineto{\pgfqpoint{2.412027in}{2.547345in}}%
\pgfpathlineto{\pgfqpoint{2.225800in}{2.547345in}}%
\pgfpathlineto{\pgfqpoint{2.225800in}{2.465617in}}%
\pgfusepath{}%
\end{pgfscope}%
\begin{pgfscope}%
\pgfpathrectangle{\pgfqpoint{0.549740in}{0.463273in}}{\pgfqpoint{9.320225in}{4.495057in}}%
\pgfusepath{clip}%
\pgfsetbuttcap%
\pgfsetroundjoin%
\pgfsetlinewidth{0.000000pt}%
\definecolor{currentstroke}{rgb}{0.000000,0.000000,0.000000}%
\pgfsetstrokecolor{currentstroke}%
\pgfsetdash{}{0pt}%
\pgfpathmoveto{\pgfqpoint{2.412027in}{2.465617in}}%
\pgfpathlineto{\pgfqpoint{2.598253in}{2.465617in}}%
\pgfpathlineto{\pgfqpoint{2.598253in}{2.547345in}}%
\pgfpathlineto{\pgfqpoint{2.412027in}{2.547345in}}%
\pgfpathlineto{\pgfqpoint{2.412027in}{2.465617in}}%
\pgfusepath{}%
\end{pgfscope}%
\begin{pgfscope}%
\pgfpathrectangle{\pgfqpoint{0.549740in}{0.463273in}}{\pgfqpoint{9.320225in}{4.495057in}}%
\pgfusepath{clip}%
\pgfsetbuttcap%
\pgfsetroundjoin%
\pgfsetlinewidth{0.000000pt}%
\definecolor{currentstroke}{rgb}{0.000000,0.000000,0.000000}%
\pgfsetstrokecolor{currentstroke}%
\pgfsetdash{}{0pt}%
\pgfpathmoveto{\pgfqpoint{2.598253in}{2.465617in}}%
\pgfpathlineto{\pgfqpoint{2.784480in}{2.465617in}}%
\pgfpathlineto{\pgfqpoint{2.784480in}{2.547345in}}%
\pgfpathlineto{\pgfqpoint{2.598253in}{2.547345in}}%
\pgfpathlineto{\pgfqpoint{2.598253in}{2.465617in}}%
\pgfusepath{}%
\end{pgfscope}%
\begin{pgfscope}%
\pgfpathrectangle{\pgfqpoint{0.549740in}{0.463273in}}{\pgfqpoint{9.320225in}{4.495057in}}%
\pgfusepath{clip}%
\pgfsetbuttcap%
\pgfsetroundjoin%
\pgfsetlinewidth{0.000000pt}%
\definecolor{currentstroke}{rgb}{0.000000,0.000000,0.000000}%
\pgfsetstrokecolor{currentstroke}%
\pgfsetdash{}{0pt}%
\pgfpathmoveto{\pgfqpoint{2.784480in}{2.465617in}}%
\pgfpathlineto{\pgfqpoint{2.970706in}{2.465617in}}%
\pgfpathlineto{\pgfqpoint{2.970706in}{2.547345in}}%
\pgfpathlineto{\pgfqpoint{2.784480in}{2.547345in}}%
\pgfpathlineto{\pgfqpoint{2.784480in}{2.465617in}}%
\pgfusepath{}%
\end{pgfscope}%
\begin{pgfscope}%
\pgfpathrectangle{\pgfqpoint{0.549740in}{0.463273in}}{\pgfqpoint{9.320225in}{4.495057in}}%
\pgfusepath{clip}%
\pgfsetbuttcap%
\pgfsetroundjoin%
\pgfsetlinewidth{0.000000pt}%
\definecolor{currentstroke}{rgb}{0.000000,0.000000,0.000000}%
\pgfsetstrokecolor{currentstroke}%
\pgfsetdash{}{0pt}%
\pgfpathmoveto{\pgfqpoint{2.970706in}{2.465617in}}%
\pgfpathlineto{\pgfqpoint{3.156933in}{2.465617in}}%
\pgfpathlineto{\pgfqpoint{3.156933in}{2.547345in}}%
\pgfpathlineto{\pgfqpoint{2.970706in}{2.547345in}}%
\pgfpathlineto{\pgfqpoint{2.970706in}{2.465617in}}%
\pgfusepath{}%
\end{pgfscope}%
\begin{pgfscope}%
\pgfpathrectangle{\pgfqpoint{0.549740in}{0.463273in}}{\pgfqpoint{9.320225in}{4.495057in}}%
\pgfusepath{clip}%
\pgfsetbuttcap%
\pgfsetroundjoin%
\pgfsetlinewidth{0.000000pt}%
\definecolor{currentstroke}{rgb}{0.000000,0.000000,0.000000}%
\pgfsetstrokecolor{currentstroke}%
\pgfsetdash{}{0pt}%
\pgfpathmoveto{\pgfqpoint{3.156933in}{2.465617in}}%
\pgfpathlineto{\pgfqpoint{3.343159in}{2.465617in}}%
\pgfpathlineto{\pgfqpoint{3.343159in}{2.547345in}}%
\pgfpathlineto{\pgfqpoint{3.156933in}{2.547345in}}%
\pgfpathlineto{\pgfqpoint{3.156933in}{2.465617in}}%
\pgfusepath{}%
\end{pgfscope}%
\begin{pgfscope}%
\pgfpathrectangle{\pgfqpoint{0.549740in}{0.463273in}}{\pgfqpoint{9.320225in}{4.495057in}}%
\pgfusepath{clip}%
\pgfsetbuttcap%
\pgfsetroundjoin%
\pgfsetlinewidth{0.000000pt}%
\definecolor{currentstroke}{rgb}{0.000000,0.000000,0.000000}%
\pgfsetstrokecolor{currentstroke}%
\pgfsetdash{}{0pt}%
\pgfpathmoveto{\pgfqpoint{3.343159in}{2.465617in}}%
\pgfpathlineto{\pgfqpoint{3.529386in}{2.465617in}}%
\pgfpathlineto{\pgfqpoint{3.529386in}{2.547345in}}%
\pgfpathlineto{\pgfqpoint{3.343159in}{2.547345in}}%
\pgfpathlineto{\pgfqpoint{3.343159in}{2.465617in}}%
\pgfusepath{}%
\end{pgfscope}%
\begin{pgfscope}%
\pgfpathrectangle{\pgfqpoint{0.549740in}{0.463273in}}{\pgfqpoint{9.320225in}{4.495057in}}%
\pgfusepath{clip}%
\pgfsetbuttcap%
\pgfsetroundjoin%
\pgfsetlinewidth{0.000000pt}%
\definecolor{currentstroke}{rgb}{0.000000,0.000000,0.000000}%
\pgfsetstrokecolor{currentstroke}%
\pgfsetdash{}{0pt}%
\pgfpathmoveto{\pgfqpoint{3.529386in}{2.465617in}}%
\pgfpathlineto{\pgfqpoint{3.715612in}{2.465617in}}%
\pgfpathlineto{\pgfqpoint{3.715612in}{2.547345in}}%
\pgfpathlineto{\pgfqpoint{3.529386in}{2.547345in}}%
\pgfpathlineto{\pgfqpoint{3.529386in}{2.465617in}}%
\pgfusepath{}%
\end{pgfscope}%
\begin{pgfscope}%
\pgfpathrectangle{\pgfqpoint{0.549740in}{0.463273in}}{\pgfqpoint{9.320225in}{4.495057in}}%
\pgfusepath{clip}%
\pgfsetbuttcap%
\pgfsetroundjoin%
\pgfsetlinewidth{0.000000pt}%
\definecolor{currentstroke}{rgb}{0.000000,0.000000,0.000000}%
\pgfsetstrokecolor{currentstroke}%
\pgfsetdash{}{0pt}%
\pgfpathmoveto{\pgfqpoint{3.715612in}{2.465617in}}%
\pgfpathlineto{\pgfqpoint{3.901839in}{2.465617in}}%
\pgfpathlineto{\pgfqpoint{3.901839in}{2.547345in}}%
\pgfpathlineto{\pgfqpoint{3.715612in}{2.547345in}}%
\pgfpathlineto{\pgfqpoint{3.715612in}{2.465617in}}%
\pgfusepath{}%
\end{pgfscope}%
\begin{pgfscope}%
\pgfpathrectangle{\pgfqpoint{0.549740in}{0.463273in}}{\pgfqpoint{9.320225in}{4.495057in}}%
\pgfusepath{clip}%
\pgfsetbuttcap%
\pgfsetroundjoin%
\pgfsetlinewidth{0.000000pt}%
\definecolor{currentstroke}{rgb}{0.000000,0.000000,0.000000}%
\pgfsetstrokecolor{currentstroke}%
\pgfsetdash{}{0pt}%
\pgfpathmoveto{\pgfqpoint{3.901839in}{2.465617in}}%
\pgfpathlineto{\pgfqpoint{4.088065in}{2.465617in}}%
\pgfpathlineto{\pgfqpoint{4.088065in}{2.547345in}}%
\pgfpathlineto{\pgfqpoint{3.901839in}{2.547345in}}%
\pgfpathlineto{\pgfqpoint{3.901839in}{2.465617in}}%
\pgfusepath{}%
\end{pgfscope}%
\begin{pgfscope}%
\pgfpathrectangle{\pgfqpoint{0.549740in}{0.463273in}}{\pgfqpoint{9.320225in}{4.495057in}}%
\pgfusepath{clip}%
\pgfsetbuttcap%
\pgfsetroundjoin%
\pgfsetlinewidth{0.000000pt}%
\definecolor{currentstroke}{rgb}{0.000000,0.000000,0.000000}%
\pgfsetstrokecolor{currentstroke}%
\pgfsetdash{}{0pt}%
\pgfpathmoveto{\pgfqpoint{4.088065in}{2.465617in}}%
\pgfpathlineto{\pgfqpoint{4.274292in}{2.465617in}}%
\pgfpathlineto{\pgfqpoint{4.274292in}{2.547345in}}%
\pgfpathlineto{\pgfqpoint{4.088065in}{2.547345in}}%
\pgfpathlineto{\pgfqpoint{4.088065in}{2.465617in}}%
\pgfusepath{}%
\end{pgfscope}%
\begin{pgfscope}%
\pgfpathrectangle{\pgfqpoint{0.549740in}{0.463273in}}{\pgfqpoint{9.320225in}{4.495057in}}%
\pgfusepath{clip}%
\pgfsetbuttcap%
\pgfsetroundjoin%
\pgfsetlinewidth{0.000000pt}%
\definecolor{currentstroke}{rgb}{0.000000,0.000000,0.000000}%
\pgfsetstrokecolor{currentstroke}%
\pgfsetdash{}{0pt}%
\pgfpathmoveto{\pgfqpoint{4.274292in}{2.465617in}}%
\pgfpathlineto{\pgfqpoint{4.460519in}{2.465617in}}%
\pgfpathlineto{\pgfqpoint{4.460519in}{2.547345in}}%
\pgfpathlineto{\pgfqpoint{4.274292in}{2.547345in}}%
\pgfpathlineto{\pgfqpoint{4.274292in}{2.465617in}}%
\pgfusepath{}%
\end{pgfscope}%
\begin{pgfscope}%
\pgfpathrectangle{\pgfqpoint{0.549740in}{0.463273in}}{\pgfqpoint{9.320225in}{4.495057in}}%
\pgfusepath{clip}%
\pgfsetbuttcap%
\pgfsetroundjoin%
\pgfsetlinewidth{0.000000pt}%
\definecolor{currentstroke}{rgb}{0.000000,0.000000,0.000000}%
\pgfsetstrokecolor{currentstroke}%
\pgfsetdash{}{0pt}%
\pgfpathmoveto{\pgfqpoint{4.460519in}{2.465617in}}%
\pgfpathlineto{\pgfqpoint{4.646745in}{2.465617in}}%
\pgfpathlineto{\pgfqpoint{4.646745in}{2.547345in}}%
\pgfpathlineto{\pgfqpoint{4.460519in}{2.547345in}}%
\pgfpathlineto{\pgfqpoint{4.460519in}{2.465617in}}%
\pgfusepath{}%
\end{pgfscope}%
\begin{pgfscope}%
\pgfpathrectangle{\pgfqpoint{0.549740in}{0.463273in}}{\pgfqpoint{9.320225in}{4.495057in}}%
\pgfusepath{clip}%
\pgfsetbuttcap%
\pgfsetroundjoin%
\pgfsetlinewidth{0.000000pt}%
\definecolor{currentstroke}{rgb}{0.000000,0.000000,0.000000}%
\pgfsetstrokecolor{currentstroke}%
\pgfsetdash{}{0pt}%
\pgfpathmoveto{\pgfqpoint{4.646745in}{2.465617in}}%
\pgfpathlineto{\pgfqpoint{4.832972in}{2.465617in}}%
\pgfpathlineto{\pgfqpoint{4.832972in}{2.547345in}}%
\pgfpathlineto{\pgfqpoint{4.646745in}{2.547345in}}%
\pgfpathlineto{\pgfqpoint{4.646745in}{2.465617in}}%
\pgfusepath{}%
\end{pgfscope}%
\begin{pgfscope}%
\pgfpathrectangle{\pgfqpoint{0.549740in}{0.463273in}}{\pgfqpoint{9.320225in}{4.495057in}}%
\pgfusepath{clip}%
\pgfsetbuttcap%
\pgfsetroundjoin%
\pgfsetlinewidth{0.000000pt}%
\definecolor{currentstroke}{rgb}{0.000000,0.000000,0.000000}%
\pgfsetstrokecolor{currentstroke}%
\pgfsetdash{}{0pt}%
\pgfpathmoveto{\pgfqpoint{4.832972in}{2.465617in}}%
\pgfpathlineto{\pgfqpoint{5.019198in}{2.465617in}}%
\pgfpathlineto{\pgfqpoint{5.019198in}{2.547345in}}%
\pgfpathlineto{\pgfqpoint{4.832972in}{2.547345in}}%
\pgfpathlineto{\pgfqpoint{4.832972in}{2.465617in}}%
\pgfusepath{}%
\end{pgfscope}%
\begin{pgfscope}%
\pgfpathrectangle{\pgfqpoint{0.549740in}{0.463273in}}{\pgfqpoint{9.320225in}{4.495057in}}%
\pgfusepath{clip}%
\pgfsetbuttcap%
\pgfsetroundjoin%
\pgfsetlinewidth{0.000000pt}%
\definecolor{currentstroke}{rgb}{0.000000,0.000000,0.000000}%
\pgfsetstrokecolor{currentstroke}%
\pgfsetdash{}{0pt}%
\pgfpathmoveto{\pgfqpoint{5.019198in}{2.465617in}}%
\pgfpathlineto{\pgfqpoint{5.205425in}{2.465617in}}%
\pgfpathlineto{\pgfqpoint{5.205425in}{2.547345in}}%
\pgfpathlineto{\pgfqpoint{5.019198in}{2.547345in}}%
\pgfpathlineto{\pgfqpoint{5.019198in}{2.465617in}}%
\pgfusepath{}%
\end{pgfscope}%
\begin{pgfscope}%
\pgfpathrectangle{\pgfqpoint{0.549740in}{0.463273in}}{\pgfqpoint{9.320225in}{4.495057in}}%
\pgfusepath{clip}%
\pgfsetbuttcap%
\pgfsetroundjoin%
\pgfsetlinewidth{0.000000pt}%
\definecolor{currentstroke}{rgb}{0.000000,0.000000,0.000000}%
\pgfsetstrokecolor{currentstroke}%
\pgfsetdash{}{0pt}%
\pgfpathmoveto{\pgfqpoint{5.205425in}{2.465617in}}%
\pgfpathlineto{\pgfqpoint{5.391651in}{2.465617in}}%
\pgfpathlineto{\pgfqpoint{5.391651in}{2.547345in}}%
\pgfpathlineto{\pgfqpoint{5.205425in}{2.547345in}}%
\pgfpathlineto{\pgfqpoint{5.205425in}{2.465617in}}%
\pgfusepath{}%
\end{pgfscope}%
\begin{pgfscope}%
\pgfpathrectangle{\pgfqpoint{0.549740in}{0.463273in}}{\pgfqpoint{9.320225in}{4.495057in}}%
\pgfusepath{clip}%
\pgfsetbuttcap%
\pgfsetroundjoin%
\definecolor{currentfill}{rgb}{0.472869,0.711325,0.955316}%
\pgfsetfillcolor{currentfill}%
\pgfsetlinewidth{0.000000pt}%
\definecolor{currentstroke}{rgb}{0.000000,0.000000,0.000000}%
\pgfsetstrokecolor{currentstroke}%
\pgfsetdash{}{0pt}%
\pgfpathmoveto{\pgfqpoint{5.391651in}{2.465617in}}%
\pgfpathlineto{\pgfqpoint{5.577878in}{2.465617in}}%
\pgfpathlineto{\pgfqpoint{5.577878in}{2.547345in}}%
\pgfpathlineto{\pgfqpoint{5.391651in}{2.547345in}}%
\pgfpathlineto{\pgfqpoint{5.391651in}{2.465617in}}%
\pgfusepath{fill}%
\end{pgfscope}%
\begin{pgfscope}%
\pgfpathrectangle{\pgfqpoint{0.549740in}{0.463273in}}{\pgfqpoint{9.320225in}{4.495057in}}%
\pgfusepath{clip}%
\pgfsetbuttcap%
\pgfsetroundjoin%
\pgfsetlinewidth{0.000000pt}%
\definecolor{currentstroke}{rgb}{0.000000,0.000000,0.000000}%
\pgfsetstrokecolor{currentstroke}%
\pgfsetdash{}{0pt}%
\pgfpathmoveto{\pgfqpoint{5.577878in}{2.465617in}}%
\pgfpathlineto{\pgfqpoint{5.764104in}{2.465617in}}%
\pgfpathlineto{\pgfqpoint{5.764104in}{2.547345in}}%
\pgfpathlineto{\pgfqpoint{5.577878in}{2.547345in}}%
\pgfpathlineto{\pgfqpoint{5.577878in}{2.465617in}}%
\pgfusepath{}%
\end{pgfscope}%
\begin{pgfscope}%
\pgfpathrectangle{\pgfqpoint{0.549740in}{0.463273in}}{\pgfqpoint{9.320225in}{4.495057in}}%
\pgfusepath{clip}%
\pgfsetbuttcap%
\pgfsetroundjoin%
\pgfsetlinewidth{0.000000pt}%
\definecolor{currentstroke}{rgb}{0.000000,0.000000,0.000000}%
\pgfsetstrokecolor{currentstroke}%
\pgfsetdash{}{0pt}%
\pgfpathmoveto{\pgfqpoint{5.764104in}{2.465617in}}%
\pgfpathlineto{\pgfqpoint{5.950331in}{2.465617in}}%
\pgfpathlineto{\pgfqpoint{5.950331in}{2.547345in}}%
\pgfpathlineto{\pgfqpoint{5.764104in}{2.547345in}}%
\pgfpathlineto{\pgfqpoint{5.764104in}{2.465617in}}%
\pgfusepath{}%
\end{pgfscope}%
\begin{pgfscope}%
\pgfpathrectangle{\pgfqpoint{0.549740in}{0.463273in}}{\pgfqpoint{9.320225in}{4.495057in}}%
\pgfusepath{clip}%
\pgfsetbuttcap%
\pgfsetroundjoin%
\pgfsetlinewidth{0.000000pt}%
\definecolor{currentstroke}{rgb}{0.000000,0.000000,0.000000}%
\pgfsetstrokecolor{currentstroke}%
\pgfsetdash{}{0pt}%
\pgfpathmoveto{\pgfqpoint{5.950331in}{2.465617in}}%
\pgfpathlineto{\pgfqpoint{6.136557in}{2.465617in}}%
\pgfpathlineto{\pgfqpoint{6.136557in}{2.547345in}}%
\pgfpathlineto{\pgfqpoint{5.950331in}{2.547345in}}%
\pgfpathlineto{\pgfqpoint{5.950331in}{2.465617in}}%
\pgfusepath{}%
\end{pgfscope}%
\begin{pgfscope}%
\pgfpathrectangle{\pgfqpoint{0.549740in}{0.463273in}}{\pgfqpoint{9.320225in}{4.495057in}}%
\pgfusepath{clip}%
\pgfsetbuttcap%
\pgfsetroundjoin%
\definecolor{currentfill}{rgb}{0.472869,0.711325,0.955316}%
\pgfsetfillcolor{currentfill}%
\pgfsetlinewidth{0.000000pt}%
\definecolor{currentstroke}{rgb}{0.000000,0.000000,0.000000}%
\pgfsetstrokecolor{currentstroke}%
\pgfsetdash{}{0pt}%
\pgfpathmoveto{\pgfqpoint{6.136557in}{2.465617in}}%
\pgfpathlineto{\pgfqpoint{6.322784in}{2.465617in}}%
\pgfpathlineto{\pgfqpoint{6.322784in}{2.547345in}}%
\pgfpathlineto{\pgfqpoint{6.136557in}{2.547345in}}%
\pgfpathlineto{\pgfqpoint{6.136557in}{2.465617in}}%
\pgfusepath{fill}%
\end{pgfscope}%
\begin{pgfscope}%
\pgfpathrectangle{\pgfqpoint{0.549740in}{0.463273in}}{\pgfqpoint{9.320225in}{4.495057in}}%
\pgfusepath{clip}%
\pgfsetbuttcap%
\pgfsetroundjoin%
\pgfsetlinewidth{0.000000pt}%
\definecolor{currentstroke}{rgb}{0.000000,0.000000,0.000000}%
\pgfsetstrokecolor{currentstroke}%
\pgfsetdash{}{0pt}%
\pgfpathmoveto{\pgfqpoint{6.322784in}{2.465617in}}%
\pgfpathlineto{\pgfqpoint{6.509011in}{2.465617in}}%
\pgfpathlineto{\pgfqpoint{6.509011in}{2.547345in}}%
\pgfpathlineto{\pgfqpoint{6.322784in}{2.547345in}}%
\pgfpathlineto{\pgfqpoint{6.322784in}{2.465617in}}%
\pgfusepath{}%
\end{pgfscope}%
\begin{pgfscope}%
\pgfpathrectangle{\pgfqpoint{0.549740in}{0.463273in}}{\pgfqpoint{9.320225in}{4.495057in}}%
\pgfusepath{clip}%
\pgfsetbuttcap%
\pgfsetroundjoin%
\pgfsetlinewidth{0.000000pt}%
\definecolor{currentstroke}{rgb}{0.000000,0.000000,0.000000}%
\pgfsetstrokecolor{currentstroke}%
\pgfsetdash{}{0pt}%
\pgfpathmoveto{\pgfqpoint{6.509011in}{2.465617in}}%
\pgfpathlineto{\pgfqpoint{6.695237in}{2.465617in}}%
\pgfpathlineto{\pgfqpoint{6.695237in}{2.547345in}}%
\pgfpathlineto{\pgfqpoint{6.509011in}{2.547345in}}%
\pgfpathlineto{\pgfqpoint{6.509011in}{2.465617in}}%
\pgfusepath{}%
\end{pgfscope}%
\begin{pgfscope}%
\pgfpathrectangle{\pgfqpoint{0.549740in}{0.463273in}}{\pgfqpoint{9.320225in}{4.495057in}}%
\pgfusepath{clip}%
\pgfsetbuttcap%
\pgfsetroundjoin%
\pgfsetlinewidth{0.000000pt}%
\definecolor{currentstroke}{rgb}{0.000000,0.000000,0.000000}%
\pgfsetstrokecolor{currentstroke}%
\pgfsetdash{}{0pt}%
\pgfpathmoveto{\pgfqpoint{6.695237in}{2.465617in}}%
\pgfpathlineto{\pgfqpoint{6.881464in}{2.465617in}}%
\pgfpathlineto{\pgfqpoint{6.881464in}{2.547345in}}%
\pgfpathlineto{\pgfqpoint{6.695237in}{2.547345in}}%
\pgfpathlineto{\pgfqpoint{6.695237in}{2.465617in}}%
\pgfusepath{}%
\end{pgfscope}%
\begin{pgfscope}%
\pgfpathrectangle{\pgfqpoint{0.549740in}{0.463273in}}{\pgfqpoint{9.320225in}{4.495057in}}%
\pgfusepath{clip}%
\pgfsetbuttcap%
\pgfsetroundjoin%
\pgfsetlinewidth{0.000000pt}%
\definecolor{currentstroke}{rgb}{0.000000,0.000000,0.000000}%
\pgfsetstrokecolor{currentstroke}%
\pgfsetdash{}{0pt}%
\pgfpathmoveto{\pgfqpoint{6.881464in}{2.465617in}}%
\pgfpathlineto{\pgfqpoint{7.067690in}{2.465617in}}%
\pgfpathlineto{\pgfqpoint{7.067690in}{2.547345in}}%
\pgfpathlineto{\pgfqpoint{6.881464in}{2.547345in}}%
\pgfpathlineto{\pgfqpoint{6.881464in}{2.465617in}}%
\pgfusepath{}%
\end{pgfscope}%
\begin{pgfscope}%
\pgfpathrectangle{\pgfqpoint{0.549740in}{0.463273in}}{\pgfqpoint{9.320225in}{4.495057in}}%
\pgfusepath{clip}%
\pgfsetbuttcap%
\pgfsetroundjoin%
\pgfsetlinewidth{0.000000pt}%
\definecolor{currentstroke}{rgb}{0.000000,0.000000,0.000000}%
\pgfsetstrokecolor{currentstroke}%
\pgfsetdash{}{0pt}%
\pgfpathmoveto{\pgfqpoint{7.067690in}{2.465617in}}%
\pgfpathlineto{\pgfqpoint{7.253917in}{2.465617in}}%
\pgfpathlineto{\pgfqpoint{7.253917in}{2.547345in}}%
\pgfpathlineto{\pgfqpoint{7.067690in}{2.547345in}}%
\pgfpathlineto{\pgfqpoint{7.067690in}{2.465617in}}%
\pgfusepath{}%
\end{pgfscope}%
\begin{pgfscope}%
\pgfpathrectangle{\pgfqpoint{0.549740in}{0.463273in}}{\pgfqpoint{9.320225in}{4.495057in}}%
\pgfusepath{clip}%
\pgfsetbuttcap%
\pgfsetroundjoin%
\definecolor{currentfill}{rgb}{0.547810,0.736432,0.947518}%
\pgfsetfillcolor{currentfill}%
\pgfsetlinewidth{0.000000pt}%
\definecolor{currentstroke}{rgb}{0.000000,0.000000,0.000000}%
\pgfsetstrokecolor{currentstroke}%
\pgfsetdash{}{0pt}%
\pgfpathmoveto{\pgfqpoint{7.253917in}{2.465617in}}%
\pgfpathlineto{\pgfqpoint{7.440143in}{2.465617in}}%
\pgfpathlineto{\pgfqpoint{7.440143in}{2.547345in}}%
\pgfpathlineto{\pgfqpoint{7.253917in}{2.547345in}}%
\pgfpathlineto{\pgfqpoint{7.253917in}{2.465617in}}%
\pgfusepath{fill}%
\end{pgfscope}%
\begin{pgfscope}%
\pgfpathrectangle{\pgfqpoint{0.549740in}{0.463273in}}{\pgfqpoint{9.320225in}{4.495057in}}%
\pgfusepath{clip}%
\pgfsetbuttcap%
\pgfsetroundjoin%
\pgfsetlinewidth{0.000000pt}%
\definecolor{currentstroke}{rgb}{0.000000,0.000000,0.000000}%
\pgfsetstrokecolor{currentstroke}%
\pgfsetdash{}{0pt}%
\pgfpathmoveto{\pgfqpoint{7.440143in}{2.465617in}}%
\pgfpathlineto{\pgfqpoint{7.626370in}{2.465617in}}%
\pgfpathlineto{\pgfqpoint{7.626370in}{2.547345in}}%
\pgfpathlineto{\pgfqpoint{7.440143in}{2.547345in}}%
\pgfpathlineto{\pgfqpoint{7.440143in}{2.465617in}}%
\pgfusepath{}%
\end{pgfscope}%
\begin{pgfscope}%
\pgfpathrectangle{\pgfqpoint{0.549740in}{0.463273in}}{\pgfqpoint{9.320225in}{4.495057in}}%
\pgfusepath{clip}%
\pgfsetbuttcap%
\pgfsetroundjoin%
\pgfsetlinewidth{0.000000pt}%
\definecolor{currentstroke}{rgb}{0.000000,0.000000,0.000000}%
\pgfsetstrokecolor{currentstroke}%
\pgfsetdash{}{0pt}%
\pgfpathmoveto{\pgfqpoint{7.626370in}{2.465617in}}%
\pgfpathlineto{\pgfqpoint{7.812596in}{2.465617in}}%
\pgfpathlineto{\pgfqpoint{7.812596in}{2.547345in}}%
\pgfpathlineto{\pgfqpoint{7.626370in}{2.547345in}}%
\pgfpathlineto{\pgfqpoint{7.626370in}{2.465617in}}%
\pgfusepath{}%
\end{pgfscope}%
\begin{pgfscope}%
\pgfpathrectangle{\pgfqpoint{0.549740in}{0.463273in}}{\pgfqpoint{9.320225in}{4.495057in}}%
\pgfusepath{clip}%
\pgfsetbuttcap%
\pgfsetroundjoin%
\pgfsetlinewidth{0.000000pt}%
\definecolor{currentstroke}{rgb}{0.000000,0.000000,0.000000}%
\pgfsetstrokecolor{currentstroke}%
\pgfsetdash{}{0pt}%
\pgfpathmoveto{\pgfqpoint{7.812596in}{2.465617in}}%
\pgfpathlineto{\pgfqpoint{7.998823in}{2.465617in}}%
\pgfpathlineto{\pgfqpoint{7.998823in}{2.547345in}}%
\pgfpathlineto{\pgfqpoint{7.812596in}{2.547345in}}%
\pgfpathlineto{\pgfqpoint{7.812596in}{2.465617in}}%
\pgfusepath{}%
\end{pgfscope}%
\begin{pgfscope}%
\pgfpathrectangle{\pgfqpoint{0.549740in}{0.463273in}}{\pgfqpoint{9.320225in}{4.495057in}}%
\pgfusepath{clip}%
\pgfsetbuttcap%
\pgfsetroundjoin%
\pgfsetlinewidth{0.000000pt}%
\definecolor{currentstroke}{rgb}{0.000000,0.000000,0.000000}%
\pgfsetstrokecolor{currentstroke}%
\pgfsetdash{}{0pt}%
\pgfpathmoveto{\pgfqpoint{7.998823in}{2.465617in}}%
\pgfpathlineto{\pgfqpoint{8.185049in}{2.465617in}}%
\pgfpathlineto{\pgfqpoint{8.185049in}{2.547345in}}%
\pgfpathlineto{\pgfqpoint{7.998823in}{2.547345in}}%
\pgfpathlineto{\pgfqpoint{7.998823in}{2.465617in}}%
\pgfusepath{}%
\end{pgfscope}%
\begin{pgfscope}%
\pgfpathrectangle{\pgfqpoint{0.549740in}{0.463273in}}{\pgfqpoint{9.320225in}{4.495057in}}%
\pgfusepath{clip}%
\pgfsetbuttcap%
\pgfsetroundjoin%
\pgfsetlinewidth{0.000000pt}%
\definecolor{currentstroke}{rgb}{0.000000,0.000000,0.000000}%
\pgfsetstrokecolor{currentstroke}%
\pgfsetdash{}{0pt}%
\pgfpathmoveto{\pgfqpoint{8.185049in}{2.465617in}}%
\pgfpathlineto{\pgfqpoint{8.371276in}{2.465617in}}%
\pgfpathlineto{\pgfqpoint{8.371276in}{2.547345in}}%
\pgfpathlineto{\pgfqpoint{8.185049in}{2.547345in}}%
\pgfpathlineto{\pgfqpoint{8.185049in}{2.465617in}}%
\pgfusepath{}%
\end{pgfscope}%
\begin{pgfscope}%
\pgfpathrectangle{\pgfqpoint{0.549740in}{0.463273in}}{\pgfqpoint{9.320225in}{4.495057in}}%
\pgfusepath{clip}%
\pgfsetbuttcap%
\pgfsetroundjoin%
\definecolor{currentfill}{rgb}{0.472869,0.711325,0.955316}%
\pgfsetfillcolor{currentfill}%
\pgfsetlinewidth{0.000000pt}%
\definecolor{currentstroke}{rgb}{0.000000,0.000000,0.000000}%
\pgfsetstrokecolor{currentstroke}%
\pgfsetdash{}{0pt}%
\pgfpathmoveto{\pgfqpoint{8.371276in}{2.465617in}}%
\pgfpathlineto{\pgfqpoint{8.557503in}{2.465617in}}%
\pgfpathlineto{\pgfqpoint{8.557503in}{2.547345in}}%
\pgfpathlineto{\pgfqpoint{8.371276in}{2.547345in}}%
\pgfpathlineto{\pgfqpoint{8.371276in}{2.465617in}}%
\pgfusepath{fill}%
\end{pgfscope}%
\begin{pgfscope}%
\pgfpathrectangle{\pgfqpoint{0.549740in}{0.463273in}}{\pgfqpoint{9.320225in}{4.495057in}}%
\pgfusepath{clip}%
\pgfsetbuttcap%
\pgfsetroundjoin%
\pgfsetlinewidth{0.000000pt}%
\definecolor{currentstroke}{rgb}{0.000000,0.000000,0.000000}%
\pgfsetstrokecolor{currentstroke}%
\pgfsetdash{}{0pt}%
\pgfpathmoveto{\pgfqpoint{8.557503in}{2.465617in}}%
\pgfpathlineto{\pgfqpoint{8.743729in}{2.465617in}}%
\pgfpathlineto{\pgfqpoint{8.743729in}{2.547345in}}%
\pgfpathlineto{\pgfqpoint{8.557503in}{2.547345in}}%
\pgfpathlineto{\pgfqpoint{8.557503in}{2.465617in}}%
\pgfusepath{}%
\end{pgfscope}%
\begin{pgfscope}%
\pgfpathrectangle{\pgfqpoint{0.549740in}{0.463273in}}{\pgfqpoint{9.320225in}{4.495057in}}%
\pgfusepath{clip}%
\pgfsetbuttcap%
\pgfsetroundjoin%
\pgfsetlinewidth{0.000000pt}%
\definecolor{currentstroke}{rgb}{0.000000,0.000000,0.000000}%
\pgfsetstrokecolor{currentstroke}%
\pgfsetdash{}{0pt}%
\pgfpathmoveto{\pgfqpoint{8.743729in}{2.465617in}}%
\pgfpathlineto{\pgfqpoint{8.929956in}{2.465617in}}%
\pgfpathlineto{\pgfqpoint{8.929956in}{2.547345in}}%
\pgfpathlineto{\pgfqpoint{8.743729in}{2.547345in}}%
\pgfpathlineto{\pgfqpoint{8.743729in}{2.465617in}}%
\pgfusepath{}%
\end{pgfscope}%
\begin{pgfscope}%
\pgfpathrectangle{\pgfqpoint{0.549740in}{0.463273in}}{\pgfqpoint{9.320225in}{4.495057in}}%
\pgfusepath{clip}%
\pgfsetbuttcap%
\pgfsetroundjoin%
\pgfsetlinewidth{0.000000pt}%
\definecolor{currentstroke}{rgb}{0.000000,0.000000,0.000000}%
\pgfsetstrokecolor{currentstroke}%
\pgfsetdash{}{0pt}%
\pgfpathmoveto{\pgfqpoint{8.929956in}{2.465617in}}%
\pgfpathlineto{\pgfqpoint{9.116182in}{2.465617in}}%
\pgfpathlineto{\pgfqpoint{9.116182in}{2.547345in}}%
\pgfpathlineto{\pgfqpoint{8.929956in}{2.547345in}}%
\pgfpathlineto{\pgfqpoint{8.929956in}{2.465617in}}%
\pgfusepath{}%
\end{pgfscope}%
\begin{pgfscope}%
\pgfpathrectangle{\pgfqpoint{0.549740in}{0.463273in}}{\pgfqpoint{9.320225in}{4.495057in}}%
\pgfusepath{clip}%
\pgfsetbuttcap%
\pgfsetroundjoin%
\pgfsetlinewidth{0.000000pt}%
\definecolor{currentstroke}{rgb}{0.000000,0.000000,0.000000}%
\pgfsetstrokecolor{currentstroke}%
\pgfsetdash{}{0pt}%
\pgfpathmoveto{\pgfqpoint{9.116182in}{2.465617in}}%
\pgfpathlineto{\pgfqpoint{9.302409in}{2.465617in}}%
\pgfpathlineto{\pgfqpoint{9.302409in}{2.547345in}}%
\pgfpathlineto{\pgfqpoint{9.116182in}{2.547345in}}%
\pgfpathlineto{\pgfqpoint{9.116182in}{2.465617in}}%
\pgfusepath{}%
\end{pgfscope}%
\begin{pgfscope}%
\pgfpathrectangle{\pgfqpoint{0.549740in}{0.463273in}}{\pgfqpoint{9.320225in}{4.495057in}}%
\pgfusepath{clip}%
\pgfsetbuttcap%
\pgfsetroundjoin%
\pgfsetlinewidth{0.000000pt}%
\definecolor{currentstroke}{rgb}{0.000000,0.000000,0.000000}%
\pgfsetstrokecolor{currentstroke}%
\pgfsetdash{}{0pt}%
\pgfpathmoveto{\pgfqpoint{9.302409in}{2.465617in}}%
\pgfpathlineto{\pgfqpoint{9.488635in}{2.465617in}}%
\pgfpathlineto{\pgfqpoint{9.488635in}{2.547345in}}%
\pgfpathlineto{\pgfqpoint{9.302409in}{2.547345in}}%
\pgfpathlineto{\pgfqpoint{9.302409in}{2.465617in}}%
\pgfusepath{}%
\end{pgfscope}%
\begin{pgfscope}%
\pgfpathrectangle{\pgfqpoint{0.549740in}{0.463273in}}{\pgfqpoint{9.320225in}{4.495057in}}%
\pgfusepath{clip}%
\pgfsetbuttcap%
\pgfsetroundjoin%
\pgfsetlinewidth{0.000000pt}%
\definecolor{currentstroke}{rgb}{0.000000,0.000000,0.000000}%
\pgfsetstrokecolor{currentstroke}%
\pgfsetdash{}{0pt}%
\pgfpathmoveto{\pgfqpoint{9.488635in}{2.465617in}}%
\pgfpathlineto{\pgfqpoint{9.674862in}{2.465617in}}%
\pgfpathlineto{\pgfqpoint{9.674862in}{2.547345in}}%
\pgfpathlineto{\pgfqpoint{9.488635in}{2.547345in}}%
\pgfpathlineto{\pgfqpoint{9.488635in}{2.465617in}}%
\pgfusepath{}%
\end{pgfscope}%
\begin{pgfscope}%
\pgfpathrectangle{\pgfqpoint{0.549740in}{0.463273in}}{\pgfqpoint{9.320225in}{4.495057in}}%
\pgfusepath{clip}%
\pgfsetbuttcap%
\pgfsetroundjoin%
\definecolor{currentfill}{rgb}{0.189527,0.635753,0.950228}%
\pgfsetfillcolor{currentfill}%
\pgfsetlinewidth{0.000000pt}%
\definecolor{currentstroke}{rgb}{0.000000,0.000000,0.000000}%
\pgfsetstrokecolor{currentstroke}%
\pgfsetdash{}{0pt}%
\pgfpathmoveto{\pgfqpoint{9.674862in}{2.465617in}}%
\pgfpathlineto{\pgfqpoint{9.861088in}{2.465617in}}%
\pgfpathlineto{\pgfqpoint{9.861088in}{2.547345in}}%
\pgfpathlineto{\pgfqpoint{9.674862in}{2.547345in}}%
\pgfpathlineto{\pgfqpoint{9.674862in}{2.465617in}}%
\pgfusepath{fill}%
\end{pgfscope}%
\begin{pgfscope}%
\pgfpathrectangle{\pgfqpoint{0.549740in}{0.463273in}}{\pgfqpoint{9.320225in}{4.495057in}}%
\pgfusepath{clip}%
\pgfsetbuttcap%
\pgfsetroundjoin%
\pgfsetlinewidth{0.000000pt}%
\definecolor{currentstroke}{rgb}{0.000000,0.000000,0.000000}%
\pgfsetstrokecolor{currentstroke}%
\pgfsetdash{}{0pt}%
\pgfpathmoveto{\pgfqpoint{0.549761in}{2.547345in}}%
\pgfpathlineto{\pgfqpoint{0.735988in}{2.547345in}}%
\pgfpathlineto{\pgfqpoint{0.735988in}{2.629073in}}%
\pgfpathlineto{\pgfqpoint{0.549761in}{2.629073in}}%
\pgfpathlineto{\pgfqpoint{0.549761in}{2.547345in}}%
\pgfusepath{}%
\end{pgfscope}%
\begin{pgfscope}%
\pgfpathrectangle{\pgfqpoint{0.549740in}{0.463273in}}{\pgfqpoint{9.320225in}{4.495057in}}%
\pgfusepath{clip}%
\pgfsetbuttcap%
\pgfsetroundjoin%
\pgfsetlinewidth{0.000000pt}%
\definecolor{currentstroke}{rgb}{0.000000,0.000000,0.000000}%
\pgfsetstrokecolor{currentstroke}%
\pgfsetdash{}{0pt}%
\pgfpathmoveto{\pgfqpoint{0.735988in}{2.547345in}}%
\pgfpathlineto{\pgfqpoint{0.922214in}{2.547345in}}%
\pgfpathlineto{\pgfqpoint{0.922214in}{2.629073in}}%
\pgfpathlineto{\pgfqpoint{0.735988in}{2.629073in}}%
\pgfpathlineto{\pgfqpoint{0.735988in}{2.547345in}}%
\pgfusepath{}%
\end{pgfscope}%
\begin{pgfscope}%
\pgfpathrectangle{\pgfqpoint{0.549740in}{0.463273in}}{\pgfqpoint{9.320225in}{4.495057in}}%
\pgfusepath{clip}%
\pgfsetbuttcap%
\pgfsetroundjoin%
\pgfsetlinewidth{0.000000pt}%
\definecolor{currentstroke}{rgb}{0.000000,0.000000,0.000000}%
\pgfsetstrokecolor{currentstroke}%
\pgfsetdash{}{0pt}%
\pgfpathmoveto{\pgfqpoint{0.922214in}{2.547345in}}%
\pgfpathlineto{\pgfqpoint{1.108441in}{2.547345in}}%
\pgfpathlineto{\pgfqpoint{1.108441in}{2.629073in}}%
\pgfpathlineto{\pgfqpoint{0.922214in}{2.629073in}}%
\pgfpathlineto{\pgfqpoint{0.922214in}{2.547345in}}%
\pgfusepath{}%
\end{pgfscope}%
\begin{pgfscope}%
\pgfpathrectangle{\pgfqpoint{0.549740in}{0.463273in}}{\pgfqpoint{9.320225in}{4.495057in}}%
\pgfusepath{clip}%
\pgfsetbuttcap%
\pgfsetroundjoin%
\pgfsetlinewidth{0.000000pt}%
\definecolor{currentstroke}{rgb}{0.000000,0.000000,0.000000}%
\pgfsetstrokecolor{currentstroke}%
\pgfsetdash{}{0pt}%
\pgfpathmoveto{\pgfqpoint{1.108441in}{2.547345in}}%
\pgfpathlineto{\pgfqpoint{1.294667in}{2.547345in}}%
\pgfpathlineto{\pgfqpoint{1.294667in}{2.629073in}}%
\pgfpathlineto{\pgfqpoint{1.108441in}{2.629073in}}%
\pgfpathlineto{\pgfqpoint{1.108441in}{2.547345in}}%
\pgfusepath{}%
\end{pgfscope}%
\begin{pgfscope}%
\pgfpathrectangle{\pgfqpoint{0.549740in}{0.463273in}}{\pgfqpoint{9.320225in}{4.495057in}}%
\pgfusepath{clip}%
\pgfsetbuttcap%
\pgfsetroundjoin%
\pgfsetlinewidth{0.000000pt}%
\definecolor{currentstroke}{rgb}{0.000000,0.000000,0.000000}%
\pgfsetstrokecolor{currentstroke}%
\pgfsetdash{}{0pt}%
\pgfpathmoveto{\pgfqpoint{1.294667in}{2.547345in}}%
\pgfpathlineto{\pgfqpoint{1.480894in}{2.547345in}}%
\pgfpathlineto{\pgfqpoint{1.480894in}{2.629073in}}%
\pgfpathlineto{\pgfqpoint{1.294667in}{2.629073in}}%
\pgfpathlineto{\pgfqpoint{1.294667in}{2.547345in}}%
\pgfusepath{}%
\end{pgfscope}%
\begin{pgfscope}%
\pgfpathrectangle{\pgfqpoint{0.549740in}{0.463273in}}{\pgfqpoint{9.320225in}{4.495057in}}%
\pgfusepath{clip}%
\pgfsetbuttcap%
\pgfsetroundjoin%
\pgfsetlinewidth{0.000000pt}%
\definecolor{currentstroke}{rgb}{0.000000,0.000000,0.000000}%
\pgfsetstrokecolor{currentstroke}%
\pgfsetdash{}{0pt}%
\pgfpathmoveto{\pgfqpoint{1.480894in}{2.547345in}}%
\pgfpathlineto{\pgfqpoint{1.667120in}{2.547345in}}%
\pgfpathlineto{\pgfqpoint{1.667120in}{2.629073in}}%
\pgfpathlineto{\pgfqpoint{1.480894in}{2.629073in}}%
\pgfpathlineto{\pgfqpoint{1.480894in}{2.547345in}}%
\pgfusepath{}%
\end{pgfscope}%
\begin{pgfscope}%
\pgfpathrectangle{\pgfqpoint{0.549740in}{0.463273in}}{\pgfqpoint{9.320225in}{4.495057in}}%
\pgfusepath{clip}%
\pgfsetbuttcap%
\pgfsetroundjoin%
\pgfsetlinewidth{0.000000pt}%
\definecolor{currentstroke}{rgb}{0.000000,0.000000,0.000000}%
\pgfsetstrokecolor{currentstroke}%
\pgfsetdash{}{0pt}%
\pgfpathmoveto{\pgfqpoint{1.667120in}{2.547345in}}%
\pgfpathlineto{\pgfqpoint{1.853347in}{2.547345in}}%
\pgfpathlineto{\pgfqpoint{1.853347in}{2.629073in}}%
\pgfpathlineto{\pgfqpoint{1.667120in}{2.629073in}}%
\pgfpathlineto{\pgfqpoint{1.667120in}{2.547345in}}%
\pgfusepath{}%
\end{pgfscope}%
\begin{pgfscope}%
\pgfpathrectangle{\pgfqpoint{0.549740in}{0.463273in}}{\pgfqpoint{9.320225in}{4.495057in}}%
\pgfusepath{clip}%
\pgfsetbuttcap%
\pgfsetroundjoin%
\pgfsetlinewidth{0.000000pt}%
\definecolor{currentstroke}{rgb}{0.000000,0.000000,0.000000}%
\pgfsetstrokecolor{currentstroke}%
\pgfsetdash{}{0pt}%
\pgfpathmoveto{\pgfqpoint{1.853347in}{2.547345in}}%
\pgfpathlineto{\pgfqpoint{2.039573in}{2.547345in}}%
\pgfpathlineto{\pgfqpoint{2.039573in}{2.629073in}}%
\pgfpathlineto{\pgfqpoint{1.853347in}{2.629073in}}%
\pgfpathlineto{\pgfqpoint{1.853347in}{2.547345in}}%
\pgfusepath{}%
\end{pgfscope}%
\begin{pgfscope}%
\pgfpathrectangle{\pgfqpoint{0.549740in}{0.463273in}}{\pgfqpoint{9.320225in}{4.495057in}}%
\pgfusepath{clip}%
\pgfsetbuttcap%
\pgfsetroundjoin%
\pgfsetlinewidth{0.000000pt}%
\definecolor{currentstroke}{rgb}{0.000000,0.000000,0.000000}%
\pgfsetstrokecolor{currentstroke}%
\pgfsetdash{}{0pt}%
\pgfpathmoveto{\pgfqpoint{2.039573in}{2.547345in}}%
\pgfpathlineto{\pgfqpoint{2.225800in}{2.547345in}}%
\pgfpathlineto{\pgfqpoint{2.225800in}{2.629073in}}%
\pgfpathlineto{\pgfqpoint{2.039573in}{2.629073in}}%
\pgfpathlineto{\pgfqpoint{2.039573in}{2.547345in}}%
\pgfusepath{}%
\end{pgfscope}%
\begin{pgfscope}%
\pgfpathrectangle{\pgfqpoint{0.549740in}{0.463273in}}{\pgfqpoint{9.320225in}{4.495057in}}%
\pgfusepath{clip}%
\pgfsetbuttcap%
\pgfsetroundjoin%
\pgfsetlinewidth{0.000000pt}%
\definecolor{currentstroke}{rgb}{0.000000,0.000000,0.000000}%
\pgfsetstrokecolor{currentstroke}%
\pgfsetdash{}{0pt}%
\pgfpathmoveto{\pgfqpoint{2.225800in}{2.547345in}}%
\pgfpathlineto{\pgfqpoint{2.412027in}{2.547345in}}%
\pgfpathlineto{\pgfqpoint{2.412027in}{2.629073in}}%
\pgfpathlineto{\pgfqpoint{2.225800in}{2.629073in}}%
\pgfpathlineto{\pgfqpoint{2.225800in}{2.547345in}}%
\pgfusepath{}%
\end{pgfscope}%
\begin{pgfscope}%
\pgfpathrectangle{\pgfqpoint{0.549740in}{0.463273in}}{\pgfqpoint{9.320225in}{4.495057in}}%
\pgfusepath{clip}%
\pgfsetbuttcap%
\pgfsetroundjoin%
\pgfsetlinewidth{0.000000pt}%
\definecolor{currentstroke}{rgb}{0.000000,0.000000,0.000000}%
\pgfsetstrokecolor{currentstroke}%
\pgfsetdash{}{0pt}%
\pgfpathmoveto{\pgfqpoint{2.412027in}{2.547345in}}%
\pgfpathlineto{\pgfqpoint{2.598253in}{2.547345in}}%
\pgfpathlineto{\pgfqpoint{2.598253in}{2.629073in}}%
\pgfpathlineto{\pgfqpoint{2.412027in}{2.629073in}}%
\pgfpathlineto{\pgfqpoint{2.412027in}{2.547345in}}%
\pgfusepath{}%
\end{pgfscope}%
\begin{pgfscope}%
\pgfpathrectangle{\pgfqpoint{0.549740in}{0.463273in}}{\pgfqpoint{9.320225in}{4.495057in}}%
\pgfusepath{clip}%
\pgfsetbuttcap%
\pgfsetroundjoin%
\pgfsetlinewidth{0.000000pt}%
\definecolor{currentstroke}{rgb}{0.000000,0.000000,0.000000}%
\pgfsetstrokecolor{currentstroke}%
\pgfsetdash{}{0pt}%
\pgfpathmoveto{\pgfqpoint{2.598253in}{2.547345in}}%
\pgfpathlineto{\pgfqpoint{2.784480in}{2.547345in}}%
\pgfpathlineto{\pgfqpoint{2.784480in}{2.629073in}}%
\pgfpathlineto{\pgfqpoint{2.598253in}{2.629073in}}%
\pgfpathlineto{\pgfqpoint{2.598253in}{2.547345in}}%
\pgfusepath{}%
\end{pgfscope}%
\begin{pgfscope}%
\pgfpathrectangle{\pgfqpoint{0.549740in}{0.463273in}}{\pgfqpoint{9.320225in}{4.495057in}}%
\pgfusepath{clip}%
\pgfsetbuttcap%
\pgfsetroundjoin%
\pgfsetlinewidth{0.000000pt}%
\definecolor{currentstroke}{rgb}{0.000000,0.000000,0.000000}%
\pgfsetstrokecolor{currentstroke}%
\pgfsetdash{}{0pt}%
\pgfpathmoveto{\pgfqpoint{2.784480in}{2.547345in}}%
\pgfpathlineto{\pgfqpoint{2.970706in}{2.547345in}}%
\pgfpathlineto{\pgfqpoint{2.970706in}{2.629073in}}%
\pgfpathlineto{\pgfqpoint{2.784480in}{2.629073in}}%
\pgfpathlineto{\pgfqpoint{2.784480in}{2.547345in}}%
\pgfusepath{}%
\end{pgfscope}%
\begin{pgfscope}%
\pgfpathrectangle{\pgfqpoint{0.549740in}{0.463273in}}{\pgfqpoint{9.320225in}{4.495057in}}%
\pgfusepath{clip}%
\pgfsetbuttcap%
\pgfsetroundjoin%
\pgfsetlinewidth{0.000000pt}%
\definecolor{currentstroke}{rgb}{0.000000,0.000000,0.000000}%
\pgfsetstrokecolor{currentstroke}%
\pgfsetdash{}{0pt}%
\pgfpathmoveto{\pgfqpoint{2.970706in}{2.547345in}}%
\pgfpathlineto{\pgfqpoint{3.156933in}{2.547345in}}%
\pgfpathlineto{\pgfqpoint{3.156933in}{2.629073in}}%
\pgfpathlineto{\pgfqpoint{2.970706in}{2.629073in}}%
\pgfpathlineto{\pgfqpoint{2.970706in}{2.547345in}}%
\pgfusepath{}%
\end{pgfscope}%
\begin{pgfscope}%
\pgfpathrectangle{\pgfqpoint{0.549740in}{0.463273in}}{\pgfqpoint{9.320225in}{4.495057in}}%
\pgfusepath{clip}%
\pgfsetbuttcap%
\pgfsetroundjoin%
\pgfsetlinewidth{0.000000pt}%
\definecolor{currentstroke}{rgb}{0.000000,0.000000,0.000000}%
\pgfsetstrokecolor{currentstroke}%
\pgfsetdash{}{0pt}%
\pgfpathmoveto{\pgfqpoint{3.156933in}{2.547345in}}%
\pgfpathlineto{\pgfqpoint{3.343159in}{2.547345in}}%
\pgfpathlineto{\pgfqpoint{3.343159in}{2.629073in}}%
\pgfpathlineto{\pgfqpoint{3.156933in}{2.629073in}}%
\pgfpathlineto{\pgfqpoint{3.156933in}{2.547345in}}%
\pgfusepath{}%
\end{pgfscope}%
\begin{pgfscope}%
\pgfpathrectangle{\pgfqpoint{0.549740in}{0.463273in}}{\pgfqpoint{9.320225in}{4.495057in}}%
\pgfusepath{clip}%
\pgfsetbuttcap%
\pgfsetroundjoin%
\pgfsetlinewidth{0.000000pt}%
\definecolor{currentstroke}{rgb}{0.000000,0.000000,0.000000}%
\pgfsetstrokecolor{currentstroke}%
\pgfsetdash{}{0pt}%
\pgfpathmoveto{\pgfqpoint{3.343159in}{2.547345in}}%
\pgfpathlineto{\pgfqpoint{3.529386in}{2.547345in}}%
\pgfpathlineto{\pgfqpoint{3.529386in}{2.629073in}}%
\pgfpathlineto{\pgfqpoint{3.343159in}{2.629073in}}%
\pgfpathlineto{\pgfqpoint{3.343159in}{2.547345in}}%
\pgfusepath{}%
\end{pgfscope}%
\begin{pgfscope}%
\pgfpathrectangle{\pgfqpoint{0.549740in}{0.463273in}}{\pgfqpoint{9.320225in}{4.495057in}}%
\pgfusepath{clip}%
\pgfsetbuttcap%
\pgfsetroundjoin%
\pgfsetlinewidth{0.000000pt}%
\definecolor{currentstroke}{rgb}{0.000000,0.000000,0.000000}%
\pgfsetstrokecolor{currentstroke}%
\pgfsetdash{}{0pt}%
\pgfpathmoveto{\pgfqpoint{3.529386in}{2.547345in}}%
\pgfpathlineto{\pgfqpoint{3.715612in}{2.547345in}}%
\pgfpathlineto{\pgfqpoint{3.715612in}{2.629073in}}%
\pgfpathlineto{\pgfqpoint{3.529386in}{2.629073in}}%
\pgfpathlineto{\pgfqpoint{3.529386in}{2.547345in}}%
\pgfusepath{}%
\end{pgfscope}%
\begin{pgfscope}%
\pgfpathrectangle{\pgfqpoint{0.549740in}{0.463273in}}{\pgfqpoint{9.320225in}{4.495057in}}%
\pgfusepath{clip}%
\pgfsetbuttcap%
\pgfsetroundjoin%
\pgfsetlinewidth{0.000000pt}%
\definecolor{currentstroke}{rgb}{0.000000,0.000000,0.000000}%
\pgfsetstrokecolor{currentstroke}%
\pgfsetdash{}{0pt}%
\pgfpathmoveto{\pgfqpoint{3.715612in}{2.547345in}}%
\pgfpathlineto{\pgfqpoint{3.901839in}{2.547345in}}%
\pgfpathlineto{\pgfqpoint{3.901839in}{2.629073in}}%
\pgfpathlineto{\pgfqpoint{3.715612in}{2.629073in}}%
\pgfpathlineto{\pgfqpoint{3.715612in}{2.547345in}}%
\pgfusepath{}%
\end{pgfscope}%
\begin{pgfscope}%
\pgfpathrectangle{\pgfqpoint{0.549740in}{0.463273in}}{\pgfqpoint{9.320225in}{4.495057in}}%
\pgfusepath{clip}%
\pgfsetbuttcap%
\pgfsetroundjoin%
\pgfsetlinewidth{0.000000pt}%
\definecolor{currentstroke}{rgb}{0.000000,0.000000,0.000000}%
\pgfsetstrokecolor{currentstroke}%
\pgfsetdash{}{0pt}%
\pgfpathmoveto{\pgfqpoint{3.901839in}{2.547345in}}%
\pgfpathlineto{\pgfqpoint{4.088065in}{2.547345in}}%
\pgfpathlineto{\pgfqpoint{4.088065in}{2.629073in}}%
\pgfpathlineto{\pgfqpoint{3.901839in}{2.629073in}}%
\pgfpathlineto{\pgfqpoint{3.901839in}{2.547345in}}%
\pgfusepath{}%
\end{pgfscope}%
\begin{pgfscope}%
\pgfpathrectangle{\pgfqpoint{0.549740in}{0.463273in}}{\pgfqpoint{9.320225in}{4.495057in}}%
\pgfusepath{clip}%
\pgfsetbuttcap%
\pgfsetroundjoin%
\pgfsetlinewidth{0.000000pt}%
\definecolor{currentstroke}{rgb}{0.000000,0.000000,0.000000}%
\pgfsetstrokecolor{currentstroke}%
\pgfsetdash{}{0pt}%
\pgfpathmoveto{\pgfqpoint{4.088065in}{2.547345in}}%
\pgfpathlineto{\pgfqpoint{4.274292in}{2.547345in}}%
\pgfpathlineto{\pgfqpoint{4.274292in}{2.629073in}}%
\pgfpathlineto{\pgfqpoint{4.088065in}{2.629073in}}%
\pgfpathlineto{\pgfqpoint{4.088065in}{2.547345in}}%
\pgfusepath{}%
\end{pgfscope}%
\begin{pgfscope}%
\pgfpathrectangle{\pgfqpoint{0.549740in}{0.463273in}}{\pgfqpoint{9.320225in}{4.495057in}}%
\pgfusepath{clip}%
\pgfsetbuttcap%
\pgfsetroundjoin%
\pgfsetlinewidth{0.000000pt}%
\definecolor{currentstroke}{rgb}{0.000000,0.000000,0.000000}%
\pgfsetstrokecolor{currentstroke}%
\pgfsetdash{}{0pt}%
\pgfpathmoveto{\pgfqpoint{4.274292in}{2.547345in}}%
\pgfpathlineto{\pgfqpoint{4.460519in}{2.547345in}}%
\pgfpathlineto{\pgfqpoint{4.460519in}{2.629073in}}%
\pgfpathlineto{\pgfqpoint{4.274292in}{2.629073in}}%
\pgfpathlineto{\pgfqpoint{4.274292in}{2.547345in}}%
\pgfusepath{}%
\end{pgfscope}%
\begin{pgfscope}%
\pgfpathrectangle{\pgfqpoint{0.549740in}{0.463273in}}{\pgfqpoint{9.320225in}{4.495057in}}%
\pgfusepath{clip}%
\pgfsetbuttcap%
\pgfsetroundjoin%
\pgfsetlinewidth{0.000000pt}%
\definecolor{currentstroke}{rgb}{0.000000,0.000000,0.000000}%
\pgfsetstrokecolor{currentstroke}%
\pgfsetdash{}{0pt}%
\pgfpathmoveto{\pgfqpoint{4.460519in}{2.547345in}}%
\pgfpathlineto{\pgfqpoint{4.646745in}{2.547345in}}%
\pgfpathlineto{\pgfqpoint{4.646745in}{2.629073in}}%
\pgfpathlineto{\pgfqpoint{4.460519in}{2.629073in}}%
\pgfpathlineto{\pgfqpoint{4.460519in}{2.547345in}}%
\pgfusepath{}%
\end{pgfscope}%
\begin{pgfscope}%
\pgfpathrectangle{\pgfqpoint{0.549740in}{0.463273in}}{\pgfqpoint{9.320225in}{4.495057in}}%
\pgfusepath{clip}%
\pgfsetbuttcap%
\pgfsetroundjoin%
\pgfsetlinewidth{0.000000pt}%
\definecolor{currentstroke}{rgb}{0.000000,0.000000,0.000000}%
\pgfsetstrokecolor{currentstroke}%
\pgfsetdash{}{0pt}%
\pgfpathmoveto{\pgfqpoint{4.646745in}{2.547345in}}%
\pgfpathlineto{\pgfqpoint{4.832972in}{2.547345in}}%
\pgfpathlineto{\pgfqpoint{4.832972in}{2.629073in}}%
\pgfpathlineto{\pgfqpoint{4.646745in}{2.629073in}}%
\pgfpathlineto{\pgfqpoint{4.646745in}{2.547345in}}%
\pgfusepath{}%
\end{pgfscope}%
\begin{pgfscope}%
\pgfpathrectangle{\pgfqpoint{0.549740in}{0.463273in}}{\pgfqpoint{9.320225in}{4.495057in}}%
\pgfusepath{clip}%
\pgfsetbuttcap%
\pgfsetroundjoin%
\pgfsetlinewidth{0.000000pt}%
\definecolor{currentstroke}{rgb}{0.000000,0.000000,0.000000}%
\pgfsetstrokecolor{currentstroke}%
\pgfsetdash{}{0pt}%
\pgfpathmoveto{\pgfqpoint{4.832972in}{2.547345in}}%
\pgfpathlineto{\pgfqpoint{5.019198in}{2.547345in}}%
\pgfpathlineto{\pgfqpoint{5.019198in}{2.629073in}}%
\pgfpathlineto{\pgfqpoint{4.832972in}{2.629073in}}%
\pgfpathlineto{\pgfqpoint{4.832972in}{2.547345in}}%
\pgfusepath{}%
\end{pgfscope}%
\begin{pgfscope}%
\pgfpathrectangle{\pgfqpoint{0.549740in}{0.463273in}}{\pgfqpoint{9.320225in}{4.495057in}}%
\pgfusepath{clip}%
\pgfsetbuttcap%
\pgfsetroundjoin%
\pgfsetlinewidth{0.000000pt}%
\definecolor{currentstroke}{rgb}{0.000000,0.000000,0.000000}%
\pgfsetstrokecolor{currentstroke}%
\pgfsetdash{}{0pt}%
\pgfpathmoveto{\pgfqpoint{5.019198in}{2.547345in}}%
\pgfpathlineto{\pgfqpoint{5.205425in}{2.547345in}}%
\pgfpathlineto{\pgfqpoint{5.205425in}{2.629073in}}%
\pgfpathlineto{\pgfqpoint{5.019198in}{2.629073in}}%
\pgfpathlineto{\pgfqpoint{5.019198in}{2.547345in}}%
\pgfusepath{}%
\end{pgfscope}%
\begin{pgfscope}%
\pgfpathrectangle{\pgfqpoint{0.549740in}{0.463273in}}{\pgfqpoint{9.320225in}{4.495057in}}%
\pgfusepath{clip}%
\pgfsetbuttcap%
\pgfsetroundjoin%
\definecolor{currentfill}{rgb}{0.614330,0.761948,0.940009}%
\pgfsetfillcolor{currentfill}%
\pgfsetlinewidth{0.000000pt}%
\definecolor{currentstroke}{rgb}{0.000000,0.000000,0.000000}%
\pgfsetstrokecolor{currentstroke}%
\pgfsetdash{}{0pt}%
\pgfpathmoveto{\pgfqpoint{5.205425in}{2.547345in}}%
\pgfpathlineto{\pgfqpoint{5.391651in}{2.547345in}}%
\pgfpathlineto{\pgfqpoint{5.391651in}{2.629073in}}%
\pgfpathlineto{\pgfqpoint{5.205425in}{2.629073in}}%
\pgfpathlineto{\pgfqpoint{5.205425in}{2.547345in}}%
\pgfusepath{fill}%
\end{pgfscope}%
\begin{pgfscope}%
\pgfpathrectangle{\pgfqpoint{0.549740in}{0.463273in}}{\pgfqpoint{9.320225in}{4.495057in}}%
\pgfusepath{clip}%
\pgfsetbuttcap%
\pgfsetroundjoin%
\definecolor{currentfill}{rgb}{0.547810,0.736432,0.947518}%
\pgfsetfillcolor{currentfill}%
\pgfsetlinewidth{0.000000pt}%
\definecolor{currentstroke}{rgb}{0.000000,0.000000,0.000000}%
\pgfsetstrokecolor{currentstroke}%
\pgfsetdash{}{0pt}%
\pgfpathmoveto{\pgfqpoint{5.391651in}{2.547345in}}%
\pgfpathlineto{\pgfqpoint{5.577878in}{2.547345in}}%
\pgfpathlineto{\pgfqpoint{5.577878in}{2.629073in}}%
\pgfpathlineto{\pgfqpoint{5.391651in}{2.629073in}}%
\pgfpathlineto{\pgfqpoint{5.391651in}{2.547345in}}%
\pgfusepath{fill}%
\end{pgfscope}%
\begin{pgfscope}%
\pgfpathrectangle{\pgfqpoint{0.549740in}{0.463273in}}{\pgfqpoint{9.320225in}{4.495057in}}%
\pgfusepath{clip}%
\pgfsetbuttcap%
\pgfsetroundjoin%
\pgfsetlinewidth{0.000000pt}%
\definecolor{currentstroke}{rgb}{0.000000,0.000000,0.000000}%
\pgfsetstrokecolor{currentstroke}%
\pgfsetdash{}{0pt}%
\pgfpathmoveto{\pgfqpoint{5.577878in}{2.547345in}}%
\pgfpathlineto{\pgfqpoint{5.764104in}{2.547345in}}%
\pgfpathlineto{\pgfqpoint{5.764104in}{2.629073in}}%
\pgfpathlineto{\pgfqpoint{5.577878in}{2.629073in}}%
\pgfpathlineto{\pgfqpoint{5.577878in}{2.547345in}}%
\pgfusepath{}%
\end{pgfscope}%
\begin{pgfscope}%
\pgfpathrectangle{\pgfqpoint{0.549740in}{0.463273in}}{\pgfqpoint{9.320225in}{4.495057in}}%
\pgfusepath{clip}%
\pgfsetbuttcap%
\pgfsetroundjoin%
\pgfsetlinewidth{0.000000pt}%
\definecolor{currentstroke}{rgb}{0.000000,0.000000,0.000000}%
\pgfsetstrokecolor{currentstroke}%
\pgfsetdash{}{0pt}%
\pgfpathmoveto{\pgfqpoint{5.764104in}{2.547345in}}%
\pgfpathlineto{\pgfqpoint{5.950331in}{2.547345in}}%
\pgfpathlineto{\pgfqpoint{5.950331in}{2.629073in}}%
\pgfpathlineto{\pgfqpoint{5.764104in}{2.629073in}}%
\pgfpathlineto{\pgfqpoint{5.764104in}{2.547345in}}%
\pgfusepath{}%
\end{pgfscope}%
\begin{pgfscope}%
\pgfpathrectangle{\pgfqpoint{0.549740in}{0.463273in}}{\pgfqpoint{9.320225in}{4.495057in}}%
\pgfusepath{clip}%
\pgfsetbuttcap%
\pgfsetroundjoin%
\pgfsetlinewidth{0.000000pt}%
\definecolor{currentstroke}{rgb}{0.000000,0.000000,0.000000}%
\pgfsetstrokecolor{currentstroke}%
\pgfsetdash{}{0pt}%
\pgfpathmoveto{\pgfqpoint{5.950331in}{2.547345in}}%
\pgfpathlineto{\pgfqpoint{6.136557in}{2.547345in}}%
\pgfpathlineto{\pgfqpoint{6.136557in}{2.629073in}}%
\pgfpathlineto{\pgfqpoint{5.950331in}{2.629073in}}%
\pgfpathlineto{\pgfqpoint{5.950331in}{2.547345in}}%
\pgfusepath{}%
\end{pgfscope}%
\begin{pgfscope}%
\pgfpathrectangle{\pgfqpoint{0.549740in}{0.463273in}}{\pgfqpoint{9.320225in}{4.495057in}}%
\pgfusepath{clip}%
\pgfsetbuttcap%
\pgfsetroundjoin%
\definecolor{currentfill}{rgb}{0.547810,0.736432,0.947518}%
\pgfsetfillcolor{currentfill}%
\pgfsetlinewidth{0.000000pt}%
\definecolor{currentstroke}{rgb}{0.000000,0.000000,0.000000}%
\pgfsetstrokecolor{currentstroke}%
\pgfsetdash{}{0pt}%
\pgfpathmoveto{\pgfqpoint{6.136557in}{2.547345in}}%
\pgfpathlineto{\pgfqpoint{6.322784in}{2.547345in}}%
\pgfpathlineto{\pgfqpoint{6.322784in}{2.629073in}}%
\pgfpathlineto{\pgfqpoint{6.136557in}{2.629073in}}%
\pgfpathlineto{\pgfqpoint{6.136557in}{2.547345in}}%
\pgfusepath{fill}%
\end{pgfscope}%
\begin{pgfscope}%
\pgfpathrectangle{\pgfqpoint{0.549740in}{0.463273in}}{\pgfqpoint{9.320225in}{4.495057in}}%
\pgfusepath{clip}%
\pgfsetbuttcap%
\pgfsetroundjoin%
\pgfsetlinewidth{0.000000pt}%
\definecolor{currentstroke}{rgb}{0.000000,0.000000,0.000000}%
\pgfsetstrokecolor{currentstroke}%
\pgfsetdash{}{0pt}%
\pgfpathmoveto{\pgfqpoint{6.322784in}{2.547345in}}%
\pgfpathlineto{\pgfqpoint{6.509011in}{2.547345in}}%
\pgfpathlineto{\pgfqpoint{6.509011in}{2.629073in}}%
\pgfpathlineto{\pgfqpoint{6.322784in}{2.629073in}}%
\pgfpathlineto{\pgfqpoint{6.322784in}{2.547345in}}%
\pgfusepath{}%
\end{pgfscope}%
\begin{pgfscope}%
\pgfpathrectangle{\pgfqpoint{0.549740in}{0.463273in}}{\pgfqpoint{9.320225in}{4.495057in}}%
\pgfusepath{clip}%
\pgfsetbuttcap%
\pgfsetroundjoin%
\pgfsetlinewidth{0.000000pt}%
\definecolor{currentstroke}{rgb}{0.000000,0.000000,0.000000}%
\pgfsetstrokecolor{currentstroke}%
\pgfsetdash{}{0pt}%
\pgfpathmoveto{\pgfqpoint{6.509011in}{2.547345in}}%
\pgfpathlineto{\pgfqpoint{6.695237in}{2.547345in}}%
\pgfpathlineto{\pgfqpoint{6.695237in}{2.629073in}}%
\pgfpathlineto{\pgfqpoint{6.509011in}{2.629073in}}%
\pgfpathlineto{\pgfqpoint{6.509011in}{2.547345in}}%
\pgfusepath{}%
\end{pgfscope}%
\begin{pgfscope}%
\pgfpathrectangle{\pgfqpoint{0.549740in}{0.463273in}}{\pgfqpoint{9.320225in}{4.495057in}}%
\pgfusepath{clip}%
\pgfsetbuttcap%
\pgfsetroundjoin%
\pgfsetlinewidth{0.000000pt}%
\definecolor{currentstroke}{rgb}{0.000000,0.000000,0.000000}%
\pgfsetstrokecolor{currentstroke}%
\pgfsetdash{}{0pt}%
\pgfpathmoveto{\pgfqpoint{6.695237in}{2.547345in}}%
\pgfpathlineto{\pgfqpoint{6.881464in}{2.547345in}}%
\pgfpathlineto{\pgfqpoint{6.881464in}{2.629073in}}%
\pgfpathlineto{\pgfqpoint{6.695237in}{2.629073in}}%
\pgfpathlineto{\pgfqpoint{6.695237in}{2.547345in}}%
\pgfusepath{}%
\end{pgfscope}%
\begin{pgfscope}%
\pgfpathrectangle{\pgfqpoint{0.549740in}{0.463273in}}{\pgfqpoint{9.320225in}{4.495057in}}%
\pgfusepath{clip}%
\pgfsetbuttcap%
\pgfsetroundjoin%
\pgfsetlinewidth{0.000000pt}%
\definecolor{currentstroke}{rgb}{0.000000,0.000000,0.000000}%
\pgfsetstrokecolor{currentstroke}%
\pgfsetdash{}{0pt}%
\pgfpathmoveto{\pgfqpoint{6.881464in}{2.547345in}}%
\pgfpathlineto{\pgfqpoint{7.067690in}{2.547345in}}%
\pgfpathlineto{\pgfqpoint{7.067690in}{2.629073in}}%
\pgfpathlineto{\pgfqpoint{6.881464in}{2.629073in}}%
\pgfpathlineto{\pgfqpoint{6.881464in}{2.547345in}}%
\pgfusepath{}%
\end{pgfscope}%
\begin{pgfscope}%
\pgfpathrectangle{\pgfqpoint{0.549740in}{0.463273in}}{\pgfqpoint{9.320225in}{4.495057in}}%
\pgfusepath{clip}%
\pgfsetbuttcap%
\pgfsetroundjoin%
\pgfsetlinewidth{0.000000pt}%
\definecolor{currentstroke}{rgb}{0.000000,0.000000,0.000000}%
\pgfsetstrokecolor{currentstroke}%
\pgfsetdash{}{0pt}%
\pgfpathmoveto{\pgfqpoint{7.067690in}{2.547345in}}%
\pgfpathlineto{\pgfqpoint{7.253917in}{2.547345in}}%
\pgfpathlineto{\pgfqpoint{7.253917in}{2.629073in}}%
\pgfpathlineto{\pgfqpoint{7.067690in}{2.629073in}}%
\pgfpathlineto{\pgfqpoint{7.067690in}{2.547345in}}%
\pgfusepath{}%
\end{pgfscope}%
\begin{pgfscope}%
\pgfpathrectangle{\pgfqpoint{0.549740in}{0.463273in}}{\pgfqpoint{9.320225in}{4.495057in}}%
\pgfusepath{clip}%
\pgfsetbuttcap%
\pgfsetroundjoin%
\definecolor{currentfill}{rgb}{0.472869,0.711325,0.955316}%
\pgfsetfillcolor{currentfill}%
\pgfsetlinewidth{0.000000pt}%
\definecolor{currentstroke}{rgb}{0.000000,0.000000,0.000000}%
\pgfsetstrokecolor{currentstroke}%
\pgfsetdash{}{0pt}%
\pgfpathmoveto{\pgfqpoint{7.253917in}{2.547345in}}%
\pgfpathlineto{\pgfqpoint{7.440143in}{2.547345in}}%
\pgfpathlineto{\pgfqpoint{7.440143in}{2.629073in}}%
\pgfpathlineto{\pgfqpoint{7.253917in}{2.629073in}}%
\pgfpathlineto{\pgfqpoint{7.253917in}{2.547345in}}%
\pgfusepath{fill}%
\end{pgfscope}%
\begin{pgfscope}%
\pgfpathrectangle{\pgfqpoint{0.549740in}{0.463273in}}{\pgfqpoint{9.320225in}{4.495057in}}%
\pgfusepath{clip}%
\pgfsetbuttcap%
\pgfsetroundjoin%
\pgfsetlinewidth{0.000000pt}%
\definecolor{currentstroke}{rgb}{0.000000,0.000000,0.000000}%
\pgfsetstrokecolor{currentstroke}%
\pgfsetdash{}{0pt}%
\pgfpathmoveto{\pgfqpoint{7.440143in}{2.547345in}}%
\pgfpathlineto{\pgfqpoint{7.626370in}{2.547345in}}%
\pgfpathlineto{\pgfqpoint{7.626370in}{2.629073in}}%
\pgfpathlineto{\pgfqpoint{7.440143in}{2.629073in}}%
\pgfpathlineto{\pgfqpoint{7.440143in}{2.547345in}}%
\pgfusepath{}%
\end{pgfscope}%
\begin{pgfscope}%
\pgfpathrectangle{\pgfqpoint{0.549740in}{0.463273in}}{\pgfqpoint{9.320225in}{4.495057in}}%
\pgfusepath{clip}%
\pgfsetbuttcap%
\pgfsetroundjoin%
\pgfsetlinewidth{0.000000pt}%
\definecolor{currentstroke}{rgb}{0.000000,0.000000,0.000000}%
\pgfsetstrokecolor{currentstroke}%
\pgfsetdash{}{0pt}%
\pgfpathmoveto{\pgfqpoint{7.626370in}{2.547345in}}%
\pgfpathlineto{\pgfqpoint{7.812596in}{2.547345in}}%
\pgfpathlineto{\pgfqpoint{7.812596in}{2.629073in}}%
\pgfpathlineto{\pgfqpoint{7.626370in}{2.629073in}}%
\pgfpathlineto{\pgfqpoint{7.626370in}{2.547345in}}%
\pgfusepath{}%
\end{pgfscope}%
\begin{pgfscope}%
\pgfpathrectangle{\pgfqpoint{0.549740in}{0.463273in}}{\pgfqpoint{9.320225in}{4.495057in}}%
\pgfusepath{clip}%
\pgfsetbuttcap%
\pgfsetroundjoin%
\pgfsetlinewidth{0.000000pt}%
\definecolor{currentstroke}{rgb}{0.000000,0.000000,0.000000}%
\pgfsetstrokecolor{currentstroke}%
\pgfsetdash{}{0pt}%
\pgfpathmoveto{\pgfqpoint{7.812596in}{2.547345in}}%
\pgfpathlineto{\pgfqpoint{7.998823in}{2.547345in}}%
\pgfpathlineto{\pgfqpoint{7.998823in}{2.629073in}}%
\pgfpathlineto{\pgfqpoint{7.812596in}{2.629073in}}%
\pgfpathlineto{\pgfqpoint{7.812596in}{2.547345in}}%
\pgfusepath{}%
\end{pgfscope}%
\begin{pgfscope}%
\pgfpathrectangle{\pgfqpoint{0.549740in}{0.463273in}}{\pgfqpoint{9.320225in}{4.495057in}}%
\pgfusepath{clip}%
\pgfsetbuttcap%
\pgfsetroundjoin%
\pgfsetlinewidth{0.000000pt}%
\definecolor{currentstroke}{rgb}{0.000000,0.000000,0.000000}%
\pgfsetstrokecolor{currentstroke}%
\pgfsetdash{}{0pt}%
\pgfpathmoveto{\pgfqpoint{7.998823in}{2.547345in}}%
\pgfpathlineto{\pgfqpoint{8.185049in}{2.547345in}}%
\pgfpathlineto{\pgfqpoint{8.185049in}{2.629073in}}%
\pgfpathlineto{\pgfqpoint{7.998823in}{2.629073in}}%
\pgfpathlineto{\pgfqpoint{7.998823in}{2.547345in}}%
\pgfusepath{}%
\end{pgfscope}%
\begin{pgfscope}%
\pgfpathrectangle{\pgfqpoint{0.549740in}{0.463273in}}{\pgfqpoint{9.320225in}{4.495057in}}%
\pgfusepath{clip}%
\pgfsetbuttcap%
\pgfsetroundjoin%
\pgfsetlinewidth{0.000000pt}%
\definecolor{currentstroke}{rgb}{0.000000,0.000000,0.000000}%
\pgfsetstrokecolor{currentstroke}%
\pgfsetdash{}{0pt}%
\pgfpathmoveto{\pgfqpoint{8.185049in}{2.547345in}}%
\pgfpathlineto{\pgfqpoint{8.371276in}{2.547345in}}%
\pgfpathlineto{\pgfqpoint{8.371276in}{2.629073in}}%
\pgfpathlineto{\pgfqpoint{8.185049in}{2.629073in}}%
\pgfpathlineto{\pgfqpoint{8.185049in}{2.547345in}}%
\pgfusepath{}%
\end{pgfscope}%
\begin{pgfscope}%
\pgfpathrectangle{\pgfqpoint{0.549740in}{0.463273in}}{\pgfqpoint{9.320225in}{4.495057in}}%
\pgfusepath{clip}%
\pgfsetbuttcap%
\pgfsetroundjoin%
\definecolor{currentfill}{rgb}{0.472869,0.711325,0.955316}%
\pgfsetfillcolor{currentfill}%
\pgfsetlinewidth{0.000000pt}%
\definecolor{currentstroke}{rgb}{0.000000,0.000000,0.000000}%
\pgfsetstrokecolor{currentstroke}%
\pgfsetdash{}{0pt}%
\pgfpathmoveto{\pgfqpoint{8.371276in}{2.547345in}}%
\pgfpathlineto{\pgfqpoint{8.557503in}{2.547345in}}%
\pgfpathlineto{\pgfqpoint{8.557503in}{2.629073in}}%
\pgfpathlineto{\pgfqpoint{8.371276in}{2.629073in}}%
\pgfpathlineto{\pgfqpoint{8.371276in}{2.547345in}}%
\pgfusepath{fill}%
\end{pgfscope}%
\begin{pgfscope}%
\pgfpathrectangle{\pgfqpoint{0.549740in}{0.463273in}}{\pgfqpoint{9.320225in}{4.495057in}}%
\pgfusepath{clip}%
\pgfsetbuttcap%
\pgfsetroundjoin%
\pgfsetlinewidth{0.000000pt}%
\definecolor{currentstroke}{rgb}{0.000000,0.000000,0.000000}%
\pgfsetstrokecolor{currentstroke}%
\pgfsetdash{}{0pt}%
\pgfpathmoveto{\pgfqpoint{8.557503in}{2.547345in}}%
\pgfpathlineto{\pgfqpoint{8.743729in}{2.547345in}}%
\pgfpathlineto{\pgfqpoint{8.743729in}{2.629073in}}%
\pgfpathlineto{\pgfqpoint{8.557503in}{2.629073in}}%
\pgfpathlineto{\pgfqpoint{8.557503in}{2.547345in}}%
\pgfusepath{}%
\end{pgfscope}%
\begin{pgfscope}%
\pgfpathrectangle{\pgfqpoint{0.549740in}{0.463273in}}{\pgfqpoint{9.320225in}{4.495057in}}%
\pgfusepath{clip}%
\pgfsetbuttcap%
\pgfsetroundjoin%
\pgfsetlinewidth{0.000000pt}%
\definecolor{currentstroke}{rgb}{0.000000,0.000000,0.000000}%
\pgfsetstrokecolor{currentstroke}%
\pgfsetdash{}{0pt}%
\pgfpathmoveto{\pgfqpoint{8.743729in}{2.547345in}}%
\pgfpathlineto{\pgfqpoint{8.929956in}{2.547345in}}%
\pgfpathlineto{\pgfqpoint{8.929956in}{2.629073in}}%
\pgfpathlineto{\pgfqpoint{8.743729in}{2.629073in}}%
\pgfpathlineto{\pgfqpoint{8.743729in}{2.547345in}}%
\pgfusepath{}%
\end{pgfscope}%
\begin{pgfscope}%
\pgfpathrectangle{\pgfqpoint{0.549740in}{0.463273in}}{\pgfqpoint{9.320225in}{4.495057in}}%
\pgfusepath{clip}%
\pgfsetbuttcap%
\pgfsetroundjoin%
\pgfsetlinewidth{0.000000pt}%
\definecolor{currentstroke}{rgb}{0.000000,0.000000,0.000000}%
\pgfsetstrokecolor{currentstroke}%
\pgfsetdash{}{0pt}%
\pgfpathmoveto{\pgfqpoint{8.929956in}{2.547345in}}%
\pgfpathlineto{\pgfqpoint{9.116182in}{2.547345in}}%
\pgfpathlineto{\pgfqpoint{9.116182in}{2.629073in}}%
\pgfpathlineto{\pgfqpoint{8.929956in}{2.629073in}}%
\pgfpathlineto{\pgfqpoint{8.929956in}{2.547345in}}%
\pgfusepath{}%
\end{pgfscope}%
\begin{pgfscope}%
\pgfpathrectangle{\pgfqpoint{0.549740in}{0.463273in}}{\pgfqpoint{9.320225in}{4.495057in}}%
\pgfusepath{clip}%
\pgfsetbuttcap%
\pgfsetroundjoin%
\pgfsetlinewidth{0.000000pt}%
\definecolor{currentstroke}{rgb}{0.000000,0.000000,0.000000}%
\pgfsetstrokecolor{currentstroke}%
\pgfsetdash{}{0pt}%
\pgfpathmoveto{\pgfqpoint{9.116182in}{2.547345in}}%
\pgfpathlineto{\pgfqpoint{9.302409in}{2.547345in}}%
\pgfpathlineto{\pgfqpoint{9.302409in}{2.629073in}}%
\pgfpathlineto{\pgfqpoint{9.116182in}{2.629073in}}%
\pgfpathlineto{\pgfqpoint{9.116182in}{2.547345in}}%
\pgfusepath{}%
\end{pgfscope}%
\begin{pgfscope}%
\pgfpathrectangle{\pgfqpoint{0.549740in}{0.463273in}}{\pgfqpoint{9.320225in}{4.495057in}}%
\pgfusepath{clip}%
\pgfsetbuttcap%
\pgfsetroundjoin%
\pgfsetlinewidth{0.000000pt}%
\definecolor{currentstroke}{rgb}{0.000000,0.000000,0.000000}%
\pgfsetstrokecolor{currentstroke}%
\pgfsetdash{}{0pt}%
\pgfpathmoveto{\pgfqpoint{9.302409in}{2.547345in}}%
\pgfpathlineto{\pgfqpoint{9.488635in}{2.547345in}}%
\pgfpathlineto{\pgfqpoint{9.488635in}{2.629073in}}%
\pgfpathlineto{\pgfqpoint{9.302409in}{2.629073in}}%
\pgfpathlineto{\pgfqpoint{9.302409in}{2.547345in}}%
\pgfusepath{}%
\end{pgfscope}%
\begin{pgfscope}%
\pgfpathrectangle{\pgfqpoint{0.549740in}{0.463273in}}{\pgfqpoint{9.320225in}{4.495057in}}%
\pgfusepath{clip}%
\pgfsetbuttcap%
\pgfsetroundjoin%
\definecolor{currentfill}{rgb}{0.614330,0.761948,0.940009}%
\pgfsetfillcolor{currentfill}%
\pgfsetlinewidth{0.000000pt}%
\definecolor{currentstroke}{rgb}{0.000000,0.000000,0.000000}%
\pgfsetstrokecolor{currentstroke}%
\pgfsetdash{}{0pt}%
\pgfpathmoveto{\pgfqpoint{9.488635in}{2.547345in}}%
\pgfpathlineto{\pgfqpoint{9.674862in}{2.547345in}}%
\pgfpathlineto{\pgfqpoint{9.674862in}{2.629073in}}%
\pgfpathlineto{\pgfqpoint{9.488635in}{2.629073in}}%
\pgfpathlineto{\pgfqpoint{9.488635in}{2.547345in}}%
\pgfusepath{fill}%
\end{pgfscope}%
\begin{pgfscope}%
\pgfpathrectangle{\pgfqpoint{0.549740in}{0.463273in}}{\pgfqpoint{9.320225in}{4.495057in}}%
\pgfusepath{clip}%
\pgfsetbuttcap%
\pgfsetroundjoin%
\definecolor{currentfill}{rgb}{0.273225,0.662144,0.968515}%
\pgfsetfillcolor{currentfill}%
\pgfsetlinewidth{0.000000pt}%
\definecolor{currentstroke}{rgb}{0.000000,0.000000,0.000000}%
\pgfsetstrokecolor{currentstroke}%
\pgfsetdash{}{0pt}%
\pgfpathmoveto{\pgfqpoint{9.674862in}{2.547345in}}%
\pgfpathlineto{\pgfqpoint{9.861088in}{2.547345in}}%
\pgfpathlineto{\pgfqpoint{9.861088in}{2.629073in}}%
\pgfpathlineto{\pgfqpoint{9.674862in}{2.629073in}}%
\pgfpathlineto{\pgfqpoint{9.674862in}{2.547345in}}%
\pgfusepath{fill}%
\end{pgfscope}%
\begin{pgfscope}%
\pgfpathrectangle{\pgfqpoint{0.549740in}{0.463273in}}{\pgfqpoint{9.320225in}{4.495057in}}%
\pgfusepath{clip}%
\pgfsetbuttcap%
\pgfsetroundjoin%
\pgfsetlinewidth{0.000000pt}%
\definecolor{currentstroke}{rgb}{0.000000,0.000000,0.000000}%
\pgfsetstrokecolor{currentstroke}%
\pgfsetdash{}{0pt}%
\pgfpathmoveto{\pgfqpoint{0.549761in}{2.629073in}}%
\pgfpathlineto{\pgfqpoint{0.735988in}{2.629073in}}%
\pgfpathlineto{\pgfqpoint{0.735988in}{2.710802in}}%
\pgfpathlineto{\pgfqpoint{0.549761in}{2.710802in}}%
\pgfpathlineto{\pgfqpoint{0.549761in}{2.629073in}}%
\pgfusepath{}%
\end{pgfscope}%
\begin{pgfscope}%
\pgfpathrectangle{\pgfqpoint{0.549740in}{0.463273in}}{\pgfqpoint{9.320225in}{4.495057in}}%
\pgfusepath{clip}%
\pgfsetbuttcap%
\pgfsetroundjoin%
\pgfsetlinewidth{0.000000pt}%
\definecolor{currentstroke}{rgb}{0.000000,0.000000,0.000000}%
\pgfsetstrokecolor{currentstroke}%
\pgfsetdash{}{0pt}%
\pgfpathmoveto{\pgfqpoint{0.735988in}{2.629073in}}%
\pgfpathlineto{\pgfqpoint{0.922214in}{2.629073in}}%
\pgfpathlineto{\pgfqpoint{0.922214in}{2.710802in}}%
\pgfpathlineto{\pgfqpoint{0.735988in}{2.710802in}}%
\pgfpathlineto{\pgfqpoint{0.735988in}{2.629073in}}%
\pgfusepath{}%
\end{pgfscope}%
\begin{pgfscope}%
\pgfpathrectangle{\pgfqpoint{0.549740in}{0.463273in}}{\pgfqpoint{9.320225in}{4.495057in}}%
\pgfusepath{clip}%
\pgfsetbuttcap%
\pgfsetroundjoin%
\pgfsetlinewidth{0.000000pt}%
\definecolor{currentstroke}{rgb}{0.000000,0.000000,0.000000}%
\pgfsetstrokecolor{currentstroke}%
\pgfsetdash{}{0pt}%
\pgfpathmoveto{\pgfqpoint{0.922214in}{2.629073in}}%
\pgfpathlineto{\pgfqpoint{1.108441in}{2.629073in}}%
\pgfpathlineto{\pgfqpoint{1.108441in}{2.710802in}}%
\pgfpathlineto{\pgfqpoint{0.922214in}{2.710802in}}%
\pgfpathlineto{\pgfqpoint{0.922214in}{2.629073in}}%
\pgfusepath{}%
\end{pgfscope}%
\begin{pgfscope}%
\pgfpathrectangle{\pgfqpoint{0.549740in}{0.463273in}}{\pgfqpoint{9.320225in}{4.495057in}}%
\pgfusepath{clip}%
\pgfsetbuttcap%
\pgfsetroundjoin%
\pgfsetlinewidth{0.000000pt}%
\definecolor{currentstroke}{rgb}{0.000000,0.000000,0.000000}%
\pgfsetstrokecolor{currentstroke}%
\pgfsetdash{}{0pt}%
\pgfpathmoveto{\pgfqpoint{1.108441in}{2.629073in}}%
\pgfpathlineto{\pgfqpoint{1.294667in}{2.629073in}}%
\pgfpathlineto{\pgfqpoint{1.294667in}{2.710802in}}%
\pgfpathlineto{\pgfqpoint{1.108441in}{2.710802in}}%
\pgfpathlineto{\pgfqpoint{1.108441in}{2.629073in}}%
\pgfusepath{}%
\end{pgfscope}%
\begin{pgfscope}%
\pgfpathrectangle{\pgfqpoint{0.549740in}{0.463273in}}{\pgfqpoint{9.320225in}{4.495057in}}%
\pgfusepath{clip}%
\pgfsetbuttcap%
\pgfsetroundjoin%
\pgfsetlinewidth{0.000000pt}%
\definecolor{currentstroke}{rgb}{0.000000,0.000000,0.000000}%
\pgfsetstrokecolor{currentstroke}%
\pgfsetdash{}{0pt}%
\pgfpathmoveto{\pgfqpoint{1.294667in}{2.629073in}}%
\pgfpathlineto{\pgfqpoint{1.480894in}{2.629073in}}%
\pgfpathlineto{\pgfqpoint{1.480894in}{2.710802in}}%
\pgfpathlineto{\pgfqpoint{1.294667in}{2.710802in}}%
\pgfpathlineto{\pgfqpoint{1.294667in}{2.629073in}}%
\pgfusepath{}%
\end{pgfscope}%
\begin{pgfscope}%
\pgfpathrectangle{\pgfqpoint{0.549740in}{0.463273in}}{\pgfqpoint{9.320225in}{4.495057in}}%
\pgfusepath{clip}%
\pgfsetbuttcap%
\pgfsetroundjoin%
\pgfsetlinewidth{0.000000pt}%
\definecolor{currentstroke}{rgb}{0.000000,0.000000,0.000000}%
\pgfsetstrokecolor{currentstroke}%
\pgfsetdash{}{0pt}%
\pgfpathmoveto{\pgfqpoint{1.480894in}{2.629073in}}%
\pgfpathlineto{\pgfqpoint{1.667120in}{2.629073in}}%
\pgfpathlineto{\pgfqpoint{1.667120in}{2.710802in}}%
\pgfpathlineto{\pgfqpoint{1.480894in}{2.710802in}}%
\pgfpathlineto{\pgfqpoint{1.480894in}{2.629073in}}%
\pgfusepath{}%
\end{pgfscope}%
\begin{pgfscope}%
\pgfpathrectangle{\pgfqpoint{0.549740in}{0.463273in}}{\pgfqpoint{9.320225in}{4.495057in}}%
\pgfusepath{clip}%
\pgfsetbuttcap%
\pgfsetroundjoin%
\pgfsetlinewidth{0.000000pt}%
\definecolor{currentstroke}{rgb}{0.000000,0.000000,0.000000}%
\pgfsetstrokecolor{currentstroke}%
\pgfsetdash{}{0pt}%
\pgfpathmoveto{\pgfqpoint{1.667120in}{2.629073in}}%
\pgfpathlineto{\pgfqpoint{1.853347in}{2.629073in}}%
\pgfpathlineto{\pgfqpoint{1.853347in}{2.710802in}}%
\pgfpathlineto{\pgfqpoint{1.667120in}{2.710802in}}%
\pgfpathlineto{\pgfqpoint{1.667120in}{2.629073in}}%
\pgfusepath{}%
\end{pgfscope}%
\begin{pgfscope}%
\pgfpathrectangle{\pgfqpoint{0.549740in}{0.463273in}}{\pgfqpoint{9.320225in}{4.495057in}}%
\pgfusepath{clip}%
\pgfsetbuttcap%
\pgfsetroundjoin%
\pgfsetlinewidth{0.000000pt}%
\definecolor{currentstroke}{rgb}{0.000000,0.000000,0.000000}%
\pgfsetstrokecolor{currentstroke}%
\pgfsetdash{}{0pt}%
\pgfpathmoveto{\pgfqpoint{1.853347in}{2.629073in}}%
\pgfpathlineto{\pgfqpoint{2.039573in}{2.629073in}}%
\pgfpathlineto{\pgfqpoint{2.039573in}{2.710802in}}%
\pgfpathlineto{\pgfqpoint{1.853347in}{2.710802in}}%
\pgfpathlineto{\pgfqpoint{1.853347in}{2.629073in}}%
\pgfusepath{}%
\end{pgfscope}%
\begin{pgfscope}%
\pgfpathrectangle{\pgfqpoint{0.549740in}{0.463273in}}{\pgfqpoint{9.320225in}{4.495057in}}%
\pgfusepath{clip}%
\pgfsetbuttcap%
\pgfsetroundjoin%
\pgfsetlinewidth{0.000000pt}%
\definecolor{currentstroke}{rgb}{0.000000,0.000000,0.000000}%
\pgfsetstrokecolor{currentstroke}%
\pgfsetdash{}{0pt}%
\pgfpathmoveto{\pgfqpoint{2.039573in}{2.629073in}}%
\pgfpathlineto{\pgfqpoint{2.225800in}{2.629073in}}%
\pgfpathlineto{\pgfqpoint{2.225800in}{2.710802in}}%
\pgfpathlineto{\pgfqpoint{2.039573in}{2.710802in}}%
\pgfpathlineto{\pgfqpoint{2.039573in}{2.629073in}}%
\pgfusepath{}%
\end{pgfscope}%
\begin{pgfscope}%
\pgfpathrectangle{\pgfqpoint{0.549740in}{0.463273in}}{\pgfqpoint{9.320225in}{4.495057in}}%
\pgfusepath{clip}%
\pgfsetbuttcap%
\pgfsetroundjoin%
\pgfsetlinewidth{0.000000pt}%
\definecolor{currentstroke}{rgb}{0.000000,0.000000,0.000000}%
\pgfsetstrokecolor{currentstroke}%
\pgfsetdash{}{0pt}%
\pgfpathmoveto{\pgfqpoint{2.225800in}{2.629073in}}%
\pgfpathlineto{\pgfqpoint{2.412027in}{2.629073in}}%
\pgfpathlineto{\pgfqpoint{2.412027in}{2.710802in}}%
\pgfpathlineto{\pgfqpoint{2.225800in}{2.710802in}}%
\pgfpathlineto{\pgfqpoint{2.225800in}{2.629073in}}%
\pgfusepath{}%
\end{pgfscope}%
\begin{pgfscope}%
\pgfpathrectangle{\pgfqpoint{0.549740in}{0.463273in}}{\pgfqpoint{9.320225in}{4.495057in}}%
\pgfusepath{clip}%
\pgfsetbuttcap%
\pgfsetroundjoin%
\pgfsetlinewidth{0.000000pt}%
\definecolor{currentstroke}{rgb}{0.000000,0.000000,0.000000}%
\pgfsetstrokecolor{currentstroke}%
\pgfsetdash{}{0pt}%
\pgfpathmoveto{\pgfqpoint{2.412027in}{2.629073in}}%
\pgfpathlineto{\pgfqpoint{2.598253in}{2.629073in}}%
\pgfpathlineto{\pgfqpoint{2.598253in}{2.710802in}}%
\pgfpathlineto{\pgfqpoint{2.412027in}{2.710802in}}%
\pgfpathlineto{\pgfqpoint{2.412027in}{2.629073in}}%
\pgfusepath{}%
\end{pgfscope}%
\begin{pgfscope}%
\pgfpathrectangle{\pgfqpoint{0.549740in}{0.463273in}}{\pgfqpoint{9.320225in}{4.495057in}}%
\pgfusepath{clip}%
\pgfsetbuttcap%
\pgfsetroundjoin%
\pgfsetlinewidth{0.000000pt}%
\definecolor{currentstroke}{rgb}{0.000000,0.000000,0.000000}%
\pgfsetstrokecolor{currentstroke}%
\pgfsetdash{}{0pt}%
\pgfpathmoveto{\pgfqpoint{2.598253in}{2.629073in}}%
\pgfpathlineto{\pgfqpoint{2.784480in}{2.629073in}}%
\pgfpathlineto{\pgfqpoint{2.784480in}{2.710802in}}%
\pgfpathlineto{\pgfqpoint{2.598253in}{2.710802in}}%
\pgfpathlineto{\pgfqpoint{2.598253in}{2.629073in}}%
\pgfusepath{}%
\end{pgfscope}%
\begin{pgfscope}%
\pgfpathrectangle{\pgfqpoint{0.549740in}{0.463273in}}{\pgfqpoint{9.320225in}{4.495057in}}%
\pgfusepath{clip}%
\pgfsetbuttcap%
\pgfsetroundjoin%
\pgfsetlinewidth{0.000000pt}%
\definecolor{currentstroke}{rgb}{0.000000,0.000000,0.000000}%
\pgfsetstrokecolor{currentstroke}%
\pgfsetdash{}{0pt}%
\pgfpathmoveto{\pgfqpoint{2.784480in}{2.629073in}}%
\pgfpathlineto{\pgfqpoint{2.970706in}{2.629073in}}%
\pgfpathlineto{\pgfqpoint{2.970706in}{2.710802in}}%
\pgfpathlineto{\pgfqpoint{2.784480in}{2.710802in}}%
\pgfpathlineto{\pgfqpoint{2.784480in}{2.629073in}}%
\pgfusepath{}%
\end{pgfscope}%
\begin{pgfscope}%
\pgfpathrectangle{\pgfqpoint{0.549740in}{0.463273in}}{\pgfqpoint{9.320225in}{4.495057in}}%
\pgfusepath{clip}%
\pgfsetbuttcap%
\pgfsetroundjoin%
\pgfsetlinewidth{0.000000pt}%
\definecolor{currentstroke}{rgb}{0.000000,0.000000,0.000000}%
\pgfsetstrokecolor{currentstroke}%
\pgfsetdash{}{0pt}%
\pgfpathmoveto{\pgfqpoint{2.970706in}{2.629073in}}%
\pgfpathlineto{\pgfqpoint{3.156933in}{2.629073in}}%
\pgfpathlineto{\pgfqpoint{3.156933in}{2.710802in}}%
\pgfpathlineto{\pgfqpoint{2.970706in}{2.710802in}}%
\pgfpathlineto{\pgfqpoint{2.970706in}{2.629073in}}%
\pgfusepath{}%
\end{pgfscope}%
\begin{pgfscope}%
\pgfpathrectangle{\pgfqpoint{0.549740in}{0.463273in}}{\pgfqpoint{9.320225in}{4.495057in}}%
\pgfusepath{clip}%
\pgfsetbuttcap%
\pgfsetroundjoin%
\pgfsetlinewidth{0.000000pt}%
\definecolor{currentstroke}{rgb}{0.000000,0.000000,0.000000}%
\pgfsetstrokecolor{currentstroke}%
\pgfsetdash{}{0pt}%
\pgfpathmoveto{\pgfqpoint{3.156933in}{2.629073in}}%
\pgfpathlineto{\pgfqpoint{3.343159in}{2.629073in}}%
\pgfpathlineto{\pgfqpoint{3.343159in}{2.710802in}}%
\pgfpathlineto{\pgfqpoint{3.156933in}{2.710802in}}%
\pgfpathlineto{\pgfqpoint{3.156933in}{2.629073in}}%
\pgfusepath{}%
\end{pgfscope}%
\begin{pgfscope}%
\pgfpathrectangle{\pgfqpoint{0.549740in}{0.463273in}}{\pgfqpoint{9.320225in}{4.495057in}}%
\pgfusepath{clip}%
\pgfsetbuttcap%
\pgfsetroundjoin%
\pgfsetlinewidth{0.000000pt}%
\definecolor{currentstroke}{rgb}{0.000000,0.000000,0.000000}%
\pgfsetstrokecolor{currentstroke}%
\pgfsetdash{}{0pt}%
\pgfpathmoveto{\pgfqpoint{3.343159in}{2.629073in}}%
\pgfpathlineto{\pgfqpoint{3.529386in}{2.629073in}}%
\pgfpathlineto{\pgfqpoint{3.529386in}{2.710802in}}%
\pgfpathlineto{\pgfqpoint{3.343159in}{2.710802in}}%
\pgfpathlineto{\pgfqpoint{3.343159in}{2.629073in}}%
\pgfusepath{}%
\end{pgfscope}%
\begin{pgfscope}%
\pgfpathrectangle{\pgfqpoint{0.549740in}{0.463273in}}{\pgfqpoint{9.320225in}{4.495057in}}%
\pgfusepath{clip}%
\pgfsetbuttcap%
\pgfsetroundjoin%
\pgfsetlinewidth{0.000000pt}%
\definecolor{currentstroke}{rgb}{0.000000,0.000000,0.000000}%
\pgfsetstrokecolor{currentstroke}%
\pgfsetdash{}{0pt}%
\pgfpathmoveto{\pgfqpoint{3.529386in}{2.629073in}}%
\pgfpathlineto{\pgfqpoint{3.715612in}{2.629073in}}%
\pgfpathlineto{\pgfqpoint{3.715612in}{2.710802in}}%
\pgfpathlineto{\pgfqpoint{3.529386in}{2.710802in}}%
\pgfpathlineto{\pgfqpoint{3.529386in}{2.629073in}}%
\pgfusepath{}%
\end{pgfscope}%
\begin{pgfscope}%
\pgfpathrectangle{\pgfqpoint{0.549740in}{0.463273in}}{\pgfqpoint{9.320225in}{4.495057in}}%
\pgfusepath{clip}%
\pgfsetbuttcap%
\pgfsetroundjoin%
\pgfsetlinewidth{0.000000pt}%
\definecolor{currentstroke}{rgb}{0.000000,0.000000,0.000000}%
\pgfsetstrokecolor{currentstroke}%
\pgfsetdash{}{0pt}%
\pgfpathmoveto{\pgfqpoint{3.715612in}{2.629073in}}%
\pgfpathlineto{\pgfqpoint{3.901839in}{2.629073in}}%
\pgfpathlineto{\pgfqpoint{3.901839in}{2.710802in}}%
\pgfpathlineto{\pgfqpoint{3.715612in}{2.710802in}}%
\pgfpathlineto{\pgfqpoint{3.715612in}{2.629073in}}%
\pgfusepath{}%
\end{pgfscope}%
\begin{pgfscope}%
\pgfpathrectangle{\pgfqpoint{0.549740in}{0.463273in}}{\pgfqpoint{9.320225in}{4.495057in}}%
\pgfusepath{clip}%
\pgfsetbuttcap%
\pgfsetroundjoin%
\pgfsetlinewidth{0.000000pt}%
\definecolor{currentstroke}{rgb}{0.000000,0.000000,0.000000}%
\pgfsetstrokecolor{currentstroke}%
\pgfsetdash{}{0pt}%
\pgfpathmoveto{\pgfqpoint{3.901839in}{2.629073in}}%
\pgfpathlineto{\pgfqpoint{4.088065in}{2.629073in}}%
\pgfpathlineto{\pgfqpoint{4.088065in}{2.710802in}}%
\pgfpathlineto{\pgfqpoint{3.901839in}{2.710802in}}%
\pgfpathlineto{\pgfqpoint{3.901839in}{2.629073in}}%
\pgfusepath{}%
\end{pgfscope}%
\begin{pgfscope}%
\pgfpathrectangle{\pgfqpoint{0.549740in}{0.463273in}}{\pgfqpoint{9.320225in}{4.495057in}}%
\pgfusepath{clip}%
\pgfsetbuttcap%
\pgfsetroundjoin%
\pgfsetlinewidth{0.000000pt}%
\definecolor{currentstroke}{rgb}{0.000000,0.000000,0.000000}%
\pgfsetstrokecolor{currentstroke}%
\pgfsetdash{}{0pt}%
\pgfpathmoveto{\pgfqpoint{4.088065in}{2.629073in}}%
\pgfpathlineto{\pgfqpoint{4.274292in}{2.629073in}}%
\pgfpathlineto{\pgfqpoint{4.274292in}{2.710802in}}%
\pgfpathlineto{\pgfqpoint{4.088065in}{2.710802in}}%
\pgfpathlineto{\pgfqpoint{4.088065in}{2.629073in}}%
\pgfusepath{}%
\end{pgfscope}%
\begin{pgfscope}%
\pgfpathrectangle{\pgfqpoint{0.549740in}{0.463273in}}{\pgfqpoint{9.320225in}{4.495057in}}%
\pgfusepath{clip}%
\pgfsetbuttcap%
\pgfsetroundjoin%
\pgfsetlinewidth{0.000000pt}%
\definecolor{currentstroke}{rgb}{0.000000,0.000000,0.000000}%
\pgfsetstrokecolor{currentstroke}%
\pgfsetdash{}{0pt}%
\pgfpathmoveto{\pgfqpoint{4.274292in}{2.629073in}}%
\pgfpathlineto{\pgfqpoint{4.460519in}{2.629073in}}%
\pgfpathlineto{\pgfqpoint{4.460519in}{2.710802in}}%
\pgfpathlineto{\pgfqpoint{4.274292in}{2.710802in}}%
\pgfpathlineto{\pgfqpoint{4.274292in}{2.629073in}}%
\pgfusepath{}%
\end{pgfscope}%
\begin{pgfscope}%
\pgfpathrectangle{\pgfqpoint{0.549740in}{0.463273in}}{\pgfqpoint{9.320225in}{4.495057in}}%
\pgfusepath{clip}%
\pgfsetbuttcap%
\pgfsetroundjoin%
\pgfsetlinewidth{0.000000pt}%
\definecolor{currentstroke}{rgb}{0.000000,0.000000,0.000000}%
\pgfsetstrokecolor{currentstroke}%
\pgfsetdash{}{0pt}%
\pgfpathmoveto{\pgfqpoint{4.460519in}{2.629073in}}%
\pgfpathlineto{\pgfqpoint{4.646745in}{2.629073in}}%
\pgfpathlineto{\pgfqpoint{4.646745in}{2.710802in}}%
\pgfpathlineto{\pgfqpoint{4.460519in}{2.710802in}}%
\pgfpathlineto{\pgfqpoint{4.460519in}{2.629073in}}%
\pgfusepath{}%
\end{pgfscope}%
\begin{pgfscope}%
\pgfpathrectangle{\pgfqpoint{0.549740in}{0.463273in}}{\pgfqpoint{9.320225in}{4.495057in}}%
\pgfusepath{clip}%
\pgfsetbuttcap%
\pgfsetroundjoin%
\pgfsetlinewidth{0.000000pt}%
\definecolor{currentstroke}{rgb}{0.000000,0.000000,0.000000}%
\pgfsetstrokecolor{currentstroke}%
\pgfsetdash{}{0pt}%
\pgfpathmoveto{\pgfqpoint{4.646745in}{2.629073in}}%
\pgfpathlineto{\pgfqpoint{4.832972in}{2.629073in}}%
\pgfpathlineto{\pgfqpoint{4.832972in}{2.710802in}}%
\pgfpathlineto{\pgfqpoint{4.646745in}{2.710802in}}%
\pgfpathlineto{\pgfqpoint{4.646745in}{2.629073in}}%
\pgfusepath{}%
\end{pgfscope}%
\begin{pgfscope}%
\pgfpathrectangle{\pgfqpoint{0.549740in}{0.463273in}}{\pgfqpoint{9.320225in}{4.495057in}}%
\pgfusepath{clip}%
\pgfsetbuttcap%
\pgfsetroundjoin%
\pgfsetlinewidth{0.000000pt}%
\definecolor{currentstroke}{rgb}{0.000000,0.000000,0.000000}%
\pgfsetstrokecolor{currentstroke}%
\pgfsetdash{}{0pt}%
\pgfpathmoveto{\pgfqpoint{4.832972in}{2.629073in}}%
\pgfpathlineto{\pgfqpoint{5.019198in}{2.629073in}}%
\pgfpathlineto{\pgfqpoint{5.019198in}{2.710802in}}%
\pgfpathlineto{\pgfqpoint{4.832972in}{2.710802in}}%
\pgfpathlineto{\pgfqpoint{4.832972in}{2.629073in}}%
\pgfusepath{}%
\end{pgfscope}%
\begin{pgfscope}%
\pgfpathrectangle{\pgfqpoint{0.549740in}{0.463273in}}{\pgfqpoint{9.320225in}{4.495057in}}%
\pgfusepath{clip}%
\pgfsetbuttcap%
\pgfsetroundjoin%
\pgfsetlinewidth{0.000000pt}%
\definecolor{currentstroke}{rgb}{0.000000,0.000000,0.000000}%
\pgfsetstrokecolor{currentstroke}%
\pgfsetdash{}{0pt}%
\pgfpathmoveto{\pgfqpoint{5.019198in}{2.629073in}}%
\pgfpathlineto{\pgfqpoint{5.205425in}{2.629073in}}%
\pgfpathlineto{\pgfqpoint{5.205425in}{2.710802in}}%
\pgfpathlineto{\pgfqpoint{5.019198in}{2.710802in}}%
\pgfpathlineto{\pgfqpoint{5.019198in}{2.629073in}}%
\pgfusepath{}%
\end{pgfscope}%
\begin{pgfscope}%
\pgfpathrectangle{\pgfqpoint{0.549740in}{0.463273in}}{\pgfqpoint{9.320225in}{4.495057in}}%
\pgfusepath{clip}%
\pgfsetbuttcap%
\pgfsetroundjoin%
\definecolor{currentfill}{rgb}{0.472869,0.711325,0.955316}%
\pgfsetfillcolor{currentfill}%
\pgfsetlinewidth{0.000000pt}%
\definecolor{currentstroke}{rgb}{0.000000,0.000000,0.000000}%
\pgfsetstrokecolor{currentstroke}%
\pgfsetdash{}{0pt}%
\pgfpathmoveto{\pgfqpoint{5.205425in}{2.629073in}}%
\pgfpathlineto{\pgfqpoint{5.391651in}{2.629073in}}%
\pgfpathlineto{\pgfqpoint{5.391651in}{2.710802in}}%
\pgfpathlineto{\pgfqpoint{5.205425in}{2.710802in}}%
\pgfpathlineto{\pgfqpoint{5.205425in}{2.629073in}}%
\pgfusepath{fill}%
\end{pgfscope}%
\begin{pgfscope}%
\pgfpathrectangle{\pgfqpoint{0.549740in}{0.463273in}}{\pgfqpoint{9.320225in}{4.495057in}}%
\pgfusepath{clip}%
\pgfsetbuttcap%
\pgfsetroundjoin%
\pgfsetlinewidth{0.000000pt}%
\definecolor{currentstroke}{rgb}{0.000000,0.000000,0.000000}%
\pgfsetstrokecolor{currentstroke}%
\pgfsetdash{}{0pt}%
\pgfpathmoveto{\pgfqpoint{5.391651in}{2.629073in}}%
\pgfpathlineto{\pgfqpoint{5.577878in}{2.629073in}}%
\pgfpathlineto{\pgfqpoint{5.577878in}{2.710802in}}%
\pgfpathlineto{\pgfqpoint{5.391651in}{2.710802in}}%
\pgfpathlineto{\pgfqpoint{5.391651in}{2.629073in}}%
\pgfusepath{}%
\end{pgfscope}%
\begin{pgfscope}%
\pgfpathrectangle{\pgfqpoint{0.549740in}{0.463273in}}{\pgfqpoint{9.320225in}{4.495057in}}%
\pgfusepath{clip}%
\pgfsetbuttcap%
\pgfsetroundjoin%
\pgfsetlinewidth{0.000000pt}%
\definecolor{currentstroke}{rgb}{0.000000,0.000000,0.000000}%
\pgfsetstrokecolor{currentstroke}%
\pgfsetdash{}{0pt}%
\pgfpathmoveto{\pgfqpoint{5.577878in}{2.629073in}}%
\pgfpathlineto{\pgfqpoint{5.764104in}{2.629073in}}%
\pgfpathlineto{\pgfqpoint{5.764104in}{2.710802in}}%
\pgfpathlineto{\pgfqpoint{5.577878in}{2.710802in}}%
\pgfpathlineto{\pgfqpoint{5.577878in}{2.629073in}}%
\pgfusepath{}%
\end{pgfscope}%
\begin{pgfscope}%
\pgfpathrectangle{\pgfqpoint{0.549740in}{0.463273in}}{\pgfqpoint{9.320225in}{4.495057in}}%
\pgfusepath{clip}%
\pgfsetbuttcap%
\pgfsetroundjoin%
\pgfsetlinewidth{0.000000pt}%
\definecolor{currentstroke}{rgb}{0.000000,0.000000,0.000000}%
\pgfsetstrokecolor{currentstroke}%
\pgfsetdash{}{0pt}%
\pgfpathmoveto{\pgfqpoint{5.764104in}{2.629073in}}%
\pgfpathlineto{\pgfqpoint{5.950331in}{2.629073in}}%
\pgfpathlineto{\pgfqpoint{5.950331in}{2.710802in}}%
\pgfpathlineto{\pgfqpoint{5.764104in}{2.710802in}}%
\pgfpathlineto{\pgfqpoint{5.764104in}{2.629073in}}%
\pgfusepath{}%
\end{pgfscope}%
\begin{pgfscope}%
\pgfpathrectangle{\pgfqpoint{0.549740in}{0.463273in}}{\pgfqpoint{9.320225in}{4.495057in}}%
\pgfusepath{clip}%
\pgfsetbuttcap%
\pgfsetroundjoin%
\pgfsetlinewidth{0.000000pt}%
\definecolor{currentstroke}{rgb}{0.000000,0.000000,0.000000}%
\pgfsetstrokecolor{currentstroke}%
\pgfsetdash{}{0pt}%
\pgfpathmoveto{\pgfqpoint{5.950331in}{2.629073in}}%
\pgfpathlineto{\pgfqpoint{6.136557in}{2.629073in}}%
\pgfpathlineto{\pgfqpoint{6.136557in}{2.710802in}}%
\pgfpathlineto{\pgfqpoint{5.950331in}{2.710802in}}%
\pgfpathlineto{\pgfqpoint{5.950331in}{2.629073in}}%
\pgfusepath{}%
\end{pgfscope}%
\begin{pgfscope}%
\pgfpathrectangle{\pgfqpoint{0.549740in}{0.463273in}}{\pgfqpoint{9.320225in}{4.495057in}}%
\pgfusepath{clip}%
\pgfsetbuttcap%
\pgfsetroundjoin%
\definecolor{currentfill}{rgb}{0.472869,0.711325,0.955316}%
\pgfsetfillcolor{currentfill}%
\pgfsetlinewidth{0.000000pt}%
\definecolor{currentstroke}{rgb}{0.000000,0.000000,0.000000}%
\pgfsetstrokecolor{currentstroke}%
\pgfsetdash{}{0pt}%
\pgfpathmoveto{\pgfqpoint{6.136557in}{2.629073in}}%
\pgfpathlineto{\pgfqpoint{6.322784in}{2.629073in}}%
\pgfpathlineto{\pgfqpoint{6.322784in}{2.710802in}}%
\pgfpathlineto{\pgfqpoint{6.136557in}{2.710802in}}%
\pgfpathlineto{\pgfqpoint{6.136557in}{2.629073in}}%
\pgfusepath{fill}%
\end{pgfscope}%
\begin{pgfscope}%
\pgfpathrectangle{\pgfqpoint{0.549740in}{0.463273in}}{\pgfqpoint{9.320225in}{4.495057in}}%
\pgfusepath{clip}%
\pgfsetbuttcap%
\pgfsetroundjoin%
\pgfsetlinewidth{0.000000pt}%
\definecolor{currentstroke}{rgb}{0.000000,0.000000,0.000000}%
\pgfsetstrokecolor{currentstroke}%
\pgfsetdash{}{0pt}%
\pgfpathmoveto{\pgfqpoint{6.322784in}{2.629073in}}%
\pgfpathlineto{\pgfqpoint{6.509011in}{2.629073in}}%
\pgfpathlineto{\pgfqpoint{6.509011in}{2.710802in}}%
\pgfpathlineto{\pgfqpoint{6.322784in}{2.710802in}}%
\pgfpathlineto{\pgfqpoint{6.322784in}{2.629073in}}%
\pgfusepath{}%
\end{pgfscope}%
\begin{pgfscope}%
\pgfpathrectangle{\pgfqpoint{0.549740in}{0.463273in}}{\pgfqpoint{9.320225in}{4.495057in}}%
\pgfusepath{clip}%
\pgfsetbuttcap%
\pgfsetroundjoin%
\pgfsetlinewidth{0.000000pt}%
\definecolor{currentstroke}{rgb}{0.000000,0.000000,0.000000}%
\pgfsetstrokecolor{currentstroke}%
\pgfsetdash{}{0pt}%
\pgfpathmoveto{\pgfqpoint{6.509011in}{2.629073in}}%
\pgfpathlineto{\pgfqpoint{6.695237in}{2.629073in}}%
\pgfpathlineto{\pgfqpoint{6.695237in}{2.710802in}}%
\pgfpathlineto{\pgfqpoint{6.509011in}{2.710802in}}%
\pgfpathlineto{\pgfqpoint{6.509011in}{2.629073in}}%
\pgfusepath{}%
\end{pgfscope}%
\begin{pgfscope}%
\pgfpathrectangle{\pgfqpoint{0.549740in}{0.463273in}}{\pgfqpoint{9.320225in}{4.495057in}}%
\pgfusepath{clip}%
\pgfsetbuttcap%
\pgfsetroundjoin%
\pgfsetlinewidth{0.000000pt}%
\definecolor{currentstroke}{rgb}{0.000000,0.000000,0.000000}%
\pgfsetstrokecolor{currentstroke}%
\pgfsetdash{}{0pt}%
\pgfpathmoveto{\pgfqpoint{6.695237in}{2.629073in}}%
\pgfpathlineto{\pgfqpoint{6.881464in}{2.629073in}}%
\pgfpathlineto{\pgfqpoint{6.881464in}{2.710802in}}%
\pgfpathlineto{\pgfqpoint{6.695237in}{2.710802in}}%
\pgfpathlineto{\pgfqpoint{6.695237in}{2.629073in}}%
\pgfusepath{}%
\end{pgfscope}%
\begin{pgfscope}%
\pgfpathrectangle{\pgfqpoint{0.549740in}{0.463273in}}{\pgfqpoint{9.320225in}{4.495057in}}%
\pgfusepath{clip}%
\pgfsetbuttcap%
\pgfsetroundjoin%
\pgfsetlinewidth{0.000000pt}%
\definecolor{currentstroke}{rgb}{0.000000,0.000000,0.000000}%
\pgfsetstrokecolor{currentstroke}%
\pgfsetdash{}{0pt}%
\pgfpathmoveto{\pgfqpoint{6.881464in}{2.629073in}}%
\pgfpathlineto{\pgfqpoint{7.067690in}{2.629073in}}%
\pgfpathlineto{\pgfqpoint{7.067690in}{2.710802in}}%
\pgfpathlineto{\pgfqpoint{6.881464in}{2.710802in}}%
\pgfpathlineto{\pgfqpoint{6.881464in}{2.629073in}}%
\pgfusepath{}%
\end{pgfscope}%
\begin{pgfscope}%
\pgfpathrectangle{\pgfqpoint{0.549740in}{0.463273in}}{\pgfqpoint{9.320225in}{4.495057in}}%
\pgfusepath{clip}%
\pgfsetbuttcap%
\pgfsetroundjoin%
\pgfsetlinewidth{0.000000pt}%
\definecolor{currentstroke}{rgb}{0.000000,0.000000,0.000000}%
\pgfsetstrokecolor{currentstroke}%
\pgfsetdash{}{0pt}%
\pgfpathmoveto{\pgfqpoint{7.067690in}{2.629073in}}%
\pgfpathlineto{\pgfqpoint{7.253917in}{2.629073in}}%
\pgfpathlineto{\pgfqpoint{7.253917in}{2.710802in}}%
\pgfpathlineto{\pgfqpoint{7.067690in}{2.710802in}}%
\pgfpathlineto{\pgfqpoint{7.067690in}{2.629073in}}%
\pgfusepath{}%
\end{pgfscope}%
\begin{pgfscope}%
\pgfpathrectangle{\pgfqpoint{0.549740in}{0.463273in}}{\pgfqpoint{9.320225in}{4.495057in}}%
\pgfusepath{clip}%
\pgfsetbuttcap%
\pgfsetroundjoin%
\definecolor{currentfill}{rgb}{0.472869,0.711325,0.955316}%
\pgfsetfillcolor{currentfill}%
\pgfsetlinewidth{0.000000pt}%
\definecolor{currentstroke}{rgb}{0.000000,0.000000,0.000000}%
\pgfsetstrokecolor{currentstroke}%
\pgfsetdash{}{0pt}%
\pgfpathmoveto{\pgfqpoint{7.253917in}{2.629073in}}%
\pgfpathlineto{\pgfqpoint{7.440143in}{2.629073in}}%
\pgfpathlineto{\pgfqpoint{7.440143in}{2.710802in}}%
\pgfpathlineto{\pgfqpoint{7.253917in}{2.710802in}}%
\pgfpathlineto{\pgfqpoint{7.253917in}{2.629073in}}%
\pgfusepath{fill}%
\end{pgfscope}%
\begin{pgfscope}%
\pgfpathrectangle{\pgfqpoint{0.549740in}{0.463273in}}{\pgfqpoint{9.320225in}{4.495057in}}%
\pgfusepath{clip}%
\pgfsetbuttcap%
\pgfsetroundjoin%
\pgfsetlinewidth{0.000000pt}%
\definecolor{currentstroke}{rgb}{0.000000,0.000000,0.000000}%
\pgfsetstrokecolor{currentstroke}%
\pgfsetdash{}{0pt}%
\pgfpathmoveto{\pgfqpoint{7.440143in}{2.629073in}}%
\pgfpathlineto{\pgfqpoint{7.626370in}{2.629073in}}%
\pgfpathlineto{\pgfqpoint{7.626370in}{2.710802in}}%
\pgfpathlineto{\pgfqpoint{7.440143in}{2.710802in}}%
\pgfpathlineto{\pgfqpoint{7.440143in}{2.629073in}}%
\pgfusepath{}%
\end{pgfscope}%
\begin{pgfscope}%
\pgfpathrectangle{\pgfqpoint{0.549740in}{0.463273in}}{\pgfqpoint{9.320225in}{4.495057in}}%
\pgfusepath{clip}%
\pgfsetbuttcap%
\pgfsetroundjoin%
\pgfsetlinewidth{0.000000pt}%
\definecolor{currentstroke}{rgb}{0.000000,0.000000,0.000000}%
\pgfsetstrokecolor{currentstroke}%
\pgfsetdash{}{0pt}%
\pgfpathmoveto{\pgfqpoint{7.626370in}{2.629073in}}%
\pgfpathlineto{\pgfqpoint{7.812596in}{2.629073in}}%
\pgfpathlineto{\pgfqpoint{7.812596in}{2.710802in}}%
\pgfpathlineto{\pgfqpoint{7.626370in}{2.710802in}}%
\pgfpathlineto{\pgfqpoint{7.626370in}{2.629073in}}%
\pgfusepath{}%
\end{pgfscope}%
\begin{pgfscope}%
\pgfpathrectangle{\pgfqpoint{0.549740in}{0.463273in}}{\pgfqpoint{9.320225in}{4.495057in}}%
\pgfusepath{clip}%
\pgfsetbuttcap%
\pgfsetroundjoin%
\pgfsetlinewidth{0.000000pt}%
\definecolor{currentstroke}{rgb}{0.000000,0.000000,0.000000}%
\pgfsetstrokecolor{currentstroke}%
\pgfsetdash{}{0pt}%
\pgfpathmoveto{\pgfqpoint{7.812596in}{2.629073in}}%
\pgfpathlineto{\pgfqpoint{7.998823in}{2.629073in}}%
\pgfpathlineto{\pgfqpoint{7.998823in}{2.710802in}}%
\pgfpathlineto{\pgfqpoint{7.812596in}{2.710802in}}%
\pgfpathlineto{\pgfqpoint{7.812596in}{2.629073in}}%
\pgfusepath{}%
\end{pgfscope}%
\begin{pgfscope}%
\pgfpathrectangle{\pgfqpoint{0.549740in}{0.463273in}}{\pgfqpoint{9.320225in}{4.495057in}}%
\pgfusepath{clip}%
\pgfsetbuttcap%
\pgfsetroundjoin%
\pgfsetlinewidth{0.000000pt}%
\definecolor{currentstroke}{rgb}{0.000000,0.000000,0.000000}%
\pgfsetstrokecolor{currentstroke}%
\pgfsetdash{}{0pt}%
\pgfpathmoveto{\pgfqpoint{7.998823in}{2.629073in}}%
\pgfpathlineto{\pgfqpoint{8.185049in}{2.629073in}}%
\pgfpathlineto{\pgfqpoint{8.185049in}{2.710802in}}%
\pgfpathlineto{\pgfqpoint{7.998823in}{2.710802in}}%
\pgfpathlineto{\pgfqpoint{7.998823in}{2.629073in}}%
\pgfusepath{}%
\end{pgfscope}%
\begin{pgfscope}%
\pgfpathrectangle{\pgfqpoint{0.549740in}{0.463273in}}{\pgfqpoint{9.320225in}{4.495057in}}%
\pgfusepath{clip}%
\pgfsetbuttcap%
\pgfsetroundjoin%
\definecolor{currentfill}{rgb}{0.614330,0.761948,0.940009}%
\pgfsetfillcolor{currentfill}%
\pgfsetlinewidth{0.000000pt}%
\definecolor{currentstroke}{rgb}{0.000000,0.000000,0.000000}%
\pgfsetstrokecolor{currentstroke}%
\pgfsetdash{}{0pt}%
\pgfpathmoveto{\pgfqpoint{8.185049in}{2.629073in}}%
\pgfpathlineto{\pgfqpoint{8.371276in}{2.629073in}}%
\pgfpathlineto{\pgfqpoint{8.371276in}{2.710802in}}%
\pgfpathlineto{\pgfqpoint{8.185049in}{2.710802in}}%
\pgfpathlineto{\pgfqpoint{8.185049in}{2.629073in}}%
\pgfusepath{fill}%
\end{pgfscope}%
\begin{pgfscope}%
\pgfpathrectangle{\pgfqpoint{0.549740in}{0.463273in}}{\pgfqpoint{9.320225in}{4.495057in}}%
\pgfusepath{clip}%
\pgfsetbuttcap%
\pgfsetroundjoin%
\definecolor{currentfill}{rgb}{0.547810,0.736432,0.947518}%
\pgfsetfillcolor{currentfill}%
\pgfsetlinewidth{0.000000pt}%
\definecolor{currentstroke}{rgb}{0.000000,0.000000,0.000000}%
\pgfsetstrokecolor{currentstroke}%
\pgfsetdash{}{0pt}%
\pgfpathmoveto{\pgfqpoint{8.371276in}{2.629073in}}%
\pgfpathlineto{\pgfqpoint{8.557503in}{2.629073in}}%
\pgfpathlineto{\pgfqpoint{8.557503in}{2.710802in}}%
\pgfpathlineto{\pgfqpoint{8.371276in}{2.710802in}}%
\pgfpathlineto{\pgfqpoint{8.371276in}{2.629073in}}%
\pgfusepath{fill}%
\end{pgfscope}%
\begin{pgfscope}%
\pgfpathrectangle{\pgfqpoint{0.549740in}{0.463273in}}{\pgfqpoint{9.320225in}{4.495057in}}%
\pgfusepath{clip}%
\pgfsetbuttcap%
\pgfsetroundjoin%
\pgfsetlinewidth{0.000000pt}%
\definecolor{currentstroke}{rgb}{0.000000,0.000000,0.000000}%
\pgfsetstrokecolor{currentstroke}%
\pgfsetdash{}{0pt}%
\pgfpathmoveto{\pgfqpoint{8.557503in}{2.629073in}}%
\pgfpathlineto{\pgfqpoint{8.743729in}{2.629073in}}%
\pgfpathlineto{\pgfqpoint{8.743729in}{2.710802in}}%
\pgfpathlineto{\pgfqpoint{8.557503in}{2.710802in}}%
\pgfpathlineto{\pgfqpoint{8.557503in}{2.629073in}}%
\pgfusepath{}%
\end{pgfscope}%
\begin{pgfscope}%
\pgfpathrectangle{\pgfqpoint{0.549740in}{0.463273in}}{\pgfqpoint{9.320225in}{4.495057in}}%
\pgfusepath{clip}%
\pgfsetbuttcap%
\pgfsetroundjoin%
\pgfsetlinewidth{0.000000pt}%
\definecolor{currentstroke}{rgb}{0.000000,0.000000,0.000000}%
\pgfsetstrokecolor{currentstroke}%
\pgfsetdash{}{0pt}%
\pgfpathmoveto{\pgfqpoint{8.743729in}{2.629073in}}%
\pgfpathlineto{\pgfqpoint{8.929956in}{2.629073in}}%
\pgfpathlineto{\pgfqpoint{8.929956in}{2.710802in}}%
\pgfpathlineto{\pgfqpoint{8.743729in}{2.710802in}}%
\pgfpathlineto{\pgfqpoint{8.743729in}{2.629073in}}%
\pgfusepath{}%
\end{pgfscope}%
\begin{pgfscope}%
\pgfpathrectangle{\pgfqpoint{0.549740in}{0.463273in}}{\pgfqpoint{9.320225in}{4.495057in}}%
\pgfusepath{clip}%
\pgfsetbuttcap%
\pgfsetroundjoin%
\pgfsetlinewidth{0.000000pt}%
\definecolor{currentstroke}{rgb}{0.000000,0.000000,0.000000}%
\pgfsetstrokecolor{currentstroke}%
\pgfsetdash{}{0pt}%
\pgfpathmoveto{\pgfqpoint{8.929956in}{2.629073in}}%
\pgfpathlineto{\pgfqpoint{9.116182in}{2.629073in}}%
\pgfpathlineto{\pgfqpoint{9.116182in}{2.710802in}}%
\pgfpathlineto{\pgfqpoint{8.929956in}{2.710802in}}%
\pgfpathlineto{\pgfqpoint{8.929956in}{2.629073in}}%
\pgfusepath{}%
\end{pgfscope}%
\begin{pgfscope}%
\pgfpathrectangle{\pgfqpoint{0.549740in}{0.463273in}}{\pgfqpoint{9.320225in}{4.495057in}}%
\pgfusepath{clip}%
\pgfsetbuttcap%
\pgfsetroundjoin%
\pgfsetlinewidth{0.000000pt}%
\definecolor{currentstroke}{rgb}{0.000000,0.000000,0.000000}%
\pgfsetstrokecolor{currentstroke}%
\pgfsetdash{}{0pt}%
\pgfpathmoveto{\pgfqpoint{9.116182in}{2.629073in}}%
\pgfpathlineto{\pgfqpoint{9.302409in}{2.629073in}}%
\pgfpathlineto{\pgfqpoint{9.302409in}{2.710802in}}%
\pgfpathlineto{\pgfqpoint{9.116182in}{2.710802in}}%
\pgfpathlineto{\pgfqpoint{9.116182in}{2.629073in}}%
\pgfusepath{}%
\end{pgfscope}%
\begin{pgfscope}%
\pgfpathrectangle{\pgfqpoint{0.549740in}{0.463273in}}{\pgfqpoint{9.320225in}{4.495057in}}%
\pgfusepath{clip}%
\pgfsetbuttcap%
\pgfsetroundjoin%
\pgfsetlinewidth{0.000000pt}%
\definecolor{currentstroke}{rgb}{0.000000,0.000000,0.000000}%
\pgfsetstrokecolor{currentstroke}%
\pgfsetdash{}{0pt}%
\pgfpathmoveto{\pgfqpoint{9.302409in}{2.629073in}}%
\pgfpathlineto{\pgfqpoint{9.488635in}{2.629073in}}%
\pgfpathlineto{\pgfqpoint{9.488635in}{2.710802in}}%
\pgfpathlineto{\pgfqpoint{9.302409in}{2.710802in}}%
\pgfpathlineto{\pgfqpoint{9.302409in}{2.629073in}}%
\pgfusepath{}%
\end{pgfscope}%
\begin{pgfscope}%
\pgfpathrectangle{\pgfqpoint{0.549740in}{0.463273in}}{\pgfqpoint{9.320225in}{4.495057in}}%
\pgfusepath{clip}%
\pgfsetbuttcap%
\pgfsetroundjoin%
\definecolor{currentfill}{rgb}{0.472869,0.711325,0.955316}%
\pgfsetfillcolor{currentfill}%
\pgfsetlinewidth{0.000000pt}%
\definecolor{currentstroke}{rgb}{0.000000,0.000000,0.000000}%
\pgfsetstrokecolor{currentstroke}%
\pgfsetdash{}{0pt}%
\pgfpathmoveto{\pgfqpoint{9.488635in}{2.629073in}}%
\pgfpathlineto{\pgfqpoint{9.674862in}{2.629073in}}%
\pgfpathlineto{\pgfqpoint{9.674862in}{2.710802in}}%
\pgfpathlineto{\pgfqpoint{9.488635in}{2.710802in}}%
\pgfpathlineto{\pgfqpoint{9.488635in}{2.629073in}}%
\pgfusepath{fill}%
\end{pgfscope}%
\begin{pgfscope}%
\pgfpathrectangle{\pgfqpoint{0.549740in}{0.463273in}}{\pgfqpoint{9.320225in}{4.495057in}}%
\pgfusepath{clip}%
\pgfsetbuttcap%
\pgfsetroundjoin%
\pgfsetlinewidth{0.000000pt}%
\definecolor{currentstroke}{rgb}{0.000000,0.000000,0.000000}%
\pgfsetstrokecolor{currentstroke}%
\pgfsetdash{}{0pt}%
\pgfpathmoveto{\pgfqpoint{9.674862in}{2.629073in}}%
\pgfpathlineto{\pgfqpoint{9.861088in}{2.629073in}}%
\pgfpathlineto{\pgfqpoint{9.861088in}{2.710802in}}%
\pgfpathlineto{\pgfqpoint{9.674862in}{2.710802in}}%
\pgfpathlineto{\pgfqpoint{9.674862in}{2.629073in}}%
\pgfusepath{}%
\end{pgfscope}%
\begin{pgfscope}%
\pgfpathrectangle{\pgfqpoint{0.549740in}{0.463273in}}{\pgfqpoint{9.320225in}{4.495057in}}%
\pgfusepath{clip}%
\pgfsetbuttcap%
\pgfsetroundjoin%
\pgfsetlinewidth{0.000000pt}%
\definecolor{currentstroke}{rgb}{0.000000,0.000000,0.000000}%
\pgfsetstrokecolor{currentstroke}%
\pgfsetdash{}{0pt}%
\pgfpathmoveto{\pgfqpoint{0.549761in}{2.710802in}}%
\pgfpathlineto{\pgfqpoint{0.735988in}{2.710802in}}%
\pgfpathlineto{\pgfqpoint{0.735988in}{2.792530in}}%
\pgfpathlineto{\pgfqpoint{0.549761in}{2.792530in}}%
\pgfpathlineto{\pgfqpoint{0.549761in}{2.710802in}}%
\pgfusepath{}%
\end{pgfscope}%
\begin{pgfscope}%
\pgfpathrectangle{\pgfqpoint{0.549740in}{0.463273in}}{\pgfqpoint{9.320225in}{4.495057in}}%
\pgfusepath{clip}%
\pgfsetbuttcap%
\pgfsetroundjoin%
\pgfsetlinewidth{0.000000pt}%
\definecolor{currentstroke}{rgb}{0.000000,0.000000,0.000000}%
\pgfsetstrokecolor{currentstroke}%
\pgfsetdash{}{0pt}%
\pgfpathmoveto{\pgfqpoint{0.735988in}{2.710802in}}%
\pgfpathlineto{\pgfqpoint{0.922214in}{2.710802in}}%
\pgfpathlineto{\pgfqpoint{0.922214in}{2.792530in}}%
\pgfpathlineto{\pgfqpoint{0.735988in}{2.792530in}}%
\pgfpathlineto{\pgfqpoint{0.735988in}{2.710802in}}%
\pgfusepath{}%
\end{pgfscope}%
\begin{pgfscope}%
\pgfpathrectangle{\pgfqpoint{0.549740in}{0.463273in}}{\pgfqpoint{9.320225in}{4.495057in}}%
\pgfusepath{clip}%
\pgfsetbuttcap%
\pgfsetroundjoin%
\pgfsetlinewidth{0.000000pt}%
\definecolor{currentstroke}{rgb}{0.000000,0.000000,0.000000}%
\pgfsetstrokecolor{currentstroke}%
\pgfsetdash{}{0pt}%
\pgfpathmoveto{\pgfqpoint{0.922214in}{2.710802in}}%
\pgfpathlineto{\pgfqpoint{1.108441in}{2.710802in}}%
\pgfpathlineto{\pgfqpoint{1.108441in}{2.792530in}}%
\pgfpathlineto{\pgfqpoint{0.922214in}{2.792530in}}%
\pgfpathlineto{\pgfqpoint{0.922214in}{2.710802in}}%
\pgfusepath{}%
\end{pgfscope}%
\begin{pgfscope}%
\pgfpathrectangle{\pgfqpoint{0.549740in}{0.463273in}}{\pgfqpoint{9.320225in}{4.495057in}}%
\pgfusepath{clip}%
\pgfsetbuttcap%
\pgfsetroundjoin%
\pgfsetlinewidth{0.000000pt}%
\definecolor{currentstroke}{rgb}{0.000000,0.000000,0.000000}%
\pgfsetstrokecolor{currentstroke}%
\pgfsetdash{}{0pt}%
\pgfpathmoveto{\pgfqpoint{1.108441in}{2.710802in}}%
\pgfpathlineto{\pgfqpoint{1.294667in}{2.710802in}}%
\pgfpathlineto{\pgfqpoint{1.294667in}{2.792530in}}%
\pgfpathlineto{\pgfqpoint{1.108441in}{2.792530in}}%
\pgfpathlineto{\pgfqpoint{1.108441in}{2.710802in}}%
\pgfusepath{}%
\end{pgfscope}%
\begin{pgfscope}%
\pgfpathrectangle{\pgfqpoint{0.549740in}{0.463273in}}{\pgfqpoint{9.320225in}{4.495057in}}%
\pgfusepath{clip}%
\pgfsetbuttcap%
\pgfsetroundjoin%
\pgfsetlinewidth{0.000000pt}%
\definecolor{currentstroke}{rgb}{0.000000,0.000000,0.000000}%
\pgfsetstrokecolor{currentstroke}%
\pgfsetdash{}{0pt}%
\pgfpathmoveto{\pgfqpoint{1.294667in}{2.710802in}}%
\pgfpathlineto{\pgfqpoint{1.480894in}{2.710802in}}%
\pgfpathlineto{\pgfqpoint{1.480894in}{2.792530in}}%
\pgfpathlineto{\pgfqpoint{1.294667in}{2.792530in}}%
\pgfpathlineto{\pgfqpoint{1.294667in}{2.710802in}}%
\pgfusepath{}%
\end{pgfscope}%
\begin{pgfscope}%
\pgfpathrectangle{\pgfqpoint{0.549740in}{0.463273in}}{\pgfqpoint{9.320225in}{4.495057in}}%
\pgfusepath{clip}%
\pgfsetbuttcap%
\pgfsetroundjoin%
\pgfsetlinewidth{0.000000pt}%
\definecolor{currentstroke}{rgb}{0.000000,0.000000,0.000000}%
\pgfsetstrokecolor{currentstroke}%
\pgfsetdash{}{0pt}%
\pgfpathmoveto{\pgfqpoint{1.480894in}{2.710802in}}%
\pgfpathlineto{\pgfqpoint{1.667120in}{2.710802in}}%
\pgfpathlineto{\pgfqpoint{1.667120in}{2.792530in}}%
\pgfpathlineto{\pgfqpoint{1.480894in}{2.792530in}}%
\pgfpathlineto{\pgfqpoint{1.480894in}{2.710802in}}%
\pgfusepath{}%
\end{pgfscope}%
\begin{pgfscope}%
\pgfpathrectangle{\pgfqpoint{0.549740in}{0.463273in}}{\pgfqpoint{9.320225in}{4.495057in}}%
\pgfusepath{clip}%
\pgfsetbuttcap%
\pgfsetroundjoin%
\pgfsetlinewidth{0.000000pt}%
\definecolor{currentstroke}{rgb}{0.000000,0.000000,0.000000}%
\pgfsetstrokecolor{currentstroke}%
\pgfsetdash{}{0pt}%
\pgfpathmoveto{\pgfqpoint{1.667120in}{2.710802in}}%
\pgfpathlineto{\pgfqpoint{1.853347in}{2.710802in}}%
\pgfpathlineto{\pgfqpoint{1.853347in}{2.792530in}}%
\pgfpathlineto{\pgfqpoint{1.667120in}{2.792530in}}%
\pgfpathlineto{\pgfqpoint{1.667120in}{2.710802in}}%
\pgfusepath{}%
\end{pgfscope}%
\begin{pgfscope}%
\pgfpathrectangle{\pgfqpoint{0.549740in}{0.463273in}}{\pgfqpoint{9.320225in}{4.495057in}}%
\pgfusepath{clip}%
\pgfsetbuttcap%
\pgfsetroundjoin%
\pgfsetlinewidth{0.000000pt}%
\definecolor{currentstroke}{rgb}{0.000000,0.000000,0.000000}%
\pgfsetstrokecolor{currentstroke}%
\pgfsetdash{}{0pt}%
\pgfpathmoveto{\pgfqpoint{1.853347in}{2.710802in}}%
\pgfpathlineto{\pgfqpoint{2.039573in}{2.710802in}}%
\pgfpathlineto{\pgfqpoint{2.039573in}{2.792530in}}%
\pgfpathlineto{\pgfqpoint{1.853347in}{2.792530in}}%
\pgfpathlineto{\pgfqpoint{1.853347in}{2.710802in}}%
\pgfusepath{}%
\end{pgfscope}%
\begin{pgfscope}%
\pgfpathrectangle{\pgfqpoint{0.549740in}{0.463273in}}{\pgfqpoint{9.320225in}{4.495057in}}%
\pgfusepath{clip}%
\pgfsetbuttcap%
\pgfsetroundjoin%
\pgfsetlinewidth{0.000000pt}%
\definecolor{currentstroke}{rgb}{0.000000,0.000000,0.000000}%
\pgfsetstrokecolor{currentstroke}%
\pgfsetdash{}{0pt}%
\pgfpathmoveto{\pgfqpoint{2.039573in}{2.710802in}}%
\pgfpathlineto{\pgfqpoint{2.225800in}{2.710802in}}%
\pgfpathlineto{\pgfqpoint{2.225800in}{2.792530in}}%
\pgfpathlineto{\pgfqpoint{2.039573in}{2.792530in}}%
\pgfpathlineto{\pgfqpoint{2.039573in}{2.710802in}}%
\pgfusepath{}%
\end{pgfscope}%
\begin{pgfscope}%
\pgfpathrectangle{\pgfqpoint{0.549740in}{0.463273in}}{\pgfqpoint{9.320225in}{4.495057in}}%
\pgfusepath{clip}%
\pgfsetbuttcap%
\pgfsetroundjoin%
\pgfsetlinewidth{0.000000pt}%
\definecolor{currentstroke}{rgb}{0.000000,0.000000,0.000000}%
\pgfsetstrokecolor{currentstroke}%
\pgfsetdash{}{0pt}%
\pgfpathmoveto{\pgfqpoint{2.225800in}{2.710802in}}%
\pgfpathlineto{\pgfqpoint{2.412027in}{2.710802in}}%
\pgfpathlineto{\pgfqpoint{2.412027in}{2.792530in}}%
\pgfpathlineto{\pgfqpoint{2.225800in}{2.792530in}}%
\pgfpathlineto{\pgfqpoint{2.225800in}{2.710802in}}%
\pgfusepath{}%
\end{pgfscope}%
\begin{pgfscope}%
\pgfpathrectangle{\pgfqpoint{0.549740in}{0.463273in}}{\pgfqpoint{9.320225in}{4.495057in}}%
\pgfusepath{clip}%
\pgfsetbuttcap%
\pgfsetroundjoin%
\pgfsetlinewidth{0.000000pt}%
\definecolor{currentstroke}{rgb}{0.000000,0.000000,0.000000}%
\pgfsetstrokecolor{currentstroke}%
\pgfsetdash{}{0pt}%
\pgfpathmoveto{\pgfqpoint{2.412027in}{2.710802in}}%
\pgfpathlineto{\pgfqpoint{2.598253in}{2.710802in}}%
\pgfpathlineto{\pgfqpoint{2.598253in}{2.792530in}}%
\pgfpathlineto{\pgfqpoint{2.412027in}{2.792530in}}%
\pgfpathlineto{\pgfqpoint{2.412027in}{2.710802in}}%
\pgfusepath{}%
\end{pgfscope}%
\begin{pgfscope}%
\pgfpathrectangle{\pgfqpoint{0.549740in}{0.463273in}}{\pgfqpoint{9.320225in}{4.495057in}}%
\pgfusepath{clip}%
\pgfsetbuttcap%
\pgfsetroundjoin%
\pgfsetlinewidth{0.000000pt}%
\definecolor{currentstroke}{rgb}{0.000000,0.000000,0.000000}%
\pgfsetstrokecolor{currentstroke}%
\pgfsetdash{}{0pt}%
\pgfpathmoveto{\pgfqpoint{2.598253in}{2.710802in}}%
\pgfpathlineto{\pgfqpoint{2.784480in}{2.710802in}}%
\pgfpathlineto{\pgfqpoint{2.784480in}{2.792530in}}%
\pgfpathlineto{\pgfqpoint{2.598253in}{2.792530in}}%
\pgfpathlineto{\pgfqpoint{2.598253in}{2.710802in}}%
\pgfusepath{}%
\end{pgfscope}%
\begin{pgfscope}%
\pgfpathrectangle{\pgfqpoint{0.549740in}{0.463273in}}{\pgfqpoint{9.320225in}{4.495057in}}%
\pgfusepath{clip}%
\pgfsetbuttcap%
\pgfsetroundjoin%
\pgfsetlinewidth{0.000000pt}%
\definecolor{currentstroke}{rgb}{0.000000,0.000000,0.000000}%
\pgfsetstrokecolor{currentstroke}%
\pgfsetdash{}{0pt}%
\pgfpathmoveto{\pgfqpoint{2.784480in}{2.710802in}}%
\pgfpathlineto{\pgfqpoint{2.970706in}{2.710802in}}%
\pgfpathlineto{\pgfqpoint{2.970706in}{2.792530in}}%
\pgfpathlineto{\pgfqpoint{2.784480in}{2.792530in}}%
\pgfpathlineto{\pgfqpoint{2.784480in}{2.710802in}}%
\pgfusepath{}%
\end{pgfscope}%
\begin{pgfscope}%
\pgfpathrectangle{\pgfqpoint{0.549740in}{0.463273in}}{\pgfqpoint{9.320225in}{4.495057in}}%
\pgfusepath{clip}%
\pgfsetbuttcap%
\pgfsetroundjoin%
\pgfsetlinewidth{0.000000pt}%
\definecolor{currentstroke}{rgb}{0.000000,0.000000,0.000000}%
\pgfsetstrokecolor{currentstroke}%
\pgfsetdash{}{0pt}%
\pgfpathmoveto{\pgfqpoint{2.970706in}{2.710802in}}%
\pgfpathlineto{\pgfqpoint{3.156933in}{2.710802in}}%
\pgfpathlineto{\pgfqpoint{3.156933in}{2.792530in}}%
\pgfpathlineto{\pgfqpoint{2.970706in}{2.792530in}}%
\pgfpathlineto{\pgfqpoint{2.970706in}{2.710802in}}%
\pgfusepath{}%
\end{pgfscope}%
\begin{pgfscope}%
\pgfpathrectangle{\pgfqpoint{0.549740in}{0.463273in}}{\pgfqpoint{9.320225in}{4.495057in}}%
\pgfusepath{clip}%
\pgfsetbuttcap%
\pgfsetroundjoin%
\pgfsetlinewidth{0.000000pt}%
\definecolor{currentstroke}{rgb}{0.000000,0.000000,0.000000}%
\pgfsetstrokecolor{currentstroke}%
\pgfsetdash{}{0pt}%
\pgfpathmoveto{\pgfqpoint{3.156933in}{2.710802in}}%
\pgfpathlineto{\pgfqpoint{3.343159in}{2.710802in}}%
\pgfpathlineto{\pgfqpoint{3.343159in}{2.792530in}}%
\pgfpathlineto{\pgfqpoint{3.156933in}{2.792530in}}%
\pgfpathlineto{\pgfqpoint{3.156933in}{2.710802in}}%
\pgfusepath{}%
\end{pgfscope}%
\begin{pgfscope}%
\pgfpathrectangle{\pgfqpoint{0.549740in}{0.463273in}}{\pgfqpoint{9.320225in}{4.495057in}}%
\pgfusepath{clip}%
\pgfsetbuttcap%
\pgfsetroundjoin%
\pgfsetlinewidth{0.000000pt}%
\definecolor{currentstroke}{rgb}{0.000000,0.000000,0.000000}%
\pgfsetstrokecolor{currentstroke}%
\pgfsetdash{}{0pt}%
\pgfpathmoveto{\pgfqpoint{3.343159in}{2.710802in}}%
\pgfpathlineto{\pgfqpoint{3.529386in}{2.710802in}}%
\pgfpathlineto{\pgfqpoint{3.529386in}{2.792530in}}%
\pgfpathlineto{\pgfqpoint{3.343159in}{2.792530in}}%
\pgfpathlineto{\pgfqpoint{3.343159in}{2.710802in}}%
\pgfusepath{}%
\end{pgfscope}%
\begin{pgfscope}%
\pgfpathrectangle{\pgfqpoint{0.549740in}{0.463273in}}{\pgfqpoint{9.320225in}{4.495057in}}%
\pgfusepath{clip}%
\pgfsetbuttcap%
\pgfsetroundjoin%
\pgfsetlinewidth{0.000000pt}%
\definecolor{currentstroke}{rgb}{0.000000,0.000000,0.000000}%
\pgfsetstrokecolor{currentstroke}%
\pgfsetdash{}{0pt}%
\pgfpathmoveto{\pgfqpoint{3.529386in}{2.710802in}}%
\pgfpathlineto{\pgfqpoint{3.715612in}{2.710802in}}%
\pgfpathlineto{\pgfqpoint{3.715612in}{2.792530in}}%
\pgfpathlineto{\pgfqpoint{3.529386in}{2.792530in}}%
\pgfpathlineto{\pgfqpoint{3.529386in}{2.710802in}}%
\pgfusepath{}%
\end{pgfscope}%
\begin{pgfscope}%
\pgfpathrectangle{\pgfqpoint{0.549740in}{0.463273in}}{\pgfqpoint{9.320225in}{4.495057in}}%
\pgfusepath{clip}%
\pgfsetbuttcap%
\pgfsetroundjoin%
\pgfsetlinewidth{0.000000pt}%
\definecolor{currentstroke}{rgb}{0.000000,0.000000,0.000000}%
\pgfsetstrokecolor{currentstroke}%
\pgfsetdash{}{0pt}%
\pgfpathmoveto{\pgfqpoint{3.715612in}{2.710802in}}%
\pgfpathlineto{\pgfqpoint{3.901839in}{2.710802in}}%
\pgfpathlineto{\pgfqpoint{3.901839in}{2.792530in}}%
\pgfpathlineto{\pgfqpoint{3.715612in}{2.792530in}}%
\pgfpathlineto{\pgfqpoint{3.715612in}{2.710802in}}%
\pgfusepath{}%
\end{pgfscope}%
\begin{pgfscope}%
\pgfpathrectangle{\pgfqpoint{0.549740in}{0.463273in}}{\pgfqpoint{9.320225in}{4.495057in}}%
\pgfusepath{clip}%
\pgfsetbuttcap%
\pgfsetroundjoin%
\pgfsetlinewidth{0.000000pt}%
\definecolor{currentstroke}{rgb}{0.000000,0.000000,0.000000}%
\pgfsetstrokecolor{currentstroke}%
\pgfsetdash{}{0pt}%
\pgfpathmoveto{\pgfqpoint{3.901839in}{2.710802in}}%
\pgfpathlineto{\pgfqpoint{4.088065in}{2.710802in}}%
\pgfpathlineto{\pgfqpoint{4.088065in}{2.792530in}}%
\pgfpathlineto{\pgfqpoint{3.901839in}{2.792530in}}%
\pgfpathlineto{\pgfqpoint{3.901839in}{2.710802in}}%
\pgfusepath{}%
\end{pgfscope}%
\begin{pgfscope}%
\pgfpathrectangle{\pgfqpoint{0.549740in}{0.463273in}}{\pgfqpoint{9.320225in}{4.495057in}}%
\pgfusepath{clip}%
\pgfsetbuttcap%
\pgfsetroundjoin%
\pgfsetlinewidth{0.000000pt}%
\definecolor{currentstroke}{rgb}{0.000000,0.000000,0.000000}%
\pgfsetstrokecolor{currentstroke}%
\pgfsetdash{}{0pt}%
\pgfpathmoveto{\pgfqpoint{4.088065in}{2.710802in}}%
\pgfpathlineto{\pgfqpoint{4.274292in}{2.710802in}}%
\pgfpathlineto{\pgfqpoint{4.274292in}{2.792530in}}%
\pgfpathlineto{\pgfqpoint{4.088065in}{2.792530in}}%
\pgfpathlineto{\pgfqpoint{4.088065in}{2.710802in}}%
\pgfusepath{}%
\end{pgfscope}%
\begin{pgfscope}%
\pgfpathrectangle{\pgfqpoint{0.549740in}{0.463273in}}{\pgfqpoint{9.320225in}{4.495057in}}%
\pgfusepath{clip}%
\pgfsetbuttcap%
\pgfsetroundjoin%
\pgfsetlinewidth{0.000000pt}%
\definecolor{currentstroke}{rgb}{0.000000,0.000000,0.000000}%
\pgfsetstrokecolor{currentstroke}%
\pgfsetdash{}{0pt}%
\pgfpathmoveto{\pgfqpoint{4.274292in}{2.710802in}}%
\pgfpathlineto{\pgfqpoint{4.460519in}{2.710802in}}%
\pgfpathlineto{\pgfqpoint{4.460519in}{2.792530in}}%
\pgfpathlineto{\pgfqpoint{4.274292in}{2.792530in}}%
\pgfpathlineto{\pgfqpoint{4.274292in}{2.710802in}}%
\pgfusepath{}%
\end{pgfscope}%
\begin{pgfscope}%
\pgfpathrectangle{\pgfqpoint{0.549740in}{0.463273in}}{\pgfqpoint{9.320225in}{4.495057in}}%
\pgfusepath{clip}%
\pgfsetbuttcap%
\pgfsetroundjoin%
\pgfsetlinewidth{0.000000pt}%
\definecolor{currentstroke}{rgb}{0.000000,0.000000,0.000000}%
\pgfsetstrokecolor{currentstroke}%
\pgfsetdash{}{0pt}%
\pgfpathmoveto{\pgfqpoint{4.460519in}{2.710802in}}%
\pgfpathlineto{\pgfqpoint{4.646745in}{2.710802in}}%
\pgfpathlineto{\pgfqpoint{4.646745in}{2.792530in}}%
\pgfpathlineto{\pgfqpoint{4.460519in}{2.792530in}}%
\pgfpathlineto{\pgfqpoint{4.460519in}{2.710802in}}%
\pgfusepath{}%
\end{pgfscope}%
\begin{pgfscope}%
\pgfpathrectangle{\pgfqpoint{0.549740in}{0.463273in}}{\pgfqpoint{9.320225in}{4.495057in}}%
\pgfusepath{clip}%
\pgfsetbuttcap%
\pgfsetroundjoin%
\pgfsetlinewidth{0.000000pt}%
\definecolor{currentstroke}{rgb}{0.000000,0.000000,0.000000}%
\pgfsetstrokecolor{currentstroke}%
\pgfsetdash{}{0pt}%
\pgfpathmoveto{\pgfqpoint{4.646745in}{2.710802in}}%
\pgfpathlineto{\pgfqpoint{4.832972in}{2.710802in}}%
\pgfpathlineto{\pgfqpoint{4.832972in}{2.792530in}}%
\pgfpathlineto{\pgfqpoint{4.646745in}{2.792530in}}%
\pgfpathlineto{\pgfqpoint{4.646745in}{2.710802in}}%
\pgfusepath{}%
\end{pgfscope}%
\begin{pgfscope}%
\pgfpathrectangle{\pgfqpoint{0.549740in}{0.463273in}}{\pgfqpoint{9.320225in}{4.495057in}}%
\pgfusepath{clip}%
\pgfsetbuttcap%
\pgfsetroundjoin%
\pgfsetlinewidth{0.000000pt}%
\definecolor{currentstroke}{rgb}{0.000000,0.000000,0.000000}%
\pgfsetstrokecolor{currentstroke}%
\pgfsetdash{}{0pt}%
\pgfpathmoveto{\pgfqpoint{4.832972in}{2.710802in}}%
\pgfpathlineto{\pgfqpoint{5.019198in}{2.710802in}}%
\pgfpathlineto{\pgfqpoint{5.019198in}{2.792530in}}%
\pgfpathlineto{\pgfqpoint{4.832972in}{2.792530in}}%
\pgfpathlineto{\pgfqpoint{4.832972in}{2.710802in}}%
\pgfusepath{}%
\end{pgfscope}%
\begin{pgfscope}%
\pgfpathrectangle{\pgfqpoint{0.549740in}{0.463273in}}{\pgfqpoint{9.320225in}{4.495057in}}%
\pgfusepath{clip}%
\pgfsetbuttcap%
\pgfsetroundjoin%
\pgfsetlinewidth{0.000000pt}%
\definecolor{currentstroke}{rgb}{0.000000,0.000000,0.000000}%
\pgfsetstrokecolor{currentstroke}%
\pgfsetdash{}{0pt}%
\pgfpathmoveto{\pgfqpoint{5.019198in}{2.710802in}}%
\pgfpathlineto{\pgfqpoint{5.205425in}{2.710802in}}%
\pgfpathlineto{\pgfqpoint{5.205425in}{2.792530in}}%
\pgfpathlineto{\pgfqpoint{5.019198in}{2.792530in}}%
\pgfpathlineto{\pgfqpoint{5.019198in}{2.710802in}}%
\pgfusepath{}%
\end{pgfscope}%
\begin{pgfscope}%
\pgfpathrectangle{\pgfqpoint{0.549740in}{0.463273in}}{\pgfqpoint{9.320225in}{4.495057in}}%
\pgfusepath{clip}%
\pgfsetbuttcap%
\pgfsetroundjoin%
\definecolor{currentfill}{rgb}{0.472869,0.711325,0.955316}%
\pgfsetfillcolor{currentfill}%
\pgfsetlinewidth{0.000000pt}%
\definecolor{currentstroke}{rgb}{0.000000,0.000000,0.000000}%
\pgfsetstrokecolor{currentstroke}%
\pgfsetdash{}{0pt}%
\pgfpathmoveto{\pgfqpoint{5.205425in}{2.710802in}}%
\pgfpathlineto{\pgfqpoint{5.391651in}{2.710802in}}%
\pgfpathlineto{\pgfqpoint{5.391651in}{2.792530in}}%
\pgfpathlineto{\pgfqpoint{5.205425in}{2.792530in}}%
\pgfpathlineto{\pgfqpoint{5.205425in}{2.710802in}}%
\pgfusepath{fill}%
\end{pgfscope}%
\begin{pgfscope}%
\pgfpathrectangle{\pgfqpoint{0.549740in}{0.463273in}}{\pgfqpoint{9.320225in}{4.495057in}}%
\pgfusepath{clip}%
\pgfsetbuttcap%
\pgfsetroundjoin%
\pgfsetlinewidth{0.000000pt}%
\definecolor{currentstroke}{rgb}{0.000000,0.000000,0.000000}%
\pgfsetstrokecolor{currentstroke}%
\pgfsetdash{}{0pt}%
\pgfpathmoveto{\pgfqpoint{5.391651in}{2.710802in}}%
\pgfpathlineto{\pgfqpoint{5.577878in}{2.710802in}}%
\pgfpathlineto{\pgfqpoint{5.577878in}{2.792530in}}%
\pgfpathlineto{\pgfqpoint{5.391651in}{2.792530in}}%
\pgfpathlineto{\pgfqpoint{5.391651in}{2.710802in}}%
\pgfusepath{}%
\end{pgfscope}%
\begin{pgfscope}%
\pgfpathrectangle{\pgfqpoint{0.549740in}{0.463273in}}{\pgfqpoint{9.320225in}{4.495057in}}%
\pgfusepath{clip}%
\pgfsetbuttcap%
\pgfsetroundjoin%
\pgfsetlinewidth{0.000000pt}%
\definecolor{currentstroke}{rgb}{0.000000,0.000000,0.000000}%
\pgfsetstrokecolor{currentstroke}%
\pgfsetdash{}{0pt}%
\pgfpathmoveto{\pgfqpoint{5.577878in}{2.710802in}}%
\pgfpathlineto{\pgfqpoint{5.764104in}{2.710802in}}%
\pgfpathlineto{\pgfqpoint{5.764104in}{2.792530in}}%
\pgfpathlineto{\pgfqpoint{5.577878in}{2.792530in}}%
\pgfpathlineto{\pgfqpoint{5.577878in}{2.710802in}}%
\pgfusepath{}%
\end{pgfscope}%
\begin{pgfscope}%
\pgfpathrectangle{\pgfqpoint{0.549740in}{0.463273in}}{\pgfqpoint{9.320225in}{4.495057in}}%
\pgfusepath{clip}%
\pgfsetbuttcap%
\pgfsetroundjoin%
\pgfsetlinewidth{0.000000pt}%
\definecolor{currentstroke}{rgb}{0.000000,0.000000,0.000000}%
\pgfsetstrokecolor{currentstroke}%
\pgfsetdash{}{0pt}%
\pgfpathmoveto{\pgfqpoint{5.764104in}{2.710802in}}%
\pgfpathlineto{\pgfqpoint{5.950331in}{2.710802in}}%
\pgfpathlineto{\pgfqpoint{5.950331in}{2.792530in}}%
\pgfpathlineto{\pgfqpoint{5.764104in}{2.792530in}}%
\pgfpathlineto{\pgfqpoint{5.764104in}{2.710802in}}%
\pgfusepath{}%
\end{pgfscope}%
\begin{pgfscope}%
\pgfpathrectangle{\pgfqpoint{0.549740in}{0.463273in}}{\pgfqpoint{9.320225in}{4.495057in}}%
\pgfusepath{clip}%
\pgfsetbuttcap%
\pgfsetroundjoin%
\pgfsetlinewidth{0.000000pt}%
\definecolor{currentstroke}{rgb}{0.000000,0.000000,0.000000}%
\pgfsetstrokecolor{currentstroke}%
\pgfsetdash{}{0pt}%
\pgfpathmoveto{\pgfqpoint{5.950331in}{2.710802in}}%
\pgfpathlineto{\pgfqpoint{6.136557in}{2.710802in}}%
\pgfpathlineto{\pgfqpoint{6.136557in}{2.792530in}}%
\pgfpathlineto{\pgfqpoint{5.950331in}{2.792530in}}%
\pgfpathlineto{\pgfqpoint{5.950331in}{2.710802in}}%
\pgfusepath{}%
\end{pgfscope}%
\begin{pgfscope}%
\pgfpathrectangle{\pgfqpoint{0.549740in}{0.463273in}}{\pgfqpoint{9.320225in}{4.495057in}}%
\pgfusepath{clip}%
\pgfsetbuttcap%
\pgfsetroundjoin%
\definecolor{currentfill}{rgb}{0.472869,0.711325,0.955316}%
\pgfsetfillcolor{currentfill}%
\pgfsetlinewidth{0.000000pt}%
\definecolor{currentstroke}{rgb}{0.000000,0.000000,0.000000}%
\pgfsetstrokecolor{currentstroke}%
\pgfsetdash{}{0pt}%
\pgfpathmoveto{\pgfqpoint{6.136557in}{2.710802in}}%
\pgfpathlineto{\pgfqpoint{6.322784in}{2.710802in}}%
\pgfpathlineto{\pgfqpoint{6.322784in}{2.792530in}}%
\pgfpathlineto{\pgfqpoint{6.136557in}{2.792530in}}%
\pgfpathlineto{\pgfqpoint{6.136557in}{2.710802in}}%
\pgfusepath{fill}%
\end{pgfscope}%
\begin{pgfscope}%
\pgfpathrectangle{\pgfqpoint{0.549740in}{0.463273in}}{\pgfqpoint{9.320225in}{4.495057in}}%
\pgfusepath{clip}%
\pgfsetbuttcap%
\pgfsetroundjoin%
\pgfsetlinewidth{0.000000pt}%
\definecolor{currentstroke}{rgb}{0.000000,0.000000,0.000000}%
\pgfsetstrokecolor{currentstroke}%
\pgfsetdash{}{0pt}%
\pgfpathmoveto{\pgfqpoint{6.322784in}{2.710802in}}%
\pgfpathlineto{\pgfqpoint{6.509011in}{2.710802in}}%
\pgfpathlineto{\pgfqpoint{6.509011in}{2.792530in}}%
\pgfpathlineto{\pgfqpoint{6.322784in}{2.792530in}}%
\pgfpathlineto{\pgfqpoint{6.322784in}{2.710802in}}%
\pgfusepath{}%
\end{pgfscope}%
\begin{pgfscope}%
\pgfpathrectangle{\pgfqpoint{0.549740in}{0.463273in}}{\pgfqpoint{9.320225in}{4.495057in}}%
\pgfusepath{clip}%
\pgfsetbuttcap%
\pgfsetroundjoin%
\pgfsetlinewidth{0.000000pt}%
\definecolor{currentstroke}{rgb}{0.000000,0.000000,0.000000}%
\pgfsetstrokecolor{currentstroke}%
\pgfsetdash{}{0pt}%
\pgfpathmoveto{\pgfqpoint{6.509011in}{2.710802in}}%
\pgfpathlineto{\pgfqpoint{6.695237in}{2.710802in}}%
\pgfpathlineto{\pgfqpoint{6.695237in}{2.792530in}}%
\pgfpathlineto{\pgfqpoint{6.509011in}{2.792530in}}%
\pgfpathlineto{\pgfqpoint{6.509011in}{2.710802in}}%
\pgfusepath{}%
\end{pgfscope}%
\begin{pgfscope}%
\pgfpathrectangle{\pgfqpoint{0.549740in}{0.463273in}}{\pgfqpoint{9.320225in}{4.495057in}}%
\pgfusepath{clip}%
\pgfsetbuttcap%
\pgfsetroundjoin%
\pgfsetlinewidth{0.000000pt}%
\definecolor{currentstroke}{rgb}{0.000000,0.000000,0.000000}%
\pgfsetstrokecolor{currentstroke}%
\pgfsetdash{}{0pt}%
\pgfpathmoveto{\pgfqpoint{6.695237in}{2.710802in}}%
\pgfpathlineto{\pgfqpoint{6.881464in}{2.710802in}}%
\pgfpathlineto{\pgfqpoint{6.881464in}{2.792530in}}%
\pgfpathlineto{\pgfqpoint{6.695237in}{2.792530in}}%
\pgfpathlineto{\pgfqpoint{6.695237in}{2.710802in}}%
\pgfusepath{}%
\end{pgfscope}%
\begin{pgfscope}%
\pgfpathrectangle{\pgfqpoint{0.549740in}{0.463273in}}{\pgfqpoint{9.320225in}{4.495057in}}%
\pgfusepath{clip}%
\pgfsetbuttcap%
\pgfsetroundjoin%
\pgfsetlinewidth{0.000000pt}%
\definecolor{currentstroke}{rgb}{0.000000,0.000000,0.000000}%
\pgfsetstrokecolor{currentstroke}%
\pgfsetdash{}{0pt}%
\pgfpathmoveto{\pgfqpoint{6.881464in}{2.710802in}}%
\pgfpathlineto{\pgfqpoint{7.067690in}{2.710802in}}%
\pgfpathlineto{\pgfqpoint{7.067690in}{2.792530in}}%
\pgfpathlineto{\pgfqpoint{6.881464in}{2.792530in}}%
\pgfpathlineto{\pgfqpoint{6.881464in}{2.710802in}}%
\pgfusepath{}%
\end{pgfscope}%
\begin{pgfscope}%
\pgfpathrectangle{\pgfqpoint{0.549740in}{0.463273in}}{\pgfqpoint{9.320225in}{4.495057in}}%
\pgfusepath{clip}%
\pgfsetbuttcap%
\pgfsetroundjoin%
\pgfsetlinewidth{0.000000pt}%
\definecolor{currentstroke}{rgb}{0.000000,0.000000,0.000000}%
\pgfsetstrokecolor{currentstroke}%
\pgfsetdash{}{0pt}%
\pgfpathmoveto{\pgfqpoint{7.067690in}{2.710802in}}%
\pgfpathlineto{\pgfqpoint{7.253917in}{2.710802in}}%
\pgfpathlineto{\pgfqpoint{7.253917in}{2.792530in}}%
\pgfpathlineto{\pgfqpoint{7.067690in}{2.792530in}}%
\pgfpathlineto{\pgfqpoint{7.067690in}{2.710802in}}%
\pgfusepath{}%
\end{pgfscope}%
\begin{pgfscope}%
\pgfpathrectangle{\pgfqpoint{0.549740in}{0.463273in}}{\pgfqpoint{9.320225in}{4.495057in}}%
\pgfusepath{clip}%
\pgfsetbuttcap%
\pgfsetroundjoin%
\definecolor{currentfill}{rgb}{0.472869,0.711325,0.955316}%
\pgfsetfillcolor{currentfill}%
\pgfsetlinewidth{0.000000pt}%
\definecolor{currentstroke}{rgb}{0.000000,0.000000,0.000000}%
\pgfsetstrokecolor{currentstroke}%
\pgfsetdash{}{0pt}%
\pgfpathmoveto{\pgfqpoint{7.253917in}{2.710802in}}%
\pgfpathlineto{\pgfqpoint{7.440143in}{2.710802in}}%
\pgfpathlineto{\pgfqpoint{7.440143in}{2.792530in}}%
\pgfpathlineto{\pgfqpoint{7.253917in}{2.792530in}}%
\pgfpathlineto{\pgfqpoint{7.253917in}{2.710802in}}%
\pgfusepath{fill}%
\end{pgfscope}%
\begin{pgfscope}%
\pgfpathrectangle{\pgfqpoint{0.549740in}{0.463273in}}{\pgfqpoint{9.320225in}{4.495057in}}%
\pgfusepath{clip}%
\pgfsetbuttcap%
\pgfsetroundjoin%
\pgfsetlinewidth{0.000000pt}%
\definecolor{currentstroke}{rgb}{0.000000,0.000000,0.000000}%
\pgfsetstrokecolor{currentstroke}%
\pgfsetdash{}{0pt}%
\pgfpathmoveto{\pgfqpoint{7.440143in}{2.710802in}}%
\pgfpathlineto{\pgfqpoint{7.626370in}{2.710802in}}%
\pgfpathlineto{\pgfqpoint{7.626370in}{2.792530in}}%
\pgfpathlineto{\pgfqpoint{7.440143in}{2.792530in}}%
\pgfpathlineto{\pgfqpoint{7.440143in}{2.710802in}}%
\pgfusepath{}%
\end{pgfscope}%
\begin{pgfscope}%
\pgfpathrectangle{\pgfqpoint{0.549740in}{0.463273in}}{\pgfqpoint{9.320225in}{4.495057in}}%
\pgfusepath{clip}%
\pgfsetbuttcap%
\pgfsetroundjoin%
\pgfsetlinewidth{0.000000pt}%
\definecolor{currentstroke}{rgb}{0.000000,0.000000,0.000000}%
\pgfsetstrokecolor{currentstroke}%
\pgfsetdash{}{0pt}%
\pgfpathmoveto{\pgfqpoint{7.626370in}{2.710802in}}%
\pgfpathlineto{\pgfqpoint{7.812596in}{2.710802in}}%
\pgfpathlineto{\pgfqpoint{7.812596in}{2.792530in}}%
\pgfpathlineto{\pgfqpoint{7.626370in}{2.792530in}}%
\pgfpathlineto{\pgfqpoint{7.626370in}{2.710802in}}%
\pgfusepath{}%
\end{pgfscope}%
\begin{pgfscope}%
\pgfpathrectangle{\pgfqpoint{0.549740in}{0.463273in}}{\pgfqpoint{9.320225in}{4.495057in}}%
\pgfusepath{clip}%
\pgfsetbuttcap%
\pgfsetroundjoin%
\pgfsetlinewidth{0.000000pt}%
\definecolor{currentstroke}{rgb}{0.000000,0.000000,0.000000}%
\pgfsetstrokecolor{currentstroke}%
\pgfsetdash{}{0pt}%
\pgfpathmoveto{\pgfqpoint{7.812596in}{2.710802in}}%
\pgfpathlineto{\pgfqpoint{7.998823in}{2.710802in}}%
\pgfpathlineto{\pgfqpoint{7.998823in}{2.792530in}}%
\pgfpathlineto{\pgfqpoint{7.812596in}{2.792530in}}%
\pgfpathlineto{\pgfqpoint{7.812596in}{2.710802in}}%
\pgfusepath{}%
\end{pgfscope}%
\begin{pgfscope}%
\pgfpathrectangle{\pgfqpoint{0.549740in}{0.463273in}}{\pgfqpoint{9.320225in}{4.495057in}}%
\pgfusepath{clip}%
\pgfsetbuttcap%
\pgfsetroundjoin%
\pgfsetlinewidth{0.000000pt}%
\definecolor{currentstroke}{rgb}{0.000000,0.000000,0.000000}%
\pgfsetstrokecolor{currentstroke}%
\pgfsetdash{}{0pt}%
\pgfpathmoveto{\pgfqpoint{7.998823in}{2.710802in}}%
\pgfpathlineto{\pgfqpoint{8.185049in}{2.710802in}}%
\pgfpathlineto{\pgfqpoint{8.185049in}{2.792530in}}%
\pgfpathlineto{\pgfqpoint{7.998823in}{2.792530in}}%
\pgfpathlineto{\pgfqpoint{7.998823in}{2.710802in}}%
\pgfusepath{}%
\end{pgfscope}%
\begin{pgfscope}%
\pgfpathrectangle{\pgfqpoint{0.549740in}{0.463273in}}{\pgfqpoint{9.320225in}{4.495057in}}%
\pgfusepath{clip}%
\pgfsetbuttcap%
\pgfsetroundjoin%
\definecolor{currentfill}{rgb}{0.472869,0.711325,0.955316}%
\pgfsetfillcolor{currentfill}%
\pgfsetlinewidth{0.000000pt}%
\definecolor{currentstroke}{rgb}{0.000000,0.000000,0.000000}%
\pgfsetstrokecolor{currentstroke}%
\pgfsetdash{}{0pt}%
\pgfpathmoveto{\pgfqpoint{8.185049in}{2.710802in}}%
\pgfpathlineto{\pgfqpoint{8.371276in}{2.710802in}}%
\pgfpathlineto{\pgfqpoint{8.371276in}{2.792530in}}%
\pgfpathlineto{\pgfqpoint{8.185049in}{2.792530in}}%
\pgfpathlineto{\pgfqpoint{8.185049in}{2.710802in}}%
\pgfusepath{fill}%
\end{pgfscope}%
\begin{pgfscope}%
\pgfpathrectangle{\pgfqpoint{0.549740in}{0.463273in}}{\pgfqpoint{9.320225in}{4.495057in}}%
\pgfusepath{clip}%
\pgfsetbuttcap%
\pgfsetroundjoin%
\pgfsetlinewidth{0.000000pt}%
\definecolor{currentstroke}{rgb}{0.000000,0.000000,0.000000}%
\pgfsetstrokecolor{currentstroke}%
\pgfsetdash{}{0pt}%
\pgfpathmoveto{\pgfqpoint{8.371276in}{2.710802in}}%
\pgfpathlineto{\pgfqpoint{8.557503in}{2.710802in}}%
\pgfpathlineto{\pgfqpoint{8.557503in}{2.792530in}}%
\pgfpathlineto{\pgfqpoint{8.371276in}{2.792530in}}%
\pgfpathlineto{\pgfqpoint{8.371276in}{2.710802in}}%
\pgfusepath{}%
\end{pgfscope}%
\begin{pgfscope}%
\pgfpathrectangle{\pgfqpoint{0.549740in}{0.463273in}}{\pgfqpoint{9.320225in}{4.495057in}}%
\pgfusepath{clip}%
\pgfsetbuttcap%
\pgfsetroundjoin%
\pgfsetlinewidth{0.000000pt}%
\definecolor{currentstroke}{rgb}{0.000000,0.000000,0.000000}%
\pgfsetstrokecolor{currentstroke}%
\pgfsetdash{}{0pt}%
\pgfpathmoveto{\pgfqpoint{8.557503in}{2.710802in}}%
\pgfpathlineto{\pgfqpoint{8.743729in}{2.710802in}}%
\pgfpathlineto{\pgfqpoint{8.743729in}{2.792530in}}%
\pgfpathlineto{\pgfqpoint{8.557503in}{2.792530in}}%
\pgfpathlineto{\pgfqpoint{8.557503in}{2.710802in}}%
\pgfusepath{}%
\end{pgfscope}%
\begin{pgfscope}%
\pgfpathrectangle{\pgfqpoint{0.549740in}{0.463273in}}{\pgfqpoint{9.320225in}{4.495057in}}%
\pgfusepath{clip}%
\pgfsetbuttcap%
\pgfsetroundjoin%
\pgfsetlinewidth{0.000000pt}%
\definecolor{currentstroke}{rgb}{0.000000,0.000000,0.000000}%
\pgfsetstrokecolor{currentstroke}%
\pgfsetdash{}{0pt}%
\pgfpathmoveto{\pgfqpoint{8.743729in}{2.710802in}}%
\pgfpathlineto{\pgfqpoint{8.929956in}{2.710802in}}%
\pgfpathlineto{\pgfqpoint{8.929956in}{2.792530in}}%
\pgfpathlineto{\pgfqpoint{8.743729in}{2.792530in}}%
\pgfpathlineto{\pgfqpoint{8.743729in}{2.710802in}}%
\pgfusepath{}%
\end{pgfscope}%
\begin{pgfscope}%
\pgfpathrectangle{\pgfqpoint{0.549740in}{0.463273in}}{\pgfqpoint{9.320225in}{4.495057in}}%
\pgfusepath{clip}%
\pgfsetbuttcap%
\pgfsetroundjoin%
\pgfsetlinewidth{0.000000pt}%
\definecolor{currentstroke}{rgb}{0.000000,0.000000,0.000000}%
\pgfsetstrokecolor{currentstroke}%
\pgfsetdash{}{0pt}%
\pgfpathmoveto{\pgfqpoint{8.929956in}{2.710802in}}%
\pgfpathlineto{\pgfqpoint{9.116182in}{2.710802in}}%
\pgfpathlineto{\pgfqpoint{9.116182in}{2.792530in}}%
\pgfpathlineto{\pgfqpoint{8.929956in}{2.792530in}}%
\pgfpathlineto{\pgfqpoint{8.929956in}{2.710802in}}%
\pgfusepath{}%
\end{pgfscope}%
\begin{pgfscope}%
\pgfpathrectangle{\pgfqpoint{0.549740in}{0.463273in}}{\pgfqpoint{9.320225in}{4.495057in}}%
\pgfusepath{clip}%
\pgfsetbuttcap%
\pgfsetroundjoin%
\pgfsetlinewidth{0.000000pt}%
\definecolor{currentstroke}{rgb}{0.000000,0.000000,0.000000}%
\pgfsetstrokecolor{currentstroke}%
\pgfsetdash{}{0pt}%
\pgfpathmoveto{\pgfqpoint{9.116182in}{2.710802in}}%
\pgfpathlineto{\pgfqpoint{9.302409in}{2.710802in}}%
\pgfpathlineto{\pgfqpoint{9.302409in}{2.792530in}}%
\pgfpathlineto{\pgfqpoint{9.116182in}{2.792530in}}%
\pgfpathlineto{\pgfqpoint{9.116182in}{2.710802in}}%
\pgfusepath{}%
\end{pgfscope}%
\begin{pgfscope}%
\pgfpathrectangle{\pgfqpoint{0.549740in}{0.463273in}}{\pgfqpoint{9.320225in}{4.495057in}}%
\pgfusepath{clip}%
\pgfsetbuttcap%
\pgfsetroundjoin%
\pgfsetlinewidth{0.000000pt}%
\definecolor{currentstroke}{rgb}{0.000000,0.000000,0.000000}%
\pgfsetstrokecolor{currentstroke}%
\pgfsetdash{}{0pt}%
\pgfpathmoveto{\pgfqpoint{9.302409in}{2.710802in}}%
\pgfpathlineto{\pgfqpoint{9.488635in}{2.710802in}}%
\pgfpathlineto{\pgfqpoint{9.488635in}{2.792530in}}%
\pgfpathlineto{\pgfqpoint{9.302409in}{2.792530in}}%
\pgfpathlineto{\pgfqpoint{9.302409in}{2.710802in}}%
\pgfusepath{}%
\end{pgfscope}%
\begin{pgfscope}%
\pgfpathrectangle{\pgfqpoint{0.549740in}{0.463273in}}{\pgfqpoint{9.320225in}{4.495057in}}%
\pgfusepath{clip}%
\pgfsetbuttcap%
\pgfsetroundjoin%
\definecolor{currentfill}{rgb}{0.472869,0.711325,0.955316}%
\pgfsetfillcolor{currentfill}%
\pgfsetlinewidth{0.000000pt}%
\definecolor{currentstroke}{rgb}{0.000000,0.000000,0.000000}%
\pgfsetstrokecolor{currentstroke}%
\pgfsetdash{}{0pt}%
\pgfpathmoveto{\pgfqpoint{9.488635in}{2.710802in}}%
\pgfpathlineto{\pgfqpoint{9.674862in}{2.710802in}}%
\pgfpathlineto{\pgfqpoint{9.674862in}{2.792530in}}%
\pgfpathlineto{\pgfqpoint{9.488635in}{2.792530in}}%
\pgfpathlineto{\pgfqpoint{9.488635in}{2.710802in}}%
\pgfusepath{fill}%
\end{pgfscope}%
\begin{pgfscope}%
\pgfpathrectangle{\pgfqpoint{0.549740in}{0.463273in}}{\pgfqpoint{9.320225in}{4.495057in}}%
\pgfusepath{clip}%
\pgfsetbuttcap%
\pgfsetroundjoin%
\pgfsetlinewidth{0.000000pt}%
\definecolor{currentstroke}{rgb}{0.000000,0.000000,0.000000}%
\pgfsetstrokecolor{currentstroke}%
\pgfsetdash{}{0pt}%
\pgfpathmoveto{\pgfqpoint{9.674862in}{2.710802in}}%
\pgfpathlineto{\pgfqpoint{9.861088in}{2.710802in}}%
\pgfpathlineto{\pgfqpoint{9.861088in}{2.792530in}}%
\pgfpathlineto{\pgfqpoint{9.674862in}{2.792530in}}%
\pgfpathlineto{\pgfqpoint{9.674862in}{2.710802in}}%
\pgfusepath{}%
\end{pgfscope}%
\begin{pgfscope}%
\pgfpathrectangle{\pgfqpoint{0.549740in}{0.463273in}}{\pgfqpoint{9.320225in}{4.495057in}}%
\pgfusepath{clip}%
\pgfsetbuttcap%
\pgfsetroundjoin%
\pgfsetlinewidth{0.000000pt}%
\definecolor{currentstroke}{rgb}{0.000000,0.000000,0.000000}%
\pgfsetstrokecolor{currentstroke}%
\pgfsetdash{}{0pt}%
\pgfpathmoveto{\pgfqpoint{0.549761in}{2.792530in}}%
\pgfpathlineto{\pgfqpoint{0.735988in}{2.792530in}}%
\pgfpathlineto{\pgfqpoint{0.735988in}{2.874258in}}%
\pgfpathlineto{\pgfqpoint{0.549761in}{2.874258in}}%
\pgfpathlineto{\pgfqpoint{0.549761in}{2.792530in}}%
\pgfusepath{}%
\end{pgfscope}%
\begin{pgfscope}%
\pgfpathrectangle{\pgfqpoint{0.549740in}{0.463273in}}{\pgfqpoint{9.320225in}{4.495057in}}%
\pgfusepath{clip}%
\pgfsetbuttcap%
\pgfsetroundjoin%
\pgfsetlinewidth{0.000000pt}%
\definecolor{currentstroke}{rgb}{0.000000,0.000000,0.000000}%
\pgfsetstrokecolor{currentstroke}%
\pgfsetdash{}{0pt}%
\pgfpathmoveto{\pgfqpoint{0.735988in}{2.792530in}}%
\pgfpathlineto{\pgfqpoint{0.922214in}{2.792530in}}%
\pgfpathlineto{\pgfqpoint{0.922214in}{2.874258in}}%
\pgfpathlineto{\pgfqpoint{0.735988in}{2.874258in}}%
\pgfpathlineto{\pgfqpoint{0.735988in}{2.792530in}}%
\pgfusepath{}%
\end{pgfscope}%
\begin{pgfscope}%
\pgfpathrectangle{\pgfqpoint{0.549740in}{0.463273in}}{\pgfqpoint{9.320225in}{4.495057in}}%
\pgfusepath{clip}%
\pgfsetbuttcap%
\pgfsetroundjoin%
\pgfsetlinewidth{0.000000pt}%
\definecolor{currentstroke}{rgb}{0.000000,0.000000,0.000000}%
\pgfsetstrokecolor{currentstroke}%
\pgfsetdash{}{0pt}%
\pgfpathmoveto{\pgfqpoint{0.922214in}{2.792530in}}%
\pgfpathlineto{\pgfqpoint{1.108441in}{2.792530in}}%
\pgfpathlineto{\pgfqpoint{1.108441in}{2.874258in}}%
\pgfpathlineto{\pgfqpoint{0.922214in}{2.874258in}}%
\pgfpathlineto{\pgfqpoint{0.922214in}{2.792530in}}%
\pgfusepath{}%
\end{pgfscope}%
\begin{pgfscope}%
\pgfpathrectangle{\pgfqpoint{0.549740in}{0.463273in}}{\pgfqpoint{9.320225in}{4.495057in}}%
\pgfusepath{clip}%
\pgfsetbuttcap%
\pgfsetroundjoin%
\pgfsetlinewidth{0.000000pt}%
\definecolor{currentstroke}{rgb}{0.000000,0.000000,0.000000}%
\pgfsetstrokecolor{currentstroke}%
\pgfsetdash{}{0pt}%
\pgfpathmoveto{\pgfqpoint{1.108441in}{2.792530in}}%
\pgfpathlineto{\pgfqpoint{1.294667in}{2.792530in}}%
\pgfpathlineto{\pgfqpoint{1.294667in}{2.874258in}}%
\pgfpathlineto{\pgfqpoint{1.108441in}{2.874258in}}%
\pgfpathlineto{\pgfqpoint{1.108441in}{2.792530in}}%
\pgfusepath{}%
\end{pgfscope}%
\begin{pgfscope}%
\pgfpathrectangle{\pgfqpoint{0.549740in}{0.463273in}}{\pgfqpoint{9.320225in}{4.495057in}}%
\pgfusepath{clip}%
\pgfsetbuttcap%
\pgfsetroundjoin%
\pgfsetlinewidth{0.000000pt}%
\definecolor{currentstroke}{rgb}{0.000000,0.000000,0.000000}%
\pgfsetstrokecolor{currentstroke}%
\pgfsetdash{}{0pt}%
\pgfpathmoveto{\pgfqpoint{1.294667in}{2.792530in}}%
\pgfpathlineto{\pgfqpoint{1.480894in}{2.792530in}}%
\pgfpathlineto{\pgfqpoint{1.480894in}{2.874258in}}%
\pgfpathlineto{\pgfqpoint{1.294667in}{2.874258in}}%
\pgfpathlineto{\pgfqpoint{1.294667in}{2.792530in}}%
\pgfusepath{}%
\end{pgfscope}%
\begin{pgfscope}%
\pgfpathrectangle{\pgfqpoint{0.549740in}{0.463273in}}{\pgfqpoint{9.320225in}{4.495057in}}%
\pgfusepath{clip}%
\pgfsetbuttcap%
\pgfsetroundjoin%
\pgfsetlinewidth{0.000000pt}%
\definecolor{currentstroke}{rgb}{0.000000,0.000000,0.000000}%
\pgfsetstrokecolor{currentstroke}%
\pgfsetdash{}{0pt}%
\pgfpathmoveto{\pgfqpoint{1.480894in}{2.792530in}}%
\pgfpathlineto{\pgfqpoint{1.667120in}{2.792530in}}%
\pgfpathlineto{\pgfqpoint{1.667120in}{2.874258in}}%
\pgfpathlineto{\pgfqpoint{1.480894in}{2.874258in}}%
\pgfpathlineto{\pgfqpoint{1.480894in}{2.792530in}}%
\pgfusepath{}%
\end{pgfscope}%
\begin{pgfscope}%
\pgfpathrectangle{\pgfqpoint{0.549740in}{0.463273in}}{\pgfqpoint{9.320225in}{4.495057in}}%
\pgfusepath{clip}%
\pgfsetbuttcap%
\pgfsetroundjoin%
\pgfsetlinewidth{0.000000pt}%
\definecolor{currentstroke}{rgb}{0.000000,0.000000,0.000000}%
\pgfsetstrokecolor{currentstroke}%
\pgfsetdash{}{0pt}%
\pgfpathmoveto{\pgfqpoint{1.667120in}{2.792530in}}%
\pgfpathlineto{\pgfqpoint{1.853347in}{2.792530in}}%
\pgfpathlineto{\pgfqpoint{1.853347in}{2.874258in}}%
\pgfpathlineto{\pgfqpoint{1.667120in}{2.874258in}}%
\pgfpathlineto{\pgfqpoint{1.667120in}{2.792530in}}%
\pgfusepath{}%
\end{pgfscope}%
\begin{pgfscope}%
\pgfpathrectangle{\pgfqpoint{0.549740in}{0.463273in}}{\pgfqpoint{9.320225in}{4.495057in}}%
\pgfusepath{clip}%
\pgfsetbuttcap%
\pgfsetroundjoin%
\pgfsetlinewidth{0.000000pt}%
\definecolor{currentstroke}{rgb}{0.000000,0.000000,0.000000}%
\pgfsetstrokecolor{currentstroke}%
\pgfsetdash{}{0pt}%
\pgfpathmoveto{\pgfqpoint{1.853347in}{2.792530in}}%
\pgfpathlineto{\pgfqpoint{2.039573in}{2.792530in}}%
\pgfpathlineto{\pgfqpoint{2.039573in}{2.874258in}}%
\pgfpathlineto{\pgfqpoint{1.853347in}{2.874258in}}%
\pgfpathlineto{\pgfqpoint{1.853347in}{2.792530in}}%
\pgfusepath{}%
\end{pgfscope}%
\begin{pgfscope}%
\pgfpathrectangle{\pgfqpoint{0.549740in}{0.463273in}}{\pgfqpoint{9.320225in}{4.495057in}}%
\pgfusepath{clip}%
\pgfsetbuttcap%
\pgfsetroundjoin%
\pgfsetlinewidth{0.000000pt}%
\definecolor{currentstroke}{rgb}{0.000000,0.000000,0.000000}%
\pgfsetstrokecolor{currentstroke}%
\pgfsetdash{}{0pt}%
\pgfpathmoveto{\pgfqpoint{2.039573in}{2.792530in}}%
\pgfpathlineto{\pgfqpoint{2.225800in}{2.792530in}}%
\pgfpathlineto{\pgfqpoint{2.225800in}{2.874258in}}%
\pgfpathlineto{\pgfqpoint{2.039573in}{2.874258in}}%
\pgfpathlineto{\pgfqpoint{2.039573in}{2.792530in}}%
\pgfusepath{}%
\end{pgfscope}%
\begin{pgfscope}%
\pgfpathrectangle{\pgfqpoint{0.549740in}{0.463273in}}{\pgfqpoint{9.320225in}{4.495057in}}%
\pgfusepath{clip}%
\pgfsetbuttcap%
\pgfsetroundjoin%
\pgfsetlinewidth{0.000000pt}%
\definecolor{currentstroke}{rgb}{0.000000,0.000000,0.000000}%
\pgfsetstrokecolor{currentstroke}%
\pgfsetdash{}{0pt}%
\pgfpathmoveto{\pgfqpoint{2.225800in}{2.792530in}}%
\pgfpathlineto{\pgfqpoint{2.412027in}{2.792530in}}%
\pgfpathlineto{\pgfqpoint{2.412027in}{2.874258in}}%
\pgfpathlineto{\pgfqpoint{2.225800in}{2.874258in}}%
\pgfpathlineto{\pgfqpoint{2.225800in}{2.792530in}}%
\pgfusepath{}%
\end{pgfscope}%
\begin{pgfscope}%
\pgfpathrectangle{\pgfqpoint{0.549740in}{0.463273in}}{\pgfqpoint{9.320225in}{4.495057in}}%
\pgfusepath{clip}%
\pgfsetbuttcap%
\pgfsetroundjoin%
\pgfsetlinewidth{0.000000pt}%
\definecolor{currentstroke}{rgb}{0.000000,0.000000,0.000000}%
\pgfsetstrokecolor{currentstroke}%
\pgfsetdash{}{0pt}%
\pgfpathmoveto{\pgfqpoint{2.412027in}{2.792530in}}%
\pgfpathlineto{\pgfqpoint{2.598253in}{2.792530in}}%
\pgfpathlineto{\pgfqpoint{2.598253in}{2.874258in}}%
\pgfpathlineto{\pgfqpoint{2.412027in}{2.874258in}}%
\pgfpathlineto{\pgfqpoint{2.412027in}{2.792530in}}%
\pgfusepath{}%
\end{pgfscope}%
\begin{pgfscope}%
\pgfpathrectangle{\pgfqpoint{0.549740in}{0.463273in}}{\pgfqpoint{9.320225in}{4.495057in}}%
\pgfusepath{clip}%
\pgfsetbuttcap%
\pgfsetroundjoin%
\pgfsetlinewidth{0.000000pt}%
\definecolor{currentstroke}{rgb}{0.000000,0.000000,0.000000}%
\pgfsetstrokecolor{currentstroke}%
\pgfsetdash{}{0pt}%
\pgfpathmoveto{\pgfqpoint{2.598253in}{2.792530in}}%
\pgfpathlineto{\pgfqpoint{2.784480in}{2.792530in}}%
\pgfpathlineto{\pgfqpoint{2.784480in}{2.874258in}}%
\pgfpathlineto{\pgfqpoint{2.598253in}{2.874258in}}%
\pgfpathlineto{\pgfqpoint{2.598253in}{2.792530in}}%
\pgfusepath{}%
\end{pgfscope}%
\begin{pgfscope}%
\pgfpathrectangle{\pgfqpoint{0.549740in}{0.463273in}}{\pgfqpoint{9.320225in}{4.495057in}}%
\pgfusepath{clip}%
\pgfsetbuttcap%
\pgfsetroundjoin%
\pgfsetlinewidth{0.000000pt}%
\definecolor{currentstroke}{rgb}{0.000000,0.000000,0.000000}%
\pgfsetstrokecolor{currentstroke}%
\pgfsetdash{}{0pt}%
\pgfpathmoveto{\pgfqpoint{2.784480in}{2.792530in}}%
\pgfpathlineto{\pgfqpoint{2.970706in}{2.792530in}}%
\pgfpathlineto{\pgfqpoint{2.970706in}{2.874258in}}%
\pgfpathlineto{\pgfqpoint{2.784480in}{2.874258in}}%
\pgfpathlineto{\pgfqpoint{2.784480in}{2.792530in}}%
\pgfusepath{}%
\end{pgfscope}%
\begin{pgfscope}%
\pgfpathrectangle{\pgfqpoint{0.549740in}{0.463273in}}{\pgfqpoint{9.320225in}{4.495057in}}%
\pgfusepath{clip}%
\pgfsetbuttcap%
\pgfsetroundjoin%
\pgfsetlinewidth{0.000000pt}%
\definecolor{currentstroke}{rgb}{0.000000,0.000000,0.000000}%
\pgfsetstrokecolor{currentstroke}%
\pgfsetdash{}{0pt}%
\pgfpathmoveto{\pgfqpoint{2.970706in}{2.792530in}}%
\pgfpathlineto{\pgfqpoint{3.156933in}{2.792530in}}%
\pgfpathlineto{\pgfqpoint{3.156933in}{2.874258in}}%
\pgfpathlineto{\pgfqpoint{2.970706in}{2.874258in}}%
\pgfpathlineto{\pgfqpoint{2.970706in}{2.792530in}}%
\pgfusepath{}%
\end{pgfscope}%
\begin{pgfscope}%
\pgfpathrectangle{\pgfqpoint{0.549740in}{0.463273in}}{\pgfqpoint{9.320225in}{4.495057in}}%
\pgfusepath{clip}%
\pgfsetbuttcap%
\pgfsetroundjoin%
\pgfsetlinewidth{0.000000pt}%
\definecolor{currentstroke}{rgb}{0.000000,0.000000,0.000000}%
\pgfsetstrokecolor{currentstroke}%
\pgfsetdash{}{0pt}%
\pgfpathmoveto{\pgfqpoint{3.156933in}{2.792530in}}%
\pgfpathlineto{\pgfqpoint{3.343159in}{2.792530in}}%
\pgfpathlineto{\pgfqpoint{3.343159in}{2.874258in}}%
\pgfpathlineto{\pgfqpoint{3.156933in}{2.874258in}}%
\pgfpathlineto{\pgfqpoint{3.156933in}{2.792530in}}%
\pgfusepath{}%
\end{pgfscope}%
\begin{pgfscope}%
\pgfpathrectangle{\pgfqpoint{0.549740in}{0.463273in}}{\pgfqpoint{9.320225in}{4.495057in}}%
\pgfusepath{clip}%
\pgfsetbuttcap%
\pgfsetroundjoin%
\pgfsetlinewidth{0.000000pt}%
\definecolor{currentstroke}{rgb}{0.000000,0.000000,0.000000}%
\pgfsetstrokecolor{currentstroke}%
\pgfsetdash{}{0pt}%
\pgfpathmoveto{\pgfqpoint{3.343159in}{2.792530in}}%
\pgfpathlineto{\pgfqpoint{3.529386in}{2.792530in}}%
\pgfpathlineto{\pgfqpoint{3.529386in}{2.874258in}}%
\pgfpathlineto{\pgfqpoint{3.343159in}{2.874258in}}%
\pgfpathlineto{\pgfqpoint{3.343159in}{2.792530in}}%
\pgfusepath{}%
\end{pgfscope}%
\begin{pgfscope}%
\pgfpathrectangle{\pgfqpoint{0.549740in}{0.463273in}}{\pgfqpoint{9.320225in}{4.495057in}}%
\pgfusepath{clip}%
\pgfsetbuttcap%
\pgfsetroundjoin%
\pgfsetlinewidth{0.000000pt}%
\definecolor{currentstroke}{rgb}{0.000000,0.000000,0.000000}%
\pgfsetstrokecolor{currentstroke}%
\pgfsetdash{}{0pt}%
\pgfpathmoveto{\pgfqpoint{3.529386in}{2.792530in}}%
\pgfpathlineto{\pgfqpoint{3.715612in}{2.792530in}}%
\pgfpathlineto{\pgfqpoint{3.715612in}{2.874258in}}%
\pgfpathlineto{\pgfqpoint{3.529386in}{2.874258in}}%
\pgfpathlineto{\pgfqpoint{3.529386in}{2.792530in}}%
\pgfusepath{}%
\end{pgfscope}%
\begin{pgfscope}%
\pgfpathrectangle{\pgfqpoint{0.549740in}{0.463273in}}{\pgfqpoint{9.320225in}{4.495057in}}%
\pgfusepath{clip}%
\pgfsetbuttcap%
\pgfsetroundjoin%
\pgfsetlinewidth{0.000000pt}%
\definecolor{currentstroke}{rgb}{0.000000,0.000000,0.000000}%
\pgfsetstrokecolor{currentstroke}%
\pgfsetdash{}{0pt}%
\pgfpathmoveto{\pgfqpoint{3.715612in}{2.792530in}}%
\pgfpathlineto{\pgfqpoint{3.901839in}{2.792530in}}%
\pgfpathlineto{\pgfqpoint{3.901839in}{2.874258in}}%
\pgfpathlineto{\pgfqpoint{3.715612in}{2.874258in}}%
\pgfpathlineto{\pgfqpoint{3.715612in}{2.792530in}}%
\pgfusepath{}%
\end{pgfscope}%
\begin{pgfscope}%
\pgfpathrectangle{\pgfqpoint{0.549740in}{0.463273in}}{\pgfqpoint{9.320225in}{4.495057in}}%
\pgfusepath{clip}%
\pgfsetbuttcap%
\pgfsetroundjoin%
\pgfsetlinewidth{0.000000pt}%
\definecolor{currentstroke}{rgb}{0.000000,0.000000,0.000000}%
\pgfsetstrokecolor{currentstroke}%
\pgfsetdash{}{0pt}%
\pgfpathmoveto{\pgfqpoint{3.901839in}{2.792530in}}%
\pgfpathlineto{\pgfqpoint{4.088065in}{2.792530in}}%
\pgfpathlineto{\pgfqpoint{4.088065in}{2.874258in}}%
\pgfpathlineto{\pgfqpoint{3.901839in}{2.874258in}}%
\pgfpathlineto{\pgfqpoint{3.901839in}{2.792530in}}%
\pgfusepath{}%
\end{pgfscope}%
\begin{pgfscope}%
\pgfpathrectangle{\pgfqpoint{0.549740in}{0.463273in}}{\pgfqpoint{9.320225in}{4.495057in}}%
\pgfusepath{clip}%
\pgfsetbuttcap%
\pgfsetroundjoin%
\pgfsetlinewidth{0.000000pt}%
\definecolor{currentstroke}{rgb}{0.000000,0.000000,0.000000}%
\pgfsetstrokecolor{currentstroke}%
\pgfsetdash{}{0pt}%
\pgfpathmoveto{\pgfqpoint{4.088065in}{2.792530in}}%
\pgfpathlineto{\pgfqpoint{4.274292in}{2.792530in}}%
\pgfpathlineto{\pgfqpoint{4.274292in}{2.874258in}}%
\pgfpathlineto{\pgfqpoint{4.088065in}{2.874258in}}%
\pgfpathlineto{\pgfqpoint{4.088065in}{2.792530in}}%
\pgfusepath{}%
\end{pgfscope}%
\begin{pgfscope}%
\pgfpathrectangle{\pgfqpoint{0.549740in}{0.463273in}}{\pgfqpoint{9.320225in}{4.495057in}}%
\pgfusepath{clip}%
\pgfsetbuttcap%
\pgfsetroundjoin%
\pgfsetlinewidth{0.000000pt}%
\definecolor{currentstroke}{rgb}{0.000000,0.000000,0.000000}%
\pgfsetstrokecolor{currentstroke}%
\pgfsetdash{}{0pt}%
\pgfpathmoveto{\pgfqpoint{4.274292in}{2.792530in}}%
\pgfpathlineto{\pgfqpoint{4.460519in}{2.792530in}}%
\pgfpathlineto{\pgfqpoint{4.460519in}{2.874258in}}%
\pgfpathlineto{\pgfqpoint{4.274292in}{2.874258in}}%
\pgfpathlineto{\pgfqpoint{4.274292in}{2.792530in}}%
\pgfusepath{}%
\end{pgfscope}%
\begin{pgfscope}%
\pgfpathrectangle{\pgfqpoint{0.549740in}{0.463273in}}{\pgfqpoint{9.320225in}{4.495057in}}%
\pgfusepath{clip}%
\pgfsetbuttcap%
\pgfsetroundjoin%
\pgfsetlinewidth{0.000000pt}%
\definecolor{currentstroke}{rgb}{0.000000,0.000000,0.000000}%
\pgfsetstrokecolor{currentstroke}%
\pgfsetdash{}{0pt}%
\pgfpathmoveto{\pgfqpoint{4.460519in}{2.792530in}}%
\pgfpathlineto{\pgfqpoint{4.646745in}{2.792530in}}%
\pgfpathlineto{\pgfqpoint{4.646745in}{2.874258in}}%
\pgfpathlineto{\pgfqpoint{4.460519in}{2.874258in}}%
\pgfpathlineto{\pgfqpoint{4.460519in}{2.792530in}}%
\pgfusepath{}%
\end{pgfscope}%
\begin{pgfscope}%
\pgfpathrectangle{\pgfqpoint{0.549740in}{0.463273in}}{\pgfqpoint{9.320225in}{4.495057in}}%
\pgfusepath{clip}%
\pgfsetbuttcap%
\pgfsetroundjoin%
\pgfsetlinewidth{0.000000pt}%
\definecolor{currentstroke}{rgb}{0.000000,0.000000,0.000000}%
\pgfsetstrokecolor{currentstroke}%
\pgfsetdash{}{0pt}%
\pgfpathmoveto{\pgfqpoint{4.646745in}{2.792530in}}%
\pgfpathlineto{\pgfqpoint{4.832972in}{2.792530in}}%
\pgfpathlineto{\pgfqpoint{4.832972in}{2.874258in}}%
\pgfpathlineto{\pgfqpoint{4.646745in}{2.874258in}}%
\pgfpathlineto{\pgfqpoint{4.646745in}{2.792530in}}%
\pgfusepath{}%
\end{pgfscope}%
\begin{pgfscope}%
\pgfpathrectangle{\pgfqpoint{0.549740in}{0.463273in}}{\pgfqpoint{9.320225in}{4.495057in}}%
\pgfusepath{clip}%
\pgfsetbuttcap%
\pgfsetroundjoin%
\pgfsetlinewidth{0.000000pt}%
\definecolor{currentstroke}{rgb}{0.000000,0.000000,0.000000}%
\pgfsetstrokecolor{currentstroke}%
\pgfsetdash{}{0pt}%
\pgfpathmoveto{\pgfqpoint{4.832972in}{2.792530in}}%
\pgfpathlineto{\pgfqpoint{5.019198in}{2.792530in}}%
\pgfpathlineto{\pgfqpoint{5.019198in}{2.874258in}}%
\pgfpathlineto{\pgfqpoint{4.832972in}{2.874258in}}%
\pgfpathlineto{\pgfqpoint{4.832972in}{2.792530in}}%
\pgfusepath{}%
\end{pgfscope}%
\begin{pgfscope}%
\pgfpathrectangle{\pgfqpoint{0.549740in}{0.463273in}}{\pgfqpoint{9.320225in}{4.495057in}}%
\pgfusepath{clip}%
\pgfsetbuttcap%
\pgfsetroundjoin%
\pgfsetlinewidth{0.000000pt}%
\definecolor{currentstroke}{rgb}{0.000000,0.000000,0.000000}%
\pgfsetstrokecolor{currentstroke}%
\pgfsetdash{}{0pt}%
\pgfpathmoveto{\pgfqpoint{5.019198in}{2.792530in}}%
\pgfpathlineto{\pgfqpoint{5.205425in}{2.792530in}}%
\pgfpathlineto{\pgfqpoint{5.205425in}{2.874258in}}%
\pgfpathlineto{\pgfqpoint{5.019198in}{2.874258in}}%
\pgfpathlineto{\pgfqpoint{5.019198in}{2.792530in}}%
\pgfusepath{}%
\end{pgfscope}%
\begin{pgfscope}%
\pgfpathrectangle{\pgfqpoint{0.549740in}{0.463273in}}{\pgfqpoint{9.320225in}{4.495057in}}%
\pgfusepath{clip}%
\pgfsetbuttcap%
\pgfsetroundjoin%
\definecolor{currentfill}{rgb}{0.472869,0.711325,0.955316}%
\pgfsetfillcolor{currentfill}%
\pgfsetlinewidth{0.000000pt}%
\definecolor{currentstroke}{rgb}{0.000000,0.000000,0.000000}%
\pgfsetstrokecolor{currentstroke}%
\pgfsetdash{}{0pt}%
\pgfpathmoveto{\pgfqpoint{5.205425in}{2.792530in}}%
\pgfpathlineto{\pgfqpoint{5.391651in}{2.792530in}}%
\pgfpathlineto{\pgfqpoint{5.391651in}{2.874258in}}%
\pgfpathlineto{\pgfqpoint{5.205425in}{2.874258in}}%
\pgfpathlineto{\pgfqpoint{5.205425in}{2.792530in}}%
\pgfusepath{fill}%
\end{pgfscope}%
\begin{pgfscope}%
\pgfpathrectangle{\pgfqpoint{0.549740in}{0.463273in}}{\pgfqpoint{9.320225in}{4.495057in}}%
\pgfusepath{clip}%
\pgfsetbuttcap%
\pgfsetroundjoin%
\pgfsetlinewidth{0.000000pt}%
\definecolor{currentstroke}{rgb}{0.000000,0.000000,0.000000}%
\pgfsetstrokecolor{currentstroke}%
\pgfsetdash{}{0pt}%
\pgfpathmoveto{\pgfqpoint{5.391651in}{2.792530in}}%
\pgfpathlineto{\pgfqpoint{5.577878in}{2.792530in}}%
\pgfpathlineto{\pgfqpoint{5.577878in}{2.874258in}}%
\pgfpathlineto{\pgfqpoint{5.391651in}{2.874258in}}%
\pgfpathlineto{\pgfqpoint{5.391651in}{2.792530in}}%
\pgfusepath{}%
\end{pgfscope}%
\begin{pgfscope}%
\pgfpathrectangle{\pgfqpoint{0.549740in}{0.463273in}}{\pgfqpoint{9.320225in}{4.495057in}}%
\pgfusepath{clip}%
\pgfsetbuttcap%
\pgfsetroundjoin%
\pgfsetlinewidth{0.000000pt}%
\definecolor{currentstroke}{rgb}{0.000000,0.000000,0.000000}%
\pgfsetstrokecolor{currentstroke}%
\pgfsetdash{}{0pt}%
\pgfpathmoveto{\pgfqpoint{5.577878in}{2.792530in}}%
\pgfpathlineto{\pgfqpoint{5.764104in}{2.792530in}}%
\pgfpathlineto{\pgfqpoint{5.764104in}{2.874258in}}%
\pgfpathlineto{\pgfqpoint{5.577878in}{2.874258in}}%
\pgfpathlineto{\pgfqpoint{5.577878in}{2.792530in}}%
\pgfusepath{}%
\end{pgfscope}%
\begin{pgfscope}%
\pgfpathrectangle{\pgfqpoint{0.549740in}{0.463273in}}{\pgfqpoint{9.320225in}{4.495057in}}%
\pgfusepath{clip}%
\pgfsetbuttcap%
\pgfsetroundjoin%
\pgfsetlinewidth{0.000000pt}%
\definecolor{currentstroke}{rgb}{0.000000,0.000000,0.000000}%
\pgfsetstrokecolor{currentstroke}%
\pgfsetdash{}{0pt}%
\pgfpathmoveto{\pgfqpoint{5.764104in}{2.792530in}}%
\pgfpathlineto{\pgfqpoint{5.950331in}{2.792530in}}%
\pgfpathlineto{\pgfqpoint{5.950331in}{2.874258in}}%
\pgfpathlineto{\pgfqpoint{5.764104in}{2.874258in}}%
\pgfpathlineto{\pgfqpoint{5.764104in}{2.792530in}}%
\pgfusepath{}%
\end{pgfscope}%
\begin{pgfscope}%
\pgfpathrectangle{\pgfqpoint{0.549740in}{0.463273in}}{\pgfqpoint{9.320225in}{4.495057in}}%
\pgfusepath{clip}%
\pgfsetbuttcap%
\pgfsetroundjoin%
\pgfsetlinewidth{0.000000pt}%
\definecolor{currentstroke}{rgb}{0.000000,0.000000,0.000000}%
\pgfsetstrokecolor{currentstroke}%
\pgfsetdash{}{0pt}%
\pgfpathmoveto{\pgfqpoint{5.950331in}{2.792530in}}%
\pgfpathlineto{\pgfqpoint{6.136557in}{2.792530in}}%
\pgfpathlineto{\pgfqpoint{6.136557in}{2.874258in}}%
\pgfpathlineto{\pgfqpoint{5.950331in}{2.874258in}}%
\pgfpathlineto{\pgfqpoint{5.950331in}{2.792530in}}%
\pgfusepath{}%
\end{pgfscope}%
\begin{pgfscope}%
\pgfpathrectangle{\pgfqpoint{0.549740in}{0.463273in}}{\pgfqpoint{9.320225in}{4.495057in}}%
\pgfusepath{clip}%
\pgfsetbuttcap%
\pgfsetroundjoin%
\definecolor{currentfill}{rgb}{0.472869,0.711325,0.955316}%
\pgfsetfillcolor{currentfill}%
\pgfsetlinewidth{0.000000pt}%
\definecolor{currentstroke}{rgb}{0.000000,0.000000,0.000000}%
\pgfsetstrokecolor{currentstroke}%
\pgfsetdash{}{0pt}%
\pgfpathmoveto{\pgfqpoint{6.136557in}{2.792530in}}%
\pgfpathlineto{\pgfqpoint{6.322784in}{2.792530in}}%
\pgfpathlineto{\pgfqpoint{6.322784in}{2.874258in}}%
\pgfpathlineto{\pgfqpoint{6.136557in}{2.874258in}}%
\pgfpathlineto{\pgfqpoint{6.136557in}{2.792530in}}%
\pgfusepath{fill}%
\end{pgfscope}%
\begin{pgfscope}%
\pgfpathrectangle{\pgfqpoint{0.549740in}{0.463273in}}{\pgfqpoint{9.320225in}{4.495057in}}%
\pgfusepath{clip}%
\pgfsetbuttcap%
\pgfsetroundjoin%
\pgfsetlinewidth{0.000000pt}%
\definecolor{currentstroke}{rgb}{0.000000,0.000000,0.000000}%
\pgfsetstrokecolor{currentstroke}%
\pgfsetdash{}{0pt}%
\pgfpathmoveto{\pgfqpoint{6.322784in}{2.792530in}}%
\pgfpathlineto{\pgfqpoint{6.509011in}{2.792530in}}%
\pgfpathlineto{\pgfqpoint{6.509011in}{2.874258in}}%
\pgfpathlineto{\pgfqpoint{6.322784in}{2.874258in}}%
\pgfpathlineto{\pgfqpoint{6.322784in}{2.792530in}}%
\pgfusepath{}%
\end{pgfscope}%
\begin{pgfscope}%
\pgfpathrectangle{\pgfqpoint{0.549740in}{0.463273in}}{\pgfqpoint{9.320225in}{4.495057in}}%
\pgfusepath{clip}%
\pgfsetbuttcap%
\pgfsetroundjoin%
\pgfsetlinewidth{0.000000pt}%
\definecolor{currentstroke}{rgb}{0.000000,0.000000,0.000000}%
\pgfsetstrokecolor{currentstroke}%
\pgfsetdash{}{0pt}%
\pgfpathmoveto{\pgfqpoint{6.509011in}{2.792530in}}%
\pgfpathlineto{\pgfqpoint{6.695237in}{2.792530in}}%
\pgfpathlineto{\pgfqpoint{6.695237in}{2.874258in}}%
\pgfpathlineto{\pgfqpoint{6.509011in}{2.874258in}}%
\pgfpathlineto{\pgfqpoint{6.509011in}{2.792530in}}%
\pgfusepath{}%
\end{pgfscope}%
\begin{pgfscope}%
\pgfpathrectangle{\pgfqpoint{0.549740in}{0.463273in}}{\pgfqpoint{9.320225in}{4.495057in}}%
\pgfusepath{clip}%
\pgfsetbuttcap%
\pgfsetroundjoin%
\pgfsetlinewidth{0.000000pt}%
\definecolor{currentstroke}{rgb}{0.000000,0.000000,0.000000}%
\pgfsetstrokecolor{currentstroke}%
\pgfsetdash{}{0pt}%
\pgfpathmoveto{\pgfqpoint{6.695237in}{2.792530in}}%
\pgfpathlineto{\pgfqpoint{6.881464in}{2.792530in}}%
\pgfpathlineto{\pgfqpoint{6.881464in}{2.874258in}}%
\pgfpathlineto{\pgfqpoint{6.695237in}{2.874258in}}%
\pgfpathlineto{\pgfqpoint{6.695237in}{2.792530in}}%
\pgfusepath{}%
\end{pgfscope}%
\begin{pgfscope}%
\pgfpathrectangle{\pgfqpoint{0.549740in}{0.463273in}}{\pgfqpoint{9.320225in}{4.495057in}}%
\pgfusepath{clip}%
\pgfsetbuttcap%
\pgfsetroundjoin%
\pgfsetlinewidth{0.000000pt}%
\definecolor{currentstroke}{rgb}{0.000000,0.000000,0.000000}%
\pgfsetstrokecolor{currentstroke}%
\pgfsetdash{}{0pt}%
\pgfpathmoveto{\pgfqpoint{6.881464in}{2.792530in}}%
\pgfpathlineto{\pgfqpoint{7.067690in}{2.792530in}}%
\pgfpathlineto{\pgfqpoint{7.067690in}{2.874258in}}%
\pgfpathlineto{\pgfqpoint{6.881464in}{2.874258in}}%
\pgfpathlineto{\pgfqpoint{6.881464in}{2.792530in}}%
\pgfusepath{}%
\end{pgfscope}%
\begin{pgfscope}%
\pgfpathrectangle{\pgfqpoint{0.549740in}{0.463273in}}{\pgfqpoint{9.320225in}{4.495057in}}%
\pgfusepath{clip}%
\pgfsetbuttcap%
\pgfsetroundjoin%
\definecolor{currentfill}{rgb}{0.614330,0.761948,0.940009}%
\pgfsetfillcolor{currentfill}%
\pgfsetlinewidth{0.000000pt}%
\definecolor{currentstroke}{rgb}{0.000000,0.000000,0.000000}%
\pgfsetstrokecolor{currentstroke}%
\pgfsetdash{}{0pt}%
\pgfpathmoveto{\pgfqpoint{7.067690in}{2.792530in}}%
\pgfpathlineto{\pgfqpoint{7.253917in}{2.792530in}}%
\pgfpathlineto{\pgfqpoint{7.253917in}{2.874258in}}%
\pgfpathlineto{\pgfqpoint{7.067690in}{2.874258in}}%
\pgfpathlineto{\pgfqpoint{7.067690in}{2.792530in}}%
\pgfusepath{fill}%
\end{pgfscope}%
\begin{pgfscope}%
\pgfpathrectangle{\pgfqpoint{0.549740in}{0.463273in}}{\pgfqpoint{9.320225in}{4.495057in}}%
\pgfusepath{clip}%
\pgfsetbuttcap%
\pgfsetroundjoin%
\definecolor{currentfill}{rgb}{0.547810,0.736432,0.947518}%
\pgfsetfillcolor{currentfill}%
\pgfsetlinewidth{0.000000pt}%
\definecolor{currentstroke}{rgb}{0.000000,0.000000,0.000000}%
\pgfsetstrokecolor{currentstroke}%
\pgfsetdash{}{0pt}%
\pgfpathmoveto{\pgfqpoint{7.253917in}{2.792530in}}%
\pgfpathlineto{\pgfqpoint{7.440143in}{2.792530in}}%
\pgfpathlineto{\pgfqpoint{7.440143in}{2.874258in}}%
\pgfpathlineto{\pgfqpoint{7.253917in}{2.874258in}}%
\pgfpathlineto{\pgfqpoint{7.253917in}{2.792530in}}%
\pgfusepath{fill}%
\end{pgfscope}%
\begin{pgfscope}%
\pgfpathrectangle{\pgfqpoint{0.549740in}{0.463273in}}{\pgfqpoint{9.320225in}{4.495057in}}%
\pgfusepath{clip}%
\pgfsetbuttcap%
\pgfsetroundjoin%
\pgfsetlinewidth{0.000000pt}%
\definecolor{currentstroke}{rgb}{0.000000,0.000000,0.000000}%
\pgfsetstrokecolor{currentstroke}%
\pgfsetdash{}{0pt}%
\pgfpathmoveto{\pgfqpoint{7.440143in}{2.792530in}}%
\pgfpathlineto{\pgfqpoint{7.626370in}{2.792530in}}%
\pgfpathlineto{\pgfqpoint{7.626370in}{2.874258in}}%
\pgfpathlineto{\pgfqpoint{7.440143in}{2.874258in}}%
\pgfpathlineto{\pgfqpoint{7.440143in}{2.792530in}}%
\pgfusepath{}%
\end{pgfscope}%
\begin{pgfscope}%
\pgfpathrectangle{\pgfqpoint{0.549740in}{0.463273in}}{\pgfqpoint{9.320225in}{4.495057in}}%
\pgfusepath{clip}%
\pgfsetbuttcap%
\pgfsetroundjoin%
\pgfsetlinewidth{0.000000pt}%
\definecolor{currentstroke}{rgb}{0.000000,0.000000,0.000000}%
\pgfsetstrokecolor{currentstroke}%
\pgfsetdash{}{0pt}%
\pgfpathmoveto{\pgfqpoint{7.626370in}{2.792530in}}%
\pgfpathlineto{\pgfqpoint{7.812596in}{2.792530in}}%
\pgfpathlineto{\pgfqpoint{7.812596in}{2.874258in}}%
\pgfpathlineto{\pgfqpoint{7.626370in}{2.874258in}}%
\pgfpathlineto{\pgfqpoint{7.626370in}{2.792530in}}%
\pgfusepath{}%
\end{pgfscope}%
\begin{pgfscope}%
\pgfpathrectangle{\pgfqpoint{0.549740in}{0.463273in}}{\pgfqpoint{9.320225in}{4.495057in}}%
\pgfusepath{clip}%
\pgfsetbuttcap%
\pgfsetroundjoin%
\pgfsetlinewidth{0.000000pt}%
\definecolor{currentstroke}{rgb}{0.000000,0.000000,0.000000}%
\pgfsetstrokecolor{currentstroke}%
\pgfsetdash{}{0pt}%
\pgfpathmoveto{\pgfqpoint{7.812596in}{2.792530in}}%
\pgfpathlineto{\pgfqpoint{7.998823in}{2.792530in}}%
\pgfpathlineto{\pgfqpoint{7.998823in}{2.874258in}}%
\pgfpathlineto{\pgfqpoint{7.812596in}{2.874258in}}%
\pgfpathlineto{\pgfqpoint{7.812596in}{2.792530in}}%
\pgfusepath{}%
\end{pgfscope}%
\begin{pgfscope}%
\pgfpathrectangle{\pgfqpoint{0.549740in}{0.463273in}}{\pgfqpoint{9.320225in}{4.495057in}}%
\pgfusepath{clip}%
\pgfsetbuttcap%
\pgfsetroundjoin%
\pgfsetlinewidth{0.000000pt}%
\definecolor{currentstroke}{rgb}{0.000000,0.000000,0.000000}%
\pgfsetstrokecolor{currentstroke}%
\pgfsetdash{}{0pt}%
\pgfpathmoveto{\pgfqpoint{7.998823in}{2.792530in}}%
\pgfpathlineto{\pgfqpoint{8.185049in}{2.792530in}}%
\pgfpathlineto{\pgfqpoint{8.185049in}{2.874258in}}%
\pgfpathlineto{\pgfqpoint{7.998823in}{2.874258in}}%
\pgfpathlineto{\pgfqpoint{7.998823in}{2.792530in}}%
\pgfusepath{}%
\end{pgfscope}%
\begin{pgfscope}%
\pgfpathrectangle{\pgfqpoint{0.549740in}{0.463273in}}{\pgfqpoint{9.320225in}{4.495057in}}%
\pgfusepath{clip}%
\pgfsetbuttcap%
\pgfsetroundjoin%
\definecolor{currentfill}{rgb}{0.472869,0.711325,0.955316}%
\pgfsetfillcolor{currentfill}%
\pgfsetlinewidth{0.000000pt}%
\definecolor{currentstroke}{rgb}{0.000000,0.000000,0.000000}%
\pgfsetstrokecolor{currentstroke}%
\pgfsetdash{}{0pt}%
\pgfpathmoveto{\pgfqpoint{8.185049in}{2.792530in}}%
\pgfpathlineto{\pgfqpoint{8.371276in}{2.792530in}}%
\pgfpathlineto{\pgfqpoint{8.371276in}{2.874258in}}%
\pgfpathlineto{\pgfqpoint{8.185049in}{2.874258in}}%
\pgfpathlineto{\pgfqpoint{8.185049in}{2.792530in}}%
\pgfusepath{fill}%
\end{pgfscope}%
\begin{pgfscope}%
\pgfpathrectangle{\pgfqpoint{0.549740in}{0.463273in}}{\pgfqpoint{9.320225in}{4.495057in}}%
\pgfusepath{clip}%
\pgfsetbuttcap%
\pgfsetroundjoin%
\pgfsetlinewidth{0.000000pt}%
\definecolor{currentstroke}{rgb}{0.000000,0.000000,0.000000}%
\pgfsetstrokecolor{currentstroke}%
\pgfsetdash{}{0pt}%
\pgfpathmoveto{\pgfqpoint{8.371276in}{2.792530in}}%
\pgfpathlineto{\pgfqpoint{8.557503in}{2.792530in}}%
\pgfpathlineto{\pgfqpoint{8.557503in}{2.874258in}}%
\pgfpathlineto{\pgfqpoint{8.371276in}{2.874258in}}%
\pgfpathlineto{\pgfqpoint{8.371276in}{2.792530in}}%
\pgfusepath{}%
\end{pgfscope}%
\begin{pgfscope}%
\pgfpathrectangle{\pgfqpoint{0.549740in}{0.463273in}}{\pgfqpoint{9.320225in}{4.495057in}}%
\pgfusepath{clip}%
\pgfsetbuttcap%
\pgfsetroundjoin%
\pgfsetlinewidth{0.000000pt}%
\definecolor{currentstroke}{rgb}{0.000000,0.000000,0.000000}%
\pgfsetstrokecolor{currentstroke}%
\pgfsetdash{}{0pt}%
\pgfpathmoveto{\pgfqpoint{8.557503in}{2.792530in}}%
\pgfpathlineto{\pgfqpoint{8.743729in}{2.792530in}}%
\pgfpathlineto{\pgfqpoint{8.743729in}{2.874258in}}%
\pgfpathlineto{\pgfqpoint{8.557503in}{2.874258in}}%
\pgfpathlineto{\pgfqpoint{8.557503in}{2.792530in}}%
\pgfusepath{}%
\end{pgfscope}%
\begin{pgfscope}%
\pgfpathrectangle{\pgfqpoint{0.549740in}{0.463273in}}{\pgfqpoint{9.320225in}{4.495057in}}%
\pgfusepath{clip}%
\pgfsetbuttcap%
\pgfsetroundjoin%
\pgfsetlinewidth{0.000000pt}%
\definecolor{currentstroke}{rgb}{0.000000,0.000000,0.000000}%
\pgfsetstrokecolor{currentstroke}%
\pgfsetdash{}{0pt}%
\pgfpathmoveto{\pgfqpoint{8.743729in}{2.792530in}}%
\pgfpathlineto{\pgfqpoint{8.929956in}{2.792530in}}%
\pgfpathlineto{\pgfqpoint{8.929956in}{2.874258in}}%
\pgfpathlineto{\pgfqpoint{8.743729in}{2.874258in}}%
\pgfpathlineto{\pgfqpoint{8.743729in}{2.792530in}}%
\pgfusepath{}%
\end{pgfscope}%
\begin{pgfscope}%
\pgfpathrectangle{\pgfqpoint{0.549740in}{0.463273in}}{\pgfqpoint{9.320225in}{4.495057in}}%
\pgfusepath{clip}%
\pgfsetbuttcap%
\pgfsetroundjoin%
\pgfsetlinewidth{0.000000pt}%
\definecolor{currentstroke}{rgb}{0.000000,0.000000,0.000000}%
\pgfsetstrokecolor{currentstroke}%
\pgfsetdash{}{0pt}%
\pgfpathmoveto{\pgfqpoint{8.929956in}{2.792530in}}%
\pgfpathlineto{\pgfqpoint{9.116182in}{2.792530in}}%
\pgfpathlineto{\pgfqpoint{9.116182in}{2.874258in}}%
\pgfpathlineto{\pgfqpoint{8.929956in}{2.874258in}}%
\pgfpathlineto{\pgfqpoint{8.929956in}{2.792530in}}%
\pgfusepath{}%
\end{pgfscope}%
\begin{pgfscope}%
\pgfpathrectangle{\pgfqpoint{0.549740in}{0.463273in}}{\pgfqpoint{9.320225in}{4.495057in}}%
\pgfusepath{clip}%
\pgfsetbuttcap%
\pgfsetroundjoin%
\pgfsetlinewidth{0.000000pt}%
\definecolor{currentstroke}{rgb}{0.000000,0.000000,0.000000}%
\pgfsetstrokecolor{currentstroke}%
\pgfsetdash{}{0pt}%
\pgfpathmoveto{\pgfqpoint{9.116182in}{2.792530in}}%
\pgfpathlineto{\pgfqpoint{9.302409in}{2.792530in}}%
\pgfpathlineto{\pgfqpoint{9.302409in}{2.874258in}}%
\pgfpathlineto{\pgfqpoint{9.116182in}{2.874258in}}%
\pgfpathlineto{\pgfqpoint{9.116182in}{2.792530in}}%
\pgfusepath{}%
\end{pgfscope}%
\begin{pgfscope}%
\pgfpathrectangle{\pgfqpoint{0.549740in}{0.463273in}}{\pgfqpoint{9.320225in}{4.495057in}}%
\pgfusepath{clip}%
\pgfsetbuttcap%
\pgfsetroundjoin%
\pgfsetlinewidth{0.000000pt}%
\definecolor{currentstroke}{rgb}{0.000000,0.000000,0.000000}%
\pgfsetstrokecolor{currentstroke}%
\pgfsetdash{}{0pt}%
\pgfpathmoveto{\pgfqpoint{9.302409in}{2.792530in}}%
\pgfpathlineto{\pgfqpoint{9.488635in}{2.792530in}}%
\pgfpathlineto{\pgfqpoint{9.488635in}{2.874258in}}%
\pgfpathlineto{\pgfqpoint{9.302409in}{2.874258in}}%
\pgfpathlineto{\pgfqpoint{9.302409in}{2.792530in}}%
\pgfusepath{}%
\end{pgfscope}%
\begin{pgfscope}%
\pgfpathrectangle{\pgfqpoint{0.549740in}{0.463273in}}{\pgfqpoint{9.320225in}{4.495057in}}%
\pgfusepath{clip}%
\pgfsetbuttcap%
\pgfsetroundjoin%
\definecolor{currentfill}{rgb}{0.547810,0.736432,0.947518}%
\pgfsetfillcolor{currentfill}%
\pgfsetlinewidth{0.000000pt}%
\definecolor{currentstroke}{rgb}{0.000000,0.000000,0.000000}%
\pgfsetstrokecolor{currentstroke}%
\pgfsetdash{}{0pt}%
\pgfpathmoveto{\pgfqpoint{9.488635in}{2.792530in}}%
\pgfpathlineto{\pgfqpoint{9.674862in}{2.792530in}}%
\pgfpathlineto{\pgfqpoint{9.674862in}{2.874258in}}%
\pgfpathlineto{\pgfqpoint{9.488635in}{2.874258in}}%
\pgfpathlineto{\pgfqpoint{9.488635in}{2.792530in}}%
\pgfusepath{fill}%
\end{pgfscope}%
\begin{pgfscope}%
\pgfpathrectangle{\pgfqpoint{0.549740in}{0.463273in}}{\pgfqpoint{9.320225in}{4.495057in}}%
\pgfusepath{clip}%
\pgfsetbuttcap%
\pgfsetroundjoin%
\pgfsetlinewidth{0.000000pt}%
\definecolor{currentstroke}{rgb}{0.000000,0.000000,0.000000}%
\pgfsetstrokecolor{currentstroke}%
\pgfsetdash{}{0pt}%
\pgfpathmoveto{\pgfqpoint{9.674862in}{2.792530in}}%
\pgfpathlineto{\pgfqpoint{9.861088in}{2.792530in}}%
\pgfpathlineto{\pgfqpoint{9.861088in}{2.874258in}}%
\pgfpathlineto{\pgfqpoint{9.674862in}{2.874258in}}%
\pgfpathlineto{\pgfqpoint{9.674862in}{2.792530in}}%
\pgfusepath{}%
\end{pgfscope}%
\begin{pgfscope}%
\pgfpathrectangle{\pgfqpoint{0.549740in}{0.463273in}}{\pgfqpoint{9.320225in}{4.495057in}}%
\pgfusepath{clip}%
\pgfsetbuttcap%
\pgfsetroundjoin%
\pgfsetlinewidth{0.000000pt}%
\definecolor{currentstroke}{rgb}{0.000000,0.000000,0.000000}%
\pgfsetstrokecolor{currentstroke}%
\pgfsetdash{}{0pt}%
\pgfpathmoveto{\pgfqpoint{0.549761in}{2.874258in}}%
\pgfpathlineto{\pgfqpoint{0.735988in}{2.874258in}}%
\pgfpathlineto{\pgfqpoint{0.735988in}{2.955987in}}%
\pgfpathlineto{\pgfqpoint{0.549761in}{2.955987in}}%
\pgfpathlineto{\pgfqpoint{0.549761in}{2.874258in}}%
\pgfusepath{}%
\end{pgfscope}%
\begin{pgfscope}%
\pgfpathrectangle{\pgfqpoint{0.549740in}{0.463273in}}{\pgfqpoint{9.320225in}{4.495057in}}%
\pgfusepath{clip}%
\pgfsetbuttcap%
\pgfsetroundjoin%
\pgfsetlinewidth{0.000000pt}%
\definecolor{currentstroke}{rgb}{0.000000,0.000000,0.000000}%
\pgfsetstrokecolor{currentstroke}%
\pgfsetdash{}{0pt}%
\pgfpathmoveto{\pgfqpoint{0.735988in}{2.874258in}}%
\pgfpathlineto{\pgfqpoint{0.922214in}{2.874258in}}%
\pgfpathlineto{\pgfqpoint{0.922214in}{2.955987in}}%
\pgfpathlineto{\pgfqpoint{0.735988in}{2.955987in}}%
\pgfpathlineto{\pgfqpoint{0.735988in}{2.874258in}}%
\pgfusepath{}%
\end{pgfscope}%
\begin{pgfscope}%
\pgfpathrectangle{\pgfqpoint{0.549740in}{0.463273in}}{\pgfqpoint{9.320225in}{4.495057in}}%
\pgfusepath{clip}%
\pgfsetbuttcap%
\pgfsetroundjoin%
\pgfsetlinewidth{0.000000pt}%
\definecolor{currentstroke}{rgb}{0.000000,0.000000,0.000000}%
\pgfsetstrokecolor{currentstroke}%
\pgfsetdash{}{0pt}%
\pgfpathmoveto{\pgfqpoint{0.922214in}{2.874258in}}%
\pgfpathlineto{\pgfqpoint{1.108441in}{2.874258in}}%
\pgfpathlineto{\pgfqpoint{1.108441in}{2.955987in}}%
\pgfpathlineto{\pgfqpoint{0.922214in}{2.955987in}}%
\pgfpathlineto{\pgfqpoint{0.922214in}{2.874258in}}%
\pgfusepath{}%
\end{pgfscope}%
\begin{pgfscope}%
\pgfpathrectangle{\pgfqpoint{0.549740in}{0.463273in}}{\pgfqpoint{9.320225in}{4.495057in}}%
\pgfusepath{clip}%
\pgfsetbuttcap%
\pgfsetroundjoin%
\pgfsetlinewidth{0.000000pt}%
\definecolor{currentstroke}{rgb}{0.000000,0.000000,0.000000}%
\pgfsetstrokecolor{currentstroke}%
\pgfsetdash{}{0pt}%
\pgfpathmoveto{\pgfqpoint{1.108441in}{2.874258in}}%
\pgfpathlineto{\pgfqpoint{1.294667in}{2.874258in}}%
\pgfpathlineto{\pgfqpoint{1.294667in}{2.955987in}}%
\pgfpathlineto{\pgfqpoint{1.108441in}{2.955987in}}%
\pgfpathlineto{\pgfqpoint{1.108441in}{2.874258in}}%
\pgfusepath{}%
\end{pgfscope}%
\begin{pgfscope}%
\pgfpathrectangle{\pgfqpoint{0.549740in}{0.463273in}}{\pgfqpoint{9.320225in}{4.495057in}}%
\pgfusepath{clip}%
\pgfsetbuttcap%
\pgfsetroundjoin%
\pgfsetlinewidth{0.000000pt}%
\definecolor{currentstroke}{rgb}{0.000000,0.000000,0.000000}%
\pgfsetstrokecolor{currentstroke}%
\pgfsetdash{}{0pt}%
\pgfpathmoveto{\pgfqpoint{1.294667in}{2.874258in}}%
\pgfpathlineto{\pgfqpoint{1.480894in}{2.874258in}}%
\pgfpathlineto{\pgfqpoint{1.480894in}{2.955987in}}%
\pgfpathlineto{\pgfqpoint{1.294667in}{2.955987in}}%
\pgfpathlineto{\pgfqpoint{1.294667in}{2.874258in}}%
\pgfusepath{}%
\end{pgfscope}%
\begin{pgfscope}%
\pgfpathrectangle{\pgfqpoint{0.549740in}{0.463273in}}{\pgfqpoint{9.320225in}{4.495057in}}%
\pgfusepath{clip}%
\pgfsetbuttcap%
\pgfsetroundjoin%
\pgfsetlinewidth{0.000000pt}%
\definecolor{currentstroke}{rgb}{0.000000,0.000000,0.000000}%
\pgfsetstrokecolor{currentstroke}%
\pgfsetdash{}{0pt}%
\pgfpathmoveto{\pgfqpoint{1.480894in}{2.874258in}}%
\pgfpathlineto{\pgfqpoint{1.667120in}{2.874258in}}%
\pgfpathlineto{\pgfqpoint{1.667120in}{2.955987in}}%
\pgfpathlineto{\pgfqpoint{1.480894in}{2.955987in}}%
\pgfpathlineto{\pgfqpoint{1.480894in}{2.874258in}}%
\pgfusepath{}%
\end{pgfscope}%
\begin{pgfscope}%
\pgfpathrectangle{\pgfqpoint{0.549740in}{0.463273in}}{\pgfqpoint{9.320225in}{4.495057in}}%
\pgfusepath{clip}%
\pgfsetbuttcap%
\pgfsetroundjoin%
\pgfsetlinewidth{0.000000pt}%
\definecolor{currentstroke}{rgb}{0.000000,0.000000,0.000000}%
\pgfsetstrokecolor{currentstroke}%
\pgfsetdash{}{0pt}%
\pgfpathmoveto{\pgfqpoint{1.667120in}{2.874258in}}%
\pgfpathlineto{\pgfqpoint{1.853347in}{2.874258in}}%
\pgfpathlineto{\pgfqpoint{1.853347in}{2.955987in}}%
\pgfpathlineto{\pgfqpoint{1.667120in}{2.955987in}}%
\pgfpathlineto{\pgfqpoint{1.667120in}{2.874258in}}%
\pgfusepath{}%
\end{pgfscope}%
\begin{pgfscope}%
\pgfpathrectangle{\pgfqpoint{0.549740in}{0.463273in}}{\pgfqpoint{9.320225in}{4.495057in}}%
\pgfusepath{clip}%
\pgfsetbuttcap%
\pgfsetroundjoin%
\pgfsetlinewidth{0.000000pt}%
\definecolor{currentstroke}{rgb}{0.000000,0.000000,0.000000}%
\pgfsetstrokecolor{currentstroke}%
\pgfsetdash{}{0pt}%
\pgfpathmoveto{\pgfqpoint{1.853347in}{2.874258in}}%
\pgfpathlineto{\pgfqpoint{2.039573in}{2.874258in}}%
\pgfpathlineto{\pgfqpoint{2.039573in}{2.955987in}}%
\pgfpathlineto{\pgfqpoint{1.853347in}{2.955987in}}%
\pgfpathlineto{\pgfqpoint{1.853347in}{2.874258in}}%
\pgfusepath{}%
\end{pgfscope}%
\begin{pgfscope}%
\pgfpathrectangle{\pgfqpoint{0.549740in}{0.463273in}}{\pgfqpoint{9.320225in}{4.495057in}}%
\pgfusepath{clip}%
\pgfsetbuttcap%
\pgfsetroundjoin%
\pgfsetlinewidth{0.000000pt}%
\definecolor{currentstroke}{rgb}{0.000000,0.000000,0.000000}%
\pgfsetstrokecolor{currentstroke}%
\pgfsetdash{}{0pt}%
\pgfpathmoveto{\pgfqpoint{2.039573in}{2.874258in}}%
\pgfpathlineto{\pgfqpoint{2.225800in}{2.874258in}}%
\pgfpathlineto{\pgfqpoint{2.225800in}{2.955987in}}%
\pgfpathlineto{\pgfqpoint{2.039573in}{2.955987in}}%
\pgfpathlineto{\pgfqpoint{2.039573in}{2.874258in}}%
\pgfusepath{}%
\end{pgfscope}%
\begin{pgfscope}%
\pgfpathrectangle{\pgfqpoint{0.549740in}{0.463273in}}{\pgfqpoint{9.320225in}{4.495057in}}%
\pgfusepath{clip}%
\pgfsetbuttcap%
\pgfsetroundjoin%
\pgfsetlinewidth{0.000000pt}%
\definecolor{currentstroke}{rgb}{0.000000,0.000000,0.000000}%
\pgfsetstrokecolor{currentstroke}%
\pgfsetdash{}{0pt}%
\pgfpathmoveto{\pgfqpoint{2.225800in}{2.874258in}}%
\pgfpathlineto{\pgfqpoint{2.412027in}{2.874258in}}%
\pgfpathlineto{\pgfqpoint{2.412027in}{2.955987in}}%
\pgfpathlineto{\pgfqpoint{2.225800in}{2.955987in}}%
\pgfpathlineto{\pgfqpoint{2.225800in}{2.874258in}}%
\pgfusepath{}%
\end{pgfscope}%
\begin{pgfscope}%
\pgfpathrectangle{\pgfqpoint{0.549740in}{0.463273in}}{\pgfqpoint{9.320225in}{4.495057in}}%
\pgfusepath{clip}%
\pgfsetbuttcap%
\pgfsetroundjoin%
\pgfsetlinewidth{0.000000pt}%
\definecolor{currentstroke}{rgb}{0.000000,0.000000,0.000000}%
\pgfsetstrokecolor{currentstroke}%
\pgfsetdash{}{0pt}%
\pgfpathmoveto{\pgfqpoint{2.412027in}{2.874258in}}%
\pgfpathlineto{\pgfqpoint{2.598253in}{2.874258in}}%
\pgfpathlineto{\pgfqpoint{2.598253in}{2.955987in}}%
\pgfpathlineto{\pgfqpoint{2.412027in}{2.955987in}}%
\pgfpathlineto{\pgfqpoint{2.412027in}{2.874258in}}%
\pgfusepath{}%
\end{pgfscope}%
\begin{pgfscope}%
\pgfpathrectangle{\pgfqpoint{0.549740in}{0.463273in}}{\pgfqpoint{9.320225in}{4.495057in}}%
\pgfusepath{clip}%
\pgfsetbuttcap%
\pgfsetroundjoin%
\pgfsetlinewidth{0.000000pt}%
\definecolor{currentstroke}{rgb}{0.000000,0.000000,0.000000}%
\pgfsetstrokecolor{currentstroke}%
\pgfsetdash{}{0pt}%
\pgfpathmoveto{\pgfqpoint{2.598253in}{2.874258in}}%
\pgfpathlineto{\pgfqpoint{2.784480in}{2.874258in}}%
\pgfpathlineto{\pgfqpoint{2.784480in}{2.955987in}}%
\pgfpathlineto{\pgfqpoint{2.598253in}{2.955987in}}%
\pgfpathlineto{\pgfqpoint{2.598253in}{2.874258in}}%
\pgfusepath{}%
\end{pgfscope}%
\begin{pgfscope}%
\pgfpathrectangle{\pgfqpoint{0.549740in}{0.463273in}}{\pgfqpoint{9.320225in}{4.495057in}}%
\pgfusepath{clip}%
\pgfsetbuttcap%
\pgfsetroundjoin%
\pgfsetlinewidth{0.000000pt}%
\definecolor{currentstroke}{rgb}{0.000000,0.000000,0.000000}%
\pgfsetstrokecolor{currentstroke}%
\pgfsetdash{}{0pt}%
\pgfpathmoveto{\pgfqpoint{2.784480in}{2.874258in}}%
\pgfpathlineto{\pgfqpoint{2.970706in}{2.874258in}}%
\pgfpathlineto{\pgfqpoint{2.970706in}{2.955987in}}%
\pgfpathlineto{\pgfqpoint{2.784480in}{2.955987in}}%
\pgfpathlineto{\pgfqpoint{2.784480in}{2.874258in}}%
\pgfusepath{}%
\end{pgfscope}%
\begin{pgfscope}%
\pgfpathrectangle{\pgfqpoint{0.549740in}{0.463273in}}{\pgfqpoint{9.320225in}{4.495057in}}%
\pgfusepath{clip}%
\pgfsetbuttcap%
\pgfsetroundjoin%
\pgfsetlinewidth{0.000000pt}%
\definecolor{currentstroke}{rgb}{0.000000,0.000000,0.000000}%
\pgfsetstrokecolor{currentstroke}%
\pgfsetdash{}{0pt}%
\pgfpathmoveto{\pgfqpoint{2.970706in}{2.874258in}}%
\pgfpathlineto{\pgfqpoint{3.156933in}{2.874258in}}%
\pgfpathlineto{\pgfqpoint{3.156933in}{2.955987in}}%
\pgfpathlineto{\pgfqpoint{2.970706in}{2.955987in}}%
\pgfpathlineto{\pgfqpoint{2.970706in}{2.874258in}}%
\pgfusepath{}%
\end{pgfscope}%
\begin{pgfscope}%
\pgfpathrectangle{\pgfqpoint{0.549740in}{0.463273in}}{\pgfqpoint{9.320225in}{4.495057in}}%
\pgfusepath{clip}%
\pgfsetbuttcap%
\pgfsetroundjoin%
\pgfsetlinewidth{0.000000pt}%
\definecolor{currentstroke}{rgb}{0.000000,0.000000,0.000000}%
\pgfsetstrokecolor{currentstroke}%
\pgfsetdash{}{0pt}%
\pgfpathmoveto{\pgfqpoint{3.156933in}{2.874258in}}%
\pgfpathlineto{\pgfqpoint{3.343159in}{2.874258in}}%
\pgfpathlineto{\pgfqpoint{3.343159in}{2.955987in}}%
\pgfpathlineto{\pgfqpoint{3.156933in}{2.955987in}}%
\pgfpathlineto{\pgfqpoint{3.156933in}{2.874258in}}%
\pgfusepath{}%
\end{pgfscope}%
\begin{pgfscope}%
\pgfpathrectangle{\pgfqpoint{0.549740in}{0.463273in}}{\pgfqpoint{9.320225in}{4.495057in}}%
\pgfusepath{clip}%
\pgfsetbuttcap%
\pgfsetroundjoin%
\pgfsetlinewidth{0.000000pt}%
\definecolor{currentstroke}{rgb}{0.000000,0.000000,0.000000}%
\pgfsetstrokecolor{currentstroke}%
\pgfsetdash{}{0pt}%
\pgfpathmoveto{\pgfqpoint{3.343159in}{2.874258in}}%
\pgfpathlineto{\pgfqpoint{3.529386in}{2.874258in}}%
\pgfpathlineto{\pgfqpoint{3.529386in}{2.955987in}}%
\pgfpathlineto{\pgfqpoint{3.343159in}{2.955987in}}%
\pgfpathlineto{\pgfqpoint{3.343159in}{2.874258in}}%
\pgfusepath{}%
\end{pgfscope}%
\begin{pgfscope}%
\pgfpathrectangle{\pgfqpoint{0.549740in}{0.463273in}}{\pgfqpoint{9.320225in}{4.495057in}}%
\pgfusepath{clip}%
\pgfsetbuttcap%
\pgfsetroundjoin%
\pgfsetlinewidth{0.000000pt}%
\definecolor{currentstroke}{rgb}{0.000000,0.000000,0.000000}%
\pgfsetstrokecolor{currentstroke}%
\pgfsetdash{}{0pt}%
\pgfpathmoveto{\pgfqpoint{3.529386in}{2.874258in}}%
\pgfpathlineto{\pgfqpoint{3.715612in}{2.874258in}}%
\pgfpathlineto{\pgfqpoint{3.715612in}{2.955987in}}%
\pgfpathlineto{\pgfqpoint{3.529386in}{2.955987in}}%
\pgfpathlineto{\pgfqpoint{3.529386in}{2.874258in}}%
\pgfusepath{}%
\end{pgfscope}%
\begin{pgfscope}%
\pgfpathrectangle{\pgfqpoint{0.549740in}{0.463273in}}{\pgfqpoint{9.320225in}{4.495057in}}%
\pgfusepath{clip}%
\pgfsetbuttcap%
\pgfsetroundjoin%
\pgfsetlinewidth{0.000000pt}%
\definecolor{currentstroke}{rgb}{0.000000,0.000000,0.000000}%
\pgfsetstrokecolor{currentstroke}%
\pgfsetdash{}{0pt}%
\pgfpathmoveto{\pgfqpoint{3.715612in}{2.874258in}}%
\pgfpathlineto{\pgfqpoint{3.901839in}{2.874258in}}%
\pgfpathlineto{\pgfqpoint{3.901839in}{2.955987in}}%
\pgfpathlineto{\pgfqpoint{3.715612in}{2.955987in}}%
\pgfpathlineto{\pgfqpoint{3.715612in}{2.874258in}}%
\pgfusepath{}%
\end{pgfscope}%
\begin{pgfscope}%
\pgfpathrectangle{\pgfqpoint{0.549740in}{0.463273in}}{\pgfqpoint{9.320225in}{4.495057in}}%
\pgfusepath{clip}%
\pgfsetbuttcap%
\pgfsetroundjoin%
\pgfsetlinewidth{0.000000pt}%
\definecolor{currentstroke}{rgb}{0.000000,0.000000,0.000000}%
\pgfsetstrokecolor{currentstroke}%
\pgfsetdash{}{0pt}%
\pgfpathmoveto{\pgfqpoint{3.901839in}{2.874258in}}%
\pgfpathlineto{\pgfqpoint{4.088065in}{2.874258in}}%
\pgfpathlineto{\pgfqpoint{4.088065in}{2.955987in}}%
\pgfpathlineto{\pgfqpoint{3.901839in}{2.955987in}}%
\pgfpathlineto{\pgfqpoint{3.901839in}{2.874258in}}%
\pgfusepath{}%
\end{pgfscope}%
\begin{pgfscope}%
\pgfpathrectangle{\pgfqpoint{0.549740in}{0.463273in}}{\pgfqpoint{9.320225in}{4.495057in}}%
\pgfusepath{clip}%
\pgfsetbuttcap%
\pgfsetroundjoin%
\pgfsetlinewidth{0.000000pt}%
\definecolor{currentstroke}{rgb}{0.000000,0.000000,0.000000}%
\pgfsetstrokecolor{currentstroke}%
\pgfsetdash{}{0pt}%
\pgfpathmoveto{\pgfqpoint{4.088065in}{2.874258in}}%
\pgfpathlineto{\pgfqpoint{4.274292in}{2.874258in}}%
\pgfpathlineto{\pgfqpoint{4.274292in}{2.955987in}}%
\pgfpathlineto{\pgfqpoint{4.088065in}{2.955987in}}%
\pgfpathlineto{\pgfqpoint{4.088065in}{2.874258in}}%
\pgfusepath{}%
\end{pgfscope}%
\begin{pgfscope}%
\pgfpathrectangle{\pgfqpoint{0.549740in}{0.463273in}}{\pgfqpoint{9.320225in}{4.495057in}}%
\pgfusepath{clip}%
\pgfsetbuttcap%
\pgfsetroundjoin%
\pgfsetlinewidth{0.000000pt}%
\definecolor{currentstroke}{rgb}{0.000000,0.000000,0.000000}%
\pgfsetstrokecolor{currentstroke}%
\pgfsetdash{}{0pt}%
\pgfpathmoveto{\pgfqpoint{4.274292in}{2.874258in}}%
\pgfpathlineto{\pgfqpoint{4.460519in}{2.874258in}}%
\pgfpathlineto{\pgfqpoint{4.460519in}{2.955987in}}%
\pgfpathlineto{\pgfqpoint{4.274292in}{2.955987in}}%
\pgfpathlineto{\pgfqpoint{4.274292in}{2.874258in}}%
\pgfusepath{}%
\end{pgfscope}%
\begin{pgfscope}%
\pgfpathrectangle{\pgfqpoint{0.549740in}{0.463273in}}{\pgfqpoint{9.320225in}{4.495057in}}%
\pgfusepath{clip}%
\pgfsetbuttcap%
\pgfsetroundjoin%
\pgfsetlinewidth{0.000000pt}%
\definecolor{currentstroke}{rgb}{0.000000,0.000000,0.000000}%
\pgfsetstrokecolor{currentstroke}%
\pgfsetdash{}{0pt}%
\pgfpathmoveto{\pgfqpoint{4.460519in}{2.874258in}}%
\pgfpathlineto{\pgfqpoint{4.646745in}{2.874258in}}%
\pgfpathlineto{\pgfqpoint{4.646745in}{2.955987in}}%
\pgfpathlineto{\pgfqpoint{4.460519in}{2.955987in}}%
\pgfpathlineto{\pgfqpoint{4.460519in}{2.874258in}}%
\pgfusepath{}%
\end{pgfscope}%
\begin{pgfscope}%
\pgfpathrectangle{\pgfqpoint{0.549740in}{0.463273in}}{\pgfqpoint{9.320225in}{4.495057in}}%
\pgfusepath{clip}%
\pgfsetbuttcap%
\pgfsetroundjoin%
\pgfsetlinewidth{0.000000pt}%
\definecolor{currentstroke}{rgb}{0.000000,0.000000,0.000000}%
\pgfsetstrokecolor{currentstroke}%
\pgfsetdash{}{0pt}%
\pgfpathmoveto{\pgfqpoint{4.646745in}{2.874258in}}%
\pgfpathlineto{\pgfqpoint{4.832972in}{2.874258in}}%
\pgfpathlineto{\pgfqpoint{4.832972in}{2.955987in}}%
\pgfpathlineto{\pgfqpoint{4.646745in}{2.955987in}}%
\pgfpathlineto{\pgfqpoint{4.646745in}{2.874258in}}%
\pgfusepath{}%
\end{pgfscope}%
\begin{pgfscope}%
\pgfpathrectangle{\pgfqpoint{0.549740in}{0.463273in}}{\pgfqpoint{9.320225in}{4.495057in}}%
\pgfusepath{clip}%
\pgfsetbuttcap%
\pgfsetroundjoin%
\pgfsetlinewidth{0.000000pt}%
\definecolor{currentstroke}{rgb}{0.000000,0.000000,0.000000}%
\pgfsetstrokecolor{currentstroke}%
\pgfsetdash{}{0pt}%
\pgfpathmoveto{\pgfqpoint{4.832972in}{2.874258in}}%
\pgfpathlineto{\pgfqpoint{5.019198in}{2.874258in}}%
\pgfpathlineto{\pgfqpoint{5.019198in}{2.955987in}}%
\pgfpathlineto{\pgfqpoint{4.832972in}{2.955987in}}%
\pgfpathlineto{\pgfqpoint{4.832972in}{2.874258in}}%
\pgfusepath{}%
\end{pgfscope}%
\begin{pgfscope}%
\pgfpathrectangle{\pgfqpoint{0.549740in}{0.463273in}}{\pgfqpoint{9.320225in}{4.495057in}}%
\pgfusepath{clip}%
\pgfsetbuttcap%
\pgfsetroundjoin%
\pgfsetlinewidth{0.000000pt}%
\definecolor{currentstroke}{rgb}{0.000000,0.000000,0.000000}%
\pgfsetstrokecolor{currentstroke}%
\pgfsetdash{}{0pt}%
\pgfpathmoveto{\pgfqpoint{5.019198in}{2.874258in}}%
\pgfpathlineto{\pgfqpoint{5.205425in}{2.874258in}}%
\pgfpathlineto{\pgfqpoint{5.205425in}{2.955987in}}%
\pgfpathlineto{\pgfqpoint{5.019198in}{2.955987in}}%
\pgfpathlineto{\pgfqpoint{5.019198in}{2.874258in}}%
\pgfusepath{}%
\end{pgfscope}%
\begin{pgfscope}%
\pgfpathrectangle{\pgfqpoint{0.549740in}{0.463273in}}{\pgfqpoint{9.320225in}{4.495057in}}%
\pgfusepath{clip}%
\pgfsetbuttcap%
\pgfsetroundjoin%
\definecolor{currentfill}{rgb}{0.472869,0.711325,0.955316}%
\pgfsetfillcolor{currentfill}%
\pgfsetlinewidth{0.000000pt}%
\definecolor{currentstroke}{rgb}{0.000000,0.000000,0.000000}%
\pgfsetstrokecolor{currentstroke}%
\pgfsetdash{}{0pt}%
\pgfpathmoveto{\pgfqpoint{5.205425in}{2.874258in}}%
\pgfpathlineto{\pgfqpoint{5.391651in}{2.874258in}}%
\pgfpathlineto{\pgfqpoint{5.391651in}{2.955987in}}%
\pgfpathlineto{\pgfqpoint{5.205425in}{2.955987in}}%
\pgfpathlineto{\pgfqpoint{5.205425in}{2.874258in}}%
\pgfusepath{fill}%
\end{pgfscope}%
\begin{pgfscope}%
\pgfpathrectangle{\pgfqpoint{0.549740in}{0.463273in}}{\pgfqpoint{9.320225in}{4.495057in}}%
\pgfusepath{clip}%
\pgfsetbuttcap%
\pgfsetroundjoin%
\pgfsetlinewidth{0.000000pt}%
\definecolor{currentstroke}{rgb}{0.000000,0.000000,0.000000}%
\pgfsetstrokecolor{currentstroke}%
\pgfsetdash{}{0pt}%
\pgfpathmoveto{\pgfqpoint{5.391651in}{2.874258in}}%
\pgfpathlineto{\pgfqpoint{5.577878in}{2.874258in}}%
\pgfpathlineto{\pgfqpoint{5.577878in}{2.955987in}}%
\pgfpathlineto{\pgfqpoint{5.391651in}{2.955987in}}%
\pgfpathlineto{\pgfqpoint{5.391651in}{2.874258in}}%
\pgfusepath{}%
\end{pgfscope}%
\begin{pgfscope}%
\pgfpathrectangle{\pgfqpoint{0.549740in}{0.463273in}}{\pgfqpoint{9.320225in}{4.495057in}}%
\pgfusepath{clip}%
\pgfsetbuttcap%
\pgfsetroundjoin%
\pgfsetlinewidth{0.000000pt}%
\definecolor{currentstroke}{rgb}{0.000000,0.000000,0.000000}%
\pgfsetstrokecolor{currentstroke}%
\pgfsetdash{}{0pt}%
\pgfpathmoveto{\pgfqpoint{5.577878in}{2.874258in}}%
\pgfpathlineto{\pgfqpoint{5.764104in}{2.874258in}}%
\pgfpathlineto{\pgfqpoint{5.764104in}{2.955987in}}%
\pgfpathlineto{\pgfqpoint{5.577878in}{2.955987in}}%
\pgfpathlineto{\pgfqpoint{5.577878in}{2.874258in}}%
\pgfusepath{}%
\end{pgfscope}%
\begin{pgfscope}%
\pgfpathrectangle{\pgfqpoint{0.549740in}{0.463273in}}{\pgfqpoint{9.320225in}{4.495057in}}%
\pgfusepath{clip}%
\pgfsetbuttcap%
\pgfsetroundjoin%
\pgfsetlinewidth{0.000000pt}%
\definecolor{currentstroke}{rgb}{0.000000,0.000000,0.000000}%
\pgfsetstrokecolor{currentstroke}%
\pgfsetdash{}{0pt}%
\pgfpathmoveto{\pgfqpoint{5.764104in}{2.874258in}}%
\pgfpathlineto{\pgfqpoint{5.950331in}{2.874258in}}%
\pgfpathlineto{\pgfqpoint{5.950331in}{2.955987in}}%
\pgfpathlineto{\pgfqpoint{5.764104in}{2.955987in}}%
\pgfpathlineto{\pgfqpoint{5.764104in}{2.874258in}}%
\pgfusepath{}%
\end{pgfscope}%
\begin{pgfscope}%
\pgfpathrectangle{\pgfqpoint{0.549740in}{0.463273in}}{\pgfqpoint{9.320225in}{4.495057in}}%
\pgfusepath{clip}%
\pgfsetbuttcap%
\pgfsetroundjoin%
\pgfsetlinewidth{0.000000pt}%
\definecolor{currentstroke}{rgb}{0.000000,0.000000,0.000000}%
\pgfsetstrokecolor{currentstroke}%
\pgfsetdash{}{0pt}%
\pgfpathmoveto{\pgfqpoint{5.950331in}{2.874258in}}%
\pgfpathlineto{\pgfqpoint{6.136557in}{2.874258in}}%
\pgfpathlineto{\pgfqpoint{6.136557in}{2.955987in}}%
\pgfpathlineto{\pgfqpoint{5.950331in}{2.955987in}}%
\pgfpathlineto{\pgfqpoint{5.950331in}{2.874258in}}%
\pgfusepath{}%
\end{pgfscope}%
\begin{pgfscope}%
\pgfpathrectangle{\pgfqpoint{0.549740in}{0.463273in}}{\pgfqpoint{9.320225in}{4.495057in}}%
\pgfusepath{clip}%
\pgfsetbuttcap%
\pgfsetroundjoin%
\definecolor{currentfill}{rgb}{0.472869,0.711325,0.955316}%
\pgfsetfillcolor{currentfill}%
\pgfsetlinewidth{0.000000pt}%
\definecolor{currentstroke}{rgb}{0.000000,0.000000,0.000000}%
\pgfsetstrokecolor{currentstroke}%
\pgfsetdash{}{0pt}%
\pgfpathmoveto{\pgfqpoint{6.136557in}{2.874258in}}%
\pgfpathlineto{\pgfqpoint{6.322784in}{2.874258in}}%
\pgfpathlineto{\pgfqpoint{6.322784in}{2.955987in}}%
\pgfpathlineto{\pgfqpoint{6.136557in}{2.955987in}}%
\pgfpathlineto{\pgfqpoint{6.136557in}{2.874258in}}%
\pgfusepath{fill}%
\end{pgfscope}%
\begin{pgfscope}%
\pgfpathrectangle{\pgfqpoint{0.549740in}{0.463273in}}{\pgfqpoint{9.320225in}{4.495057in}}%
\pgfusepath{clip}%
\pgfsetbuttcap%
\pgfsetroundjoin%
\pgfsetlinewidth{0.000000pt}%
\definecolor{currentstroke}{rgb}{0.000000,0.000000,0.000000}%
\pgfsetstrokecolor{currentstroke}%
\pgfsetdash{}{0pt}%
\pgfpathmoveto{\pgfqpoint{6.322784in}{2.874258in}}%
\pgfpathlineto{\pgfqpoint{6.509011in}{2.874258in}}%
\pgfpathlineto{\pgfqpoint{6.509011in}{2.955987in}}%
\pgfpathlineto{\pgfqpoint{6.322784in}{2.955987in}}%
\pgfpathlineto{\pgfqpoint{6.322784in}{2.874258in}}%
\pgfusepath{}%
\end{pgfscope}%
\begin{pgfscope}%
\pgfpathrectangle{\pgfqpoint{0.549740in}{0.463273in}}{\pgfqpoint{9.320225in}{4.495057in}}%
\pgfusepath{clip}%
\pgfsetbuttcap%
\pgfsetroundjoin%
\pgfsetlinewidth{0.000000pt}%
\definecolor{currentstroke}{rgb}{0.000000,0.000000,0.000000}%
\pgfsetstrokecolor{currentstroke}%
\pgfsetdash{}{0pt}%
\pgfpathmoveto{\pgfqpoint{6.509011in}{2.874258in}}%
\pgfpathlineto{\pgfqpoint{6.695237in}{2.874258in}}%
\pgfpathlineto{\pgfqpoint{6.695237in}{2.955987in}}%
\pgfpathlineto{\pgfqpoint{6.509011in}{2.955987in}}%
\pgfpathlineto{\pgfqpoint{6.509011in}{2.874258in}}%
\pgfusepath{}%
\end{pgfscope}%
\begin{pgfscope}%
\pgfpathrectangle{\pgfqpoint{0.549740in}{0.463273in}}{\pgfqpoint{9.320225in}{4.495057in}}%
\pgfusepath{clip}%
\pgfsetbuttcap%
\pgfsetroundjoin%
\pgfsetlinewidth{0.000000pt}%
\definecolor{currentstroke}{rgb}{0.000000,0.000000,0.000000}%
\pgfsetstrokecolor{currentstroke}%
\pgfsetdash{}{0pt}%
\pgfpathmoveto{\pgfqpoint{6.695237in}{2.874258in}}%
\pgfpathlineto{\pgfqpoint{6.881464in}{2.874258in}}%
\pgfpathlineto{\pgfqpoint{6.881464in}{2.955987in}}%
\pgfpathlineto{\pgfqpoint{6.695237in}{2.955987in}}%
\pgfpathlineto{\pgfqpoint{6.695237in}{2.874258in}}%
\pgfusepath{}%
\end{pgfscope}%
\begin{pgfscope}%
\pgfpathrectangle{\pgfqpoint{0.549740in}{0.463273in}}{\pgfqpoint{9.320225in}{4.495057in}}%
\pgfusepath{clip}%
\pgfsetbuttcap%
\pgfsetroundjoin%
\pgfsetlinewidth{0.000000pt}%
\definecolor{currentstroke}{rgb}{0.000000,0.000000,0.000000}%
\pgfsetstrokecolor{currentstroke}%
\pgfsetdash{}{0pt}%
\pgfpathmoveto{\pgfqpoint{6.881464in}{2.874258in}}%
\pgfpathlineto{\pgfqpoint{7.067690in}{2.874258in}}%
\pgfpathlineto{\pgfqpoint{7.067690in}{2.955987in}}%
\pgfpathlineto{\pgfqpoint{6.881464in}{2.955987in}}%
\pgfpathlineto{\pgfqpoint{6.881464in}{2.874258in}}%
\pgfusepath{}%
\end{pgfscope}%
\begin{pgfscope}%
\pgfpathrectangle{\pgfqpoint{0.549740in}{0.463273in}}{\pgfqpoint{9.320225in}{4.495057in}}%
\pgfusepath{clip}%
\pgfsetbuttcap%
\pgfsetroundjoin%
\definecolor{currentfill}{rgb}{0.472869,0.711325,0.955316}%
\pgfsetfillcolor{currentfill}%
\pgfsetlinewidth{0.000000pt}%
\definecolor{currentstroke}{rgb}{0.000000,0.000000,0.000000}%
\pgfsetstrokecolor{currentstroke}%
\pgfsetdash{}{0pt}%
\pgfpathmoveto{\pgfqpoint{7.067690in}{2.874258in}}%
\pgfpathlineto{\pgfqpoint{7.253917in}{2.874258in}}%
\pgfpathlineto{\pgfqpoint{7.253917in}{2.955987in}}%
\pgfpathlineto{\pgfqpoint{7.067690in}{2.955987in}}%
\pgfpathlineto{\pgfqpoint{7.067690in}{2.874258in}}%
\pgfusepath{fill}%
\end{pgfscope}%
\begin{pgfscope}%
\pgfpathrectangle{\pgfqpoint{0.549740in}{0.463273in}}{\pgfqpoint{9.320225in}{4.495057in}}%
\pgfusepath{clip}%
\pgfsetbuttcap%
\pgfsetroundjoin%
\pgfsetlinewidth{0.000000pt}%
\definecolor{currentstroke}{rgb}{0.000000,0.000000,0.000000}%
\pgfsetstrokecolor{currentstroke}%
\pgfsetdash{}{0pt}%
\pgfpathmoveto{\pgfqpoint{7.253917in}{2.874258in}}%
\pgfpathlineto{\pgfqpoint{7.440143in}{2.874258in}}%
\pgfpathlineto{\pgfqpoint{7.440143in}{2.955987in}}%
\pgfpathlineto{\pgfqpoint{7.253917in}{2.955987in}}%
\pgfpathlineto{\pgfqpoint{7.253917in}{2.874258in}}%
\pgfusepath{}%
\end{pgfscope}%
\begin{pgfscope}%
\pgfpathrectangle{\pgfqpoint{0.549740in}{0.463273in}}{\pgfqpoint{9.320225in}{4.495057in}}%
\pgfusepath{clip}%
\pgfsetbuttcap%
\pgfsetroundjoin%
\pgfsetlinewidth{0.000000pt}%
\definecolor{currentstroke}{rgb}{0.000000,0.000000,0.000000}%
\pgfsetstrokecolor{currentstroke}%
\pgfsetdash{}{0pt}%
\pgfpathmoveto{\pgfqpoint{7.440143in}{2.874258in}}%
\pgfpathlineto{\pgfqpoint{7.626370in}{2.874258in}}%
\pgfpathlineto{\pgfqpoint{7.626370in}{2.955987in}}%
\pgfpathlineto{\pgfqpoint{7.440143in}{2.955987in}}%
\pgfpathlineto{\pgfqpoint{7.440143in}{2.874258in}}%
\pgfusepath{}%
\end{pgfscope}%
\begin{pgfscope}%
\pgfpathrectangle{\pgfqpoint{0.549740in}{0.463273in}}{\pgfqpoint{9.320225in}{4.495057in}}%
\pgfusepath{clip}%
\pgfsetbuttcap%
\pgfsetroundjoin%
\pgfsetlinewidth{0.000000pt}%
\definecolor{currentstroke}{rgb}{0.000000,0.000000,0.000000}%
\pgfsetstrokecolor{currentstroke}%
\pgfsetdash{}{0pt}%
\pgfpathmoveto{\pgfqpoint{7.626370in}{2.874258in}}%
\pgfpathlineto{\pgfqpoint{7.812596in}{2.874258in}}%
\pgfpathlineto{\pgfqpoint{7.812596in}{2.955987in}}%
\pgfpathlineto{\pgfqpoint{7.626370in}{2.955987in}}%
\pgfpathlineto{\pgfqpoint{7.626370in}{2.874258in}}%
\pgfusepath{}%
\end{pgfscope}%
\begin{pgfscope}%
\pgfpathrectangle{\pgfqpoint{0.549740in}{0.463273in}}{\pgfqpoint{9.320225in}{4.495057in}}%
\pgfusepath{clip}%
\pgfsetbuttcap%
\pgfsetroundjoin%
\pgfsetlinewidth{0.000000pt}%
\definecolor{currentstroke}{rgb}{0.000000,0.000000,0.000000}%
\pgfsetstrokecolor{currentstroke}%
\pgfsetdash{}{0pt}%
\pgfpathmoveto{\pgfqpoint{7.812596in}{2.874258in}}%
\pgfpathlineto{\pgfqpoint{7.998823in}{2.874258in}}%
\pgfpathlineto{\pgfqpoint{7.998823in}{2.955987in}}%
\pgfpathlineto{\pgfqpoint{7.812596in}{2.955987in}}%
\pgfpathlineto{\pgfqpoint{7.812596in}{2.874258in}}%
\pgfusepath{}%
\end{pgfscope}%
\begin{pgfscope}%
\pgfpathrectangle{\pgfqpoint{0.549740in}{0.463273in}}{\pgfqpoint{9.320225in}{4.495057in}}%
\pgfusepath{clip}%
\pgfsetbuttcap%
\pgfsetroundjoin%
\pgfsetlinewidth{0.000000pt}%
\definecolor{currentstroke}{rgb}{0.000000,0.000000,0.000000}%
\pgfsetstrokecolor{currentstroke}%
\pgfsetdash{}{0pt}%
\pgfpathmoveto{\pgfqpoint{7.998823in}{2.874258in}}%
\pgfpathlineto{\pgfqpoint{8.185049in}{2.874258in}}%
\pgfpathlineto{\pgfqpoint{8.185049in}{2.955987in}}%
\pgfpathlineto{\pgfqpoint{7.998823in}{2.955987in}}%
\pgfpathlineto{\pgfqpoint{7.998823in}{2.874258in}}%
\pgfusepath{}%
\end{pgfscope}%
\begin{pgfscope}%
\pgfpathrectangle{\pgfqpoint{0.549740in}{0.463273in}}{\pgfqpoint{9.320225in}{4.495057in}}%
\pgfusepath{clip}%
\pgfsetbuttcap%
\pgfsetroundjoin%
\definecolor{currentfill}{rgb}{0.472869,0.711325,0.955316}%
\pgfsetfillcolor{currentfill}%
\pgfsetlinewidth{0.000000pt}%
\definecolor{currentstroke}{rgb}{0.000000,0.000000,0.000000}%
\pgfsetstrokecolor{currentstroke}%
\pgfsetdash{}{0pt}%
\pgfpathmoveto{\pgfqpoint{8.185049in}{2.874258in}}%
\pgfpathlineto{\pgfqpoint{8.371276in}{2.874258in}}%
\pgfpathlineto{\pgfqpoint{8.371276in}{2.955987in}}%
\pgfpathlineto{\pgfqpoint{8.185049in}{2.955987in}}%
\pgfpathlineto{\pgfqpoint{8.185049in}{2.874258in}}%
\pgfusepath{fill}%
\end{pgfscope}%
\begin{pgfscope}%
\pgfpathrectangle{\pgfqpoint{0.549740in}{0.463273in}}{\pgfqpoint{9.320225in}{4.495057in}}%
\pgfusepath{clip}%
\pgfsetbuttcap%
\pgfsetroundjoin%
\pgfsetlinewidth{0.000000pt}%
\definecolor{currentstroke}{rgb}{0.000000,0.000000,0.000000}%
\pgfsetstrokecolor{currentstroke}%
\pgfsetdash{}{0pt}%
\pgfpathmoveto{\pgfqpoint{8.371276in}{2.874258in}}%
\pgfpathlineto{\pgfqpoint{8.557503in}{2.874258in}}%
\pgfpathlineto{\pgfqpoint{8.557503in}{2.955987in}}%
\pgfpathlineto{\pgfqpoint{8.371276in}{2.955987in}}%
\pgfpathlineto{\pgfqpoint{8.371276in}{2.874258in}}%
\pgfusepath{}%
\end{pgfscope}%
\begin{pgfscope}%
\pgfpathrectangle{\pgfqpoint{0.549740in}{0.463273in}}{\pgfqpoint{9.320225in}{4.495057in}}%
\pgfusepath{clip}%
\pgfsetbuttcap%
\pgfsetroundjoin%
\pgfsetlinewidth{0.000000pt}%
\definecolor{currentstroke}{rgb}{0.000000,0.000000,0.000000}%
\pgfsetstrokecolor{currentstroke}%
\pgfsetdash{}{0pt}%
\pgfpathmoveto{\pgfqpoint{8.557503in}{2.874258in}}%
\pgfpathlineto{\pgfqpoint{8.743729in}{2.874258in}}%
\pgfpathlineto{\pgfqpoint{8.743729in}{2.955987in}}%
\pgfpathlineto{\pgfqpoint{8.557503in}{2.955987in}}%
\pgfpathlineto{\pgfqpoint{8.557503in}{2.874258in}}%
\pgfusepath{}%
\end{pgfscope}%
\begin{pgfscope}%
\pgfpathrectangle{\pgfqpoint{0.549740in}{0.463273in}}{\pgfqpoint{9.320225in}{4.495057in}}%
\pgfusepath{clip}%
\pgfsetbuttcap%
\pgfsetroundjoin%
\pgfsetlinewidth{0.000000pt}%
\definecolor{currentstroke}{rgb}{0.000000,0.000000,0.000000}%
\pgfsetstrokecolor{currentstroke}%
\pgfsetdash{}{0pt}%
\pgfpathmoveto{\pgfqpoint{8.743729in}{2.874258in}}%
\pgfpathlineto{\pgfqpoint{8.929956in}{2.874258in}}%
\pgfpathlineto{\pgfqpoint{8.929956in}{2.955987in}}%
\pgfpathlineto{\pgfqpoint{8.743729in}{2.955987in}}%
\pgfpathlineto{\pgfqpoint{8.743729in}{2.874258in}}%
\pgfusepath{}%
\end{pgfscope}%
\begin{pgfscope}%
\pgfpathrectangle{\pgfqpoint{0.549740in}{0.463273in}}{\pgfqpoint{9.320225in}{4.495057in}}%
\pgfusepath{clip}%
\pgfsetbuttcap%
\pgfsetroundjoin%
\pgfsetlinewidth{0.000000pt}%
\definecolor{currentstroke}{rgb}{0.000000,0.000000,0.000000}%
\pgfsetstrokecolor{currentstroke}%
\pgfsetdash{}{0pt}%
\pgfpathmoveto{\pgfqpoint{8.929956in}{2.874258in}}%
\pgfpathlineto{\pgfqpoint{9.116182in}{2.874258in}}%
\pgfpathlineto{\pgfqpoint{9.116182in}{2.955987in}}%
\pgfpathlineto{\pgfqpoint{8.929956in}{2.955987in}}%
\pgfpathlineto{\pgfqpoint{8.929956in}{2.874258in}}%
\pgfusepath{}%
\end{pgfscope}%
\begin{pgfscope}%
\pgfpathrectangle{\pgfqpoint{0.549740in}{0.463273in}}{\pgfqpoint{9.320225in}{4.495057in}}%
\pgfusepath{clip}%
\pgfsetbuttcap%
\pgfsetroundjoin%
\pgfsetlinewidth{0.000000pt}%
\definecolor{currentstroke}{rgb}{0.000000,0.000000,0.000000}%
\pgfsetstrokecolor{currentstroke}%
\pgfsetdash{}{0pt}%
\pgfpathmoveto{\pgfqpoint{9.116182in}{2.874258in}}%
\pgfpathlineto{\pgfqpoint{9.302409in}{2.874258in}}%
\pgfpathlineto{\pgfqpoint{9.302409in}{2.955987in}}%
\pgfpathlineto{\pgfqpoint{9.116182in}{2.955987in}}%
\pgfpathlineto{\pgfqpoint{9.116182in}{2.874258in}}%
\pgfusepath{}%
\end{pgfscope}%
\begin{pgfscope}%
\pgfpathrectangle{\pgfqpoint{0.549740in}{0.463273in}}{\pgfqpoint{9.320225in}{4.495057in}}%
\pgfusepath{clip}%
\pgfsetbuttcap%
\pgfsetroundjoin%
\pgfsetlinewidth{0.000000pt}%
\definecolor{currentstroke}{rgb}{0.000000,0.000000,0.000000}%
\pgfsetstrokecolor{currentstroke}%
\pgfsetdash{}{0pt}%
\pgfpathmoveto{\pgfqpoint{9.302409in}{2.874258in}}%
\pgfpathlineto{\pgfqpoint{9.488635in}{2.874258in}}%
\pgfpathlineto{\pgfqpoint{9.488635in}{2.955987in}}%
\pgfpathlineto{\pgfqpoint{9.302409in}{2.955987in}}%
\pgfpathlineto{\pgfqpoint{9.302409in}{2.874258in}}%
\pgfusepath{}%
\end{pgfscope}%
\begin{pgfscope}%
\pgfpathrectangle{\pgfqpoint{0.549740in}{0.463273in}}{\pgfqpoint{9.320225in}{4.495057in}}%
\pgfusepath{clip}%
\pgfsetbuttcap%
\pgfsetroundjoin%
\definecolor{currentfill}{rgb}{0.472869,0.711325,0.955316}%
\pgfsetfillcolor{currentfill}%
\pgfsetlinewidth{0.000000pt}%
\definecolor{currentstroke}{rgb}{0.000000,0.000000,0.000000}%
\pgfsetstrokecolor{currentstroke}%
\pgfsetdash{}{0pt}%
\pgfpathmoveto{\pgfqpoint{9.488635in}{2.874258in}}%
\pgfpathlineto{\pgfqpoint{9.674862in}{2.874258in}}%
\pgfpathlineto{\pgfqpoint{9.674862in}{2.955987in}}%
\pgfpathlineto{\pgfqpoint{9.488635in}{2.955987in}}%
\pgfpathlineto{\pgfqpoint{9.488635in}{2.874258in}}%
\pgfusepath{fill}%
\end{pgfscope}%
\begin{pgfscope}%
\pgfpathrectangle{\pgfqpoint{0.549740in}{0.463273in}}{\pgfqpoint{9.320225in}{4.495057in}}%
\pgfusepath{clip}%
\pgfsetbuttcap%
\pgfsetroundjoin%
\pgfsetlinewidth{0.000000pt}%
\definecolor{currentstroke}{rgb}{0.000000,0.000000,0.000000}%
\pgfsetstrokecolor{currentstroke}%
\pgfsetdash{}{0pt}%
\pgfpathmoveto{\pgfqpoint{9.674862in}{2.874258in}}%
\pgfpathlineto{\pgfqpoint{9.861088in}{2.874258in}}%
\pgfpathlineto{\pgfqpoint{9.861088in}{2.955987in}}%
\pgfpathlineto{\pgfqpoint{9.674862in}{2.955987in}}%
\pgfpathlineto{\pgfqpoint{9.674862in}{2.874258in}}%
\pgfusepath{}%
\end{pgfscope}%
\begin{pgfscope}%
\pgfpathrectangle{\pgfqpoint{0.549740in}{0.463273in}}{\pgfqpoint{9.320225in}{4.495057in}}%
\pgfusepath{clip}%
\pgfsetbuttcap%
\pgfsetroundjoin%
\pgfsetlinewidth{0.000000pt}%
\definecolor{currentstroke}{rgb}{0.000000,0.000000,0.000000}%
\pgfsetstrokecolor{currentstroke}%
\pgfsetdash{}{0pt}%
\pgfpathmoveto{\pgfqpoint{0.549761in}{2.955987in}}%
\pgfpathlineto{\pgfqpoint{0.735988in}{2.955987in}}%
\pgfpathlineto{\pgfqpoint{0.735988in}{3.037715in}}%
\pgfpathlineto{\pgfqpoint{0.549761in}{3.037715in}}%
\pgfpathlineto{\pgfqpoint{0.549761in}{2.955987in}}%
\pgfusepath{}%
\end{pgfscope}%
\begin{pgfscope}%
\pgfpathrectangle{\pgfqpoint{0.549740in}{0.463273in}}{\pgfqpoint{9.320225in}{4.495057in}}%
\pgfusepath{clip}%
\pgfsetbuttcap%
\pgfsetroundjoin%
\pgfsetlinewidth{0.000000pt}%
\definecolor{currentstroke}{rgb}{0.000000,0.000000,0.000000}%
\pgfsetstrokecolor{currentstroke}%
\pgfsetdash{}{0pt}%
\pgfpathmoveto{\pgfqpoint{0.735988in}{2.955987in}}%
\pgfpathlineto{\pgfqpoint{0.922214in}{2.955987in}}%
\pgfpathlineto{\pgfqpoint{0.922214in}{3.037715in}}%
\pgfpathlineto{\pgfqpoint{0.735988in}{3.037715in}}%
\pgfpathlineto{\pgfqpoint{0.735988in}{2.955987in}}%
\pgfusepath{}%
\end{pgfscope}%
\begin{pgfscope}%
\pgfpathrectangle{\pgfqpoint{0.549740in}{0.463273in}}{\pgfqpoint{9.320225in}{4.495057in}}%
\pgfusepath{clip}%
\pgfsetbuttcap%
\pgfsetroundjoin%
\pgfsetlinewidth{0.000000pt}%
\definecolor{currentstroke}{rgb}{0.000000,0.000000,0.000000}%
\pgfsetstrokecolor{currentstroke}%
\pgfsetdash{}{0pt}%
\pgfpathmoveto{\pgfqpoint{0.922214in}{2.955987in}}%
\pgfpathlineto{\pgfqpoint{1.108441in}{2.955987in}}%
\pgfpathlineto{\pgfqpoint{1.108441in}{3.037715in}}%
\pgfpathlineto{\pgfqpoint{0.922214in}{3.037715in}}%
\pgfpathlineto{\pgfqpoint{0.922214in}{2.955987in}}%
\pgfusepath{}%
\end{pgfscope}%
\begin{pgfscope}%
\pgfpathrectangle{\pgfqpoint{0.549740in}{0.463273in}}{\pgfqpoint{9.320225in}{4.495057in}}%
\pgfusepath{clip}%
\pgfsetbuttcap%
\pgfsetroundjoin%
\pgfsetlinewidth{0.000000pt}%
\definecolor{currentstroke}{rgb}{0.000000,0.000000,0.000000}%
\pgfsetstrokecolor{currentstroke}%
\pgfsetdash{}{0pt}%
\pgfpathmoveto{\pgfqpoint{1.108441in}{2.955987in}}%
\pgfpathlineto{\pgfqpoint{1.294667in}{2.955987in}}%
\pgfpathlineto{\pgfqpoint{1.294667in}{3.037715in}}%
\pgfpathlineto{\pgfqpoint{1.108441in}{3.037715in}}%
\pgfpathlineto{\pgfqpoint{1.108441in}{2.955987in}}%
\pgfusepath{}%
\end{pgfscope}%
\begin{pgfscope}%
\pgfpathrectangle{\pgfqpoint{0.549740in}{0.463273in}}{\pgfqpoint{9.320225in}{4.495057in}}%
\pgfusepath{clip}%
\pgfsetbuttcap%
\pgfsetroundjoin%
\pgfsetlinewidth{0.000000pt}%
\definecolor{currentstroke}{rgb}{0.000000,0.000000,0.000000}%
\pgfsetstrokecolor{currentstroke}%
\pgfsetdash{}{0pt}%
\pgfpathmoveto{\pgfqpoint{1.294667in}{2.955987in}}%
\pgfpathlineto{\pgfqpoint{1.480894in}{2.955987in}}%
\pgfpathlineto{\pgfqpoint{1.480894in}{3.037715in}}%
\pgfpathlineto{\pgfqpoint{1.294667in}{3.037715in}}%
\pgfpathlineto{\pgfqpoint{1.294667in}{2.955987in}}%
\pgfusepath{}%
\end{pgfscope}%
\begin{pgfscope}%
\pgfpathrectangle{\pgfqpoint{0.549740in}{0.463273in}}{\pgfqpoint{9.320225in}{4.495057in}}%
\pgfusepath{clip}%
\pgfsetbuttcap%
\pgfsetroundjoin%
\pgfsetlinewidth{0.000000pt}%
\definecolor{currentstroke}{rgb}{0.000000,0.000000,0.000000}%
\pgfsetstrokecolor{currentstroke}%
\pgfsetdash{}{0pt}%
\pgfpathmoveto{\pgfqpoint{1.480894in}{2.955987in}}%
\pgfpathlineto{\pgfqpoint{1.667120in}{2.955987in}}%
\pgfpathlineto{\pgfqpoint{1.667120in}{3.037715in}}%
\pgfpathlineto{\pgfqpoint{1.480894in}{3.037715in}}%
\pgfpathlineto{\pgfqpoint{1.480894in}{2.955987in}}%
\pgfusepath{}%
\end{pgfscope}%
\begin{pgfscope}%
\pgfpathrectangle{\pgfqpoint{0.549740in}{0.463273in}}{\pgfqpoint{9.320225in}{4.495057in}}%
\pgfusepath{clip}%
\pgfsetbuttcap%
\pgfsetroundjoin%
\pgfsetlinewidth{0.000000pt}%
\definecolor{currentstroke}{rgb}{0.000000,0.000000,0.000000}%
\pgfsetstrokecolor{currentstroke}%
\pgfsetdash{}{0pt}%
\pgfpathmoveto{\pgfqpoint{1.667120in}{2.955987in}}%
\pgfpathlineto{\pgfqpoint{1.853347in}{2.955987in}}%
\pgfpathlineto{\pgfqpoint{1.853347in}{3.037715in}}%
\pgfpathlineto{\pgfqpoint{1.667120in}{3.037715in}}%
\pgfpathlineto{\pgfqpoint{1.667120in}{2.955987in}}%
\pgfusepath{}%
\end{pgfscope}%
\begin{pgfscope}%
\pgfpathrectangle{\pgfqpoint{0.549740in}{0.463273in}}{\pgfqpoint{9.320225in}{4.495057in}}%
\pgfusepath{clip}%
\pgfsetbuttcap%
\pgfsetroundjoin%
\pgfsetlinewidth{0.000000pt}%
\definecolor{currentstroke}{rgb}{0.000000,0.000000,0.000000}%
\pgfsetstrokecolor{currentstroke}%
\pgfsetdash{}{0pt}%
\pgfpathmoveto{\pgfqpoint{1.853347in}{2.955987in}}%
\pgfpathlineto{\pgfqpoint{2.039573in}{2.955987in}}%
\pgfpathlineto{\pgfqpoint{2.039573in}{3.037715in}}%
\pgfpathlineto{\pgfqpoint{1.853347in}{3.037715in}}%
\pgfpathlineto{\pgfqpoint{1.853347in}{2.955987in}}%
\pgfusepath{}%
\end{pgfscope}%
\begin{pgfscope}%
\pgfpathrectangle{\pgfqpoint{0.549740in}{0.463273in}}{\pgfqpoint{9.320225in}{4.495057in}}%
\pgfusepath{clip}%
\pgfsetbuttcap%
\pgfsetroundjoin%
\pgfsetlinewidth{0.000000pt}%
\definecolor{currentstroke}{rgb}{0.000000,0.000000,0.000000}%
\pgfsetstrokecolor{currentstroke}%
\pgfsetdash{}{0pt}%
\pgfpathmoveto{\pgfqpoint{2.039573in}{2.955987in}}%
\pgfpathlineto{\pgfqpoint{2.225800in}{2.955987in}}%
\pgfpathlineto{\pgfqpoint{2.225800in}{3.037715in}}%
\pgfpathlineto{\pgfqpoint{2.039573in}{3.037715in}}%
\pgfpathlineto{\pgfqpoint{2.039573in}{2.955987in}}%
\pgfusepath{}%
\end{pgfscope}%
\begin{pgfscope}%
\pgfpathrectangle{\pgfqpoint{0.549740in}{0.463273in}}{\pgfqpoint{9.320225in}{4.495057in}}%
\pgfusepath{clip}%
\pgfsetbuttcap%
\pgfsetroundjoin%
\pgfsetlinewidth{0.000000pt}%
\definecolor{currentstroke}{rgb}{0.000000,0.000000,0.000000}%
\pgfsetstrokecolor{currentstroke}%
\pgfsetdash{}{0pt}%
\pgfpathmoveto{\pgfqpoint{2.225800in}{2.955987in}}%
\pgfpathlineto{\pgfqpoint{2.412027in}{2.955987in}}%
\pgfpathlineto{\pgfqpoint{2.412027in}{3.037715in}}%
\pgfpathlineto{\pgfqpoint{2.225800in}{3.037715in}}%
\pgfpathlineto{\pgfqpoint{2.225800in}{2.955987in}}%
\pgfusepath{}%
\end{pgfscope}%
\begin{pgfscope}%
\pgfpathrectangle{\pgfqpoint{0.549740in}{0.463273in}}{\pgfqpoint{9.320225in}{4.495057in}}%
\pgfusepath{clip}%
\pgfsetbuttcap%
\pgfsetroundjoin%
\pgfsetlinewidth{0.000000pt}%
\definecolor{currentstroke}{rgb}{0.000000,0.000000,0.000000}%
\pgfsetstrokecolor{currentstroke}%
\pgfsetdash{}{0pt}%
\pgfpathmoveto{\pgfqpoint{2.412027in}{2.955987in}}%
\pgfpathlineto{\pgfqpoint{2.598253in}{2.955987in}}%
\pgfpathlineto{\pgfqpoint{2.598253in}{3.037715in}}%
\pgfpathlineto{\pgfqpoint{2.412027in}{3.037715in}}%
\pgfpathlineto{\pgfqpoint{2.412027in}{2.955987in}}%
\pgfusepath{}%
\end{pgfscope}%
\begin{pgfscope}%
\pgfpathrectangle{\pgfqpoint{0.549740in}{0.463273in}}{\pgfqpoint{9.320225in}{4.495057in}}%
\pgfusepath{clip}%
\pgfsetbuttcap%
\pgfsetroundjoin%
\pgfsetlinewidth{0.000000pt}%
\definecolor{currentstroke}{rgb}{0.000000,0.000000,0.000000}%
\pgfsetstrokecolor{currentstroke}%
\pgfsetdash{}{0pt}%
\pgfpathmoveto{\pgfqpoint{2.598253in}{2.955987in}}%
\pgfpathlineto{\pgfqpoint{2.784480in}{2.955987in}}%
\pgfpathlineto{\pgfqpoint{2.784480in}{3.037715in}}%
\pgfpathlineto{\pgfqpoint{2.598253in}{3.037715in}}%
\pgfpathlineto{\pgfqpoint{2.598253in}{2.955987in}}%
\pgfusepath{}%
\end{pgfscope}%
\begin{pgfscope}%
\pgfpathrectangle{\pgfqpoint{0.549740in}{0.463273in}}{\pgfqpoint{9.320225in}{4.495057in}}%
\pgfusepath{clip}%
\pgfsetbuttcap%
\pgfsetroundjoin%
\pgfsetlinewidth{0.000000pt}%
\definecolor{currentstroke}{rgb}{0.000000,0.000000,0.000000}%
\pgfsetstrokecolor{currentstroke}%
\pgfsetdash{}{0pt}%
\pgfpathmoveto{\pgfqpoint{2.784480in}{2.955987in}}%
\pgfpathlineto{\pgfqpoint{2.970706in}{2.955987in}}%
\pgfpathlineto{\pgfqpoint{2.970706in}{3.037715in}}%
\pgfpathlineto{\pgfqpoint{2.784480in}{3.037715in}}%
\pgfpathlineto{\pgfqpoint{2.784480in}{2.955987in}}%
\pgfusepath{}%
\end{pgfscope}%
\begin{pgfscope}%
\pgfpathrectangle{\pgfqpoint{0.549740in}{0.463273in}}{\pgfqpoint{9.320225in}{4.495057in}}%
\pgfusepath{clip}%
\pgfsetbuttcap%
\pgfsetroundjoin%
\pgfsetlinewidth{0.000000pt}%
\definecolor{currentstroke}{rgb}{0.000000,0.000000,0.000000}%
\pgfsetstrokecolor{currentstroke}%
\pgfsetdash{}{0pt}%
\pgfpathmoveto{\pgfqpoint{2.970706in}{2.955987in}}%
\pgfpathlineto{\pgfqpoint{3.156933in}{2.955987in}}%
\pgfpathlineto{\pgfqpoint{3.156933in}{3.037715in}}%
\pgfpathlineto{\pgfqpoint{2.970706in}{3.037715in}}%
\pgfpathlineto{\pgfqpoint{2.970706in}{2.955987in}}%
\pgfusepath{}%
\end{pgfscope}%
\begin{pgfscope}%
\pgfpathrectangle{\pgfqpoint{0.549740in}{0.463273in}}{\pgfqpoint{9.320225in}{4.495057in}}%
\pgfusepath{clip}%
\pgfsetbuttcap%
\pgfsetroundjoin%
\pgfsetlinewidth{0.000000pt}%
\definecolor{currentstroke}{rgb}{0.000000,0.000000,0.000000}%
\pgfsetstrokecolor{currentstroke}%
\pgfsetdash{}{0pt}%
\pgfpathmoveto{\pgfqpoint{3.156933in}{2.955987in}}%
\pgfpathlineto{\pgfqpoint{3.343159in}{2.955987in}}%
\pgfpathlineto{\pgfqpoint{3.343159in}{3.037715in}}%
\pgfpathlineto{\pgfqpoint{3.156933in}{3.037715in}}%
\pgfpathlineto{\pgfqpoint{3.156933in}{2.955987in}}%
\pgfusepath{}%
\end{pgfscope}%
\begin{pgfscope}%
\pgfpathrectangle{\pgfqpoint{0.549740in}{0.463273in}}{\pgfqpoint{9.320225in}{4.495057in}}%
\pgfusepath{clip}%
\pgfsetbuttcap%
\pgfsetroundjoin%
\pgfsetlinewidth{0.000000pt}%
\definecolor{currentstroke}{rgb}{0.000000,0.000000,0.000000}%
\pgfsetstrokecolor{currentstroke}%
\pgfsetdash{}{0pt}%
\pgfpathmoveto{\pgfqpoint{3.343159in}{2.955987in}}%
\pgfpathlineto{\pgfqpoint{3.529386in}{2.955987in}}%
\pgfpathlineto{\pgfqpoint{3.529386in}{3.037715in}}%
\pgfpathlineto{\pgfqpoint{3.343159in}{3.037715in}}%
\pgfpathlineto{\pgfqpoint{3.343159in}{2.955987in}}%
\pgfusepath{}%
\end{pgfscope}%
\begin{pgfscope}%
\pgfpathrectangle{\pgfqpoint{0.549740in}{0.463273in}}{\pgfqpoint{9.320225in}{4.495057in}}%
\pgfusepath{clip}%
\pgfsetbuttcap%
\pgfsetroundjoin%
\pgfsetlinewidth{0.000000pt}%
\definecolor{currentstroke}{rgb}{0.000000,0.000000,0.000000}%
\pgfsetstrokecolor{currentstroke}%
\pgfsetdash{}{0pt}%
\pgfpathmoveto{\pgfqpoint{3.529386in}{2.955987in}}%
\pgfpathlineto{\pgfqpoint{3.715612in}{2.955987in}}%
\pgfpathlineto{\pgfqpoint{3.715612in}{3.037715in}}%
\pgfpathlineto{\pgfqpoint{3.529386in}{3.037715in}}%
\pgfpathlineto{\pgfqpoint{3.529386in}{2.955987in}}%
\pgfusepath{}%
\end{pgfscope}%
\begin{pgfscope}%
\pgfpathrectangle{\pgfqpoint{0.549740in}{0.463273in}}{\pgfqpoint{9.320225in}{4.495057in}}%
\pgfusepath{clip}%
\pgfsetbuttcap%
\pgfsetroundjoin%
\pgfsetlinewidth{0.000000pt}%
\definecolor{currentstroke}{rgb}{0.000000,0.000000,0.000000}%
\pgfsetstrokecolor{currentstroke}%
\pgfsetdash{}{0pt}%
\pgfpathmoveto{\pgfqpoint{3.715612in}{2.955987in}}%
\pgfpathlineto{\pgfqpoint{3.901839in}{2.955987in}}%
\pgfpathlineto{\pgfqpoint{3.901839in}{3.037715in}}%
\pgfpathlineto{\pgfqpoint{3.715612in}{3.037715in}}%
\pgfpathlineto{\pgfqpoint{3.715612in}{2.955987in}}%
\pgfusepath{}%
\end{pgfscope}%
\begin{pgfscope}%
\pgfpathrectangle{\pgfqpoint{0.549740in}{0.463273in}}{\pgfqpoint{9.320225in}{4.495057in}}%
\pgfusepath{clip}%
\pgfsetbuttcap%
\pgfsetroundjoin%
\pgfsetlinewidth{0.000000pt}%
\definecolor{currentstroke}{rgb}{0.000000,0.000000,0.000000}%
\pgfsetstrokecolor{currentstroke}%
\pgfsetdash{}{0pt}%
\pgfpathmoveto{\pgfqpoint{3.901839in}{2.955987in}}%
\pgfpathlineto{\pgfqpoint{4.088065in}{2.955987in}}%
\pgfpathlineto{\pgfqpoint{4.088065in}{3.037715in}}%
\pgfpathlineto{\pgfqpoint{3.901839in}{3.037715in}}%
\pgfpathlineto{\pgfqpoint{3.901839in}{2.955987in}}%
\pgfusepath{}%
\end{pgfscope}%
\begin{pgfscope}%
\pgfpathrectangle{\pgfqpoint{0.549740in}{0.463273in}}{\pgfqpoint{9.320225in}{4.495057in}}%
\pgfusepath{clip}%
\pgfsetbuttcap%
\pgfsetroundjoin%
\pgfsetlinewidth{0.000000pt}%
\definecolor{currentstroke}{rgb}{0.000000,0.000000,0.000000}%
\pgfsetstrokecolor{currentstroke}%
\pgfsetdash{}{0pt}%
\pgfpathmoveto{\pgfqpoint{4.088065in}{2.955987in}}%
\pgfpathlineto{\pgfqpoint{4.274292in}{2.955987in}}%
\pgfpathlineto{\pgfqpoint{4.274292in}{3.037715in}}%
\pgfpathlineto{\pgfqpoint{4.088065in}{3.037715in}}%
\pgfpathlineto{\pgfqpoint{4.088065in}{2.955987in}}%
\pgfusepath{}%
\end{pgfscope}%
\begin{pgfscope}%
\pgfpathrectangle{\pgfqpoint{0.549740in}{0.463273in}}{\pgfqpoint{9.320225in}{4.495057in}}%
\pgfusepath{clip}%
\pgfsetbuttcap%
\pgfsetroundjoin%
\pgfsetlinewidth{0.000000pt}%
\definecolor{currentstroke}{rgb}{0.000000,0.000000,0.000000}%
\pgfsetstrokecolor{currentstroke}%
\pgfsetdash{}{0pt}%
\pgfpathmoveto{\pgfqpoint{4.274292in}{2.955987in}}%
\pgfpathlineto{\pgfqpoint{4.460519in}{2.955987in}}%
\pgfpathlineto{\pgfqpoint{4.460519in}{3.037715in}}%
\pgfpathlineto{\pgfqpoint{4.274292in}{3.037715in}}%
\pgfpathlineto{\pgfqpoint{4.274292in}{2.955987in}}%
\pgfusepath{}%
\end{pgfscope}%
\begin{pgfscope}%
\pgfpathrectangle{\pgfqpoint{0.549740in}{0.463273in}}{\pgfqpoint{9.320225in}{4.495057in}}%
\pgfusepath{clip}%
\pgfsetbuttcap%
\pgfsetroundjoin%
\pgfsetlinewidth{0.000000pt}%
\definecolor{currentstroke}{rgb}{0.000000,0.000000,0.000000}%
\pgfsetstrokecolor{currentstroke}%
\pgfsetdash{}{0pt}%
\pgfpathmoveto{\pgfqpoint{4.460519in}{2.955987in}}%
\pgfpathlineto{\pgfqpoint{4.646745in}{2.955987in}}%
\pgfpathlineto{\pgfqpoint{4.646745in}{3.037715in}}%
\pgfpathlineto{\pgfqpoint{4.460519in}{3.037715in}}%
\pgfpathlineto{\pgfqpoint{4.460519in}{2.955987in}}%
\pgfusepath{}%
\end{pgfscope}%
\begin{pgfscope}%
\pgfpathrectangle{\pgfqpoint{0.549740in}{0.463273in}}{\pgfqpoint{9.320225in}{4.495057in}}%
\pgfusepath{clip}%
\pgfsetbuttcap%
\pgfsetroundjoin%
\pgfsetlinewidth{0.000000pt}%
\definecolor{currentstroke}{rgb}{0.000000,0.000000,0.000000}%
\pgfsetstrokecolor{currentstroke}%
\pgfsetdash{}{0pt}%
\pgfpathmoveto{\pgfqpoint{4.646745in}{2.955987in}}%
\pgfpathlineto{\pgfqpoint{4.832972in}{2.955987in}}%
\pgfpathlineto{\pgfqpoint{4.832972in}{3.037715in}}%
\pgfpathlineto{\pgfqpoint{4.646745in}{3.037715in}}%
\pgfpathlineto{\pgfqpoint{4.646745in}{2.955987in}}%
\pgfusepath{}%
\end{pgfscope}%
\begin{pgfscope}%
\pgfpathrectangle{\pgfqpoint{0.549740in}{0.463273in}}{\pgfqpoint{9.320225in}{4.495057in}}%
\pgfusepath{clip}%
\pgfsetbuttcap%
\pgfsetroundjoin%
\pgfsetlinewidth{0.000000pt}%
\definecolor{currentstroke}{rgb}{0.000000,0.000000,0.000000}%
\pgfsetstrokecolor{currentstroke}%
\pgfsetdash{}{0pt}%
\pgfpathmoveto{\pgfqpoint{4.832972in}{2.955987in}}%
\pgfpathlineto{\pgfqpoint{5.019198in}{2.955987in}}%
\pgfpathlineto{\pgfqpoint{5.019198in}{3.037715in}}%
\pgfpathlineto{\pgfqpoint{4.832972in}{3.037715in}}%
\pgfpathlineto{\pgfqpoint{4.832972in}{2.955987in}}%
\pgfusepath{}%
\end{pgfscope}%
\begin{pgfscope}%
\pgfpathrectangle{\pgfqpoint{0.549740in}{0.463273in}}{\pgfqpoint{9.320225in}{4.495057in}}%
\pgfusepath{clip}%
\pgfsetbuttcap%
\pgfsetroundjoin%
\pgfsetlinewidth{0.000000pt}%
\definecolor{currentstroke}{rgb}{0.000000,0.000000,0.000000}%
\pgfsetstrokecolor{currentstroke}%
\pgfsetdash{}{0pt}%
\pgfpathmoveto{\pgfqpoint{5.019198in}{2.955987in}}%
\pgfpathlineto{\pgfqpoint{5.205425in}{2.955987in}}%
\pgfpathlineto{\pgfqpoint{5.205425in}{3.037715in}}%
\pgfpathlineto{\pgfqpoint{5.019198in}{3.037715in}}%
\pgfpathlineto{\pgfqpoint{5.019198in}{2.955987in}}%
\pgfusepath{}%
\end{pgfscope}%
\begin{pgfscope}%
\pgfpathrectangle{\pgfqpoint{0.549740in}{0.463273in}}{\pgfqpoint{9.320225in}{4.495057in}}%
\pgfusepath{clip}%
\pgfsetbuttcap%
\pgfsetroundjoin%
\definecolor{currentfill}{rgb}{0.547810,0.736432,0.947518}%
\pgfsetfillcolor{currentfill}%
\pgfsetlinewidth{0.000000pt}%
\definecolor{currentstroke}{rgb}{0.000000,0.000000,0.000000}%
\pgfsetstrokecolor{currentstroke}%
\pgfsetdash{}{0pt}%
\pgfpathmoveto{\pgfqpoint{5.205425in}{2.955987in}}%
\pgfpathlineto{\pgfqpoint{5.391651in}{2.955987in}}%
\pgfpathlineto{\pgfqpoint{5.391651in}{3.037715in}}%
\pgfpathlineto{\pgfqpoint{5.205425in}{3.037715in}}%
\pgfpathlineto{\pgfqpoint{5.205425in}{2.955987in}}%
\pgfusepath{fill}%
\end{pgfscope}%
\begin{pgfscope}%
\pgfpathrectangle{\pgfqpoint{0.549740in}{0.463273in}}{\pgfqpoint{9.320225in}{4.495057in}}%
\pgfusepath{clip}%
\pgfsetbuttcap%
\pgfsetroundjoin%
\pgfsetlinewidth{0.000000pt}%
\definecolor{currentstroke}{rgb}{0.000000,0.000000,0.000000}%
\pgfsetstrokecolor{currentstroke}%
\pgfsetdash{}{0pt}%
\pgfpathmoveto{\pgfqpoint{5.391651in}{2.955987in}}%
\pgfpathlineto{\pgfqpoint{5.577878in}{2.955987in}}%
\pgfpathlineto{\pgfqpoint{5.577878in}{3.037715in}}%
\pgfpathlineto{\pgfqpoint{5.391651in}{3.037715in}}%
\pgfpathlineto{\pgfqpoint{5.391651in}{2.955987in}}%
\pgfusepath{}%
\end{pgfscope}%
\begin{pgfscope}%
\pgfpathrectangle{\pgfqpoint{0.549740in}{0.463273in}}{\pgfqpoint{9.320225in}{4.495057in}}%
\pgfusepath{clip}%
\pgfsetbuttcap%
\pgfsetroundjoin%
\pgfsetlinewidth{0.000000pt}%
\definecolor{currentstroke}{rgb}{0.000000,0.000000,0.000000}%
\pgfsetstrokecolor{currentstroke}%
\pgfsetdash{}{0pt}%
\pgfpathmoveto{\pgfqpoint{5.577878in}{2.955987in}}%
\pgfpathlineto{\pgfqpoint{5.764104in}{2.955987in}}%
\pgfpathlineto{\pgfqpoint{5.764104in}{3.037715in}}%
\pgfpathlineto{\pgfqpoint{5.577878in}{3.037715in}}%
\pgfpathlineto{\pgfqpoint{5.577878in}{2.955987in}}%
\pgfusepath{}%
\end{pgfscope}%
\begin{pgfscope}%
\pgfpathrectangle{\pgfqpoint{0.549740in}{0.463273in}}{\pgfqpoint{9.320225in}{4.495057in}}%
\pgfusepath{clip}%
\pgfsetbuttcap%
\pgfsetroundjoin%
\pgfsetlinewidth{0.000000pt}%
\definecolor{currentstroke}{rgb}{0.000000,0.000000,0.000000}%
\pgfsetstrokecolor{currentstroke}%
\pgfsetdash{}{0pt}%
\pgfpathmoveto{\pgfqpoint{5.764104in}{2.955987in}}%
\pgfpathlineto{\pgfqpoint{5.950331in}{2.955987in}}%
\pgfpathlineto{\pgfqpoint{5.950331in}{3.037715in}}%
\pgfpathlineto{\pgfqpoint{5.764104in}{3.037715in}}%
\pgfpathlineto{\pgfqpoint{5.764104in}{2.955987in}}%
\pgfusepath{}%
\end{pgfscope}%
\begin{pgfscope}%
\pgfpathrectangle{\pgfqpoint{0.549740in}{0.463273in}}{\pgfqpoint{9.320225in}{4.495057in}}%
\pgfusepath{clip}%
\pgfsetbuttcap%
\pgfsetroundjoin%
\pgfsetlinewidth{0.000000pt}%
\definecolor{currentstroke}{rgb}{0.000000,0.000000,0.000000}%
\pgfsetstrokecolor{currentstroke}%
\pgfsetdash{}{0pt}%
\pgfpathmoveto{\pgfqpoint{5.950331in}{2.955987in}}%
\pgfpathlineto{\pgfqpoint{6.136557in}{2.955987in}}%
\pgfpathlineto{\pgfqpoint{6.136557in}{3.037715in}}%
\pgfpathlineto{\pgfqpoint{5.950331in}{3.037715in}}%
\pgfpathlineto{\pgfqpoint{5.950331in}{2.955987in}}%
\pgfusepath{}%
\end{pgfscope}%
\begin{pgfscope}%
\pgfpathrectangle{\pgfqpoint{0.549740in}{0.463273in}}{\pgfqpoint{9.320225in}{4.495057in}}%
\pgfusepath{clip}%
\pgfsetbuttcap%
\pgfsetroundjoin%
\definecolor{currentfill}{rgb}{0.472869,0.711325,0.955316}%
\pgfsetfillcolor{currentfill}%
\pgfsetlinewidth{0.000000pt}%
\definecolor{currentstroke}{rgb}{0.000000,0.000000,0.000000}%
\pgfsetstrokecolor{currentstroke}%
\pgfsetdash{}{0pt}%
\pgfpathmoveto{\pgfqpoint{6.136557in}{2.955987in}}%
\pgfpathlineto{\pgfqpoint{6.322784in}{2.955987in}}%
\pgfpathlineto{\pgfqpoint{6.322784in}{3.037715in}}%
\pgfpathlineto{\pgfqpoint{6.136557in}{3.037715in}}%
\pgfpathlineto{\pgfqpoint{6.136557in}{2.955987in}}%
\pgfusepath{fill}%
\end{pgfscope}%
\begin{pgfscope}%
\pgfpathrectangle{\pgfqpoint{0.549740in}{0.463273in}}{\pgfqpoint{9.320225in}{4.495057in}}%
\pgfusepath{clip}%
\pgfsetbuttcap%
\pgfsetroundjoin%
\pgfsetlinewidth{0.000000pt}%
\definecolor{currentstroke}{rgb}{0.000000,0.000000,0.000000}%
\pgfsetstrokecolor{currentstroke}%
\pgfsetdash{}{0pt}%
\pgfpathmoveto{\pgfqpoint{6.322784in}{2.955987in}}%
\pgfpathlineto{\pgfqpoint{6.509011in}{2.955987in}}%
\pgfpathlineto{\pgfqpoint{6.509011in}{3.037715in}}%
\pgfpathlineto{\pgfqpoint{6.322784in}{3.037715in}}%
\pgfpathlineto{\pgfqpoint{6.322784in}{2.955987in}}%
\pgfusepath{}%
\end{pgfscope}%
\begin{pgfscope}%
\pgfpathrectangle{\pgfqpoint{0.549740in}{0.463273in}}{\pgfqpoint{9.320225in}{4.495057in}}%
\pgfusepath{clip}%
\pgfsetbuttcap%
\pgfsetroundjoin%
\pgfsetlinewidth{0.000000pt}%
\definecolor{currentstroke}{rgb}{0.000000,0.000000,0.000000}%
\pgfsetstrokecolor{currentstroke}%
\pgfsetdash{}{0pt}%
\pgfpathmoveto{\pgfqpoint{6.509011in}{2.955987in}}%
\pgfpathlineto{\pgfqpoint{6.695237in}{2.955987in}}%
\pgfpathlineto{\pgfqpoint{6.695237in}{3.037715in}}%
\pgfpathlineto{\pgfqpoint{6.509011in}{3.037715in}}%
\pgfpathlineto{\pgfqpoint{6.509011in}{2.955987in}}%
\pgfusepath{}%
\end{pgfscope}%
\begin{pgfscope}%
\pgfpathrectangle{\pgfqpoint{0.549740in}{0.463273in}}{\pgfqpoint{9.320225in}{4.495057in}}%
\pgfusepath{clip}%
\pgfsetbuttcap%
\pgfsetroundjoin%
\pgfsetlinewidth{0.000000pt}%
\definecolor{currentstroke}{rgb}{0.000000,0.000000,0.000000}%
\pgfsetstrokecolor{currentstroke}%
\pgfsetdash{}{0pt}%
\pgfpathmoveto{\pgfqpoint{6.695237in}{2.955987in}}%
\pgfpathlineto{\pgfqpoint{6.881464in}{2.955987in}}%
\pgfpathlineto{\pgfqpoint{6.881464in}{3.037715in}}%
\pgfpathlineto{\pgfqpoint{6.695237in}{3.037715in}}%
\pgfpathlineto{\pgfqpoint{6.695237in}{2.955987in}}%
\pgfusepath{}%
\end{pgfscope}%
\begin{pgfscope}%
\pgfpathrectangle{\pgfqpoint{0.549740in}{0.463273in}}{\pgfqpoint{9.320225in}{4.495057in}}%
\pgfusepath{clip}%
\pgfsetbuttcap%
\pgfsetroundjoin%
\pgfsetlinewidth{0.000000pt}%
\definecolor{currentstroke}{rgb}{0.000000,0.000000,0.000000}%
\pgfsetstrokecolor{currentstroke}%
\pgfsetdash{}{0pt}%
\pgfpathmoveto{\pgfqpoint{6.881464in}{2.955987in}}%
\pgfpathlineto{\pgfqpoint{7.067690in}{2.955987in}}%
\pgfpathlineto{\pgfqpoint{7.067690in}{3.037715in}}%
\pgfpathlineto{\pgfqpoint{6.881464in}{3.037715in}}%
\pgfpathlineto{\pgfqpoint{6.881464in}{2.955987in}}%
\pgfusepath{}%
\end{pgfscope}%
\begin{pgfscope}%
\pgfpathrectangle{\pgfqpoint{0.549740in}{0.463273in}}{\pgfqpoint{9.320225in}{4.495057in}}%
\pgfusepath{clip}%
\pgfsetbuttcap%
\pgfsetroundjoin%
\definecolor{currentfill}{rgb}{0.472869,0.711325,0.955316}%
\pgfsetfillcolor{currentfill}%
\pgfsetlinewidth{0.000000pt}%
\definecolor{currentstroke}{rgb}{0.000000,0.000000,0.000000}%
\pgfsetstrokecolor{currentstroke}%
\pgfsetdash{}{0pt}%
\pgfpathmoveto{\pgfqpoint{7.067690in}{2.955987in}}%
\pgfpathlineto{\pgfqpoint{7.253917in}{2.955987in}}%
\pgfpathlineto{\pgfqpoint{7.253917in}{3.037715in}}%
\pgfpathlineto{\pgfqpoint{7.067690in}{3.037715in}}%
\pgfpathlineto{\pgfqpoint{7.067690in}{2.955987in}}%
\pgfusepath{fill}%
\end{pgfscope}%
\begin{pgfscope}%
\pgfpathrectangle{\pgfqpoint{0.549740in}{0.463273in}}{\pgfqpoint{9.320225in}{4.495057in}}%
\pgfusepath{clip}%
\pgfsetbuttcap%
\pgfsetroundjoin%
\pgfsetlinewidth{0.000000pt}%
\definecolor{currentstroke}{rgb}{0.000000,0.000000,0.000000}%
\pgfsetstrokecolor{currentstroke}%
\pgfsetdash{}{0pt}%
\pgfpathmoveto{\pgfqpoint{7.253917in}{2.955987in}}%
\pgfpathlineto{\pgfqpoint{7.440143in}{2.955987in}}%
\pgfpathlineto{\pgfqpoint{7.440143in}{3.037715in}}%
\pgfpathlineto{\pgfqpoint{7.253917in}{3.037715in}}%
\pgfpathlineto{\pgfqpoint{7.253917in}{2.955987in}}%
\pgfusepath{}%
\end{pgfscope}%
\begin{pgfscope}%
\pgfpathrectangle{\pgfqpoint{0.549740in}{0.463273in}}{\pgfqpoint{9.320225in}{4.495057in}}%
\pgfusepath{clip}%
\pgfsetbuttcap%
\pgfsetroundjoin%
\pgfsetlinewidth{0.000000pt}%
\definecolor{currentstroke}{rgb}{0.000000,0.000000,0.000000}%
\pgfsetstrokecolor{currentstroke}%
\pgfsetdash{}{0pt}%
\pgfpathmoveto{\pgfqpoint{7.440143in}{2.955987in}}%
\pgfpathlineto{\pgfqpoint{7.626370in}{2.955987in}}%
\pgfpathlineto{\pgfqpoint{7.626370in}{3.037715in}}%
\pgfpathlineto{\pgfqpoint{7.440143in}{3.037715in}}%
\pgfpathlineto{\pgfqpoint{7.440143in}{2.955987in}}%
\pgfusepath{}%
\end{pgfscope}%
\begin{pgfscope}%
\pgfpathrectangle{\pgfqpoint{0.549740in}{0.463273in}}{\pgfqpoint{9.320225in}{4.495057in}}%
\pgfusepath{clip}%
\pgfsetbuttcap%
\pgfsetroundjoin%
\pgfsetlinewidth{0.000000pt}%
\definecolor{currentstroke}{rgb}{0.000000,0.000000,0.000000}%
\pgfsetstrokecolor{currentstroke}%
\pgfsetdash{}{0pt}%
\pgfpathmoveto{\pgfqpoint{7.626370in}{2.955987in}}%
\pgfpathlineto{\pgfqpoint{7.812596in}{2.955987in}}%
\pgfpathlineto{\pgfqpoint{7.812596in}{3.037715in}}%
\pgfpathlineto{\pgfqpoint{7.626370in}{3.037715in}}%
\pgfpathlineto{\pgfqpoint{7.626370in}{2.955987in}}%
\pgfusepath{}%
\end{pgfscope}%
\begin{pgfscope}%
\pgfpathrectangle{\pgfqpoint{0.549740in}{0.463273in}}{\pgfqpoint{9.320225in}{4.495057in}}%
\pgfusepath{clip}%
\pgfsetbuttcap%
\pgfsetroundjoin%
\pgfsetlinewidth{0.000000pt}%
\definecolor{currentstroke}{rgb}{0.000000,0.000000,0.000000}%
\pgfsetstrokecolor{currentstroke}%
\pgfsetdash{}{0pt}%
\pgfpathmoveto{\pgfqpoint{7.812596in}{2.955987in}}%
\pgfpathlineto{\pgfqpoint{7.998823in}{2.955987in}}%
\pgfpathlineto{\pgfqpoint{7.998823in}{3.037715in}}%
\pgfpathlineto{\pgfqpoint{7.812596in}{3.037715in}}%
\pgfpathlineto{\pgfqpoint{7.812596in}{2.955987in}}%
\pgfusepath{}%
\end{pgfscope}%
\begin{pgfscope}%
\pgfpathrectangle{\pgfqpoint{0.549740in}{0.463273in}}{\pgfqpoint{9.320225in}{4.495057in}}%
\pgfusepath{clip}%
\pgfsetbuttcap%
\pgfsetroundjoin%
\pgfsetlinewidth{0.000000pt}%
\definecolor{currentstroke}{rgb}{0.000000,0.000000,0.000000}%
\pgfsetstrokecolor{currentstroke}%
\pgfsetdash{}{0pt}%
\pgfpathmoveto{\pgfqpoint{7.998823in}{2.955987in}}%
\pgfpathlineto{\pgfqpoint{8.185049in}{2.955987in}}%
\pgfpathlineto{\pgfqpoint{8.185049in}{3.037715in}}%
\pgfpathlineto{\pgfqpoint{7.998823in}{3.037715in}}%
\pgfpathlineto{\pgfqpoint{7.998823in}{2.955987in}}%
\pgfusepath{}%
\end{pgfscope}%
\begin{pgfscope}%
\pgfpathrectangle{\pgfqpoint{0.549740in}{0.463273in}}{\pgfqpoint{9.320225in}{4.495057in}}%
\pgfusepath{clip}%
\pgfsetbuttcap%
\pgfsetroundjoin%
\definecolor{currentfill}{rgb}{0.472869,0.711325,0.955316}%
\pgfsetfillcolor{currentfill}%
\pgfsetlinewidth{0.000000pt}%
\definecolor{currentstroke}{rgb}{0.000000,0.000000,0.000000}%
\pgfsetstrokecolor{currentstroke}%
\pgfsetdash{}{0pt}%
\pgfpathmoveto{\pgfqpoint{8.185049in}{2.955987in}}%
\pgfpathlineto{\pgfqpoint{8.371276in}{2.955987in}}%
\pgfpathlineto{\pgfqpoint{8.371276in}{3.037715in}}%
\pgfpathlineto{\pgfqpoint{8.185049in}{3.037715in}}%
\pgfpathlineto{\pgfqpoint{8.185049in}{2.955987in}}%
\pgfusepath{fill}%
\end{pgfscope}%
\begin{pgfscope}%
\pgfpathrectangle{\pgfqpoint{0.549740in}{0.463273in}}{\pgfqpoint{9.320225in}{4.495057in}}%
\pgfusepath{clip}%
\pgfsetbuttcap%
\pgfsetroundjoin%
\pgfsetlinewidth{0.000000pt}%
\definecolor{currentstroke}{rgb}{0.000000,0.000000,0.000000}%
\pgfsetstrokecolor{currentstroke}%
\pgfsetdash{}{0pt}%
\pgfpathmoveto{\pgfqpoint{8.371276in}{2.955987in}}%
\pgfpathlineto{\pgfqpoint{8.557503in}{2.955987in}}%
\pgfpathlineto{\pgfqpoint{8.557503in}{3.037715in}}%
\pgfpathlineto{\pgfqpoint{8.371276in}{3.037715in}}%
\pgfpathlineto{\pgfqpoint{8.371276in}{2.955987in}}%
\pgfusepath{}%
\end{pgfscope}%
\begin{pgfscope}%
\pgfpathrectangle{\pgfqpoint{0.549740in}{0.463273in}}{\pgfqpoint{9.320225in}{4.495057in}}%
\pgfusepath{clip}%
\pgfsetbuttcap%
\pgfsetroundjoin%
\pgfsetlinewidth{0.000000pt}%
\definecolor{currentstroke}{rgb}{0.000000,0.000000,0.000000}%
\pgfsetstrokecolor{currentstroke}%
\pgfsetdash{}{0pt}%
\pgfpathmoveto{\pgfqpoint{8.557503in}{2.955987in}}%
\pgfpathlineto{\pgfqpoint{8.743729in}{2.955987in}}%
\pgfpathlineto{\pgfqpoint{8.743729in}{3.037715in}}%
\pgfpathlineto{\pgfqpoint{8.557503in}{3.037715in}}%
\pgfpathlineto{\pgfqpoint{8.557503in}{2.955987in}}%
\pgfusepath{}%
\end{pgfscope}%
\begin{pgfscope}%
\pgfpathrectangle{\pgfqpoint{0.549740in}{0.463273in}}{\pgfqpoint{9.320225in}{4.495057in}}%
\pgfusepath{clip}%
\pgfsetbuttcap%
\pgfsetroundjoin%
\pgfsetlinewidth{0.000000pt}%
\definecolor{currentstroke}{rgb}{0.000000,0.000000,0.000000}%
\pgfsetstrokecolor{currentstroke}%
\pgfsetdash{}{0pt}%
\pgfpathmoveto{\pgfqpoint{8.743729in}{2.955987in}}%
\pgfpathlineto{\pgfqpoint{8.929956in}{2.955987in}}%
\pgfpathlineto{\pgfqpoint{8.929956in}{3.037715in}}%
\pgfpathlineto{\pgfqpoint{8.743729in}{3.037715in}}%
\pgfpathlineto{\pgfqpoint{8.743729in}{2.955987in}}%
\pgfusepath{}%
\end{pgfscope}%
\begin{pgfscope}%
\pgfpathrectangle{\pgfqpoint{0.549740in}{0.463273in}}{\pgfqpoint{9.320225in}{4.495057in}}%
\pgfusepath{clip}%
\pgfsetbuttcap%
\pgfsetroundjoin%
\pgfsetlinewidth{0.000000pt}%
\definecolor{currentstroke}{rgb}{0.000000,0.000000,0.000000}%
\pgfsetstrokecolor{currentstroke}%
\pgfsetdash{}{0pt}%
\pgfpathmoveto{\pgfqpoint{8.929956in}{2.955987in}}%
\pgfpathlineto{\pgfqpoint{9.116182in}{2.955987in}}%
\pgfpathlineto{\pgfqpoint{9.116182in}{3.037715in}}%
\pgfpathlineto{\pgfqpoint{8.929956in}{3.037715in}}%
\pgfpathlineto{\pgfqpoint{8.929956in}{2.955987in}}%
\pgfusepath{}%
\end{pgfscope}%
\begin{pgfscope}%
\pgfpathrectangle{\pgfqpoint{0.549740in}{0.463273in}}{\pgfqpoint{9.320225in}{4.495057in}}%
\pgfusepath{clip}%
\pgfsetbuttcap%
\pgfsetroundjoin%
\pgfsetlinewidth{0.000000pt}%
\definecolor{currentstroke}{rgb}{0.000000,0.000000,0.000000}%
\pgfsetstrokecolor{currentstroke}%
\pgfsetdash{}{0pt}%
\pgfpathmoveto{\pgfqpoint{9.116182in}{2.955987in}}%
\pgfpathlineto{\pgfqpoint{9.302409in}{2.955987in}}%
\pgfpathlineto{\pgfqpoint{9.302409in}{3.037715in}}%
\pgfpathlineto{\pgfqpoint{9.116182in}{3.037715in}}%
\pgfpathlineto{\pgfqpoint{9.116182in}{2.955987in}}%
\pgfusepath{}%
\end{pgfscope}%
\begin{pgfscope}%
\pgfpathrectangle{\pgfqpoint{0.549740in}{0.463273in}}{\pgfqpoint{9.320225in}{4.495057in}}%
\pgfusepath{clip}%
\pgfsetbuttcap%
\pgfsetroundjoin%
\pgfsetlinewidth{0.000000pt}%
\definecolor{currentstroke}{rgb}{0.000000,0.000000,0.000000}%
\pgfsetstrokecolor{currentstroke}%
\pgfsetdash{}{0pt}%
\pgfpathmoveto{\pgfqpoint{9.302409in}{2.955987in}}%
\pgfpathlineto{\pgfqpoint{9.488635in}{2.955987in}}%
\pgfpathlineto{\pgfqpoint{9.488635in}{3.037715in}}%
\pgfpathlineto{\pgfqpoint{9.302409in}{3.037715in}}%
\pgfpathlineto{\pgfqpoint{9.302409in}{2.955987in}}%
\pgfusepath{}%
\end{pgfscope}%
\begin{pgfscope}%
\pgfpathrectangle{\pgfqpoint{0.549740in}{0.463273in}}{\pgfqpoint{9.320225in}{4.495057in}}%
\pgfusepath{clip}%
\pgfsetbuttcap%
\pgfsetroundjoin%
\definecolor{currentfill}{rgb}{0.472869,0.711325,0.955316}%
\pgfsetfillcolor{currentfill}%
\pgfsetlinewidth{0.000000pt}%
\definecolor{currentstroke}{rgb}{0.000000,0.000000,0.000000}%
\pgfsetstrokecolor{currentstroke}%
\pgfsetdash{}{0pt}%
\pgfpathmoveto{\pgfqpoint{9.488635in}{2.955987in}}%
\pgfpathlineto{\pgfqpoint{9.674862in}{2.955987in}}%
\pgfpathlineto{\pgfqpoint{9.674862in}{3.037715in}}%
\pgfpathlineto{\pgfqpoint{9.488635in}{3.037715in}}%
\pgfpathlineto{\pgfqpoint{9.488635in}{2.955987in}}%
\pgfusepath{fill}%
\end{pgfscope}%
\begin{pgfscope}%
\pgfpathrectangle{\pgfqpoint{0.549740in}{0.463273in}}{\pgfqpoint{9.320225in}{4.495057in}}%
\pgfusepath{clip}%
\pgfsetbuttcap%
\pgfsetroundjoin%
\pgfsetlinewidth{0.000000pt}%
\definecolor{currentstroke}{rgb}{0.000000,0.000000,0.000000}%
\pgfsetstrokecolor{currentstroke}%
\pgfsetdash{}{0pt}%
\pgfpathmoveto{\pgfqpoint{9.674862in}{2.955987in}}%
\pgfpathlineto{\pgfqpoint{9.861088in}{2.955987in}}%
\pgfpathlineto{\pgfqpoint{9.861088in}{3.037715in}}%
\pgfpathlineto{\pgfqpoint{9.674862in}{3.037715in}}%
\pgfpathlineto{\pgfqpoint{9.674862in}{2.955987in}}%
\pgfusepath{}%
\end{pgfscope}%
\begin{pgfscope}%
\pgfpathrectangle{\pgfqpoint{0.549740in}{0.463273in}}{\pgfqpoint{9.320225in}{4.495057in}}%
\pgfusepath{clip}%
\pgfsetbuttcap%
\pgfsetroundjoin%
\pgfsetlinewidth{0.000000pt}%
\definecolor{currentstroke}{rgb}{0.000000,0.000000,0.000000}%
\pgfsetstrokecolor{currentstroke}%
\pgfsetdash{}{0pt}%
\pgfpathmoveto{\pgfqpoint{0.549761in}{3.037715in}}%
\pgfpathlineto{\pgfqpoint{0.735988in}{3.037715in}}%
\pgfpathlineto{\pgfqpoint{0.735988in}{3.119443in}}%
\pgfpathlineto{\pgfqpoint{0.549761in}{3.119443in}}%
\pgfpathlineto{\pgfqpoint{0.549761in}{3.037715in}}%
\pgfusepath{}%
\end{pgfscope}%
\begin{pgfscope}%
\pgfpathrectangle{\pgfqpoint{0.549740in}{0.463273in}}{\pgfqpoint{9.320225in}{4.495057in}}%
\pgfusepath{clip}%
\pgfsetbuttcap%
\pgfsetroundjoin%
\pgfsetlinewidth{0.000000pt}%
\definecolor{currentstroke}{rgb}{0.000000,0.000000,0.000000}%
\pgfsetstrokecolor{currentstroke}%
\pgfsetdash{}{0pt}%
\pgfpathmoveto{\pgfqpoint{0.735988in}{3.037715in}}%
\pgfpathlineto{\pgfqpoint{0.922214in}{3.037715in}}%
\pgfpathlineto{\pgfqpoint{0.922214in}{3.119443in}}%
\pgfpathlineto{\pgfqpoint{0.735988in}{3.119443in}}%
\pgfpathlineto{\pgfqpoint{0.735988in}{3.037715in}}%
\pgfusepath{}%
\end{pgfscope}%
\begin{pgfscope}%
\pgfpathrectangle{\pgfqpoint{0.549740in}{0.463273in}}{\pgfqpoint{9.320225in}{4.495057in}}%
\pgfusepath{clip}%
\pgfsetbuttcap%
\pgfsetroundjoin%
\pgfsetlinewidth{0.000000pt}%
\definecolor{currentstroke}{rgb}{0.000000,0.000000,0.000000}%
\pgfsetstrokecolor{currentstroke}%
\pgfsetdash{}{0pt}%
\pgfpathmoveto{\pgfqpoint{0.922214in}{3.037715in}}%
\pgfpathlineto{\pgfqpoint{1.108441in}{3.037715in}}%
\pgfpathlineto{\pgfqpoint{1.108441in}{3.119443in}}%
\pgfpathlineto{\pgfqpoint{0.922214in}{3.119443in}}%
\pgfpathlineto{\pgfqpoint{0.922214in}{3.037715in}}%
\pgfusepath{}%
\end{pgfscope}%
\begin{pgfscope}%
\pgfpathrectangle{\pgfqpoint{0.549740in}{0.463273in}}{\pgfqpoint{9.320225in}{4.495057in}}%
\pgfusepath{clip}%
\pgfsetbuttcap%
\pgfsetroundjoin%
\pgfsetlinewidth{0.000000pt}%
\definecolor{currentstroke}{rgb}{0.000000,0.000000,0.000000}%
\pgfsetstrokecolor{currentstroke}%
\pgfsetdash{}{0pt}%
\pgfpathmoveto{\pgfqpoint{1.108441in}{3.037715in}}%
\pgfpathlineto{\pgfqpoint{1.294667in}{3.037715in}}%
\pgfpathlineto{\pgfqpoint{1.294667in}{3.119443in}}%
\pgfpathlineto{\pgfqpoint{1.108441in}{3.119443in}}%
\pgfpathlineto{\pgfqpoint{1.108441in}{3.037715in}}%
\pgfusepath{}%
\end{pgfscope}%
\begin{pgfscope}%
\pgfpathrectangle{\pgfqpoint{0.549740in}{0.463273in}}{\pgfqpoint{9.320225in}{4.495057in}}%
\pgfusepath{clip}%
\pgfsetbuttcap%
\pgfsetroundjoin%
\pgfsetlinewidth{0.000000pt}%
\definecolor{currentstroke}{rgb}{0.000000,0.000000,0.000000}%
\pgfsetstrokecolor{currentstroke}%
\pgfsetdash{}{0pt}%
\pgfpathmoveto{\pgfqpoint{1.294667in}{3.037715in}}%
\pgfpathlineto{\pgfqpoint{1.480894in}{3.037715in}}%
\pgfpathlineto{\pgfqpoint{1.480894in}{3.119443in}}%
\pgfpathlineto{\pgfqpoint{1.294667in}{3.119443in}}%
\pgfpathlineto{\pgfqpoint{1.294667in}{3.037715in}}%
\pgfusepath{}%
\end{pgfscope}%
\begin{pgfscope}%
\pgfpathrectangle{\pgfqpoint{0.549740in}{0.463273in}}{\pgfqpoint{9.320225in}{4.495057in}}%
\pgfusepath{clip}%
\pgfsetbuttcap%
\pgfsetroundjoin%
\pgfsetlinewidth{0.000000pt}%
\definecolor{currentstroke}{rgb}{0.000000,0.000000,0.000000}%
\pgfsetstrokecolor{currentstroke}%
\pgfsetdash{}{0pt}%
\pgfpathmoveto{\pgfqpoint{1.480894in}{3.037715in}}%
\pgfpathlineto{\pgfqpoint{1.667120in}{3.037715in}}%
\pgfpathlineto{\pgfqpoint{1.667120in}{3.119443in}}%
\pgfpathlineto{\pgfqpoint{1.480894in}{3.119443in}}%
\pgfpathlineto{\pgfqpoint{1.480894in}{3.037715in}}%
\pgfusepath{}%
\end{pgfscope}%
\begin{pgfscope}%
\pgfpathrectangle{\pgfqpoint{0.549740in}{0.463273in}}{\pgfqpoint{9.320225in}{4.495057in}}%
\pgfusepath{clip}%
\pgfsetbuttcap%
\pgfsetroundjoin%
\pgfsetlinewidth{0.000000pt}%
\definecolor{currentstroke}{rgb}{0.000000,0.000000,0.000000}%
\pgfsetstrokecolor{currentstroke}%
\pgfsetdash{}{0pt}%
\pgfpathmoveto{\pgfqpoint{1.667120in}{3.037715in}}%
\pgfpathlineto{\pgfqpoint{1.853347in}{3.037715in}}%
\pgfpathlineto{\pgfqpoint{1.853347in}{3.119443in}}%
\pgfpathlineto{\pgfqpoint{1.667120in}{3.119443in}}%
\pgfpathlineto{\pgfqpoint{1.667120in}{3.037715in}}%
\pgfusepath{}%
\end{pgfscope}%
\begin{pgfscope}%
\pgfpathrectangle{\pgfqpoint{0.549740in}{0.463273in}}{\pgfqpoint{9.320225in}{4.495057in}}%
\pgfusepath{clip}%
\pgfsetbuttcap%
\pgfsetroundjoin%
\pgfsetlinewidth{0.000000pt}%
\definecolor{currentstroke}{rgb}{0.000000,0.000000,0.000000}%
\pgfsetstrokecolor{currentstroke}%
\pgfsetdash{}{0pt}%
\pgfpathmoveto{\pgfqpoint{1.853347in}{3.037715in}}%
\pgfpathlineto{\pgfqpoint{2.039573in}{3.037715in}}%
\pgfpathlineto{\pgfqpoint{2.039573in}{3.119443in}}%
\pgfpathlineto{\pgfqpoint{1.853347in}{3.119443in}}%
\pgfpathlineto{\pgfqpoint{1.853347in}{3.037715in}}%
\pgfusepath{}%
\end{pgfscope}%
\begin{pgfscope}%
\pgfpathrectangle{\pgfqpoint{0.549740in}{0.463273in}}{\pgfqpoint{9.320225in}{4.495057in}}%
\pgfusepath{clip}%
\pgfsetbuttcap%
\pgfsetroundjoin%
\pgfsetlinewidth{0.000000pt}%
\definecolor{currentstroke}{rgb}{0.000000,0.000000,0.000000}%
\pgfsetstrokecolor{currentstroke}%
\pgfsetdash{}{0pt}%
\pgfpathmoveto{\pgfqpoint{2.039573in}{3.037715in}}%
\pgfpathlineto{\pgfqpoint{2.225800in}{3.037715in}}%
\pgfpathlineto{\pgfqpoint{2.225800in}{3.119443in}}%
\pgfpathlineto{\pgfqpoint{2.039573in}{3.119443in}}%
\pgfpathlineto{\pgfqpoint{2.039573in}{3.037715in}}%
\pgfusepath{}%
\end{pgfscope}%
\begin{pgfscope}%
\pgfpathrectangle{\pgfqpoint{0.549740in}{0.463273in}}{\pgfqpoint{9.320225in}{4.495057in}}%
\pgfusepath{clip}%
\pgfsetbuttcap%
\pgfsetroundjoin%
\pgfsetlinewidth{0.000000pt}%
\definecolor{currentstroke}{rgb}{0.000000,0.000000,0.000000}%
\pgfsetstrokecolor{currentstroke}%
\pgfsetdash{}{0pt}%
\pgfpathmoveto{\pgfqpoint{2.225800in}{3.037715in}}%
\pgfpathlineto{\pgfqpoint{2.412027in}{3.037715in}}%
\pgfpathlineto{\pgfqpoint{2.412027in}{3.119443in}}%
\pgfpathlineto{\pgfqpoint{2.225800in}{3.119443in}}%
\pgfpathlineto{\pgfqpoint{2.225800in}{3.037715in}}%
\pgfusepath{}%
\end{pgfscope}%
\begin{pgfscope}%
\pgfpathrectangle{\pgfqpoint{0.549740in}{0.463273in}}{\pgfqpoint{9.320225in}{4.495057in}}%
\pgfusepath{clip}%
\pgfsetbuttcap%
\pgfsetroundjoin%
\pgfsetlinewidth{0.000000pt}%
\definecolor{currentstroke}{rgb}{0.000000,0.000000,0.000000}%
\pgfsetstrokecolor{currentstroke}%
\pgfsetdash{}{0pt}%
\pgfpathmoveto{\pgfqpoint{2.412027in}{3.037715in}}%
\pgfpathlineto{\pgfqpoint{2.598253in}{3.037715in}}%
\pgfpathlineto{\pgfqpoint{2.598253in}{3.119443in}}%
\pgfpathlineto{\pgfqpoint{2.412027in}{3.119443in}}%
\pgfpathlineto{\pgfqpoint{2.412027in}{3.037715in}}%
\pgfusepath{}%
\end{pgfscope}%
\begin{pgfscope}%
\pgfpathrectangle{\pgfqpoint{0.549740in}{0.463273in}}{\pgfqpoint{9.320225in}{4.495057in}}%
\pgfusepath{clip}%
\pgfsetbuttcap%
\pgfsetroundjoin%
\pgfsetlinewidth{0.000000pt}%
\definecolor{currentstroke}{rgb}{0.000000,0.000000,0.000000}%
\pgfsetstrokecolor{currentstroke}%
\pgfsetdash{}{0pt}%
\pgfpathmoveto{\pgfqpoint{2.598253in}{3.037715in}}%
\pgfpathlineto{\pgfqpoint{2.784480in}{3.037715in}}%
\pgfpathlineto{\pgfqpoint{2.784480in}{3.119443in}}%
\pgfpathlineto{\pgfqpoint{2.598253in}{3.119443in}}%
\pgfpathlineto{\pgfqpoint{2.598253in}{3.037715in}}%
\pgfusepath{}%
\end{pgfscope}%
\begin{pgfscope}%
\pgfpathrectangle{\pgfqpoint{0.549740in}{0.463273in}}{\pgfqpoint{9.320225in}{4.495057in}}%
\pgfusepath{clip}%
\pgfsetbuttcap%
\pgfsetroundjoin%
\pgfsetlinewidth{0.000000pt}%
\definecolor{currentstroke}{rgb}{0.000000,0.000000,0.000000}%
\pgfsetstrokecolor{currentstroke}%
\pgfsetdash{}{0pt}%
\pgfpathmoveto{\pgfqpoint{2.784480in}{3.037715in}}%
\pgfpathlineto{\pgfqpoint{2.970706in}{3.037715in}}%
\pgfpathlineto{\pgfqpoint{2.970706in}{3.119443in}}%
\pgfpathlineto{\pgfqpoint{2.784480in}{3.119443in}}%
\pgfpathlineto{\pgfqpoint{2.784480in}{3.037715in}}%
\pgfusepath{}%
\end{pgfscope}%
\begin{pgfscope}%
\pgfpathrectangle{\pgfqpoint{0.549740in}{0.463273in}}{\pgfqpoint{9.320225in}{4.495057in}}%
\pgfusepath{clip}%
\pgfsetbuttcap%
\pgfsetroundjoin%
\pgfsetlinewidth{0.000000pt}%
\definecolor{currentstroke}{rgb}{0.000000,0.000000,0.000000}%
\pgfsetstrokecolor{currentstroke}%
\pgfsetdash{}{0pt}%
\pgfpathmoveto{\pgfqpoint{2.970706in}{3.037715in}}%
\pgfpathlineto{\pgfqpoint{3.156933in}{3.037715in}}%
\pgfpathlineto{\pgfqpoint{3.156933in}{3.119443in}}%
\pgfpathlineto{\pgfqpoint{2.970706in}{3.119443in}}%
\pgfpathlineto{\pgfqpoint{2.970706in}{3.037715in}}%
\pgfusepath{}%
\end{pgfscope}%
\begin{pgfscope}%
\pgfpathrectangle{\pgfqpoint{0.549740in}{0.463273in}}{\pgfqpoint{9.320225in}{4.495057in}}%
\pgfusepath{clip}%
\pgfsetbuttcap%
\pgfsetroundjoin%
\pgfsetlinewidth{0.000000pt}%
\definecolor{currentstroke}{rgb}{0.000000,0.000000,0.000000}%
\pgfsetstrokecolor{currentstroke}%
\pgfsetdash{}{0pt}%
\pgfpathmoveto{\pgfqpoint{3.156933in}{3.037715in}}%
\pgfpathlineto{\pgfqpoint{3.343159in}{3.037715in}}%
\pgfpathlineto{\pgfqpoint{3.343159in}{3.119443in}}%
\pgfpathlineto{\pgfqpoint{3.156933in}{3.119443in}}%
\pgfpathlineto{\pgfqpoint{3.156933in}{3.037715in}}%
\pgfusepath{}%
\end{pgfscope}%
\begin{pgfscope}%
\pgfpathrectangle{\pgfqpoint{0.549740in}{0.463273in}}{\pgfqpoint{9.320225in}{4.495057in}}%
\pgfusepath{clip}%
\pgfsetbuttcap%
\pgfsetroundjoin%
\pgfsetlinewidth{0.000000pt}%
\definecolor{currentstroke}{rgb}{0.000000,0.000000,0.000000}%
\pgfsetstrokecolor{currentstroke}%
\pgfsetdash{}{0pt}%
\pgfpathmoveto{\pgfqpoint{3.343159in}{3.037715in}}%
\pgfpathlineto{\pgfqpoint{3.529386in}{3.037715in}}%
\pgfpathlineto{\pgfqpoint{3.529386in}{3.119443in}}%
\pgfpathlineto{\pgfqpoint{3.343159in}{3.119443in}}%
\pgfpathlineto{\pgfqpoint{3.343159in}{3.037715in}}%
\pgfusepath{}%
\end{pgfscope}%
\begin{pgfscope}%
\pgfpathrectangle{\pgfqpoint{0.549740in}{0.463273in}}{\pgfqpoint{9.320225in}{4.495057in}}%
\pgfusepath{clip}%
\pgfsetbuttcap%
\pgfsetroundjoin%
\pgfsetlinewidth{0.000000pt}%
\definecolor{currentstroke}{rgb}{0.000000,0.000000,0.000000}%
\pgfsetstrokecolor{currentstroke}%
\pgfsetdash{}{0pt}%
\pgfpathmoveto{\pgfqpoint{3.529386in}{3.037715in}}%
\pgfpathlineto{\pgfqpoint{3.715612in}{3.037715in}}%
\pgfpathlineto{\pgfqpoint{3.715612in}{3.119443in}}%
\pgfpathlineto{\pgfqpoint{3.529386in}{3.119443in}}%
\pgfpathlineto{\pgfqpoint{3.529386in}{3.037715in}}%
\pgfusepath{}%
\end{pgfscope}%
\begin{pgfscope}%
\pgfpathrectangle{\pgfqpoint{0.549740in}{0.463273in}}{\pgfqpoint{9.320225in}{4.495057in}}%
\pgfusepath{clip}%
\pgfsetbuttcap%
\pgfsetroundjoin%
\pgfsetlinewidth{0.000000pt}%
\definecolor{currentstroke}{rgb}{0.000000,0.000000,0.000000}%
\pgfsetstrokecolor{currentstroke}%
\pgfsetdash{}{0pt}%
\pgfpathmoveto{\pgfqpoint{3.715612in}{3.037715in}}%
\pgfpathlineto{\pgfqpoint{3.901839in}{3.037715in}}%
\pgfpathlineto{\pgfqpoint{3.901839in}{3.119443in}}%
\pgfpathlineto{\pgfqpoint{3.715612in}{3.119443in}}%
\pgfpathlineto{\pgfqpoint{3.715612in}{3.037715in}}%
\pgfusepath{}%
\end{pgfscope}%
\begin{pgfscope}%
\pgfpathrectangle{\pgfqpoint{0.549740in}{0.463273in}}{\pgfqpoint{9.320225in}{4.495057in}}%
\pgfusepath{clip}%
\pgfsetbuttcap%
\pgfsetroundjoin%
\pgfsetlinewidth{0.000000pt}%
\definecolor{currentstroke}{rgb}{0.000000,0.000000,0.000000}%
\pgfsetstrokecolor{currentstroke}%
\pgfsetdash{}{0pt}%
\pgfpathmoveto{\pgfqpoint{3.901839in}{3.037715in}}%
\pgfpathlineto{\pgfqpoint{4.088065in}{3.037715in}}%
\pgfpathlineto{\pgfqpoint{4.088065in}{3.119443in}}%
\pgfpathlineto{\pgfqpoint{3.901839in}{3.119443in}}%
\pgfpathlineto{\pgfqpoint{3.901839in}{3.037715in}}%
\pgfusepath{}%
\end{pgfscope}%
\begin{pgfscope}%
\pgfpathrectangle{\pgfqpoint{0.549740in}{0.463273in}}{\pgfqpoint{9.320225in}{4.495057in}}%
\pgfusepath{clip}%
\pgfsetbuttcap%
\pgfsetroundjoin%
\pgfsetlinewidth{0.000000pt}%
\definecolor{currentstroke}{rgb}{0.000000,0.000000,0.000000}%
\pgfsetstrokecolor{currentstroke}%
\pgfsetdash{}{0pt}%
\pgfpathmoveto{\pgfqpoint{4.088065in}{3.037715in}}%
\pgfpathlineto{\pgfqpoint{4.274292in}{3.037715in}}%
\pgfpathlineto{\pgfqpoint{4.274292in}{3.119443in}}%
\pgfpathlineto{\pgfqpoint{4.088065in}{3.119443in}}%
\pgfpathlineto{\pgfqpoint{4.088065in}{3.037715in}}%
\pgfusepath{}%
\end{pgfscope}%
\begin{pgfscope}%
\pgfpathrectangle{\pgfqpoint{0.549740in}{0.463273in}}{\pgfqpoint{9.320225in}{4.495057in}}%
\pgfusepath{clip}%
\pgfsetbuttcap%
\pgfsetroundjoin%
\pgfsetlinewidth{0.000000pt}%
\definecolor{currentstroke}{rgb}{0.000000,0.000000,0.000000}%
\pgfsetstrokecolor{currentstroke}%
\pgfsetdash{}{0pt}%
\pgfpathmoveto{\pgfqpoint{4.274292in}{3.037715in}}%
\pgfpathlineto{\pgfqpoint{4.460519in}{3.037715in}}%
\pgfpathlineto{\pgfqpoint{4.460519in}{3.119443in}}%
\pgfpathlineto{\pgfqpoint{4.274292in}{3.119443in}}%
\pgfpathlineto{\pgfqpoint{4.274292in}{3.037715in}}%
\pgfusepath{}%
\end{pgfscope}%
\begin{pgfscope}%
\pgfpathrectangle{\pgfqpoint{0.549740in}{0.463273in}}{\pgfqpoint{9.320225in}{4.495057in}}%
\pgfusepath{clip}%
\pgfsetbuttcap%
\pgfsetroundjoin%
\pgfsetlinewidth{0.000000pt}%
\definecolor{currentstroke}{rgb}{0.000000,0.000000,0.000000}%
\pgfsetstrokecolor{currentstroke}%
\pgfsetdash{}{0pt}%
\pgfpathmoveto{\pgfqpoint{4.460519in}{3.037715in}}%
\pgfpathlineto{\pgfqpoint{4.646745in}{3.037715in}}%
\pgfpathlineto{\pgfqpoint{4.646745in}{3.119443in}}%
\pgfpathlineto{\pgfqpoint{4.460519in}{3.119443in}}%
\pgfpathlineto{\pgfqpoint{4.460519in}{3.037715in}}%
\pgfusepath{}%
\end{pgfscope}%
\begin{pgfscope}%
\pgfpathrectangle{\pgfqpoint{0.549740in}{0.463273in}}{\pgfqpoint{9.320225in}{4.495057in}}%
\pgfusepath{clip}%
\pgfsetbuttcap%
\pgfsetroundjoin%
\pgfsetlinewidth{0.000000pt}%
\definecolor{currentstroke}{rgb}{0.000000,0.000000,0.000000}%
\pgfsetstrokecolor{currentstroke}%
\pgfsetdash{}{0pt}%
\pgfpathmoveto{\pgfqpoint{4.646745in}{3.037715in}}%
\pgfpathlineto{\pgfqpoint{4.832972in}{3.037715in}}%
\pgfpathlineto{\pgfqpoint{4.832972in}{3.119443in}}%
\pgfpathlineto{\pgfqpoint{4.646745in}{3.119443in}}%
\pgfpathlineto{\pgfqpoint{4.646745in}{3.037715in}}%
\pgfusepath{}%
\end{pgfscope}%
\begin{pgfscope}%
\pgfpathrectangle{\pgfqpoint{0.549740in}{0.463273in}}{\pgfqpoint{9.320225in}{4.495057in}}%
\pgfusepath{clip}%
\pgfsetbuttcap%
\pgfsetroundjoin%
\pgfsetlinewidth{0.000000pt}%
\definecolor{currentstroke}{rgb}{0.000000,0.000000,0.000000}%
\pgfsetstrokecolor{currentstroke}%
\pgfsetdash{}{0pt}%
\pgfpathmoveto{\pgfqpoint{4.832972in}{3.037715in}}%
\pgfpathlineto{\pgfqpoint{5.019198in}{3.037715in}}%
\pgfpathlineto{\pgfqpoint{5.019198in}{3.119443in}}%
\pgfpathlineto{\pgfqpoint{4.832972in}{3.119443in}}%
\pgfpathlineto{\pgfqpoint{4.832972in}{3.037715in}}%
\pgfusepath{}%
\end{pgfscope}%
\begin{pgfscope}%
\pgfpathrectangle{\pgfqpoint{0.549740in}{0.463273in}}{\pgfqpoint{9.320225in}{4.495057in}}%
\pgfusepath{clip}%
\pgfsetbuttcap%
\pgfsetroundjoin%
\pgfsetlinewidth{0.000000pt}%
\definecolor{currentstroke}{rgb}{0.000000,0.000000,0.000000}%
\pgfsetstrokecolor{currentstroke}%
\pgfsetdash{}{0pt}%
\pgfpathmoveto{\pgfqpoint{5.019198in}{3.037715in}}%
\pgfpathlineto{\pgfqpoint{5.205425in}{3.037715in}}%
\pgfpathlineto{\pgfqpoint{5.205425in}{3.119443in}}%
\pgfpathlineto{\pgfqpoint{5.019198in}{3.119443in}}%
\pgfpathlineto{\pgfqpoint{5.019198in}{3.037715in}}%
\pgfusepath{}%
\end{pgfscope}%
\begin{pgfscope}%
\pgfpathrectangle{\pgfqpoint{0.549740in}{0.463273in}}{\pgfqpoint{9.320225in}{4.495057in}}%
\pgfusepath{clip}%
\pgfsetbuttcap%
\pgfsetroundjoin%
\definecolor{currentfill}{rgb}{0.385185,0.686583,0.962589}%
\pgfsetfillcolor{currentfill}%
\pgfsetlinewidth{0.000000pt}%
\definecolor{currentstroke}{rgb}{0.000000,0.000000,0.000000}%
\pgfsetstrokecolor{currentstroke}%
\pgfsetdash{}{0pt}%
\pgfpathmoveto{\pgfqpoint{5.205425in}{3.037715in}}%
\pgfpathlineto{\pgfqpoint{5.391651in}{3.037715in}}%
\pgfpathlineto{\pgfqpoint{5.391651in}{3.119443in}}%
\pgfpathlineto{\pgfqpoint{5.205425in}{3.119443in}}%
\pgfpathlineto{\pgfqpoint{5.205425in}{3.037715in}}%
\pgfusepath{fill}%
\end{pgfscope}%
\begin{pgfscope}%
\pgfpathrectangle{\pgfqpoint{0.549740in}{0.463273in}}{\pgfqpoint{9.320225in}{4.495057in}}%
\pgfusepath{clip}%
\pgfsetbuttcap%
\pgfsetroundjoin%
\pgfsetlinewidth{0.000000pt}%
\definecolor{currentstroke}{rgb}{0.000000,0.000000,0.000000}%
\pgfsetstrokecolor{currentstroke}%
\pgfsetdash{}{0pt}%
\pgfpathmoveto{\pgfqpoint{5.391651in}{3.037715in}}%
\pgfpathlineto{\pgfqpoint{5.577878in}{3.037715in}}%
\pgfpathlineto{\pgfqpoint{5.577878in}{3.119443in}}%
\pgfpathlineto{\pgfqpoint{5.391651in}{3.119443in}}%
\pgfpathlineto{\pgfqpoint{5.391651in}{3.037715in}}%
\pgfusepath{}%
\end{pgfscope}%
\begin{pgfscope}%
\pgfpathrectangle{\pgfqpoint{0.549740in}{0.463273in}}{\pgfqpoint{9.320225in}{4.495057in}}%
\pgfusepath{clip}%
\pgfsetbuttcap%
\pgfsetroundjoin%
\pgfsetlinewidth{0.000000pt}%
\definecolor{currentstroke}{rgb}{0.000000,0.000000,0.000000}%
\pgfsetstrokecolor{currentstroke}%
\pgfsetdash{}{0pt}%
\pgfpathmoveto{\pgfqpoint{5.577878in}{3.037715in}}%
\pgfpathlineto{\pgfqpoint{5.764104in}{3.037715in}}%
\pgfpathlineto{\pgfqpoint{5.764104in}{3.119443in}}%
\pgfpathlineto{\pgfqpoint{5.577878in}{3.119443in}}%
\pgfpathlineto{\pgfqpoint{5.577878in}{3.037715in}}%
\pgfusepath{}%
\end{pgfscope}%
\begin{pgfscope}%
\pgfpathrectangle{\pgfqpoint{0.549740in}{0.463273in}}{\pgfqpoint{9.320225in}{4.495057in}}%
\pgfusepath{clip}%
\pgfsetbuttcap%
\pgfsetroundjoin%
\pgfsetlinewidth{0.000000pt}%
\definecolor{currentstroke}{rgb}{0.000000,0.000000,0.000000}%
\pgfsetstrokecolor{currentstroke}%
\pgfsetdash{}{0pt}%
\pgfpathmoveto{\pgfqpoint{5.764104in}{3.037715in}}%
\pgfpathlineto{\pgfqpoint{5.950331in}{3.037715in}}%
\pgfpathlineto{\pgfqpoint{5.950331in}{3.119443in}}%
\pgfpathlineto{\pgfqpoint{5.764104in}{3.119443in}}%
\pgfpathlineto{\pgfqpoint{5.764104in}{3.037715in}}%
\pgfusepath{}%
\end{pgfscope}%
\begin{pgfscope}%
\pgfpathrectangle{\pgfqpoint{0.549740in}{0.463273in}}{\pgfqpoint{9.320225in}{4.495057in}}%
\pgfusepath{clip}%
\pgfsetbuttcap%
\pgfsetroundjoin%
\definecolor{currentfill}{rgb}{0.472869,0.711325,0.955316}%
\pgfsetfillcolor{currentfill}%
\pgfsetlinewidth{0.000000pt}%
\definecolor{currentstroke}{rgb}{0.000000,0.000000,0.000000}%
\pgfsetstrokecolor{currentstroke}%
\pgfsetdash{}{0pt}%
\pgfpathmoveto{\pgfqpoint{5.950331in}{3.037715in}}%
\pgfpathlineto{\pgfqpoint{6.136557in}{3.037715in}}%
\pgfpathlineto{\pgfqpoint{6.136557in}{3.119443in}}%
\pgfpathlineto{\pgfqpoint{5.950331in}{3.119443in}}%
\pgfpathlineto{\pgfqpoint{5.950331in}{3.037715in}}%
\pgfusepath{fill}%
\end{pgfscope}%
\begin{pgfscope}%
\pgfpathrectangle{\pgfqpoint{0.549740in}{0.463273in}}{\pgfqpoint{9.320225in}{4.495057in}}%
\pgfusepath{clip}%
\pgfsetbuttcap%
\pgfsetroundjoin%
\pgfsetlinewidth{0.000000pt}%
\definecolor{currentstroke}{rgb}{0.000000,0.000000,0.000000}%
\pgfsetstrokecolor{currentstroke}%
\pgfsetdash{}{0pt}%
\pgfpathmoveto{\pgfqpoint{6.136557in}{3.037715in}}%
\pgfpathlineto{\pgfqpoint{6.322784in}{3.037715in}}%
\pgfpathlineto{\pgfqpoint{6.322784in}{3.119443in}}%
\pgfpathlineto{\pgfqpoint{6.136557in}{3.119443in}}%
\pgfpathlineto{\pgfqpoint{6.136557in}{3.037715in}}%
\pgfusepath{}%
\end{pgfscope}%
\begin{pgfscope}%
\pgfpathrectangle{\pgfqpoint{0.549740in}{0.463273in}}{\pgfqpoint{9.320225in}{4.495057in}}%
\pgfusepath{clip}%
\pgfsetbuttcap%
\pgfsetroundjoin%
\pgfsetlinewidth{0.000000pt}%
\definecolor{currentstroke}{rgb}{0.000000,0.000000,0.000000}%
\pgfsetstrokecolor{currentstroke}%
\pgfsetdash{}{0pt}%
\pgfpathmoveto{\pgfqpoint{6.322784in}{3.037715in}}%
\pgfpathlineto{\pgfqpoint{6.509011in}{3.037715in}}%
\pgfpathlineto{\pgfqpoint{6.509011in}{3.119443in}}%
\pgfpathlineto{\pgfqpoint{6.322784in}{3.119443in}}%
\pgfpathlineto{\pgfqpoint{6.322784in}{3.037715in}}%
\pgfusepath{}%
\end{pgfscope}%
\begin{pgfscope}%
\pgfpathrectangle{\pgfqpoint{0.549740in}{0.463273in}}{\pgfqpoint{9.320225in}{4.495057in}}%
\pgfusepath{clip}%
\pgfsetbuttcap%
\pgfsetroundjoin%
\pgfsetlinewidth{0.000000pt}%
\definecolor{currentstroke}{rgb}{0.000000,0.000000,0.000000}%
\pgfsetstrokecolor{currentstroke}%
\pgfsetdash{}{0pt}%
\pgfpathmoveto{\pgfqpoint{6.509011in}{3.037715in}}%
\pgfpathlineto{\pgfqpoint{6.695237in}{3.037715in}}%
\pgfpathlineto{\pgfqpoint{6.695237in}{3.119443in}}%
\pgfpathlineto{\pgfqpoint{6.509011in}{3.119443in}}%
\pgfpathlineto{\pgfqpoint{6.509011in}{3.037715in}}%
\pgfusepath{}%
\end{pgfscope}%
\begin{pgfscope}%
\pgfpathrectangle{\pgfqpoint{0.549740in}{0.463273in}}{\pgfqpoint{9.320225in}{4.495057in}}%
\pgfusepath{clip}%
\pgfsetbuttcap%
\pgfsetroundjoin%
\pgfsetlinewidth{0.000000pt}%
\definecolor{currentstroke}{rgb}{0.000000,0.000000,0.000000}%
\pgfsetstrokecolor{currentstroke}%
\pgfsetdash{}{0pt}%
\pgfpathmoveto{\pgfqpoint{6.695237in}{3.037715in}}%
\pgfpathlineto{\pgfqpoint{6.881464in}{3.037715in}}%
\pgfpathlineto{\pgfqpoint{6.881464in}{3.119443in}}%
\pgfpathlineto{\pgfqpoint{6.695237in}{3.119443in}}%
\pgfpathlineto{\pgfqpoint{6.695237in}{3.037715in}}%
\pgfusepath{}%
\end{pgfscope}%
\begin{pgfscope}%
\pgfpathrectangle{\pgfqpoint{0.549740in}{0.463273in}}{\pgfqpoint{9.320225in}{4.495057in}}%
\pgfusepath{clip}%
\pgfsetbuttcap%
\pgfsetroundjoin%
\pgfsetlinewidth{0.000000pt}%
\definecolor{currentstroke}{rgb}{0.000000,0.000000,0.000000}%
\pgfsetstrokecolor{currentstroke}%
\pgfsetdash{}{0pt}%
\pgfpathmoveto{\pgfqpoint{6.881464in}{3.037715in}}%
\pgfpathlineto{\pgfqpoint{7.067690in}{3.037715in}}%
\pgfpathlineto{\pgfqpoint{7.067690in}{3.119443in}}%
\pgfpathlineto{\pgfqpoint{6.881464in}{3.119443in}}%
\pgfpathlineto{\pgfqpoint{6.881464in}{3.037715in}}%
\pgfusepath{}%
\end{pgfscope}%
\begin{pgfscope}%
\pgfpathrectangle{\pgfqpoint{0.549740in}{0.463273in}}{\pgfqpoint{9.320225in}{4.495057in}}%
\pgfusepath{clip}%
\pgfsetbuttcap%
\pgfsetroundjoin%
\definecolor{currentfill}{rgb}{0.472869,0.711325,0.955316}%
\pgfsetfillcolor{currentfill}%
\pgfsetlinewidth{0.000000pt}%
\definecolor{currentstroke}{rgb}{0.000000,0.000000,0.000000}%
\pgfsetstrokecolor{currentstroke}%
\pgfsetdash{}{0pt}%
\pgfpathmoveto{\pgfqpoint{7.067690in}{3.037715in}}%
\pgfpathlineto{\pgfqpoint{7.253917in}{3.037715in}}%
\pgfpathlineto{\pgfqpoint{7.253917in}{3.119443in}}%
\pgfpathlineto{\pgfqpoint{7.067690in}{3.119443in}}%
\pgfpathlineto{\pgfqpoint{7.067690in}{3.037715in}}%
\pgfusepath{fill}%
\end{pgfscope}%
\begin{pgfscope}%
\pgfpathrectangle{\pgfqpoint{0.549740in}{0.463273in}}{\pgfqpoint{9.320225in}{4.495057in}}%
\pgfusepath{clip}%
\pgfsetbuttcap%
\pgfsetroundjoin%
\pgfsetlinewidth{0.000000pt}%
\definecolor{currentstroke}{rgb}{0.000000,0.000000,0.000000}%
\pgfsetstrokecolor{currentstroke}%
\pgfsetdash{}{0pt}%
\pgfpathmoveto{\pgfqpoint{7.253917in}{3.037715in}}%
\pgfpathlineto{\pgfqpoint{7.440143in}{3.037715in}}%
\pgfpathlineto{\pgfqpoint{7.440143in}{3.119443in}}%
\pgfpathlineto{\pgfqpoint{7.253917in}{3.119443in}}%
\pgfpathlineto{\pgfqpoint{7.253917in}{3.037715in}}%
\pgfusepath{}%
\end{pgfscope}%
\begin{pgfscope}%
\pgfpathrectangle{\pgfqpoint{0.549740in}{0.463273in}}{\pgfqpoint{9.320225in}{4.495057in}}%
\pgfusepath{clip}%
\pgfsetbuttcap%
\pgfsetroundjoin%
\pgfsetlinewidth{0.000000pt}%
\definecolor{currentstroke}{rgb}{0.000000,0.000000,0.000000}%
\pgfsetstrokecolor{currentstroke}%
\pgfsetdash{}{0pt}%
\pgfpathmoveto{\pgfqpoint{7.440143in}{3.037715in}}%
\pgfpathlineto{\pgfqpoint{7.626370in}{3.037715in}}%
\pgfpathlineto{\pgfqpoint{7.626370in}{3.119443in}}%
\pgfpathlineto{\pgfqpoint{7.440143in}{3.119443in}}%
\pgfpathlineto{\pgfqpoint{7.440143in}{3.037715in}}%
\pgfusepath{}%
\end{pgfscope}%
\begin{pgfscope}%
\pgfpathrectangle{\pgfqpoint{0.549740in}{0.463273in}}{\pgfqpoint{9.320225in}{4.495057in}}%
\pgfusepath{clip}%
\pgfsetbuttcap%
\pgfsetroundjoin%
\pgfsetlinewidth{0.000000pt}%
\definecolor{currentstroke}{rgb}{0.000000,0.000000,0.000000}%
\pgfsetstrokecolor{currentstroke}%
\pgfsetdash{}{0pt}%
\pgfpathmoveto{\pgfqpoint{7.626370in}{3.037715in}}%
\pgfpathlineto{\pgfqpoint{7.812596in}{3.037715in}}%
\pgfpathlineto{\pgfqpoint{7.812596in}{3.119443in}}%
\pgfpathlineto{\pgfqpoint{7.626370in}{3.119443in}}%
\pgfpathlineto{\pgfqpoint{7.626370in}{3.037715in}}%
\pgfusepath{}%
\end{pgfscope}%
\begin{pgfscope}%
\pgfpathrectangle{\pgfqpoint{0.549740in}{0.463273in}}{\pgfqpoint{9.320225in}{4.495057in}}%
\pgfusepath{clip}%
\pgfsetbuttcap%
\pgfsetroundjoin%
\pgfsetlinewidth{0.000000pt}%
\definecolor{currentstroke}{rgb}{0.000000,0.000000,0.000000}%
\pgfsetstrokecolor{currentstroke}%
\pgfsetdash{}{0pt}%
\pgfpathmoveto{\pgfqpoint{7.812596in}{3.037715in}}%
\pgfpathlineto{\pgfqpoint{7.998823in}{3.037715in}}%
\pgfpathlineto{\pgfqpoint{7.998823in}{3.119443in}}%
\pgfpathlineto{\pgfqpoint{7.812596in}{3.119443in}}%
\pgfpathlineto{\pgfqpoint{7.812596in}{3.037715in}}%
\pgfusepath{}%
\end{pgfscope}%
\begin{pgfscope}%
\pgfpathrectangle{\pgfqpoint{0.549740in}{0.463273in}}{\pgfqpoint{9.320225in}{4.495057in}}%
\pgfusepath{clip}%
\pgfsetbuttcap%
\pgfsetroundjoin%
\pgfsetlinewidth{0.000000pt}%
\definecolor{currentstroke}{rgb}{0.000000,0.000000,0.000000}%
\pgfsetstrokecolor{currentstroke}%
\pgfsetdash{}{0pt}%
\pgfpathmoveto{\pgfqpoint{7.998823in}{3.037715in}}%
\pgfpathlineto{\pgfqpoint{8.185049in}{3.037715in}}%
\pgfpathlineto{\pgfqpoint{8.185049in}{3.119443in}}%
\pgfpathlineto{\pgfqpoint{7.998823in}{3.119443in}}%
\pgfpathlineto{\pgfqpoint{7.998823in}{3.037715in}}%
\pgfusepath{}%
\end{pgfscope}%
\begin{pgfscope}%
\pgfpathrectangle{\pgfqpoint{0.549740in}{0.463273in}}{\pgfqpoint{9.320225in}{4.495057in}}%
\pgfusepath{clip}%
\pgfsetbuttcap%
\pgfsetroundjoin%
\definecolor{currentfill}{rgb}{0.472869,0.711325,0.955316}%
\pgfsetfillcolor{currentfill}%
\pgfsetlinewidth{0.000000pt}%
\definecolor{currentstroke}{rgb}{0.000000,0.000000,0.000000}%
\pgfsetstrokecolor{currentstroke}%
\pgfsetdash{}{0pt}%
\pgfpathmoveto{\pgfqpoint{8.185049in}{3.037715in}}%
\pgfpathlineto{\pgfqpoint{8.371276in}{3.037715in}}%
\pgfpathlineto{\pgfqpoint{8.371276in}{3.119443in}}%
\pgfpathlineto{\pgfqpoint{8.185049in}{3.119443in}}%
\pgfpathlineto{\pgfqpoint{8.185049in}{3.037715in}}%
\pgfusepath{fill}%
\end{pgfscope}%
\begin{pgfscope}%
\pgfpathrectangle{\pgfqpoint{0.549740in}{0.463273in}}{\pgfqpoint{9.320225in}{4.495057in}}%
\pgfusepath{clip}%
\pgfsetbuttcap%
\pgfsetroundjoin%
\pgfsetlinewidth{0.000000pt}%
\definecolor{currentstroke}{rgb}{0.000000,0.000000,0.000000}%
\pgfsetstrokecolor{currentstroke}%
\pgfsetdash{}{0pt}%
\pgfpathmoveto{\pgfqpoint{8.371276in}{3.037715in}}%
\pgfpathlineto{\pgfqpoint{8.557503in}{3.037715in}}%
\pgfpathlineto{\pgfqpoint{8.557503in}{3.119443in}}%
\pgfpathlineto{\pgfqpoint{8.371276in}{3.119443in}}%
\pgfpathlineto{\pgfqpoint{8.371276in}{3.037715in}}%
\pgfusepath{}%
\end{pgfscope}%
\begin{pgfscope}%
\pgfpathrectangle{\pgfqpoint{0.549740in}{0.463273in}}{\pgfqpoint{9.320225in}{4.495057in}}%
\pgfusepath{clip}%
\pgfsetbuttcap%
\pgfsetroundjoin%
\pgfsetlinewidth{0.000000pt}%
\definecolor{currentstroke}{rgb}{0.000000,0.000000,0.000000}%
\pgfsetstrokecolor{currentstroke}%
\pgfsetdash{}{0pt}%
\pgfpathmoveto{\pgfqpoint{8.557503in}{3.037715in}}%
\pgfpathlineto{\pgfqpoint{8.743729in}{3.037715in}}%
\pgfpathlineto{\pgfqpoint{8.743729in}{3.119443in}}%
\pgfpathlineto{\pgfqpoint{8.557503in}{3.119443in}}%
\pgfpathlineto{\pgfqpoint{8.557503in}{3.037715in}}%
\pgfusepath{}%
\end{pgfscope}%
\begin{pgfscope}%
\pgfpathrectangle{\pgfqpoint{0.549740in}{0.463273in}}{\pgfqpoint{9.320225in}{4.495057in}}%
\pgfusepath{clip}%
\pgfsetbuttcap%
\pgfsetroundjoin%
\pgfsetlinewidth{0.000000pt}%
\definecolor{currentstroke}{rgb}{0.000000,0.000000,0.000000}%
\pgfsetstrokecolor{currentstroke}%
\pgfsetdash{}{0pt}%
\pgfpathmoveto{\pgfqpoint{8.743729in}{3.037715in}}%
\pgfpathlineto{\pgfqpoint{8.929956in}{3.037715in}}%
\pgfpathlineto{\pgfqpoint{8.929956in}{3.119443in}}%
\pgfpathlineto{\pgfqpoint{8.743729in}{3.119443in}}%
\pgfpathlineto{\pgfqpoint{8.743729in}{3.037715in}}%
\pgfusepath{}%
\end{pgfscope}%
\begin{pgfscope}%
\pgfpathrectangle{\pgfqpoint{0.549740in}{0.463273in}}{\pgfqpoint{9.320225in}{4.495057in}}%
\pgfusepath{clip}%
\pgfsetbuttcap%
\pgfsetroundjoin%
\pgfsetlinewidth{0.000000pt}%
\definecolor{currentstroke}{rgb}{0.000000,0.000000,0.000000}%
\pgfsetstrokecolor{currentstroke}%
\pgfsetdash{}{0pt}%
\pgfpathmoveto{\pgfqpoint{8.929956in}{3.037715in}}%
\pgfpathlineto{\pgfqpoint{9.116182in}{3.037715in}}%
\pgfpathlineto{\pgfqpoint{9.116182in}{3.119443in}}%
\pgfpathlineto{\pgfqpoint{8.929956in}{3.119443in}}%
\pgfpathlineto{\pgfqpoint{8.929956in}{3.037715in}}%
\pgfusepath{}%
\end{pgfscope}%
\begin{pgfscope}%
\pgfpathrectangle{\pgfqpoint{0.549740in}{0.463273in}}{\pgfqpoint{9.320225in}{4.495057in}}%
\pgfusepath{clip}%
\pgfsetbuttcap%
\pgfsetroundjoin%
\pgfsetlinewidth{0.000000pt}%
\definecolor{currentstroke}{rgb}{0.000000,0.000000,0.000000}%
\pgfsetstrokecolor{currentstroke}%
\pgfsetdash{}{0pt}%
\pgfpathmoveto{\pgfqpoint{9.116182in}{3.037715in}}%
\pgfpathlineto{\pgfqpoint{9.302409in}{3.037715in}}%
\pgfpathlineto{\pgfqpoint{9.302409in}{3.119443in}}%
\pgfpathlineto{\pgfqpoint{9.116182in}{3.119443in}}%
\pgfpathlineto{\pgfqpoint{9.116182in}{3.037715in}}%
\pgfusepath{}%
\end{pgfscope}%
\begin{pgfscope}%
\pgfpathrectangle{\pgfqpoint{0.549740in}{0.463273in}}{\pgfqpoint{9.320225in}{4.495057in}}%
\pgfusepath{clip}%
\pgfsetbuttcap%
\pgfsetroundjoin%
\pgfsetlinewidth{0.000000pt}%
\definecolor{currentstroke}{rgb}{0.000000,0.000000,0.000000}%
\pgfsetstrokecolor{currentstroke}%
\pgfsetdash{}{0pt}%
\pgfpathmoveto{\pgfqpoint{9.302409in}{3.037715in}}%
\pgfpathlineto{\pgfqpoint{9.488635in}{3.037715in}}%
\pgfpathlineto{\pgfqpoint{9.488635in}{3.119443in}}%
\pgfpathlineto{\pgfqpoint{9.302409in}{3.119443in}}%
\pgfpathlineto{\pgfqpoint{9.302409in}{3.037715in}}%
\pgfusepath{}%
\end{pgfscope}%
\begin{pgfscope}%
\pgfpathrectangle{\pgfqpoint{0.549740in}{0.463273in}}{\pgfqpoint{9.320225in}{4.495057in}}%
\pgfusepath{clip}%
\pgfsetbuttcap%
\pgfsetroundjoin%
\definecolor{currentfill}{rgb}{0.472869,0.711325,0.955316}%
\pgfsetfillcolor{currentfill}%
\pgfsetlinewidth{0.000000pt}%
\definecolor{currentstroke}{rgb}{0.000000,0.000000,0.000000}%
\pgfsetstrokecolor{currentstroke}%
\pgfsetdash{}{0pt}%
\pgfpathmoveto{\pgfqpoint{9.488635in}{3.037715in}}%
\pgfpathlineto{\pgfqpoint{9.674862in}{3.037715in}}%
\pgfpathlineto{\pgfqpoint{9.674862in}{3.119443in}}%
\pgfpathlineto{\pgfqpoint{9.488635in}{3.119443in}}%
\pgfpathlineto{\pgfqpoint{9.488635in}{3.037715in}}%
\pgfusepath{fill}%
\end{pgfscope}%
\begin{pgfscope}%
\pgfpathrectangle{\pgfqpoint{0.549740in}{0.463273in}}{\pgfqpoint{9.320225in}{4.495057in}}%
\pgfusepath{clip}%
\pgfsetbuttcap%
\pgfsetroundjoin%
\pgfsetlinewidth{0.000000pt}%
\definecolor{currentstroke}{rgb}{0.000000,0.000000,0.000000}%
\pgfsetstrokecolor{currentstroke}%
\pgfsetdash{}{0pt}%
\pgfpathmoveto{\pgfqpoint{9.674862in}{3.037715in}}%
\pgfpathlineto{\pgfqpoint{9.861088in}{3.037715in}}%
\pgfpathlineto{\pgfqpoint{9.861088in}{3.119443in}}%
\pgfpathlineto{\pgfqpoint{9.674862in}{3.119443in}}%
\pgfpathlineto{\pgfqpoint{9.674862in}{3.037715in}}%
\pgfusepath{}%
\end{pgfscope}%
\begin{pgfscope}%
\pgfpathrectangle{\pgfqpoint{0.549740in}{0.463273in}}{\pgfqpoint{9.320225in}{4.495057in}}%
\pgfusepath{clip}%
\pgfsetbuttcap%
\pgfsetroundjoin%
\pgfsetlinewidth{0.000000pt}%
\definecolor{currentstroke}{rgb}{0.000000,0.000000,0.000000}%
\pgfsetstrokecolor{currentstroke}%
\pgfsetdash{}{0pt}%
\pgfpathmoveto{\pgfqpoint{0.549761in}{3.119443in}}%
\pgfpathlineto{\pgfqpoint{0.735988in}{3.119443in}}%
\pgfpathlineto{\pgfqpoint{0.735988in}{3.201171in}}%
\pgfpathlineto{\pgfqpoint{0.549761in}{3.201171in}}%
\pgfpathlineto{\pgfqpoint{0.549761in}{3.119443in}}%
\pgfusepath{}%
\end{pgfscope}%
\begin{pgfscope}%
\pgfpathrectangle{\pgfqpoint{0.549740in}{0.463273in}}{\pgfqpoint{9.320225in}{4.495057in}}%
\pgfusepath{clip}%
\pgfsetbuttcap%
\pgfsetroundjoin%
\pgfsetlinewidth{0.000000pt}%
\definecolor{currentstroke}{rgb}{0.000000,0.000000,0.000000}%
\pgfsetstrokecolor{currentstroke}%
\pgfsetdash{}{0pt}%
\pgfpathmoveto{\pgfqpoint{0.735988in}{3.119443in}}%
\pgfpathlineto{\pgfqpoint{0.922214in}{3.119443in}}%
\pgfpathlineto{\pgfqpoint{0.922214in}{3.201171in}}%
\pgfpathlineto{\pgfqpoint{0.735988in}{3.201171in}}%
\pgfpathlineto{\pgfqpoint{0.735988in}{3.119443in}}%
\pgfusepath{}%
\end{pgfscope}%
\begin{pgfscope}%
\pgfpathrectangle{\pgfqpoint{0.549740in}{0.463273in}}{\pgfqpoint{9.320225in}{4.495057in}}%
\pgfusepath{clip}%
\pgfsetbuttcap%
\pgfsetroundjoin%
\pgfsetlinewidth{0.000000pt}%
\definecolor{currentstroke}{rgb}{0.000000,0.000000,0.000000}%
\pgfsetstrokecolor{currentstroke}%
\pgfsetdash{}{0pt}%
\pgfpathmoveto{\pgfqpoint{0.922214in}{3.119443in}}%
\pgfpathlineto{\pgfqpoint{1.108441in}{3.119443in}}%
\pgfpathlineto{\pgfqpoint{1.108441in}{3.201171in}}%
\pgfpathlineto{\pgfqpoint{0.922214in}{3.201171in}}%
\pgfpathlineto{\pgfqpoint{0.922214in}{3.119443in}}%
\pgfusepath{}%
\end{pgfscope}%
\begin{pgfscope}%
\pgfpathrectangle{\pgfqpoint{0.549740in}{0.463273in}}{\pgfqpoint{9.320225in}{4.495057in}}%
\pgfusepath{clip}%
\pgfsetbuttcap%
\pgfsetroundjoin%
\pgfsetlinewidth{0.000000pt}%
\definecolor{currentstroke}{rgb}{0.000000,0.000000,0.000000}%
\pgfsetstrokecolor{currentstroke}%
\pgfsetdash{}{0pt}%
\pgfpathmoveto{\pgfqpoint{1.108441in}{3.119443in}}%
\pgfpathlineto{\pgfqpoint{1.294667in}{3.119443in}}%
\pgfpathlineto{\pgfqpoint{1.294667in}{3.201171in}}%
\pgfpathlineto{\pgfqpoint{1.108441in}{3.201171in}}%
\pgfpathlineto{\pgfqpoint{1.108441in}{3.119443in}}%
\pgfusepath{}%
\end{pgfscope}%
\begin{pgfscope}%
\pgfpathrectangle{\pgfqpoint{0.549740in}{0.463273in}}{\pgfqpoint{9.320225in}{4.495057in}}%
\pgfusepath{clip}%
\pgfsetbuttcap%
\pgfsetroundjoin%
\pgfsetlinewidth{0.000000pt}%
\definecolor{currentstroke}{rgb}{0.000000,0.000000,0.000000}%
\pgfsetstrokecolor{currentstroke}%
\pgfsetdash{}{0pt}%
\pgfpathmoveto{\pgfqpoint{1.294667in}{3.119443in}}%
\pgfpathlineto{\pgfqpoint{1.480894in}{3.119443in}}%
\pgfpathlineto{\pgfqpoint{1.480894in}{3.201171in}}%
\pgfpathlineto{\pgfqpoint{1.294667in}{3.201171in}}%
\pgfpathlineto{\pgfqpoint{1.294667in}{3.119443in}}%
\pgfusepath{}%
\end{pgfscope}%
\begin{pgfscope}%
\pgfpathrectangle{\pgfqpoint{0.549740in}{0.463273in}}{\pgfqpoint{9.320225in}{4.495057in}}%
\pgfusepath{clip}%
\pgfsetbuttcap%
\pgfsetroundjoin%
\pgfsetlinewidth{0.000000pt}%
\definecolor{currentstroke}{rgb}{0.000000,0.000000,0.000000}%
\pgfsetstrokecolor{currentstroke}%
\pgfsetdash{}{0pt}%
\pgfpathmoveto{\pgfqpoint{1.480894in}{3.119443in}}%
\pgfpathlineto{\pgfqpoint{1.667120in}{3.119443in}}%
\pgfpathlineto{\pgfqpoint{1.667120in}{3.201171in}}%
\pgfpathlineto{\pgfqpoint{1.480894in}{3.201171in}}%
\pgfpathlineto{\pgfqpoint{1.480894in}{3.119443in}}%
\pgfusepath{}%
\end{pgfscope}%
\begin{pgfscope}%
\pgfpathrectangle{\pgfqpoint{0.549740in}{0.463273in}}{\pgfqpoint{9.320225in}{4.495057in}}%
\pgfusepath{clip}%
\pgfsetbuttcap%
\pgfsetroundjoin%
\pgfsetlinewidth{0.000000pt}%
\definecolor{currentstroke}{rgb}{0.000000,0.000000,0.000000}%
\pgfsetstrokecolor{currentstroke}%
\pgfsetdash{}{0pt}%
\pgfpathmoveto{\pgfqpoint{1.667120in}{3.119443in}}%
\pgfpathlineto{\pgfqpoint{1.853347in}{3.119443in}}%
\pgfpathlineto{\pgfqpoint{1.853347in}{3.201171in}}%
\pgfpathlineto{\pgfqpoint{1.667120in}{3.201171in}}%
\pgfpathlineto{\pgfqpoint{1.667120in}{3.119443in}}%
\pgfusepath{}%
\end{pgfscope}%
\begin{pgfscope}%
\pgfpathrectangle{\pgfqpoint{0.549740in}{0.463273in}}{\pgfqpoint{9.320225in}{4.495057in}}%
\pgfusepath{clip}%
\pgfsetbuttcap%
\pgfsetroundjoin%
\pgfsetlinewidth{0.000000pt}%
\definecolor{currentstroke}{rgb}{0.000000,0.000000,0.000000}%
\pgfsetstrokecolor{currentstroke}%
\pgfsetdash{}{0pt}%
\pgfpathmoveto{\pgfqpoint{1.853347in}{3.119443in}}%
\pgfpathlineto{\pgfqpoint{2.039573in}{3.119443in}}%
\pgfpathlineto{\pgfqpoint{2.039573in}{3.201171in}}%
\pgfpathlineto{\pgfqpoint{1.853347in}{3.201171in}}%
\pgfpathlineto{\pgfqpoint{1.853347in}{3.119443in}}%
\pgfusepath{}%
\end{pgfscope}%
\begin{pgfscope}%
\pgfpathrectangle{\pgfqpoint{0.549740in}{0.463273in}}{\pgfqpoint{9.320225in}{4.495057in}}%
\pgfusepath{clip}%
\pgfsetbuttcap%
\pgfsetroundjoin%
\pgfsetlinewidth{0.000000pt}%
\definecolor{currentstroke}{rgb}{0.000000,0.000000,0.000000}%
\pgfsetstrokecolor{currentstroke}%
\pgfsetdash{}{0pt}%
\pgfpathmoveto{\pgfqpoint{2.039573in}{3.119443in}}%
\pgfpathlineto{\pgfqpoint{2.225800in}{3.119443in}}%
\pgfpathlineto{\pgfqpoint{2.225800in}{3.201171in}}%
\pgfpathlineto{\pgfqpoint{2.039573in}{3.201171in}}%
\pgfpathlineto{\pgfqpoint{2.039573in}{3.119443in}}%
\pgfusepath{}%
\end{pgfscope}%
\begin{pgfscope}%
\pgfpathrectangle{\pgfqpoint{0.549740in}{0.463273in}}{\pgfqpoint{9.320225in}{4.495057in}}%
\pgfusepath{clip}%
\pgfsetbuttcap%
\pgfsetroundjoin%
\pgfsetlinewidth{0.000000pt}%
\definecolor{currentstroke}{rgb}{0.000000,0.000000,0.000000}%
\pgfsetstrokecolor{currentstroke}%
\pgfsetdash{}{0pt}%
\pgfpathmoveto{\pgfqpoint{2.225800in}{3.119443in}}%
\pgfpathlineto{\pgfqpoint{2.412027in}{3.119443in}}%
\pgfpathlineto{\pgfqpoint{2.412027in}{3.201171in}}%
\pgfpathlineto{\pgfqpoint{2.225800in}{3.201171in}}%
\pgfpathlineto{\pgfqpoint{2.225800in}{3.119443in}}%
\pgfusepath{}%
\end{pgfscope}%
\begin{pgfscope}%
\pgfpathrectangle{\pgfqpoint{0.549740in}{0.463273in}}{\pgfqpoint{9.320225in}{4.495057in}}%
\pgfusepath{clip}%
\pgfsetbuttcap%
\pgfsetroundjoin%
\pgfsetlinewidth{0.000000pt}%
\definecolor{currentstroke}{rgb}{0.000000,0.000000,0.000000}%
\pgfsetstrokecolor{currentstroke}%
\pgfsetdash{}{0pt}%
\pgfpathmoveto{\pgfqpoint{2.412027in}{3.119443in}}%
\pgfpathlineto{\pgfqpoint{2.598253in}{3.119443in}}%
\pgfpathlineto{\pgfqpoint{2.598253in}{3.201171in}}%
\pgfpathlineto{\pgfqpoint{2.412027in}{3.201171in}}%
\pgfpathlineto{\pgfqpoint{2.412027in}{3.119443in}}%
\pgfusepath{}%
\end{pgfscope}%
\begin{pgfscope}%
\pgfpathrectangle{\pgfqpoint{0.549740in}{0.463273in}}{\pgfqpoint{9.320225in}{4.495057in}}%
\pgfusepath{clip}%
\pgfsetbuttcap%
\pgfsetroundjoin%
\pgfsetlinewidth{0.000000pt}%
\definecolor{currentstroke}{rgb}{0.000000,0.000000,0.000000}%
\pgfsetstrokecolor{currentstroke}%
\pgfsetdash{}{0pt}%
\pgfpathmoveto{\pgfqpoint{2.598253in}{3.119443in}}%
\pgfpathlineto{\pgfqpoint{2.784480in}{3.119443in}}%
\pgfpathlineto{\pgfqpoint{2.784480in}{3.201171in}}%
\pgfpathlineto{\pgfqpoint{2.598253in}{3.201171in}}%
\pgfpathlineto{\pgfqpoint{2.598253in}{3.119443in}}%
\pgfusepath{}%
\end{pgfscope}%
\begin{pgfscope}%
\pgfpathrectangle{\pgfqpoint{0.549740in}{0.463273in}}{\pgfqpoint{9.320225in}{4.495057in}}%
\pgfusepath{clip}%
\pgfsetbuttcap%
\pgfsetroundjoin%
\pgfsetlinewidth{0.000000pt}%
\definecolor{currentstroke}{rgb}{0.000000,0.000000,0.000000}%
\pgfsetstrokecolor{currentstroke}%
\pgfsetdash{}{0pt}%
\pgfpathmoveto{\pgfqpoint{2.784480in}{3.119443in}}%
\pgfpathlineto{\pgfqpoint{2.970706in}{3.119443in}}%
\pgfpathlineto{\pgfqpoint{2.970706in}{3.201171in}}%
\pgfpathlineto{\pgfqpoint{2.784480in}{3.201171in}}%
\pgfpathlineto{\pgfqpoint{2.784480in}{3.119443in}}%
\pgfusepath{}%
\end{pgfscope}%
\begin{pgfscope}%
\pgfpathrectangle{\pgfqpoint{0.549740in}{0.463273in}}{\pgfqpoint{9.320225in}{4.495057in}}%
\pgfusepath{clip}%
\pgfsetbuttcap%
\pgfsetroundjoin%
\pgfsetlinewidth{0.000000pt}%
\definecolor{currentstroke}{rgb}{0.000000,0.000000,0.000000}%
\pgfsetstrokecolor{currentstroke}%
\pgfsetdash{}{0pt}%
\pgfpathmoveto{\pgfqpoint{2.970706in}{3.119443in}}%
\pgfpathlineto{\pgfqpoint{3.156933in}{3.119443in}}%
\pgfpathlineto{\pgfqpoint{3.156933in}{3.201171in}}%
\pgfpathlineto{\pgfqpoint{2.970706in}{3.201171in}}%
\pgfpathlineto{\pgfqpoint{2.970706in}{3.119443in}}%
\pgfusepath{}%
\end{pgfscope}%
\begin{pgfscope}%
\pgfpathrectangle{\pgfqpoint{0.549740in}{0.463273in}}{\pgfqpoint{9.320225in}{4.495057in}}%
\pgfusepath{clip}%
\pgfsetbuttcap%
\pgfsetroundjoin%
\pgfsetlinewidth{0.000000pt}%
\definecolor{currentstroke}{rgb}{0.000000,0.000000,0.000000}%
\pgfsetstrokecolor{currentstroke}%
\pgfsetdash{}{0pt}%
\pgfpathmoveto{\pgfqpoint{3.156933in}{3.119443in}}%
\pgfpathlineto{\pgfqpoint{3.343159in}{3.119443in}}%
\pgfpathlineto{\pgfqpoint{3.343159in}{3.201171in}}%
\pgfpathlineto{\pgfqpoint{3.156933in}{3.201171in}}%
\pgfpathlineto{\pgfqpoint{3.156933in}{3.119443in}}%
\pgfusepath{}%
\end{pgfscope}%
\begin{pgfscope}%
\pgfpathrectangle{\pgfqpoint{0.549740in}{0.463273in}}{\pgfqpoint{9.320225in}{4.495057in}}%
\pgfusepath{clip}%
\pgfsetbuttcap%
\pgfsetroundjoin%
\pgfsetlinewidth{0.000000pt}%
\definecolor{currentstroke}{rgb}{0.000000,0.000000,0.000000}%
\pgfsetstrokecolor{currentstroke}%
\pgfsetdash{}{0pt}%
\pgfpathmoveto{\pgfqpoint{3.343159in}{3.119443in}}%
\pgfpathlineto{\pgfqpoint{3.529386in}{3.119443in}}%
\pgfpathlineto{\pgfqpoint{3.529386in}{3.201171in}}%
\pgfpathlineto{\pgfqpoint{3.343159in}{3.201171in}}%
\pgfpathlineto{\pgfqpoint{3.343159in}{3.119443in}}%
\pgfusepath{}%
\end{pgfscope}%
\begin{pgfscope}%
\pgfpathrectangle{\pgfqpoint{0.549740in}{0.463273in}}{\pgfqpoint{9.320225in}{4.495057in}}%
\pgfusepath{clip}%
\pgfsetbuttcap%
\pgfsetroundjoin%
\pgfsetlinewidth{0.000000pt}%
\definecolor{currentstroke}{rgb}{0.000000,0.000000,0.000000}%
\pgfsetstrokecolor{currentstroke}%
\pgfsetdash{}{0pt}%
\pgfpathmoveto{\pgfqpoint{3.529386in}{3.119443in}}%
\pgfpathlineto{\pgfqpoint{3.715612in}{3.119443in}}%
\pgfpathlineto{\pgfqpoint{3.715612in}{3.201171in}}%
\pgfpathlineto{\pgfqpoint{3.529386in}{3.201171in}}%
\pgfpathlineto{\pgfqpoint{3.529386in}{3.119443in}}%
\pgfusepath{}%
\end{pgfscope}%
\begin{pgfscope}%
\pgfpathrectangle{\pgfqpoint{0.549740in}{0.463273in}}{\pgfqpoint{9.320225in}{4.495057in}}%
\pgfusepath{clip}%
\pgfsetbuttcap%
\pgfsetroundjoin%
\pgfsetlinewidth{0.000000pt}%
\definecolor{currentstroke}{rgb}{0.000000,0.000000,0.000000}%
\pgfsetstrokecolor{currentstroke}%
\pgfsetdash{}{0pt}%
\pgfpathmoveto{\pgfqpoint{3.715612in}{3.119443in}}%
\pgfpathlineto{\pgfqpoint{3.901839in}{3.119443in}}%
\pgfpathlineto{\pgfqpoint{3.901839in}{3.201171in}}%
\pgfpathlineto{\pgfqpoint{3.715612in}{3.201171in}}%
\pgfpathlineto{\pgfqpoint{3.715612in}{3.119443in}}%
\pgfusepath{}%
\end{pgfscope}%
\begin{pgfscope}%
\pgfpathrectangle{\pgfqpoint{0.549740in}{0.463273in}}{\pgfqpoint{9.320225in}{4.495057in}}%
\pgfusepath{clip}%
\pgfsetbuttcap%
\pgfsetroundjoin%
\pgfsetlinewidth{0.000000pt}%
\definecolor{currentstroke}{rgb}{0.000000,0.000000,0.000000}%
\pgfsetstrokecolor{currentstroke}%
\pgfsetdash{}{0pt}%
\pgfpathmoveto{\pgfqpoint{3.901839in}{3.119443in}}%
\pgfpathlineto{\pgfqpoint{4.088065in}{3.119443in}}%
\pgfpathlineto{\pgfqpoint{4.088065in}{3.201171in}}%
\pgfpathlineto{\pgfqpoint{3.901839in}{3.201171in}}%
\pgfpathlineto{\pgfqpoint{3.901839in}{3.119443in}}%
\pgfusepath{}%
\end{pgfscope}%
\begin{pgfscope}%
\pgfpathrectangle{\pgfqpoint{0.549740in}{0.463273in}}{\pgfqpoint{9.320225in}{4.495057in}}%
\pgfusepath{clip}%
\pgfsetbuttcap%
\pgfsetroundjoin%
\pgfsetlinewidth{0.000000pt}%
\definecolor{currentstroke}{rgb}{0.000000,0.000000,0.000000}%
\pgfsetstrokecolor{currentstroke}%
\pgfsetdash{}{0pt}%
\pgfpathmoveto{\pgfqpoint{4.088065in}{3.119443in}}%
\pgfpathlineto{\pgfqpoint{4.274292in}{3.119443in}}%
\pgfpathlineto{\pgfqpoint{4.274292in}{3.201171in}}%
\pgfpathlineto{\pgfqpoint{4.088065in}{3.201171in}}%
\pgfpathlineto{\pgfqpoint{4.088065in}{3.119443in}}%
\pgfusepath{}%
\end{pgfscope}%
\begin{pgfscope}%
\pgfpathrectangle{\pgfqpoint{0.549740in}{0.463273in}}{\pgfqpoint{9.320225in}{4.495057in}}%
\pgfusepath{clip}%
\pgfsetbuttcap%
\pgfsetroundjoin%
\pgfsetlinewidth{0.000000pt}%
\definecolor{currentstroke}{rgb}{0.000000,0.000000,0.000000}%
\pgfsetstrokecolor{currentstroke}%
\pgfsetdash{}{0pt}%
\pgfpathmoveto{\pgfqpoint{4.274292in}{3.119443in}}%
\pgfpathlineto{\pgfqpoint{4.460519in}{3.119443in}}%
\pgfpathlineto{\pgfqpoint{4.460519in}{3.201171in}}%
\pgfpathlineto{\pgfqpoint{4.274292in}{3.201171in}}%
\pgfpathlineto{\pgfqpoint{4.274292in}{3.119443in}}%
\pgfusepath{}%
\end{pgfscope}%
\begin{pgfscope}%
\pgfpathrectangle{\pgfqpoint{0.549740in}{0.463273in}}{\pgfqpoint{9.320225in}{4.495057in}}%
\pgfusepath{clip}%
\pgfsetbuttcap%
\pgfsetroundjoin%
\pgfsetlinewidth{0.000000pt}%
\definecolor{currentstroke}{rgb}{0.000000,0.000000,0.000000}%
\pgfsetstrokecolor{currentstroke}%
\pgfsetdash{}{0pt}%
\pgfpathmoveto{\pgfqpoint{4.460519in}{3.119443in}}%
\pgfpathlineto{\pgfqpoint{4.646745in}{3.119443in}}%
\pgfpathlineto{\pgfqpoint{4.646745in}{3.201171in}}%
\pgfpathlineto{\pgfqpoint{4.460519in}{3.201171in}}%
\pgfpathlineto{\pgfqpoint{4.460519in}{3.119443in}}%
\pgfusepath{}%
\end{pgfscope}%
\begin{pgfscope}%
\pgfpathrectangle{\pgfqpoint{0.549740in}{0.463273in}}{\pgfqpoint{9.320225in}{4.495057in}}%
\pgfusepath{clip}%
\pgfsetbuttcap%
\pgfsetroundjoin%
\pgfsetlinewidth{0.000000pt}%
\definecolor{currentstroke}{rgb}{0.000000,0.000000,0.000000}%
\pgfsetstrokecolor{currentstroke}%
\pgfsetdash{}{0pt}%
\pgfpathmoveto{\pgfqpoint{4.646745in}{3.119443in}}%
\pgfpathlineto{\pgfqpoint{4.832972in}{3.119443in}}%
\pgfpathlineto{\pgfqpoint{4.832972in}{3.201171in}}%
\pgfpathlineto{\pgfqpoint{4.646745in}{3.201171in}}%
\pgfpathlineto{\pgfqpoint{4.646745in}{3.119443in}}%
\pgfusepath{}%
\end{pgfscope}%
\begin{pgfscope}%
\pgfpathrectangle{\pgfqpoint{0.549740in}{0.463273in}}{\pgfqpoint{9.320225in}{4.495057in}}%
\pgfusepath{clip}%
\pgfsetbuttcap%
\pgfsetroundjoin%
\pgfsetlinewidth{0.000000pt}%
\definecolor{currentstroke}{rgb}{0.000000,0.000000,0.000000}%
\pgfsetstrokecolor{currentstroke}%
\pgfsetdash{}{0pt}%
\pgfpathmoveto{\pgfqpoint{4.832972in}{3.119443in}}%
\pgfpathlineto{\pgfqpoint{5.019198in}{3.119443in}}%
\pgfpathlineto{\pgfqpoint{5.019198in}{3.201171in}}%
\pgfpathlineto{\pgfqpoint{4.832972in}{3.201171in}}%
\pgfpathlineto{\pgfqpoint{4.832972in}{3.119443in}}%
\pgfusepath{}%
\end{pgfscope}%
\begin{pgfscope}%
\pgfpathrectangle{\pgfqpoint{0.549740in}{0.463273in}}{\pgfqpoint{9.320225in}{4.495057in}}%
\pgfusepath{clip}%
\pgfsetbuttcap%
\pgfsetroundjoin%
\definecolor{currentfill}{rgb}{0.547810,0.736432,0.947518}%
\pgfsetfillcolor{currentfill}%
\pgfsetlinewidth{0.000000pt}%
\definecolor{currentstroke}{rgb}{0.000000,0.000000,0.000000}%
\pgfsetstrokecolor{currentstroke}%
\pgfsetdash{}{0pt}%
\pgfpathmoveto{\pgfqpoint{5.019198in}{3.119443in}}%
\pgfpathlineto{\pgfqpoint{5.205425in}{3.119443in}}%
\pgfpathlineto{\pgfqpoint{5.205425in}{3.201171in}}%
\pgfpathlineto{\pgfqpoint{5.019198in}{3.201171in}}%
\pgfpathlineto{\pgfqpoint{5.019198in}{3.119443in}}%
\pgfusepath{fill}%
\end{pgfscope}%
\begin{pgfscope}%
\pgfpathrectangle{\pgfqpoint{0.549740in}{0.463273in}}{\pgfqpoint{9.320225in}{4.495057in}}%
\pgfusepath{clip}%
\pgfsetbuttcap%
\pgfsetroundjoin%
\definecolor{currentfill}{rgb}{0.614330,0.761948,0.940009}%
\pgfsetfillcolor{currentfill}%
\pgfsetlinewidth{0.000000pt}%
\definecolor{currentstroke}{rgb}{0.000000,0.000000,0.000000}%
\pgfsetstrokecolor{currentstroke}%
\pgfsetdash{}{0pt}%
\pgfpathmoveto{\pgfqpoint{5.205425in}{3.119443in}}%
\pgfpathlineto{\pgfqpoint{5.391651in}{3.119443in}}%
\pgfpathlineto{\pgfqpoint{5.391651in}{3.201171in}}%
\pgfpathlineto{\pgfqpoint{5.205425in}{3.201171in}}%
\pgfpathlineto{\pgfqpoint{5.205425in}{3.119443in}}%
\pgfusepath{fill}%
\end{pgfscope}%
\begin{pgfscope}%
\pgfpathrectangle{\pgfqpoint{0.549740in}{0.463273in}}{\pgfqpoint{9.320225in}{4.495057in}}%
\pgfusepath{clip}%
\pgfsetbuttcap%
\pgfsetroundjoin%
\pgfsetlinewidth{0.000000pt}%
\definecolor{currentstroke}{rgb}{0.000000,0.000000,0.000000}%
\pgfsetstrokecolor{currentstroke}%
\pgfsetdash{}{0pt}%
\pgfpathmoveto{\pgfqpoint{5.391651in}{3.119443in}}%
\pgfpathlineto{\pgfqpoint{5.577878in}{3.119443in}}%
\pgfpathlineto{\pgfqpoint{5.577878in}{3.201171in}}%
\pgfpathlineto{\pgfqpoint{5.391651in}{3.201171in}}%
\pgfpathlineto{\pgfqpoint{5.391651in}{3.119443in}}%
\pgfusepath{}%
\end{pgfscope}%
\begin{pgfscope}%
\pgfpathrectangle{\pgfqpoint{0.549740in}{0.463273in}}{\pgfqpoint{9.320225in}{4.495057in}}%
\pgfusepath{clip}%
\pgfsetbuttcap%
\pgfsetroundjoin%
\pgfsetlinewidth{0.000000pt}%
\definecolor{currentstroke}{rgb}{0.000000,0.000000,0.000000}%
\pgfsetstrokecolor{currentstroke}%
\pgfsetdash{}{0pt}%
\pgfpathmoveto{\pgfqpoint{5.577878in}{3.119443in}}%
\pgfpathlineto{\pgfqpoint{5.764104in}{3.119443in}}%
\pgfpathlineto{\pgfqpoint{5.764104in}{3.201171in}}%
\pgfpathlineto{\pgfqpoint{5.577878in}{3.201171in}}%
\pgfpathlineto{\pgfqpoint{5.577878in}{3.119443in}}%
\pgfusepath{}%
\end{pgfscope}%
\begin{pgfscope}%
\pgfpathrectangle{\pgfqpoint{0.549740in}{0.463273in}}{\pgfqpoint{9.320225in}{4.495057in}}%
\pgfusepath{clip}%
\pgfsetbuttcap%
\pgfsetroundjoin%
\pgfsetlinewidth{0.000000pt}%
\definecolor{currentstroke}{rgb}{0.000000,0.000000,0.000000}%
\pgfsetstrokecolor{currentstroke}%
\pgfsetdash{}{0pt}%
\pgfpathmoveto{\pgfqpoint{5.764104in}{3.119443in}}%
\pgfpathlineto{\pgfqpoint{5.950331in}{3.119443in}}%
\pgfpathlineto{\pgfqpoint{5.950331in}{3.201171in}}%
\pgfpathlineto{\pgfqpoint{5.764104in}{3.201171in}}%
\pgfpathlineto{\pgfqpoint{5.764104in}{3.119443in}}%
\pgfusepath{}%
\end{pgfscope}%
\begin{pgfscope}%
\pgfpathrectangle{\pgfqpoint{0.549740in}{0.463273in}}{\pgfqpoint{9.320225in}{4.495057in}}%
\pgfusepath{clip}%
\pgfsetbuttcap%
\pgfsetroundjoin%
\definecolor{currentfill}{rgb}{0.472869,0.711325,0.955316}%
\pgfsetfillcolor{currentfill}%
\pgfsetlinewidth{0.000000pt}%
\definecolor{currentstroke}{rgb}{0.000000,0.000000,0.000000}%
\pgfsetstrokecolor{currentstroke}%
\pgfsetdash{}{0pt}%
\pgfpathmoveto{\pgfqpoint{5.950331in}{3.119443in}}%
\pgfpathlineto{\pgfqpoint{6.136557in}{3.119443in}}%
\pgfpathlineto{\pgfqpoint{6.136557in}{3.201171in}}%
\pgfpathlineto{\pgfqpoint{5.950331in}{3.201171in}}%
\pgfpathlineto{\pgfqpoint{5.950331in}{3.119443in}}%
\pgfusepath{fill}%
\end{pgfscope}%
\begin{pgfscope}%
\pgfpathrectangle{\pgfqpoint{0.549740in}{0.463273in}}{\pgfqpoint{9.320225in}{4.495057in}}%
\pgfusepath{clip}%
\pgfsetbuttcap%
\pgfsetroundjoin%
\pgfsetlinewidth{0.000000pt}%
\definecolor{currentstroke}{rgb}{0.000000,0.000000,0.000000}%
\pgfsetstrokecolor{currentstroke}%
\pgfsetdash{}{0pt}%
\pgfpathmoveto{\pgfqpoint{6.136557in}{3.119443in}}%
\pgfpathlineto{\pgfqpoint{6.322784in}{3.119443in}}%
\pgfpathlineto{\pgfqpoint{6.322784in}{3.201171in}}%
\pgfpathlineto{\pgfqpoint{6.136557in}{3.201171in}}%
\pgfpathlineto{\pgfqpoint{6.136557in}{3.119443in}}%
\pgfusepath{}%
\end{pgfscope}%
\begin{pgfscope}%
\pgfpathrectangle{\pgfqpoint{0.549740in}{0.463273in}}{\pgfqpoint{9.320225in}{4.495057in}}%
\pgfusepath{clip}%
\pgfsetbuttcap%
\pgfsetroundjoin%
\pgfsetlinewidth{0.000000pt}%
\definecolor{currentstroke}{rgb}{0.000000,0.000000,0.000000}%
\pgfsetstrokecolor{currentstroke}%
\pgfsetdash{}{0pt}%
\pgfpathmoveto{\pgfqpoint{6.322784in}{3.119443in}}%
\pgfpathlineto{\pgfqpoint{6.509011in}{3.119443in}}%
\pgfpathlineto{\pgfqpoint{6.509011in}{3.201171in}}%
\pgfpathlineto{\pgfqpoint{6.322784in}{3.201171in}}%
\pgfpathlineto{\pgfqpoint{6.322784in}{3.119443in}}%
\pgfusepath{}%
\end{pgfscope}%
\begin{pgfscope}%
\pgfpathrectangle{\pgfqpoint{0.549740in}{0.463273in}}{\pgfqpoint{9.320225in}{4.495057in}}%
\pgfusepath{clip}%
\pgfsetbuttcap%
\pgfsetroundjoin%
\pgfsetlinewidth{0.000000pt}%
\definecolor{currentstroke}{rgb}{0.000000,0.000000,0.000000}%
\pgfsetstrokecolor{currentstroke}%
\pgfsetdash{}{0pt}%
\pgfpathmoveto{\pgfqpoint{6.509011in}{3.119443in}}%
\pgfpathlineto{\pgfqpoint{6.695237in}{3.119443in}}%
\pgfpathlineto{\pgfqpoint{6.695237in}{3.201171in}}%
\pgfpathlineto{\pgfqpoint{6.509011in}{3.201171in}}%
\pgfpathlineto{\pgfqpoint{6.509011in}{3.119443in}}%
\pgfusepath{}%
\end{pgfscope}%
\begin{pgfscope}%
\pgfpathrectangle{\pgfqpoint{0.549740in}{0.463273in}}{\pgfqpoint{9.320225in}{4.495057in}}%
\pgfusepath{clip}%
\pgfsetbuttcap%
\pgfsetroundjoin%
\pgfsetlinewidth{0.000000pt}%
\definecolor{currentstroke}{rgb}{0.000000,0.000000,0.000000}%
\pgfsetstrokecolor{currentstroke}%
\pgfsetdash{}{0pt}%
\pgfpathmoveto{\pgfqpoint{6.695237in}{3.119443in}}%
\pgfpathlineto{\pgfqpoint{6.881464in}{3.119443in}}%
\pgfpathlineto{\pgfqpoint{6.881464in}{3.201171in}}%
\pgfpathlineto{\pgfqpoint{6.695237in}{3.201171in}}%
\pgfpathlineto{\pgfqpoint{6.695237in}{3.119443in}}%
\pgfusepath{}%
\end{pgfscope}%
\begin{pgfscope}%
\pgfpathrectangle{\pgfqpoint{0.549740in}{0.463273in}}{\pgfqpoint{9.320225in}{4.495057in}}%
\pgfusepath{clip}%
\pgfsetbuttcap%
\pgfsetroundjoin%
\pgfsetlinewidth{0.000000pt}%
\definecolor{currentstroke}{rgb}{0.000000,0.000000,0.000000}%
\pgfsetstrokecolor{currentstroke}%
\pgfsetdash{}{0pt}%
\pgfpathmoveto{\pgfqpoint{6.881464in}{3.119443in}}%
\pgfpathlineto{\pgfqpoint{7.067690in}{3.119443in}}%
\pgfpathlineto{\pgfqpoint{7.067690in}{3.201171in}}%
\pgfpathlineto{\pgfqpoint{6.881464in}{3.201171in}}%
\pgfpathlineto{\pgfqpoint{6.881464in}{3.119443in}}%
\pgfusepath{}%
\end{pgfscope}%
\begin{pgfscope}%
\pgfpathrectangle{\pgfqpoint{0.549740in}{0.463273in}}{\pgfqpoint{9.320225in}{4.495057in}}%
\pgfusepath{clip}%
\pgfsetbuttcap%
\pgfsetroundjoin%
\definecolor{currentfill}{rgb}{0.472869,0.711325,0.955316}%
\pgfsetfillcolor{currentfill}%
\pgfsetlinewidth{0.000000pt}%
\definecolor{currentstroke}{rgb}{0.000000,0.000000,0.000000}%
\pgfsetstrokecolor{currentstroke}%
\pgfsetdash{}{0pt}%
\pgfpathmoveto{\pgfqpoint{7.067690in}{3.119443in}}%
\pgfpathlineto{\pgfqpoint{7.253917in}{3.119443in}}%
\pgfpathlineto{\pgfqpoint{7.253917in}{3.201171in}}%
\pgfpathlineto{\pgfqpoint{7.067690in}{3.201171in}}%
\pgfpathlineto{\pgfqpoint{7.067690in}{3.119443in}}%
\pgfusepath{fill}%
\end{pgfscope}%
\begin{pgfscope}%
\pgfpathrectangle{\pgfqpoint{0.549740in}{0.463273in}}{\pgfqpoint{9.320225in}{4.495057in}}%
\pgfusepath{clip}%
\pgfsetbuttcap%
\pgfsetroundjoin%
\pgfsetlinewidth{0.000000pt}%
\definecolor{currentstroke}{rgb}{0.000000,0.000000,0.000000}%
\pgfsetstrokecolor{currentstroke}%
\pgfsetdash{}{0pt}%
\pgfpathmoveto{\pgfqpoint{7.253917in}{3.119443in}}%
\pgfpathlineto{\pgfqpoint{7.440143in}{3.119443in}}%
\pgfpathlineto{\pgfqpoint{7.440143in}{3.201171in}}%
\pgfpathlineto{\pgfqpoint{7.253917in}{3.201171in}}%
\pgfpathlineto{\pgfqpoint{7.253917in}{3.119443in}}%
\pgfusepath{}%
\end{pgfscope}%
\begin{pgfscope}%
\pgfpathrectangle{\pgfqpoint{0.549740in}{0.463273in}}{\pgfqpoint{9.320225in}{4.495057in}}%
\pgfusepath{clip}%
\pgfsetbuttcap%
\pgfsetroundjoin%
\pgfsetlinewidth{0.000000pt}%
\definecolor{currentstroke}{rgb}{0.000000,0.000000,0.000000}%
\pgfsetstrokecolor{currentstroke}%
\pgfsetdash{}{0pt}%
\pgfpathmoveto{\pgfqpoint{7.440143in}{3.119443in}}%
\pgfpathlineto{\pgfqpoint{7.626370in}{3.119443in}}%
\pgfpathlineto{\pgfqpoint{7.626370in}{3.201171in}}%
\pgfpathlineto{\pgfqpoint{7.440143in}{3.201171in}}%
\pgfpathlineto{\pgfqpoint{7.440143in}{3.119443in}}%
\pgfusepath{}%
\end{pgfscope}%
\begin{pgfscope}%
\pgfpathrectangle{\pgfqpoint{0.549740in}{0.463273in}}{\pgfqpoint{9.320225in}{4.495057in}}%
\pgfusepath{clip}%
\pgfsetbuttcap%
\pgfsetroundjoin%
\pgfsetlinewidth{0.000000pt}%
\definecolor{currentstroke}{rgb}{0.000000,0.000000,0.000000}%
\pgfsetstrokecolor{currentstroke}%
\pgfsetdash{}{0pt}%
\pgfpathmoveto{\pgfqpoint{7.626370in}{3.119443in}}%
\pgfpathlineto{\pgfqpoint{7.812596in}{3.119443in}}%
\pgfpathlineto{\pgfqpoint{7.812596in}{3.201171in}}%
\pgfpathlineto{\pgfqpoint{7.626370in}{3.201171in}}%
\pgfpathlineto{\pgfqpoint{7.626370in}{3.119443in}}%
\pgfusepath{}%
\end{pgfscope}%
\begin{pgfscope}%
\pgfpathrectangle{\pgfqpoint{0.549740in}{0.463273in}}{\pgfqpoint{9.320225in}{4.495057in}}%
\pgfusepath{clip}%
\pgfsetbuttcap%
\pgfsetroundjoin%
\pgfsetlinewidth{0.000000pt}%
\definecolor{currentstroke}{rgb}{0.000000,0.000000,0.000000}%
\pgfsetstrokecolor{currentstroke}%
\pgfsetdash{}{0pt}%
\pgfpathmoveto{\pgfqpoint{7.812596in}{3.119443in}}%
\pgfpathlineto{\pgfqpoint{7.998823in}{3.119443in}}%
\pgfpathlineto{\pgfqpoint{7.998823in}{3.201171in}}%
\pgfpathlineto{\pgfqpoint{7.812596in}{3.201171in}}%
\pgfpathlineto{\pgfqpoint{7.812596in}{3.119443in}}%
\pgfusepath{}%
\end{pgfscope}%
\begin{pgfscope}%
\pgfpathrectangle{\pgfqpoint{0.549740in}{0.463273in}}{\pgfqpoint{9.320225in}{4.495057in}}%
\pgfusepath{clip}%
\pgfsetbuttcap%
\pgfsetroundjoin%
\pgfsetlinewidth{0.000000pt}%
\definecolor{currentstroke}{rgb}{0.000000,0.000000,0.000000}%
\pgfsetstrokecolor{currentstroke}%
\pgfsetdash{}{0pt}%
\pgfpathmoveto{\pgfqpoint{7.998823in}{3.119443in}}%
\pgfpathlineto{\pgfqpoint{8.185049in}{3.119443in}}%
\pgfpathlineto{\pgfqpoint{8.185049in}{3.201171in}}%
\pgfpathlineto{\pgfqpoint{7.998823in}{3.201171in}}%
\pgfpathlineto{\pgfqpoint{7.998823in}{3.119443in}}%
\pgfusepath{}%
\end{pgfscope}%
\begin{pgfscope}%
\pgfpathrectangle{\pgfqpoint{0.549740in}{0.463273in}}{\pgfqpoint{9.320225in}{4.495057in}}%
\pgfusepath{clip}%
\pgfsetbuttcap%
\pgfsetroundjoin%
\definecolor{currentfill}{rgb}{0.472869,0.711325,0.955316}%
\pgfsetfillcolor{currentfill}%
\pgfsetlinewidth{0.000000pt}%
\definecolor{currentstroke}{rgb}{0.000000,0.000000,0.000000}%
\pgfsetstrokecolor{currentstroke}%
\pgfsetdash{}{0pt}%
\pgfpathmoveto{\pgfqpoint{8.185049in}{3.119443in}}%
\pgfpathlineto{\pgfqpoint{8.371276in}{3.119443in}}%
\pgfpathlineto{\pgfqpoint{8.371276in}{3.201171in}}%
\pgfpathlineto{\pgfqpoint{8.185049in}{3.201171in}}%
\pgfpathlineto{\pgfqpoint{8.185049in}{3.119443in}}%
\pgfusepath{fill}%
\end{pgfscope}%
\begin{pgfscope}%
\pgfpathrectangle{\pgfqpoint{0.549740in}{0.463273in}}{\pgfqpoint{9.320225in}{4.495057in}}%
\pgfusepath{clip}%
\pgfsetbuttcap%
\pgfsetroundjoin%
\pgfsetlinewidth{0.000000pt}%
\definecolor{currentstroke}{rgb}{0.000000,0.000000,0.000000}%
\pgfsetstrokecolor{currentstroke}%
\pgfsetdash{}{0pt}%
\pgfpathmoveto{\pgfqpoint{8.371276in}{3.119443in}}%
\pgfpathlineto{\pgfqpoint{8.557503in}{3.119443in}}%
\pgfpathlineto{\pgfqpoint{8.557503in}{3.201171in}}%
\pgfpathlineto{\pgfqpoint{8.371276in}{3.201171in}}%
\pgfpathlineto{\pgfqpoint{8.371276in}{3.119443in}}%
\pgfusepath{}%
\end{pgfscope}%
\begin{pgfscope}%
\pgfpathrectangle{\pgfqpoint{0.549740in}{0.463273in}}{\pgfqpoint{9.320225in}{4.495057in}}%
\pgfusepath{clip}%
\pgfsetbuttcap%
\pgfsetroundjoin%
\pgfsetlinewidth{0.000000pt}%
\definecolor{currentstroke}{rgb}{0.000000,0.000000,0.000000}%
\pgfsetstrokecolor{currentstroke}%
\pgfsetdash{}{0pt}%
\pgfpathmoveto{\pgfqpoint{8.557503in}{3.119443in}}%
\pgfpathlineto{\pgfqpoint{8.743729in}{3.119443in}}%
\pgfpathlineto{\pgfqpoint{8.743729in}{3.201171in}}%
\pgfpathlineto{\pgfqpoint{8.557503in}{3.201171in}}%
\pgfpathlineto{\pgfqpoint{8.557503in}{3.119443in}}%
\pgfusepath{}%
\end{pgfscope}%
\begin{pgfscope}%
\pgfpathrectangle{\pgfqpoint{0.549740in}{0.463273in}}{\pgfqpoint{9.320225in}{4.495057in}}%
\pgfusepath{clip}%
\pgfsetbuttcap%
\pgfsetroundjoin%
\pgfsetlinewidth{0.000000pt}%
\definecolor{currentstroke}{rgb}{0.000000,0.000000,0.000000}%
\pgfsetstrokecolor{currentstroke}%
\pgfsetdash{}{0pt}%
\pgfpathmoveto{\pgfqpoint{8.743729in}{3.119443in}}%
\pgfpathlineto{\pgfqpoint{8.929956in}{3.119443in}}%
\pgfpathlineto{\pgfqpoint{8.929956in}{3.201171in}}%
\pgfpathlineto{\pgfqpoint{8.743729in}{3.201171in}}%
\pgfpathlineto{\pgfqpoint{8.743729in}{3.119443in}}%
\pgfusepath{}%
\end{pgfscope}%
\begin{pgfscope}%
\pgfpathrectangle{\pgfqpoint{0.549740in}{0.463273in}}{\pgfqpoint{9.320225in}{4.495057in}}%
\pgfusepath{clip}%
\pgfsetbuttcap%
\pgfsetroundjoin%
\pgfsetlinewidth{0.000000pt}%
\definecolor{currentstroke}{rgb}{0.000000,0.000000,0.000000}%
\pgfsetstrokecolor{currentstroke}%
\pgfsetdash{}{0pt}%
\pgfpathmoveto{\pgfqpoint{8.929956in}{3.119443in}}%
\pgfpathlineto{\pgfqpoint{9.116182in}{3.119443in}}%
\pgfpathlineto{\pgfqpoint{9.116182in}{3.201171in}}%
\pgfpathlineto{\pgfqpoint{8.929956in}{3.201171in}}%
\pgfpathlineto{\pgfqpoint{8.929956in}{3.119443in}}%
\pgfusepath{}%
\end{pgfscope}%
\begin{pgfscope}%
\pgfpathrectangle{\pgfqpoint{0.549740in}{0.463273in}}{\pgfqpoint{9.320225in}{4.495057in}}%
\pgfusepath{clip}%
\pgfsetbuttcap%
\pgfsetroundjoin%
\pgfsetlinewidth{0.000000pt}%
\definecolor{currentstroke}{rgb}{0.000000,0.000000,0.000000}%
\pgfsetstrokecolor{currentstroke}%
\pgfsetdash{}{0pt}%
\pgfpathmoveto{\pgfqpoint{9.116182in}{3.119443in}}%
\pgfpathlineto{\pgfqpoint{9.302409in}{3.119443in}}%
\pgfpathlineto{\pgfqpoint{9.302409in}{3.201171in}}%
\pgfpathlineto{\pgfqpoint{9.116182in}{3.201171in}}%
\pgfpathlineto{\pgfqpoint{9.116182in}{3.119443in}}%
\pgfusepath{}%
\end{pgfscope}%
\begin{pgfscope}%
\pgfpathrectangle{\pgfqpoint{0.549740in}{0.463273in}}{\pgfqpoint{9.320225in}{4.495057in}}%
\pgfusepath{clip}%
\pgfsetbuttcap%
\pgfsetroundjoin%
\definecolor{currentfill}{rgb}{0.547810,0.736432,0.947518}%
\pgfsetfillcolor{currentfill}%
\pgfsetlinewidth{0.000000pt}%
\definecolor{currentstroke}{rgb}{0.000000,0.000000,0.000000}%
\pgfsetstrokecolor{currentstroke}%
\pgfsetdash{}{0pt}%
\pgfpathmoveto{\pgfqpoint{9.302409in}{3.119443in}}%
\pgfpathlineto{\pgfqpoint{9.488635in}{3.119443in}}%
\pgfpathlineto{\pgfqpoint{9.488635in}{3.201171in}}%
\pgfpathlineto{\pgfqpoint{9.302409in}{3.201171in}}%
\pgfpathlineto{\pgfqpoint{9.302409in}{3.119443in}}%
\pgfusepath{fill}%
\end{pgfscope}%
\begin{pgfscope}%
\pgfpathrectangle{\pgfqpoint{0.549740in}{0.463273in}}{\pgfqpoint{9.320225in}{4.495057in}}%
\pgfusepath{clip}%
\pgfsetbuttcap%
\pgfsetroundjoin%
\definecolor{currentfill}{rgb}{0.614330,0.761948,0.940009}%
\pgfsetfillcolor{currentfill}%
\pgfsetlinewidth{0.000000pt}%
\definecolor{currentstroke}{rgb}{0.000000,0.000000,0.000000}%
\pgfsetstrokecolor{currentstroke}%
\pgfsetdash{}{0pt}%
\pgfpathmoveto{\pgfqpoint{9.488635in}{3.119443in}}%
\pgfpathlineto{\pgfqpoint{9.674862in}{3.119443in}}%
\pgfpathlineto{\pgfqpoint{9.674862in}{3.201171in}}%
\pgfpathlineto{\pgfqpoint{9.488635in}{3.201171in}}%
\pgfpathlineto{\pgfqpoint{9.488635in}{3.119443in}}%
\pgfusepath{fill}%
\end{pgfscope}%
\begin{pgfscope}%
\pgfpathrectangle{\pgfqpoint{0.549740in}{0.463273in}}{\pgfqpoint{9.320225in}{4.495057in}}%
\pgfusepath{clip}%
\pgfsetbuttcap%
\pgfsetroundjoin%
\pgfsetlinewidth{0.000000pt}%
\definecolor{currentstroke}{rgb}{0.000000,0.000000,0.000000}%
\pgfsetstrokecolor{currentstroke}%
\pgfsetdash{}{0pt}%
\pgfpathmoveto{\pgfqpoint{9.674862in}{3.119443in}}%
\pgfpathlineto{\pgfqpoint{9.861088in}{3.119443in}}%
\pgfpathlineto{\pgfqpoint{9.861088in}{3.201171in}}%
\pgfpathlineto{\pgfqpoint{9.674862in}{3.201171in}}%
\pgfpathlineto{\pgfqpoint{9.674862in}{3.119443in}}%
\pgfusepath{}%
\end{pgfscope}%
\begin{pgfscope}%
\pgfpathrectangle{\pgfqpoint{0.549740in}{0.463273in}}{\pgfqpoint{9.320225in}{4.495057in}}%
\pgfusepath{clip}%
\pgfsetbuttcap%
\pgfsetroundjoin%
\pgfsetlinewidth{0.000000pt}%
\definecolor{currentstroke}{rgb}{0.000000,0.000000,0.000000}%
\pgfsetstrokecolor{currentstroke}%
\pgfsetdash{}{0pt}%
\pgfpathmoveto{\pgfqpoint{0.549761in}{3.201171in}}%
\pgfpathlineto{\pgfqpoint{0.735988in}{3.201171in}}%
\pgfpathlineto{\pgfqpoint{0.735988in}{3.282900in}}%
\pgfpathlineto{\pgfqpoint{0.549761in}{3.282900in}}%
\pgfpathlineto{\pgfqpoint{0.549761in}{3.201171in}}%
\pgfusepath{}%
\end{pgfscope}%
\begin{pgfscope}%
\pgfpathrectangle{\pgfqpoint{0.549740in}{0.463273in}}{\pgfqpoint{9.320225in}{4.495057in}}%
\pgfusepath{clip}%
\pgfsetbuttcap%
\pgfsetroundjoin%
\pgfsetlinewidth{0.000000pt}%
\definecolor{currentstroke}{rgb}{0.000000,0.000000,0.000000}%
\pgfsetstrokecolor{currentstroke}%
\pgfsetdash{}{0pt}%
\pgfpathmoveto{\pgfqpoint{0.735988in}{3.201171in}}%
\pgfpathlineto{\pgfqpoint{0.922214in}{3.201171in}}%
\pgfpathlineto{\pgfqpoint{0.922214in}{3.282900in}}%
\pgfpathlineto{\pgfqpoint{0.735988in}{3.282900in}}%
\pgfpathlineto{\pgfqpoint{0.735988in}{3.201171in}}%
\pgfusepath{}%
\end{pgfscope}%
\begin{pgfscope}%
\pgfpathrectangle{\pgfqpoint{0.549740in}{0.463273in}}{\pgfqpoint{9.320225in}{4.495057in}}%
\pgfusepath{clip}%
\pgfsetbuttcap%
\pgfsetroundjoin%
\pgfsetlinewidth{0.000000pt}%
\definecolor{currentstroke}{rgb}{0.000000,0.000000,0.000000}%
\pgfsetstrokecolor{currentstroke}%
\pgfsetdash{}{0pt}%
\pgfpathmoveto{\pgfqpoint{0.922214in}{3.201171in}}%
\pgfpathlineto{\pgfqpoint{1.108441in}{3.201171in}}%
\pgfpathlineto{\pgfqpoint{1.108441in}{3.282900in}}%
\pgfpathlineto{\pgfqpoint{0.922214in}{3.282900in}}%
\pgfpathlineto{\pgfqpoint{0.922214in}{3.201171in}}%
\pgfusepath{}%
\end{pgfscope}%
\begin{pgfscope}%
\pgfpathrectangle{\pgfqpoint{0.549740in}{0.463273in}}{\pgfqpoint{9.320225in}{4.495057in}}%
\pgfusepath{clip}%
\pgfsetbuttcap%
\pgfsetroundjoin%
\pgfsetlinewidth{0.000000pt}%
\definecolor{currentstroke}{rgb}{0.000000,0.000000,0.000000}%
\pgfsetstrokecolor{currentstroke}%
\pgfsetdash{}{0pt}%
\pgfpathmoveto{\pgfqpoint{1.108441in}{3.201171in}}%
\pgfpathlineto{\pgfqpoint{1.294667in}{3.201171in}}%
\pgfpathlineto{\pgfqpoint{1.294667in}{3.282900in}}%
\pgfpathlineto{\pgfqpoint{1.108441in}{3.282900in}}%
\pgfpathlineto{\pgfqpoint{1.108441in}{3.201171in}}%
\pgfusepath{}%
\end{pgfscope}%
\begin{pgfscope}%
\pgfpathrectangle{\pgfqpoint{0.549740in}{0.463273in}}{\pgfqpoint{9.320225in}{4.495057in}}%
\pgfusepath{clip}%
\pgfsetbuttcap%
\pgfsetroundjoin%
\pgfsetlinewidth{0.000000pt}%
\definecolor{currentstroke}{rgb}{0.000000,0.000000,0.000000}%
\pgfsetstrokecolor{currentstroke}%
\pgfsetdash{}{0pt}%
\pgfpathmoveto{\pgfqpoint{1.294667in}{3.201171in}}%
\pgfpathlineto{\pgfqpoint{1.480894in}{3.201171in}}%
\pgfpathlineto{\pgfqpoint{1.480894in}{3.282900in}}%
\pgfpathlineto{\pgfqpoint{1.294667in}{3.282900in}}%
\pgfpathlineto{\pgfqpoint{1.294667in}{3.201171in}}%
\pgfusepath{}%
\end{pgfscope}%
\begin{pgfscope}%
\pgfpathrectangle{\pgfqpoint{0.549740in}{0.463273in}}{\pgfqpoint{9.320225in}{4.495057in}}%
\pgfusepath{clip}%
\pgfsetbuttcap%
\pgfsetroundjoin%
\pgfsetlinewidth{0.000000pt}%
\definecolor{currentstroke}{rgb}{0.000000,0.000000,0.000000}%
\pgfsetstrokecolor{currentstroke}%
\pgfsetdash{}{0pt}%
\pgfpathmoveto{\pgfqpoint{1.480894in}{3.201171in}}%
\pgfpathlineto{\pgfqpoint{1.667120in}{3.201171in}}%
\pgfpathlineto{\pgfqpoint{1.667120in}{3.282900in}}%
\pgfpathlineto{\pgfqpoint{1.480894in}{3.282900in}}%
\pgfpathlineto{\pgfqpoint{1.480894in}{3.201171in}}%
\pgfusepath{}%
\end{pgfscope}%
\begin{pgfscope}%
\pgfpathrectangle{\pgfqpoint{0.549740in}{0.463273in}}{\pgfqpoint{9.320225in}{4.495057in}}%
\pgfusepath{clip}%
\pgfsetbuttcap%
\pgfsetroundjoin%
\pgfsetlinewidth{0.000000pt}%
\definecolor{currentstroke}{rgb}{0.000000,0.000000,0.000000}%
\pgfsetstrokecolor{currentstroke}%
\pgfsetdash{}{0pt}%
\pgfpathmoveto{\pgfqpoint{1.667120in}{3.201171in}}%
\pgfpathlineto{\pgfqpoint{1.853347in}{3.201171in}}%
\pgfpathlineto{\pgfqpoint{1.853347in}{3.282900in}}%
\pgfpathlineto{\pgfqpoint{1.667120in}{3.282900in}}%
\pgfpathlineto{\pgfqpoint{1.667120in}{3.201171in}}%
\pgfusepath{}%
\end{pgfscope}%
\begin{pgfscope}%
\pgfpathrectangle{\pgfqpoint{0.549740in}{0.463273in}}{\pgfqpoint{9.320225in}{4.495057in}}%
\pgfusepath{clip}%
\pgfsetbuttcap%
\pgfsetroundjoin%
\pgfsetlinewidth{0.000000pt}%
\definecolor{currentstroke}{rgb}{0.000000,0.000000,0.000000}%
\pgfsetstrokecolor{currentstroke}%
\pgfsetdash{}{0pt}%
\pgfpathmoveto{\pgfqpoint{1.853347in}{3.201171in}}%
\pgfpathlineto{\pgfqpoint{2.039573in}{3.201171in}}%
\pgfpathlineto{\pgfqpoint{2.039573in}{3.282900in}}%
\pgfpathlineto{\pgfqpoint{1.853347in}{3.282900in}}%
\pgfpathlineto{\pgfqpoint{1.853347in}{3.201171in}}%
\pgfusepath{}%
\end{pgfscope}%
\begin{pgfscope}%
\pgfpathrectangle{\pgfqpoint{0.549740in}{0.463273in}}{\pgfqpoint{9.320225in}{4.495057in}}%
\pgfusepath{clip}%
\pgfsetbuttcap%
\pgfsetroundjoin%
\pgfsetlinewidth{0.000000pt}%
\definecolor{currentstroke}{rgb}{0.000000,0.000000,0.000000}%
\pgfsetstrokecolor{currentstroke}%
\pgfsetdash{}{0pt}%
\pgfpathmoveto{\pgfqpoint{2.039573in}{3.201171in}}%
\pgfpathlineto{\pgfqpoint{2.225800in}{3.201171in}}%
\pgfpathlineto{\pgfqpoint{2.225800in}{3.282900in}}%
\pgfpathlineto{\pgfqpoint{2.039573in}{3.282900in}}%
\pgfpathlineto{\pgfqpoint{2.039573in}{3.201171in}}%
\pgfusepath{}%
\end{pgfscope}%
\begin{pgfscope}%
\pgfpathrectangle{\pgfqpoint{0.549740in}{0.463273in}}{\pgfqpoint{9.320225in}{4.495057in}}%
\pgfusepath{clip}%
\pgfsetbuttcap%
\pgfsetroundjoin%
\pgfsetlinewidth{0.000000pt}%
\definecolor{currentstroke}{rgb}{0.000000,0.000000,0.000000}%
\pgfsetstrokecolor{currentstroke}%
\pgfsetdash{}{0pt}%
\pgfpathmoveto{\pgfqpoint{2.225800in}{3.201171in}}%
\pgfpathlineto{\pgfqpoint{2.412027in}{3.201171in}}%
\pgfpathlineto{\pgfqpoint{2.412027in}{3.282900in}}%
\pgfpathlineto{\pgfqpoint{2.225800in}{3.282900in}}%
\pgfpathlineto{\pgfqpoint{2.225800in}{3.201171in}}%
\pgfusepath{}%
\end{pgfscope}%
\begin{pgfscope}%
\pgfpathrectangle{\pgfqpoint{0.549740in}{0.463273in}}{\pgfqpoint{9.320225in}{4.495057in}}%
\pgfusepath{clip}%
\pgfsetbuttcap%
\pgfsetroundjoin%
\pgfsetlinewidth{0.000000pt}%
\definecolor{currentstroke}{rgb}{0.000000,0.000000,0.000000}%
\pgfsetstrokecolor{currentstroke}%
\pgfsetdash{}{0pt}%
\pgfpathmoveto{\pgfqpoint{2.412027in}{3.201171in}}%
\pgfpathlineto{\pgfqpoint{2.598253in}{3.201171in}}%
\pgfpathlineto{\pgfqpoint{2.598253in}{3.282900in}}%
\pgfpathlineto{\pgfqpoint{2.412027in}{3.282900in}}%
\pgfpathlineto{\pgfqpoint{2.412027in}{3.201171in}}%
\pgfusepath{}%
\end{pgfscope}%
\begin{pgfscope}%
\pgfpathrectangle{\pgfqpoint{0.549740in}{0.463273in}}{\pgfqpoint{9.320225in}{4.495057in}}%
\pgfusepath{clip}%
\pgfsetbuttcap%
\pgfsetroundjoin%
\pgfsetlinewidth{0.000000pt}%
\definecolor{currentstroke}{rgb}{0.000000,0.000000,0.000000}%
\pgfsetstrokecolor{currentstroke}%
\pgfsetdash{}{0pt}%
\pgfpathmoveto{\pgfqpoint{2.598253in}{3.201171in}}%
\pgfpathlineto{\pgfqpoint{2.784480in}{3.201171in}}%
\pgfpathlineto{\pgfqpoint{2.784480in}{3.282900in}}%
\pgfpathlineto{\pgfqpoint{2.598253in}{3.282900in}}%
\pgfpathlineto{\pgfqpoint{2.598253in}{3.201171in}}%
\pgfusepath{}%
\end{pgfscope}%
\begin{pgfscope}%
\pgfpathrectangle{\pgfqpoint{0.549740in}{0.463273in}}{\pgfqpoint{9.320225in}{4.495057in}}%
\pgfusepath{clip}%
\pgfsetbuttcap%
\pgfsetroundjoin%
\pgfsetlinewidth{0.000000pt}%
\definecolor{currentstroke}{rgb}{0.000000,0.000000,0.000000}%
\pgfsetstrokecolor{currentstroke}%
\pgfsetdash{}{0pt}%
\pgfpathmoveto{\pgfqpoint{2.784480in}{3.201171in}}%
\pgfpathlineto{\pgfqpoint{2.970706in}{3.201171in}}%
\pgfpathlineto{\pgfqpoint{2.970706in}{3.282900in}}%
\pgfpathlineto{\pgfqpoint{2.784480in}{3.282900in}}%
\pgfpathlineto{\pgfqpoint{2.784480in}{3.201171in}}%
\pgfusepath{}%
\end{pgfscope}%
\begin{pgfscope}%
\pgfpathrectangle{\pgfqpoint{0.549740in}{0.463273in}}{\pgfqpoint{9.320225in}{4.495057in}}%
\pgfusepath{clip}%
\pgfsetbuttcap%
\pgfsetroundjoin%
\pgfsetlinewidth{0.000000pt}%
\definecolor{currentstroke}{rgb}{0.000000,0.000000,0.000000}%
\pgfsetstrokecolor{currentstroke}%
\pgfsetdash{}{0pt}%
\pgfpathmoveto{\pgfqpoint{2.970706in}{3.201171in}}%
\pgfpathlineto{\pgfqpoint{3.156933in}{3.201171in}}%
\pgfpathlineto{\pgfqpoint{3.156933in}{3.282900in}}%
\pgfpathlineto{\pgfqpoint{2.970706in}{3.282900in}}%
\pgfpathlineto{\pgfqpoint{2.970706in}{3.201171in}}%
\pgfusepath{}%
\end{pgfscope}%
\begin{pgfscope}%
\pgfpathrectangle{\pgfqpoint{0.549740in}{0.463273in}}{\pgfqpoint{9.320225in}{4.495057in}}%
\pgfusepath{clip}%
\pgfsetbuttcap%
\pgfsetroundjoin%
\pgfsetlinewidth{0.000000pt}%
\definecolor{currentstroke}{rgb}{0.000000,0.000000,0.000000}%
\pgfsetstrokecolor{currentstroke}%
\pgfsetdash{}{0pt}%
\pgfpathmoveto{\pgfqpoint{3.156933in}{3.201171in}}%
\pgfpathlineto{\pgfqpoint{3.343159in}{3.201171in}}%
\pgfpathlineto{\pgfqpoint{3.343159in}{3.282900in}}%
\pgfpathlineto{\pgfqpoint{3.156933in}{3.282900in}}%
\pgfpathlineto{\pgfqpoint{3.156933in}{3.201171in}}%
\pgfusepath{}%
\end{pgfscope}%
\begin{pgfscope}%
\pgfpathrectangle{\pgfqpoint{0.549740in}{0.463273in}}{\pgfqpoint{9.320225in}{4.495057in}}%
\pgfusepath{clip}%
\pgfsetbuttcap%
\pgfsetroundjoin%
\pgfsetlinewidth{0.000000pt}%
\definecolor{currentstroke}{rgb}{0.000000,0.000000,0.000000}%
\pgfsetstrokecolor{currentstroke}%
\pgfsetdash{}{0pt}%
\pgfpathmoveto{\pgfqpoint{3.343159in}{3.201171in}}%
\pgfpathlineto{\pgfqpoint{3.529386in}{3.201171in}}%
\pgfpathlineto{\pgfqpoint{3.529386in}{3.282900in}}%
\pgfpathlineto{\pgfqpoint{3.343159in}{3.282900in}}%
\pgfpathlineto{\pgfqpoint{3.343159in}{3.201171in}}%
\pgfusepath{}%
\end{pgfscope}%
\begin{pgfscope}%
\pgfpathrectangle{\pgfqpoint{0.549740in}{0.463273in}}{\pgfqpoint{9.320225in}{4.495057in}}%
\pgfusepath{clip}%
\pgfsetbuttcap%
\pgfsetroundjoin%
\pgfsetlinewidth{0.000000pt}%
\definecolor{currentstroke}{rgb}{0.000000,0.000000,0.000000}%
\pgfsetstrokecolor{currentstroke}%
\pgfsetdash{}{0pt}%
\pgfpathmoveto{\pgfqpoint{3.529386in}{3.201171in}}%
\pgfpathlineto{\pgfqpoint{3.715612in}{3.201171in}}%
\pgfpathlineto{\pgfqpoint{3.715612in}{3.282900in}}%
\pgfpathlineto{\pgfqpoint{3.529386in}{3.282900in}}%
\pgfpathlineto{\pgfqpoint{3.529386in}{3.201171in}}%
\pgfusepath{}%
\end{pgfscope}%
\begin{pgfscope}%
\pgfpathrectangle{\pgfqpoint{0.549740in}{0.463273in}}{\pgfqpoint{9.320225in}{4.495057in}}%
\pgfusepath{clip}%
\pgfsetbuttcap%
\pgfsetroundjoin%
\pgfsetlinewidth{0.000000pt}%
\definecolor{currentstroke}{rgb}{0.000000,0.000000,0.000000}%
\pgfsetstrokecolor{currentstroke}%
\pgfsetdash{}{0pt}%
\pgfpathmoveto{\pgfqpoint{3.715612in}{3.201171in}}%
\pgfpathlineto{\pgfqpoint{3.901839in}{3.201171in}}%
\pgfpathlineto{\pgfqpoint{3.901839in}{3.282900in}}%
\pgfpathlineto{\pgfqpoint{3.715612in}{3.282900in}}%
\pgfpathlineto{\pgfqpoint{3.715612in}{3.201171in}}%
\pgfusepath{}%
\end{pgfscope}%
\begin{pgfscope}%
\pgfpathrectangle{\pgfqpoint{0.549740in}{0.463273in}}{\pgfqpoint{9.320225in}{4.495057in}}%
\pgfusepath{clip}%
\pgfsetbuttcap%
\pgfsetroundjoin%
\pgfsetlinewidth{0.000000pt}%
\definecolor{currentstroke}{rgb}{0.000000,0.000000,0.000000}%
\pgfsetstrokecolor{currentstroke}%
\pgfsetdash{}{0pt}%
\pgfpathmoveto{\pgfqpoint{3.901839in}{3.201171in}}%
\pgfpathlineto{\pgfqpoint{4.088065in}{3.201171in}}%
\pgfpathlineto{\pgfqpoint{4.088065in}{3.282900in}}%
\pgfpathlineto{\pgfqpoint{3.901839in}{3.282900in}}%
\pgfpathlineto{\pgfqpoint{3.901839in}{3.201171in}}%
\pgfusepath{}%
\end{pgfscope}%
\begin{pgfscope}%
\pgfpathrectangle{\pgfqpoint{0.549740in}{0.463273in}}{\pgfqpoint{9.320225in}{4.495057in}}%
\pgfusepath{clip}%
\pgfsetbuttcap%
\pgfsetroundjoin%
\pgfsetlinewidth{0.000000pt}%
\definecolor{currentstroke}{rgb}{0.000000,0.000000,0.000000}%
\pgfsetstrokecolor{currentstroke}%
\pgfsetdash{}{0pt}%
\pgfpathmoveto{\pgfqpoint{4.088065in}{3.201171in}}%
\pgfpathlineto{\pgfqpoint{4.274292in}{3.201171in}}%
\pgfpathlineto{\pgfqpoint{4.274292in}{3.282900in}}%
\pgfpathlineto{\pgfqpoint{4.088065in}{3.282900in}}%
\pgfpathlineto{\pgfqpoint{4.088065in}{3.201171in}}%
\pgfusepath{}%
\end{pgfscope}%
\begin{pgfscope}%
\pgfpathrectangle{\pgfqpoint{0.549740in}{0.463273in}}{\pgfqpoint{9.320225in}{4.495057in}}%
\pgfusepath{clip}%
\pgfsetbuttcap%
\pgfsetroundjoin%
\pgfsetlinewidth{0.000000pt}%
\definecolor{currentstroke}{rgb}{0.000000,0.000000,0.000000}%
\pgfsetstrokecolor{currentstroke}%
\pgfsetdash{}{0pt}%
\pgfpathmoveto{\pgfqpoint{4.274292in}{3.201171in}}%
\pgfpathlineto{\pgfqpoint{4.460519in}{3.201171in}}%
\pgfpathlineto{\pgfqpoint{4.460519in}{3.282900in}}%
\pgfpathlineto{\pgfqpoint{4.274292in}{3.282900in}}%
\pgfpathlineto{\pgfqpoint{4.274292in}{3.201171in}}%
\pgfusepath{}%
\end{pgfscope}%
\begin{pgfscope}%
\pgfpathrectangle{\pgfqpoint{0.549740in}{0.463273in}}{\pgfqpoint{9.320225in}{4.495057in}}%
\pgfusepath{clip}%
\pgfsetbuttcap%
\pgfsetroundjoin%
\pgfsetlinewidth{0.000000pt}%
\definecolor{currentstroke}{rgb}{0.000000,0.000000,0.000000}%
\pgfsetstrokecolor{currentstroke}%
\pgfsetdash{}{0pt}%
\pgfpathmoveto{\pgfqpoint{4.460519in}{3.201171in}}%
\pgfpathlineto{\pgfqpoint{4.646745in}{3.201171in}}%
\pgfpathlineto{\pgfqpoint{4.646745in}{3.282900in}}%
\pgfpathlineto{\pgfqpoint{4.460519in}{3.282900in}}%
\pgfpathlineto{\pgfqpoint{4.460519in}{3.201171in}}%
\pgfusepath{}%
\end{pgfscope}%
\begin{pgfscope}%
\pgfpathrectangle{\pgfqpoint{0.549740in}{0.463273in}}{\pgfqpoint{9.320225in}{4.495057in}}%
\pgfusepath{clip}%
\pgfsetbuttcap%
\pgfsetroundjoin%
\pgfsetlinewidth{0.000000pt}%
\definecolor{currentstroke}{rgb}{0.000000,0.000000,0.000000}%
\pgfsetstrokecolor{currentstroke}%
\pgfsetdash{}{0pt}%
\pgfpathmoveto{\pgfqpoint{4.646745in}{3.201171in}}%
\pgfpathlineto{\pgfqpoint{4.832972in}{3.201171in}}%
\pgfpathlineto{\pgfqpoint{4.832972in}{3.282900in}}%
\pgfpathlineto{\pgfqpoint{4.646745in}{3.282900in}}%
\pgfpathlineto{\pgfqpoint{4.646745in}{3.201171in}}%
\pgfusepath{}%
\end{pgfscope}%
\begin{pgfscope}%
\pgfpathrectangle{\pgfqpoint{0.549740in}{0.463273in}}{\pgfqpoint{9.320225in}{4.495057in}}%
\pgfusepath{clip}%
\pgfsetbuttcap%
\pgfsetroundjoin%
\pgfsetlinewidth{0.000000pt}%
\definecolor{currentstroke}{rgb}{0.000000,0.000000,0.000000}%
\pgfsetstrokecolor{currentstroke}%
\pgfsetdash{}{0pt}%
\pgfpathmoveto{\pgfqpoint{4.832972in}{3.201171in}}%
\pgfpathlineto{\pgfqpoint{5.019198in}{3.201171in}}%
\pgfpathlineto{\pgfqpoint{5.019198in}{3.282900in}}%
\pgfpathlineto{\pgfqpoint{4.832972in}{3.282900in}}%
\pgfpathlineto{\pgfqpoint{4.832972in}{3.201171in}}%
\pgfusepath{}%
\end{pgfscope}%
\begin{pgfscope}%
\pgfpathrectangle{\pgfqpoint{0.549740in}{0.463273in}}{\pgfqpoint{9.320225in}{4.495057in}}%
\pgfusepath{clip}%
\pgfsetbuttcap%
\pgfsetroundjoin%
\definecolor{currentfill}{rgb}{0.614330,0.761948,0.940009}%
\pgfsetfillcolor{currentfill}%
\pgfsetlinewidth{0.000000pt}%
\definecolor{currentstroke}{rgb}{0.000000,0.000000,0.000000}%
\pgfsetstrokecolor{currentstroke}%
\pgfsetdash{}{0pt}%
\pgfpathmoveto{\pgfqpoint{5.019198in}{3.201171in}}%
\pgfpathlineto{\pgfqpoint{5.205425in}{3.201171in}}%
\pgfpathlineto{\pgfqpoint{5.205425in}{3.282900in}}%
\pgfpathlineto{\pgfqpoint{5.019198in}{3.282900in}}%
\pgfpathlineto{\pgfqpoint{5.019198in}{3.201171in}}%
\pgfusepath{fill}%
\end{pgfscope}%
\begin{pgfscope}%
\pgfpathrectangle{\pgfqpoint{0.549740in}{0.463273in}}{\pgfqpoint{9.320225in}{4.495057in}}%
\pgfusepath{clip}%
\pgfsetbuttcap%
\pgfsetroundjoin%
\pgfsetlinewidth{0.000000pt}%
\definecolor{currentstroke}{rgb}{0.000000,0.000000,0.000000}%
\pgfsetstrokecolor{currentstroke}%
\pgfsetdash{}{0pt}%
\pgfpathmoveto{\pgfqpoint{5.205425in}{3.201171in}}%
\pgfpathlineto{\pgfqpoint{5.391651in}{3.201171in}}%
\pgfpathlineto{\pgfqpoint{5.391651in}{3.282900in}}%
\pgfpathlineto{\pgfqpoint{5.205425in}{3.282900in}}%
\pgfpathlineto{\pgfqpoint{5.205425in}{3.201171in}}%
\pgfusepath{}%
\end{pgfscope}%
\begin{pgfscope}%
\pgfpathrectangle{\pgfqpoint{0.549740in}{0.463273in}}{\pgfqpoint{9.320225in}{4.495057in}}%
\pgfusepath{clip}%
\pgfsetbuttcap%
\pgfsetroundjoin%
\pgfsetlinewidth{0.000000pt}%
\definecolor{currentstroke}{rgb}{0.000000,0.000000,0.000000}%
\pgfsetstrokecolor{currentstroke}%
\pgfsetdash{}{0pt}%
\pgfpathmoveto{\pgfqpoint{5.391651in}{3.201171in}}%
\pgfpathlineto{\pgfqpoint{5.577878in}{3.201171in}}%
\pgfpathlineto{\pgfqpoint{5.577878in}{3.282900in}}%
\pgfpathlineto{\pgfqpoint{5.391651in}{3.282900in}}%
\pgfpathlineto{\pgfqpoint{5.391651in}{3.201171in}}%
\pgfusepath{}%
\end{pgfscope}%
\begin{pgfscope}%
\pgfpathrectangle{\pgfqpoint{0.549740in}{0.463273in}}{\pgfqpoint{9.320225in}{4.495057in}}%
\pgfusepath{clip}%
\pgfsetbuttcap%
\pgfsetroundjoin%
\pgfsetlinewidth{0.000000pt}%
\definecolor{currentstroke}{rgb}{0.000000,0.000000,0.000000}%
\pgfsetstrokecolor{currentstroke}%
\pgfsetdash{}{0pt}%
\pgfpathmoveto{\pgfqpoint{5.577878in}{3.201171in}}%
\pgfpathlineto{\pgfqpoint{5.764104in}{3.201171in}}%
\pgfpathlineto{\pgfqpoint{5.764104in}{3.282900in}}%
\pgfpathlineto{\pgfqpoint{5.577878in}{3.282900in}}%
\pgfpathlineto{\pgfqpoint{5.577878in}{3.201171in}}%
\pgfusepath{}%
\end{pgfscope}%
\begin{pgfscope}%
\pgfpathrectangle{\pgfqpoint{0.549740in}{0.463273in}}{\pgfqpoint{9.320225in}{4.495057in}}%
\pgfusepath{clip}%
\pgfsetbuttcap%
\pgfsetroundjoin%
\pgfsetlinewidth{0.000000pt}%
\definecolor{currentstroke}{rgb}{0.000000,0.000000,0.000000}%
\pgfsetstrokecolor{currentstroke}%
\pgfsetdash{}{0pt}%
\pgfpathmoveto{\pgfqpoint{5.764104in}{3.201171in}}%
\pgfpathlineto{\pgfqpoint{5.950331in}{3.201171in}}%
\pgfpathlineto{\pgfqpoint{5.950331in}{3.282900in}}%
\pgfpathlineto{\pgfqpoint{5.764104in}{3.282900in}}%
\pgfpathlineto{\pgfqpoint{5.764104in}{3.201171in}}%
\pgfusepath{}%
\end{pgfscope}%
\begin{pgfscope}%
\pgfpathrectangle{\pgfqpoint{0.549740in}{0.463273in}}{\pgfqpoint{9.320225in}{4.495057in}}%
\pgfusepath{clip}%
\pgfsetbuttcap%
\pgfsetroundjoin%
\definecolor{currentfill}{rgb}{0.472869,0.711325,0.955316}%
\pgfsetfillcolor{currentfill}%
\pgfsetlinewidth{0.000000pt}%
\definecolor{currentstroke}{rgb}{0.000000,0.000000,0.000000}%
\pgfsetstrokecolor{currentstroke}%
\pgfsetdash{}{0pt}%
\pgfpathmoveto{\pgfqpoint{5.950331in}{3.201171in}}%
\pgfpathlineto{\pgfqpoint{6.136557in}{3.201171in}}%
\pgfpathlineto{\pgfqpoint{6.136557in}{3.282900in}}%
\pgfpathlineto{\pgfqpoint{5.950331in}{3.282900in}}%
\pgfpathlineto{\pgfqpoint{5.950331in}{3.201171in}}%
\pgfusepath{fill}%
\end{pgfscope}%
\begin{pgfscope}%
\pgfpathrectangle{\pgfqpoint{0.549740in}{0.463273in}}{\pgfqpoint{9.320225in}{4.495057in}}%
\pgfusepath{clip}%
\pgfsetbuttcap%
\pgfsetroundjoin%
\pgfsetlinewidth{0.000000pt}%
\definecolor{currentstroke}{rgb}{0.000000,0.000000,0.000000}%
\pgfsetstrokecolor{currentstroke}%
\pgfsetdash{}{0pt}%
\pgfpathmoveto{\pgfqpoint{6.136557in}{3.201171in}}%
\pgfpathlineto{\pgfqpoint{6.322784in}{3.201171in}}%
\pgfpathlineto{\pgfqpoint{6.322784in}{3.282900in}}%
\pgfpathlineto{\pgfqpoint{6.136557in}{3.282900in}}%
\pgfpathlineto{\pgfqpoint{6.136557in}{3.201171in}}%
\pgfusepath{}%
\end{pgfscope}%
\begin{pgfscope}%
\pgfpathrectangle{\pgfqpoint{0.549740in}{0.463273in}}{\pgfqpoint{9.320225in}{4.495057in}}%
\pgfusepath{clip}%
\pgfsetbuttcap%
\pgfsetroundjoin%
\pgfsetlinewidth{0.000000pt}%
\definecolor{currentstroke}{rgb}{0.000000,0.000000,0.000000}%
\pgfsetstrokecolor{currentstroke}%
\pgfsetdash{}{0pt}%
\pgfpathmoveto{\pgfqpoint{6.322784in}{3.201171in}}%
\pgfpathlineto{\pgfqpoint{6.509011in}{3.201171in}}%
\pgfpathlineto{\pgfqpoint{6.509011in}{3.282900in}}%
\pgfpathlineto{\pgfqpoint{6.322784in}{3.282900in}}%
\pgfpathlineto{\pgfqpoint{6.322784in}{3.201171in}}%
\pgfusepath{}%
\end{pgfscope}%
\begin{pgfscope}%
\pgfpathrectangle{\pgfqpoint{0.549740in}{0.463273in}}{\pgfqpoint{9.320225in}{4.495057in}}%
\pgfusepath{clip}%
\pgfsetbuttcap%
\pgfsetroundjoin%
\pgfsetlinewidth{0.000000pt}%
\definecolor{currentstroke}{rgb}{0.000000,0.000000,0.000000}%
\pgfsetstrokecolor{currentstroke}%
\pgfsetdash{}{0pt}%
\pgfpathmoveto{\pgfqpoint{6.509011in}{3.201171in}}%
\pgfpathlineto{\pgfqpoint{6.695237in}{3.201171in}}%
\pgfpathlineto{\pgfqpoint{6.695237in}{3.282900in}}%
\pgfpathlineto{\pgfqpoint{6.509011in}{3.282900in}}%
\pgfpathlineto{\pgfqpoint{6.509011in}{3.201171in}}%
\pgfusepath{}%
\end{pgfscope}%
\begin{pgfscope}%
\pgfpathrectangle{\pgfqpoint{0.549740in}{0.463273in}}{\pgfqpoint{9.320225in}{4.495057in}}%
\pgfusepath{clip}%
\pgfsetbuttcap%
\pgfsetroundjoin%
\pgfsetlinewidth{0.000000pt}%
\definecolor{currentstroke}{rgb}{0.000000,0.000000,0.000000}%
\pgfsetstrokecolor{currentstroke}%
\pgfsetdash{}{0pt}%
\pgfpathmoveto{\pgfqpoint{6.695237in}{3.201171in}}%
\pgfpathlineto{\pgfqpoint{6.881464in}{3.201171in}}%
\pgfpathlineto{\pgfqpoint{6.881464in}{3.282900in}}%
\pgfpathlineto{\pgfqpoint{6.695237in}{3.282900in}}%
\pgfpathlineto{\pgfqpoint{6.695237in}{3.201171in}}%
\pgfusepath{}%
\end{pgfscope}%
\begin{pgfscope}%
\pgfpathrectangle{\pgfqpoint{0.549740in}{0.463273in}}{\pgfqpoint{9.320225in}{4.495057in}}%
\pgfusepath{clip}%
\pgfsetbuttcap%
\pgfsetroundjoin%
\pgfsetlinewidth{0.000000pt}%
\definecolor{currentstroke}{rgb}{0.000000,0.000000,0.000000}%
\pgfsetstrokecolor{currentstroke}%
\pgfsetdash{}{0pt}%
\pgfpathmoveto{\pgfqpoint{6.881464in}{3.201171in}}%
\pgfpathlineto{\pgfqpoint{7.067690in}{3.201171in}}%
\pgfpathlineto{\pgfqpoint{7.067690in}{3.282900in}}%
\pgfpathlineto{\pgfqpoint{6.881464in}{3.282900in}}%
\pgfpathlineto{\pgfqpoint{6.881464in}{3.201171in}}%
\pgfusepath{}%
\end{pgfscope}%
\begin{pgfscope}%
\pgfpathrectangle{\pgfqpoint{0.549740in}{0.463273in}}{\pgfqpoint{9.320225in}{4.495057in}}%
\pgfusepath{clip}%
\pgfsetbuttcap%
\pgfsetroundjoin%
\definecolor{currentfill}{rgb}{0.472869,0.711325,0.955316}%
\pgfsetfillcolor{currentfill}%
\pgfsetlinewidth{0.000000pt}%
\definecolor{currentstroke}{rgb}{0.000000,0.000000,0.000000}%
\pgfsetstrokecolor{currentstroke}%
\pgfsetdash{}{0pt}%
\pgfpathmoveto{\pgfqpoint{7.067690in}{3.201171in}}%
\pgfpathlineto{\pgfqpoint{7.253917in}{3.201171in}}%
\pgfpathlineto{\pgfqpoint{7.253917in}{3.282900in}}%
\pgfpathlineto{\pgfqpoint{7.067690in}{3.282900in}}%
\pgfpathlineto{\pgfqpoint{7.067690in}{3.201171in}}%
\pgfusepath{fill}%
\end{pgfscope}%
\begin{pgfscope}%
\pgfpathrectangle{\pgfqpoint{0.549740in}{0.463273in}}{\pgfqpoint{9.320225in}{4.495057in}}%
\pgfusepath{clip}%
\pgfsetbuttcap%
\pgfsetroundjoin%
\pgfsetlinewidth{0.000000pt}%
\definecolor{currentstroke}{rgb}{0.000000,0.000000,0.000000}%
\pgfsetstrokecolor{currentstroke}%
\pgfsetdash{}{0pt}%
\pgfpathmoveto{\pgfqpoint{7.253917in}{3.201171in}}%
\pgfpathlineto{\pgfqpoint{7.440143in}{3.201171in}}%
\pgfpathlineto{\pgfqpoint{7.440143in}{3.282900in}}%
\pgfpathlineto{\pgfqpoint{7.253917in}{3.282900in}}%
\pgfpathlineto{\pgfqpoint{7.253917in}{3.201171in}}%
\pgfusepath{}%
\end{pgfscope}%
\begin{pgfscope}%
\pgfpathrectangle{\pgfqpoint{0.549740in}{0.463273in}}{\pgfqpoint{9.320225in}{4.495057in}}%
\pgfusepath{clip}%
\pgfsetbuttcap%
\pgfsetroundjoin%
\pgfsetlinewidth{0.000000pt}%
\definecolor{currentstroke}{rgb}{0.000000,0.000000,0.000000}%
\pgfsetstrokecolor{currentstroke}%
\pgfsetdash{}{0pt}%
\pgfpathmoveto{\pgfqpoint{7.440143in}{3.201171in}}%
\pgfpathlineto{\pgfqpoint{7.626370in}{3.201171in}}%
\pgfpathlineto{\pgfqpoint{7.626370in}{3.282900in}}%
\pgfpathlineto{\pgfqpoint{7.440143in}{3.282900in}}%
\pgfpathlineto{\pgfqpoint{7.440143in}{3.201171in}}%
\pgfusepath{}%
\end{pgfscope}%
\begin{pgfscope}%
\pgfpathrectangle{\pgfqpoint{0.549740in}{0.463273in}}{\pgfqpoint{9.320225in}{4.495057in}}%
\pgfusepath{clip}%
\pgfsetbuttcap%
\pgfsetroundjoin%
\pgfsetlinewidth{0.000000pt}%
\definecolor{currentstroke}{rgb}{0.000000,0.000000,0.000000}%
\pgfsetstrokecolor{currentstroke}%
\pgfsetdash{}{0pt}%
\pgfpathmoveto{\pgfqpoint{7.626370in}{3.201171in}}%
\pgfpathlineto{\pgfqpoint{7.812596in}{3.201171in}}%
\pgfpathlineto{\pgfqpoint{7.812596in}{3.282900in}}%
\pgfpathlineto{\pgfqpoint{7.626370in}{3.282900in}}%
\pgfpathlineto{\pgfqpoint{7.626370in}{3.201171in}}%
\pgfusepath{}%
\end{pgfscope}%
\begin{pgfscope}%
\pgfpathrectangle{\pgfqpoint{0.549740in}{0.463273in}}{\pgfqpoint{9.320225in}{4.495057in}}%
\pgfusepath{clip}%
\pgfsetbuttcap%
\pgfsetroundjoin%
\pgfsetlinewidth{0.000000pt}%
\definecolor{currentstroke}{rgb}{0.000000,0.000000,0.000000}%
\pgfsetstrokecolor{currentstroke}%
\pgfsetdash{}{0pt}%
\pgfpathmoveto{\pgfqpoint{7.812596in}{3.201171in}}%
\pgfpathlineto{\pgfqpoint{7.998823in}{3.201171in}}%
\pgfpathlineto{\pgfqpoint{7.998823in}{3.282900in}}%
\pgfpathlineto{\pgfqpoint{7.812596in}{3.282900in}}%
\pgfpathlineto{\pgfqpoint{7.812596in}{3.201171in}}%
\pgfusepath{}%
\end{pgfscope}%
\begin{pgfscope}%
\pgfpathrectangle{\pgfqpoint{0.549740in}{0.463273in}}{\pgfqpoint{9.320225in}{4.495057in}}%
\pgfusepath{clip}%
\pgfsetbuttcap%
\pgfsetroundjoin%
\definecolor{currentfill}{rgb}{0.547810,0.736432,0.947518}%
\pgfsetfillcolor{currentfill}%
\pgfsetlinewidth{0.000000pt}%
\definecolor{currentstroke}{rgb}{0.000000,0.000000,0.000000}%
\pgfsetstrokecolor{currentstroke}%
\pgfsetdash{}{0pt}%
\pgfpathmoveto{\pgfqpoint{7.998823in}{3.201171in}}%
\pgfpathlineto{\pgfqpoint{8.185049in}{3.201171in}}%
\pgfpathlineto{\pgfqpoint{8.185049in}{3.282900in}}%
\pgfpathlineto{\pgfqpoint{7.998823in}{3.282900in}}%
\pgfpathlineto{\pgfqpoint{7.998823in}{3.201171in}}%
\pgfusepath{fill}%
\end{pgfscope}%
\begin{pgfscope}%
\pgfpathrectangle{\pgfqpoint{0.549740in}{0.463273in}}{\pgfqpoint{9.320225in}{4.495057in}}%
\pgfusepath{clip}%
\pgfsetbuttcap%
\pgfsetroundjoin%
\definecolor{currentfill}{rgb}{0.614330,0.761948,0.940009}%
\pgfsetfillcolor{currentfill}%
\pgfsetlinewidth{0.000000pt}%
\definecolor{currentstroke}{rgb}{0.000000,0.000000,0.000000}%
\pgfsetstrokecolor{currentstroke}%
\pgfsetdash{}{0pt}%
\pgfpathmoveto{\pgfqpoint{8.185049in}{3.201171in}}%
\pgfpathlineto{\pgfqpoint{8.371276in}{3.201171in}}%
\pgfpathlineto{\pgfqpoint{8.371276in}{3.282900in}}%
\pgfpathlineto{\pgfqpoint{8.185049in}{3.282900in}}%
\pgfpathlineto{\pgfqpoint{8.185049in}{3.201171in}}%
\pgfusepath{fill}%
\end{pgfscope}%
\begin{pgfscope}%
\pgfpathrectangle{\pgfqpoint{0.549740in}{0.463273in}}{\pgfqpoint{9.320225in}{4.495057in}}%
\pgfusepath{clip}%
\pgfsetbuttcap%
\pgfsetroundjoin%
\pgfsetlinewidth{0.000000pt}%
\definecolor{currentstroke}{rgb}{0.000000,0.000000,0.000000}%
\pgfsetstrokecolor{currentstroke}%
\pgfsetdash{}{0pt}%
\pgfpathmoveto{\pgfqpoint{8.371276in}{3.201171in}}%
\pgfpathlineto{\pgfqpoint{8.557503in}{3.201171in}}%
\pgfpathlineto{\pgfqpoint{8.557503in}{3.282900in}}%
\pgfpathlineto{\pgfqpoint{8.371276in}{3.282900in}}%
\pgfpathlineto{\pgfqpoint{8.371276in}{3.201171in}}%
\pgfusepath{}%
\end{pgfscope}%
\begin{pgfscope}%
\pgfpathrectangle{\pgfqpoint{0.549740in}{0.463273in}}{\pgfqpoint{9.320225in}{4.495057in}}%
\pgfusepath{clip}%
\pgfsetbuttcap%
\pgfsetroundjoin%
\pgfsetlinewidth{0.000000pt}%
\definecolor{currentstroke}{rgb}{0.000000,0.000000,0.000000}%
\pgfsetstrokecolor{currentstroke}%
\pgfsetdash{}{0pt}%
\pgfpathmoveto{\pgfqpoint{8.557503in}{3.201171in}}%
\pgfpathlineto{\pgfqpoint{8.743729in}{3.201171in}}%
\pgfpathlineto{\pgfqpoint{8.743729in}{3.282900in}}%
\pgfpathlineto{\pgfqpoint{8.557503in}{3.282900in}}%
\pgfpathlineto{\pgfqpoint{8.557503in}{3.201171in}}%
\pgfusepath{}%
\end{pgfscope}%
\begin{pgfscope}%
\pgfpathrectangle{\pgfqpoint{0.549740in}{0.463273in}}{\pgfqpoint{9.320225in}{4.495057in}}%
\pgfusepath{clip}%
\pgfsetbuttcap%
\pgfsetroundjoin%
\pgfsetlinewidth{0.000000pt}%
\definecolor{currentstroke}{rgb}{0.000000,0.000000,0.000000}%
\pgfsetstrokecolor{currentstroke}%
\pgfsetdash{}{0pt}%
\pgfpathmoveto{\pgfqpoint{8.743729in}{3.201171in}}%
\pgfpathlineto{\pgfqpoint{8.929956in}{3.201171in}}%
\pgfpathlineto{\pgfqpoint{8.929956in}{3.282900in}}%
\pgfpathlineto{\pgfqpoint{8.743729in}{3.282900in}}%
\pgfpathlineto{\pgfqpoint{8.743729in}{3.201171in}}%
\pgfusepath{}%
\end{pgfscope}%
\begin{pgfscope}%
\pgfpathrectangle{\pgfqpoint{0.549740in}{0.463273in}}{\pgfqpoint{9.320225in}{4.495057in}}%
\pgfusepath{clip}%
\pgfsetbuttcap%
\pgfsetroundjoin%
\pgfsetlinewidth{0.000000pt}%
\definecolor{currentstroke}{rgb}{0.000000,0.000000,0.000000}%
\pgfsetstrokecolor{currentstroke}%
\pgfsetdash{}{0pt}%
\pgfpathmoveto{\pgfqpoint{8.929956in}{3.201171in}}%
\pgfpathlineto{\pgfqpoint{9.116182in}{3.201171in}}%
\pgfpathlineto{\pgfqpoint{9.116182in}{3.282900in}}%
\pgfpathlineto{\pgfqpoint{8.929956in}{3.282900in}}%
\pgfpathlineto{\pgfqpoint{8.929956in}{3.201171in}}%
\pgfusepath{}%
\end{pgfscope}%
\begin{pgfscope}%
\pgfpathrectangle{\pgfqpoint{0.549740in}{0.463273in}}{\pgfqpoint{9.320225in}{4.495057in}}%
\pgfusepath{clip}%
\pgfsetbuttcap%
\pgfsetroundjoin%
\pgfsetlinewidth{0.000000pt}%
\definecolor{currentstroke}{rgb}{0.000000,0.000000,0.000000}%
\pgfsetstrokecolor{currentstroke}%
\pgfsetdash{}{0pt}%
\pgfpathmoveto{\pgfqpoint{9.116182in}{3.201171in}}%
\pgfpathlineto{\pgfqpoint{9.302409in}{3.201171in}}%
\pgfpathlineto{\pgfqpoint{9.302409in}{3.282900in}}%
\pgfpathlineto{\pgfqpoint{9.116182in}{3.282900in}}%
\pgfpathlineto{\pgfqpoint{9.116182in}{3.201171in}}%
\pgfusepath{}%
\end{pgfscope}%
\begin{pgfscope}%
\pgfpathrectangle{\pgfqpoint{0.549740in}{0.463273in}}{\pgfqpoint{9.320225in}{4.495057in}}%
\pgfusepath{clip}%
\pgfsetbuttcap%
\pgfsetroundjoin%
\definecolor{currentfill}{rgb}{0.472869,0.711325,0.955316}%
\pgfsetfillcolor{currentfill}%
\pgfsetlinewidth{0.000000pt}%
\definecolor{currentstroke}{rgb}{0.000000,0.000000,0.000000}%
\pgfsetstrokecolor{currentstroke}%
\pgfsetdash{}{0pt}%
\pgfpathmoveto{\pgfqpoint{9.302409in}{3.201171in}}%
\pgfpathlineto{\pgfqpoint{9.488635in}{3.201171in}}%
\pgfpathlineto{\pgfqpoint{9.488635in}{3.282900in}}%
\pgfpathlineto{\pgfqpoint{9.302409in}{3.282900in}}%
\pgfpathlineto{\pgfqpoint{9.302409in}{3.201171in}}%
\pgfusepath{fill}%
\end{pgfscope}%
\begin{pgfscope}%
\pgfpathrectangle{\pgfqpoint{0.549740in}{0.463273in}}{\pgfqpoint{9.320225in}{4.495057in}}%
\pgfusepath{clip}%
\pgfsetbuttcap%
\pgfsetroundjoin%
\pgfsetlinewidth{0.000000pt}%
\definecolor{currentstroke}{rgb}{0.000000,0.000000,0.000000}%
\pgfsetstrokecolor{currentstroke}%
\pgfsetdash{}{0pt}%
\pgfpathmoveto{\pgfqpoint{9.488635in}{3.201171in}}%
\pgfpathlineto{\pgfqpoint{9.674862in}{3.201171in}}%
\pgfpathlineto{\pgfqpoint{9.674862in}{3.282900in}}%
\pgfpathlineto{\pgfqpoint{9.488635in}{3.282900in}}%
\pgfpathlineto{\pgfqpoint{9.488635in}{3.201171in}}%
\pgfusepath{}%
\end{pgfscope}%
\begin{pgfscope}%
\pgfpathrectangle{\pgfqpoint{0.549740in}{0.463273in}}{\pgfqpoint{9.320225in}{4.495057in}}%
\pgfusepath{clip}%
\pgfsetbuttcap%
\pgfsetroundjoin%
\pgfsetlinewidth{0.000000pt}%
\definecolor{currentstroke}{rgb}{0.000000,0.000000,0.000000}%
\pgfsetstrokecolor{currentstroke}%
\pgfsetdash{}{0pt}%
\pgfpathmoveto{\pgfqpoint{9.674862in}{3.201171in}}%
\pgfpathlineto{\pgfqpoint{9.861088in}{3.201171in}}%
\pgfpathlineto{\pgfqpoint{9.861088in}{3.282900in}}%
\pgfpathlineto{\pgfqpoint{9.674862in}{3.282900in}}%
\pgfpathlineto{\pgfqpoint{9.674862in}{3.201171in}}%
\pgfusepath{}%
\end{pgfscope}%
\begin{pgfscope}%
\pgfpathrectangle{\pgfqpoint{0.549740in}{0.463273in}}{\pgfqpoint{9.320225in}{4.495057in}}%
\pgfusepath{clip}%
\pgfsetbuttcap%
\pgfsetroundjoin%
\pgfsetlinewidth{0.000000pt}%
\definecolor{currentstroke}{rgb}{0.000000,0.000000,0.000000}%
\pgfsetstrokecolor{currentstroke}%
\pgfsetdash{}{0pt}%
\pgfpathmoveto{\pgfqpoint{0.549761in}{3.282900in}}%
\pgfpathlineto{\pgfqpoint{0.735988in}{3.282900in}}%
\pgfpathlineto{\pgfqpoint{0.735988in}{3.364628in}}%
\pgfpathlineto{\pgfqpoint{0.549761in}{3.364628in}}%
\pgfpathlineto{\pgfqpoint{0.549761in}{3.282900in}}%
\pgfusepath{}%
\end{pgfscope}%
\begin{pgfscope}%
\pgfpathrectangle{\pgfqpoint{0.549740in}{0.463273in}}{\pgfqpoint{9.320225in}{4.495057in}}%
\pgfusepath{clip}%
\pgfsetbuttcap%
\pgfsetroundjoin%
\pgfsetlinewidth{0.000000pt}%
\definecolor{currentstroke}{rgb}{0.000000,0.000000,0.000000}%
\pgfsetstrokecolor{currentstroke}%
\pgfsetdash{}{0pt}%
\pgfpathmoveto{\pgfqpoint{0.735988in}{3.282900in}}%
\pgfpathlineto{\pgfqpoint{0.922214in}{3.282900in}}%
\pgfpathlineto{\pgfqpoint{0.922214in}{3.364628in}}%
\pgfpathlineto{\pgfqpoint{0.735988in}{3.364628in}}%
\pgfpathlineto{\pgfqpoint{0.735988in}{3.282900in}}%
\pgfusepath{}%
\end{pgfscope}%
\begin{pgfscope}%
\pgfpathrectangle{\pgfqpoint{0.549740in}{0.463273in}}{\pgfqpoint{9.320225in}{4.495057in}}%
\pgfusepath{clip}%
\pgfsetbuttcap%
\pgfsetroundjoin%
\pgfsetlinewidth{0.000000pt}%
\definecolor{currentstroke}{rgb}{0.000000,0.000000,0.000000}%
\pgfsetstrokecolor{currentstroke}%
\pgfsetdash{}{0pt}%
\pgfpathmoveto{\pgfqpoint{0.922214in}{3.282900in}}%
\pgfpathlineto{\pgfqpoint{1.108441in}{3.282900in}}%
\pgfpathlineto{\pgfqpoint{1.108441in}{3.364628in}}%
\pgfpathlineto{\pgfqpoint{0.922214in}{3.364628in}}%
\pgfpathlineto{\pgfqpoint{0.922214in}{3.282900in}}%
\pgfusepath{}%
\end{pgfscope}%
\begin{pgfscope}%
\pgfpathrectangle{\pgfqpoint{0.549740in}{0.463273in}}{\pgfqpoint{9.320225in}{4.495057in}}%
\pgfusepath{clip}%
\pgfsetbuttcap%
\pgfsetroundjoin%
\pgfsetlinewidth{0.000000pt}%
\definecolor{currentstroke}{rgb}{0.000000,0.000000,0.000000}%
\pgfsetstrokecolor{currentstroke}%
\pgfsetdash{}{0pt}%
\pgfpathmoveto{\pgfqpoint{1.108441in}{3.282900in}}%
\pgfpathlineto{\pgfqpoint{1.294667in}{3.282900in}}%
\pgfpathlineto{\pgfqpoint{1.294667in}{3.364628in}}%
\pgfpathlineto{\pgfqpoint{1.108441in}{3.364628in}}%
\pgfpathlineto{\pgfqpoint{1.108441in}{3.282900in}}%
\pgfusepath{}%
\end{pgfscope}%
\begin{pgfscope}%
\pgfpathrectangle{\pgfqpoint{0.549740in}{0.463273in}}{\pgfqpoint{9.320225in}{4.495057in}}%
\pgfusepath{clip}%
\pgfsetbuttcap%
\pgfsetroundjoin%
\pgfsetlinewidth{0.000000pt}%
\definecolor{currentstroke}{rgb}{0.000000,0.000000,0.000000}%
\pgfsetstrokecolor{currentstroke}%
\pgfsetdash{}{0pt}%
\pgfpathmoveto{\pgfqpoint{1.294667in}{3.282900in}}%
\pgfpathlineto{\pgfqpoint{1.480894in}{3.282900in}}%
\pgfpathlineto{\pgfqpoint{1.480894in}{3.364628in}}%
\pgfpathlineto{\pgfqpoint{1.294667in}{3.364628in}}%
\pgfpathlineto{\pgfqpoint{1.294667in}{3.282900in}}%
\pgfusepath{}%
\end{pgfscope}%
\begin{pgfscope}%
\pgfpathrectangle{\pgfqpoint{0.549740in}{0.463273in}}{\pgfqpoint{9.320225in}{4.495057in}}%
\pgfusepath{clip}%
\pgfsetbuttcap%
\pgfsetroundjoin%
\pgfsetlinewidth{0.000000pt}%
\definecolor{currentstroke}{rgb}{0.000000,0.000000,0.000000}%
\pgfsetstrokecolor{currentstroke}%
\pgfsetdash{}{0pt}%
\pgfpathmoveto{\pgfqpoint{1.480894in}{3.282900in}}%
\pgfpathlineto{\pgfqpoint{1.667120in}{3.282900in}}%
\pgfpathlineto{\pgfqpoint{1.667120in}{3.364628in}}%
\pgfpathlineto{\pgfqpoint{1.480894in}{3.364628in}}%
\pgfpathlineto{\pgfqpoint{1.480894in}{3.282900in}}%
\pgfusepath{}%
\end{pgfscope}%
\begin{pgfscope}%
\pgfpathrectangle{\pgfqpoint{0.549740in}{0.463273in}}{\pgfqpoint{9.320225in}{4.495057in}}%
\pgfusepath{clip}%
\pgfsetbuttcap%
\pgfsetroundjoin%
\pgfsetlinewidth{0.000000pt}%
\definecolor{currentstroke}{rgb}{0.000000,0.000000,0.000000}%
\pgfsetstrokecolor{currentstroke}%
\pgfsetdash{}{0pt}%
\pgfpathmoveto{\pgfqpoint{1.667120in}{3.282900in}}%
\pgfpathlineto{\pgfqpoint{1.853347in}{3.282900in}}%
\pgfpathlineto{\pgfqpoint{1.853347in}{3.364628in}}%
\pgfpathlineto{\pgfqpoint{1.667120in}{3.364628in}}%
\pgfpathlineto{\pgfqpoint{1.667120in}{3.282900in}}%
\pgfusepath{}%
\end{pgfscope}%
\begin{pgfscope}%
\pgfpathrectangle{\pgfqpoint{0.549740in}{0.463273in}}{\pgfqpoint{9.320225in}{4.495057in}}%
\pgfusepath{clip}%
\pgfsetbuttcap%
\pgfsetroundjoin%
\pgfsetlinewidth{0.000000pt}%
\definecolor{currentstroke}{rgb}{0.000000,0.000000,0.000000}%
\pgfsetstrokecolor{currentstroke}%
\pgfsetdash{}{0pt}%
\pgfpathmoveto{\pgfqpoint{1.853347in}{3.282900in}}%
\pgfpathlineto{\pgfqpoint{2.039573in}{3.282900in}}%
\pgfpathlineto{\pgfqpoint{2.039573in}{3.364628in}}%
\pgfpathlineto{\pgfqpoint{1.853347in}{3.364628in}}%
\pgfpathlineto{\pgfqpoint{1.853347in}{3.282900in}}%
\pgfusepath{}%
\end{pgfscope}%
\begin{pgfscope}%
\pgfpathrectangle{\pgfqpoint{0.549740in}{0.463273in}}{\pgfqpoint{9.320225in}{4.495057in}}%
\pgfusepath{clip}%
\pgfsetbuttcap%
\pgfsetroundjoin%
\pgfsetlinewidth{0.000000pt}%
\definecolor{currentstroke}{rgb}{0.000000,0.000000,0.000000}%
\pgfsetstrokecolor{currentstroke}%
\pgfsetdash{}{0pt}%
\pgfpathmoveto{\pgfqpoint{2.039573in}{3.282900in}}%
\pgfpathlineto{\pgfqpoint{2.225800in}{3.282900in}}%
\pgfpathlineto{\pgfqpoint{2.225800in}{3.364628in}}%
\pgfpathlineto{\pgfqpoint{2.039573in}{3.364628in}}%
\pgfpathlineto{\pgfqpoint{2.039573in}{3.282900in}}%
\pgfusepath{}%
\end{pgfscope}%
\begin{pgfscope}%
\pgfpathrectangle{\pgfqpoint{0.549740in}{0.463273in}}{\pgfqpoint{9.320225in}{4.495057in}}%
\pgfusepath{clip}%
\pgfsetbuttcap%
\pgfsetroundjoin%
\pgfsetlinewidth{0.000000pt}%
\definecolor{currentstroke}{rgb}{0.000000,0.000000,0.000000}%
\pgfsetstrokecolor{currentstroke}%
\pgfsetdash{}{0pt}%
\pgfpathmoveto{\pgfqpoint{2.225800in}{3.282900in}}%
\pgfpathlineto{\pgfqpoint{2.412027in}{3.282900in}}%
\pgfpathlineto{\pgfqpoint{2.412027in}{3.364628in}}%
\pgfpathlineto{\pgfqpoint{2.225800in}{3.364628in}}%
\pgfpathlineto{\pgfqpoint{2.225800in}{3.282900in}}%
\pgfusepath{}%
\end{pgfscope}%
\begin{pgfscope}%
\pgfpathrectangle{\pgfqpoint{0.549740in}{0.463273in}}{\pgfqpoint{9.320225in}{4.495057in}}%
\pgfusepath{clip}%
\pgfsetbuttcap%
\pgfsetroundjoin%
\pgfsetlinewidth{0.000000pt}%
\definecolor{currentstroke}{rgb}{0.000000,0.000000,0.000000}%
\pgfsetstrokecolor{currentstroke}%
\pgfsetdash{}{0pt}%
\pgfpathmoveto{\pgfqpoint{2.412027in}{3.282900in}}%
\pgfpathlineto{\pgfqpoint{2.598253in}{3.282900in}}%
\pgfpathlineto{\pgfqpoint{2.598253in}{3.364628in}}%
\pgfpathlineto{\pgfqpoint{2.412027in}{3.364628in}}%
\pgfpathlineto{\pgfqpoint{2.412027in}{3.282900in}}%
\pgfusepath{}%
\end{pgfscope}%
\begin{pgfscope}%
\pgfpathrectangle{\pgfqpoint{0.549740in}{0.463273in}}{\pgfqpoint{9.320225in}{4.495057in}}%
\pgfusepath{clip}%
\pgfsetbuttcap%
\pgfsetroundjoin%
\pgfsetlinewidth{0.000000pt}%
\definecolor{currentstroke}{rgb}{0.000000,0.000000,0.000000}%
\pgfsetstrokecolor{currentstroke}%
\pgfsetdash{}{0pt}%
\pgfpathmoveto{\pgfqpoint{2.598253in}{3.282900in}}%
\pgfpathlineto{\pgfqpoint{2.784480in}{3.282900in}}%
\pgfpathlineto{\pgfqpoint{2.784480in}{3.364628in}}%
\pgfpathlineto{\pgfqpoint{2.598253in}{3.364628in}}%
\pgfpathlineto{\pgfqpoint{2.598253in}{3.282900in}}%
\pgfusepath{}%
\end{pgfscope}%
\begin{pgfscope}%
\pgfpathrectangle{\pgfqpoint{0.549740in}{0.463273in}}{\pgfqpoint{9.320225in}{4.495057in}}%
\pgfusepath{clip}%
\pgfsetbuttcap%
\pgfsetroundjoin%
\pgfsetlinewidth{0.000000pt}%
\definecolor{currentstroke}{rgb}{0.000000,0.000000,0.000000}%
\pgfsetstrokecolor{currentstroke}%
\pgfsetdash{}{0pt}%
\pgfpathmoveto{\pgfqpoint{2.784480in}{3.282900in}}%
\pgfpathlineto{\pgfqpoint{2.970706in}{3.282900in}}%
\pgfpathlineto{\pgfqpoint{2.970706in}{3.364628in}}%
\pgfpathlineto{\pgfqpoint{2.784480in}{3.364628in}}%
\pgfpathlineto{\pgfqpoint{2.784480in}{3.282900in}}%
\pgfusepath{}%
\end{pgfscope}%
\begin{pgfscope}%
\pgfpathrectangle{\pgfqpoint{0.549740in}{0.463273in}}{\pgfqpoint{9.320225in}{4.495057in}}%
\pgfusepath{clip}%
\pgfsetbuttcap%
\pgfsetroundjoin%
\pgfsetlinewidth{0.000000pt}%
\definecolor{currentstroke}{rgb}{0.000000,0.000000,0.000000}%
\pgfsetstrokecolor{currentstroke}%
\pgfsetdash{}{0pt}%
\pgfpathmoveto{\pgfqpoint{2.970706in}{3.282900in}}%
\pgfpathlineto{\pgfqpoint{3.156933in}{3.282900in}}%
\pgfpathlineto{\pgfqpoint{3.156933in}{3.364628in}}%
\pgfpathlineto{\pgfqpoint{2.970706in}{3.364628in}}%
\pgfpathlineto{\pgfqpoint{2.970706in}{3.282900in}}%
\pgfusepath{}%
\end{pgfscope}%
\begin{pgfscope}%
\pgfpathrectangle{\pgfqpoint{0.549740in}{0.463273in}}{\pgfqpoint{9.320225in}{4.495057in}}%
\pgfusepath{clip}%
\pgfsetbuttcap%
\pgfsetroundjoin%
\pgfsetlinewidth{0.000000pt}%
\definecolor{currentstroke}{rgb}{0.000000,0.000000,0.000000}%
\pgfsetstrokecolor{currentstroke}%
\pgfsetdash{}{0pt}%
\pgfpathmoveto{\pgfqpoint{3.156933in}{3.282900in}}%
\pgfpathlineto{\pgfqpoint{3.343159in}{3.282900in}}%
\pgfpathlineto{\pgfqpoint{3.343159in}{3.364628in}}%
\pgfpathlineto{\pgfqpoint{3.156933in}{3.364628in}}%
\pgfpathlineto{\pgfqpoint{3.156933in}{3.282900in}}%
\pgfusepath{}%
\end{pgfscope}%
\begin{pgfscope}%
\pgfpathrectangle{\pgfqpoint{0.549740in}{0.463273in}}{\pgfqpoint{9.320225in}{4.495057in}}%
\pgfusepath{clip}%
\pgfsetbuttcap%
\pgfsetroundjoin%
\pgfsetlinewidth{0.000000pt}%
\definecolor{currentstroke}{rgb}{0.000000,0.000000,0.000000}%
\pgfsetstrokecolor{currentstroke}%
\pgfsetdash{}{0pt}%
\pgfpathmoveto{\pgfqpoint{3.343159in}{3.282900in}}%
\pgfpathlineto{\pgfqpoint{3.529386in}{3.282900in}}%
\pgfpathlineto{\pgfqpoint{3.529386in}{3.364628in}}%
\pgfpathlineto{\pgfqpoint{3.343159in}{3.364628in}}%
\pgfpathlineto{\pgfqpoint{3.343159in}{3.282900in}}%
\pgfusepath{}%
\end{pgfscope}%
\begin{pgfscope}%
\pgfpathrectangle{\pgfqpoint{0.549740in}{0.463273in}}{\pgfqpoint{9.320225in}{4.495057in}}%
\pgfusepath{clip}%
\pgfsetbuttcap%
\pgfsetroundjoin%
\pgfsetlinewidth{0.000000pt}%
\definecolor{currentstroke}{rgb}{0.000000,0.000000,0.000000}%
\pgfsetstrokecolor{currentstroke}%
\pgfsetdash{}{0pt}%
\pgfpathmoveto{\pgfqpoint{3.529386in}{3.282900in}}%
\pgfpathlineto{\pgfqpoint{3.715612in}{3.282900in}}%
\pgfpathlineto{\pgfqpoint{3.715612in}{3.364628in}}%
\pgfpathlineto{\pgfqpoint{3.529386in}{3.364628in}}%
\pgfpathlineto{\pgfqpoint{3.529386in}{3.282900in}}%
\pgfusepath{}%
\end{pgfscope}%
\begin{pgfscope}%
\pgfpathrectangle{\pgfqpoint{0.549740in}{0.463273in}}{\pgfqpoint{9.320225in}{4.495057in}}%
\pgfusepath{clip}%
\pgfsetbuttcap%
\pgfsetroundjoin%
\pgfsetlinewidth{0.000000pt}%
\definecolor{currentstroke}{rgb}{0.000000,0.000000,0.000000}%
\pgfsetstrokecolor{currentstroke}%
\pgfsetdash{}{0pt}%
\pgfpathmoveto{\pgfqpoint{3.715612in}{3.282900in}}%
\pgfpathlineto{\pgfqpoint{3.901839in}{3.282900in}}%
\pgfpathlineto{\pgfqpoint{3.901839in}{3.364628in}}%
\pgfpathlineto{\pgfqpoint{3.715612in}{3.364628in}}%
\pgfpathlineto{\pgfqpoint{3.715612in}{3.282900in}}%
\pgfusepath{}%
\end{pgfscope}%
\begin{pgfscope}%
\pgfpathrectangle{\pgfqpoint{0.549740in}{0.463273in}}{\pgfqpoint{9.320225in}{4.495057in}}%
\pgfusepath{clip}%
\pgfsetbuttcap%
\pgfsetroundjoin%
\pgfsetlinewidth{0.000000pt}%
\definecolor{currentstroke}{rgb}{0.000000,0.000000,0.000000}%
\pgfsetstrokecolor{currentstroke}%
\pgfsetdash{}{0pt}%
\pgfpathmoveto{\pgfqpoint{3.901839in}{3.282900in}}%
\pgfpathlineto{\pgfqpoint{4.088065in}{3.282900in}}%
\pgfpathlineto{\pgfqpoint{4.088065in}{3.364628in}}%
\pgfpathlineto{\pgfqpoint{3.901839in}{3.364628in}}%
\pgfpathlineto{\pgfqpoint{3.901839in}{3.282900in}}%
\pgfusepath{}%
\end{pgfscope}%
\begin{pgfscope}%
\pgfpathrectangle{\pgfqpoint{0.549740in}{0.463273in}}{\pgfqpoint{9.320225in}{4.495057in}}%
\pgfusepath{clip}%
\pgfsetbuttcap%
\pgfsetroundjoin%
\pgfsetlinewidth{0.000000pt}%
\definecolor{currentstroke}{rgb}{0.000000,0.000000,0.000000}%
\pgfsetstrokecolor{currentstroke}%
\pgfsetdash{}{0pt}%
\pgfpathmoveto{\pgfqpoint{4.088065in}{3.282900in}}%
\pgfpathlineto{\pgfqpoint{4.274292in}{3.282900in}}%
\pgfpathlineto{\pgfqpoint{4.274292in}{3.364628in}}%
\pgfpathlineto{\pgfqpoint{4.088065in}{3.364628in}}%
\pgfpathlineto{\pgfqpoint{4.088065in}{3.282900in}}%
\pgfusepath{}%
\end{pgfscope}%
\begin{pgfscope}%
\pgfpathrectangle{\pgfqpoint{0.549740in}{0.463273in}}{\pgfqpoint{9.320225in}{4.495057in}}%
\pgfusepath{clip}%
\pgfsetbuttcap%
\pgfsetroundjoin%
\pgfsetlinewidth{0.000000pt}%
\definecolor{currentstroke}{rgb}{0.000000,0.000000,0.000000}%
\pgfsetstrokecolor{currentstroke}%
\pgfsetdash{}{0pt}%
\pgfpathmoveto{\pgfqpoint{4.274292in}{3.282900in}}%
\pgfpathlineto{\pgfqpoint{4.460519in}{3.282900in}}%
\pgfpathlineto{\pgfqpoint{4.460519in}{3.364628in}}%
\pgfpathlineto{\pgfqpoint{4.274292in}{3.364628in}}%
\pgfpathlineto{\pgfqpoint{4.274292in}{3.282900in}}%
\pgfusepath{}%
\end{pgfscope}%
\begin{pgfscope}%
\pgfpathrectangle{\pgfqpoint{0.549740in}{0.463273in}}{\pgfqpoint{9.320225in}{4.495057in}}%
\pgfusepath{clip}%
\pgfsetbuttcap%
\pgfsetroundjoin%
\pgfsetlinewidth{0.000000pt}%
\definecolor{currentstroke}{rgb}{0.000000,0.000000,0.000000}%
\pgfsetstrokecolor{currentstroke}%
\pgfsetdash{}{0pt}%
\pgfpathmoveto{\pgfqpoint{4.460519in}{3.282900in}}%
\pgfpathlineto{\pgfqpoint{4.646745in}{3.282900in}}%
\pgfpathlineto{\pgfqpoint{4.646745in}{3.364628in}}%
\pgfpathlineto{\pgfqpoint{4.460519in}{3.364628in}}%
\pgfpathlineto{\pgfqpoint{4.460519in}{3.282900in}}%
\pgfusepath{}%
\end{pgfscope}%
\begin{pgfscope}%
\pgfpathrectangle{\pgfqpoint{0.549740in}{0.463273in}}{\pgfqpoint{9.320225in}{4.495057in}}%
\pgfusepath{clip}%
\pgfsetbuttcap%
\pgfsetroundjoin%
\pgfsetlinewidth{0.000000pt}%
\definecolor{currentstroke}{rgb}{0.000000,0.000000,0.000000}%
\pgfsetstrokecolor{currentstroke}%
\pgfsetdash{}{0pt}%
\pgfpathmoveto{\pgfqpoint{4.646745in}{3.282900in}}%
\pgfpathlineto{\pgfqpoint{4.832972in}{3.282900in}}%
\pgfpathlineto{\pgfqpoint{4.832972in}{3.364628in}}%
\pgfpathlineto{\pgfqpoint{4.646745in}{3.364628in}}%
\pgfpathlineto{\pgfqpoint{4.646745in}{3.282900in}}%
\pgfusepath{}%
\end{pgfscope}%
\begin{pgfscope}%
\pgfpathrectangle{\pgfqpoint{0.549740in}{0.463273in}}{\pgfqpoint{9.320225in}{4.495057in}}%
\pgfusepath{clip}%
\pgfsetbuttcap%
\pgfsetroundjoin%
\pgfsetlinewidth{0.000000pt}%
\definecolor{currentstroke}{rgb}{0.000000,0.000000,0.000000}%
\pgfsetstrokecolor{currentstroke}%
\pgfsetdash{}{0pt}%
\pgfpathmoveto{\pgfqpoint{4.832972in}{3.282900in}}%
\pgfpathlineto{\pgfqpoint{5.019198in}{3.282900in}}%
\pgfpathlineto{\pgfqpoint{5.019198in}{3.364628in}}%
\pgfpathlineto{\pgfqpoint{4.832972in}{3.364628in}}%
\pgfpathlineto{\pgfqpoint{4.832972in}{3.282900in}}%
\pgfusepath{}%
\end{pgfscope}%
\begin{pgfscope}%
\pgfpathrectangle{\pgfqpoint{0.549740in}{0.463273in}}{\pgfqpoint{9.320225in}{4.495057in}}%
\pgfusepath{clip}%
\pgfsetbuttcap%
\pgfsetroundjoin%
\pgfsetlinewidth{0.000000pt}%
\definecolor{currentstroke}{rgb}{0.000000,0.000000,0.000000}%
\pgfsetstrokecolor{currentstroke}%
\pgfsetdash{}{0pt}%
\pgfpathmoveto{\pgfqpoint{5.019198in}{3.282900in}}%
\pgfpathlineto{\pgfqpoint{5.205425in}{3.282900in}}%
\pgfpathlineto{\pgfqpoint{5.205425in}{3.364628in}}%
\pgfpathlineto{\pgfqpoint{5.019198in}{3.364628in}}%
\pgfpathlineto{\pgfqpoint{5.019198in}{3.282900in}}%
\pgfusepath{}%
\end{pgfscope}%
\begin{pgfscope}%
\pgfpathrectangle{\pgfqpoint{0.549740in}{0.463273in}}{\pgfqpoint{9.320225in}{4.495057in}}%
\pgfusepath{clip}%
\pgfsetbuttcap%
\pgfsetroundjoin%
\pgfsetlinewidth{0.000000pt}%
\definecolor{currentstroke}{rgb}{0.000000,0.000000,0.000000}%
\pgfsetstrokecolor{currentstroke}%
\pgfsetdash{}{0pt}%
\pgfpathmoveto{\pgfqpoint{5.205425in}{3.282900in}}%
\pgfpathlineto{\pgfqpoint{5.391651in}{3.282900in}}%
\pgfpathlineto{\pgfqpoint{5.391651in}{3.364628in}}%
\pgfpathlineto{\pgfqpoint{5.205425in}{3.364628in}}%
\pgfpathlineto{\pgfqpoint{5.205425in}{3.282900in}}%
\pgfusepath{}%
\end{pgfscope}%
\begin{pgfscope}%
\pgfpathrectangle{\pgfqpoint{0.549740in}{0.463273in}}{\pgfqpoint{9.320225in}{4.495057in}}%
\pgfusepath{clip}%
\pgfsetbuttcap%
\pgfsetroundjoin%
\pgfsetlinewidth{0.000000pt}%
\definecolor{currentstroke}{rgb}{0.000000,0.000000,0.000000}%
\pgfsetstrokecolor{currentstroke}%
\pgfsetdash{}{0pt}%
\pgfpathmoveto{\pgfqpoint{5.391651in}{3.282900in}}%
\pgfpathlineto{\pgfqpoint{5.577878in}{3.282900in}}%
\pgfpathlineto{\pgfqpoint{5.577878in}{3.364628in}}%
\pgfpathlineto{\pgfqpoint{5.391651in}{3.364628in}}%
\pgfpathlineto{\pgfqpoint{5.391651in}{3.282900in}}%
\pgfusepath{}%
\end{pgfscope}%
\begin{pgfscope}%
\pgfpathrectangle{\pgfqpoint{0.549740in}{0.463273in}}{\pgfqpoint{9.320225in}{4.495057in}}%
\pgfusepath{clip}%
\pgfsetbuttcap%
\pgfsetroundjoin%
\pgfsetlinewidth{0.000000pt}%
\definecolor{currentstroke}{rgb}{0.000000,0.000000,0.000000}%
\pgfsetstrokecolor{currentstroke}%
\pgfsetdash{}{0pt}%
\pgfpathmoveto{\pgfqpoint{5.577878in}{3.282900in}}%
\pgfpathlineto{\pgfqpoint{5.764104in}{3.282900in}}%
\pgfpathlineto{\pgfqpoint{5.764104in}{3.364628in}}%
\pgfpathlineto{\pgfqpoint{5.577878in}{3.364628in}}%
\pgfpathlineto{\pgfqpoint{5.577878in}{3.282900in}}%
\pgfusepath{}%
\end{pgfscope}%
\begin{pgfscope}%
\pgfpathrectangle{\pgfqpoint{0.549740in}{0.463273in}}{\pgfqpoint{9.320225in}{4.495057in}}%
\pgfusepath{clip}%
\pgfsetbuttcap%
\pgfsetroundjoin%
\pgfsetlinewidth{0.000000pt}%
\definecolor{currentstroke}{rgb}{0.000000,0.000000,0.000000}%
\pgfsetstrokecolor{currentstroke}%
\pgfsetdash{}{0pt}%
\pgfpathmoveto{\pgfqpoint{5.764104in}{3.282900in}}%
\pgfpathlineto{\pgfqpoint{5.950331in}{3.282900in}}%
\pgfpathlineto{\pgfqpoint{5.950331in}{3.364628in}}%
\pgfpathlineto{\pgfqpoint{5.764104in}{3.364628in}}%
\pgfpathlineto{\pgfqpoint{5.764104in}{3.282900in}}%
\pgfusepath{}%
\end{pgfscope}%
\begin{pgfscope}%
\pgfpathrectangle{\pgfqpoint{0.549740in}{0.463273in}}{\pgfqpoint{9.320225in}{4.495057in}}%
\pgfusepath{clip}%
\pgfsetbuttcap%
\pgfsetroundjoin%
\definecolor{currentfill}{rgb}{0.472869,0.711325,0.955316}%
\pgfsetfillcolor{currentfill}%
\pgfsetlinewidth{0.000000pt}%
\definecolor{currentstroke}{rgb}{0.000000,0.000000,0.000000}%
\pgfsetstrokecolor{currentstroke}%
\pgfsetdash{}{0pt}%
\pgfpathmoveto{\pgfqpoint{5.950331in}{3.282900in}}%
\pgfpathlineto{\pgfqpoint{6.136557in}{3.282900in}}%
\pgfpathlineto{\pgfqpoint{6.136557in}{3.364628in}}%
\pgfpathlineto{\pgfqpoint{5.950331in}{3.364628in}}%
\pgfpathlineto{\pgfqpoint{5.950331in}{3.282900in}}%
\pgfusepath{fill}%
\end{pgfscope}%
\begin{pgfscope}%
\pgfpathrectangle{\pgfqpoint{0.549740in}{0.463273in}}{\pgfqpoint{9.320225in}{4.495057in}}%
\pgfusepath{clip}%
\pgfsetbuttcap%
\pgfsetroundjoin%
\pgfsetlinewidth{0.000000pt}%
\definecolor{currentstroke}{rgb}{0.000000,0.000000,0.000000}%
\pgfsetstrokecolor{currentstroke}%
\pgfsetdash{}{0pt}%
\pgfpathmoveto{\pgfqpoint{6.136557in}{3.282900in}}%
\pgfpathlineto{\pgfqpoint{6.322784in}{3.282900in}}%
\pgfpathlineto{\pgfqpoint{6.322784in}{3.364628in}}%
\pgfpathlineto{\pgfqpoint{6.136557in}{3.364628in}}%
\pgfpathlineto{\pgfqpoint{6.136557in}{3.282900in}}%
\pgfusepath{}%
\end{pgfscope}%
\begin{pgfscope}%
\pgfpathrectangle{\pgfqpoint{0.549740in}{0.463273in}}{\pgfqpoint{9.320225in}{4.495057in}}%
\pgfusepath{clip}%
\pgfsetbuttcap%
\pgfsetroundjoin%
\pgfsetlinewidth{0.000000pt}%
\definecolor{currentstroke}{rgb}{0.000000,0.000000,0.000000}%
\pgfsetstrokecolor{currentstroke}%
\pgfsetdash{}{0pt}%
\pgfpathmoveto{\pgfqpoint{6.322784in}{3.282900in}}%
\pgfpathlineto{\pgfqpoint{6.509011in}{3.282900in}}%
\pgfpathlineto{\pgfqpoint{6.509011in}{3.364628in}}%
\pgfpathlineto{\pgfqpoint{6.322784in}{3.364628in}}%
\pgfpathlineto{\pgfqpoint{6.322784in}{3.282900in}}%
\pgfusepath{}%
\end{pgfscope}%
\begin{pgfscope}%
\pgfpathrectangle{\pgfqpoint{0.549740in}{0.463273in}}{\pgfqpoint{9.320225in}{4.495057in}}%
\pgfusepath{clip}%
\pgfsetbuttcap%
\pgfsetroundjoin%
\pgfsetlinewidth{0.000000pt}%
\definecolor{currentstroke}{rgb}{0.000000,0.000000,0.000000}%
\pgfsetstrokecolor{currentstroke}%
\pgfsetdash{}{0pt}%
\pgfpathmoveto{\pgfqpoint{6.509011in}{3.282900in}}%
\pgfpathlineto{\pgfqpoint{6.695237in}{3.282900in}}%
\pgfpathlineto{\pgfqpoint{6.695237in}{3.364628in}}%
\pgfpathlineto{\pgfqpoint{6.509011in}{3.364628in}}%
\pgfpathlineto{\pgfqpoint{6.509011in}{3.282900in}}%
\pgfusepath{}%
\end{pgfscope}%
\begin{pgfscope}%
\pgfpathrectangle{\pgfqpoint{0.549740in}{0.463273in}}{\pgfqpoint{9.320225in}{4.495057in}}%
\pgfusepath{clip}%
\pgfsetbuttcap%
\pgfsetroundjoin%
\pgfsetlinewidth{0.000000pt}%
\definecolor{currentstroke}{rgb}{0.000000,0.000000,0.000000}%
\pgfsetstrokecolor{currentstroke}%
\pgfsetdash{}{0pt}%
\pgfpathmoveto{\pgfqpoint{6.695237in}{3.282900in}}%
\pgfpathlineto{\pgfqpoint{6.881464in}{3.282900in}}%
\pgfpathlineto{\pgfqpoint{6.881464in}{3.364628in}}%
\pgfpathlineto{\pgfqpoint{6.695237in}{3.364628in}}%
\pgfpathlineto{\pgfqpoint{6.695237in}{3.282900in}}%
\pgfusepath{}%
\end{pgfscope}%
\begin{pgfscope}%
\pgfpathrectangle{\pgfqpoint{0.549740in}{0.463273in}}{\pgfqpoint{9.320225in}{4.495057in}}%
\pgfusepath{clip}%
\pgfsetbuttcap%
\pgfsetroundjoin%
\pgfsetlinewidth{0.000000pt}%
\definecolor{currentstroke}{rgb}{0.000000,0.000000,0.000000}%
\pgfsetstrokecolor{currentstroke}%
\pgfsetdash{}{0pt}%
\pgfpathmoveto{\pgfqpoint{6.881464in}{3.282900in}}%
\pgfpathlineto{\pgfqpoint{7.067690in}{3.282900in}}%
\pgfpathlineto{\pgfqpoint{7.067690in}{3.364628in}}%
\pgfpathlineto{\pgfqpoint{6.881464in}{3.364628in}}%
\pgfpathlineto{\pgfqpoint{6.881464in}{3.282900in}}%
\pgfusepath{}%
\end{pgfscope}%
\begin{pgfscope}%
\pgfpathrectangle{\pgfqpoint{0.549740in}{0.463273in}}{\pgfqpoint{9.320225in}{4.495057in}}%
\pgfusepath{clip}%
\pgfsetbuttcap%
\pgfsetroundjoin%
\definecolor{currentfill}{rgb}{0.472869,0.711325,0.955316}%
\pgfsetfillcolor{currentfill}%
\pgfsetlinewidth{0.000000pt}%
\definecolor{currentstroke}{rgb}{0.000000,0.000000,0.000000}%
\pgfsetstrokecolor{currentstroke}%
\pgfsetdash{}{0pt}%
\pgfpathmoveto{\pgfqpoint{7.067690in}{3.282900in}}%
\pgfpathlineto{\pgfqpoint{7.253917in}{3.282900in}}%
\pgfpathlineto{\pgfqpoint{7.253917in}{3.364628in}}%
\pgfpathlineto{\pgfqpoint{7.067690in}{3.364628in}}%
\pgfpathlineto{\pgfqpoint{7.067690in}{3.282900in}}%
\pgfusepath{fill}%
\end{pgfscope}%
\begin{pgfscope}%
\pgfpathrectangle{\pgfqpoint{0.549740in}{0.463273in}}{\pgfqpoint{9.320225in}{4.495057in}}%
\pgfusepath{clip}%
\pgfsetbuttcap%
\pgfsetroundjoin%
\pgfsetlinewidth{0.000000pt}%
\definecolor{currentstroke}{rgb}{0.000000,0.000000,0.000000}%
\pgfsetstrokecolor{currentstroke}%
\pgfsetdash{}{0pt}%
\pgfpathmoveto{\pgfqpoint{7.253917in}{3.282900in}}%
\pgfpathlineto{\pgfqpoint{7.440143in}{3.282900in}}%
\pgfpathlineto{\pgfqpoint{7.440143in}{3.364628in}}%
\pgfpathlineto{\pgfqpoint{7.253917in}{3.364628in}}%
\pgfpathlineto{\pgfqpoint{7.253917in}{3.282900in}}%
\pgfusepath{}%
\end{pgfscope}%
\begin{pgfscope}%
\pgfpathrectangle{\pgfqpoint{0.549740in}{0.463273in}}{\pgfqpoint{9.320225in}{4.495057in}}%
\pgfusepath{clip}%
\pgfsetbuttcap%
\pgfsetroundjoin%
\pgfsetlinewidth{0.000000pt}%
\definecolor{currentstroke}{rgb}{0.000000,0.000000,0.000000}%
\pgfsetstrokecolor{currentstroke}%
\pgfsetdash{}{0pt}%
\pgfpathmoveto{\pgfqpoint{7.440143in}{3.282900in}}%
\pgfpathlineto{\pgfqpoint{7.626370in}{3.282900in}}%
\pgfpathlineto{\pgfqpoint{7.626370in}{3.364628in}}%
\pgfpathlineto{\pgfqpoint{7.440143in}{3.364628in}}%
\pgfpathlineto{\pgfqpoint{7.440143in}{3.282900in}}%
\pgfusepath{}%
\end{pgfscope}%
\begin{pgfscope}%
\pgfpathrectangle{\pgfqpoint{0.549740in}{0.463273in}}{\pgfqpoint{9.320225in}{4.495057in}}%
\pgfusepath{clip}%
\pgfsetbuttcap%
\pgfsetroundjoin%
\pgfsetlinewidth{0.000000pt}%
\definecolor{currentstroke}{rgb}{0.000000,0.000000,0.000000}%
\pgfsetstrokecolor{currentstroke}%
\pgfsetdash{}{0pt}%
\pgfpathmoveto{\pgfqpoint{7.626370in}{3.282900in}}%
\pgfpathlineto{\pgfqpoint{7.812596in}{3.282900in}}%
\pgfpathlineto{\pgfqpoint{7.812596in}{3.364628in}}%
\pgfpathlineto{\pgfqpoint{7.626370in}{3.364628in}}%
\pgfpathlineto{\pgfqpoint{7.626370in}{3.282900in}}%
\pgfusepath{}%
\end{pgfscope}%
\begin{pgfscope}%
\pgfpathrectangle{\pgfqpoint{0.549740in}{0.463273in}}{\pgfqpoint{9.320225in}{4.495057in}}%
\pgfusepath{clip}%
\pgfsetbuttcap%
\pgfsetroundjoin%
\pgfsetlinewidth{0.000000pt}%
\definecolor{currentstroke}{rgb}{0.000000,0.000000,0.000000}%
\pgfsetstrokecolor{currentstroke}%
\pgfsetdash{}{0pt}%
\pgfpathmoveto{\pgfqpoint{7.812596in}{3.282900in}}%
\pgfpathlineto{\pgfqpoint{7.998823in}{3.282900in}}%
\pgfpathlineto{\pgfqpoint{7.998823in}{3.364628in}}%
\pgfpathlineto{\pgfqpoint{7.812596in}{3.364628in}}%
\pgfpathlineto{\pgfqpoint{7.812596in}{3.282900in}}%
\pgfusepath{}%
\end{pgfscope}%
\begin{pgfscope}%
\pgfpathrectangle{\pgfqpoint{0.549740in}{0.463273in}}{\pgfqpoint{9.320225in}{4.495057in}}%
\pgfusepath{clip}%
\pgfsetbuttcap%
\pgfsetroundjoin%
\definecolor{currentfill}{rgb}{0.472869,0.711325,0.955316}%
\pgfsetfillcolor{currentfill}%
\pgfsetlinewidth{0.000000pt}%
\definecolor{currentstroke}{rgb}{0.000000,0.000000,0.000000}%
\pgfsetstrokecolor{currentstroke}%
\pgfsetdash{}{0pt}%
\pgfpathmoveto{\pgfqpoint{7.998823in}{3.282900in}}%
\pgfpathlineto{\pgfqpoint{8.185049in}{3.282900in}}%
\pgfpathlineto{\pgfqpoint{8.185049in}{3.364628in}}%
\pgfpathlineto{\pgfqpoint{7.998823in}{3.364628in}}%
\pgfpathlineto{\pgfqpoint{7.998823in}{3.282900in}}%
\pgfusepath{fill}%
\end{pgfscope}%
\begin{pgfscope}%
\pgfpathrectangle{\pgfqpoint{0.549740in}{0.463273in}}{\pgfqpoint{9.320225in}{4.495057in}}%
\pgfusepath{clip}%
\pgfsetbuttcap%
\pgfsetroundjoin%
\pgfsetlinewidth{0.000000pt}%
\definecolor{currentstroke}{rgb}{0.000000,0.000000,0.000000}%
\pgfsetstrokecolor{currentstroke}%
\pgfsetdash{}{0pt}%
\pgfpathmoveto{\pgfqpoint{8.185049in}{3.282900in}}%
\pgfpathlineto{\pgfqpoint{8.371276in}{3.282900in}}%
\pgfpathlineto{\pgfqpoint{8.371276in}{3.364628in}}%
\pgfpathlineto{\pgfqpoint{8.185049in}{3.364628in}}%
\pgfpathlineto{\pgfqpoint{8.185049in}{3.282900in}}%
\pgfusepath{}%
\end{pgfscope}%
\begin{pgfscope}%
\pgfpathrectangle{\pgfqpoint{0.549740in}{0.463273in}}{\pgfqpoint{9.320225in}{4.495057in}}%
\pgfusepath{clip}%
\pgfsetbuttcap%
\pgfsetroundjoin%
\pgfsetlinewidth{0.000000pt}%
\definecolor{currentstroke}{rgb}{0.000000,0.000000,0.000000}%
\pgfsetstrokecolor{currentstroke}%
\pgfsetdash{}{0pt}%
\pgfpathmoveto{\pgfqpoint{8.371276in}{3.282900in}}%
\pgfpathlineto{\pgfqpoint{8.557503in}{3.282900in}}%
\pgfpathlineto{\pgfqpoint{8.557503in}{3.364628in}}%
\pgfpathlineto{\pgfqpoint{8.371276in}{3.364628in}}%
\pgfpathlineto{\pgfqpoint{8.371276in}{3.282900in}}%
\pgfusepath{}%
\end{pgfscope}%
\begin{pgfscope}%
\pgfpathrectangle{\pgfqpoint{0.549740in}{0.463273in}}{\pgfqpoint{9.320225in}{4.495057in}}%
\pgfusepath{clip}%
\pgfsetbuttcap%
\pgfsetroundjoin%
\pgfsetlinewidth{0.000000pt}%
\definecolor{currentstroke}{rgb}{0.000000,0.000000,0.000000}%
\pgfsetstrokecolor{currentstroke}%
\pgfsetdash{}{0pt}%
\pgfpathmoveto{\pgfqpoint{8.557503in}{3.282900in}}%
\pgfpathlineto{\pgfqpoint{8.743729in}{3.282900in}}%
\pgfpathlineto{\pgfqpoint{8.743729in}{3.364628in}}%
\pgfpathlineto{\pgfqpoint{8.557503in}{3.364628in}}%
\pgfpathlineto{\pgfqpoint{8.557503in}{3.282900in}}%
\pgfusepath{}%
\end{pgfscope}%
\begin{pgfscope}%
\pgfpathrectangle{\pgfqpoint{0.549740in}{0.463273in}}{\pgfqpoint{9.320225in}{4.495057in}}%
\pgfusepath{clip}%
\pgfsetbuttcap%
\pgfsetroundjoin%
\pgfsetlinewidth{0.000000pt}%
\definecolor{currentstroke}{rgb}{0.000000,0.000000,0.000000}%
\pgfsetstrokecolor{currentstroke}%
\pgfsetdash{}{0pt}%
\pgfpathmoveto{\pgfqpoint{8.743729in}{3.282900in}}%
\pgfpathlineto{\pgfqpoint{8.929956in}{3.282900in}}%
\pgfpathlineto{\pgfqpoint{8.929956in}{3.364628in}}%
\pgfpathlineto{\pgfqpoint{8.743729in}{3.364628in}}%
\pgfpathlineto{\pgfqpoint{8.743729in}{3.282900in}}%
\pgfusepath{}%
\end{pgfscope}%
\begin{pgfscope}%
\pgfpathrectangle{\pgfqpoint{0.549740in}{0.463273in}}{\pgfqpoint{9.320225in}{4.495057in}}%
\pgfusepath{clip}%
\pgfsetbuttcap%
\pgfsetroundjoin%
\pgfsetlinewidth{0.000000pt}%
\definecolor{currentstroke}{rgb}{0.000000,0.000000,0.000000}%
\pgfsetstrokecolor{currentstroke}%
\pgfsetdash{}{0pt}%
\pgfpathmoveto{\pgfqpoint{8.929956in}{3.282900in}}%
\pgfpathlineto{\pgfqpoint{9.116182in}{3.282900in}}%
\pgfpathlineto{\pgfqpoint{9.116182in}{3.364628in}}%
\pgfpathlineto{\pgfqpoint{8.929956in}{3.364628in}}%
\pgfpathlineto{\pgfqpoint{8.929956in}{3.282900in}}%
\pgfusepath{}%
\end{pgfscope}%
\begin{pgfscope}%
\pgfpathrectangle{\pgfqpoint{0.549740in}{0.463273in}}{\pgfqpoint{9.320225in}{4.495057in}}%
\pgfusepath{clip}%
\pgfsetbuttcap%
\pgfsetroundjoin%
\pgfsetlinewidth{0.000000pt}%
\definecolor{currentstroke}{rgb}{0.000000,0.000000,0.000000}%
\pgfsetstrokecolor{currentstroke}%
\pgfsetdash{}{0pt}%
\pgfpathmoveto{\pgfqpoint{9.116182in}{3.282900in}}%
\pgfpathlineto{\pgfqpoint{9.302409in}{3.282900in}}%
\pgfpathlineto{\pgfqpoint{9.302409in}{3.364628in}}%
\pgfpathlineto{\pgfqpoint{9.116182in}{3.364628in}}%
\pgfpathlineto{\pgfqpoint{9.116182in}{3.282900in}}%
\pgfusepath{}%
\end{pgfscope}%
\begin{pgfscope}%
\pgfpathrectangle{\pgfqpoint{0.549740in}{0.463273in}}{\pgfqpoint{9.320225in}{4.495057in}}%
\pgfusepath{clip}%
\pgfsetbuttcap%
\pgfsetroundjoin%
\definecolor{currentfill}{rgb}{0.472869,0.711325,0.955316}%
\pgfsetfillcolor{currentfill}%
\pgfsetlinewidth{0.000000pt}%
\definecolor{currentstroke}{rgb}{0.000000,0.000000,0.000000}%
\pgfsetstrokecolor{currentstroke}%
\pgfsetdash{}{0pt}%
\pgfpathmoveto{\pgfqpoint{9.302409in}{3.282900in}}%
\pgfpathlineto{\pgfqpoint{9.488635in}{3.282900in}}%
\pgfpathlineto{\pgfqpoint{9.488635in}{3.364628in}}%
\pgfpathlineto{\pgfqpoint{9.302409in}{3.364628in}}%
\pgfpathlineto{\pgfqpoint{9.302409in}{3.282900in}}%
\pgfusepath{fill}%
\end{pgfscope}%
\begin{pgfscope}%
\pgfpathrectangle{\pgfqpoint{0.549740in}{0.463273in}}{\pgfqpoint{9.320225in}{4.495057in}}%
\pgfusepath{clip}%
\pgfsetbuttcap%
\pgfsetroundjoin%
\pgfsetlinewidth{0.000000pt}%
\definecolor{currentstroke}{rgb}{0.000000,0.000000,0.000000}%
\pgfsetstrokecolor{currentstroke}%
\pgfsetdash{}{0pt}%
\pgfpathmoveto{\pgfqpoint{9.488635in}{3.282900in}}%
\pgfpathlineto{\pgfqpoint{9.674862in}{3.282900in}}%
\pgfpathlineto{\pgfqpoint{9.674862in}{3.364628in}}%
\pgfpathlineto{\pgfqpoint{9.488635in}{3.364628in}}%
\pgfpathlineto{\pgfqpoint{9.488635in}{3.282900in}}%
\pgfusepath{}%
\end{pgfscope}%
\begin{pgfscope}%
\pgfpathrectangle{\pgfqpoint{0.549740in}{0.463273in}}{\pgfqpoint{9.320225in}{4.495057in}}%
\pgfusepath{clip}%
\pgfsetbuttcap%
\pgfsetroundjoin%
\pgfsetlinewidth{0.000000pt}%
\definecolor{currentstroke}{rgb}{0.000000,0.000000,0.000000}%
\pgfsetstrokecolor{currentstroke}%
\pgfsetdash{}{0pt}%
\pgfpathmoveto{\pgfqpoint{9.674862in}{3.282900in}}%
\pgfpathlineto{\pgfqpoint{9.861088in}{3.282900in}}%
\pgfpathlineto{\pgfqpoint{9.861088in}{3.364628in}}%
\pgfpathlineto{\pgfqpoint{9.674862in}{3.364628in}}%
\pgfpathlineto{\pgfqpoint{9.674862in}{3.282900in}}%
\pgfusepath{}%
\end{pgfscope}%
\begin{pgfscope}%
\pgfpathrectangle{\pgfqpoint{0.549740in}{0.463273in}}{\pgfqpoint{9.320225in}{4.495057in}}%
\pgfusepath{clip}%
\pgfsetbuttcap%
\pgfsetroundjoin%
\pgfsetlinewidth{0.000000pt}%
\definecolor{currentstroke}{rgb}{0.000000,0.000000,0.000000}%
\pgfsetstrokecolor{currentstroke}%
\pgfsetdash{}{0pt}%
\pgfpathmoveto{\pgfqpoint{0.549761in}{3.364628in}}%
\pgfpathlineto{\pgfqpoint{0.735988in}{3.364628in}}%
\pgfpathlineto{\pgfqpoint{0.735988in}{3.446356in}}%
\pgfpathlineto{\pgfqpoint{0.549761in}{3.446356in}}%
\pgfpathlineto{\pgfqpoint{0.549761in}{3.364628in}}%
\pgfusepath{}%
\end{pgfscope}%
\begin{pgfscope}%
\pgfpathrectangle{\pgfqpoint{0.549740in}{0.463273in}}{\pgfqpoint{9.320225in}{4.495057in}}%
\pgfusepath{clip}%
\pgfsetbuttcap%
\pgfsetroundjoin%
\pgfsetlinewidth{0.000000pt}%
\definecolor{currentstroke}{rgb}{0.000000,0.000000,0.000000}%
\pgfsetstrokecolor{currentstroke}%
\pgfsetdash{}{0pt}%
\pgfpathmoveto{\pgfqpoint{0.735988in}{3.364628in}}%
\pgfpathlineto{\pgfqpoint{0.922214in}{3.364628in}}%
\pgfpathlineto{\pgfqpoint{0.922214in}{3.446356in}}%
\pgfpathlineto{\pgfqpoint{0.735988in}{3.446356in}}%
\pgfpathlineto{\pgfqpoint{0.735988in}{3.364628in}}%
\pgfusepath{}%
\end{pgfscope}%
\begin{pgfscope}%
\pgfpathrectangle{\pgfqpoint{0.549740in}{0.463273in}}{\pgfqpoint{9.320225in}{4.495057in}}%
\pgfusepath{clip}%
\pgfsetbuttcap%
\pgfsetroundjoin%
\pgfsetlinewidth{0.000000pt}%
\definecolor{currentstroke}{rgb}{0.000000,0.000000,0.000000}%
\pgfsetstrokecolor{currentstroke}%
\pgfsetdash{}{0pt}%
\pgfpathmoveto{\pgfqpoint{0.922214in}{3.364628in}}%
\pgfpathlineto{\pgfqpoint{1.108441in}{3.364628in}}%
\pgfpathlineto{\pgfqpoint{1.108441in}{3.446356in}}%
\pgfpathlineto{\pgfqpoint{0.922214in}{3.446356in}}%
\pgfpathlineto{\pgfqpoint{0.922214in}{3.364628in}}%
\pgfusepath{}%
\end{pgfscope}%
\begin{pgfscope}%
\pgfpathrectangle{\pgfqpoint{0.549740in}{0.463273in}}{\pgfqpoint{9.320225in}{4.495057in}}%
\pgfusepath{clip}%
\pgfsetbuttcap%
\pgfsetroundjoin%
\pgfsetlinewidth{0.000000pt}%
\definecolor{currentstroke}{rgb}{0.000000,0.000000,0.000000}%
\pgfsetstrokecolor{currentstroke}%
\pgfsetdash{}{0pt}%
\pgfpathmoveto{\pgfqpoint{1.108441in}{3.364628in}}%
\pgfpathlineto{\pgfqpoint{1.294667in}{3.364628in}}%
\pgfpathlineto{\pgfqpoint{1.294667in}{3.446356in}}%
\pgfpathlineto{\pgfqpoint{1.108441in}{3.446356in}}%
\pgfpathlineto{\pgfqpoint{1.108441in}{3.364628in}}%
\pgfusepath{}%
\end{pgfscope}%
\begin{pgfscope}%
\pgfpathrectangle{\pgfqpoint{0.549740in}{0.463273in}}{\pgfqpoint{9.320225in}{4.495057in}}%
\pgfusepath{clip}%
\pgfsetbuttcap%
\pgfsetroundjoin%
\pgfsetlinewidth{0.000000pt}%
\definecolor{currentstroke}{rgb}{0.000000,0.000000,0.000000}%
\pgfsetstrokecolor{currentstroke}%
\pgfsetdash{}{0pt}%
\pgfpathmoveto{\pgfqpoint{1.294667in}{3.364628in}}%
\pgfpathlineto{\pgfqpoint{1.480894in}{3.364628in}}%
\pgfpathlineto{\pgfqpoint{1.480894in}{3.446356in}}%
\pgfpathlineto{\pgfqpoint{1.294667in}{3.446356in}}%
\pgfpathlineto{\pgfqpoint{1.294667in}{3.364628in}}%
\pgfusepath{}%
\end{pgfscope}%
\begin{pgfscope}%
\pgfpathrectangle{\pgfqpoint{0.549740in}{0.463273in}}{\pgfqpoint{9.320225in}{4.495057in}}%
\pgfusepath{clip}%
\pgfsetbuttcap%
\pgfsetroundjoin%
\pgfsetlinewidth{0.000000pt}%
\definecolor{currentstroke}{rgb}{0.000000,0.000000,0.000000}%
\pgfsetstrokecolor{currentstroke}%
\pgfsetdash{}{0pt}%
\pgfpathmoveto{\pgfqpoint{1.480894in}{3.364628in}}%
\pgfpathlineto{\pgfqpoint{1.667120in}{3.364628in}}%
\pgfpathlineto{\pgfqpoint{1.667120in}{3.446356in}}%
\pgfpathlineto{\pgfqpoint{1.480894in}{3.446356in}}%
\pgfpathlineto{\pgfqpoint{1.480894in}{3.364628in}}%
\pgfusepath{}%
\end{pgfscope}%
\begin{pgfscope}%
\pgfpathrectangle{\pgfqpoint{0.549740in}{0.463273in}}{\pgfqpoint{9.320225in}{4.495057in}}%
\pgfusepath{clip}%
\pgfsetbuttcap%
\pgfsetroundjoin%
\pgfsetlinewidth{0.000000pt}%
\definecolor{currentstroke}{rgb}{0.000000,0.000000,0.000000}%
\pgfsetstrokecolor{currentstroke}%
\pgfsetdash{}{0pt}%
\pgfpathmoveto{\pgfqpoint{1.667120in}{3.364628in}}%
\pgfpathlineto{\pgfqpoint{1.853347in}{3.364628in}}%
\pgfpathlineto{\pgfqpoint{1.853347in}{3.446356in}}%
\pgfpathlineto{\pgfqpoint{1.667120in}{3.446356in}}%
\pgfpathlineto{\pgfqpoint{1.667120in}{3.364628in}}%
\pgfusepath{}%
\end{pgfscope}%
\begin{pgfscope}%
\pgfpathrectangle{\pgfqpoint{0.549740in}{0.463273in}}{\pgfqpoint{9.320225in}{4.495057in}}%
\pgfusepath{clip}%
\pgfsetbuttcap%
\pgfsetroundjoin%
\pgfsetlinewidth{0.000000pt}%
\definecolor{currentstroke}{rgb}{0.000000,0.000000,0.000000}%
\pgfsetstrokecolor{currentstroke}%
\pgfsetdash{}{0pt}%
\pgfpathmoveto{\pgfqpoint{1.853347in}{3.364628in}}%
\pgfpathlineto{\pgfqpoint{2.039573in}{3.364628in}}%
\pgfpathlineto{\pgfqpoint{2.039573in}{3.446356in}}%
\pgfpathlineto{\pgfqpoint{1.853347in}{3.446356in}}%
\pgfpathlineto{\pgfqpoint{1.853347in}{3.364628in}}%
\pgfusepath{}%
\end{pgfscope}%
\begin{pgfscope}%
\pgfpathrectangle{\pgfqpoint{0.549740in}{0.463273in}}{\pgfqpoint{9.320225in}{4.495057in}}%
\pgfusepath{clip}%
\pgfsetbuttcap%
\pgfsetroundjoin%
\pgfsetlinewidth{0.000000pt}%
\definecolor{currentstroke}{rgb}{0.000000,0.000000,0.000000}%
\pgfsetstrokecolor{currentstroke}%
\pgfsetdash{}{0pt}%
\pgfpathmoveto{\pgfqpoint{2.039573in}{3.364628in}}%
\pgfpathlineto{\pgfqpoint{2.225800in}{3.364628in}}%
\pgfpathlineto{\pgfqpoint{2.225800in}{3.446356in}}%
\pgfpathlineto{\pgfqpoint{2.039573in}{3.446356in}}%
\pgfpathlineto{\pgfqpoint{2.039573in}{3.364628in}}%
\pgfusepath{}%
\end{pgfscope}%
\begin{pgfscope}%
\pgfpathrectangle{\pgfqpoint{0.549740in}{0.463273in}}{\pgfqpoint{9.320225in}{4.495057in}}%
\pgfusepath{clip}%
\pgfsetbuttcap%
\pgfsetroundjoin%
\pgfsetlinewidth{0.000000pt}%
\definecolor{currentstroke}{rgb}{0.000000,0.000000,0.000000}%
\pgfsetstrokecolor{currentstroke}%
\pgfsetdash{}{0pt}%
\pgfpathmoveto{\pgfqpoint{2.225800in}{3.364628in}}%
\pgfpathlineto{\pgfqpoint{2.412027in}{3.364628in}}%
\pgfpathlineto{\pgfqpoint{2.412027in}{3.446356in}}%
\pgfpathlineto{\pgfqpoint{2.225800in}{3.446356in}}%
\pgfpathlineto{\pgfqpoint{2.225800in}{3.364628in}}%
\pgfusepath{}%
\end{pgfscope}%
\begin{pgfscope}%
\pgfpathrectangle{\pgfqpoint{0.549740in}{0.463273in}}{\pgfqpoint{9.320225in}{4.495057in}}%
\pgfusepath{clip}%
\pgfsetbuttcap%
\pgfsetroundjoin%
\pgfsetlinewidth{0.000000pt}%
\definecolor{currentstroke}{rgb}{0.000000,0.000000,0.000000}%
\pgfsetstrokecolor{currentstroke}%
\pgfsetdash{}{0pt}%
\pgfpathmoveto{\pgfqpoint{2.412027in}{3.364628in}}%
\pgfpathlineto{\pgfqpoint{2.598253in}{3.364628in}}%
\pgfpathlineto{\pgfqpoint{2.598253in}{3.446356in}}%
\pgfpathlineto{\pgfqpoint{2.412027in}{3.446356in}}%
\pgfpathlineto{\pgfqpoint{2.412027in}{3.364628in}}%
\pgfusepath{}%
\end{pgfscope}%
\begin{pgfscope}%
\pgfpathrectangle{\pgfqpoint{0.549740in}{0.463273in}}{\pgfqpoint{9.320225in}{4.495057in}}%
\pgfusepath{clip}%
\pgfsetbuttcap%
\pgfsetroundjoin%
\pgfsetlinewidth{0.000000pt}%
\definecolor{currentstroke}{rgb}{0.000000,0.000000,0.000000}%
\pgfsetstrokecolor{currentstroke}%
\pgfsetdash{}{0pt}%
\pgfpathmoveto{\pgfqpoint{2.598253in}{3.364628in}}%
\pgfpathlineto{\pgfqpoint{2.784480in}{3.364628in}}%
\pgfpathlineto{\pgfqpoint{2.784480in}{3.446356in}}%
\pgfpathlineto{\pgfqpoint{2.598253in}{3.446356in}}%
\pgfpathlineto{\pgfqpoint{2.598253in}{3.364628in}}%
\pgfusepath{}%
\end{pgfscope}%
\begin{pgfscope}%
\pgfpathrectangle{\pgfqpoint{0.549740in}{0.463273in}}{\pgfqpoint{9.320225in}{4.495057in}}%
\pgfusepath{clip}%
\pgfsetbuttcap%
\pgfsetroundjoin%
\pgfsetlinewidth{0.000000pt}%
\definecolor{currentstroke}{rgb}{0.000000,0.000000,0.000000}%
\pgfsetstrokecolor{currentstroke}%
\pgfsetdash{}{0pt}%
\pgfpathmoveto{\pgfqpoint{2.784480in}{3.364628in}}%
\pgfpathlineto{\pgfqpoint{2.970706in}{3.364628in}}%
\pgfpathlineto{\pgfqpoint{2.970706in}{3.446356in}}%
\pgfpathlineto{\pgfqpoint{2.784480in}{3.446356in}}%
\pgfpathlineto{\pgfqpoint{2.784480in}{3.364628in}}%
\pgfusepath{}%
\end{pgfscope}%
\begin{pgfscope}%
\pgfpathrectangle{\pgfqpoint{0.549740in}{0.463273in}}{\pgfqpoint{9.320225in}{4.495057in}}%
\pgfusepath{clip}%
\pgfsetbuttcap%
\pgfsetroundjoin%
\pgfsetlinewidth{0.000000pt}%
\definecolor{currentstroke}{rgb}{0.000000,0.000000,0.000000}%
\pgfsetstrokecolor{currentstroke}%
\pgfsetdash{}{0pt}%
\pgfpathmoveto{\pgfqpoint{2.970706in}{3.364628in}}%
\pgfpathlineto{\pgfqpoint{3.156933in}{3.364628in}}%
\pgfpathlineto{\pgfqpoint{3.156933in}{3.446356in}}%
\pgfpathlineto{\pgfqpoint{2.970706in}{3.446356in}}%
\pgfpathlineto{\pgfqpoint{2.970706in}{3.364628in}}%
\pgfusepath{}%
\end{pgfscope}%
\begin{pgfscope}%
\pgfpathrectangle{\pgfqpoint{0.549740in}{0.463273in}}{\pgfqpoint{9.320225in}{4.495057in}}%
\pgfusepath{clip}%
\pgfsetbuttcap%
\pgfsetroundjoin%
\pgfsetlinewidth{0.000000pt}%
\definecolor{currentstroke}{rgb}{0.000000,0.000000,0.000000}%
\pgfsetstrokecolor{currentstroke}%
\pgfsetdash{}{0pt}%
\pgfpathmoveto{\pgfqpoint{3.156933in}{3.364628in}}%
\pgfpathlineto{\pgfqpoint{3.343159in}{3.364628in}}%
\pgfpathlineto{\pgfqpoint{3.343159in}{3.446356in}}%
\pgfpathlineto{\pgfqpoint{3.156933in}{3.446356in}}%
\pgfpathlineto{\pgfqpoint{3.156933in}{3.364628in}}%
\pgfusepath{}%
\end{pgfscope}%
\begin{pgfscope}%
\pgfpathrectangle{\pgfqpoint{0.549740in}{0.463273in}}{\pgfqpoint{9.320225in}{4.495057in}}%
\pgfusepath{clip}%
\pgfsetbuttcap%
\pgfsetroundjoin%
\pgfsetlinewidth{0.000000pt}%
\definecolor{currentstroke}{rgb}{0.000000,0.000000,0.000000}%
\pgfsetstrokecolor{currentstroke}%
\pgfsetdash{}{0pt}%
\pgfpathmoveto{\pgfqpoint{3.343159in}{3.364628in}}%
\pgfpathlineto{\pgfqpoint{3.529386in}{3.364628in}}%
\pgfpathlineto{\pgfqpoint{3.529386in}{3.446356in}}%
\pgfpathlineto{\pgfqpoint{3.343159in}{3.446356in}}%
\pgfpathlineto{\pgfqpoint{3.343159in}{3.364628in}}%
\pgfusepath{}%
\end{pgfscope}%
\begin{pgfscope}%
\pgfpathrectangle{\pgfqpoint{0.549740in}{0.463273in}}{\pgfqpoint{9.320225in}{4.495057in}}%
\pgfusepath{clip}%
\pgfsetbuttcap%
\pgfsetroundjoin%
\pgfsetlinewidth{0.000000pt}%
\definecolor{currentstroke}{rgb}{0.000000,0.000000,0.000000}%
\pgfsetstrokecolor{currentstroke}%
\pgfsetdash{}{0pt}%
\pgfpathmoveto{\pgfqpoint{3.529386in}{3.364628in}}%
\pgfpathlineto{\pgfqpoint{3.715612in}{3.364628in}}%
\pgfpathlineto{\pgfqpoint{3.715612in}{3.446356in}}%
\pgfpathlineto{\pgfqpoint{3.529386in}{3.446356in}}%
\pgfpathlineto{\pgfqpoint{3.529386in}{3.364628in}}%
\pgfusepath{}%
\end{pgfscope}%
\begin{pgfscope}%
\pgfpathrectangle{\pgfqpoint{0.549740in}{0.463273in}}{\pgfqpoint{9.320225in}{4.495057in}}%
\pgfusepath{clip}%
\pgfsetbuttcap%
\pgfsetroundjoin%
\pgfsetlinewidth{0.000000pt}%
\definecolor{currentstroke}{rgb}{0.000000,0.000000,0.000000}%
\pgfsetstrokecolor{currentstroke}%
\pgfsetdash{}{0pt}%
\pgfpathmoveto{\pgfqpoint{3.715612in}{3.364628in}}%
\pgfpathlineto{\pgfqpoint{3.901839in}{3.364628in}}%
\pgfpathlineto{\pgfqpoint{3.901839in}{3.446356in}}%
\pgfpathlineto{\pgfqpoint{3.715612in}{3.446356in}}%
\pgfpathlineto{\pgfqpoint{3.715612in}{3.364628in}}%
\pgfusepath{}%
\end{pgfscope}%
\begin{pgfscope}%
\pgfpathrectangle{\pgfqpoint{0.549740in}{0.463273in}}{\pgfqpoint{9.320225in}{4.495057in}}%
\pgfusepath{clip}%
\pgfsetbuttcap%
\pgfsetroundjoin%
\pgfsetlinewidth{0.000000pt}%
\definecolor{currentstroke}{rgb}{0.000000,0.000000,0.000000}%
\pgfsetstrokecolor{currentstroke}%
\pgfsetdash{}{0pt}%
\pgfpathmoveto{\pgfqpoint{3.901839in}{3.364628in}}%
\pgfpathlineto{\pgfqpoint{4.088065in}{3.364628in}}%
\pgfpathlineto{\pgfqpoint{4.088065in}{3.446356in}}%
\pgfpathlineto{\pgfqpoint{3.901839in}{3.446356in}}%
\pgfpathlineto{\pgfqpoint{3.901839in}{3.364628in}}%
\pgfusepath{}%
\end{pgfscope}%
\begin{pgfscope}%
\pgfpathrectangle{\pgfqpoint{0.549740in}{0.463273in}}{\pgfqpoint{9.320225in}{4.495057in}}%
\pgfusepath{clip}%
\pgfsetbuttcap%
\pgfsetroundjoin%
\pgfsetlinewidth{0.000000pt}%
\definecolor{currentstroke}{rgb}{0.000000,0.000000,0.000000}%
\pgfsetstrokecolor{currentstroke}%
\pgfsetdash{}{0pt}%
\pgfpathmoveto{\pgfqpoint{4.088065in}{3.364628in}}%
\pgfpathlineto{\pgfqpoint{4.274292in}{3.364628in}}%
\pgfpathlineto{\pgfqpoint{4.274292in}{3.446356in}}%
\pgfpathlineto{\pgfqpoint{4.088065in}{3.446356in}}%
\pgfpathlineto{\pgfqpoint{4.088065in}{3.364628in}}%
\pgfusepath{}%
\end{pgfscope}%
\begin{pgfscope}%
\pgfpathrectangle{\pgfqpoint{0.549740in}{0.463273in}}{\pgfqpoint{9.320225in}{4.495057in}}%
\pgfusepath{clip}%
\pgfsetbuttcap%
\pgfsetroundjoin%
\pgfsetlinewidth{0.000000pt}%
\definecolor{currentstroke}{rgb}{0.000000,0.000000,0.000000}%
\pgfsetstrokecolor{currentstroke}%
\pgfsetdash{}{0pt}%
\pgfpathmoveto{\pgfqpoint{4.274292in}{3.364628in}}%
\pgfpathlineto{\pgfqpoint{4.460519in}{3.364628in}}%
\pgfpathlineto{\pgfqpoint{4.460519in}{3.446356in}}%
\pgfpathlineto{\pgfqpoint{4.274292in}{3.446356in}}%
\pgfpathlineto{\pgfqpoint{4.274292in}{3.364628in}}%
\pgfusepath{}%
\end{pgfscope}%
\begin{pgfscope}%
\pgfpathrectangle{\pgfqpoint{0.549740in}{0.463273in}}{\pgfqpoint{9.320225in}{4.495057in}}%
\pgfusepath{clip}%
\pgfsetbuttcap%
\pgfsetroundjoin%
\pgfsetlinewidth{0.000000pt}%
\definecolor{currentstroke}{rgb}{0.000000,0.000000,0.000000}%
\pgfsetstrokecolor{currentstroke}%
\pgfsetdash{}{0pt}%
\pgfpathmoveto{\pgfqpoint{4.460519in}{3.364628in}}%
\pgfpathlineto{\pgfqpoint{4.646745in}{3.364628in}}%
\pgfpathlineto{\pgfqpoint{4.646745in}{3.446356in}}%
\pgfpathlineto{\pgfqpoint{4.460519in}{3.446356in}}%
\pgfpathlineto{\pgfqpoint{4.460519in}{3.364628in}}%
\pgfusepath{}%
\end{pgfscope}%
\begin{pgfscope}%
\pgfpathrectangle{\pgfqpoint{0.549740in}{0.463273in}}{\pgfqpoint{9.320225in}{4.495057in}}%
\pgfusepath{clip}%
\pgfsetbuttcap%
\pgfsetroundjoin%
\pgfsetlinewidth{0.000000pt}%
\definecolor{currentstroke}{rgb}{0.000000,0.000000,0.000000}%
\pgfsetstrokecolor{currentstroke}%
\pgfsetdash{}{0pt}%
\pgfpathmoveto{\pgfqpoint{4.646745in}{3.364628in}}%
\pgfpathlineto{\pgfqpoint{4.832972in}{3.364628in}}%
\pgfpathlineto{\pgfqpoint{4.832972in}{3.446356in}}%
\pgfpathlineto{\pgfqpoint{4.646745in}{3.446356in}}%
\pgfpathlineto{\pgfqpoint{4.646745in}{3.364628in}}%
\pgfusepath{}%
\end{pgfscope}%
\begin{pgfscope}%
\pgfpathrectangle{\pgfqpoint{0.549740in}{0.463273in}}{\pgfqpoint{9.320225in}{4.495057in}}%
\pgfusepath{clip}%
\pgfsetbuttcap%
\pgfsetroundjoin%
\pgfsetlinewidth{0.000000pt}%
\definecolor{currentstroke}{rgb}{0.000000,0.000000,0.000000}%
\pgfsetstrokecolor{currentstroke}%
\pgfsetdash{}{0pt}%
\pgfpathmoveto{\pgfqpoint{4.832972in}{3.364628in}}%
\pgfpathlineto{\pgfqpoint{5.019198in}{3.364628in}}%
\pgfpathlineto{\pgfqpoint{5.019198in}{3.446356in}}%
\pgfpathlineto{\pgfqpoint{4.832972in}{3.446356in}}%
\pgfpathlineto{\pgfqpoint{4.832972in}{3.364628in}}%
\pgfusepath{}%
\end{pgfscope}%
\begin{pgfscope}%
\pgfpathrectangle{\pgfqpoint{0.549740in}{0.463273in}}{\pgfqpoint{9.320225in}{4.495057in}}%
\pgfusepath{clip}%
\pgfsetbuttcap%
\pgfsetroundjoin%
\pgfsetlinewidth{0.000000pt}%
\definecolor{currentstroke}{rgb}{0.000000,0.000000,0.000000}%
\pgfsetstrokecolor{currentstroke}%
\pgfsetdash{}{0pt}%
\pgfpathmoveto{\pgfqpoint{5.019198in}{3.364628in}}%
\pgfpathlineto{\pgfqpoint{5.205425in}{3.364628in}}%
\pgfpathlineto{\pgfqpoint{5.205425in}{3.446356in}}%
\pgfpathlineto{\pgfqpoint{5.019198in}{3.446356in}}%
\pgfpathlineto{\pgfqpoint{5.019198in}{3.364628in}}%
\pgfusepath{}%
\end{pgfscope}%
\begin{pgfscope}%
\pgfpathrectangle{\pgfqpoint{0.549740in}{0.463273in}}{\pgfqpoint{9.320225in}{4.495057in}}%
\pgfusepath{clip}%
\pgfsetbuttcap%
\pgfsetroundjoin%
\pgfsetlinewidth{0.000000pt}%
\definecolor{currentstroke}{rgb}{0.000000,0.000000,0.000000}%
\pgfsetstrokecolor{currentstroke}%
\pgfsetdash{}{0pt}%
\pgfpathmoveto{\pgfqpoint{5.205425in}{3.364628in}}%
\pgfpathlineto{\pgfqpoint{5.391651in}{3.364628in}}%
\pgfpathlineto{\pgfqpoint{5.391651in}{3.446356in}}%
\pgfpathlineto{\pgfqpoint{5.205425in}{3.446356in}}%
\pgfpathlineto{\pgfqpoint{5.205425in}{3.364628in}}%
\pgfusepath{}%
\end{pgfscope}%
\begin{pgfscope}%
\pgfpathrectangle{\pgfqpoint{0.549740in}{0.463273in}}{\pgfqpoint{9.320225in}{4.495057in}}%
\pgfusepath{clip}%
\pgfsetbuttcap%
\pgfsetroundjoin%
\pgfsetlinewidth{0.000000pt}%
\definecolor{currentstroke}{rgb}{0.000000,0.000000,0.000000}%
\pgfsetstrokecolor{currentstroke}%
\pgfsetdash{}{0pt}%
\pgfpathmoveto{\pgfqpoint{5.391651in}{3.364628in}}%
\pgfpathlineto{\pgfqpoint{5.577878in}{3.364628in}}%
\pgfpathlineto{\pgfqpoint{5.577878in}{3.446356in}}%
\pgfpathlineto{\pgfqpoint{5.391651in}{3.446356in}}%
\pgfpathlineto{\pgfqpoint{5.391651in}{3.364628in}}%
\pgfusepath{}%
\end{pgfscope}%
\begin{pgfscope}%
\pgfpathrectangle{\pgfqpoint{0.549740in}{0.463273in}}{\pgfqpoint{9.320225in}{4.495057in}}%
\pgfusepath{clip}%
\pgfsetbuttcap%
\pgfsetroundjoin%
\pgfsetlinewidth{0.000000pt}%
\definecolor{currentstroke}{rgb}{0.000000,0.000000,0.000000}%
\pgfsetstrokecolor{currentstroke}%
\pgfsetdash{}{0pt}%
\pgfpathmoveto{\pgfqpoint{5.577878in}{3.364628in}}%
\pgfpathlineto{\pgfqpoint{5.764104in}{3.364628in}}%
\pgfpathlineto{\pgfqpoint{5.764104in}{3.446356in}}%
\pgfpathlineto{\pgfqpoint{5.577878in}{3.446356in}}%
\pgfpathlineto{\pgfqpoint{5.577878in}{3.364628in}}%
\pgfusepath{}%
\end{pgfscope}%
\begin{pgfscope}%
\pgfpathrectangle{\pgfqpoint{0.549740in}{0.463273in}}{\pgfqpoint{9.320225in}{4.495057in}}%
\pgfusepath{clip}%
\pgfsetbuttcap%
\pgfsetroundjoin%
\pgfsetlinewidth{0.000000pt}%
\definecolor{currentstroke}{rgb}{0.000000,0.000000,0.000000}%
\pgfsetstrokecolor{currentstroke}%
\pgfsetdash{}{0pt}%
\pgfpathmoveto{\pgfqpoint{5.764104in}{3.364628in}}%
\pgfpathlineto{\pgfqpoint{5.950331in}{3.364628in}}%
\pgfpathlineto{\pgfqpoint{5.950331in}{3.446356in}}%
\pgfpathlineto{\pgfqpoint{5.764104in}{3.446356in}}%
\pgfpathlineto{\pgfqpoint{5.764104in}{3.364628in}}%
\pgfusepath{}%
\end{pgfscope}%
\begin{pgfscope}%
\pgfpathrectangle{\pgfqpoint{0.549740in}{0.463273in}}{\pgfqpoint{9.320225in}{4.495057in}}%
\pgfusepath{clip}%
\pgfsetbuttcap%
\pgfsetroundjoin%
\pgfsetlinewidth{0.000000pt}%
\definecolor{currentstroke}{rgb}{0.000000,0.000000,0.000000}%
\pgfsetstrokecolor{currentstroke}%
\pgfsetdash{}{0pt}%
\pgfpathmoveto{\pgfqpoint{5.950331in}{3.364628in}}%
\pgfpathlineto{\pgfqpoint{6.136557in}{3.364628in}}%
\pgfpathlineto{\pgfqpoint{6.136557in}{3.446356in}}%
\pgfpathlineto{\pgfqpoint{5.950331in}{3.446356in}}%
\pgfpathlineto{\pgfqpoint{5.950331in}{3.364628in}}%
\pgfusepath{}%
\end{pgfscope}%
\begin{pgfscope}%
\pgfpathrectangle{\pgfqpoint{0.549740in}{0.463273in}}{\pgfqpoint{9.320225in}{4.495057in}}%
\pgfusepath{clip}%
\pgfsetbuttcap%
\pgfsetroundjoin%
\pgfsetlinewidth{0.000000pt}%
\definecolor{currentstroke}{rgb}{0.000000,0.000000,0.000000}%
\pgfsetstrokecolor{currentstroke}%
\pgfsetdash{}{0pt}%
\pgfpathmoveto{\pgfqpoint{6.136557in}{3.364628in}}%
\pgfpathlineto{\pgfqpoint{6.322784in}{3.364628in}}%
\pgfpathlineto{\pgfqpoint{6.322784in}{3.446356in}}%
\pgfpathlineto{\pgfqpoint{6.136557in}{3.446356in}}%
\pgfpathlineto{\pgfqpoint{6.136557in}{3.364628in}}%
\pgfusepath{}%
\end{pgfscope}%
\begin{pgfscope}%
\pgfpathrectangle{\pgfqpoint{0.549740in}{0.463273in}}{\pgfqpoint{9.320225in}{4.495057in}}%
\pgfusepath{clip}%
\pgfsetbuttcap%
\pgfsetroundjoin%
\pgfsetlinewidth{0.000000pt}%
\definecolor{currentstroke}{rgb}{0.000000,0.000000,0.000000}%
\pgfsetstrokecolor{currentstroke}%
\pgfsetdash{}{0pt}%
\pgfpathmoveto{\pgfqpoint{6.322784in}{3.364628in}}%
\pgfpathlineto{\pgfqpoint{6.509011in}{3.364628in}}%
\pgfpathlineto{\pgfqpoint{6.509011in}{3.446356in}}%
\pgfpathlineto{\pgfqpoint{6.322784in}{3.446356in}}%
\pgfpathlineto{\pgfqpoint{6.322784in}{3.364628in}}%
\pgfusepath{}%
\end{pgfscope}%
\begin{pgfscope}%
\pgfpathrectangle{\pgfqpoint{0.549740in}{0.463273in}}{\pgfqpoint{9.320225in}{4.495057in}}%
\pgfusepath{clip}%
\pgfsetbuttcap%
\pgfsetroundjoin%
\pgfsetlinewidth{0.000000pt}%
\definecolor{currentstroke}{rgb}{0.000000,0.000000,0.000000}%
\pgfsetstrokecolor{currentstroke}%
\pgfsetdash{}{0pt}%
\pgfpathmoveto{\pgfqpoint{6.509011in}{3.364628in}}%
\pgfpathlineto{\pgfqpoint{6.695237in}{3.364628in}}%
\pgfpathlineto{\pgfqpoint{6.695237in}{3.446356in}}%
\pgfpathlineto{\pgfqpoint{6.509011in}{3.446356in}}%
\pgfpathlineto{\pgfqpoint{6.509011in}{3.364628in}}%
\pgfusepath{}%
\end{pgfscope}%
\begin{pgfscope}%
\pgfpathrectangle{\pgfqpoint{0.549740in}{0.463273in}}{\pgfqpoint{9.320225in}{4.495057in}}%
\pgfusepath{clip}%
\pgfsetbuttcap%
\pgfsetroundjoin%
\pgfsetlinewidth{0.000000pt}%
\definecolor{currentstroke}{rgb}{0.000000,0.000000,0.000000}%
\pgfsetstrokecolor{currentstroke}%
\pgfsetdash{}{0pt}%
\pgfpathmoveto{\pgfqpoint{6.695237in}{3.364628in}}%
\pgfpathlineto{\pgfqpoint{6.881464in}{3.364628in}}%
\pgfpathlineto{\pgfqpoint{6.881464in}{3.446356in}}%
\pgfpathlineto{\pgfqpoint{6.695237in}{3.446356in}}%
\pgfpathlineto{\pgfqpoint{6.695237in}{3.364628in}}%
\pgfusepath{}%
\end{pgfscope}%
\begin{pgfscope}%
\pgfpathrectangle{\pgfqpoint{0.549740in}{0.463273in}}{\pgfqpoint{9.320225in}{4.495057in}}%
\pgfusepath{clip}%
\pgfsetbuttcap%
\pgfsetroundjoin%
\definecolor{currentfill}{rgb}{0.547810,0.736432,0.947518}%
\pgfsetfillcolor{currentfill}%
\pgfsetlinewidth{0.000000pt}%
\definecolor{currentstroke}{rgb}{0.000000,0.000000,0.000000}%
\pgfsetstrokecolor{currentstroke}%
\pgfsetdash{}{0pt}%
\pgfpathmoveto{\pgfqpoint{6.881464in}{3.364628in}}%
\pgfpathlineto{\pgfqpoint{7.067690in}{3.364628in}}%
\pgfpathlineto{\pgfqpoint{7.067690in}{3.446356in}}%
\pgfpathlineto{\pgfqpoint{6.881464in}{3.446356in}}%
\pgfpathlineto{\pgfqpoint{6.881464in}{3.364628in}}%
\pgfusepath{fill}%
\end{pgfscope}%
\begin{pgfscope}%
\pgfpathrectangle{\pgfqpoint{0.549740in}{0.463273in}}{\pgfqpoint{9.320225in}{4.495057in}}%
\pgfusepath{clip}%
\pgfsetbuttcap%
\pgfsetroundjoin%
\definecolor{currentfill}{rgb}{0.614330,0.761948,0.940009}%
\pgfsetfillcolor{currentfill}%
\pgfsetlinewidth{0.000000pt}%
\definecolor{currentstroke}{rgb}{0.000000,0.000000,0.000000}%
\pgfsetstrokecolor{currentstroke}%
\pgfsetdash{}{0pt}%
\pgfpathmoveto{\pgfqpoint{7.067690in}{3.364628in}}%
\pgfpathlineto{\pgfqpoint{7.253917in}{3.364628in}}%
\pgfpathlineto{\pgfqpoint{7.253917in}{3.446356in}}%
\pgfpathlineto{\pgfqpoint{7.067690in}{3.446356in}}%
\pgfpathlineto{\pgfqpoint{7.067690in}{3.364628in}}%
\pgfusepath{fill}%
\end{pgfscope}%
\begin{pgfscope}%
\pgfpathrectangle{\pgfqpoint{0.549740in}{0.463273in}}{\pgfqpoint{9.320225in}{4.495057in}}%
\pgfusepath{clip}%
\pgfsetbuttcap%
\pgfsetroundjoin%
\pgfsetlinewidth{0.000000pt}%
\definecolor{currentstroke}{rgb}{0.000000,0.000000,0.000000}%
\pgfsetstrokecolor{currentstroke}%
\pgfsetdash{}{0pt}%
\pgfpathmoveto{\pgfqpoint{7.253917in}{3.364628in}}%
\pgfpathlineto{\pgfqpoint{7.440143in}{3.364628in}}%
\pgfpathlineto{\pgfqpoint{7.440143in}{3.446356in}}%
\pgfpathlineto{\pgfqpoint{7.253917in}{3.446356in}}%
\pgfpathlineto{\pgfqpoint{7.253917in}{3.364628in}}%
\pgfusepath{}%
\end{pgfscope}%
\begin{pgfscope}%
\pgfpathrectangle{\pgfqpoint{0.549740in}{0.463273in}}{\pgfqpoint{9.320225in}{4.495057in}}%
\pgfusepath{clip}%
\pgfsetbuttcap%
\pgfsetroundjoin%
\pgfsetlinewidth{0.000000pt}%
\definecolor{currentstroke}{rgb}{0.000000,0.000000,0.000000}%
\pgfsetstrokecolor{currentstroke}%
\pgfsetdash{}{0pt}%
\pgfpathmoveto{\pgfqpoint{7.440143in}{3.364628in}}%
\pgfpathlineto{\pgfqpoint{7.626370in}{3.364628in}}%
\pgfpathlineto{\pgfqpoint{7.626370in}{3.446356in}}%
\pgfpathlineto{\pgfqpoint{7.440143in}{3.446356in}}%
\pgfpathlineto{\pgfqpoint{7.440143in}{3.364628in}}%
\pgfusepath{}%
\end{pgfscope}%
\begin{pgfscope}%
\pgfpathrectangle{\pgfqpoint{0.549740in}{0.463273in}}{\pgfqpoint{9.320225in}{4.495057in}}%
\pgfusepath{clip}%
\pgfsetbuttcap%
\pgfsetroundjoin%
\pgfsetlinewidth{0.000000pt}%
\definecolor{currentstroke}{rgb}{0.000000,0.000000,0.000000}%
\pgfsetstrokecolor{currentstroke}%
\pgfsetdash{}{0pt}%
\pgfpathmoveto{\pgfqpoint{7.626370in}{3.364628in}}%
\pgfpathlineto{\pgfqpoint{7.812596in}{3.364628in}}%
\pgfpathlineto{\pgfqpoint{7.812596in}{3.446356in}}%
\pgfpathlineto{\pgfqpoint{7.626370in}{3.446356in}}%
\pgfpathlineto{\pgfqpoint{7.626370in}{3.364628in}}%
\pgfusepath{}%
\end{pgfscope}%
\begin{pgfscope}%
\pgfpathrectangle{\pgfqpoint{0.549740in}{0.463273in}}{\pgfqpoint{9.320225in}{4.495057in}}%
\pgfusepath{clip}%
\pgfsetbuttcap%
\pgfsetroundjoin%
\pgfsetlinewidth{0.000000pt}%
\definecolor{currentstroke}{rgb}{0.000000,0.000000,0.000000}%
\pgfsetstrokecolor{currentstroke}%
\pgfsetdash{}{0pt}%
\pgfpathmoveto{\pgfqpoint{7.812596in}{3.364628in}}%
\pgfpathlineto{\pgfqpoint{7.998823in}{3.364628in}}%
\pgfpathlineto{\pgfqpoint{7.998823in}{3.446356in}}%
\pgfpathlineto{\pgfqpoint{7.812596in}{3.446356in}}%
\pgfpathlineto{\pgfqpoint{7.812596in}{3.364628in}}%
\pgfusepath{}%
\end{pgfscope}%
\begin{pgfscope}%
\pgfpathrectangle{\pgfqpoint{0.549740in}{0.463273in}}{\pgfqpoint{9.320225in}{4.495057in}}%
\pgfusepath{clip}%
\pgfsetbuttcap%
\pgfsetroundjoin%
\definecolor{currentfill}{rgb}{0.547810,0.736432,0.947518}%
\pgfsetfillcolor{currentfill}%
\pgfsetlinewidth{0.000000pt}%
\definecolor{currentstroke}{rgb}{0.000000,0.000000,0.000000}%
\pgfsetstrokecolor{currentstroke}%
\pgfsetdash{}{0pt}%
\pgfpathmoveto{\pgfqpoint{7.998823in}{3.364628in}}%
\pgfpathlineto{\pgfqpoint{8.185049in}{3.364628in}}%
\pgfpathlineto{\pgfqpoint{8.185049in}{3.446356in}}%
\pgfpathlineto{\pgfqpoint{7.998823in}{3.446356in}}%
\pgfpathlineto{\pgfqpoint{7.998823in}{3.364628in}}%
\pgfusepath{fill}%
\end{pgfscope}%
\begin{pgfscope}%
\pgfpathrectangle{\pgfqpoint{0.549740in}{0.463273in}}{\pgfqpoint{9.320225in}{4.495057in}}%
\pgfusepath{clip}%
\pgfsetbuttcap%
\pgfsetroundjoin%
\pgfsetlinewidth{0.000000pt}%
\definecolor{currentstroke}{rgb}{0.000000,0.000000,0.000000}%
\pgfsetstrokecolor{currentstroke}%
\pgfsetdash{}{0pt}%
\pgfpathmoveto{\pgfqpoint{8.185049in}{3.364628in}}%
\pgfpathlineto{\pgfqpoint{8.371276in}{3.364628in}}%
\pgfpathlineto{\pgfqpoint{8.371276in}{3.446356in}}%
\pgfpathlineto{\pgfqpoint{8.185049in}{3.446356in}}%
\pgfpathlineto{\pgfqpoint{8.185049in}{3.364628in}}%
\pgfusepath{}%
\end{pgfscope}%
\begin{pgfscope}%
\pgfpathrectangle{\pgfqpoint{0.549740in}{0.463273in}}{\pgfqpoint{9.320225in}{4.495057in}}%
\pgfusepath{clip}%
\pgfsetbuttcap%
\pgfsetroundjoin%
\pgfsetlinewidth{0.000000pt}%
\definecolor{currentstroke}{rgb}{0.000000,0.000000,0.000000}%
\pgfsetstrokecolor{currentstroke}%
\pgfsetdash{}{0pt}%
\pgfpathmoveto{\pgfqpoint{8.371276in}{3.364628in}}%
\pgfpathlineto{\pgfqpoint{8.557503in}{3.364628in}}%
\pgfpathlineto{\pgfqpoint{8.557503in}{3.446356in}}%
\pgfpathlineto{\pgfqpoint{8.371276in}{3.446356in}}%
\pgfpathlineto{\pgfqpoint{8.371276in}{3.364628in}}%
\pgfusepath{}%
\end{pgfscope}%
\begin{pgfscope}%
\pgfpathrectangle{\pgfqpoint{0.549740in}{0.463273in}}{\pgfqpoint{9.320225in}{4.495057in}}%
\pgfusepath{clip}%
\pgfsetbuttcap%
\pgfsetroundjoin%
\pgfsetlinewidth{0.000000pt}%
\definecolor{currentstroke}{rgb}{0.000000,0.000000,0.000000}%
\pgfsetstrokecolor{currentstroke}%
\pgfsetdash{}{0pt}%
\pgfpathmoveto{\pgfqpoint{8.557503in}{3.364628in}}%
\pgfpathlineto{\pgfqpoint{8.743729in}{3.364628in}}%
\pgfpathlineto{\pgfqpoint{8.743729in}{3.446356in}}%
\pgfpathlineto{\pgfqpoint{8.557503in}{3.446356in}}%
\pgfpathlineto{\pgfqpoint{8.557503in}{3.364628in}}%
\pgfusepath{}%
\end{pgfscope}%
\begin{pgfscope}%
\pgfpathrectangle{\pgfqpoint{0.549740in}{0.463273in}}{\pgfqpoint{9.320225in}{4.495057in}}%
\pgfusepath{clip}%
\pgfsetbuttcap%
\pgfsetroundjoin%
\pgfsetlinewidth{0.000000pt}%
\definecolor{currentstroke}{rgb}{0.000000,0.000000,0.000000}%
\pgfsetstrokecolor{currentstroke}%
\pgfsetdash{}{0pt}%
\pgfpathmoveto{\pgfqpoint{8.743729in}{3.364628in}}%
\pgfpathlineto{\pgfqpoint{8.929956in}{3.364628in}}%
\pgfpathlineto{\pgfqpoint{8.929956in}{3.446356in}}%
\pgfpathlineto{\pgfqpoint{8.743729in}{3.446356in}}%
\pgfpathlineto{\pgfqpoint{8.743729in}{3.364628in}}%
\pgfusepath{}%
\end{pgfscope}%
\begin{pgfscope}%
\pgfpathrectangle{\pgfqpoint{0.549740in}{0.463273in}}{\pgfqpoint{9.320225in}{4.495057in}}%
\pgfusepath{clip}%
\pgfsetbuttcap%
\pgfsetroundjoin%
\pgfsetlinewidth{0.000000pt}%
\definecolor{currentstroke}{rgb}{0.000000,0.000000,0.000000}%
\pgfsetstrokecolor{currentstroke}%
\pgfsetdash{}{0pt}%
\pgfpathmoveto{\pgfqpoint{8.929956in}{3.364628in}}%
\pgfpathlineto{\pgfqpoint{9.116182in}{3.364628in}}%
\pgfpathlineto{\pgfqpoint{9.116182in}{3.446356in}}%
\pgfpathlineto{\pgfqpoint{8.929956in}{3.446356in}}%
\pgfpathlineto{\pgfqpoint{8.929956in}{3.364628in}}%
\pgfusepath{}%
\end{pgfscope}%
\begin{pgfscope}%
\pgfpathrectangle{\pgfqpoint{0.549740in}{0.463273in}}{\pgfqpoint{9.320225in}{4.495057in}}%
\pgfusepath{clip}%
\pgfsetbuttcap%
\pgfsetroundjoin%
\pgfsetlinewidth{0.000000pt}%
\definecolor{currentstroke}{rgb}{0.000000,0.000000,0.000000}%
\pgfsetstrokecolor{currentstroke}%
\pgfsetdash{}{0pt}%
\pgfpathmoveto{\pgfqpoint{9.116182in}{3.364628in}}%
\pgfpathlineto{\pgfqpoint{9.302409in}{3.364628in}}%
\pgfpathlineto{\pgfqpoint{9.302409in}{3.446356in}}%
\pgfpathlineto{\pgfqpoint{9.116182in}{3.446356in}}%
\pgfpathlineto{\pgfqpoint{9.116182in}{3.364628in}}%
\pgfusepath{}%
\end{pgfscope}%
\begin{pgfscope}%
\pgfpathrectangle{\pgfqpoint{0.549740in}{0.463273in}}{\pgfqpoint{9.320225in}{4.495057in}}%
\pgfusepath{clip}%
\pgfsetbuttcap%
\pgfsetroundjoin%
\definecolor{currentfill}{rgb}{0.472869,0.711325,0.955316}%
\pgfsetfillcolor{currentfill}%
\pgfsetlinewidth{0.000000pt}%
\definecolor{currentstroke}{rgb}{0.000000,0.000000,0.000000}%
\pgfsetstrokecolor{currentstroke}%
\pgfsetdash{}{0pt}%
\pgfpathmoveto{\pgfqpoint{9.302409in}{3.364628in}}%
\pgfpathlineto{\pgfqpoint{9.488635in}{3.364628in}}%
\pgfpathlineto{\pgfqpoint{9.488635in}{3.446356in}}%
\pgfpathlineto{\pgfqpoint{9.302409in}{3.446356in}}%
\pgfpathlineto{\pgfqpoint{9.302409in}{3.364628in}}%
\pgfusepath{fill}%
\end{pgfscope}%
\begin{pgfscope}%
\pgfpathrectangle{\pgfqpoint{0.549740in}{0.463273in}}{\pgfqpoint{9.320225in}{4.495057in}}%
\pgfusepath{clip}%
\pgfsetbuttcap%
\pgfsetroundjoin%
\pgfsetlinewidth{0.000000pt}%
\definecolor{currentstroke}{rgb}{0.000000,0.000000,0.000000}%
\pgfsetstrokecolor{currentstroke}%
\pgfsetdash{}{0pt}%
\pgfpathmoveto{\pgfqpoint{9.488635in}{3.364628in}}%
\pgfpathlineto{\pgfqpoint{9.674862in}{3.364628in}}%
\pgfpathlineto{\pgfqpoint{9.674862in}{3.446356in}}%
\pgfpathlineto{\pgfqpoint{9.488635in}{3.446356in}}%
\pgfpathlineto{\pgfqpoint{9.488635in}{3.364628in}}%
\pgfusepath{}%
\end{pgfscope}%
\begin{pgfscope}%
\pgfpathrectangle{\pgfqpoint{0.549740in}{0.463273in}}{\pgfqpoint{9.320225in}{4.495057in}}%
\pgfusepath{clip}%
\pgfsetbuttcap%
\pgfsetroundjoin%
\pgfsetlinewidth{0.000000pt}%
\definecolor{currentstroke}{rgb}{0.000000,0.000000,0.000000}%
\pgfsetstrokecolor{currentstroke}%
\pgfsetdash{}{0pt}%
\pgfpathmoveto{\pgfqpoint{9.674862in}{3.364628in}}%
\pgfpathlineto{\pgfqpoint{9.861088in}{3.364628in}}%
\pgfpathlineto{\pgfqpoint{9.861088in}{3.446356in}}%
\pgfpathlineto{\pgfqpoint{9.674862in}{3.446356in}}%
\pgfpathlineto{\pgfqpoint{9.674862in}{3.364628in}}%
\pgfusepath{}%
\end{pgfscope}%
\begin{pgfscope}%
\pgfpathrectangle{\pgfqpoint{0.549740in}{0.463273in}}{\pgfqpoint{9.320225in}{4.495057in}}%
\pgfusepath{clip}%
\pgfsetbuttcap%
\pgfsetroundjoin%
\pgfsetlinewidth{0.000000pt}%
\definecolor{currentstroke}{rgb}{0.000000,0.000000,0.000000}%
\pgfsetstrokecolor{currentstroke}%
\pgfsetdash{}{0pt}%
\pgfpathmoveto{\pgfqpoint{0.549761in}{3.446356in}}%
\pgfpathlineto{\pgfqpoint{0.735988in}{3.446356in}}%
\pgfpathlineto{\pgfqpoint{0.735988in}{3.528085in}}%
\pgfpathlineto{\pgfqpoint{0.549761in}{3.528085in}}%
\pgfpathlineto{\pgfqpoint{0.549761in}{3.446356in}}%
\pgfusepath{}%
\end{pgfscope}%
\begin{pgfscope}%
\pgfpathrectangle{\pgfqpoint{0.549740in}{0.463273in}}{\pgfqpoint{9.320225in}{4.495057in}}%
\pgfusepath{clip}%
\pgfsetbuttcap%
\pgfsetroundjoin%
\pgfsetlinewidth{0.000000pt}%
\definecolor{currentstroke}{rgb}{0.000000,0.000000,0.000000}%
\pgfsetstrokecolor{currentstroke}%
\pgfsetdash{}{0pt}%
\pgfpathmoveto{\pgfqpoint{0.735988in}{3.446356in}}%
\pgfpathlineto{\pgfqpoint{0.922214in}{3.446356in}}%
\pgfpathlineto{\pgfqpoint{0.922214in}{3.528085in}}%
\pgfpathlineto{\pgfqpoint{0.735988in}{3.528085in}}%
\pgfpathlineto{\pgfqpoint{0.735988in}{3.446356in}}%
\pgfusepath{}%
\end{pgfscope}%
\begin{pgfscope}%
\pgfpathrectangle{\pgfqpoint{0.549740in}{0.463273in}}{\pgfqpoint{9.320225in}{4.495057in}}%
\pgfusepath{clip}%
\pgfsetbuttcap%
\pgfsetroundjoin%
\pgfsetlinewidth{0.000000pt}%
\definecolor{currentstroke}{rgb}{0.000000,0.000000,0.000000}%
\pgfsetstrokecolor{currentstroke}%
\pgfsetdash{}{0pt}%
\pgfpathmoveto{\pgfqpoint{0.922214in}{3.446356in}}%
\pgfpathlineto{\pgfqpoint{1.108441in}{3.446356in}}%
\pgfpathlineto{\pgfqpoint{1.108441in}{3.528085in}}%
\pgfpathlineto{\pgfqpoint{0.922214in}{3.528085in}}%
\pgfpathlineto{\pgfqpoint{0.922214in}{3.446356in}}%
\pgfusepath{}%
\end{pgfscope}%
\begin{pgfscope}%
\pgfpathrectangle{\pgfqpoint{0.549740in}{0.463273in}}{\pgfqpoint{9.320225in}{4.495057in}}%
\pgfusepath{clip}%
\pgfsetbuttcap%
\pgfsetroundjoin%
\pgfsetlinewidth{0.000000pt}%
\definecolor{currentstroke}{rgb}{0.000000,0.000000,0.000000}%
\pgfsetstrokecolor{currentstroke}%
\pgfsetdash{}{0pt}%
\pgfpathmoveto{\pgfqpoint{1.108441in}{3.446356in}}%
\pgfpathlineto{\pgfqpoint{1.294667in}{3.446356in}}%
\pgfpathlineto{\pgfqpoint{1.294667in}{3.528085in}}%
\pgfpathlineto{\pgfqpoint{1.108441in}{3.528085in}}%
\pgfpathlineto{\pgfqpoint{1.108441in}{3.446356in}}%
\pgfusepath{}%
\end{pgfscope}%
\begin{pgfscope}%
\pgfpathrectangle{\pgfqpoint{0.549740in}{0.463273in}}{\pgfqpoint{9.320225in}{4.495057in}}%
\pgfusepath{clip}%
\pgfsetbuttcap%
\pgfsetroundjoin%
\pgfsetlinewidth{0.000000pt}%
\definecolor{currentstroke}{rgb}{0.000000,0.000000,0.000000}%
\pgfsetstrokecolor{currentstroke}%
\pgfsetdash{}{0pt}%
\pgfpathmoveto{\pgfqpoint{1.294667in}{3.446356in}}%
\pgfpathlineto{\pgfqpoint{1.480894in}{3.446356in}}%
\pgfpathlineto{\pgfqpoint{1.480894in}{3.528085in}}%
\pgfpathlineto{\pgfqpoint{1.294667in}{3.528085in}}%
\pgfpathlineto{\pgfqpoint{1.294667in}{3.446356in}}%
\pgfusepath{}%
\end{pgfscope}%
\begin{pgfscope}%
\pgfpathrectangle{\pgfqpoint{0.549740in}{0.463273in}}{\pgfqpoint{9.320225in}{4.495057in}}%
\pgfusepath{clip}%
\pgfsetbuttcap%
\pgfsetroundjoin%
\pgfsetlinewidth{0.000000pt}%
\definecolor{currentstroke}{rgb}{0.000000,0.000000,0.000000}%
\pgfsetstrokecolor{currentstroke}%
\pgfsetdash{}{0pt}%
\pgfpathmoveto{\pgfqpoint{1.480894in}{3.446356in}}%
\pgfpathlineto{\pgfqpoint{1.667120in}{3.446356in}}%
\pgfpathlineto{\pgfqpoint{1.667120in}{3.528085in}}%
\pgfpathlineto{\pgfqpoint{1.480894in}{3.528085in}}%
\pgfpathlineto{\pgfqpoint{1.480894in}{3.446356in}}%
\pgfusepath{}%
\end{pgfscope}%
\begin{pgfscope}%
\pgfpathrectangle{\pgfqpoint{0.549740in}{0.463273in}}{\pgfqpoint{9.320225in}{4.495057in}}%
\pgfusepath{clip}%
\pgfsetbuttcap%
\pgfsetroundjoin%
\pgfsetlinewidth{0.000000pt}%
\definecolor{currentstroke}{rgb}{0.000000,0.000000,0.000000}%
\pgfsetstrokecolor{currentstroke}%
\pgfsetdash{}{0pt}%
\pgfpathmoveto{\pgfqpoint{1.667120in}{3.446356in}}%
\pgfpathlineto{\pgfqpoint{1.853347in}{3.446356in}}%
\pgfpathlineto{\pgfqpoint{1.853347in}{3.528085in}}%
\pgfpathlineto{\pgfqpoint{1.667120in}{3.528085in}}%
\pgfpathlineto{\pgfqpoint{1.667120in}{3.446356in}}%
\pgfusepath{}%
\end{pgfscope}%
\begin{pgfscope}%
\pgfpathrectangle{\pgfqpoint{0.549740in}{0.463273in}}{\pgfqpoint{9.320225in}{4.495057in}}%
\pgfusepath{clip}%
\pgfsetbuttcap%
\pgfsetroundjoin%
\pgfsetlinewidth{0.000000pt}%
\definecolor{currentstroke}{rgb}{0.000000,0.000000,0.000000}%
\pgfsetstrokecolor{currentstroke}%
\pgfsetdash{}{0pt}%
\pgfpathmoveto{\pgfqpoint{1.853347in}{3.446356in}}%
\pgfpathlineto{\pgfqpoint{2.039573in}{3.446356in}}%
\pgfpathlineto{\pgfqpoint{2.039573in}{3.528085in}}%
\pgfpathlineto{\pgfqpoint{1.853347in}{3.528085in}}%
\pgfpathlineto{\pgfqpoint{1.853347in}{3.446356in}}%
\pgfusepath{}%
\end{pgfscope}%
\begin{pgfscope}%
\pgfpathrectangle{\pgfqpoint{0.549740in}{0.463273in}}{\pgfqpoint{9.320225in}{4.495057in}}%
\pgfusepath{clip}%
\pgfsetbuttcap%
\pgfsetroundjoin%
\pgfsetlinewidth{0.000000pt}%
\definecolor{currentstroke}{rgb}{0.000000,0.000000,0.000000}%
\pgfsetstrokecolor{currentstroke}%
\pgfsetdash{}{0pt}%
\pgfpathmoveto{\pgfqpoint{2.039573in}{3.446356in}}%
\pgfpathlineto{\pgfqpoint{2.225800in}{3.446356in}}%
\pgfpathlineto{\pgfqpoint{2.225800in}{3.528085in}}%
\pgfpathlineto{\pgfqpoint{2.039573in}{3.528085in}}%
\pgfpathlineto{\pgfqpoint{2.039573in}{3.446356in}}%
\pgfusepath{}%
\end{pgfscope}%
\begin{pgfscope}%
\pgfpathrectangle{\pgfqpoint{0.549740in}{0.463273in}}{\pgfqpoint{9.320225in}{4.495057in}}%
\pgfusepath{clip}%
\pgfsetbuttcap%
\pgfsetroundjoin%
\pgfsetlinewidth{0.000000pt}%
\definecolor{currentstroke}{rgb}{0.000000,0.000000,0.000000}%
\pgfsetstrokecolor{currentstroke}%
\pgfsetdash{}{0pt}%
\pgfpathmoveto{\pgfqpoint{2.225800in}{3.446356in}}%
\pgfpathlineto{\pgfqpoint{2.412027in}{3.446356in}}%
\pgfpathlineto{\pgfqpoint{2.412027in}{3.528085in}}%
\pgfpathlineto{\pgfqpoint{2.225800in}{3.528085in}}%
\pgfpathlineto{\pgfqpoint{2.225800in}{3.446356in}}%
\pgfusepath{}%
\end{pgfscope}%
\begin{pgfscope}%
\pgfpathrectangle{\pgfqpoint{0.549740in}{0.463273in}}{\pgfqpoint{9.320225in}{4.495057in}}%
\pgfusepath{clip}%
\pgfsetbuttcap%
\pgfsetroundjoin%
\pgfsetlinewidth{0.000000pt}%
\definecolor{currentstroke}{rgb}{0.000000,0.000000,0.000000}%
\pgfsetstrokecolor{currentstroke}%
\pgfsetdash{}{0pt}%
\pgfpathmoveto{\pgfqpoint{2.412027in}{3.446356in}}%
\pgfpathlineto{\pgfqpoint{2.598253in}{3.446356in}}%
\pgfpathlineto{\pgfqpoint{2.598253in}{3.528085in}}%
\pgfpathlineto{\pgfqpoint{2.412027in}{3.528085in}}%
\pgfpathlineto{\pgfqpoint{2.412027in}{3.446356in}}%
\pgfusepath{}%
\end{pgfscope}%
\begin{pgfscope}%
\pgfpathrectangle{\pgfqpoint{0.549740in}{0.463273in}}{\pgfqpoint{9.320225in}{4.495057in}}%
\pgfusepath{clip}%
\pgfsetbuttcap%
\pgfsetroundjoin%
\pgfsetlinewidth{0.000000pt}%
\definecolor{currentstroke}{rgb}{0.000000,0.000000,0.000000}%
\pgfsetstrokecolor{currentstroke}%
\pgfsetdash{}{0pt}%
\pgfpathmoveto{\pgfqpoint{2.598253in}{3.446356in}}%
\pgfpathlineto{\pgfqpoint{2.784480in}{3.446356in}}%
\pgfpathlineto{\pgfqpoint{2.784480in}{3.528085in}}%
\pgfpathlineto{\pgfqpoint{2.598253in}{3.528085in}}%
\pgfpathlineto{\pgfqpoint{2.598253in}{3.446356in}}%
\pgfusepath{}%
\end{pgfscope}%
\begin{pgfscope}%
\pgfpathrectangle{\pgfqpoint{0.549740in}{0.463273in}}{\pgfqpoint{9.320225in}{4.495057in}}%
\pgfusepath{clip}%
\pgfsetbuttcap%
\pgfsetroundjoin%
\pgfsetlinewidth{0.000000pt}%
\definecolor{currentstroke}{rgb}{0.000000,0.000000,0.000000}%
\pgfsetstrokecolor{currentstroke}%
\pgfsetdash{}{0pt}%
\pgfpathmoveto{\pgfqpoint{2.784480in}{3.446356in}}%
\pgfpathlineto{\pgfqpoint{2.970706in}{3.446356in}}%
\pgfpathlineto{\pgfqpoint{2.970706in}{3.528085in}}%
\pgfpathlineto{\pgfqpoint{2.784480in}{3.528085in}}%
\pgfpathlineto{\pgfqpoint{2.784480in}{3.446356in}}%
\pgfusepath{}%
\end{pgfscope}%
\begin{pgfscope}%
\pgfpathrectangle{\pgfqpoint{0.549740in}{0.463273in}}{\pgfqpoint{9.320225in}{4.495057in}}%
\pgfusepath{clip}%
\pgfsetbuttcap%
\pgfsetroundjoin%
\pgfsetlinewidth{0.000000pt}%
\definecolor{currentstroke}{rgb}{0.000000,0.000000,0.000000}%
\pgfsetstrokecolor{currentstroke}%
\pgfsetdash{}{0pt}%
\pgfpathmoveto{\pgfqpoint{2.970706in}{3.446356in}}%
\pgfpathlineto{\pgfqpoint{3.156933in}{3.446356in}}%
\pgfpathlineto{\pgfqpoint{3.156933in}{3.528085in}}%
\pgfpathlineto{\pgfqpoint{2.970706in}{3.528085in}}%
\pgfpathlineto{\pgfqpoint{2.970706in}{3.446356in}}%
\pgfusepath{}%
\end{pgfscope}%
\begin{pgfscope}%
\pgfpathrectangle{\pgfqpoint{0.549740in}{0.463273in}}{\pgfqpoint{9.320225in}{4.495057in}}%
\pgfusepath{clip}%
\pgfsetbuttcap%
\pgfsetroundjoin%
\pgfsetlinewidth{0.000000pt}%
\definecolor{currentstroke}{rgb}{0.000000,0.000000,0.000000}%
\pgfsetstrokecolor{currentstroke}%
\pgfsetdash{}{0pt}%
\pgfpathmoveto{\pgfqpoint{3.156933in}{3.446356in}}%
\pgfpathlineto{\pgfqpoint{3.343159in}{3.446356in}}%
\pgfpathlineto{\pgfqpoint{3.343159in}{3.528085in}}%
\pgfpathlineto{\pgfqpoint{3.156933in}{3.528085in}}%
\pgfpathlineto{\pgfqpoint{3.156933in}{3.446356in}}%
\pgfusepath{}%
\end{pgfscope}%
\begin{pgfscope}%
\pgfpathrectangle{\pgfqpoint{0.549740in}{0.463273in}}{\pgfqpoint{9.320225in}{4.495057in}}%
\pgfusepath{clip}%
\pgfsetbuttcap%
\pgfsetroundjoin%
\pgfsetlinewidth{0.000000pt}%
\definecolor{currentstroke}{rgb}{0.000000,0.000000,0.000000}%
\pgfsetstrokecolor{currentstroke}%
\pgfsetdash{}{0pt}%
\pgfpathmoveto{\pgfqpoint{3.343159in}{3.446356in}}%
\pgfpathlineto{\pgfqpoint{3.529386in}{3.446356in}}%
\pgfpathlineto{\pgfqpoint{3.529386in}{3.528085in}}%
\pgfpathlineto{\pgfqpoint{3.343159in}{3.528085in}}%
\pgfpathlineto{\pgfqpoint{3.343159in}{3.446356in}}%
\pgfusepath{}%
\end{pgfscope}%
\begin{pgfscope}%
\pgfpathrectangle{\pgfqpoint{0.549740in}{0.463273in}}{\pgfqpoint{9.320225in}{4.495057in}}%
\pgfusepath{clip}%
\pgfsetbuttcap%
\pgfsetroundjoin%
\pgfsetlinewidth{0.000000pt}%
\definecolor{currentstroke}{rgb}{0.000000,0.000000,0.000000}%
\pgfsetstrokecolor{currentstroke}%
\pgfsetdash{}{0pt}%
\pgfpathmoveto{\pgfqpoint{3.529386in}{3.446356in}}%
\pgfpathlineto{\pgfqpoint{3.715612in}{3.446356in}}%
\pgfpathlineto{\pgfqpoint{3.715612in}{3.528085in}}%
\pgfpathlineto{\pgfqpoint{3.529386in}{3.528085in}}%
\pgfpathlineto{\pgfqpoint{3.529386in}{3.446356in}}%
\pgfusepath{}%
\end{pgfscope}%
\begin{pgfscope}%
\pgfpathrectangle{\pgfqpoint{0.549740in}{0.463273in}}{\pgfqpoint{9.320225in}{4.495057in}}%
\pgfusepath{clip}%
\pgfsetbuttcap%
\pgfsetroundjoin%
\pgfsetlinewidth{0.000000pt}%
\definecolor{currentstroke}{rgb}{0.000000,0.000000,0.000000}%
\pgfsetstrokecolor{currentstroke}%
\pgfsetdash{}{0pt}%
\pgfpathmoveto{\pgfqpoint{3.715612in}{3.446356in}}%
\pgfpathlineto{\pgfqpoint{3.901839in}{3.446356in}}%
\pgfpathlineto{\pgfqpoint{3.901839in}{3.528085in}}%
\pgfpathlineto{\pgfqpoint{3.715612in}{3.528085in}}%
\pgfpathlineto{\pgfqpoint{3.715612in}{3.446356in}}%
\pgfusepath{}%
\end{pgfscope}%
\begin{pgfscope}%
\pgfpathrectangle{\pgfqpoint{0.549740in}{0.463273in}}{\pgfqpoint{9.320225in}{4.495057in}}%
\pgfusepath{clip}%
\pgfsetbuttcap%
\pgfsetroundjoin%
\pgfsetlinewidth{0.000000pt}%
\definecolor{currentstroke}{rgb}{0.000000,0.000000,0.000000}%
\pgfsetstrokecolor{currentstroke}%
\pgfsetdash{}{0pt}%
\pgfpathmoveto{\pgfqpoint{3.901839in}{3.446356in}}%
\pgfpathlineto{\pgfqpoint{4.088065in}{3.446356in}}%
\pgfpathlineto{\pgfqpoint{4.088065in}{3.528085in}}%
\pgfpathlineto{\pgfqpoint{3.901839in}{3.528085in}}%
\pgfpathlineto{\pgfqpoint{3.901839in}{3.446356in}}%
\pgfusepath{}%
\end{pgfscope}%
\begin{pgfscope}%
\pgfpathrectangle{\pgfqpoint{0.549740in}{0.463273in}}{\pgfqpoint{9.320225in}{4.495057in}}%
\pgfusepath{clip}%
\pgfsetbuttcap%
\pgfsetroundjoin%
\pgfsetlinewidth{0.000000pt}%
\definecolor{currentstroke}{rgb}{0.000000,0.000000,0.000000}%
\pgfsetstrokecolor{currentstroke}%
\pgfsetdash{}{0pt}%
\pgfpathmoveto{\pgfqpoint{4.088065in}{3.446356in}}%
\pgfpathlineto{\pgfqpoint{4.274292in}{3.446356in}}%
\pgfpathlineto{\pgfqpoint{4.274292in}{3.528085in}}%
\pgfpathlineto{\pgfqpoint{4.088065in}{3.528085in}}%
\pgfpathlineto{\pgfqpoint{4.088065in}{3.446356in}}%
\pgfusepath{}%
\end{pgfscope}%
\begin{pgfscope}%
\pgfpathrectangle{\pgfqpoint{0.549740in}{0.463273in}}{\pgfqpoint{9.320225in}{4.495057in}}%
\pgfusepath{clip}%
\pgfsetbuttcap%
\pgfsetroundjoin%
\pgfsetlinewidth{0.000000pt}%
\definecolor{currentstroke}{rgb}{0.000000,0.000000,0.000000}%
\pgfsetstrokecolor{currentstroke}%
\pgfsetdash{}{0pt}%
\pgfpathmoveto{\pgfqpoint{4.274292in}{3.446356in}}%
\pgfpathlineto{\pgfqpoint{4.460519in}{3.446356in}}%
\pgfpathlineto{\pgfqpoint{4.460519in}{3.528085in}}%
\pgfpathlineto{\pgfqpoint{4.274292in}{3.528085in}}%
\pgfpathlineto{\pgfqpoint{4.274292in}{3.446356in}}%
\pgfusepath{}%
\end{pgfscope}%
\begin{pgfscope}%
\pgfpathrectangle{\pgfqpoint{0.549740in}{0.463273in}}{\pgfqpoint{9.320225in}{4.495057in}}%
\pgfusepath{clip}%
\pgfsetbuttcap%
\pgfsetroundjoin%
\pgfsetlinewidth{0.000000pt}%
\definecolor{currentstroke}{rgb}{0.000000,0.000000,0.000000}%
\pgfsetstrokecolor{currentstroke}%
\pgfsetdash{}{0pt}%
\pgfpathmoveto{\pgfqpoint{4.460519in}{3.446356in}}%
\pgfpathlineto{\pgfqpoint{4.646745in}{3.446356in}}%
\pgfpathlineto{\pgfqpoint{4.646745in}{3.528085in}}%
\pgfpathlineto{\pgfqpoint{4.460519in}{3.528085in}}%
\pgfpathlineto{\pgfqpoint{4.460519in}{3.446356in}}%
\pgfusepath{}%
\end{pgfscope}%
\begin{pgfscope}%
\pgfpathrectangle{\pgfqpoint{0.549740in}{0.463273in}}{\pgfqpoint{9.320225in}{4.495057in}}%
\pgfusepath{clip}%
\pgfsetbuttcap%
\pgfsetroundjoin%
\pgfsetlinewidth{0.000000pt}%
\definecolor{currentstroke}{rgb}{0.000000,0.000000,0.000000}%
\pgfsetstrokecolor{currentstroke}%
\pgfsetdash{}{0pt}%
\pgfpathmoveto{\pgfqpoint{4.646745in}{3.446356in}}%
\pgfpathlineto{\pgfqpoint{4.832972in}{3.446356in}}%
\pgfpathlineto{\pgfqpoint{4.832972in}{3.528085in}}%
\pgfpathlineto{\pgfqpoint{4.646745in}{3.528085in}}%
\pgfpathlineto{\pgfqpoint{4.646745in}{3.446356in}}%
\pgfusepath{}%
\end{pgfscope}%
\begin{pgfscope}%
\pgfpathrectangle{\pgfqpoint{0.549740in}{0.463273in}}{\pgfqpoint{9.320225in}{4.495057in}}%
\pgfusepath{clip}%
\pgfsetbuttcap%
\pgfsetroundjoin%
\pgfsetlinewidth{0.000000pt}%
\definecolor{currentstroke}{rgb}{0.000000,0.000000,0.000000}%
\pgfsetstrokecolor{currentstroke}%
\pgfsetdash{}{0pt}%
\pgfpathmoveto{\pgfqpoint{4.832972in}{3.446356in}}%
\pgfpathlineto{\pgfqpoint{5.019198in}{3.446356in}}%
\pgfpathlineto{\pgfqpoint{5.019198in}{3.528085in}}%
\pgfpathlineto{\pgfqpoint{4.832972in}{3.528085in}}%
\pgfpathlineto{\pgfqpoint{4.832972in}{3.446356in}}%
\pgfusepath{}%
\end{pgfscope}%
\begin{pgfscope}%
\pgfpathrectangle{\pgfqpoint{0.549740in}{0.463273in}}{\pgfqpoint{9.320225in}{4.495057in}}%
\pgfusepath{clip}%
\pgfsetbuttcap%
\pgfsetroundjoin%
\pgfsetlinewidth{0.000000pt}%
\definecolor{currentstroke}{rgb}{0.000000,0.000000,0.000000}%
\pgfsetstrokecolor{currentstroke}%
\pgfsetdash{}{0pt}%
\pgfpathmoveto{\pgfqpoint{5.019198in}{3.446356in}}%
\pgfpathlineto{\pgfqpoint{5.205425in}{3.446356in}}%
\pgfpathlineto{\pgfqpoint{5.205425in}{3.528085in}}%
\pgfpathlineto{\pgfqpoint{5.019198in}{3.528085in}}%
\pgfpathlineto{\pgfqpoint{5.019198in}{3.446356in}}%
\pgfusepath{}%
\end{pgfscope}%
\begin{pgfscope}%
\pgfpathrectangle{\pgfqpoint{0.549740in}{0.463273in}}{\pgfqpoint{9.320225in}{4.495057in}}%
\pgfusepath{clip}%
\pgfsetbuttcap%
\pgfsetroundjoin%
\pgfsetlinewidth{0.000000pt}%
\definecolor{currentstroke}{rgb}{0.000000,0.000000,0.000000}%
\pgfsetstrokecolor{currentstroke}%
\pgfsetdash{}{0pt}%
\pgfpathmoveto{\pgfqpoint{5.205425in}{3.446356in}}%
\pgfpathlineto{\pgfqpoint{5.391651in}{3.446356in}}%
\pgfpathlineto{\pgfqpoint{5.391651in}{3.528085in}}%
\pgfpathlineto{\pgfqpoint{5.205425in}{3.528085in}}%
\pgfpathlineto{\pgfqpoint{5.205425in}{3.446356in}}%
\pgfusepath{}%
\end{pgfscope}%
\begin{pgfscope}%
\pgfpathrectangle{\pgfqpoint{0.549740in}{0.463273in}}{\pgfqpoint{9.320225in}{4.495057in}}%
\pgfusepath{clip}%
\pgfsetbuttcap%
\pgfsetroundjoin%
\pgfsetlinewidth{0.000000pt}%
\definecolor{currentstroke}{rgb}{0.000000,0.000000,0.000000}%
\pgfsetstrokecolor{currentstroke}%
\pgfsetdash{}{0pt}%
\pgfpathmoveto{\pgfqpoint{5.391651in}{3.446356in}}%
\pgfpathlineto{\pgfqpoint{5.577878in}{3.446356in}}%
\pgfpathlineto{\pgfqpoint{5.577878in}{3.528085in}}%
\pgfpathlineto{\pgfqpoint{5.391651in}{3.528085in}}%
\pgfpathlineto{\pgfqpoint{5.391651in}{3.446356in}}%
\pgfusepath{}%
\end{pgfscope}%
\begin{pgfscope}%
\pgfpathrectangle{\pgfqpoint{0.549740in}{0.463273in}}{\pgfqpoint{9.320225in}{4.495057in}}%
\pgfusepath{clip}%
\pgfsetbuttcap%
\pgfsetroundjoin%
\pgfsetlinewidth{0.000000pt}%
\definecolor{currentstroke}{rgb}{0.000000,0.000000,0.000000}%
\pgfsetstrokecolor{currentstroke}%
\pgfsetdash{}{0pt}%
\pgfpathmoveto{\pgfqpoint{5.577878in}{3.446356in}}%
\pgfpathlineto{\pgfqpoint{5.764104in}{3.446356in}}%
\pgfpathlineto{\pgfqpoint{5.764104in}{3.528085in}}%
\pgfpathlineto{\pgfqpoint{5.577878in}{3.528085in}}%
\pgfpathlineto{\pgfqpoint{5.577878in}{3.446356in}}%
\pgfusepath{}%
\end{pgfscope}%
\begin{pgfscope}%
\pgfpathrectangle{\pgfqpoint{0.549740in}{0.463273in}}{\pgfqpoint{9.320225in}{4.495057in}}%
\pgfusepath{clip}%
\pgfsetbuttcap%
\pgfsetroundjoin%
\pgfsetlinewidth{0.000000pt}%
\definecolor{currentstroke}{rgb}{0.000000,0.000000,0.000000}%
\pgfsetstrokecolor{currentstroke}%
\pgfsetdash{}{0pt}%
\pgfpathmoveto{\pgfqpoint{5.764104in}{3.446356in}}%
\pgfpathlineto{\pgfqpoint{5.950331in}{3.446356in}}%
\pgfpathlineto{\pgfqpoint{5.950331in}{3.528085in}}%
\pgfpathlineto{\pgfqpoint{5.764104in}{3.528085in}}%
\pgfpathlineto{\pgfqpoint{5.764104in}{3.446356in}}%
\pgfusepath{}%
\end{pgfscope}%
\begin{pgfscope}%
\pgfpathrectangle{\pgfqpoint{0.549740in}{0.463273in}}{\pgfqpoint{9.320225in}{4.495057in}}%
\pgfusepath{clip}%
\pgfsetbuttcap%
\pgfsetroundjoin%
\pgfsetlinewidth{0.000000pt}%
\definecolor{currentstroke}{rgb}{0.000000,0.000000,0.000000}%
\pgfsetstrokecolor{currentstroke}%
\pgfsetdash{}{0pt}%
\pgfpathmoveto{\pgfqpoint{5.950331in}{3.446356in}}%
\pgfpathlineto{\pgfqpoint{6.136557in}{3.446356in}}%
\pgfpathlineto{\pgfqpoint{6.136557in}{3.528085in}}%
\pgfpathlineto{\pgfqpoint{5.950331in}{3.528085in}}%
\pgfpathlineto{\pgfqpoint{5.950331in}{3.446356in}}%
\pgfusepath{}%
\end{pgfscope}%
\begin{pgfscope}%
\pgfpathrectangle{\pgfqpoint{0.549740in}{0.463273in}}{\pgfqpoint{9.320225in}{4.495057in}}%
\pgfusepath{clip}%
\pgfsetbuttcap%
\pgfsetroundjoin%
\pgfsetlinewidth{0.000000pt}%
\definecolor{currentstroke}{rgb}{0.000000,0.000000,0.000000}%
\pgfsetstrokecolor{currentstroke}%
\pgfsetdash{}{0pt}%
\pgfpathmoveto{\pgfqpoint{6.136557in}{3.446356in}}%
\pgfpathlineto{\pgfqpoint{6.322784in}{3.446356in}}%
\pgfpathlineto{\pgfqpoint{6.322784in}{3.528085in}}%
\pgfpathlineto{\pgfqpoint{6.136557in}{3.528085in}}%
\pgfpathlineto{\pgfqpoint{6.136557in}{3.446356in}}%
\pgfusepath{}%
\end{pgfscope}%
\begin{pgfscope}%
\pgfpathrectangle{\pgfqpoint{0.549740in}{0.463273in}}{\pgfqpoint{9.320225in}{4.495057in}}%
\pgfusepath{clip}%
\pgfsetbuttcap%
\pgfsetroundjoin%
\pgfsetlinewidth{0.000000pt}%
\definecolor{currentstroke}{rgb}{0.000000,0.000000,0.000000}%
\pgfsetstrokecolor{currentstroke}%
\pgfsetdash{}{0pt}%
\pgfpathmoveto{\pgfqpoint{6.322784in}{3.446356in}}%
\pgfpathlineto{\pgfqpoint{6.509011in}{3.446356in}}%
\pgfpathlineto{\pgfqpoint{6.509011in}{3.528085in}}%
\pgfpathlineto{\pgfqpoint{6.322784in}{3.528085in}}%
\pgfpathlineto{\pgfqpoint{6.322784in}{3.446356in}}%
\pgfusepath{}%
\end{pgfscope}%
\begin{pgfscope}%
\pgfpathrectangle{\pgfqpoint{0.549740in}{0.463273in}}{\pgfqpoint{9.320225in}{4.495057in}}%
\pgfusepath{clip}%
\pgfsetbuttcap%
\pgfsetroundjoin%
\pgfsetlinewidth{0.000000pt}%
\definecolor{currentstroke}{rgb}{0.000000,0.000000,0.000000}%
\pgfsetstrokecolor{currentstroke}%
\pgfsetdash{}{0pt}%
\pgfpathmoveto{\pgfqpoint{6.509011in}{3.446356in}}%
\pgfpathlineto{\pgfqpoint{6.695237in}{3.446356in}}%
\pgfpathlineto{\pgfqpoint{6.695237in}{3.528085in}}%
\pgfpathlineto{\pgfqpoint{6.509011in}{3.528085in}}%
\pgfpathlineto{\pgfqpoint{6.509011in}{3.446356in}}%
\pgfusepath{}%
\end{pgfscope}%
\begin{pgfscope}%
\pgfpathrectangle{\pgfqpoint{0.549740in}{0.463273in}}{\pgfqpoint{9.320225in}{4.495057in}}%
\pgfusepath{clip}%
\pgfsetbuttcap%
\pgfsetroundjoin%
\pgfsetlinewidth{0.000000pt}%
\definecolor{currentstroke}{rgb}{0.000000,0.000000,0.000000}%
\pgfsetstrokecolor{currentstroke}%
\pgfsetdash{}{0pt}%
\pgfpathmoveto{\pgfqpoint{6.695237in}{3.446356in}}%
\pgfpathlineto{\pgfqpoint{6.881464in}{3.446356in}}%
\pgfpathlineto{\pgfqpoint{6.881464in}{3.528085in}}%
\pgfpathlineto{\pgfqpoint{6.695237in}{3.528085in}}%
\pgfpathlineto{\pgfqpoint{6.695237in}{3.446356in}}%
\pgfusepath{}%
\end{pgfscope}%
\begin{pgfscope}%
\pgfpathrectangle{\pgfqpoint{0.549740in}{0.463273in}}{\pgfqpoint{9.320225in}{4.495057in}}%
\pgfusepath{clip}%
\pgfsetbuttcap%
\pgfsetroundjoin%
\definecolor{currentfill}{rgb}{0.472869,0.711325,0.955316}%
\pgfsetfillcolor{currentfill}%
\pgfsetlinewidth{0.000000pt}%
\definecolor{currentstroke}{rgb}{0.000000,0.000000,0.000000}%
\pgfsetstrokecolor{currentstroke}%
\pgfsetdash{}{0pt}%
\pgfpathmoveto{\pgfqpoint{6.881464in}{3.446356in}}%
\pgfpathlineto{\pgfqpoint{7.067690in}{3.446356in}}%
\pgfpathlineto{\pgfqpoint{7.067690in}{3.528085in}}%
\pgfpathlineto{\pgfqpoint{6.881464in}{3.528085in}}%
\pgfpathlineto{\pgfqpoint{6.881464in}{3.446356in}}%
\pgfusepath{fill}%
\end{pgfscope}%
\begin{pgfscope}%
\pgfpathrectangle{\pgfqpoint{0.549740in}{0.463273in}}{\pgfqpoint{9.320225in}{4.495057in}}%
\pgfusepath{clip}%
\pgfsetbuttcap%
\pgfsetroundjoin%
\pgfsetlinewidth{0.000000pt}%
\definecolor{currentstroke}{rgb}{0.000000,0.000000,0.000000}%
\pgfsetstrokecolor{currentstroke}%
\pgfsetdash{}{0pt}%
\pgfpathmoveto{\pgfqpoint{7.067690in}{3.446356in}}%
\pgfpathlineto{\pgfqpoint{7.253917in}{3.446356in}}%
\pgfpathlineto{\pgfqpoint{7.253917in}{3.528085in}}%
\pgfpathlineto{\pgfqpoint{7.067690in}{3.528085in}}%
\pgfpathlineto{\pgfqpoint{7.067690in}{3.446356in}}%
\pgfusepath{}%
\end{pgfscope}%
\begin{pgfscope}%
\pgfpathrectangle{\pgfqpoint{0.549740in}{0.463273in}}{\pgfqpoint{9.320225in}{4.495057in}}%
\pgfusepath{clip}%
\pgfsetbuttcap%
\pgfsetroundjoin%
\pgfsetlinewidth{0.000000pt}%
\definecolor{currentstroke}{rgb}{0.000000,0.000000,0.000000}%
\pgfsetstrokecolor{currentstroke}%
\pgfsetdash{}{0pt}%
\pgfpathmoveto{\pgfqpoint{7.253917in}{3.446356in}}%
\pgfpathlineto{\pgfqpoint{7.440143in}{3.446356in}}%
\pgfpathlineto{\pgfqpoint{7.440143in}{3.528085in}}%
\pgfpathlineto{\pgfqpoint{7.253917in}{3.528085in}}%
\pgfpathlineto{\pgfqpoint{7.253917in}{3.446356in}}%
\pgfusepath{}%
\end{pgfscope}%
\begin{pgfscope}%
\pgfpathrectangle{\pgfqpoint{0.549740in}{0.463273in}}{\pgfqpoint{9.320225in}{4.495057in}}%
\pgfusepath{clip}%
\pgfsetbuttcap%
\pgfsetroundjoin%
\pgfsetlinewidth{0.000000pt}%
\definecolor{currentstroke}{rgb}{0.000000,0.000000,0.000000}%
\pgfsetstrokecolor{currentstroke}%
\pgfsetdash{}{0pt}%
\pgfpathmoveto{\pgfqpoint{7.440143in}{3.446356in}}%
\pgfpathlineto{\pgfqpoint{7.626370in}{3.446356in}}%
\pgfpathlineto{\pgfqpoint{7.626370in}{3.528085in}}%
\pgfpathlineto{\pgfqpoint{7.440143in}{3.528085in}}%
\pgfpathlineto{\pgfqpoint{7.440143in}{3.446356in}}%
\pgfusepath{}%
\end{pgfscope}%
\begin{pgfscope}%
\pgfpathrectangle{\pgfqpoint{0.549740in}{0.463273in}}{\pgfqpoint{9.320225in}{4.495057in}}%
\pgfusepath{clip}%
\pgfsetbuttcap%
\pgfsetroundjoin%
\pgfsetlinewidth{0.000000pt}%
\definecolor{currentstroke}{rgb}{0.000000,0.000000,0.000000}%
\pgfsetstrokecolor{currentstroke}%
\pgfsetdash{}{0pt}%
\pgfpathmoveto{\pgfqpoint{7.626370in}{3.446356in}}%
\pgfpathlineto{\pgfqpoint{7.812596in}{3.446356in}}%
\pgfpathlineto{\pgfqpoint{7.812596in}{3.528085in}}%
\pgfpathlineto{\pgfqpoint{7.626370in}{3.528085in}}%
\pgfpathlineto{\pgfqpoint{7.626370in}{3.446356in}}%
\pgfusepath{}%
\end{pgfscope}%
\begin{pgfscope}%
\pgfpathrectangle{\pgfqpoint{0.549740in}{0.463273in}}{\pgfqpoint{9.320225in}{4.495057in}}%
\pgfusepath{clip}%
\pgfsetbuttcap%
\pgfsetroundjoin%
\pgfsetlinewidth{0.000000pt}%
\definecolor{currentstroke}{rgb}{0.000000,0.000000,0.000000}%
\pgfsetstrokecolor{currentstroke}%
\pgfsetdash{}{0pt}%
\pgfpathmoveto{\pgfqpoint{7.812596in}{3.446356in}}%
\pgfpathlineto{\pgfqpoint{7.998823in}{3.446356in}}%
\pgfpathlineto{\pgfqpoint{7.998823in}{3.528085in}}%
\pgfpathlineto{\pgfqpoint{7.812596in}{3.528085in}}%
\pgfpathlineto{\pgfqpoint{7.812596in}{3.446356in}}%
\pgfusepath{}%
\end{pgfscope}%
\begin{pgfscope}%
\pgfpathrectangle{\pgfqpoint{0.549740in}{0.463273in}}{\pgfqpoint{9.320225in}{4.495057in}}%
\pgfusepath{clip}%
\pgfsetbuttcap%
\pgfsetroundjoin%
\definecolor{currentfill}{rgb}{0.472869,0.711325,0.955316}%
\pgfsetfillcolor{currentfill}%
\pgfsetlinewidth{0.000000pt}%
\definecolor{currentstroke}{rgb}{0.000000,0.000000,0.000000}%
\pgfsetstrokecolor{currentstroke}%
\pgfsetdash{}{0pt}%
\pgfpathmoveto{\pgfqpoint{7.998823in}{3.446356in}}%
\pgfpathlineto{\pgfqpoint{8.185049in}{3.446356in}}%
\pgfpathlineto{\pgfqpoint{8.185049in}{3.528085in}}%
\pgfpathlineto{\pgfqpoint{7.998823in}{3.528085in}}%
\pgfpathlineto{\pgfqpoint{7.998823in}{3.446356in}}%
\pgfusepath{fill}%
\end{pgfscope}%
\begin{pgfscope}%
\pgfpathrectangle{\pgfqpoint{0.549740in}{0.463273in}}{\pgfqpoint{9.320225in}{4.495057in}}%
\pgfusepath{clip}%
\pgfsetbuttcap%
\pgfsetroundjoin%
\pgfsetlinewidth{0.000000pt}%
\definecolor{currentstroke}{rgb}{0.000000,0.000000,0.000000}%
\pgfsetstrokecolor{currentstroke}%
\pgfsetdash{}{0pt}%
\pgfpathmoveto{\pgfqpoint{8.185049in}{3.446356in}}%
\pgfpathlineto{\pgfqpoint{8.371276in}{3.446356in}}%
\pgfpathlineto{\pgfqpoint{8.371276in}{3.528085in}}%
\pgfpathlineto{\pgfqpoint{8.185049in}{3.528085in}}%
\pgfpathlineto{\pgfqpoint{8.185049in}{3.446356in}}%
\pgfusepath{}%
\end{pgfscope}%
\begin{pgfscope}%
\pgfpathrectangle{\pgfqpoint{0.549740in}{0.463273in}}{\pgfqpoint{9.320225in}{4.495057in}}%
\pgfusepath{clip}%
\pgfsetbuttcap%
\pgfsetroundjoin%
\pgfsetlinewidth{0.000000pt}%
\definecolor{currentstroke}{rgb}{0.000000,0.000000,0.000000}%
\pgfsetstrokecolor{currentstroke}%
\pgfsetdash{}{0pt}%
\pgfpathmoveto{\pgfqpoint{8.371276in}{3.446356in}}%
\pgfpathlineto{\pgfqpoint{8.557503in}{3.446356in}}%
\pgfpathlineto{\pgfqpoint{8.557503in}{3.528085in}}%
\pgfpathlineto{\pgfqpoint{8.371276in}{3.528085in}}%
\pgfpathlineto{\pgfqpoint{8.371276in}{3.446356in}}%
\pgfusepath{}%
\end{pgfscope}%
\begin{pgfscope}%
\pgfpathrectangle{\pgfqpoint{0.549740in}{0.463273in}}{\pgfqpoint{9.320225in}{4.495057in}}%
\pgfusepath{clip}%
\pgfsetbuttcap%
\pgfsetroundjoin%
\pgfsetlinewidth{0.000000pt}%
\definecolor{currentstroke}{rgb}{0.000000,0.000000,0.000000}%
\pgfsetstrokecolor{currentstroke}%
\pgfsetdash{}{0pt}%
\pgfpathmoveto{\pgfqpoint{8.557503in}{3.446356in}}%
\pgfpathlineto{\pgfqpoint{8.743729in}{3.446356in}}%
\pgfpathlineto{\pgfqpoint{8.743729in}{3.528085in}}%
\pgfpathlineto{\pgfqpoint{8.557503in}{3.528085in}}%
\pgfpathlineto{\pgfqpoint{8.557503in}{3.446356in}}%
\pgfusepath{}%
\end{pgfscope}%
\begin{pgfscope}%
\pgfpathrectangle{\pgfqpoint{0.549740in}{0.463273in}}{\pgfqpoint{9.320225in}{4.495057in}}%
\pgfusepath{clip}%
\pgfsetbuttcap%
\pgfsetroundjoin%
\pgfsetlinewidth{0.000000pt}%
\definecolor{currentstroke}{rgb}{0.000000,0.000000,0.000000}%
\pgfsetstrokecolor{currentstroke}%
\pgfsetdash{}{0pt}%
\pgfpathmoveto{\pgfqpoint{8.743729in}{3.446356in}}%
\pgfpathlineto{\pgfqpoint{8.929956in}{3.446356in}}%
\pgfpathlineto{\pgfqpoint{8.929956in}{3.528085in}}%
\pgfpathlineto{\pgfqpoint{8.743729in}{3.528085in}}%
\pgfpathlineto{\pgfqpoint{8.743729in}{3.446356in}}%
\pgfusepath{}%
\end{pgfscope}%
\begin{pgfscope}%
\pgfpathrectangle{\pgfqpoint{0.549740in}{0.463273in}}{\pgfqpoint{9.320225in}{4.495057in}}%
\pgfusepath{clip}%
\pgfsetbuttcap%
\pgfsetroundjoin%
\pgfsetlinewidth{0.000000pt}%
\definecolor{currentstroke}{rgb}{0.000000,0.000000,0.000000}%
\pgfsetstrokecolor{currentstroke}%
\pgfsetdash{}{0pt}%
\pgfpathmoveto{\pgfqpoint{8.929956in}{3.446356in}}%
\pgfpathlineto{\pgfqpoint{9.116182in}{3.446356in}}%
\pgfpathlineto{\pgfqpoint{9.116182in}{3.528085in}}%
\pgfpathlineto{\pgfqpoint{8.929956in}{3.528085in}}%
\pgfpathlineto{\pgfqpoint{8.929956in}{3.446356in}}%
\pgfusepath{}%
\end{pgfscope}%
\begin{pgfscope}%
\pgfpathrectangle{\pgfqpoint{0.549740in}{0.463273in}}{\pgfqpoint{9.320225in}{4.495057in}}%
\pgfusepath{clip}%
\pgfsetbuttcap%
\pgfsetroundjoin%
\pgfsetlinewidth{0.000000pt}%
\definecolor{currentstroke}{rgb}{0.000000,0.000000,0.000000}%
\pgfsetstrokecolor{currentstroke}%
\pgfsetdash{}{0pt}%
\pgfpathmoveto{\pgfqpoint{9.116182in}{3.446356in}}%
\pgfpathlineto{\pgfqpoint{9.302409in}{3.446356in}}%
\pgfpathlineto{\pgfqpoint{9.302409in}{3.528085in}}%
\pgfpathlineto{\pgfqpoint{9.116182in}{3.528085in}}%
\pgfpathlineto{\pgfqpoint{9.116182in}{3.446356in}}%
\pgfusepath{}%
\end{pgfscope}%
\begin{pgfscope}%
\pgfpathrectangle{\pgfqpoint{0.549740in}{0.463273in}}{\pgfqpoint{9.320225in}{4.495057in}}%
\pgfusepath{clip}%
\pgfsetbuttcap%
\pgfsetroundjoin%
\definecolor{currentfill}{rgb}{0.472869,0.711325,0.955316}%
\pgfsetfillcolor{currentfill}%
\pgfsetlinewidth{0.000000pt}%
\definecolor{currentstroke}{rgb}{0.000000,0.000000,0.000000}%
\pgfsetstrokecolor{currentstroke}%
\pgfsetdash{}{0pt}%
\pgfpathmoveto{\pgfqpoint{9.302409in}{3.446356in}}%
\pgfpathlineto{\pgfqpoint{9.488635in}{3.446356in}}%
\pgfpathlineto{\pgfqpoint{9.488635in}{3.528085in}}%
\pgfpathlineto{\pgfqpoint{9.302409in}{3.528085in}}%
\pgfpathlineto{\pgfqpoint{9.302409in}{3.446356in}}%
\pgfusepath{fill}%
\end{pgfscope}%
\begin{pgfscope}%
\pgfpathrectangle{\pgfqpoint{0.549740in}{0.463273in}}{\pgfqpoint{9.320225in}{4.495057in}}%
\pgfusepath{clip}%
\pgfsetbuttcap%
\pgfsetroundjoin%
\pgfsetlinewidth{0.000000pt}%
\definecolor{currentstroke}{rgb}{0.000000,0.000000,0.000000}%
\pgfsetstrokecolor{currentstroke}%
\pgfsetdash{}{0pt}%
\pgfpathmoveto{\pgfqpoint{9.488635in}{3.446356in}}%
\pgfpathlineto{\pgfqpoint{9.674862in}{3.446356in}}%
\pgfpathlineto{\pgfqpoint{9.674862in}{3.528085in}}%
\pgfpathlineto{\pgfqpoint{9.488635in}{3.528085in}}%
\pgfpathlineto{\pgfqpoint{9.488635in}{3.446356in}}%
\pgfusepath{}%
\end{pgfscope}%
\begin{pgfscope}%
\pgfpathrectangle{\pgfqpoint{0.549740in}{0.463273in}}{\pgfqpoint{9.320225in}{4.495057in}}%
\pgfusepath{clip}%
\pgfsetbuttcap%
\pgfsetroundjoin%
\pgfsetlinewidth{0.000000pt}%
\definecolor{currentstroke}{rgb}{0.000000,0.000000,0.000000}%
\pgfsetstrokecolor{currentstroke}%
\pgfsetdash{}{0pt}%
\pgfpathmoveto{\pgfqpoint{9.674862in}{3.446356in}}%
\pgfpathlineto{\pgfqpoint{9.861088in}{3.446356in}}%
\pgfpathlineto{\pgfqpoint{9.861088in}{3.528085in}}%
\pgfpathlineto{\pgfqpoint{9.674862in}{3.528085in}}%
\pgfpathlineto{\pgfqpoint{9.674862in}{3.446356in}}%
\pgfusepath{}%
\end{pgfscope}%
\begin{pgfscope}%
\pgfpathrectangle{\pgfqpoint{0.549740in}{0.463273in}}{\pgfqpoint{9.320225in}{4.495057in}}%
\pgfusepath{clip}%
\pgfsetbuttcap%
\pgfsetroundjoin%
\pgfsetlinewidth{0.000000pt}%
\definecolor{currentstroke}{rgb}{0.000000,0.000000,0.000000}%
\pgfsetstrokecolor{currentstroke}%
\pgfsetdash{}{0pt}%
\pgfpathmoveto{\pgfqpoint{0.549761in}{3.528085in}}%
\pgfpathlineto{\pgfqpoint{0.735988in}{3.528085in}}%
\pgfpathlineto{\pgfqpoint{0.735988in}{3.609813in}}%
\pgfpathlineto{\pgfqpoint{0.549761in}{3.609813in}}%
\pgfpathlineto{\pgfqpoint{0.549761in}{3.528085in}}%
\pgfusepath{}%
\end{pgfscope}%
\begin{pgfscope}%
\pgfpathrectangle{\pgfqpoint{0.549740in}{0.463273in}}{\pgfqpoint{9.320225in}{4.495057in}}%
\pgfusepath{clip}%
\pgfsetbuttcap%
\pgfsetroundjoin%
\pgfsetlinewidth{0.000000pt}%
\definecolor{currentstroke}{rgb}{0.000000,0.000000,0.000000}%
\pgfsetstrokecolor{currentstroke}%
\pgfsetdash{}{0pt}%
\pgfpathmoveto{\pgfqpoint{0.735988in}{3.528085in}}%
\pgfpathlineto{\pgfqpoint{0.922214in}{3.528085in}}%
\pgfpathlineto{\pgfqpoint{0.922214in}{3.609813in}}%
\pgfpathlineto{\pgfqpoint{0.735988in}{3.609813in}}%
\pgfpathlineto{\pgfqpoint{0.735988in}{3.528085in}}%
\pgfusepath{}%
\end{pgfscope}%
\begin{pgfscope}%
\pgfpathrectangle{\pgfqpoint{0.549740in}{0.463273in}}{\pgfqpoint{9.320225in}{4.495057in}}%
\pgfusepath{clip}%
\pgfsetbuttcap%
\pgfsetroundjoin%
\pgfsetlinewidth{0.000000pt}%
\definecolor{currentstroke}{rgb}{0.000000,0.000000,0.000000}%
\pgfsetstrokecolor{currentstroke}%
\pgfsetdash{}{0pt}%
\pgfpathmoveto{\pgfqpoint{0.922214in}{3.528085in}}%
\pgfpathlineto{\pgfqpoint{1.108441in}{3.528085in}}%
\pgfpathlineto{\pgfqpoint{1.108441in}{3.609813in}}%
\pgfpathlineto{\pgfqpoint{0.922214in}{3.609813in}}%
\pgfpathlineto{\pgfqpoint{0.922214in}{3.528085in}}%
\pgfusepath{}%
\end{pgfscope}%
\begin{pgfscope}%
\pgfpathrectangle{\pgfqpoint{0.549740in}{0.463273in}}{\pgfqpoint{9.320225in}{4.495057in}}%
\pgfusepath{clip}%
\pgfsetbuttcap%
\pgfsetroundjoin%
\pgfsetlinewidth{0.000000pt}%
\definecolor{currentstroke}{rgb}{0.000000,0.000000,0.000000}%
\pgfsetstrokecolor{currentstroke}%
\pgfsetdash{}{0pt}%
\pgfpathmoveto{\pgfqpoint{1.108441in}{3.528085in}}%
\pgfpathlineto{\pgfqpoint{1.294667in}{3.528085in}}%
\pgfpathlineto{\pgfqpoint{1.294667in}{3.609813in}}%
\pgfpathlineto{\pgfqpoint{1.108441in}{3.609813in}}%
\pgfpathlineto{\pgfqpoint{1.108441in}{3.528085in}}%
\pgfusepath{}%
\end{pgfscope}%
\begin{pgfscope}%
\pgfpathrectangle{\pgfqpoint{0.549740in}{0.463273in}}{\pgfqpoint{9.320225in}{4.495057in}}%
\pgfusepath{clip}%
\pgfsetbuttcap%
\pgfsetroundjoin%
\pgfsetlinewidth{0.000000pt}%
\definecolor{currentstroke}{rgb}{0.000000,0.000000,0.000000}%
\pgfsetstrokecolor{currentstroke}%
\pgfsetdash{}{0pt}%
\pgfpathmoveto{\pgfqpoint{1.294667in}{3.528085in}}%
\pgfpathlineto{\pgfqpoint{1.480894in}{3.528085in}}%
\pgfpathlineto{\pgfqpoint{1.480894in}{3.609813in}}%
\pgfpathlineto{\pgfqpoint{1.294667in}{3.609813in}}%
\pgfpathlineto{\pgfqpoint{1.294667in}{3.528085in}}%
\pgfusepath{}%
\end{pgfscope}%
\begin{pgfscope}%
\pgfpathrectangle{\pgfqpoint{0.549740in}{0.463273in}}{\pgfqpoint{9.320225in}{4.495057in}}%
\pgfusepath{clip}%
\pgfsetbuttcap%
\pgfsetroundjoin%
\pgfsetlinewidth{0.000000pt}%
\definecolor{currentstroke}{rgb}{0.000000,0.000000,0.000000}%
\pgfsetstrokecolor{currentstroke}%
\pgfsetdash{}{0pt}%
\pgfpathmoveto{\pgfqpoint{1.480894in}{3.528085in}}%
\pgfpathlineto{\pgfqpoint{1.667120in}{3.528085in}}%
\pgfpathlineto{\pgfqpoint{1.667120in}{3.609813in}}%
\pgfpathlineto{\pgfqpoint{1.480894in}{3.609813in}}%
\pgfpathlineto{\pgfqpoint{1.480894in}{3.528085in}}%
\pgfusepath{}%
\end{pgfscope}%
\begin{pgfscope}%
\pgfpathrectangle{\pgfqpoint{0.549740in}{0.463273in}}{\pgfqpoint{9.320225in}{4.495057in}}%
\pgfusepath{clip}%
\pgfsetbuttcap%
\pgfsetroundjoin%
\pgfsetlinewidth{0.000000pt}%
\definecolor{currentstroke}{rgb}{0.000000,0.000000,0.000000}%
\pgfsetstrokecolor{currentstroke}%
\pgfsetdash{}{0pt}%
\pgfpathmoveto{\pgfqpoint{1.667120in}{3.528085in}}%
\pgfpathlineto{\pgfqpoint{1.853347in}{3.528085in}}%
\pgfpathlineto{\pgfqpoint{1.853347in}{3.609813in}}%
\pgfpathlineto{\pgfqpoint{1.667120in}{3.609813in}}%
\pgfpathlineto{\pgfqpoint{1.667120in}{3.528085in}}%
\pgfusepath{}%
\end{pgfscope}%
\begin{pgfscope}%
\pgfpathrectangle{\pgfqpoint{0.549740in}{0.463273in}}{\pgfqpoint{9.320225in}{4.495057in}}%
\pgfusepath{clip}%
\pgfsetbuttcap%
\pgfsetroundjoin%
\pgfsetlinewidth{0.000000pt}%
\definecolor{currentstroke}{rgb}{0.000000,0.000000,0.000000}%
\pgfsetstrokecolor{currentstroke}%
\pgfsetdash{}{0pt}%
\pgfpathmoveto{\pgfqpoint{1.853347in}{3.528085in}}%
\pgfpathlineto{\pgfqpoint{2.039573in}{3.528085in}}%
\pgfpathlineto{\pgfqpoint{2.039573in}{3.609813in}}%
\pgfpathlineto{\pgfqpoint{1.853347in}{3.609813in}}%
\pgfpathlineto{\pgfqpoint{1.853347in}{3.528085in}}%
\pgfusepath{}%
\end{pgfscope}%
\begin{pgfscope}%
\pgfpathrectangle{\pgfqpoint{0.549740in}{0.463273in}}{\pgfqpoint{9.320225in}{4.495057in}}%
\pgfusepath{clip}%
\pgfsetbuttcap%
\pgfsetroundjoin%
\pgfsetlinewidth{0.000000pt}%
\definecolor{currentstroke}{rgb}{0.000000,0.000000,0.000000}%
\pgfsetstrokecolor{currentstroke}%
\pgfsetdash{}{0pt}%
\pgfpathmoveto{\pgfqpoint{2.039573in}{3.528085in}}%
\pgfpathlineto{\pgfqpoint{2.225800in}{3.528085in}}%
\pgfpathlineto{\pgfqpoint{2.225800in}{3.609813in}}%
\pgfpathlineto{\pgfqpoint{2.039573in}{3.609813in}}%
\pgfpathlineto{\pgfqpoint{2.039573in}{3.528085in}}%
\pgfusepath{}%
\end{pgfscope}%
\begin{pgfscope}%
\pgfpathrectangle{\pgfqpoint{0.549740in}{0.463273in}}{\pgfqpoint{9.320225in}{4.495057in}}%
\pgfusepath{clip}%
\pgfsetbuttcap%
\pgfsetroundjoin%
\pgfsetlinewidth{0.000000pt}%
\definecolor{currentstroke}{rgb}{0.000000,0.000000,0.000000}%
\pgfsetstrokecolor{currentstroke}%
\pgfsetdash{}{0pt}%
\pgfpathmoveto{\pgfqpoint{2.225800in}{3.528085in}}%
\pgfpathlineto{\pgfqpoint{2.412027in}{3.528085in}}%
\pgfpathlineto{\pgfqpoint{2.412027in}{3.609813in}}%
\pgfpathlineto{\pgfqpoint{2.225800in}{3.609813in}}%
\pgfpathlineto{\pgfqpoint{2.225800in}{3.528085in}}%
\pgfusepath{}%
\end{pgfscope}%
\begin{pgfscope}%
\pgfpathrectangle{\pgfqpoint{0.549740in}{0.463273in}}{\pgfqpoint{9.320225in}{4.495057in}}%
\pgfusepath{clip}%
\pgfsetbuttcap%
\pgfsetroundjoin%
\pgfsetlinewidth{0.000000pt}%
\definecolor{currentstroke}{rgb}{0.000000,0.000000,0.000000}%
\pgfsetstrokecolor{currentstroke}%
\pgfsetdash{}{0pt}%
\pgfpathmoveto{\pgfqpoint{2.412027in}{3.528085in}}%
\pgfpathlineto{\pgfqpoint{2.598253in}{3.528085in}}%
\pgfpathlineto{\pgfqpoint{2.598253in}{3.609813in}}%
\pgfpathlineto{\pgfqpoint{2.412027in}{3.609813in}}%
\pgfpathlineto{\pgfqpoint{2.412027in}{3.528085in}}%
\pgfusepath{}%
\end{pgfscope}%
\begin{pgfscope}%
\pgfpathrectangle{\pgfqpoint{0.549740in}{0.463273in}}{\pgfqpoint{9.320225in}{4.495057in}}%
\pgfusepath{clip}%
\pgfsetbuttcap%
\pgfsetroundjoin%
\pgfsetlinewidth{0.000000pt}%
\definecolor{currentstroke}{rgb}{0.000000,0.000000,0.000000}%
\pgfsetstrokecolor{currentstroke}%
\pgfsetdash{}{0pt}%
\pgfpathmoveto{\pgfqpoint{2.598253in}{3.528085in}}%
\pgfpathlineto{\pgfqpoint{2.784480in}{3.528085in}}%
\pgfpathlineto{\pgfqpoint{2.784480in}{3.609813in}}%
\pgfpathlineto{\pgfqpoint{2.598253in}{3.609813in}}%
\pgfpathlineto{\pgfqpoint{2.598253in}{3.528085in}}%
\pgfusepath{}%
\end{pgfscope}%
\begin{pgfscope}%
\pgfpathrectangle{\pgfqpoint{0.549740in}{0.463273in}}{\pgfqpoint{9.320225in}{4.495057in}}%
\pgfusepath{clip}%
\pgfsetbuttcap%
\pgfsetroundjoin%
\pgfsetlinewidth{0.000000pt}%
\definecolor{currentstroke}{rgb}{0.000000,0.000000,0.000000}%
\pgfsetstrokecolor{currentstroke}%
\pgfsetdash{}{0pt}%
\pgfpathmoveto{\pgfqpoint{2.784480in}{3.528085in}}%
\pgfpathlineto{\pgfqpoint{2.970706in}{3.528085in}}%
\pgfpathlineto{\pgfqpoint{2.970706in}{3.609813in}}%
\pgfpathlineto{\pgfqpoint{2.784480in}{3.609813in}}%
\pgfpathlineto{\pgfqpoint{2.784480in}{3.528085in}}%
\pgfusepath{}%
\end{pgfscope}%
\begin{pgfscope}%
\pgfpathrectangle{\pgfqpoint{0.549740in}{0.463273in}}{\pgfqpoint{9.320225in}{4.495057in}}%
\pgfusepath{clip}%
\pgfsetbuttcap%
\pgfsetroundjoin%
\pgfsetlinewidth{0.000000pt}%
\definecolor{currentstroke}{rgb}{0.000000,0.000000,0.000000}%
\pgfsetstrokecolor{currentstroke}%
\pgfsetdash{}{0pt}%
\pgfpathmoveto{\pgfqpoint{2.970706in}{3.528085in}}%
\pgfpathlineto{\pgfqpoint{3.156933in}{3.528085in}}%
\pgfpathlineto{\pgfqpoint{3.156933in}{3.609813in}}%
\pgfpathlineto{\pgfqpoint{2.970706in}{3.609813in}}%
\pgfpathlineto{\pgfqpoint{2.970706in}{3.528085in}}%
\pgfusepath{}%
\end{pgfscope}%
\begin{pgfscope}%
\pgfpathrectangle{\pgfqpoint{0.549740in}{0.463273in}}{\pgfqpoint{9.320225in}{4.495057in}}%
\pgfusepath{clip}%
\pgfsetbuttcap%
\pgfsetroundjoin%
\pgfsetlinewidth{0.000000pt}%
\definecolor{currentstroke}{rgb}{0.000000,0.000000,0.000000}%
\pgfsetstrokecolor{currentstroke}%
\pgfsetdash{}{0pt}%
\pgfpathmoveto{\pgfqpoint{3.156933in}{3.528085in}}%
\pgfpathlineto{\pgfqpoint{3.343159in}{3.528085in}}%
\pgfpathlineto{\pgfqpoint{3.343159in}{3.609813in}}%
\pgfpathlineto{\pgfqpoint{3.156933in}{3.609813in}}%
\pgfpathlineto{\pgfqpoint{3.156933in}{3.528085in}}%
\pgfusepath{}%
\end{pgfscope}%
\begin{pgfscope}%
\pgfpathrectangle{\pgfqpoint{0.549740in}{0.463273in}}{\pgfqpoint{9.320225in}{4.495057in}}%
\pgfusepath{clip}%
\pgfsetbuttcap%
\pgfsetroundjoin%
\pgfsetlinewidth{0.000000pt}%
\definecolor{currentstroke}{rgb}{0.000000,0.000000,0.000000}%
\pgfsetstrokecolor{currentstroke}%
\pgfsetdash{}{0pt}%
\pgfpathmoveto{\pgfqpoint{3.343159in}{3.528085in}}%
\pgfpathlineto{\pgfqpoint{3.529386in}{3.528085in}}%
\pgfpathlineto{\pgfqpoint{3.529386in}{3.609813in}}%
\pgfpathlineto{\pgfqpoint{3.343159in}{3.609813in}}%
\pgfpathlineto{\pgfqpoint{3.343159in}{3.528085in}}%
\pgfusepath{}%
\end{pgfscope}%
\begin{pgfscope}%
\pgfpathrectangle{\pgfqpoint{0.549740in}{0.463273in}}{\pgfqpoint{9.320225in}{4.495057in}}%
\pgfusepath{clip}%
\pgfsetbuttcap%
\pgfsetroundjoin%
\pgfsetlinewidth{0.000000pt}%
\definecolor{currentstroke}{rgb}{0.000000,0.000000,0.000000}%
\pgfsetstrokecolor{currentstroke}%
\pgfsetdash{}{0pt}%
\pgfpathmoveto{\pgfqpoint{3.529386in}{3.528085in}}%
\pgfpathlineto{\pgfqpoint{3.715612in}{3.528085in}}%
\pgfpathlineto{\pgfqpoint{3.715612in}{3.609813in}}%
\pgfpathlineto{\pgfqpoint{3.529386in}{3.609813in}}%
\pgfpathlineto{\pgfqpoint{3.529386in}{3.528085in}}%
\pgfusepath{}%
\end{pgfscope}%
\begin{pgfscope}%
\pgfpathrectangle{\pgfqpoint{0.549740in}{0.463273in}}{\pgfqpoint{9.320225in}{4.495057in}}%
\pgfusepath{clip}%
\pgfsetbuttcap%
\pgfsetroundjoin%
\pgfsetlinewidth{0.000000pt}%
\definecolor{currentstroke}{rgb}{0.000000,0.000000,0.000000}%
\pgfsetstrokecolor{currentstroke}%
\pgfsetdash{}{0pt}%
\pgfpathmoveto{\pgfqpoint{3.715612in}{3.528085in}}%
\pgfpathlineto{\pgfqpoint{3.901839in}{3.528085in}}%
\pgfpathlineto{\pgfqpoint{3.901839in}{3.609813in}}%
\pgfpathlineto{\pgfqpoint{3.715612in}{3.609813in}}%
\pgfpathlineto{\pgfqpoint{3.715612in}{3.528085in}}%
\pgfusepath{}%
\end{pgfscope}%
\begin{pgfscope}%
\pgfpathrectangle{\pgfqpoint{0.549740in}{0.463273in}}{\pgfqpoint{9.320225in}{4.495057in}}%
\pgfusepath{clip}%
\pgfsetbuttcap%
\pgfsetroundjoin%
\pgfsetlinewidth{0.000000pt}%
\definecolor{currentstroke}{rgb}{0.000000,0.000000,0.000000}%
\pgfsetstrokecolor{currentstroke}%
\pgfsetdash{}{0pt}%
\pgfpathmoveto{\pgfqpoint{3.901839in}{3.528085in}}%
\pgfpathlineto{\pgfqpoint{4.088065in}{3.528085in}}%
\pgfpathlineto{\pgfqpoint{4.088065in}{3.609813in}}%
\pgfpathlineto{\pgfqpoint{3.901839in}{3.609813in}}%
\pgfpathlineto{\pgfqpoint{3.901839in}{3.528085in}}%
\pgfusepath{}%
\end{pgfscope}%
\begin{pgfscope}%
\pgfpathrectangle{\pgfqpoint{0.549740in}{0.463273in}}{\pgfqpoint{9.320225in}{4.495057in}}%
\pgfusepath{clip}%
\pgfsetbuttcap%
\pgfsetroundjoin%
\pgfsetlinewidth{0.000000pt}%
\definecolor{currentstroke}{rgb}{0.000000,0.000000,0.000000}%
\pgfsetstrokecolor{currentstroke}%
\pgfsetdash{}{0pt}%
\pgfpathmoveto{\pgfqpoint{4.088065in}{3.528085in}}%
\pgfpathlineto{\pgfqpoint{4.274292in}{3.528085in}}%
\pgfpathlineto{\pgfqpoint{4.274292in}{3.609813in}}%
\pgfpathlineto{\pgfqpoint{4.088065in}{3.609813in}}%
\pgfpathlineto{\pgfqpoint{4.088065in}{3.528085in}}%
\pgfusepath{}%
\end{pgfscope}%
\begin{pgfscope}%
\pgfpathrectangle{\pgfqpoint{0.549740in}{0.463273in}}{\pgfqpoint{9.320225in}{4.495057in}}%
\pgfusepath{clip}%
\pgfsetbuttcap%
\pgfsetroundjoin%
\pgfsetlinewidth{0.000000pt}%
\definecolor{currentstroke}{rgb}{0.000000,0.000000,0.000000}%
\pgfsetstrokecolor{currentstroke}%
\pgfsetdash{}{0pt}%
\pgfpathmoveto{\pgfqpoint{4.274292in}{3.528085in}}%
\pgfpathlineto{\pgfqpoint{4.460519in}{3.528085in}}%
\pgfpathlineto{\pgfqpoint{4.460519in}{3.609813in}}%
\pgfpathlineto{\pgfqpoint{4.274292in}{3.609813in}}%
\pgfpathlineto{\pgfqpoint{4.274292in}{3.528085in}}%
\pgfusepath{}%
\end{pgfscope}%
\begin{pgfscope}%
\pgfpathrectangle{\pgfqpoint{0.549740in}{0.463273in}}{\pgfqpoint{9.320225in}{4.495057in}}%
\pgfusepath{clip}%
\pgfsetbuttcap%
\pgfsetroundjoin%
\pgfsetlinewidth{0.000000pt}%
\definecolor{currentstroke}{rgb}{0.000000,0.000000,0.000000}%
\pgfsetstrokecolor{currentstroke}%
\pgfsetdash{}{0pt}%
\pgfpathmoveto{\pgfqpoint{4.460519in}{3.528085in}}%
\pgfpathlineto{\pgfqpoint{4.646745in}{3.528085in}}%
\pgfpathlineto{\pgfqpoint{4.646745in}{3.609813in}}%
\pgfpathlineto{\pgfqpoint{4.460519in}{3.609813in}}%
\pgfpathlineto{\pgfqpoint{4.460519in}{3.528085in}}%
\pgfusepath{}%
\end{pgfscope}%
\begin{pgfscope}%
\pgfpathrectangle{\pgfqpoint{0.549740in}{0.463273in}}{\pgfqpoint{9.320225in}{4.495057in}}%
\pgfusepath{clip}%
\pgfsetbuttcap%
\pgfsetroundjoin%
\pgfsetlinewidth{0.000000pt}%
\definecolor{currentstroke}{rgb}{0.000000,0.000000,0.000000}%
\pgfsetstrokecolor{currentstroke}%
\pgfsetdash{}{0pt}%
\pgfpathmoveto{\pgfqpoint{4.646745in}{3.528085in}}%
\pgfpathlineto{\pgfqpoint{4.832972in}{3.528085in}}%
\pgfpathlineto{\pgfqpoint{4.832972in}{3.609813in}}%
\pgfpathlineto{\pgfqpoint{4.646745in}{3.609813in}}%
\pgfpathlineto{\pgfqpoint{4.646745in}{3.528085in}}%
\pgfusepath{}%
\end{pgfscope}%
\begin{pgfscope}%
\pgfpathrectangle{\pgfqpoint{0.549740in}{0.463273in}}{\pgfqpoint{9.320225in}{4.495057in}}%
\pgfusepath{clip}%
\pgfsetbuttcap%
\pgfsetroundjoin%
\pgfsetlinewidth{0.000000pt}%
\definecolor{currentstroke}{rgb}{0.000000,0.000000,0.000000}%
\pgfsetstrokecolor{currentstroke}%
\pgfsetdash{}{0pt}%
\pgfpathmoveto{\pgfqpoint{4.832972in}{3.528085in}}%
\pgfpathlineto{\pgfqpoint{5.019198in}{3.528085in}}%
\pgfpathlineto{\pgfqpoint{5.019198in}{3.609813in}}%
\pgfpathlineto{\pgfqpoint{4.832972in}{3.609813in}}%
\pgfpathlineto{\pgfqpoint{4.832972in}{3.528085in}}%
\pgfusepath{}%
\end{pgfscope}%
\begin{pgfscope}%
\pgfpathrectangle{\pgfqpoint{0.549740in}{0.463273in}}{\pgfqpoint{9.320225in}{4.495057in}}%
\pgfusepath{clip}%
\pgfsetbuttcap%
\pgfsetroundjoin%
\pgfsetlinewidth{0.000000pt}%
\definecolor{currentstroke}{rgb}{0.000000,0.000000,0.000000}%
\pgfsetstrokecolor{currentstroke}%
\pgfsetdash{}{0pt}%
\pgfpathmoveto{\pgfqpoint{5.019198in}{3.528085in}}%
\pgfpathlineto{\pgfqpoint{5.205425in}{3.528085in}}%
\pgfpathlineto{\pgfqpoint{5.205425in}{3.609813in}}%
\pgfpathlineto{\pgfqpoint{5.019198in}{3.609813in}}%
\pgfpathlineto{\pgfqpoint{5.019198in}{3.528085in}}%
\pgfusepath{}%
\end{pgfscope}%
\begin{pgfscope}%
\pgfpathrectangle{\pgfqpoint{0.549740in}{0.463273in}}{\pgfqpoint{9.320225in}{4.495057in}}%
\pgfusepath{clip}%
\pgfsetbuttcap%
\pgfsetroundjoin%
\pgfsetlinewidth{0.000000pt}%
\definecolor{currentstroke}{rgb}{0.000000,0.000000,0.000000}%
\pgfsetstrokecolor{currentstroke}%
\pgfsetdash{}{0pt}%
\pgfpathmoveto{\pgfqpoint{5.205425in}{3.528085in}}%
\pgfpathlineto{\pgfqpoint{5.391651in}{3.528085in}}%
\pgfpathlineto{\pgfqpoint{5.391651in}{3.609813in}}%
\pgfpathlineto{\pgfqpoint{5.205425in}{3.609813in}}%
\pgfpathlineto{\pgfqpoint{5.205425in}{3.528085in}}%
\pgfusepath{}%
\end{pgfscope}%
\begin{pgfscope}%
\pgfpathrectangle{\pgfqpoint{0.549740in}{0.463273in}}{\pgfqpoint{9.320225in}{4.495057in}}%
\pgfusepath{clip}%
\pgfsetbuttcap%
\pgfsetroundjoin%
\pgfsetlinewidth{0.000000pt}%
\definecolor{currentstroke}{rgb}{0.000000,0.000000,0.000000}%
\pgfsetstrokecolor{currentstroke}%
\pgfsetdash{}{0pt}%
\pgfpathmoveto{\pgfqpoint{5.391651in}{3.528085in}}%
\pgfpathlineto{\pgfqpoint{5.577878in}{3.528085in}}%
\pgfpathlineto{\pgfqpoint{5.577878in}{3.609813in}}%
\pgfpathlineto{\pgfqpoint{5.391651in}{3.609813in}}%
\pgfpathlineto{\pgfqpoint{5.391651in}{3.528085in}}%
\pgfusepath{}%
\end{pgfscope}%
\begin{pgfscope}%
\pgfpathrectangle{\pgfqpoint{0.549740in}{0.463273in}}{\pgfqpoint{9.320225in}{4.495057in}}%
\pgfusepath{clip}%
\pgfsetbuttcap%
\pgfsetroundjoin%
\pgfsetlinewidth{0.000000pt}%
\definecolor{currentstroke}{rgb}{0.000000,0.000000,0.000000}%
\pgfsetstrokecolor{currentstroke}%
\pgfsetdash{}{0pt}%
\pgfpathmoveto{\pgfqpoint{5.577878in}{3.528085in}}%
\pgfpathlineto{\pgfqpoint{5.764104in}{3.528085in}}%
\pgfpathlineto{\pgfqpoint{5.764104in}{3.609813in}}%
\pgfpathlineto{\pgfqpoint{5.577878in}{3.609813in}}%
\pgfpathlineto{\pgfqpoint{5.577878in}{3.528085in}}%
\pgfusepath{}%
\end{pgfscope}%
\begin{pgfscope}%
\pgfpathrectangle{\pgfqpoint{0.549740in}{0.463273in}}{\pgfqpoint{9.320225in}{4.495057in}}%
\pgfusepath{clip}%
\pgfsetbuttcap%
\pgfsetroundjoin%
\pgfsetlinewidth{0.000000pt}%
\definecolor{currentstroke}{rgb}{0.000000,0.000000,0.000000}%
\pgfsetstrokecolor{currentstroke}%
\pgfsetdash{}{0pt}%
\pgfpathmoveto{\pgfqpoint{5.764104in}{3.528085in}}%
\pgfpathlineto{\pgfqpoint{5.950331in}{3.528085in}}%
\pgfpathlineto{\pgfqpoint{5.950331in}{3.609813in}}%
\pgfpathlineto{\pgfqpoint{5.764104in}{3.609813in}}%
\pgfpathlineto{\pgfqpoint{5.764104in}{3.528085in}}%
\pgfusepath{}%
\end{pgfscope}%
\begin{pgfscope}%
\pgfpathrectangle{\pgfqpoint{0.549740in}{0.463273in}}{\pgfqpoint{9.320225in}{4.495057in}}%
\pgfusepath{clip}%
\pgfsetbuttcap%
\pgfsetroundjoin%
\pgfsetlinewidth{0.000000pt}%
\definecolor{currentstroke}{rgb}{0.000000,0.000000,0.000000}%
\pgfsetstrokecolor{currentstroke}%
\pgfsetdash{}{0pt}%
\pgfpathmoveto{\pgfqpoint{5.950331in}{3.528085in}}%
\pgfpathlineto{\pgfqpoint{6.136557in}{3.528085in}}%
\pgfpathlineto{\pgfqpoint{6.136557in}{3.609813in}}%
\pgfpathlineto{\pgfqpoint{5.950331in}{3.609813in}}%
\pgfpathlineto{\pgfqpoint{5.950331in}{3.528085in}}%
\pgfusepath{}%
\end{pgfscope}%
\begin{pgfscope}%
\pgfpathrectangle{\pgfqpoint{0.549740in}{0.463273in}}{\pgfqpoint{9.320225in}{4.495057in}}%
\pgfusepath{clip}%
\pgfsetbuttcap%
\pgfsetroundjoin%
\pgfsetlinewidth{0.000000pt}%
\definecolor{currentstroke}{rgb}{0.000000,0.000000,0.000000}%
\pgfsetstrokecolor{currentstroke}%
\pgfsetdash{}{0pt}%
\pgfpathmoveto{\pgfqpoint{6.136557in}{3.528085in}}%
\pgfpathlineto{\pgfqpoint{6.322784in}{3.528085in}}%
\pgfpathlineto{\pgfqpoint{6.322784in}{3.609813in}}%
\pgfpathlineto{\pgfqpoint{6.136557in}{3.609813in}}%
\pgfpathlineto{\pgfqpoint{6.136557in}{3.528085in}}%
\pgfusepath{}%
\end{pgfscope}%
\begin{pgfscope}%
\pgfpathrectangle{\pgfqpoint{0.549740in}{0.463273in}}{\pgfqpoint{9.320225in}{4.495057in}}%
\pgfusepath{clip}%
\pgfsetbuttcap%
\pgfsetroundjoin%
\pgfsetlinewidth{0.000000pt}%
\definecolor{currentstroke}{rgb}{0.000000,0.000000,0.000000}%
\pgfsetstrokecolor{currentstroke}%
\pgfsetdash{}{0pt}%
\pgfpathmoveto{\pgfqpoint{6.322784in}{3.528085in}}%
\pgfpathlineto{\pgfqpoint{6.509011in}{3.528085in}}%
\pgfpathlineto{\pgfqpoint{6.509011in}{3.609813in}}%
\pgfpathlineto{\pgfqpoint{6.322784in}{3.609813in}}%
\pgfpathlineto{\pgfqpoint{6.322784in}{3.528085in}}%
\pgfusepath{}%
\end{pgfscope}%
\begin{pgfscope}%
\pgfpathrectangle{\pgfqpoint{0.549740in}{0.463273in}}{\pgfqpoint{9.320225in}{4.495057in}}%
\pgfusepath{clip}%
\pgfsetbuttcap%
\pgfsetroundjoin%
\pgfsetlinewidth{0.000000pt}%
\definecolor{currentstroke}{rgb}{0.000000,0.000000,0.000000}%
\pgfsetstrokecolor{currentstroke}%
\pgfsetdash{}{0pt}%
\pgfpathmoveto{\pgfqpoint{6.509011in}{3.528085in}}%
\pgfpathlineto{\pgfqpoint{6.695237in}{3.528085in}}%
\pgfpathlineto{\pgfqpoint{6.695237in}{3.609813in}}%
\pgfpathlineto{\pgfqpoint{6.509011in}{3.609813in}}%
\pgfpathlineto{\pgfqpoint{6.509011in}{3.528085in}}%
\pgfusepath{}%
\end{pgfscope}%
\begin{pgfscope}%
\pgfpathrectangle{\pgfqpoint{0.549740in}{0.463273in}}{\pgfqpoint{9.320225in}{4.495057in}}%
\pgfusepath{clip}%
\pgfsetbuttcap%
\pgfsetroundjoin%
\pgfsetlinewidth{0.000000pt}%
\definecolor{currentstroke}{rgb}{0.000000,0.000000,0.000000}%
\pgfsetstrokecolor{currentstroke}%
\pgfsetdash{}{0pt}%
\pgfpathmoveto{\pgfqpoint{6.695237in}{3.528085in}}%
\pgfpathlineto{\pgfqpoint{6.881464in}{3.528085in}}%
\pgfpathlineto{\pgfqpoint{6.881464in}{3.609813in}}%
\pgfpathlineto{\pgfqpoint{6.695237in}{3.609813in}}%
\pgfpathlineto{\pgfqpoint{6.695237in}{3.528085in}}%
\pgfusepath{}%
\end{pgfscope}%
\begin{pgfscope}%
\pgfpathrectangle{\pgfqpoint{0.549740in}{0.463273in}}{\pgfqpoint{9.320225in}{4.495057in}}%
\pgfusepath{clip}%
\pgfsetbuttcap%
\pgfsetroundjoin%
\definecolor{currentfill}{rgb}{0.472869,0.711325,0.955316}%
\pgfsetfillcolor{currentfill}%
\pgfsetlinewidth{0.000000pt}%
\definecolor{currentstroke}{rgb}{0.000000,0.000000,0.000000}%
\pgfsetstrokecolor{currentstroke}%
\pgfsetdash{}{0pt}%
\pgfpathmoveto{\pgfqpoint{6.881464in}{3.528085in}}%
\pgfpathlineto{\pgfqpoint{7.067690in}{3.528085in}}%
\pgfpathlineto{\pgfqpoint{7.067690in}{3.609813in}}%
\pgfpathlineto{\pgfqpoint{6.881464in}{3.609813in}}%
\pgfpathlineto{\pgfqpoint{6.881464in}{3.528085in}}%
\pgfusepath{fill}%
\end{pgfscope}%
\begin{pgfscope}%
\pgfpathrectangle{\pgfqpoint{0.549740in}{0.463273in}}{\pgfqpoint{9.320225in}{4.495057in}}%
\pgfusepath{clip}%
\pgfsetbuttcap%
\pgfsetroundjoin%
\pgfsetlinewidth{0.000000pt}%
\definecolor{currentstroke}{rgb}{0.000000,0.000000,0.000000}%
\pgfsetstrokecolor{currentstroke}%
\pgfsetdash{}{0pt}%
\pgfpathmoveto{\pgfqpoint{7.067690in}{3.528085in}}%
\pgfpathlineto{\pgfqpoint{7.253917in}{3.528085in}}%
\pgfpathlineto{\pgfqpoint{7.253917in}{3.609813in}}%
\pgfpathlineto{\pgfqpoint{7.067690in}{3.609813in}}%
\pgfpathlineto{\pgfqpoint{7.067690in}{3.528085in}}%
\pgfusepath{}%
\end{pgfscope}%
\begin{pgfscope}%
\pgfpathrectangle{\pgfqpoint{0.549740in}{0.463273in}}{\pgfqpoint{9.320225in}{4.495057in}}%
\pgfusepath{clip}%
\pgfsetbuttcap%
\pgfsetroundjoin%
\pgfsetlinewidth{0.000000pt}%
\definecolor{currentstroke}{rgb}{0.000000,0.000000,0.000000}%
\pgfsetstrokecolor{currentstroke}%
\pgfsetdash{}{0pt}%
\pgfpathmoveto{\pgfqpoint{7.253917in}{3.528085in}}%
\pgfpathlineto{\pgfqpoint{7.440143in}{3.528085in}}%
\pgfpathlineto{\pgfqpoint{7.440143in}{3.609813in}}%
\pgfpathlineto{\pgfqpoint{7.253917in}{3.609813in}}%
\pgfpathlineto{\pgfqpoint{7.253917in}{3.528085in}}%
\pgfusepath{}%
\end{pgfscope}%
\begin{pgfscope}%
\pgfpathrectangle{\pgfqpoint{0.549740in}{0.463273in}}{\pgfqpoint{9.320225in}{4.495057in}}%
\pgfusepath{clip}%
\pgfsetbuttcap%
\pgfsetroundjoin%
\pgfsetlinewidth{0.000000pt}%
\definecolor{currentstroke}{rgb}{0.000000,0.000000,0.000000}%
\pgfsetstrokecolor{currentstroke}%
\pgfsetdash{}{0pt}%
\pgfpathmoveto{\pgfqpoint{7.440143in}{3.528085in}}%
\pgfpathlineto{\pgfqpoint{7.626370in}{3.528085in}}%
\pgfpathlineto{\pgfqpoint{7.626370in}{3.609813in}}%
\pgfpathlineto{\pgfqpoint{7.440143in}{3.609813in}}%
\pgfpathlineto{\pgfqpoint{7.440143in}{3.528085in}}%
\pgfusepath{}%
\end{pgfscope}%
\begin{pgfscope}%
\pgfpathrectangle{\pgfqpoint{0.549740in}{0.463273in}}{\pgfqpoint{9.320225in}{4.495057in}}%
\pgfusepath{clip}%
\pgfsetbuttcap%
\pgfsetroundjoin%
\pgfsetlinewidth{0.000000pt}%
\definecolor{currentstroke}{rgb}{0.000000,0.000000,0.000000}%
\pgfsetstrokecolor{currentstroke}%
\pgfsetdash{}{0pt}%
\pgfpathmoveto{\pgfqpoint{7.626370in}{3.528085in}}%
\pgfpathlineto{\pgfqpoint{7.812596in}{3.528085in}}%
\pgfpathlineto{\pgfqpoint{7.812596in}{3.609813in}}%
\pgfpathlineto{\pgfqpoint{7.626370in}{3.609813in}}%
\pgfpathlineto{\pgfqpoint{7.626370in}{3.528085in}}%
\pgfusepath{}%
\end{pgfscope}%
\begin{pgfscope}%
\pgfpathrectangle{\pgfqpoint{0.549740in}{0.463273in}}{\pgfqpoint{9.320225in}{4.495057in}}%
\pgfusepath{clip}%
\pgfsetbuttcap%
\pgfsetroundjoin%
\pgfsetlinewidth{0.000000pt}%
\definecolor{currentstroke}{rgb}{0.000000,0.000000,0.000000}%
\pgfsetstrokecolor{currentstroke}%
\pgfsetdash{}{0pt}%
\pgfpathmoveto{\pgfqpoint{7.812596in}{3.528085in}}%
\pgfpathlineto{\pgfqpoint{7.998823in}{3.528085in}}%
\pgfpathlineto{\pgfqpoint{7.998823in}{3.609813in}}%
\pgfpathlineto{\pgfqpoint{7.812596in}{3.609813in}}%
\pgfpathlineto{\pgfqpoint{7.812596in}{3.528085in}}%
\pgfusepath{}%
\end{pgfscope}%
\begin{pgfscope}%
\pgfpathrectangle{\pgfqpoint{0.549740in}{0.463273in}}{\pgfqpoint{9.320225in}{4.495057in}}%
\pgfusepath{clip}%
\pgfsetbuttcap%
\pgfsetroundjoin%
\definecolor{currentfill}{rgb}{0.385185,0.686583,0.962589}%
\pgfsetfillcolor{currentfill}%
\pgfsetlinewidth{0.000000pt}%
\definecolor{currentstroke}{rgb}{0.000000,0.000000,0.000000}%
\pgfsetstrokecolor{currentstroke}%
\pgfsetdash{}{0pt}%
\pgfpathmoveto{\pgfqpoint{7.998823in}{3.528085in}}%
\pgfpathlineto{\pgfqpoint{8.185049in}{3.528085in}}%
\pgfpathlineto{\pgfqpoint{8.185049in}{3.609813in}}%
\pgfpathlineto{\pgfqpoint{7.998823in}{3.609813in}}%
\pgfpathlineto{\pgfqpoint{7.998823in}{3.528085in}}%
\pgfusepath{fill}%
\end{pgfscope}%
\begin{pgfscope}%
\pgfpathrectangle{\pgfqpoint{0.549740in}{0.463273in}}{\pgfqpoint{9.320225in}{4.495057in}}%
\pgfusepath{clip}%
\pgfsetbuttcap%
\pgfsetroundjoin%
\pgfsetlinewidth{0.000000pt}%
\definecolor{currentstroke}{rgb}{0.000000,0.000000,0.000000}%
\pgfsetstrokecolor{currentstroke}%
\pgfsetdash{}{0pt}%
\pgfpathmoveto{\pgfqpoint{8.185049in}{3.528085in}}%
\pgfpathlineto{\pgfqpoint{8.371276in}{3.528085in}}%
\pgfpathlineto{\pgfqpoint{8.371276in}{3.609813in}}%
\pgfpathlineto{\pgfqpoint{8.185049in}{3.609813in}}%
\pgfpathlineto{\pgfqpoint{8.185049in}{3.528085in}}%
\pgfusepath{}%
\end{pgfscope}%
\begin{pgfscope}%
\pgfpathrectangle{\pgfqpoint{0.549740in}{0.463273in}}{\pgfqpoint{9.320225in}{4.495057in}}%
\pgfusepath{clip}%
\pgfsetbuttcap%
\pgfsetroundjoin%
\pgfsetlinewidth{0.000000pt}%
\definecolor{currentstroke}{rgb}{0.000000,0.000000,0.000000}%
\pgfsetstrokecolor{currentstroke}%
\pgfsetdash{}{0pt}%
\pgfpathmoveto{\pgfqpoint{8.371276in}{3.528085in}}%
\pgfpathlineto{\pgfqpoint{8.557503in}{3.528085in}}%
\pgfpathlineto{\pgfqpoint{8.557503in}{3.609813in}}%
\pgfpathlineto{\pgfqpoint{8.371276in}{3.609813in}}%
\pgfpathlineto{\pgfqpoint{8.371276in}{3.528085in}}%
\pgfusepath{}%
\end{pgfscope}%
\begin{pgfscope}%
\pgfpathrectangle{\pgfqpoint{0.549740in}{0.463273in}}{\pgfqpoint{9.320225in}{4.495057in}}%
\pgfusepath{clip}%
\pgfsetbuttcap%
\pgfsetroundjoin%
\pgfsetlinewidth{0.000000pt}%
\definecolor{currentstroke}{rgb}{0.000000,0.000000,0.000000}%
\pgfsetstrokecolor{currentstroke}%
\pgfsetdash{}{0pt}%
\pgfpathmoveto{\pgfqpoint{8.557503in}{3.528085in}}%
\pgfpathlineto{\pgfqpoint{8.743729in}{3.528085in}}%
\pgfpathlineto{\pgfqpoint{8.743729in}{3.609813in}}%
\pgfpathlineto{\pgfqpoint{8.557503in}{3.609813in}}%
\pgfpathlineto{\pgfqpoint{8.557503in}{3.528085in}}%
\pgfusepath{}%
\end{pgfscope}%
\begin{pgfscope}%
\pgfpathrectangle{\pgfqpoint{0.549740in}{0.463273in}}{\pgfqpoint{9.320225in}{4.495057in}}%
\pgfusepath{clip}%
\pgfsetbuttcap%
\pgfsetroundjoin%
\pgfsetlinewidth{0.000000pt}%
\definecolor{currentstroke}{rgb}{0.000000,0.000000,0.000000}%
\pgfsetstrokecolor{currentstroke}%
\pgfsetdash{}{0pt}%
\pgfpathmoveto{\pgfqpoint{8.743729in}{3.528085in}}%
\pgfpathlineto{\pgfqpoint{8.929956in}{3.528085in}}%
\pgfpathlineto{\pgfqpoint{8.929956in}{3.609813in}}%
\pgfpathlineto{\pgfqpoint{8.743729in}{3.609813in}}%
\pgfpathlineto{\pgfqpoint{8.743729in}{3.528085in}}%
\pgfusepath{}%
\end{pgfscope}%
\begin{pgfscope}%
\pgfpathrectangle{\pgfqpoint{0.549740in}{0.463273in}}{\pgfqpoint{9.320225in}{4.495057in}}%
\pgfusepath{clip}%
\pgfsetbuttcap%
\pgfsetroundjoin%
\pgfsetlinewidth{0.000000pt}%
\definecolor{currentstroke}{rgb}{0.000000,0.000000,0.000000}%
\pgfsetstrokecolor{currentstroke}%
\pgfsetdash{}{0pt}%
\pgfpathmoveto{\pgfqpoint{8.929956in}{3.528085in}}%
\pgfpathlineto{\pgfqpoint{9.116182in}{3.528085in}}%
\pgfpathlineto{\pgfqpoint{9.116182in}{3.609813in}}%
\pgfpathlineto{\pgfqpoint{8.929956in}{3.609813in}}%
\pgfpathlineto{\pgfqpoint{8.929956in}{3.528085in}}%
\pgfusepath{}%
\end{pgfscope}%
\begin{pgfscope}%
\pgfpathrectangle{\pgfqpoint{0.549740in}{0.463273in}}{\pgfqpoint{9.320225in}{4.495057in}}%
\pgfusepath{clip}%
\pgfsetbuttcap%
\pgfsetroundjoin%
\pgfsetlinewidth{0.000000pt}%
\definecolor{currentstroke}{rgb}{0.000000,0.000000,0.000000}%
\pgfsetstrokecolor{currentstroke}%
\pgfsetdash{}{0pt}%
\pgfpathmoveto{\pgfqpoint{9.116182in}{3.528085in}}%
\pgfpathlineto{\pgfqpoint{9.302409in}{3.528085in}}%
\pgfpathlineto{\pgfqpoint{9.302409in}{3.609813in}}%
\pgfpathlineto{\pgfqpoint{9.116182in}{3.609813in}}%
\pgfpathlineto{\pgfqpoint{9.116182in}{3.528085in}}%
\pgfusepath{}%
\end{pgfscope}%
\begin{pgfscope}%
\pgfpathrectangle{\pgfqpoint{0.549740in}{0.463273in}}{\pgfqpoint{9.320225in}{4.495057in}}%
\pgfusepath{clip}%
\pgfsetbuttcap%
\pgfsetroundjoin%
\definecolor{currentfill}{rgb}{0.472869,0.711325,0.955316}%
\pgfsetfillcolor{currentfill}%
\pgfsetlinewidth{0.000000pt}%
\definecolor{currentstroke}{rgb}{0.000000,0.000000,0.000000}%
\pgfsetstrokecolor{currentstroke}%
\pgfsetdash{}{0pt}%
\pgfpathmoveto{\pgfqpoint{9.302409in}{3.528085in}}%
\pgfpathlineto{\pgfqpoint{9.488635in}{3.528085in}}%
\pgfpathlineto{\pgfqpoint{9.488635in}{3.609813in}}%
\pgfpathlineto{\pgfqpoint{9.302409in}{3.609813in}}%
\pgfpathlineto{\pgfqpoint{9.302409in}{3.528085in}}%
\pgfusepath{fill}%
\end{pgfscope}%
\begin{pgfscope}%
\pgfpathrectangle{\pgfqpoint{0.549740in}{0.463273in}}{\pgfqpoint{9.320225in}{4.495057in}}%
\pgfusepath{clip}%
\pgfsetbuttcap%
\pgfsetroundjoin%
\pgfsetlinewidth{0.000000pt}%
\definecolor{currentstroke}{rgb}{0.000000,0.000000,0.000000}%
\pgfsetstrokecolor{currentstroke}%
\pgfsetdash{}{0pt}%
\pgfpathmoveto{\pgfqpoint{9.488635in}{3.528085in}}%
\pgfpathlineto{\pgfqpoint{9.674862in}{3.528085in}}%
\pgfpathlineto{\pgfqpoint{9.674862in}{3.609813in}}%
\pgfpathlineto{\pgfqpoint{9.488635in}{3.609813in}}%
\pgfpathlineto{\pgfqpoint{9.488635in}{3.528085in}}%
\pgfusepath{}%
\end{pgfscope}%
\begin{pgfscope}%
\pgfpathrectangle{\pgfqpoint{0.549740in}{0.463273in}}{\pgfqpoint{9.320225in}{4.495057in}}%
\pgfusepath{clip}%
\pgfsetbuttcap%
\pgfsetroundjoin%
\pgfsetlinewidth{0.000000pt}%
\definecolor{currentstroke}{rgb}{0.000000,0.000000,0.000000}%
\pgfsetstrokecolor{currentstroke}%
\pgfsetdash{}{0pt}%
\pgfpathmoveto{\pgfqpoint{9.674862in}{3.528085in}}%
\pgfpathlineto{\pgfqpoint{9.861088in}{3.528085in}}%
\pgfpathlineto{\pgfqpoint{9.861088in}{3.609813in}}%
\pgfpathlineto{\pgfqpoint{9.674862in}{3.609813in}}%
\pgfpathlineto{\pgfqpoint{9.674862in}{3.528085in}}%
\pgfusepath{}%
\end{pgfscope}%
\begin{pgfscope}%
\pgfpathrectangle{\pgfqpoint{0.549740in}{0.463273in}}{\pgfqpoint{9.320225in}{4.495057in}}%
\pgfusepath{clip}%
\pgfsetbuttcap%
\pgfsetroundjoin%
\pgfsetlinewidth{0.000000pt}%
\definecolor{currentstroke}{rgb}{0.000000,0.000000,0.000000}%
\pgfsetstrokecolor{currentstroke}%
\pgfsetdash{}{0pt}%
\pgfpathmoveto{\pgfqpoint{0.549761in}{3.609813in}}%
\pgfpathlineto{\pgfqpoint{0.735988in}{3.609813in}}%
\pgfpathlineto{\pgfqpoint{0.735988in}{3.691541in}}%
\pgfpathlineto{\pgfqpoint{0.549761in}{3.691541in}}%
\pgfpathlineto{\pgfqpoint{0.549761in}{3.609813in}}%
\pgfusepath{}%
\end{pgfscope}%
\begin{pgfscope}%
\pgfpathrectangle{\pgfqpoint{0.549740in}{0.463273in}}{\pgfqpoint{9.320225in}{4.495057in}}%
\pgfusepath{clip}%
\pgfsetbuttcap%
\pgfsetroundjoin%
\pgfsetlinewidth{0.000000pt}%
\definecolor{currentstroke}{rgb}{0.000000,0.000000,0.000000}%
\pgfsetstrokecolor{currentstroke}%
\pgfsetdash{}{0pt}%
\pgfpathmoveto{\pgfqpoint{0.735988in}{3.609813in}}%
\pgfpathlineto{\pgfqpoint{0.922214in}{3.609813in}}%
\pgfpathlineto{\pgfqpoint{0.922214in}{3.691541in}}%
\pgfpathlineto{\pgfqpoint{0.735988in}{3.691541in}}%
\pgfpathlineto{\pgfqpoint{0.735988in}{3.609813in}}%
\pgfusepath{}%
\end{pgfscope}%
\begin{pgfscope}%
\pgfpathrectangle{\pgfqpoint{0.549740in}{0.463273in}}{\pgfqpoint{9.320225in}{4.495057in}}%
\pgfusepath{clip}%
\pgfsetbuttcap%
\pgfsetroundjoin%
\pgfsetlinewidth{0.000000pt}%
\definecolor{currentstroke}{rgb}{0.000000,0.000000,0.000000}%
\pgfsetstrokecolor{currentstroke}%
\pgfsetdash{}{0pt}%
\pgfpathmoveto{\pgfqpoint{0.922214in}{3.609813in}}%
\pgfpathlineto{\pgfqpoint{1.108441in}{3.609813in}}%
\pgfpathlineto{\pgfqpoint{1.108441in}{3.691541in}}%
\pgfpathlineto{\pgfqpoint{0.922214in}{3.691541in}}%
\pgfpathlineto{\pgfqpoint{0.922214in}{3.609813in}}%
\pgfusepath{}%
\end{pgfscope}%
\begin{pgfscope}%
\pgfpathrectangle{\pgfqpoint{0.549740in}{0.463273in}}{\pgfqpoint{9.320225in}{4.495057in}}%
\pgfusepath{clip}%
\pgfsetbuttcap%
\pgfsetroundjoin%
\pgfsetlinewidth{0.000000pt}%
\definecolor{currentstroke}{rgb}{0.000000,0.000000,0.000000}%
\pgfsetstrokecolor{currentstroke}%
\pgfsetdash{}{0pt}%
\pgfpathmoveto{\pgfqpoint{1.108441in}{3.609813in}}%
\pgfpathlineto{\pgfqpoint{1.294667in}{3.609813in}}%
\pgfpathlineto{\pgfqpoint{1.294667in}{3.691541in}}%
\pgfpathlineto{\pgfqpoint{1.108441in}{3.691541in}}%
\pgfpathlineto{\pgfqpoint{1.108441in}{3.609813in}}%
\pgfusepath{}%
\end{pgfscope}%
\begin{pgfscope}%
\pgfpathrectangle{\pgfqpoint{0.549740in}{0.463273in}}{\pgfqpoint{9.320225in}{4.495057in}}%
\pgfusepath{clip}%
\pgfsetbuttcap%
\pgfsetroundjoin%
\pgfsetlinewidth{0.000000pt}%
\definecolor{currentstroke}{rgb}{0.000000,0.000000,0.000000}%
\pgfsetstrokecolor{currentstroke}%
\pgfsetdash{}{0pt}%
\pgfpathmoveto{\pgfqpoint{1.294667in}{3.609813in}}%
\pgfpathlineto{\pgfqpoint{1.480894in}{3.609813in}}%
\pgfpathlineto{\pgfqpoint{1.480894in}{3.691541in}}%
\pgfpathlineto{\pgfqpoint{1.294667in}{3.691541in}}%
\pgfpathlineto{\pgfqpoint{1.294667in}{3.609813in}}%
\pgfusepath{}%
\end{pgfscope}%
\begin{pgfscope}%
\pgfpathrectangle{\pgfqpoint{0.549740in}{0.463273in}}{\pgfqpoint{9.320225in}{4.495057in}}%
\pgfusepath{clip}%
\pgfsetbuttcap%
\pgfsetroundjoin%
\pgfsetlinewidth{0.000000pt}%
\definecolor{currentstroke}{rgb}{0.000000,0.000000,0.000000}%
\pgfsetstrokecolor{currentstroke}%
\pgfsetdash{}{0pt}%
\pgfpathmoveto{\pgfqpoint{1.480894in}{3.609813in}}%
\pgfpathlineto{\pgfqpoint{1.667120in}{3.609813in}}%
\pgfpathlineto{\pgfqpoint{1.667120in}{3.691541in}}%
\pgfpathlineto{\pgfqpoint{1.480894in}{3.691541in}}%
\pgfpathlineto{\pgfqpoint{1.480894in}{3.609813in}}%
\pgfusepath{}%
\end{pgfscope}%
\begin{pgfscope}%
\pgfpathrectangle{\pgfqpoint{0.549740in}{0.463273in}}{\pgfqpoint{9.320225in}{4.495057in}}%
\pgfusepath{clip}%
\pgfsetbuttcap%
\pgfsetroundjoin%
\pgfsetlinewidth{0.000000pt}%
\definecolor{currentstroke}{rgb}{0.000000,0.000000,0.000000}%
\pgfsetstrokecolor{currentstroke}%
\pgfsetdash{}{0pt}%
\pgfpathmoveto{\pgfqpoint{1.667120in}{3.609813in}}%
\pgfpathlineto{\pgfqpoint{1.853347in}{3.609813in}}%
\pgfpathlineto{\pgfqpoint{1.853347in}{3.691541in}}%
\pgfpathlineto{\pgfqpoint{1.667120in}{3.691541in}}%
\pgfpathlineto{\pgfqpoint{1.667120in}{3.609813in}}%
\pgfusepath{}%
\end{pgfscope}%
\begin{pgfscope}%
\pgfpathrectangle{\pgfqpoint{0.549740in}{0.463273in}}{\pgfqpoint{9.320225in}{4.495057in}}%
\pgfusepath{clip}%
\pgfsetbuttcap%
\pgfsetroundjoin%
\pgfsetlinewidth{0.000000pt}%
\definecolor{currentstroke}{rgb}{0.000000,0.000000,0.000000}%
\pgfsetstrokecolor{currentstroke}%
\pgfsetdash{}{0pt}%
\pgfpathmoveto{\pgfqpoint{1.853347in}{3.609813in}}%
\pgfpathlineto{\pgfqpoint{2.039573in}{3.609813in}}%
\pgfpathlineto{\pgfqpoint{2.039573in}{3.691541in}}%
\pgfpathlineto{\pgfqpoint{1.853347in}{3.691541in}}%
\pgfpathlineto{\pgfqpoint{1.853347in}{3.609813in}}%
\pgfusepath{}%
\end{pgfscope}%
\begin{pgfscope}%
\pgfpathrectangle{\pgfqpoint{0.549740in}{0.463273in}}{\pgfqpoint{9.320225in}{4.495057in}}%
\pgfusepath{clip}%
\pgfsetbuttcap%
\pgfsetroundjoin%
\pgfsetlinewidth{0.000000pt}%
\definecolor{currentstroke}{rgb}{0.000000,0.000000,0.000000}%
\pgfsetstrokecolor{currentstroke}%
\pgfsetdash{}{0pt}%
\pgfpathmoveto{\pgfqpoint{2.039573in}{3.609813in}}%
\pgfpathlineto{\pgfqpoint{2.225800in}{3.609813in}}%
\pgfpathlineto{\pgfqpoint{2.225800in}{3.691541in}}%
\pgfpathlineto{\pgfqpoint{2.039573in}{3.691541in}}%
\pgfpathlineto{\pgfqpoint{2.039573in}{3.609813in}}%
\pgfusepath{}%
\end{pgfscope}%
\begin{pgfscope}%
\pgfpathrectangle{\pgfqpoint{0.549740in}{0.463273in}}{\pgfqpoint{9.320225in}{4.495057in}}%
\pgfusepath{clip}%
\pgfsetbuttcap%
\pgfsetroundjoin%
\pgfsetlinewidth{0.000000pt}%
\definecolor{currentstroke}{rgb}{0.000000,0.000000,0.000000}%
\pgfsetstrokecolor{currentstroke}%
\pgfsetdash{}{0pt}%
\pgfpathmoveto{\pgfqpoint{2.225800in}{3.609813in}}%
\pgfpathlineto{\pgfqpoint{2.412027in}{3.609813in}}%
\pgfpathlineto{\pgfqpoint{2.412027in}{3.691541in}}%
\pgfpathlineto{\pgfqpoint{2.225800in}{3.691541in}}%
\pgfpathlineto{\pgfqpoint{2.225800in}{3.609813in}}%
\pgfusepath{}%
\end{pgfscope}%
\begin{pgfscope}%
\pgfpathrectangle{\pgfqpoint{0.549740in}{0.463273in}}{\pgfqpoint{9.320225in}{4.495057in}}%
\pgfusepath{clip}%
\pgfsetbuttcap%
\pgfsetroundjoin%
\pgfsetlinewidth{0.000000pt}%
\definecolor{currentstroke}{rgb}{0.000000,0.000000,0.000000}%
\pgfsetstrokecolor{currentstroke}%
\pgfsetdash{}{0pt}%
\pgfpathmoveto{\pgfqpoint{2.412027in}{3.609813in}}%
\pgfpathlineto{\pgfqpoint{2.598253in}{3.609813in}}%
\pgfpathlineto{\pgfqpoint{2.598253in}{3.691541in}}%
\pgfpathlineto{\pgfqpoint{2.412027in}{3.691541in}}%
\pgfpathlineto{\pgfqpoint{2.412027in}{3.609813in}}%
\pgfusepath{}%
\end{pgfscope}%
\begin{pgfscope}%
\pgfpathrectangle{\pgfqpoint{0.549740in}{0.463273in}}{\pgfqpoint{9.320225in}{4.495057in}}%
\pgfusepath{clip}%
\pgfsetbuttcap%
\pgfsetroundjoin%
\pgfsetlinewidth{0.000000pt}%
\definecolor{currentstroke}{rgb}{0.000000,0.000000,0.000000}%
\pgfsetstrokecolor{currentstroke}%
\pgfsetdash{}{0pt}%
\pgfpathmoveto{\pgfqpoint{2.598253in}{3.609813in}}%
\pgfpathlineto{\pgfqpoint{2.784480in}{3.609813in}}%
\pgfpathlineto{\pgfqpoint{2.784480in}{3.691541in}}%
\pgfpathlineto{\pgfqpoint{2.598253in}{3.691541in}}%
\pgfpathlineto{\pgfqpoint{2.598253in}{3.609813in}}%
\pgfusepath{}%
\end{pgfscope}%
\begin{pgfscope}%
\pgfpathrectangle{\pgfqpoint{0.549740in}{0.463273in}}{\pgfqpoint{9.320225in}{4.495057in}}%
\pgfusepath{clip}%
\pgfsetbuttcap%
\pgfsetroundjoin%
\pgfsetlinewidth{0.000000pt}%
\definecolor{currentstroke}{rgb}{0.000000,0.000000,0.000000}%
\pgfsetstrokecolor{currentstroke}%
\pgfsetdash{}{0pt}%
\pgfpathmoveto{\pgfqpoint{2.784480in}{3.609813in}}%
\pgfpathlineto{\pgfqpoint{2.970706in}{3.609813in}}%
\pgfpathlineto{\pgfqpoint{2.970706in}{3.691541in}}%
\pgfpathlineto{\pgfqpoint{2.784480in}{3.691541in}}%
\pgfpathlineto{\pgfqpoint{2.784480in}{3.609813in}}%
\pgfusepath{}%
\end{pgfscope}%
\begin{pgfscope}%
\pgfpathrectangle{\pgfqpoint{0.549740in}{0.463273in}}{\pgfqpoint{9.320225in}{4.495057in}}%
\pgfusepath{clip}%
\pgfsetbuttcap%
\pgfsetroundjoin%
\pgfsetlinewidth{0.000000pt}%
\definecolor{currentstroke}{rgb}{0.000000,0.000000,0.000000}%
\pgfsetstrokecolor{currentstroke}%
\pgfsetdash{}{0pt}%
\pgfpathmoveto{\pgfqpoint{2.970706in}{3.609813in}}%
\pgfpathlineto{\pgfqpoint{3.156933in}{3.609813in}}%
\pgfpathlineto{\pgfqpoint{3.156933in}{3.691541in}}%
\pgfpathlineto{\pgfqpoint{2.970706in}{3.691541in}}%
\pgfpathlineto{\pgfqpoint{2.970706in}{3.609813in}}%
\pgfusepath{}%
\end{pgfscope}%
\begin{pgfscope}%
\pgfpathrectangle{\pgfqpoint{0.549740in}{0.463273in}}{\pgfqpoint{9.320225in}{4.495057in}}%
\pgfusepath{clip}%
\pgfsetbuttcap%
\pgfsetroundjoin%
\pgfsetlinewidth{0.000000pt}%
\definecolor{currentstroke}{rgb}{0.000000,0.000000,0.000000}%
\pgfsetstrokecolor{currentstroke}%
\pgfsetdash{}{0pt}%
\pgfpathmoveto{\pgfqpoint{3.156933in}{3.609813in}}%
\pgfpathlineto{\pgfqpoint{3.343159in}{3.609813in}}%
\pgfpathlineto{\pgfqpoint{3.343159in}{3.691541in}}%
\pgfpathlineto{\pgfqpoint{3.156933in}{3.691541in}}%
\pgfpathlineto{\pgfqpoint{3.156933in}{3.609813in}}%
\pgfusepath{}%
\end{pgfscope}%
\begin{pgfscope}%
\pgfpathrectangle{\pgfqpoint{0.549740in}{0.463273in}}{\pgfqpoint{9.320225in}{4.495057in}}%
\pgfusepath{clip}%
\pgfsetbuttcap%
\pgfsetroundjoin%
\pgfsetlinewidth{0.000000pt}%
\definecolor{currentstroke}{rgb}{0.000000,0.000000,0.000000}%
\pgfsetstrokecolor{currentstroke}%
\pgfsetdash{}{0pt}%
\pgfpathmoveto{\pgfqpoint{3.343159in}{3.609813in}}%
\pgfpathlineto{\pgfqpoint{3.529386in}{3.609813in}}%
\pgfpathlineto{\pgfqpoint{3.529386in}{3.691541in}}%
\pgfpathlineto{\pgfqpoint{3.343159in}{3.691541in}}%
\pgfpathlineto{\pgfqpoint{3.343159in}{3.609813in}}%
\pgfusepath{}%
\end{pgfscope}%
\begin{pgfscope}%
\pgfpathrectangle{\pgfqpoint{0.549740in}{0.463273in}}{\pgfqpoint{9.320225in}{4.495057in}}%
\pgfusepath{clip}%
\pgfsetbuttcap%
\pgfsetroundjoin%
\pgfsetlinewidth{0.000000pt}%
\definecolor{currentstroke}{rgb}{0.000000,0.000000,0.000000}%
\pgfsetstrokecolor{currentstroke}%
\pgfsetdash{}{0pt}%
\pgfpathmoveto{\pgfqpoint{3.529386in}{3.609813in}}%
\pgfpathlineto{\pgfqpoint{3.715612in}{3.609813in}}%
\pgfpathlineto{\pgfqpoint{3.715612in}{3.691541in}}%
\pgfpathlineto{\pgfqpoint{3.529386in}{3.691541in}}%
\pgfpathlineto{\pgfqpoint{3.529386in}{3.609813in}}%
\pgfusepath{}%
\end{pgfscope}%
\begin{pgfscope}%
\pgfpathrectangle{\pgfqpoint{0.549740in}{0.463273in}}{\pgfqpoint{9.320225in}{4.495057in}}%
\pgfusepath{clip}%
\pgfsetbuttcap%
\pgfsetroundjoin%
\pgfsetlinewidth{0.000000pt}%
\definecolor{currentstroke}{rgb}{0.000000,0.000000,0.000000}%
\pgfsetstrokecolor{currentstroke}%
\pgfsetdash{}{0pt}%
\pgfpathmoveto{\pgfqpoint{3.715612in}{3.609813in}}%
\pgfpathlineto{\pgfqpoint{3.901839in}{3.609813in}}%
\pgfpathlineto{\pgfqpoint{3.901839in}{3.691541in}}%
\pgfpathlineto{\pgfqpoint{3.715612in}{3.691541in}}%
\pgfpathlineto{\pgfqpoint{3.715612in}{3.609813in}}%
\pgfusepath{}%
\end{pgfscope}%
\begin{pgfscope}%
\pgfpathrectangle{\pgfqpoint{0.549740in}{0.463273in}}{\pgfqpoint{9.320225in}{4.495057in}}%
\pgfusepath{clip}%
\pgfsetbuttcap%
\pgfsetroundjoin%
\pgfsetlinewidth{0.000000pt}%
\definecolor{currentstroke}{rgb}{0.000000,0.000000,0.000000}%
\pgfsetstrokecolor{currentstroke}%
\pgfsetdash{}{0pt}%
\pgfpathmoveto{\pgfqpoint{3.901839in}{3.609813in}}%
\pgfpathlineto{\pgfqpoint{4.088065in}{3.609813in}}%
\pgfpathlineto{\pgfqpoint{4.088065in}{3.691541in}}%
\pgfpathlineto{\pgfqpoint{3.901839in}{3.691541in}}%
\pgfpathlineto{\pgfqpoint{3.901839in}{3.609813in}}%
\pgfusepath{}%
\end{pgfscope}%
\begin{pgfscope}%
\pgfpathrectangle{\pgfqpoint{0.549740in}{0.463273in}}{\pgfqpoint{9.320225in}{4.495057in}}%
\pgfusepath{clip}%
\pgfsetbuttcap%
\pgfsetroundjoin%
\pgfsetlinewidth{0.000000pt}%
\definecolor{currentstroke}{rgb}{0.000000,0.000000,0.000000}%
\pgfsetstrokecolor{currentstroke}%
\pgfsetdash{}{0pt}%
\pgfpathmoveto{\pgfqpoint{4.088065in}{3.609813in}}%
\pgfpathlineto{\pgfqpoint{4.274292in}{3.609813in}}%
\pgfpathlineto{\pgfqpoint{4.274292in}{3.691541in}}%
\pgfpathlineto{\pgfqpoint{4.088065in}{3.691541in}}%
\pgfpathlineto{\pgfqpoint{4.088065in}{3.609813in}}%
\pgfusepath{}%
\end{pgfscope}%
\begin{pgfscope}%
\pgfpathrectangle{\pgfqpoint{0.549740in}{0.463273in}}{\pgfqpoint{9.320225in}{4.495057in}}%
\pgfusepath{clip}%
\pgfsetbuttcap%
\pgfsetroundjoin%
\pgfsetlinewidth{0.000000pt}%
\definecolor{currentstroke}{rgb}{0.000000,0.000000,0.000000}%
\pgfsetstrokecolor{currentstroke}%
\pgfsetdash{}{0pt}%
\pgfpathmoveto{\pgfqpoint{4.274292in}{3.609813in}}%
\pgfpathlineto{\pgfqpoint{4.460519in}{3.609813in}}%
\pgfpathlineto{\pgfqpoint{4.460519in}{3.691541in}}%
\pgfpathlineto{\pgfqpoint{4.274292in}{3.691541in}}%
\pgfpathlineto{\pgfqpoint{4.274292in}{3.609813in}}%
\pgfusepath{}%
\end{pgfscope}%
\begin{pgfscope}%
\pgfpathrectangle{\pgfqpoint{0.549740in}{0.463273in}}{\pgfqpoint{9.320225in}{4.495057in}}%
\pgfusepath{clip}%
\pgfsetbuttcap%
\pgfsetroundjoin%
\pgfsetlinewidth{0.000000pt}%
\definecolor{currentstroke}{rgb}{0.000000,0.000000,0.000000}%
\pgfsetstrokecolor{currentstroke}%
\pgfsetdash{}{0pt}%
\pgfpathmoveto{\pgfqpoint{4.460519in}{3.609813in}}%
\pgfpathlineto{\pgfqpoint{4.646745in}{3.609813in}}%
\pgfpathlineto{\pgfqpoint{4.646745in}{3.691541in}}%
\pgfpathlineto{\pgfqpoint{4.460519in}{3.691541in}}%
\pgfpathlineto{\pgfqpoint{4.460519in}{3.609813in}}%
\pgfusepath{}%
\end{pgfscope}%
\begin{pgfscope}%
\pgfpathrectangle{\pgfqpoint{0.549740in}{0.463273in}}{\pgfqpoint{9.320225in}{4.495057in}}%
\pgfusepath{clip}%
\pgfsetbuttcap%
\pgfsetroundjoin%
\pgfsetlinewidth{0.000000pt}%
\definecolor{currentstroke}{rgb}{0.000000,0.000000,0.000000}%
\pgfsetstrokecolor{currentstroke}%
\pgfsetdash{}{0pt}%
\pgfpathmoveto{\pgfqpoint{4.646745in}{3.609813in}}%
\pgfpathlineto{\pgfqpoint{4.832972in}{3.609813in}}%
\pgfpathlineto{\pgfqpoint{4.832972in}{3.691541in}}%
\pgfpathlineto{\pgfqpoint{4.646745in}{3.691541in}}%
\pgfpathlineto{\pgfqpoint{4.646745in}{3.609813in}}%
\pgfusepath{}%
\end{pgfscope}%
\begin{pgfscope}%
\pgfpathrectangle{\pgfqpoint{0.549740in}{0.463273in}}{\pgfqpoint{9.320225in}{4.495057in}}%
\pgfusepath{clip}%
\pgfsetbuttcap%
\pgfsetroundjoin%
\pgfsetlinewidth{0.000000pt}%
\definecolor{currentstroke}{rgb}{0.000000,0.000000,0.000000}%
\pgfsetstrokecolor{currentstroke}%
\pgfsetdash{}{0pt}%
\pgfpathmoveto{\pgfqpoint{4.832972in}{3.609813in}}%
\pgfpathlineto{\pgfqpoint{5.019198in}{3.609813in}}%
\pgfpathlineto{\pgfqpoint{5.019198in}{3.691541in}}%
\pgfpathlineto{\pgfqpoint{4.832972in}{3.691541in}}%
\pgfpathlineto{\pgfqpoint{4.832972in}{3.609813in}}%
\pgfusepath{}%
\end{pgfscope}%
\begin{pgfscope}%
\pgfpathrectangle{\pgfqpoint{0.549740in}{0.463273in}}{\pgfqpoint{9.320225in}{4.495057in}}%
\pgfusepath{clip}%
\pgfsetbuttcap%
\pgfsetroundjoin%
\pgfsetlinewidth{0.000000pt}%
\definecolor{currentstroke}{rgb}{0.000000,0.000000,0.000000}%
\pgfsetstrokecolor{currentstroke}%
\pgfsetdash{}{0pt}%
\pgfpathmoveto{\pgfqpoint{5.019198in}{3.609813in}}%
\pgfpathlineto{\pgfqpoint{5.205425in}{3.609813in}}%
\pgfpathlineto{\pgfqpoint{5.205425in}{3.691541in}}%
\pgfpathlineto{\pgfqpoint{5.019198in}{3.691541in}}%
\pgfpathlineto{\pgfqpoint{5.019198in}{3.609813in}}%
\pgfusepath{}%
\end{pgfscope}%
\begin{pgfscope}%
\pgfpathrectangle{\pgfqpoint{0.549740in}{0.463273in}}{\pgfqpoint{9.320225in}{4.495057in}}%
\pgfusepath{clip}%
\pgfsetbuttcap%
\pgfsetroundjoin%
\pgfsetlinewidth{0.000000pt}%
\definecolor{currentstroke}{rgb}{0.000000,0.000000,0.000000}%
\pgfsetstrokecolor{currentstroke}%
\pgfsetdash{}{0pt}%
\pgfpathmoveto{\pgfqpoint{5.205425in}{3.609813in}}%
\pgfpathlineto{\pgfqpoint{5.391651in}{3.609813in}}%
\pgfpathlineto{\pgfqpoint{5.391651in}{3.691541in}}%
\pgfpathlineto{\pgfqpoint{5.205425in}{3.691541in}}%
\pgfpathlineto{\pgfqpoint{5.205425in}{3.609813in}}%
\pgfusepath{}%
\end{pgfscope}%
\begin{pgfscope}%
\pgfpathrectangle{\pgfqpoint{0.549740in}{0.463273in}}{\pgfqpoint{9.320225in}{4.495057in}}%
\pgfusepath{clip}%
\pgfsetbuttcap%
\pgfsetroundjoin%
\pgfsetlinewidth{0.000000pt}%
\definecolor{currentstroke}{rgb}{0.000000,0.000000,0.000000}%
\pgfsetstrokecolor{currentstroke}%
\pgfsetdash{}{0pt}%
\pgfpathmoveto{\pgfqpoint{5.391651in}{3.609813in}}%
\pgfpathlineto{\pgfqpoint{5.577878in}{3.609813in}}%
\pgfpathlineto{\pgfqpoint{5.577878in}{3.691541in}}%
\pgfpathlineto{\pgfqpoint{5.391651in}{3.691541in}}%
\pgfpathlineto{\pgfqpoint{5.391651in}{3.609813in}}%
\pgfusepath{}%
\end{pgfscope}%
\begin{pgfscope}%
\pgfpathrectangle{\pgfqpoint{0.549740in}{0.463273in}}{\pgfqpoint{9.320225in}{4.495057in}}%
\pgfusepath{clip}%
\pgfsetbuttcap%
\pgfsetroundjoin%
\pgfsetlinewidth{0.000000pt}%
\definecolor{currentstroke}{rgb}{0.000000,0.000000,0.000000}%
\pgfsetstrokecolor{currentstroke}%
\pgfsetdash{}{0pt}%
\pgfpathmoveto{\pgfqpoint{5.577878in}{3.609813in}}%
\pgfpathlineto{\pgfqpoint{5.764104in}{3.609813in}}%
\pgfpathlineto{\pgfqpoint{5.764104in}{3.691541in}}%
\pgfpathlineto{\pgfqpoint{5.577878in}{3.691541in}}%
\pgfpathlineto{\pgfqpoint{5.577878in}{3.609813in}}%
\pgfusepath{}%
\end{pgfscope}%
\begin{pgfscope}%
\pgfpathrectangle{\pgfqpoint{0.549740in}{0.463273in}}{\pgfqpoint{9.320225in}{4.495057in}}%
\pgfusepath{clip}%
\pgfsetbuttcap%
\pgfsetroundjoin%
\pgfsetlinewidth{0.000000pt}%
\definecolor{currentstroke}{rgb}{0.000000,0.000000,0.000000}%
\pgfsetstrokecolor{currentstroke}%
\pgfsetdash{}{0pt}%
\pgfpathmoveto{\pgfqpoint{5.764104in}{3.609813in}}%
\pgfpathlineto{\pgfqpoint{5.950331in}{3.609813in}}%
\pgfpathlineto{\pgfqpoint{5.950331in}{3.691541in}}%
\pgfpathlineto{\pgfqpoint{5.764104in}{3.691541in}}%
\pgfpathlineto{\pgfqpoint{5.764104in}{3.609813in}}%
\pgfusepath{}%
\end{pgfscope}%
\begin{pgfscope}%
\pgfpathrectangle{\pgfqpoint{0.549740in}{0.463273in}}{\pgfqpoint{9.320225in}{4.495057in}}%
\pgfusepath{clip}%
\pgfsetbuttcap%
\pgfsetroundjoin%
\pgfsetlinewidth{0.000000pt}%
\definecolor{currentstroke}{rgb}{0.000000,0.000000,0.000000}%
\pgfsetstrokecolor{currentstroke}%
\pgfsetdash{}{0pt}%
\pgfpathmoveto{\pgfqpoint{5.950331in}{3.609813in}}%
\pgfpathlineto{\pgfqpoint{6.136557in}{3.609813in}}%
\pgfpathlineto{\pgfqpoint{6.136557in}{3.691541in}}%
\pgfpathlineto{\pgfqpoint{5.950331in}{3.691541in}}%
\pgfpathlineto{\pgfqpoint{5.950331in}{3.609813in}}%
\pgfusepath{}%
\end{pgfscope}%
\begin{pgfscope}%
\pgfpathrectangle{\pgfqpoint{0.549740in}{0.463273in}}{\pgfqpoint{9.320225in}{4.495057in}}%
\pgfusepath{clip}%
\pgfsetbuttcap%
\pgfsetroundjoin%
\pgfsetlinewidth{0.000000pt}%
\definecolor{currentstroke}{rgb}{0.000000,0.000000,0.000000}%
\pgfsetstrokecolor{currentstroke}%
\pgfsetdash{}{0pt}%
\pgfpathmoveto{\pgfqpoint{6.136557in}{3.609813in}}%
\pgfpathlineto{\pgfqpoint{6.322784in}{3.609813in}}%
\pgfpathlineto{\pgfqpoint{6.322784in}{3.691541in}}%
\pgfpathlineto{\pgfqpoint{6.136557in}{3.691541in}}%
\pgfpathlineto{\pgfqpoint{6.136557in}{3.609813in}}%
\pgfusepath{}%
\end{pgfscope}%
\begin{pgfscope}%
\pgfpathrectangle{\pgfqpoint{0.549740in}{0.463273in}}{\pgfqpoint{9.320225in}{4.495057in}}%
\pgfusepath{clip}%
\pgfsetbuttcap%
\pgfsetroundjoin%
\pgfsetlinewidth{0.000000pt}%
\definecolor{currentstroke}{rgb}{0.000000,0.000000,0.000000}%
\pgfsetstrokecolor{currentstroke}%
\pgfsetdash{}{0pt}%
\pgfpathmoveto{\pgfqpoint{6.322784in}{3.609813in}}%
\pgfpathlineto{\pgfqpoint{6.509011in}{3.609813in}}%
\pgfpathlineto{\pgfqpoint{6.509011in}{3.691541in}}%
\pgfpathlineto{\pgfqpoint{6.322784in}{3.691541in}}%
\pgfpathlineto{\pgfqpoint{6.322784in}{3.609813in}}%
\pgfusepath{}%
\end{pgfscope}%
\begin{pgfscope}%
\pgfpathrectangle{\pgfqpoint{0.549740in}{0.463273in}}{\pgfqpoint{9.320225in}{4.495057in}}%
\pgfusepath{clip}%
\pgfsetbuttcap%
\pgfsetroundjoin%
\pgfsetlinewidth{0.000000pt}%
\definecolor{currentstroke}{rgb}{0.000000,0.000000,0.000000}%
\pgfsetstrokecolor{currentstroke}%
\pgfsetdash{}{0pt}%
\pgfpathmoveto{\pgfqpoint{6.509011in}{3.609813in}}%
\pgfpathlineto{\pgfqpoint{6.695237in}{3.609813in}}%
\pgfpathlineto{\pgfqpoint{6.695237in}{3.691541in}}%
\pgfpathlineto{\pgfqpoint{6.509011in}{3.691541in}}%
\pgfpathlineto{\pgfqpoint{6.509011in}{3.609813in}}%
\pgfusepath{}%
\end{pgfscope}%
\begin{pgfscope}%
\pgfpathrectangle{\pgfqpoint{0.549740in}{0.463273in}}{\pgfqpoint{9.320225in}{4.495057in}}%
\pgfusepath{clip}%
\pgfsetbuttcap%
\pgfsetroundjoin%
\pgfsetlinewidth{0.000000pt}%
\definecolor{currentstroke}{rgb}{0.000000,0.000000,0.000000}%
\pgfsetstrokecolor{currentstroke}%
\pgfsetdash{}{0pt}%
\pgfpathmoveto{\pgfqpoint{6.695237in}{3.609813in}}%
\pgfpathlineto{\pgfqpoint{6.881464in}{3.609813in}}%
\pgfpathlineto{\pgfqpoint{6.881464in}{3.691541in}}%
\pgfpathlineto{\pgfqpoint{6.695237in}{3.691541in}}%
\pgfpathlineto{\pgfqpoint{6.695237in}{3.609813in}}%
\pgfusepath{}%
\end{pgfscope}%
\begin{pgfscope}%
\pgfpathrectangle{\pgfqpoint{0.549740in}{0.463273in}}{\pgfqpoint{9.320225in}{4.495057in}}%
\pgfusepath{clip}%
\pgfsetbuttcap%
\pgfsetroundjoin%
\definecolor{currentfill}{rgb}{0.472869,0.711325,0.955316}%
\pgfsetfillcolor{currentfill}%
\pgfsetlinewidth{0.000000pt}%
\definecolor{currentstroke}{rgb}{0.000000,0.000000,0.000000}%
\pgfsetstrokecolor{currentstroke}%
\pgfsetdash{}{0pt}%
\pgfpathmoveto{\pgfqpoint{6.881464in}{3.609813in}}%
\pgfpathlineto{\pgfqpoint{7.067690in}{3.609813in}}%
\pgfpathlineto{\pgfqpoint{7.067690in}{3.691541in}}%
\pgfpathlineto{\pgfqpoint{6.881464in}{3.691541in}}%
\pgfpathlineto{\pgfqpoint{6.881464in}{3.609813in}}%
\pgfusepath{fill}%
\end{pgfscope}%
\begin{pgfscope}%
\pgfpathrectangle{\pgfqpoint{0.549740in}{0.463273in}}{\pgfqpoint{9.320225in}{4.495057in}}%
\pgfusepath{clip}%
\pgfsetbuttcap%
\pgfsetroundjoin%
\pgfsetlinewidth{0.000000pt}%
\definecolor{currentstroke}{rgb}{0.000000,0.000000,0.000000}%
\pgfsetstrokecolor{currentstroke}%
\pgfsetdash{}{0pt}%
\pgfpathmoveto{\pgfqpoint{7.067690in}{3.609813in}}%
\pgfpathlineto{\pgfqpoint{7.253917in}{3.609813in}}%
\pgfpathlineto{\pgfqpoint{7.253917in}{3.691541in}}%
\pgfpathlineto{\pgfqpoint{7.067690in}{3.691541in}}%
\pgfpathlineto{\pgfqpoint{7.067690in}{3.609813in}}%
\pgfusepath{}%
\end{pgfscope}%
\begin{pgfscope}%
\pgfpathrectangle{\pgfqpoint{0.549740in}{0.463273in}}{\pgfqpoint{9.320225in}{4.495057in}}%
\pgfusepath{clip}%
\pgfsetbuttcap%
\pgfsetroundjoin%
\pgfsetlinewidth{0.000000pt}%
\definecolor{currentstroke}{rgb}{0.000000,0.000000,0.000000}%
\pgfsetstrokecolor{currentstroke}%
\pgfsetdash{}{0pt}%
\pgfpathmoveto{\pgfqpoint{7.253917in}{3.609813in}}%
\pgfpathlineto{\pgfqpoint{7.440143in}{3.609813in}}%
\pgfpathlineto{\pgfqpoint{7.440143in}{3.691541in}}%
\pgfpathlineto{\pgfqpoint{7.253917in}{3.691541in}}%
\pgfpathlineto{\pgfqpoint{7.253917in}{3.609813in}}%
\pgfusepath{}%
\end{pgfscope}%
\begin{pgfscope}%
\pgfpathrectangle{\pgfqpoint{0.549740in}{0.463273in}}{\pgfqpoint{9.320225in}{4.495057in}}%
\pgfusepath{clip}%
\pgfsetbuttcap%
\pgfsetroundjoin%
\pgfsetlinewidth{0.000000pt}%
\definecolor{currentstroke}{rgb}{0.000000,0.000000,0.000000}%
\pgfsetstrokecolor{currentstroke}%
\pgfsetdash{}{0pt}%
\pgfpathmoveto{\pgfqpoint{7.440143in}{3.609813in}}%
\pgfpathlineto{\pgfqpoint{7.626370in}{3.609813in}}%
\pgfpathlineto{\pgfqpoint{7.626370in}{3.691541in}}%
\pgfpathlineto{\pgfqpoint{7.440143in}{3.691541in}}%
\pgfpathlineto{\pgfqpoint{7.440143in}{3.609813in}}%
\pgfusepath{}%
\end{pgfscope}%
\begin{pgfscope}%
\pgfpathrectangle{\pgfqpoint{0.549740in}{0.463273in}}{\pgfqpoint{9.320225in}{4.495057in}}%
\pgfusepath{clip}%
\pgfsetbuttcap%
\pgfsetroundjoin%
\pgfsetlinewidth{0.000000pt}%
\definecolor{currentstroke}{rgb}{0.000000,0.000000,0.000000}%
\pgfsetstrokecolor{currentstroke}%
\pgfsetdash{}{0pt}%
\pgfpathmoveto{\pgfqpoint{7.626370in}{3.609813in}}%
\pgfpathlineto{\pgfqpoint{7.812596in}{3.609813in}}%
\pgfpathlineto{\pgfqpoint{7.812596in}{3.691541in}}%
\pgfpathlineto{\pgfqpoint{7.626370in}{3.691541in}}%
\pgfpathlineto{\pgfqpoint{7.626370in}{3.609813in}}%
\pgfusepath{}%
\end{pgfscope}%
\begin{pgfscope}%
\pgfpathrectangle{\pgfqpoint{0.549740in}{0.463273in}}{\pgfqpoint{9.320225in}{4.495057in}}%
\pgfusepath{clip}%
\pgfsetbuttcap%
\pgfsetroundjoin%
\pgfsetlinewidth{0.000000pt}%
\definecolor{currentstroke}{rgb}{0.000000,0.000000,0.000000}%
\pgfsetstrokecolor{currentstroke}%
\pgfsetdash{}{0pt}%
\pgfpathmoveto{\pgfqpoint{7.812596in}{3.609813in}}%
\pgfpathlineto{\pgfqpoint{7.998823in}{3.609813in}}%
\pgfpathlineto{\pgfqpoint{7.998823in}{3.691541in}}%
\pgfpathlineto{\pgfqpoint{7.812596in}{3.691541in}}%
\pgfpathlineto{\pgfqpoint{7.812596in}{3.609813in}}%
\pgfusepath{}%
\end{pgfscope}%
\begin{pgfscope}%
\pgfpathrectangle{\pgfqpoint{0.549740in}{0.463273in}}{\pgfqpoint{9.320225in}{4.495057in}}%
\pgfusepath{clip}%
\pgfsetbuttcap%
\pgfsetroundjoin%
\definecolor{currentfill}{rgb}{0.472869,0.711325,0.955316}%
\pgfsetfillcolor{currentfill}%
\pgfsetlinewidth{0.000000pt}%
\definecolor{currentstroke}{rgb}{0.000000,0.000000,0.000000}%
\pgfsetstrokecolor{currentstroke}%
\pgfsetdash{}{0pt}%
\pgfpathmoveto{\pgfqpoint{7.998823in}{3.609813in}}%
\pgfpathlineto{\pgfqpoint{8.185049in}{3.609813in}}%
\pgfpathlineto{\pgfqpoint{8.185049in}{3.691541in}}%
\pgfpathlineto{\pgfqpoint{7.998823in}{3.691541in}}%
\pgfpathlineto{\pgfqpoint{7.998823in}{3.609813in}}%
\pgfusepath{fill}%
\end{pgfscope}%
\begin{pgfscope}%
\pgfpathrectangle{\pgfqpoint{0.549740in}{0.463273in}}{\pgfqpoint{9.320225in}{4.495057in}}%
\pgfusepath{clip}%
\pgfsetbuttcap%
\pgfsetroundjoin%
\pgfsetlinewidth{0.000000pt}%
\definecolor{currentstroke}{rgb}{0.000000,0.000000,0.000000}%
\pgfsetstrokecolor{currentstroke}%
\pgfsetdash{}{0pt}%
\pgfpathmoveto{\pgfqpoint{8.185049in}{3.609813in}}%
\pgfpathlineto{\pgfqpoint{8.371276in}{3.609813in}}%
\pgfpathlineto{\pgfqpoint{8.371276in}{3.691541in}}%
\pgfpathlineto{\pgfqpoint{8.185049in}{3.691541in}}%
\pgfpathlineto{\pgfqpoint{8.185049in}{3.609813in}}%
\pgfusepath{}%
\end{pgfscope}%
\begin{pgfscope}%
\pgfpathrectangle{\pgfqpoint{0.549740in}{0.463273in}}{\pgfqpoint{9.320225in}{4.495057in}}%
\pgfusepath{clip}%
\pgfsetbuttcap%
\pgfsetroundjoin%
\pgfsetlinewidth{0.000000pt}%
\definecolor{currentstroke}{rgb}{0.000000,0.000000,0.000000}%
\pgfsetstrokecolor{currentstroke}%
\pgfsetdash{}{0pt}%
\pgfpathmoveto{\pgfqpoint{8.371276in}{3.609813in}}%
\pgfpathlineto{\pgfqpoint{8.557503in}{3.609813in}}%
\pgfpathlineto{\pgfqpoint{8.557503in}{3.691541in}}%
\pgfpathlineto{\pgfqpoint{8.371276in}{3.691541in}}%
\pgfpathlineto{\pgfqpoint{8.371276in}{3.609813in}}%
\pgfusepath{}%
\end{pgfscope}%
\begin{pgfscope}%
\pgfpathrectangle{\pgfqpoint{0.549740in}{0.463273in}}{\pgfqpoint{9.320225in}{4.495057in}}%
\pgfusepath{clip}%
\pgfsetbuttcap%
\pgfsetroundjoin%
\pgfsetlinewidth{0.000000pt}%
\definecolor{currentstroke}{rgb}{0.000000,0.000000,0.000000}%
\pgfsetstrokecolor{currentstroke}%
\pgfsetdash{}{0pt}%
\pgfpathmoveto{\pgfqpoint{8.557503in}{3.609813in}}%
\pgfpathlineto{\pgfqpoint{8.743729in}{3.609813in}}%
\pgfpathlineto{\pgfqpoint{8.743729in}{3.691541in}}%
\pgfpathlineto{\pgfqpoint{8.557503in}{3.691541in}}%
\pgfpathlineto{\pgfqpoint{8.557503in}{3.609813in}}%
\pgfusepath{}%
\end{pgfscope}%
\begin{pgfscope}%
\pgfpathrectangle{\pgfqpoint{0.549740in}{0.463273in}}{\pgfqpoint{9.320225in}{4.495057in}}%
\pgfusepath{clip}%
\pgfsetbuttcap%
\pgfsetroundjoin%
\pgfsetlinewidth{0.000000pt}%
\definecolor{currentstroke}{rgb}{0.000000,0.000000,0.000000}%
\pgfsetstrokecolor{currentstroke}%
\pgfsetdash{}{0pt}%
\pgfpathmoveto{\pgfqpoint{8.743729in}{3.609813in}}%
\pgfpathlineto{\pgfqpoint{8.929956in}{3.609813in}}%
\pgfpathlineto{\pgfqpoint{8.929956in}{3.691541in}}%
\pgfpathlineto{\pgfqpoint{8.743729in}{3.691541in}}%
\pgfpathlineto{\pgfqpoint{8.743729in}{3.609813in}}%
\pgfusepath{}%
\end{pgfscope}%
\begin{pgfscope}%
\pgfpathrectangle{\pgfqpoint{0.549740in}{0.463273in}}{\pgfqpoint{9.320225in}{4.495057in}}%
\pgfusepath{clip}%
\pgfsetbuttcap%
\pgfsetroundjoin%
\pgfsetlinewidth{0.000000pt}%
\definecolor{currentstroke}{rgb}{0.000000,0.000000,0.000000}%
\pgfsetstrokecolor{currentstroke}%
\pgfsetdash{}{0pt}%
\pgfpathmoveto{\pgfqpoint{8.929956in}{3.609813in}}%
\pgfpathlineto{\pgfqpoint{9.116182in}{3.609813in}}%
\pgfpathlineto{\pgfqpoint{9.116182in}{3.691541in}}%
\pgfpathlineto{\pgfqpoint{8.929956in}{3.691541in}}%
\pgfpathlineto{\pgfqpoint{8.929956in}{3.609813in}}%
\pgfusepath{}%
\end{pgfscope}%
\begin{pgfscope}%
\pgfpathrectangle{\pgfqpoint{0.549740in}{0.463273in}}{\pgfqpoint{9.320225in}{4.495057in}}%
\pgfusepath{clip}%
\pgfsetbuttcap%
\pgfsetroundjoin%
\pgfsetlinewidth{0.000000pt}%
\definecolor{currentstroke}{rgb}{0.000000,0.000000,0.000000}%
\pgfsetstrokecolor{currentstroke}%
\pgfsetdash{}{0pt}%
\pgfpathmoveto{\pgfqpoint{9.116182in}{3.609813in}}%
\pgfpathlineto{\pgfqpoint{9.302409in}{3.609813in}}%
\pgfpathlineto{\pgfqpoint{9.302409in}{3.691541in}}%
\pgfpathlineto{\pgfqpoint{9.116182in}{3.691541in}}%
\pgfpathlineto{\pgfqpoint{9.116182in}{3.609813in}}%
\pgfusepath{}%
\end{pgfscope}%
\begin{pgfscope}%
\pgfpathrectangle{\pgfqpoint{0.549740in}{0.463273in}}{\pgfqpoint{9.320225in}{4.495057in}}%
\pgfusepath{clip}%
\pgfsetbuttcap%
\pgfsetroundjoin%
\definecolor{currentfill}{rgb}{0.472869,0.711325,0.955316}%
\pgfsetfillcolor{currentfill}%
\pgfsetlinewidth{0.000000pt}%
\definecolor{currentstroke}{rgb}{0.000000,0.000000,0.000000}%
\pgfsetstrokecolor{currentstroke}%
\pgfsetdash{}{0pt}%
\pgfpathmoveto{\pgfqpoint{9.302409in}{3.609813in}}%
\pgfpathlineto{\pgfqpoint{9.488635in}{3.609813in}}%
\pgfpathlineto{\pgfqpoint{9.488635in}{3.691541in}}%
\pgfpathlineto{\pgfqpoint{9.302409in}{3.691541in}}%
\pgfpathlineto{\pgfqpoint{9.302409in}{3.609813in}}%
\pgfusepath{fill}%
\end{pgfscope}%
\begin{pgfscope}%
\pgfpathrectangle{\pgfqpoint{0.549740in}{0.463273in}}{\pgfqpoint{9.320225in}{4.495057in}}%
\pgfusepath{clip}%
\pgfsetbuttcap%
\pgfsetroundjoin%
\pgfsetlinewidth{0.000000pt}%
\definecolor{currentstroke}{rgb}{0.000000,0.000000,0.000000}%
\pgfsetstrokecolor{currentstroke}%
\pgfsetdash{}{0pt}%
\pgfpathmoveto{\pgfqpoint{9.488635in}{3.609813in}}%
\pgfpathlineto{\pgfqpoint{9.674862in}{3.609813in}}%
\pgfpathlineto{\pgfqpoint{9.674862in}{3.691541in}}%
\pgfpathlineto{\pgfqpoint{9.488635in}{3.691541in}}%
\pgfpathlineto{\pgfqpoint{9.488635in}{3.609813in}}%
\pgfusepath{}%
\end{pgfscope}%
\begin{pgfscope}%
\pgfpathrectangle{\pgfqpoint{0.549740in}{0.463273in}}{\pgfqpoint{9.320225in}{4.495057in}}%
\pgfusepath{clip}%
\pgfsetbuttcap%
\pgfsetroundjoin%
\pgfsetlinewidth{0.000000pt}%
\definecolor{currentstroke}{rgb}{0.000000,0.000000,0.000000}%
\pgfsetstrokecolor{currentstroke}%
\pgfsetdash{}{0pt}%
\pgfpathmoveto{\pgfqpoint{9.674862in}{3.609813in}}%
\pgfpathlineto{\pgfqpoint{9.861088in}{3.609813in}}%
\pgfpathlineto{\pgfqpoint{9.861088in}{3.691541in}}%
\pgfpathlineto{\pgfqpoint{9.674862in}{3.691541in}}%
\pgfpathlineto{\pgfqpoint{9.674862in}{3.609813in}}%
\pgfusepath{}%
\end{pgfscope}%
\begin{pgfscope}%
\pgfpathrectangle{\pgfqpoint{0.549740in}{0.463273in}}{\pgfqpoint{9.320225in}{4.495057in}}%
\pgfusepath{clip}%
\pgfsetbuttcap%
\pgfsetroundjoin%
\pgfsetlinewidth{0.000000pt}%
\definecolor{currentstroke}{rgb}{0.000000,0.000000,0.000000}%
\pgfsetstrokecolor{currentstroke}%
\pgfsetdash{}{0pt}%
\pgfpathmoveto{\pgfqpoint{0.549761in}{3.691541in}}%
\pgfpathlineto{\pgfqpoint{0.735988in}{3.691541in}}%
\pgfpathlineto{\pgfqpoint{0.735988in}{3.773270in}}%
\pgfpathlineto{\pgfqpoint{0.549761in}{3.773270in}}%
\pgfpathlineto{\pgfqpoint{0.549761in}{3.691541in}}%
\pgfusepath{}%
\end{pgfscope}%
\begin{pgfscope}%
\pgfpathrectangle{\pgfqpoint{0.549740in}{0.463273in}}{\pgfqpoint{9.320225in}{4.495057in}}%
\pgfusepath{clip}%
\pgfsetbuttcap%
\pgfsetroundjoin%
\pgfsetlinewidth{0.000000pt}%
\definecolor{currentstroke}{rgb}{0.000000,0.000000,0.000000}%
\pgfsetstrokecolor{currentstroke}%
\pgfsetdash{}{0pt}%
\pgfpathmoveto{\pgfqpoint{0.735988in}{3.691541in}}%
\pgfpathlineto{\pgfqpoint{0.922214in}{3.691541in}}%
\pgfpathlineto{\pgfqpoint{0.922214in}{3.773270in}}%
\pgfpathlineto{\pgfqpoint{0.735988in}{3.773270in}}%
\pgfpathlineto{\pgfqpoint{0.735988in}{3.691541in}}%
\pgfusepath{}%
\end{pgfscope}%
\begin{pgfscope}%
\pgfpathrectangle{\pgfqpoint{0.549740in}{0.463273in}}{\pgfqpoint{9.320225in}{4.495057in}}%
\pgfusepath{clip}%
\pgfsetbuttcap%
\pgfsetroundjoin%
\pgfsetlinewidth{0.000000pt}%
\definecolor{currentstroke}{rgb}{0.000000,0.000000,0.000000}%
\pgfsetstrokecolor{currentstroke}%
\pgfsetdash{}{0pt}%
\pgfpathmoveto{\pgfqpoint{0.922214in}{3.691541in}}%
\pgfpathlineto{\pgfqpoint{1.108441in}{3.691541in}}%
\pgfpathlineto{\pgfqpoint{1.108441in}{3.773270in}}%
\pgfpathlineto{\pgfqpoint{0.922214in}{3.773270in}}%
\pgfpathlineto{\pgfqpoint{0.922214in}{3.691541in}}%
\pgfusepath{}%
\end{pgfscope}%
\begin{pgfscope}%
\pgfpathrectangle{\pgfqpoint{0.549740in}{0.463273in}}{\pgfqpoint{9.320225in}{4.495057in}}%
\pgfusepath{clip}%
\pgfsetbuttcap%
\pgfsetroundjoin%
\pgfsetlinewidth{0.000000pt}%
\definecolor{currentstroke}{rgb}{0.000000,0.000000,0.000000}%
\pgfsetstrokecolor{currentstroke}%
\pgfsetdash{}{0pt}%
\pgfpathmoveto{\pgfqpoint{1.108441in}{3.691541in}}%
\pgfpathlineto{\pgfqpoint{1.294667in}{3.691541in}}%
\pgfpathlineto{\pgfqpoint{1.294667in}{3.773270in}}%
\pgfpathlineto{\pgfqpoint{1.108441in}{3.773270in}}%
\pgfpathlineto{\pgfqpoint{1.108441in}{3.691541in}}%
\pgfusepath{}%
\end{pgfscope}%
\begin{pgfscope}%
\pgfpathrectangle{\pgfqpoint{0.549740in}{0.463273in}}{\pgfqpoint{9.320225in}{4.495057in}}%
\pgfusepath{clip}%
\pgfsetbuttcap%
\pgfsetroundjoin%
\pgfsetlinewidth{0.000000pt}%
\definecolor{currentstroke}{rgb}{0.000000,0.000000,0.000000}%
\pgfsetstrokecolor{currentstroke}%
\pgfsetdash{}{0pt}%
\pgfpathmoveto{\pgfqpoint{1.294667in}{3.691541in}}%
\pgfpathlineto{\pgfqpoint{1.480894in}{3.691541in}}%
\pgfpathlineto{\pgfqpoint{1.480894in}{3.773270in}}%
\pgfpathlineto{\pgfqpoint{1.294667in}{3.773270in}}%
\pgfpathlineto{\pgfqpoint{1.294667in}{3.691541in}}%
\pgfusepath{}%
\end{pgfscope}%
\begin{pgfscope}%
\pgfpathrectangle{\pgfqpoint{0.549740in}{0.463273in}}{\pgfqpoint{9.320225in}{4.495057in}}%
\pgfusepath{clip}%
\pgfsetbuttcap%
\pgfsetroundjoin%
\pgfsetlinewidth{0.000000pt}%
\definecolor{currentstroke}{rgb}{0.000000,0.000000,0.000000}%
\pgfsetstrokecolor{currentstroke}%
\pgfsetdash{}{0pt}%
\pgfpathmoveto{\pgfqpoint{1.480894in}{3.691541in}}%
\pgfpathlineto{\pgfqpoint{1.667120in}{3.691541in}}%
\pgfpathlineto{\pgfqpoint{1.667120in}{3.773270in}}%
\pgfpathlineto{\pgfqpoint{1.480894in}{3.773270in}}%
\pgfpathlineto{\pgfqpoint{1.480894in}{3.691541in}}%
\pgfusepath{}%
\end{pgfscope}%
\begin{pgfscope}%
\pgfpathrectangle{\pgfqpoint{0.549740in}{0.463273in}}{\pgfqpoint{9.320225in}{4.495057in}}%
\pgfusepath{clip}%
\pgfsetbuttcap%
\pgfsetroundjoin%
\pgfsetlinewidth{0.000000pt}%
\definecolor{currentstroke}{rgb}{0.000000,0.000000,0.000000}%
\pgfsetstrokecolor{currentstroke}%
\pgfsetdash{}{0pt}%
\pgfpathmoveto{\pgfqpoint{1.667120in}{3.691541in}}%
\pgfpathlineto{\pgfqpoint{1.853347in}{3.691541in}}%
\pgfpathlineto{\pgfqpoint{1.853347in}{3.773270in}}%
\pgfpathlineto{\pgfqpoint{1.667120in}{3.773270in}}%
\pgfpathlineto{\pgfqpoint{1.667120in}{3.691541in}}%
\pgfusepath{}%
\end{pgfscope}%
\begin{pgfscope}%
\pgfpathrectangle{\pgfqpoint{0.549740in}{0.463273in}}{\pgfqpoint{9.320225in}{4.495057in}}%
\pgfusepath{clip}%
\pgfsetbuttcap%
\pgfsetroundjoin%
\pgfsetlinewidth{0.000000pt}%
\definecolor{currentstroke}{rgb}{0.000000,0.000000,0.000000}%
\pgfsetstrokecolor{currentstroke}%
\pgfsetdash{}{0pt}%
\pgfpathmoveto{\pgfqpoint{1.853347in}{3.691541in}}%
\pgfpathlineto{\pgfqpoint{2.039573in}{3.691541in}}%
\pgfpathlineto{\pgfqpoint{2.039573in}{3.773270in}}%
\pgfpathlineto{\pgfqpoint{1.853347in}{3.773270in}}%
\pgfpathlineto{\pgfqpoint{1.853347in}{3.691541in}}%
\pgfusepath{}%
\end{pgfscope}%
\begin{pgfscope}%
\pgfpathrectangle{\pgfqpoint{0.549740in}{0.463273in}}{\pgfqpoint{9.320225in}{4.495057in}}%
\pgfusepath{clip}%
\pgfsetbuttcap%
\pgfsetroundjoin%
\pgfsetlinewidth{0.000000pt}%
\definecolor{currentstroke}{rgb}{0.000000,0.000000,0.000000}%
\pgfsetstrokecolor{currentstroke}%
\pgfsetdash{}{0pt}%
\pgfpathmoveto{\pgfqpoint{2.039573in}{3.691541in}}%
\pgfpathlineto{\pgfqpoint{2.225800in}{3.691541in}}%
\pgfpathlineto{\pgfqpoint{2.225800in}{3.773270in}}%
\pgfpathlineto{\pgfqpoint{2.039573in}{3.773270in}}%
\pgfpathlineto{\pgfqpoint{2.039573in}{3.691541in}}%
\pgfusepath{}%
\end{pgfscope}%
\begin{pgfscope}%
\pgfpathrectangle{\pgfqpoint{0.549740in}{0.463273in}}{\pgfqpoint{9.320225in}{4.495057in}}%
\pgfusepath{clip}%
\pgfsetbuttcap%
\pgfsetroundjoin%
\pgfsetlinewidth{0.000000pt}%
\definecolor{currentstroke}{rgb}{0.000000,0.000000,0.000000}%
\pgfsetstrokecolor{currentstroke}%
\pgfsetdash{}{0pt}%
\pgfpathmoveto{\pgfqpoint{2.225800in}{3.691541in}}%
\pgfpathlineto{\pgfqpoint{2.412027in}{3.691541in}}%
\pgfpathlineto{\pgfqpoint{2.412027in}{3.773270in}}%
\pgfpathlineto{\pgfqpoint{2.225800in}{3.773270in}}%
\pgfpathlineto{\pgfqpoint{2.225800in}{3.691541in}}%
\pgfusepath{}%
\end{pgfscope}%
\begin{pgfscope}%
\pgfpathrectangle{\pgfqpoint{0.549740in}{0.463273in}}{\pgfqpoint{9.320225in}{4.495057in}}%
\pgfusepath{clip}%
\pgfsetbuttcap%
\pgfsetroundjoin%
\pgfsetlinewidth{0.000000pt}%
\definecolor{currentstroke}{rgb}{0.000000,0.000000,0.000000}%
\pgfsetstrokecolor{currentstroke}%
\pgfsetdash{}{0pt}%
\pgfpathmoveto{\pgfqpoint{2.412027in}{3.691541in}}%
\pgfpathlineto{\pgfqpoint{2.598253in}{3.691541in}}%
\pgfpathlineto{\pgfqpoint{2.598253in}{3.773270in}}%
\pgfpathlineto{\pgfqpoint{2.412027in}{3.773270in}}%
\pgfpathlineto{\pgfqpoint{2.412027in}{3.691541in}}%
\pgfusepath{}%
\end{pgfscope}%
\begin{pgfscope}%
\pgfpathrectangle{\pgfqpoint{0.549740in}{0.463273in}}{\pgfqpoint{9.320225in}{4.495057in}}%
\pgfusepath{clip}%
\pgfsetbuttcap%
\pgfsetroundjoin%
\pgfsetlinewidth{0.000000pt}%
\definecolor{currentstroke}{rgb}{0.000000,0.000000,0.000000}%
\pgfsetstrokecolor{currentstroke}%
\pgfsetdash{}{0pt}%
\pgfpathmoveto{\pgfqpoint{2.598253in}{3.691541in}}%
\pgfpathlineto{\pgfqpoint{2.784480in}{3.691541in}}%
\pgfpathlineto{\pgfqpoint{2.784480in}{3.773270in}}%
\pgfpathlineto{\pgfqpoint{2.598253in}{3.773270in}}%
\pgfpathlineto{\pgfqpoint{2.598253in}{3.691541in}}%
\pgfusepath{}%
\end{pgfscope}%
\begin{pgfscope}%
\pgfpathrectangle{\pgfqpoint{0.549740in}{0.463273in}}{\pgfqpoint{9.320225in}{4.495057in}}%
\pgfusepath{clip}%
\pgfsetbuttcap%
\pgfsetroundjoin%
\pgfsetlinewidth{0.000000pt}%
\definecolor{currentstroke}{rgb}{0.000000,0.000000,0.000000}%
\pgfsetstrokecolor{currentstroke}%
\pgfsetdash{}{0pt}%
\pgfpathmoveto{\pgfqpoint{2.784480in}{3.691541in}}%
\pgfpathlineto{\pgfqpoint{2.970706in}{3.691541in}}%
\pgfpathlineto{\pgfqpoint{2.970706in}{3.773270in}}%
\pgfpathlineto{\pgfqpoint{2.784480in}{3.773270in}}%
\pgfpathlineto{\pgfqpoint{2.784480in}{3.691541in}}%
\pgfusepath{}%
\end{pgfscope}%
\begin{pgfscope}%
\pgfpathrectangle{\pgfqpoint{0.549740in}{0.463273in}}{\pgfqpoint{9.320225in}{4.495057in}}%
\pgfusepath{clip}%
\pgfsetbuttcap%
\pgfsetroundjoin%
\pgfsetlinewidth{0.000000pt}%
\definecolor{currentstroke}{rgb}{0.000000,0.000000,0.000000}%
\pgfsetstrokecolor{currentstroke}%
\pgfsetdash{}{0pt}%
\pgfpathmoveto{\pgfqpoint{2.970706in}{3.691541in}}%
\pgfpathlineto{\pgfqpoint{3.156933in}{3.691541in}}%
\pgfpathlineto{\pgfqpoint{3.156933in}{3.773270in}}%
\pgfpathlineto{\pgfqpoint{2.970706in}{3.773270in}}%
\pgfpathlineto{\pgfqpoint{2.970706in}{3.691541in}}%
\pgfusepath{}%
\end{pgfscope}%
\begin{pgfscope}%
\pgfpathrectangle{\pgfqpoint{0.549740in}{0.463273in}}{\pgfqpoint{9.320225in}{4.495057in}}%
\pgfusepath{clip}%
\pgfsetbuttcap%
\pgfsetroundjoin%
\pgfsetlinewidth{0.000000pt}%
\definecolor{currentstroke}{rgb}{0.000000,0.000000,0.000000}%
\pgfsetstrokecolor{currentstroke}%
\pgfsetdash{}{0pt}%
\pgfpathmoveto{\pgfqpoint{3.156933in}{3.691541in}}%
\pgfpathlineto{\pgfqpoint{3.343159in}{3.691541in}}%
\pgfpathlineto{\pgfqpoint{3.343159in}{3.773270in}}%
\pgfpathlineto{\pgfqpoint{3.156933in}{3.773270in}}%
\pgfpathlineto{\pgfqpoint{3.156933in}{3.691541in}}%
\pgfusepath{}%
\end{pgfscope}%
\begin{pgfscope}%
\pgfpathrectangle{\pgfqpoint{0.549740in}{0.463273in}}{\pgfqpoint{9.320225in}{4.495057in}}%
\pgfusepath{clip}%
\pgfsetbuttcap%
\pgfsetroundjoin%
\pgfsetlinewidth{0.000000pt}%
\definecolor{currentstroke}{rgb}{0.000000,0.000000,0.000000}%
\pgfsetstrokecolor{currentstroke}%
\pgfsetdash{}{0pt}%
\pgfpathmoveto{\pgfqpoint{3.343159in}{3.691541in}}%
\pgfpathlineto{\pgfqpoint{3.529386in}{3.691541in}}%
\pgfpathlineto{\pgfqpoint{3.529386in}{3.773270in}}%
\pgfpathlineto{\pgfqpoint{3.343159in}{3.773270in}}%
\pgfpathlineto{\pgfqpoint{3.343159in}{3.691541in}}%
\pgfusepath{}%
\end{pgfscope}%
\begin{pgfscope}%
\pgfpathrectangle{\pgfqpoint{0.549740in}{0.463273in}}{\pgfqpoint{9.320225in}{4.495057in}}%
\pgfusepath{clip}%
\pgfsetbuttcap%
\pgfsetroundjoin%
\pgfsetlinewidth{0.000000pt}%
\definecolor{currentstroke}{rgb}{0.000000,0.000000,0.000000}%
\pgfsetstrokecolor{currentstroke}%
\pgfsetdash{}{0pt}%
\pgfpathmoveto{\pgfqpoint{3.529386in}{3.691541in}}%
\pgfpathlineto{\pgfqpoint{3.715612in}{3.691541in}}%
\pgfpathlineto{\pgfqpoint{3.715612in}{3.773270in}}%
\pgfpathlineto{\pgfqpoint{3.529386in}{3.773270in}}%
\pgfpathlineto{\pgfqpoint{3.529386in}{3.691541in}}%
\pgfusepath{}%
\end{pgfscope}%
\begin{pgfscope}%
\pgfpathrectangle{\pgfqpoint{0.549740in}{0.463273in}}{\pgfqpoint{9.320225in}{4.495057in}}%
\pgfusepath{clip}%
\pgfsetbuttcap%
\pgfsetroundjoin%
\pgfsetlinewidth{0.000000pt}%
\definecolor{currentstroke}{rgb}{0.000000,0.000000,0.000000}%
\pgfsetstrokecolor{currentstroke}%
\pgfsetdash{}{0pt}%
\pgfpathmoveto{\pgfqpoint{3.715612in}{3.691541in}}%
\pgfpathlineto{\pgfqpoint{3.901839in}{3.691541in}}%
\pgfpathlineto{\pgfqpoint{3.901839in}{3.773270in}}%
\pgfpathlineto{\pgfqpoint{3.715612in}{3.773270in}}%
\pgfpathlineto{\pgfqpoint{3.715612in}{3.691541in}}%
\pgfusepath{}%
\end{pgfscope}%
\begin{pgfscope}%
\pgfpathrectangle{\pgfqpoint{0.549740in}{0.463273in}}{\pgfqpoint{9.320225in}{4.495057in}}%
\pgfusepath{clip}%
\pgfsetbuttcap%
\pgfsetroundjoin%
\pgfsetlinewidth{0.000000pt}%
\definecolor{currentstroke}{rgb}{0.000000,0.000000,0.000000}%
\pgfsetstrokecolor{currentstroke}%
\pgfsetdash{}{0pt}%
\pgfpathmoveto{\pgfqpoint{3.901839in}{3.691541in}}%
\pgfpathlineto{\pgfqpoint{4.088065in}{3.691541in}}%
\pgfpathlineto{\pgfqpoint{4.088065in}{3.773270in}}%
\pgfpathlineto{\pgfqpoint{3.901839in}{3.773270in}}%
\pgfpathlineto{\pgfqpoint{3.901839in}{3.691541in}}%
\pgfusepath{}%
\end{pgfscope}%
\begin{pgfscope}%
\pgfpathrectangle{\pgfqpoint{0.549740in}{0.463273in}}{\pgfqpoint{9.320225in}{4.495057in}}%
\pgfusepath{clip}%
\pgfsetbuttcap%
\pgfsetroundjoin%
\pgfsetlinewidth{0.000000pt}%
\definecolor{currentstroke}{rgb}{0.000000,0.000000,0.000000}%
\pgfsetstrokecolor{currentstroke}%
\pgfsetdash{}{0pt}%
\pgfpathmoveto{\pgfqpoint{4.088065in}{3.691541in}}%
\pgfpathlineto{\pgfqpoint{4.274292in}{3.691541in}}%
\pgfpathlineto{\pgfqpoint{4.274292in}{3.773270in}}%
\pgfpathlineto{\pgfqpoint{4.088065in}{3.773270in}}%
\pgfpathlineto{\pgfqpoint{4.088065in}{3.691541in}}%
\pgfusepath{}%
\end{pgfscope}%
\begin{pgfscope}%
\pgfpathrectangle{\pgfqpoint{0.549740in}{0.463273in}}{\pgfqpoint{9.320225in}{4.495057in}}%
\pgfusepath{clip}%
\pgfsetbuttcap%
\pgfsetroundjoin%
\pgfsetlinewidth{0.000000pt}%
\definecolor{currentstroke}{rgb}{0.000000,0.000000,0.000000}%
\pgfsetstrokecolor{currentstroke}%
\pgfsetdash{}{0pt}%
\pgfpathmoveto{\pgfqpoint{4.274292in}{3.691541in}}%
\pgfpathlineto{\pgfqpoint{4.460519in}{3.691541in}}%
\pgfpathlineto{\pgfqpoint{4.460519in}{3.773270in}}%
\pgfpathlineto{\pgfqpoint{4.274292in}{3.773270in}}%
\pgfpathlineto{\pgfqpoint{4.274292in}{3.691541in}}%
\pgfusepath{}%
\end{pgfscope}%
\begin{pgfscope}%
\pgfpathrectangle{\pgfqpoint{0.549740in}{0.463273in}}{\pgfqpoint{9.320225in}{4.495057in}}%
\pgfusepath{clip}%
\pgfsetbuttcap%
\pgfsetroundjoin%
\pgfsetlinewidth{0.000000pt}%
\definecolor{currentstroke}{rgb}{0.000000,0.000000,0.000000}%
\pgfsetstrokecolor{currentstroke}%
\pgfsetdash{}{0pt}%
\pgfpathmoveto{\pgfqpoint{4.460519in}{3.691541in}}%
\pgfpathlineto{\pgfqpoint{4.646745in}{3.691541in}}%
\pgfpathlineto{\pgfqpoint{4.646745in}{3.773270in}}%
\pgfpathlineto{\pgfqpoint{4.460519in}{3.773270in}}%
\pgfpathlineto{\pgfqpoint{4.460519in}{3.691541in}}%
\pgfusepath{}%
\end{pgfscope}%
\begin{pgfscope}%
\pgfpathrectangle{\pgfqpoint{0.549740in}{0.463273in}}{\pgfqpoint{9.320225in}{4.495057in}}%
\pgfusepath{clip}%
\pgfsetbuttcap%
\pgfsetroundjoin%
\pgfsetlinewidth{0.000000pt}%
\definecolor{currentstroke}{rgb}{0.000000,0.000000,0.000000}%
\pgfsetstrokecolor{currentstroke}%
\pgfsetdash{}{0pt}%
\pgfpathmoveto{\pgfqpoint{4.646745in}{3.691541in}}%
\pgfpathlineto{\pgfqpoint{4.832972in}{3.691541in}}%
\pgfpathlineto{\pgfqpoint{4.832972in}{3.773270in}}%
\pgfpathlineto{\pgfqpoint{4.646745in}{3.773270in}}%
\pgfpathlineto{\pgfqpoint{4.646745in}{3.691541in}}%
\pgfusepath{}%
\end{pgfscope}%
\begin{pgfscope}%
\pgfpathrectangle{\pgfqpoint{0.549740in}{0.463273in}}{\pgfqpoint{9.320225in}{4.495057in}}%
\pgfusepath{clip}%
\pgfsetbuttcap%
\pgfsetroundjoin%
\pgfsetlinewidth{0.000000pt}%
\definecolor{currentstroke}{rgb}{0.000000,0.000000,0.000000}%
\pgfsetstrokecolor{currentstroke}%
\pgfsetdash{}{0pt}%
\pgfpathmoveto{\pgfqpoint{4.832972in}{3.691541in}}%
\pgfpathlineto{\pgfqpoint{5.019198in}{3.691541in}}%
\pgfpathlineto{\pgfqpoint{5.019198in}{3.773270in}}%
\pgfpathlineto{\pgfqpoint{4.832972in}{3.773270in}}%
\pgfpathlineto{\pgfqpoint{4.832972in}{3.691541in}}%
\pgfusepath{}%
\end{pgfscope}%
\begin{pgfscope}%
\pgfpathrectangle{\pgfqpoint{0.549740in}{0.463273in}}{\pgfqpoint{9.320225in}{4.495057in}}%
\pgfusepath{clip}%
\pgfsetbuttcap%
\pgfsetroundjoin%
\pgfsetlinewidth{0.000000pt}%
\definecolor{currentstroke}{rgb}{0.000000,0.000000,0.000000}%
\pgfsetstrokecolor{currentstroke}%
\pgfsetdash{}{0pt}%
\pgfpathmoveto{\pgfqpoint{5.019198in}{3.691541in}}%
\pgfpathlineto{\pgfqpoint{5.205425in}{3.691541in}}%
\pgfpathlineto{\pgfqpoint{5.205425in}{3.773270in}}%
\pgfpathlineto{\pgfqpoint{5.019198in}{3.773270in}}%
\pgfpathlineto{\pgfqpoint{5.019198in}{3.691541in}}%
\pgfusepath{}%
\end{pgfscope}%
\begin{pgfscope}%
\pgfpathrectangle{\pgfqpoint{0.549740in}{0.463273in}}{\pgfqpoint{9.320225in}{4.495057in}}%
\pgfusepath{clip}%
\pgfsetbuttcap%
\pgfsetroundjoin%
\pgfsetlinewidth{0.000000pt}%
\definecolor{currentstroke}{rgb}{0.000000,0.000000,0.000000}%
\pgfsetstrokecolor{currentstroke}%
\pgfsetdash{}{0pt}%
\pgfpathmoveto{\pgfqpoint{5.205425in}{3.691541in}}%
\pgfpathlineto{\pgfqpoint{5.391651in}{3.691541in}}%
\pgfpathlineto{\pgfqpoint{5.391651in}{3.773270in}}%
\pgfpathlineto{\pgfqpoint{5.205425in}{3.773270in}}%
\pgfpathlineto{\pgfqpoint{5.205425in}{3.691541in}}%
\pgfusepath{}%
\end{pgfscope}%
\begin{pgfscope}%
\pgfpathrectangle{\pgfqpoint{0.549740in}{0.463273in}}{\pgfqpoint{9.320225in}{4.495057in}}%
\pgfusepath{clip}%
\pgfsetbuttcap%
\pgfsetroundjoin%
\pgfsetlinewidth{0.000000pt}%
\definecolor{currentstroke}{rgb}{0.000000,0.000000,0.000000}%
\pgfsetstrokecolor{currentstroke}%
\pgfsetdash{}{0pt}%
\pgfpathmoveto{\pgfqpoint{5.391651in}{3.691541in}}%
\pgfpathlineto{\pgfqpoint{5.577878in}{3.691541in}}%
\pgfpathlineto{\pgfqpoint{5.577878in}{3.773270in}}%
\pgfpathlineto{\pgfqpoint{5.391651in}{3.773270in}}%
\pgfpathlineto{\pgfqpoint{5.391651in}{3.691541in}}%
\pgfusepath{}%
\end{pgfscope}%
\begin{pgfscope}%
\pgfpathrectangle{\pgfqpoint{0.549740in}{0.463273in}}{\pgfqpoint{9.320225in}{4.495057in}}%
\pgfusepath{clip}%
\pgfsetbuttcap%
\pgfsetroundjoin%
\pgfsetlinewidth{0.000000pt}%
\definecolor{currentstroke}{rgb}{0.000000,0.000000,0.000000}%
\pgfsetstrokecolor{currentstroke}%
\pgfsetdash{}{0pt}%
\pgfpathmoveto{\pgfqpoint{5.577878in}{3.691541in}}%
\pgfpathlineto{\pgfqpoint{5.764104in}{3.691541in}}%
\pgfpathlineto{\pgfqpoint{5.764104in}{3.773270in}}%
\pgfpathlineto{\pgfqpoint{5.577878in}{3.773270in}}%
\pgfpathlineto{\pgfqpoint{5.577878in}{3.691541in}}%
\pgfusepath{}%
\end{pgfscope}%
\begin{pgfscope}%
\pgfpathrectangle{\pgfqpoint{0.549740in}{0.463273in}}{\pgfqpoint{9.320225in}{4.495057in}}%
\pgfusepath{clip}%
\pgfsetbuttcap%
\pgfsetroundjoin%
\pgfsetlinewidth{0.000000pt}%
\definecolor{currentstroke}{rgb}{0.000000,0.000000,0.000000}%
\pgfsetstrokecolor{currentstroke}%
\pgfsetdash{}{0pt}%
\pgfpathmoveto{\pgfqpoint{5.764104in}{3.691541in}}%
\pgfpathlineto{\pgfqpoint{5.950331in}{3.691541in}}%
\pgfpathlineto{\pgfqpoint{5.950331in}{3.773270in}}%
\pgfpathlineto{\pgfqpoint{5.764104in}{3.773270in}}%
\pgfpathlineto{\pgfqpoint{5.764104in}{3.691541in}}%
\pgfusepath{}%
\end{pgfscope}%
\begin{pgfscope}%
\pgfpathrectangle{\pgfqpoint{0.549740in}{0.463273in}}{\pgfqpoint{9.320225in}{4.495057in}}%
\pgfusepath{clip}%
\pgfsetbuttcap%
\pgfsetroundjoin%
\pgfsetlinewidth{0.000000pt}%
\definecolor{currentstroke}{rgb}{0.000000,0.000000,0.000000}%
\pgfsetstrokecolor{currentstroke}%
\pgfsetdash{}{0pt}%
\pgfpathmoveto{\pgfqpoint{5.950331in}{3.691541in}}%
\pgfpathlineto{\pgfqpoint{6.136557in}{3.691541in}}%
\pgfpathlineto{\pgfqpoint{6.136557in}{3.773270in}}%
\pgfpathlineto{\pgfqpoint{5.950331in}{3.773270in}}%
\pgfpathlineto{\pgfqpoint{5.950331in}{3.691541in}}%
\pgfusepath{}%
\end{pgfscope}%
\begin{pgfscope}%
\pgfpathrectangle{\pgfqpoint{0.549740in}{0.463273in}}{\pgfqpoint{9.320225in}{4.495057in}}%
\pgfusepath{clip}%
\pgfsetbuttcap%
\pgfsetroundjoin%
\pgfsetlinewidth{0.000000pt}%
\definecolor{currentstroke}{rgb}{0.000000,0.000000,0.000000}%
\pgfsetstrokecolor{currentstroke}%
\pgfsetdash{}{0pt}%
\pgfpathmoveto{\pgfqpoint{6.136557in}{3.691541in}}%
\pgfpathlineto{\pgfqpoint{6.322784in}{3.691541in}}%
\pgfpathlineto{\pgfqpoint{6.322784in}{3.773270in}}%
\pgfpathlineto{\pgfqpoint{6.136557in}{3.773270in}}%
\pgfpathlineto{\pgfqpoint{6.136557in}{3.691541in}}%
\pgfusepath{}%
\end{pgfscope}%
\begin{pgfscope}%
\pgfpathrectangle{\pgfqpoint{0.549740in}{0.463273in}}{\pgfqpoint{9.320225in}{4.495057in}}%
\pgfusepath{clip}%
\pgfsetbuttcap%
\pgfsetroundjoin%
\pgfsetlinewidth{0.000000pt}%
\definecolor{currentstroke}{rgb}{0.000000,0.000000,0.000000}%
\pgfsetstrokecolor{currentstroke}%
\pgfsetdash{}{0pt}%
\pgfpathmoveto{\pgfqpoint{6.322784in}{3.691541in}}%
\pgfpathlineto{\pgfqpoint{6.509011in}{3.691541in}}%
\pgfpathlineto{\pgfqpoint{6.509011in}{3.773270in}}%
\pgfpathlineto{\pgfqpoint{6.322784in}{3.773270in}}%
\pgfpathlineto{\pgfqpoint{6.322784in}{3.691541in}}%
\pgfusepath{}%
\end{pgfscope}%
\begin{pgfscope}%
\pgfpathrectangle{\pgfqpoint{0.549740in}{0.463273in}}{\pgfqpoint{9.320225in}{4.495057in}}%
\pgfusepath{clip}%
\pgfsetbuttcap%
\pgfsetroundjoin%
\pgfsetlinewidth{0.000000pt}%
\definecolor{currentstroke}{rgb}{0.000000,0.000000,0.000000}%
\pgfsetstrokecolor{currentstroke}%
\pgfsetdash{}{0pt}%
\pgfpathmoveto{\pgfqpoint{6.509011in}{3.691541in}}%
\pgfpathlineto{\pgfqpoint{6.695237in}{3.691541in}}%
\pgfpathlineto{\pgfqpoint{6.695237in}{3.773270in}}%
\pgfpathlineto{\pgfqpoint{6.509011in}{3.773270in}}%
\pgfpathlineto{\pgfqpoint{6.509011in}{3.691541in}}%
\pgfusepath{}%
\end{pgfscope}%
\begin{pgfscope}%
\pgfpathrectangle{\pgfqpoint{0.549740in}{0.463273in}}{\pgfqpoint{9.320225in}{4.495057in}}%
\pgfusepath{clip}%
\pgfsetbuttcap%
\pgfsetroundjoin%
\pgfsetlinewidth{0.000000pt}%
\definecolor{currentstroke}{rgb}{0.000000,0.000000,0.000000}%
\pgfsetstrokecolor{currentstroke}%
\pgfsetdash{}{0pt}%
\pgfpathmoveto{\pgfqpoint{6.695237in}{3.691541in}}%
\pgfpathlineto{\pgfqpoint{6.881464in}{3.691541in}}%
\pgfpathlineto{\pgfqpoint{6.881464in}{3.773270in}}%
\pgfpathlineto{\pgfqpoint{6.695237in}{3.773270in}}%
\pgfpathlineto{\pgfqpoint{6.695237in}{3.691541in}}%
\pgfusepath{}%
\end{pgfscope}%
\begin{pgfscope}%
\pgfpathrectangle{\pgfqpoint{0.549740in}{0.463273in}}{\pgfqpoint{9.320225in}{4.495057in}}%
\pgfusepath{clip}%
\pgfsetbuttcap%
\pgfsetroundjoin%
\definecolor{currentfill}{rgb}{0.472869,0.711325,0.955316}%
\pgfsetfillcolor{currentfill}%
\pgfsetlinewidth{0.000000pt}%
\definecolor{currentstroke}{rgb}{0.000000,0.000000,0.000000}%
\pgfsetstrokecolor{currentstroke}%
\pgfsetdash{}{0pt}%
\pgfpathmoveto{\pgfqpoint{6.881464in}{3.691541in}}%
\pgfpathlineto{\pgfqpoint{7.067690in}{3.691541in}}%
\pgfpathlineto{\pgfqpoint{7.067690in}{3.773270in}}%
\pgfpathlineto{\pgfqpoint{6.881464in}{3.773270in}}%
\pgfpathlineto{\pgfqpoint{6.881464in}{3.691541in}}%
\pgfusepath{fill}%
\end{pgfscope}%
\begin{pgfscope}%
\pgfpathrectangle{\pgfqpoint{0.549740in}{0.463273in}}{\pgfqpoint{9.320225in}{4.495057in}}%
\pgfusepath{clip}%
\pgfsetbuttcap%
\pgfsetroundjoin%
\pgfsetlinewidth{0.000000pt}%
\definecolor{currentstroke}{rgb}{0.000000,0.000000,0.000000}%
\pgfsetstrokecolor{currentstroke}%
\pgfsetdash{}{0pt}%
\pgfpathmoveto{\pgfqpoint{7.067690in}{3.691541in}}%
\pgfpathlineto{\pgfqpoint{7.253917in}{3.691541in}}%
\pgfpathlineto{\pgfqpoint{7.253917in}{3.773270in}}%
\pgfpathlineto{\pgfqpoint{7.067690in}{3.773270in}}%
\pgfpathlineto{\pgfqpoint{7.067690in}{3.691541in}}%
\pgfusepath{}%
\end{pgfscope}%
\begin{pgfscope}%
\pgfpathrectangle{\pgfqpoint{0.549740in}{0.463273in}}{\pgfqpoint{9.320225in}{4.495057in}}%
\pgfusepath{clip}%
\pgfsetbuttcap%
\pgfsetroundjoin%
\pgfsetlinewidth{0.000000pt}%
\definecolor{currentstroke}{rgb}{0.000000,0.000000,0.000000}%
\pgfsetstrokecolor{currentstroke}%
\pgfsetdash{}{0pt}%
\pgfpathmoveto{\pgfqpoint{7.253917in}{3.691541in}}%
\pgfpathlineto{\pgfqpoint{7.440143in}{3.691541in}}%
\pgfpathlineto{\pgfqpoint{7.440143in}{3.773270in}}%
\pgfpathlineto{\pgfqpoint{7.253917in}{3.773270in}}%
\pgfpathlineto{\pgfqpoint{7.253917in}{3.691541in}}%
\pgfusepath{}%
\end{pgfscope}%
\begin{pgfscope}%
\pgfpathrectangle{\pgfqpoint{0.549740in}{0.463273in}}{\pgfqpoint{9.320225in}{4.495057in}}%
\pgfusepath{clip}%
\pgfsetbuttcap%
\pgfsetroundjoin%
\pgfsetlinewidth{0.000000pt}%
\definecolor{currentstroke}{rgb}{0.000000,0.000000,0.000000}%
\pgfsetstrokecolor{currentstroke}%
\pgfsetdash{}{0pt}%
\pgfpathmoveto{\pgfqpoint{7.440143in}{3.691541in}}%
\pgfpathlineto{\pgfqpoint{7.626370in}{3.691541in}}%
\pgfpathlineto{\pgfqpoint{7.626370in}{3.773270in}}%
\pgfpathlineto{\pgfqpoint{7.440143in}{3.773270in}}%
\pgfpathlineto{\pgfqpoint{7.440143in}{3.691541in}}%
\pgfusepath{}%
\end{pgfscope}%
\begin{pgfscope}%
\pgfpathrectangle{\pgfqpoint{0.549740in}{0.463273in}}{\pgfqpoint{9.320225in}{4.495057in}}%
\pgfusepath{clip}%
\pgfsetbuttcap%
\pgfsetroundjoin%
\pgfsetlinewidth{0.000000pt}%
\definecolor{currentstroke}{rgb}{0.000000,0.000000,0.000000}%
\pgfsetstrokecolor{currentstroke}%
\pgfsetdash{}{0pt}%
\pgfpathmoveto{\pgfqpoint{7.626370in}{3.691541in}}%
\pgfpathlineto{\pgfqpoint{7.812596in}{3.691541in}}%
\pgfpathlineto{\pgfqpoint{7.812596in}{3.773270in}}%
\pgfpathlineto{\pgfqpoint{7.626370in}{3.773270in}}%
\pgfpathlineto{\pgfqpoint{7.626370in}{3.691541in}}%
\pgfusepath{}%
\end{pgfscope}%
\begin{pgfscope}%
\pgfpathrectangle{\pgfqpoint{0.549740in}{0.463273in}}{\pgfqpoint{9.320225in}{4.495057in}}%
\pgfusepath{clip}%
\pgfsetbuttcap%
\pgfsetroundjoin%
\pgfsetlinewidth{0.000000pt}%
\definecolor{currentstroke}{rgb}{0.000000,0.000000,0.000000}%
\pgfsetstrokecolor{currentstroke}%
\pgfsetdash{}{0pt}%
\pgfpathmoveto{\pgfqpoint{7.812596in}{3.691541in}}%
\pgfpathlineto{\pgfqpoint{7.998823in}{3.691541in}}%
\pgfpathlineto{\pgfqpoint{7.998823in}{3.773270in}}%
\pgfpathlineto{\pgfqpoint{7.812596in}{3.773270in}}%
\pgfpathlineto{\pgfqpoint{7.812596in}{3.691541in}}%
\pgfusepath{}%
\end{pgfscope}%
\begin{pgfscope}%
\pgfpathrectangle{\pgfqpoint{0.549740in}{0.463273in}}{\pgfqpoint{9.320225in}{4.495057in}}%
\pgfusepath{clip}%
\pgfsetbuttcap%
\pgfsetroundjoin%
\definecolor{currentfill}{rgb}{0.472869,0.711325,0.955316}%
\pgfsetfillcolor{currentfill}%
\pgfsetlinewidth{0.000000pt}%
\definecolor{currentstroke}{rgb}{0.000000,0.000000,0.000000}%
\pgfsetstrokecolor{currentstroke}%
\pgfsetdash{}{0pt}%
\pgfpathmoveto{\pgfqpoint{7.998823in}{3.691541in}}%
\pgfpathlineto{\pgfqpoint{8.185049in}{3.691541in}}%
\pgfpathlineto{\pgfqpoint{8.185049in}{3.773270in}}%
\pgfpathlineto{\pgfqpoint{7.998823in}{3.773270in}}%
\pgfpathlineto{\pgfqpoint{7.998823in}{3.691541in}}%
\pgfusepath{fill}%
\end{pgfscope}%
\begin{pgfscope}%
\pgfpathrectangle{\pgfqpoint{0.549740in}{0.463273in}}{\pgfqpoint{9.320225in}{4.495057in}}%
\pgfusepath{clip}%
\pgfsetbuttcap%
\pgfsetroundjoin%
\pgfsetlinewidth{0.000000pt}%
\definecolor{currentstroke}{rgb}{0.000000,0.000000,0.000000}%
\pgfsetstrokecolor{currentstroke}%
\pgfsetdash{}{0pt}%
\pgfpathmoveto{\pgfqpoint{8.185049in}{3.691541in}}%
\pgfpathlineto{\pgfqpoint{8.371276in}{3.691541in}}%
\pgfpathlineto{\pgfqpoint{8.371276in}{3.773270in}}%
\pgfpathlineto{\pgfqpoint{8.185049in}{3.773270in}}%
\pgfpathlineto{\pgfqpoint{8.185049in}{3.691541in}}%
\pgfusepath{}%
\end{pgfscope}%
\begin{pgfscope}%
\pgfpathrectangle{\pgfqpoint{0.549740in}{0.463273in}}{\pgfqpoint{9.320225in}{4.495057in}}%
\pgfusepath{clip}%
\pgfsetbuttcap%
\pgfsetroundjoin%
\pgfsetlinewidth{0.000000pt}%
\definecolor{currentstroke}{rgb}{0.000000,0.000000,0.000000}%
\pgfsetstrokecolor{currentstroke}%
\pgfsetdash{}{0pt}%
\pgfpathmoveto{\pgfqpoint{8.371276in}{3.691541in}}%
\pgfpathlineto{\pgfqpoint{8.557503in}{3.691541in}}%
\pgfpathlineto{\pgfqpoint{8.557503in}{3.773270in}}%
\pgfpathlineto{\pgfqpoint{8.371276in}{3.773270in}}%
\pgfpathlineto{\pgfqpoint{8.371276in}{3.691541in}}%
\pgfusepath{}%
\end{pgfscope}%
\begin{pgfscope}%
\pgfpathrectangle{\pgfqpoint{0.549740in}{0.463273in}}{\pgfqpoint{9.320225in}{4.495057in}}%
\pgfusepath{clip}%
\pgfsetbuttcap%
\pgfsetroundjoin%
\pgfsetlinewidth{0.000000pt}%
\definecolor{currentstroke}{rgb}{0.000000,0.000000,0.000000}%
\pgfsetstrokecolor{currentstroke}%
\pgfsetdash{}{0pt}%
\pgfpathmoveto{\pgfqpoint{8.557503in}{3.691541in}}%
\pgfpathlineto{\pgfqpoint{8.743729in}{3.691541in}}%
\pgfpathlineto{\pgfqpoint{8.743729in}{3.773270in}}%
\pgfpathlineto{\pgfqpoint{8.557503in}{3.773270in}}%
\pgfpathlineto{\pgfqpoint{8.557503in}{3.691541in}}%
\pgfusepath{}%
\end{pgfscope}%
\begin{pgfscope}%
\pgfpathrectangle{\pgfqpoint{0.549740in}{0.463273in}}{\pgfqpoint{9.320225in}{4.495057in}}%
\pgfusepath{clip}%
\pgfsetbuttcap%
\pgfsetroundjoin%
\pgfsetlinewidth{0.000000pt}%
\definecolor{currentstroke}{rgb}{0.000000,0.000000,0.000000}%
\pgfsetstrokecolor{currentstroke}%
\pgfsetdash{}{0pt}%
\pgfpathmoveto{\pgfqpoint{8.743729in}{3.691541in}}%
\pgfpathlineto{\pgfqpoint{8.929956in}{3.691541in}}%
\pgfpathlineto{\pgfqpoint{8.929956in}{3.773270in}}%
\pgfpathlineto{\pgfqpoint{8.743729in}{3.773270in}}%
\pgfpathlineto{\pgfqpoint{8.743729in}{3.691541in}}%
\pgfusepath{}%
\end{pgfscope}%
\begin{pgfscope}%
\pgfpathrectangle{\pgfqpoint{0.549740in}{0.463273in}}{\pgfqpoint{9.320225in}{4.495057in}}%
\pgfusepath{clip}%
\pgfsetbuttcap%
\pgfsetroundjoin%
\pgfsetlinewidth{0.000000pt}%
\definecolor{currentstroke}{rgb}{0.000000,0.000000,0.000000}%
\pgfsetstrokecolor{currentstroke}%
\pgfsetdash{}{0pt}%
\pgfpathmoveto{\pgfqpoint{8.929956in}{3.691541in}}%
\pgfpathlineto{\pgfqpoint{9.116182in}{3.691541in}}%
\pgfpathlineto{\pgfqpoint{9.116182in}{3.773270in}}%
\pgfpathlineto{\pgfqpoint{8.929956in}{3.773270in}}%
\pgfpathlineto{\pgfqpoint{8.929956in}{3.691541in}}%
\pgfusepath{}%
\end{pgfscope}%
\begin{pgfscope}%
\pgfpathrectangle{\pgfqpoint{0.549740in}{0.463273in}}{\pgfqpoint{9.320225in}{4.495057in}}%
\pgfusepath{clip}%
\pgfsetbuttcap%
\pgfsetroundjoin%
\definecolor{currentfill}{rgb}{0.547810,0.736432,0.947518}%
\pgfsetfillcolor{currentfill}%
\pgfsetlinewidth{0.000000pt}%
\definecolor{currentstroke}{rgb}{0.000000,0.000000,0.000000}%
\pgfsetstrokecolor{currentstroke}%
\pgfsetdash{}{0pt}%
\pgfpathmoveto{\pgfqpoint{9.116182in}{3.691541in}}%
\pgfpathlineto{\pgfqpoint{9.302409in}{3.691541in}}%
\pgfpathlineto{\pgfqpoint{9.302409in}{3.773270in}}%
\pgfpathlineto{\pgfqpoint{9.116182in}{3.773270in}}%
\pgfpathlineto{\pgfqpoint{9.116182in}{3.691541in}}%
\pgfusepath{fill}%
\end{pgfscope}%
\begin{pgfscope}%
\pgfpathrectangle{\pgfqpoint{0.549740in}{0.463273in}}{\pgfqpoint{9.320225in}{4.495057in}}%
\pgfusepath{clip}%
\pgfsetbuttcap%
\pgfsetroundjoin%
\definecolor{currentfill}{rgb}{0.614330,0.761948,0.940009}%
\pgfsetfillcolor{currentfill}%
\pgfsetlinewidth{0.000000pt}%
\definecolor{currentstroke}{rgb}{0.000000,0.000000,0.000000}%
\pgfsetstrokecolor{currentstroke}%
\pgfsetdash{}{0pt}%
\pgfpathmoveto{\pgfqpoint{9.302409in}{3.691541in}}%
\pgfpathlineto{\pgfqpoint{9.488635in}{3.691541in}}%
\pgfpathlineto{\pgfqpoint{9.488635in}{3.773270in}}%
\pgfpathlineto{\pgfqpoint{9.302409in}{3.773270in}}%
\pgfpathlineto{\pgfqpoint{9.302409in}{3.691541in}}%
\pgfusepath{fill}%
\end{pgfscope}%
\begin{pgfscope}%
\pgfpathrectangle{\pgfqpoint{0.549740in}{0.463273in}}{\pgfqpoint{9.320225in}{4.495057in}}%
\pgfusepath{clip}%
\pgfsetbuttcap%
\pgfsetroundjoin%
\pgfsetlinewidth{0.000000pt}%
\definecolor{currentstroke}{rgb}{0.000000,0.000000,0.000000}%
\pgfsetstrokecolor{currentstroke}%
\pgfsetdash{}{0pt}%
\pgfpathmoveto{\pgfqpoint{9.488635in}{3.691541in}}%
\pgfpathlineto{\pgfqpoint{9.674862in}{3.691541in}}%
\pgfpathlineto{\pgfqpoint{9.674862in}{3.773270in}}%
\pgfpathlineto{\pgfqpoint{9.488635in}{3.773270in}}%
\pgfpathlineto{\pgfqpoint{9.488635in}{3.691541in}}%
\pgfusepath{}%
\end{pgfscope}%
\begin{pgfscope}%
\pgfpathrectangle{\pgfqpoint{0.549740in}{0.463273in}}{\pgfqpoint{9.320225in}{4.495057in}}%
\pgfusepath{clip}%
\pgfsetbuttcap%
\pgfsetroundjoin%
\pgfsetlinewidth{0.000000pt}%
\definecolor{currentstroke}{rgb}{0.000000,0.000000,0.000000}%
\pgfsetstrokecolor{currentstroke}%
\pgfsetdash{}{0pt}%
\pgfpathmoveto{\pgfqpoint{9.674862in}{3.691541in}}%
\pgfpathlineto{\pgfqpoint{9.861088in}{3.691541in}}%
\pgfpathlineto{\pgfqpoint{9.861088in}{3.773270in}}%
\pgfpathlineto{\pgfqpoint{9.674862in}{3.773270in}}%
\pgfpathlineto{\pgfqpoint{9.674862in}{3.691541in}}%
\pgfusepath{}%
\end{pgfscope}%
\begin{pgfscope}%
\pgfpathrectangle{\pgfqpoint{0.549740in}{0.463273in}}{\pgfqpoint{9.320225in}{4.495057in}}%
\pgfusepath{clip}%
\pgfsetbuttcap%
\pgfsetroundjoin%
\pgfsetlinewidth{0.000000pt}%
\definecolor{currentstroke}{rgb}{0.000000,0.000000,0.000000}%
\pgfsetstrokecolor{currentstroke}%
\pgfsetdash{}{0pt}%
\pgfpathmoveto{\pgfqpoint{0.549761in}{3.773270in}}%
\pgfpathlineto{\pgfqpoint{0.735988in}{3.773270in}}%
\pgfpathlineto{\pgfqpoint{0.735988in}{3.854998in}}%
\pgfpathlineto{\pgfqpoint{0.549761in}{3.854998in}}%
\pgfpathlineto{\pgfqpoint{0.549761in}{3.773270in}}%
\pgfusepath{}%
\end{pgfscope}%
\begin{pgfscope}%
\pgfpathrectangle{\pgfqpoint{0.549740in}{0.463273in}}{\pgfqpoint{9.320225in}{4.495057in}}%
\pgfusepath{clip}%
\pgfsetbuttcap%
\pgfsetroundjoin%
\pgfsetlinewidth{0.000000pt}%
\definecolor{currentstroke}{rgb}{0.000000,0.000000,0.000000}%
\pgfsetstrokecolor{currentstroke}%
\pgfsetdash{}{0pt}%
\pgfpathmoveto{\pgfqpoint{0.735988in}{3.773270in}}%
\pgfpathlineto{\pgfqpoint{0.922214in}{3.773270in}}%
\pgfpathlineto{\pgfqpoint{0.922214in}{3.854998in}}%
\pgfpathlineto{\pgfqpoint{0.735988in}{3.854998in}}%
\pgfpathlineto{\pgfqpoint{0.735988in}{3.773270in}}%
\pgfusepath{}%
\end{pgfscope}%
\begin{pgfscope}%
\pgfpathrectangle{\pgfqpoint{0.549740in}{0.463273in}}{\pgfqpoint{9.320225in}{4.495057in}}%
\pgfusepath{clip}%
\pgfsetbuttcap%
\pgfsetroundjoin%
\pgfsetlinewidth{0.000000pt}%
\definecolor{currentstroke}{rgb}{0.000000,0.000000,0.000000}%
\pgfsetstrokecolor{currentstroke}%
\pgfsetdash{}{0pt}%
\pgfpathmoveto{\pgfqpoint{0.922214in}{3.773270in}}%
\pgfpathlineto{\pgfqpoint{1.108441in}{3.773270in}}%
\pgfpathlineto{\pgfqpoint{1.108441in}{3.854998in}}%
\pgfpathlineto{\pgfqpoint{0.922214in}{3.854998in}}%
\pgfpathlineto{\pgfqpoint{0.922214in}{3.773270in}}%
\pgfusepath{}%
\end{pgfscope}%
\begin{pgfscope}%
\pgfpathrectangle{\pgfqpoint{0.549740in}{0.463273in}}{\pgfqpoint{9.320225in}{4.495057in}}%
\pgfusepath{clip}%
\pgfsetbuttcap%
\pgfsetroundjoin%
\pgfsetlinewidth{0.000000pt}%
\definecolor{currentstroke}{rgb}{0.000000,0.000000,0.000000}%
\pgfsetstrokecolor{currentstroke}%
\pgfsetdash{}{0pt}%
\pgfpathmoveto{\pgfqpoint{1.108441in}{3.773270in}}%
\pgfpathlineto{\pgfqpoint{1.294667in}{3.773270in}}%
\pgfpathlineto{\pgfqpoint{1.294667in}{3.854998in}}%
\pgfpathlineto{\pgfqpoint{1.108441in}{3.854998in}}%
\pgfpathlineto{\pgfqpoint{1.108441in}{3.773270in}}%
\pgfusepath{}%
\end{pgfscope}%
\begin{pgfscope}%
\pgfpathrectangle{\pgfqpoint{0.549740in}{0.463273in}}{\pgfqpoint{9.320225in}{4.495057in}}%
\pgfusepath{clip}%
\pgfsetbuttcap%
\pgfsetroundjoin%
\pgfsetlinewidth{0.000000pt}%
\definecolor{currentstroke}{rgb}{0.000000,0.000000,0.000000}%
\pgfsetstrokecolor{currentstroke}%
\pgfsetdash{}{0pt}%
\pgfpathmoveto{\pgfqpoint{1.294667in}{3.773270in}}%
\pgfpathlineto{\pgfqpoint{1.480894in}{3.773270in}}%
\pgfpathlineto{\pgfqpoint{1.480894in}{3.854998in}}%
\pgfpathlineto{\pgfqpoint{1.294667in}{3.854998in}}%
\pgfpathlineto{\pgfqpoint{1.294667in}{3.773270in}}%
\pgfusepath{}%
\end{pgfscope}%
\begin{pgfscope}%
\pgfpathrectangle{\pgfqpoint{0.549740in}{0.463273in}}{\pgfqpoint{9.320225in}{4.495057in}}%
\pgfusepath{clip}%
\pgfsetbuttcap%
\pgfsetroundjoin%
\pgfsetlinewidth{0.000000pt}%
\definecolor{currentstroke}{rgb}{0.000000,0.000000,0.000000}%
\pgfsetstrokecolor{currentstroke}%
\pgfsetdash{}{0pt}%
\pgfpathmoveto{\pgfqpoint{1.480894in}{3.773270in}}%
\pgfpathlineto{\pgfqpoint{1.667120in}{3.773270in}}%
\pgfpathlineto{\pgfqpoint{1.667120in}{3.854998in}}%
\pgfpathlineto{\pgfqpoint{1.480894in}{3.854998in}}%
\pgfpathlineto{\pgfqpoint{1.480894in}{3.773270in}}%
\pgfusepath{}%
\end{pgfscope}%
\begin{pgfscope}%
\pgfpathrectangle{\pgfqpoint{0.549740in}{0.463273in}}{\pgfqpoint{9.320225in}{4.495057in}}%
\pgfusepath{clip}%
\pgfsetbuttcap%
\pgfsetroundjoin%
\pgfsetlinewidth{0.000000pt}%
\definecolor{currentstroke}{rgb}{0.000000,0.000000,0.000000}%
\pgfsetstrokecolor{currentstroke}%
\pgfsetdash{}{0pt}%
\pgfpathmoveto{\pgfqpoint{1.667120in}{3.773270in}}%
\pgfpathlineto{\pgfqpoint{1.853347in}{3.773270in}}%
\pgfpathlineto{\pgfqpoint{1.853347in}{3.854998in}}%
\pgfpathlineto{\pgfqpoint{1.667120in}{3.854998in}}%
\pgfpathlineto{\pgfqpoint{1.667120in}{3.773270in}}%
\pgfusepath{}%
\end{pgfscope}%
\begin{pgfscope}%
\pgfpathrectangle{\pgfqpoint{0.549740in}{0.463273in}}{\pgfqpoint{9.320225in}{4.495057in}}%
\pgfusepath{clip}%
\pgfsetbuttcap%
\pgfsetroundjoin%
\pgfsetlinewidth{0.000000pt}%
\definecolor{currentstroke}{rgb}{0.000000,0.000000,0.000000}%
\pgfsetstrokecolor{currentstroke}%
\pgfsetdash{}{0pt}%
\pgfpathmoveto{\pgfqpoint{1.853347in}{3.773270in}}%
\pgfpathlineto{\pgfqpoint{2.039573in}{3.773270in}}%
\pgfpathlineto{\pgfqpoint{2.039573in}{3.854998in}}%
\pgfpathlineto{\pgfqpoint{1.853347in}{3.854998in}}%
\pgfpathlineto{\pgfqpoint{1.853347in}{3.773270in}}%
\pgfusepath{}%
\end{pgfscope}%
\begin{pgfscope}%
\pgfpathrectangle{\pgfqpoint{0.549740in}{0.463273in}}{\pgfqpoint{9.320225in}{4.495057in}}%
\pgfusepath{clip}%
\pgfsetbuttcap%
\pgfsetroundjoin%
\pgfsetlinewidth{0.000000pt}%
\definecolor{currentstroke}{rgb}{0.000000,0.000000,0.000000}%
\pgfsetstrokecolor{currentstroke}%
\pgfsetdash{}{0pt}%
\pgfpathmoveto{\pgfqpoint{2.039573in}{3.773270in}}%
\pgfpathlineto{\pgfqpoint{2.225800in}{3.773270in}}%
\pgfpathlineto{\pgfqpoint{2.225800in}{3.854998in}}%
\pgfpathlineto{\pgfqpoint{2.039573in}{3.854998in}}%
\pgfpathlineto{\pgfqpoint{2.039573in}{3.773270in}}%
\pgfusepath{}%
\end{pgfscope}%
\begin{pgfscope}%
\pgfpathrectangle{\pgfqpoint{0.549740in}{0.463273in}}{\pgfqpoint{9.320225in}{4.495057in}}%
\pgfusepath{clip}%
\pgfsetbuttcap%
\pgfsetroundjoin%
\pgfsetlinewidth{0.000000pt}%
\definecolor{currentstroke}{rgb}{0.000000,0.000000,0.000000}%
\pgfsetstrokecolor{currentstroke}%
\pgfsetdash{}{0pt}%
\pgfpathmoveto{\pgfqpoint{2.225800in}{3.773270in}}%
\pgfpathlineto{\pgfqpoint{2.412027in}{3.773270in}}%
\pgfpathlineto{\pgfqpoint{2.412027in}{3.854998in}}%
\pgfpathlineto{\pgfqpoint{2.225800in}{3.854998in}}%
\pgfpathlineto{\pgfqpoint{2.225800in}{3.773270in}}%
\pgfusepath{}%
\end{pgfscope}%
\begin{pgfscope}%
\pgfpathrectangle{\pgfqpoint{0.549740in}{0.463273in}}{\pgfqpoint{9.320225in}{4.495057in}}%
\pgfusepath{clip}%
\pgfsetbuttcap%
\pgfsetroundjoin%
\pgfsetlinewidth{0.000000pt}%
\definecolor{currentstroke}{rgb}{0.000000,0.000000,0.000000}%
\pgfsetstrokecolor{currentstroke}%
\pgfsetdash{}{0pt}%
\pgfpathmoveto{\pgfqpoint{2.412027in}{3.773270in}}%
\pgfpathlineto{\pgfqpoint{2.598253in}{3.773270in}}%
\pgfpathlineto{\pgfqpoint{2.598253in}{3.854998in}}%
\pgfpathlineto{\pgfqpoint{2.412027in}{3.854998in}}%
\pgfpathlineto{\pgfqpoint{2.412027in}{3.773270in}}%
\pgfusepath{}%
\end{pgfscope}%
\begin{pgfscope}%
\pgfpathrectangle{\pgfqpoint{0.549740in}{0.463273in}}{\pgfqpoint{9.320225in}{4.495057in}}%
\pgfusepath{clip}%
\pgfsetbuttcap%
\pgfsetroundjoin%
\pgfsetlinewidth{0.000000pt}%
\definecolor{currentstroke}{rgb}{0.000000,0.000000,0.000000}%
\pgfsetstrokecolor{currentstroke}%
\pgfsetdash{}{0pt}%
\pgfpathmoveto{\pgfqpoint{2.598253in}{3.773270in}}%
\pgfpathlineto{\pgfqpoint{2.784480in}{3.773270in}}%
\pgfpathlineto{\pgfqpoint{2.784480in}{3.854998in}}%
\pgfpathlineto{\pgfqpoint{2.598253in}{3.854998in}}%
\pgfpathlineto{\pgfqpoint{2.598253in}{3.773270in}}%
\pgfusepath{}%
\end{pgfscope}%
\begin{pgfscope}%
\pgfpathrectangle{\pgfqpoint{0.549740in}{0.463273in}}{\pgfqpoint{9.320225in}{4.495057in}}%
\pgfusepath{clip}%
\pgfsetbuttcap%
\pgfsetroundjoin%
\pgfsetlinewidth{0.000000pt}%
\definecolor{currentstroke}{rgb}{0.000000,0.000000,0.000000}%
\pgfsetstrokecolor{currentstroke}%
\pgfsetdash{}{0pt}%
\pgfpathmoveto{\pgfqpoint{2.784480in}{3.773270in}}%
\pgfpathlineto{\pgfqpoint{2.970706in}{3.773270in}}%
\pgfpathlineto{\pgfqpoint{2.970706in}{3.854998in}}%
\pgfpathlineto{\pgfqpoint{2.784480in}{3.854998in}}%
\pgfpathlineto{\pgfqpoint{2.784480in}{3.773270in}}%
\pgfusepath{}%
\end{pgfscope}%
\begin{pgfscope}%
\pgfpathrectangle{\pgfqpoint{0.549740in}{0.463273in}}{\pgfqpoint{9.320225in}{4.495057in}}%
\pgfusepath{clip}%
\pgfsetbuttcap%
\pgfsetroundjoin%
\pgfsetlinewidth{0.000000pt}%
\definecolor{currentstroke}{rgb}{0.000000,0.000000,0.000000}%
\pgfsetstrokecolor{currentstroke}%
\pgfsetdash{}{0pt}%
\pgfpathmoveto{\pgfqpoint{2.970706in}{3.773270in}}%
\pgfpathlineto{\pgfqpoint{3.156933in}{3.773270in}}%
\pgfpathlineto{\pgfqpoint{3.156933in}{3.854998in}}%
\pgfpathlineto{\pgfqpoint{2.970706in}{3.854998in}}%
\pgfpathlineto{\pgfqpoint{2.970706in}{3.773270in}}%
\pgfusepath{}%
\end{pgfscope}%
\begin{pgfscope}%
\pgfpathrectangle{\pgfqpoint{0.549740in}{0.463273in}}{\pgfqpoint{9.320225in}{4.495057in}}%
\pgfusepath{clip}%
\pgfsetbuttcap%
\pgfsetroundjoin%
\pgfsetlinewidth{0.000000pt}%
\definecolor{currentstroke}{rgb}{0.000000,0.000000,0.000000}%
\pgfsetstrokecolor{currentstroke}%
\pgfsetdash{}{0pt}%
\pgfpathmoveto{\pgfqpoint{3.156933in}{3.773270in}}%
\pgfpathlineto{\pgfqpoint{3.343159in}{3.773270in}}%
\pgfpathlineto{\pgfqpoint{3.343159in}{3.854998in}}%
\pgfpathlineto{\pgfqpoint{3.156933in}{3.854998in}}%
\pgfpathlineto{\pgfqpoint{3.156933in}{3.773270in}}%
\pgfusepath{}%
\end{pgfscope}%
\begin{pgfscope}%
\pgfpathrectangle{\pgfqpoint{0.549740in}{0.463273in}}{\pgfqpoint{9.320225in}{4.495057in}}%
\pgfusepath{clip}%
\pgfsetbuttcap%
\pgfsetroundjoin%
\pgfsetlinewidth{0.000000pt}%
\definecolor{currentstroke}{rgb}{0.000000,0.000000,0.000000}%
\pgfsetstrokecolor{currentstroke}%
\pgfsetdash{}{0pt}%
\pgfpathmoveto{\pgfqpoint{3.343159in}{3.773270in}}%
\pgfpathlineto{\pgfqpoint{3.529386in}{3.773270in}}%
\pgfpathlineto{\pgfqpoint{3.529386in}{3.854998in}}%
\pgfpathlineto{\pgfqpoint{3.343159in}{3.854998in}}%
\pgfpathlineto{\pgfqpoint{3.343159in}{3.773270in}}%
\pgfusepath{}%
\end{pgfscope}%
\begin{pgfscope}%
\pgfpathrectangle{\pgfqpoint{0.549740in}{0.463273in}}{\pgfqpoint{9.320225in}{4.495057in}}%
\pgfusepath{clip}%
\pgfsetbuttcap%
\pgfsetroundjoin%
\pgfsetlinewidth{0.000000pt}%
\definecolor{currentstroke}{rgb}{0.000000,0.000000,0.000000}%
\pgfsetstrokecolor{currentstroke}%
\pgfsetdash{}{0pt}%
\pgfpathmoveto{\pgfqpoint{3.529386in}{3.773270in}}%
\pgfpathlineto{\pgfqpoint{3.715612in}{3.773270in}}%
\pgfpathlineto{\pgfqpoint{3.715612in}{3.854998in}}%
\pgfpathlineto{\pgfqpoint{3.529386in}{3.854998in}}%
\pgfpathlineto{\pgfqpoint{3.529386in}{3.773270in}}%
\pgfusepath{}%
\end{pgfscope}%
\begin{pgfscope}%
\pgfpathrectangle{\pgfqpoint{0.549740in}{0.463273in}}{\pgfqpoint{9.320225in}{4.495057in}}%
\pgfusepath{clip}%
\pgfsetbuttcap%
\pgfsetroundjoin%
\pgfsetlinewidth{0.000000pt}%
\definecolor{currentstroke}{rgb}{0.000000,0.000000,0.000000}%
\pgfsetstrokecolor{currentstroke}%
\pgfsetdash{}{0pt}%
\pgfpathmoveto{\pgfqpoint{3.715612in}{3.773270in}}%
\pgfpathlineto{\pgfqpoint{3.901839in}{3.773270in}}%
\pgfpathlineto{\pgfqpoint{3.901839in}{3.854998in}}%
\pgfpathlineto{\pgfqpoint{3.715612in}{3.854998in}}%
\pgfpathlineto{\pgfqpoint{3.715612in}{3.773270in}}%
\pgfusepath{}%
\end{pgfscope}%
\begin{pgfscope}%
\pgfpathrectangle{\pgfqpoint{0.549740in}{0.463273in}}{\pgfqpoint{9.320225in}{4.495057in}}%
\pgfusepath{clip}%
\pgfsetbuttcap%
\pgfsetroundjoin%
\pgfsetlinewidth{0.000000pt}%
\definecolor{currentstroke}{rgb}{0.000000,0.000000,0.000000}%
\pgfsetstrokecolor{currentstroke}%
\pgfsetdash{}{0pt}%
\pgfpathmoveto{\pgfqpoint{3.901839in}{3.773270in}}%
\pgfpathlineto{\pgfqpoint{4.088065in}{3.773270in}}%
\pgfpathlineto{\pgfqpoint{4.088065in}{3.854998in}}%
\pgfpathlineto{\pgfqpoint{3.901839in}{3.854998in}}%
\pgfpathlineto{\pgfqpoint{3.901839in}{3.773270in}}%
\pgfusepath{}%
\end{pgfscope}%
\begin{pgfscope}%
\pgfpathrectangle{\pgfqpoint{0.549740in}{0.463273in}}{\pgfqpoint{9.320225in}{4.495057in}}%
\pgfusepath{clip}%
\pgfsetbuttcap%
\pgfsetroundjoin%
\pgfsetlinewidth{0.000000pt}%
\definecolor{currentstroke}{rgb}{0.000000,0.000000,0.000000}%
\pgfsetstrokecolor{currentstroke}%
\pgfsetdash{}{0pt}%
\pgfpathmoveto{\pgfqpoint{4.088065in}{3.773270in}}%
\pgfpathlineto{\pgfqpoint{4.274292in}{3.773270in}}%
\pgfpathlineto{\pgfqpoint{4.274292in}{3.854998in}}%
\pgfpathlineto{\pgfqpoint{4.088065in}{3.854998in}}%
\pgfpathlineto{\pgfqpoint{4.088065in}{3.773270in}}%
\pgfusepath{}%
\end{pgfscope}%
\begin{pgfscope}%
\pgfpathrectangle{\pgfqpoint{0.549740in}{0.463273in}}{\pgfqpoint{9.320225in}{4.495057in}}%
\pgfusepath{clip}%
\pgfsetbuttcap%
\pgfsetroundjoin%
\pgfsetlinewidth{0.000000pt}%
\definecolor{currentstroke}{rgb}{0.000000,0.000000,0.000000}%
\pgfsetstrokecolor{currentstroke}%
\pgfsetdash{}{0pt}%
\pgfpathmoveto{\pgfqpoint{4.274292in}{3.773270in}}%
\pgfpathlineto{\pgfqpoint{4.460519in}{3.773270in}}%
\pgfpathlineto{\pgfqpoint{4.460519in}{3.854998in}}%
\pgfpathlineto{\pgfqpoint{4.274292in}{3.854998in}}%
\pgfpathlineto{\pgfqpoint{4.274292in}{3.773270in}}%
\pgfusepath{}%
\end{pgfscope}%
\begin{pgfscope}%
\pgfpathrectangle{\pgfqpoint{0.549740in}{0.463273in}}{\pgfqpoint{9.320225in}{4.495057in}}%
\pgfusepath{clip}%
\pgfsetbuttcap%
\pgfsetroundjoin%
\pgfsetlinewidth{0.000000pt}%
\definecolor{currentstroke}{rgb}{0.000000,0.000000,0.000000}%
\pgfsetstrokecolor{currentstroke}%
\pgfsetdash{}{0pt}%
\pgfpathmoveto{\pgfqpoint{4.460519in}{3.773270in}}%
\pgfpathlineto{\pgfqpoint{4.646745in}{3.773270in}}%
\pgfpathlineto{\pgfqpoint{4.646745in}{3.854998in}}%
\pgfpathlineto{\pgfqpoint{4.460519in}{3.854998in}}%
\pgfpathlineto{\pgfqpoint{4.460519in}{3.773270in}}%
\pgfusepath{}%
\end{pgfscope}%
\begin{pgfscope}%
\pgfpathrectangle{\pgfqpoint{0.549740in}{0.463273in}}{\pgfqpoint{9.320225in}{4.495057in}}%
\pgfusepath{clip}%
\pgfsetbuttcap%
\pgfsetroundjoin%
\pgfsetlinewidth{0.000000pt}%
\definecolor{currentstroke}{rgb}{0.000000,0.000000,0.000000}%
\pgfsetstrokecolor{currentstroke}%
\pgfsetdash{}{0pt}%
\pgfpathmoveto{\pgfqpoint{4.646745in}{3.773270in}}%
\pgfpathlineto{\pgfqpoint{4.832972in}{3.773270in}}%
\pgfpathlineto{\pgfqpoint{4.832972in}{3.854998in}}%
\pgfpathlineto{\pgfqpoint{4.646745in}{3.854998in}}%
\pgfpathlineto{\pgfqpoint{4.646745in}{3.773270in}}%
\pgfusepath{}%
\end{pgfscope}%
\begin{pgfscope}%
\pgfpathrectangle{\pgfqpoint{0.549740in}{0.463273in}}{\pgfqpoint{9.320225in}{4.495057in}}%
\pgfusepath{clip}%
\pgfsetbuttcap%
\pgfsetroundjoin%
\pgfsetlinewidth{0.000000pt}%
\definecolor{currentstroke}{rgb}{0.000000,0.000000,0.000000}%
\pgfsetstrokecolor{currentstroke}%
\pgfsetdash{}{0pt}%
\pgfpathmoveto{\pgfqpoint{4.832972in}{3.773270in}}%
\pgfpathlineto{\pgfqpoint{5.019198in}{3.773270in}}%
\pgfpathlineto{\pgfqpoint{5.019198in}{3.854998in}}%
\pgfpathlineto{\pgfqpoint{4.832972in}{3.854998in}}%
\pgfpathlineto{\pgfqpoint{4.832972in}{3.773270in}}%
\pgfusepath{}%
\end{pgfscope}%
\begin{pgfscope}%
\pgfpathrectangle{\pgfqpoint{0.549740in}{0.463273in}}{\pgfqpoint{9.320225in}{4.495057in}}%
\pgfusepath{clip}%
\pgfsetbuttcap%
\pgfsetroundjoin%
\pgfsetlinewidth{0.000000pt}%
\definecolor{currentstroke}{rgb}{0.000000,0.000000,0.000000}%
\pgfsetstrokecolor{currentstroke}%
\pgfsetdash{}{0pt}%
\pgfpathmoveto{\pgfqpoint{5.019198in}{3.773270in}}%
\pgfpathlineto{\pgfqpoint{5.205425in}{3.773270in}}%
\pgfpathlineto{\pgfqpoint{5.205425in}{3.854998in}}%
\pgfpathlineto{\pgfqpoint{5.019198in}{3.854998in}}%
\pgfpathlineto{\pgfqpoint{5.019198in}{3.773270in}}%
\pgfusepath{}%
\end{pgfscope}%
\begin{pgfscope}%
\pgfpathrectangle{\pgfqpoint{0.549740in}{0.463273in}}{\pgfqpoint{9.320225in}{4.495057in}}%
\pgfusepath{clip}%
\pgfsetbuttcap%
\pgfsetroundjoin%
\pgfsetlinewidth{0.000000pt}%
\definecolor{currentstroke}{rgb}{0.000000,0.000000,0.000000}%
\pgfsetstrokecolor{currentstroke}%
\pgfsetdash{}{0pt}%
\pgfpathmoveto{\pgfqpoint{5.205425in}{3.773270in}}%
\pgfpathlineto{\pgfqpoint{5.391651in}{3.773270in}}%
\pgfpathlineto{\pgfqpoint{5.391651in}{3.854998in}}%
\pgfpathlineto{\pgfqpoint{5.205425in}{3.854998in}}%
\pgfpathlineto{\pgfqpoint{5.205425in}{3.773270in}}%
\pgfusepath{}%
\end{pgfscope}%
\begin{pgfscope}%
\pgfpathrectangle{\pgfqpoint{0.549740in}{0.463273in}}{\pgfqpoint{9.320225in}{4.495057in}}%
\pgfusepath{clip}%
\pgfsetbuttcap%
\pgfsetroundjoin%
\pgfsetlinewidth{0.000000pt}%
\definecolor{currentstroke}{rgb}{0.000000,0.000000,0.000000}%
\pgfsetstrokecolor{currentstroke}%
\pgfsetdash{}{0pt}%
\pgfpathmoveto{\pgfqpoint{5.391651in}{3.773270in}}%
\pgfpathlineto{\pgfqpoint{5.577878in}{3.773270in}}%
\pgfpathlineto{\pgfqpoint{5.577878in}{3.854998in}}%
\pgfpathlineto{\pgfqpoint{5.391651in}{3.854998in}}%
\pgfpathlineto{\pgfqpoint{5.391651in}{3.773270in}}%
\pgfusepath{}%
\end{pgfscope}%
\begin{pgfscope}%
\pgfpathrectangle{\pgfqpoint{0.549740in}{0.463273in}}{\pgfqpoint{9.320225in}{4.495057in}}%
\pgfusepath{clip}%
\pgfsetbuttcap%
\pgfsetroundjoin%
\pgfsetlinewidth{0.000000pt}%
\definecolor{currentstroke}{rgb}{0.000000,0.000000,0.000000}%
\pgfsetstrokecolor{currentstroke}%
\pgfsetdash{}{0pt}%
\pgfpathmoveto{\pgfqpoint{5.577878in}{3.773270in}}%
\pgfpathlineto{\pgfqpoint{5.764104in}{3.773270in}}%
\pgfpathlineto{\pgfqpoint{5.764104in}{3.854998in}}%
\pgfpathlineto{\pgfqpoint{5.577878in}{3.854998in}}%
\pgfpathlineto{\pgfqpoint{5.577878in}{3.773270in}}%
\pgfusepath{}%
\end{pgfscope}%
\begin{pgfscope}%
\pgfpathrectangle{\pgfqpoint{0.549740in}{0.463273in}}{\pgfqpoint{9.320225in}{4.495057in}}%
\pgfusepath{clip}%
\pgfsetbuttcap%
\pgfsetroundjoin%
\pgfsetlinewidth{0.000000pt}%
\definecolor{currentstroke}{rgb}{0.000000,0.000000,0.000000}%
\pgfsetstrokecolor{currentstroke}%
\pgfsetdash{}{0pt}%
\pgfpathmoveto{\pgfqpoint{5.764104in}{3.773270in}}%
\pgfpathlineto{\pgfqpoint{5.950331in}{3.773270in}}%
\pgfpathlineto{\pgfqpoint{5.950331in}{3.854998in}}%
\pgfpathlineto{\pgfqpoint{5.764104in}{3.854998in}}%
\pgfpathlineto{\pgfqpoint{5.764104in}{3.773270in}}%
\pgfusepath{}%
\end{pgfscope}%
\begin{pgfscope}%
\pgfpathrectangle{\pgfqpoint{0.549740in}{0.463273in}}{\pgfqpoint{9.320225in}{4.495057in}}%
\pgfusepath{clip}%
\pgfsetbuttcap%
\pgfsetroundjoin%
\pgfsetlinewidth{0.000000pt}%
\definecolor{currentstroke}{rgb}{0.000000,0.000000,0.000000}%
\pgfsetstrokecolor{currentstroke}%
\pgfsetdash{}{0pt}%
\pgfpathmoveto{\pgfqpoint{5.950331in}{3.773270in}}%
\pgfpathlineto{\pgfqpoint{6.136557in}{3.773270in}}%
\pgfpathlineto{\pgfqpoint{6.136557in}{3.854998in}}%
\pgfpathlineto{\pgfqpoint{5.950331in}{3.854998in}}%
\pgfpathlineto{\pgfqpoint{5.950331in}{3.773270in}}%
\pgfusepath{}%
\end{pgfscope}%
\begin{pgfscope}%
\pgfpathrectangle{\pgfqpoint{0.549740in}{0.463273in}}{\pgfqpoint{9.320225in}{4.495057in}}%
\pgfusepath{clip}%
\pgfsetbuttcap%
\pgfsetroundjoin%
\pgfsetlinewidth{0.000000pt}%
\definecolor{currentstroke}{rgb}{0.000000,0.000000,0.000000}%
\pgfsetstrokecolor{currentstroke}%
\pgfsetdash{}{0pt}%
\pgfpathmoveto{\pgfqpoint{6.136557in}{3.773270in}}%
\pgfpathlineto{\pgfqpoint{6.322784in}{3.773270in}}%
\pgfpathlineto{\pgfqpoint{6.322784in}{3.854998in}}%
\pgfpathlineto{\pgfqpoint{6.136557in}{3.854998in}}%
\pgfpathlineto{\pgfqpoint{6.136557in}{3.773270in}}%
\pgfusepath{}%
\end{pgfscope}%
\begin{pgfscope}%
\pgfpathrectangle{\pgfqpoint{0.549740in}{0.463273in}}{\pgfqpoint{9.320225in}{4.495057in}}%
\pgfusepath{clip}%
\pgfsetbuttcap%
\pgfsetroundjoin%
\pgfsetlinewidth{0.000000pt}%
\definecolor{currentstroke}{rgb}{0.000000,0.000000,0.000000}%
\pgfsetstrokecolor{currentstroke}%
\pgfsetdash{}{0pt}%
\pgfpathmoveto{\pgfqpoint{6.322784in}{3.773270in}}%
\pgfpathlineto{\pgfqpoint{6.509011in}{3.773270in}}%
\pgfpathlineto{\pgfqpoint{6.509011in}{3.854998in}}%
\pgfpathlineto{\pgfqpoint{6.322784in}{3.854998in}}%
\pgfpathlineto{\pgfqpoint{6.322784in}{3.773270in}}%
\pgfusepath{}%
\end{pgfscope}%
\begin{pgfscope}%
\pgfpathrectangle{\pgfqpoint{0.549740in}{0.463273in}}{\pgfqpoint{9.320225in}{4.495057in}}%
\pgfusepath{clip}%
\pgfsetbuttcap%
\pgfsetroundjoin%
\pgfsetlinewidth{0.000000pt}%
\definecolor{currentstroke}{rgb}{0.000000,0.000000,0.000000}%
\pgfsetstrokecolor{currentstroke}%
\pgfsetdash{}{0pt}%
\pgfpathmoveto{\pgfqpoint{6.509011in}{3.773270in}}%
\pgfpathlineto{\pgfqpoint{6.695237in}{3.773270in}}%
\pgfpathlineto{\pgfqpoint{6.695237in}{3.854998in}}%
\pgfpathlineto{\pgfqpoint{6.509011in}{3.854998in}}%
\pgfpathlineto{\pgfqpoint{6.509011in}{3.773270in}}%
\pgfusepath{}%
\end{pgfscope}%
\begin{pgfscope}%
\pgfpathrectangle{\pgfqpoint{0.549740in}{0.463273in}}{\pgfqpoint{9.320225in}{4.495057in}}%
\pgfusepath{clip}%
\pgfsetbuttcap%
\pgfsetroundjoin%
\pgfsetlinewidth{0.000000pt}%
\definecolor{currentstroke}{rgb}{0.000000,0.000000,0.000000}%
\pgfsetstrokecolor{currentstroke}%
\pgfsetdash{}{0pt}%
\pgfpathmoveto{\pgfqpoint{6.695237in}{3.773270in}}%
\pgfpathlineto{\pgfqpoint{6.881464in}{3.773270in}}%
\pgfpathlineto{\pgfqpoint{6.881464in}{3.854998in}}%
\pgfpathlineto{\pgfqpoint{6.695237in}{3.854998in}}%
\pgfpathlineto{\pgfqpoint{6.695237in}{3.773270in}}%
\pgfusepath{}%
\end{pgfscope}%
\begin{pgfscope}%
\pgfpathrectangle{\pgfqpoint{0.549740in}{0.463273in}}{\pgfqpoint{9.320225in}{4.495057in}}%
\pgfusepath{clip}%
\pgfsetbuttcap%
\pgfsetroundjoin%
\definecolor{currentfill}{rgb}{0.472869,0.711325,0.955316}%
\pgfsetfillcolor{currentfill}%
\pgfsetlinewidth{0.000000pt}%
\definecolor{currentstroke}{rgb}{0.000000,0.000000,0.000000}%
\pgfsetstrokecolor{currentstroke}%
\pgfsetdash{}{0pt}%
\pgfpathmoveto{\pgfqpoint{6.881464in}{3.773270in}}%
\pgfpathlineto{\pgfqpoint{7.067690in}{3.773270in}}%
\pgfpathlineto{\pgfqpoint{7.067690in}{3.854998in}}%
\pgfpathlineto{\pgfqpoint{6.881464in}{3.854998in}}%
\pgfpathlineto{\pgfqpoint{6.881464in}{3.773270in}}%
\pgfusepath{fill}%
\end{pgfscope}%
\begin{pgfscope}%
\pgfpathrectangle{\pgfqpoint{0.549740in}{0.463273in}}{\pgfqpoint{9.320225in}{4.495057in}}%
\pgfusepath{clip}%
\pgfsetbuttcap%
\pgfsetroundjoin%
\pgfsetlinewidth{0.000000pt}%
\definecolor{currentstroke}{rgb}{0.000000,0.000000,0.000000}%
\pgfsetstrokecolor{currentstroke}%
\pgfsetdash{}{0pt}%
\pgfpathmoveto{\pgfqpoint{7.067690in}{3.773270in}}%
\pgfpathlineto{\pgfqpoint{7.253917in}{3.773270in}}%
\pgfpathlineto{\pgfqpoint{7.253917in}{3.854998in}}%
\pgfpathlineto{\pgfqpoint{7.067690in}{3.854998in}}%
\pgfpathlineto{\pgfqpoint{7.067690in}{3.773270in}}%
\pgfusepath{}%
\end{pgfscope}%
\begin{pgfscope}%
\pgfpathrectangle{\pgfqpoint{0.549740in}{0.463273in}}{\pgfqpoint{9.320225in}{4.495057in}}%
\pgfusepath{clip}%
\pgfsetbuttcap%
\pgfsetroundjoin%
\pgfsetlinewidth{0.000000pt}%
\definecolor{currentstroke}{rgb}{0.000000,0.000000,0.000000}%
\pgfsetstrokecolor{currentstroke}%
\pgfsetdash{}{0pt}%
\pgfpathmoveto{\pgfqpoint{7.253917in}{3.773270in}}%
\pgfpathlineto{\pgfqpoint{7.440143in}{3.773270in}}%
\pgfpathlineto{\pgfqpoint{7.440143in}{3.854998in}}%
\pgfpathlineto{\pgfqpoint{7.253917in}{3.854998in}}%
\pgfpathlineto{\pgfqpoint{7.253917in}{3.773270in}}%
\pgfusepath{}%
\end{pgfscope}%
\begin{pgfscope}%
\pgfpathrectangle{\pgfqpoint{0.549740in}{0.463273in}}{\pgfqpoint{9.320225in}{4.495057in}}%
\pgfusepath{clip}%
\pgfsetbuttcap%
\pgfsetroundjoin%
\pgfsetlinewidth{0.000000pt}%
\definecolor{currentstroke}{rgb}{0.000000,0.000000,0.000000}%
\pgfsetstrokecolor{currentstroke}%
\pgfsetdash{}{0pt}%
\pgfpathmoveto{\pgfqpoint{7.440143in}{3.773270in}}%
\pgfpathlineto{\pgfqpoint{7.626370in}{3.773270in}}%
\pgfpathlineto{\pgfqpoint{7.626370in}{3.854998in}}%
\pgfpathlineto{\pgfqpoint{7.440143in}{3.854998in}}%
\pgfpathlineto{\pgfqpoint{7.440143in}{3.773270in}}%
\pgfusepath{}%
\end{pgfscope}%
\begin{pgfscope}%
\pgfpathrectangle{\pgfqpoint{0.549740in}{0.463273in}}{\pgfqpoint{9.320225in}{4.495057in}}%
\pgfusepath{clip}%
\pgfsetbuttcap%
\pgfsetroundjoin%
\pgfsetlinewidth{0.000000pt}%
\definecolor{currentstroke}{rgb}{0.000000,0.000000,0.000000}%
\pgfsetstrokecolor{currentstroke}%
\pgfsetdash{}{0pt}%
\pgfpathmoveto{\pgfqpoint{7.626370in}{3.773270in}}%
\pgfpathlineto{\pgfqpoint{7.812596in}{3.773270in}}%
\pgfpathlineto{\pgfqpoint{7.812596in}{3.854998in}}%
\pgfpathlineto{\pgfqpoint{7.626370in}{3.854998in}}%
\pgfpathlineto{\pgfqpoint{7.626370in}{3.773270in}}%
\pgfusepath{}%
\end{pgfscope}%
\begin{pgfscope}%
\pgfpathrectangle{\pgfqpoint{0.549740in}{0.463273in}}{\pgfqpoint{9.320225in}{4.495057in}}%
\pgfusepath{clip}%
\pgfsetbuttcap%
\pgfsetroundjoin%
\definecolor{currentfill}{rgb}{0.547810,0.736432,0.947518}%
\pgfsetfillcolor{currentfill}%
\pgfsetlinewidth{0.000000pt}%
\definecolor{currentstroke}{rgb}{0.000000,0.000000,0.000000}%
\pgfsetstrokecolor{currentstroke}%
\pgfsetdash{}{0pt}%
\pgfpathmoveto{\pgfqpoint{7.812596in}{3.773270in}}%
\pgfpathlineto{\pgfqpoint{7.998823in}{3.773270in}}%
\pgfpathlineto{\pgfqpoint{7.998823in}{3.854998in}}%
\pgfpathlineto{\pgfqpoint{7.812596in}{3.854998in}}%
\pgfpathlineto{\pgfqpoint{7.812596in}{3.773270in}}%
\pgfusepath{fill}%
\end{pgfscope}%
\begin{pgfscope}%
\pgfpathrectangle{\pgfqpoint{0.549740in}{0.463273in}}{\pgfqpoint{9.320225in}{4.495057in}}%
\pgfusepath{clip}%
\pgfsetbuttcap%
\pgfsetroundjoin%
\pgfsetlinewidth{0.000000pt}%
\definecolor{currentstroke}{rgb}{0.000000,0.000000,0.000000}%
\pgfsetstrokecolor{currentstroke}%
\pgfsetdash{}{0pt}%
\pgfpathmoveto{\pgfqpoint{7.998823in}{3.773270in}}%
\pgfpathlineto{\pgfqpoint{8.185049in}{3.773270in}}%
\pgfpathlineto{\pgfqpoint{8.185049in}{3.854998in}}%
\pgfpathlineto{\pgfqpoint{7.998823in}{3.854998in}}%
\pgfpathlineto{\pgfqpoint{7.998823in}{3.773270in}}%
\pgfusepath{}%
\end{pgfscope}%
\begin{pgfscope}%
\pgfpathrectangle{\pgfqpoint{0.549740in}{0.463273in}}{\pgfqpoint{9.320225in}{4.495057in}}%
\pgfusepath{clip}%
\pgfsetbuttcap%
\pgfsetroundjoin%
\pgfsetlinewidth{0.000000pt}%
\definecolor{currentstroke}{rgb}{0.000000,0.000000,0.000000}%
\pgfsetstrokecolor{currentstroke}%
\pgfsetdash{}{0pt}%
\pgfpathmoveto{\pgfqpoint{8.185049in}{3.773270in}}%
\pgfpathlineto{\pgfqpoint{8.371276in}{3.773270in}}%
\pgfpathlineto{\pgfqpoint{8.371276in}{3.854998in}}%
\pgfpathlineto{\pgfqpoint{8.185049in}{3.854998in}}%
\pgfpathlineto{\pgfqpoint{8.185049in}{3.773270in}}%
\pgfusepath{}%
\end{pgfscope}%
\begin{pgfscope}%
\pgfpathrectangle{\pgfqpoint{0.549740in}{0.463273in}}{\pgfqpoint{9.320225in}{4.495057in}}%
\pgfusepath{clip}%
\pgfsetbuttcap%
\pgfsetroundjoin%
\pgfsetlinewidth{0.000000pt}%
\definecolor{currentstroke}{rgb}{0.000000,0.000000,0.000000}%
\pgfsetstrokecolor{currentstroke}%
\pgfsetdash{}{0pt}%
\pgfpathmoveto{\pgfqpoint{8.371276in}{3.773270in}}%
\pgfpathlineto{\pgfqpoint{8.557503in}{3.773270in}}%
\pgfpathlineto{\pgfqpoint{8.557503in}{3.854998in}}%
\pgfpathlineto{\pgfqpoint{8.371276in}{3.854998in}}%
\pgfpathlineto{\pgfqpoint{8.371276in}{3.773270in}}%
\pgfusepath{}%
\end{pgfscope}%
\begin{pgfscope}%
\pgfpathrectangle{\pgfqpoint{0.549740in}{0.463273in}}{\pgfqpoint{9.320225in}{4.495057in}}%
\pgfusepath{clip}%
\pgfsetbuttcap%
\pgfsetroundjoin%
\pgfsetlinewidth{0.000000pt}%
\definecolor{currentstroke}{rgb}{0.000000,0.000000,0.000000}%
\pgfsetstrokecolor{currentstroke}%
\pgfsetdash{}{0pt}%
\pgfpathmoveto{\pgfqpoint{8.557503in}{3.773270in}}%
\pgfpathlineto{\pgfqpoint{8.743729in}{3.773270in}}%
\pgfpathlineto{\pgfqpoint{8.743729in}{3.854998in}}%
\pgfpathlineto{\pgfqpoint{8.557503in}{3.854998in}}%
\pgfpathlineto{\pgfqpoint{8.557503in}{3.773270in}}%
\pgfusepath{}%
\end{pgfscope}%
\begin{pgfscope}%
\pgfpathrectangle{\pgfqpoint{0.549740in}{0.463273in}}{\pgfqpoint{9.320225in}{4.495057in}}%
\pgfusepath{clip}%
\pgfsetbuttcap%
\pgfsetroundjoin%
\pgfsetlinewidth{0.000000pt}%
\definecolor{currentstroke}{rgb}{0.000000,0.000000,0.000000}%
\pgfsetstrokecolor{currentstroke}%
\pgfsetdash{}{0pt}%
\pgfpathmoveto{\pgfqpoint{8.743729in}{3.773270in}}%
\pgfpathlineto{\pgfqpoint{8.929956in}{3.773270in}}%
\pgfpathlineto{\pgfqpoint{8.929956in}{3.854998in}}%
\pgfpathlineto{\pgfqpoint{8.743729in}{3.854998in}}%
\pgfpathlineto{\pgfqpoint{8.743729in}{3.773270in}}%
\pgfusepath{}%
\end{pgfscope}%
\begin{pgfscope}%
\pgfpathrectangle{\pgfqpoint{0.549740in}{0.463273in}}{\pgfqpoint{9.320225in}{4.495057in}}%
\pgfusepath{clip}%
\pgfsetbuttcap%
\pgfsetroundjoin%
\pgfsetlinewidth{0.000000pt}%
\definecolor{currentstroke}{rgb}{0.000000,0.000000,0.000000}%
\pgfsetstrokecolor{currentstroke}%
\pgfsetdash{}{0pt}%
\pgfpathmoveto{\pgfqpoint{8.929956in}{3.773270in}}%
\pgfpathlineto{\pgfqpoint{9.116182in}{3.773270in}}%
\pgfpathlineto{\pgfqpoint{9.116182in}{3.854998in}}%
\pgfpathlineto{\pgfqpoint{8.929956in}{3.854998in}}%
\pgfpathlineto{\pgfqpoint{8.929956in}{3.773270in}}%
\pgfusepath{}%
\end{pgfscope}%
\begin{pgfscope}%
\pgfpathrectangle{\pgfqpoint{0.549740in}{0.463273in}}{\pgfqpoint{9.320225in}{4.495057in}}%
\pgfusepath{clip}%
\pgfsetbuttcap%
\pgfsetroundjoin%
\definecolor{currentfill}{rgb}{0.472869,0.711325,0.955316}%
\pgfsetfillcolor{currentfill}%
\pgfsetlinewidth{0.000000pt}%
\definecolor{currentstroke}{rgb}{0.000000,0.000000,0.000000}%
\pgfsetstrokecolor{currentstroke}%
\pgfsetdash{}{0pt}%
\pgfpathmoveto{\pgfqpoint{9.116182in}{3.773270in}}%
\pgfpathlineto{\pgfqpoint{9.302409in}{3.773270in}}%
\pgfpathlineto{\pgfqpoint{9.302409in}{3.854998in}}%
\pgfpathlineto{\pgfqpoint{9.116182in}{3.854998in}}%
\pgfpathlineto{\pgfqpoint{9.116182in}{3.773270in}}%
\pgfusepath{fill}%
\end{pgfscope}%
\begin{pgfscope}%
\pgfpathrectangle{\pgfqpoint{0.549740in}{0.463273in}}{\pgfqpoint{9.320225in}{4.495057in}}%
\pgfusepath{clip}%
\pgfsetbuttcap%
\pgfsetroundjoin%
\pgfsetlinewidth{0.000000pt}%
\definecolor{currentstroke}{rgb}{0.000000,0.000000,0.000000}%
\pgfsetstrokecolor{currentstroke}%
\pgfsetdash{}{0pt}%
\pgfpathmoveto{\pgfqpoint{9.302409in}{3.773270in}}%
\pgfpathlineto{\pgfqpoint{9.488635in}{3.773270in}}%
\pgfpathlineto{\pgfqpoint{9.488635in}{3.854998in}}%
\pgfpathlineto{\pgfqpoint{9.302409in}{3.854998in}}%
\pgfpathlineto{\pgfqpoint{9.302409in}{3.773270in}}%
\pgfusepath{}%
\end{pgfscope}%
\begin{pgfscope}%
\pgfpathrectangle{\pgfqpoint{0.549740in}{0.463273in}}{\pgfqpoint{9.320225in}{4.495057in}}%
\pgfusepath{clip}%
\pgfsetbuttcap%
\pgfsetroundjoin%
\pgfsetlinewidth{0.000000pt}%
\definecolor{currentstroke}{rgb}{0.000000,0.000000,0.000000}%
\pgfsetstrokecolor{currentstroke}%
\pgfsetdash{}{0pt}%
\pgfpathmoveto{\pgfqpoint{9.488635in}{3.773270in}}%
\pgfpathlineto{\pgfqpoint{9.674862in}{3.773270in}}%
\pgfpathlineto{\pgfqpoint{9.674862in}{3.854998in}}%
\pgfpathlineto{\pgfqpoint{9.488635in}{3.854998in}}%
\pgfpathlineto{\pgfqpoint{9.488635in}{3.773270in}}%
\pgfusepath{}%
\end{pgfscope}%
\begin{pgfscope}%
\pgfpathrectangle{\pgfqpoint{0.549740in}{0.463273in}}{\pgfqpoint{9.320225in}{4.495057in}}%
\pgfusepath{clip}%
\pgfsetbuttcap%
\pgfsetroundjoin%
\pgfsetlinewidth{0.000000pt}%
\definecolor{currentstroke}{rgb}{0.000000,0.000000,0.000000}%
\pgfsetstrokecolor{currentstroke}%
\pgfsetdash{}{0pt}%
\pgfpathmoveto{\pgfqpoint{9.674862in}{3.773270in}}%
\pgfpathlineto{\pgfqpoint{9.861088in}{3.773270in}}%
\pgfpathlineto{\pgfqpoint{9.861088in}{3.854998in}}%
\pgfpathlineto{\pgfqpoint{9.674862in}{3.854998in}}%
\pgfpathlineto{\pgfqpoint{9.674862in}{3.773270in}}%
\pgfusepath{}%
\end{pgfscope}%
\begin{pgfscope}%
\pgfpathrectangle{\pgfqpoint{0.549740in}{0.463273in}}{\pgfqpoint{9.320225in}{4.495057in}}%
\pgfusepath{clip}%
\pgfsetbuttcap%
\pgfsetroundjoin%
\pgfsetlinewidth{0.000000pt}%
\definecolor{currentstroke}{rgb}{0.000000,0.000000,0.000000}%
\pgfsetstrokecolor{currentstroke}%
\pgfsetdash{}{0pt}%
\pgfpathmoveto{\pgfqpoint{0.549761in}{3.854998in}}%
\pgfpathlineto{\pgfqpoint{0.735988in}{3.854998in}}%
\pgfpathlineto{\pgfqpoint{0.735988in}{3.936726in}}%
\pgfpathlineto{\pgfqpoint{0.549761in}{3.936726in}}%
\pgfpathlineto{\pgfqpoint{0.549761in}{3.854998in}}%
\pgfusepath{}%
\end{pgfscope}%
\begin{pgfscope}%
\pgfpathrectangle{\pgfqpoint{0.549740in}{0.463273in}}{\pgfqpoint{9.320225in}{4.495057in}}%
\pgfusepath{clip}%
\pgfsetbuttcap%
\pgfsetroundjoin%
\pgfsetlinewidth{0.000000pt}%
\definecolor{currentstroke}{rgb}{0.000000,0.000000,0.000000}%
\pgfsetstrokecolor{currentstroke}%
\pgfsetdash{}{0pt}%
\pgfpathmoveto{\pgfqpoint{0.735988in}{3.854998in}}%
\pgfpathlineto{\pgfqpoint{0.922214in}{3.854998in}}%
\pgfpathlineto{\pgfqpoint{0.922214in}{3.936726in}}%
\pgfpathlineto{\pgfqpoint{0.735988in}{3.936726in}}%
\pgfpathlineto{\pgfqpoint{0.735988in}{3.854998in}}%
\pgfusepath{}%
\end{pgfscope}%
\begin{pgfscope}%
\pgfpathrectangle{\pgfqpoint{0.549740in}{0.463273in}}{\pgfqpoint{9.320225in}{4.495057in}}%
\pgfusepath{clip}%
\pgfsetbuttcap%
\pgfsetroundjoin%
\pgfsetlinewidth{0.000000pt}%
\definecolor{currentstroke}{rgb}{0.000000,0.000000,0.000000}%
\pgfsetstrokecolor{currentstroke}%
\pgfsetdash{}{0pt}%
\pgfpathmoveto{\pgfqpoint{0.922214in}{3.854998in}}%
\pgfpathlineto{\pgfqpoint{1.108441in}{3.854998in}}%
\pgfpathlineto{\pgfqpoint{1.108441in}{3.936726in}}%
\pgfpathlineto{\pgfqpoint{0.922214in}{3.936726in}}%
\pgfpathlineto{\pgfqpoint{0.922214in}{3.854998in}}%
\pgfusepath{}%
\end{pgfscope}%
\begin{pgfscope}%
\pgfpathrectangle{\pgfqpoint{0.549740in}{0.463273in}}{\pgfqpoint{9.320225in}{4.495057in}}%
\pgfusepath{clip}%
\pgfsetbuttcap%
\pgfsetroundjoin%
\pgfsetlinewidth{0.000000pt}%
\definecolor{currentstroke}{rgb}{0.000000,0.000000,0.000000}%
\pgfsetstrokecolor{currentstroke}%
\pgfsetdash{}{0pt}%
\pgfpathmoveto{\pgfqpoint{1.108441in}{3.854998in}}%
\pgfpathlineto{\pgfqpoint{1.294667in}{3.854998in}}%
\pgfpathlineto{\pgfqpoint{1.294667in}{3.936726in}}%
\pgfpathlineto{\pgfqpoint{1.108441in}{3.936726in}}%
\pgfpathlineto{\pgfqpoint{1.108441in}{3.854998in}}%
\pgfusepath{}%
\end{pgfscope}%
\begin{pgfscope}%
\pgfpathrectangle{\pgfqpoint{0.549740in}{0.463273in}}{\pgfqpoint{9.320225in}{4.495057in}}%
\pgfusepath{clip}%
\pgfsetbuttcap%
\pgfsetroundjoin%
\pgfsetlinewidth{0.000000pt}%
\definecolor{currentstroke}{rgb}{0.000000,0.000000,0.000000}%
\pgfsetstrokecolor{currentstroke}%
\pgfsetdash{}{0pt}%
\pgfpathmoveto{\pgfqpoint{1.294667in}{3.854998in}}%
\pgfpathlineto{\pgfqpoint{1.480894in}{3.854998in}}%
\pgfpathlineto{\pgfqpoint{1.480894in}{3.936726in}}%
\pgfpathlineto{\pgfqpoint{1.294667in}{3.936726in}}%
\pgfpathlineto{\pgfqpoint{1.294667in}{3.854998in}}%
\pgfusepath{}%
\end{pgfscope}%
\begin{pgfscope}%
\pgfpathrectangle{\pgfqpoint{0.549740in}{0.463273in}}{\pgfqpoint{9.320225in}{4.495057in}}%
\pgfusepath{clip}%
\pgfsetbuttcap%
\pgfsetroundjoin%
\pgfsetlinewidth{0.000000pt}%
\definecolor{currentstroke}{rgb}{0.000000,0.000000,0.000000}%
\pgfsetstrokecolor{currentstroke}%
\pgfsetdash{}{0pt}%
\pgfpathmoveto{\pgfqpoint{1.480894in}{3.854998in}}%
\pgfpathlineto{\pgfqpoint{1.667120in}{3.854998in}}%
\pgfpathlineto{\pgfqpoint{1.667120in}{3.936726in}}%
\pgfpathlineto{\pgfqpoint{1.480894in}{3.936726in}}%
\pgfpathlineto{\pgfqpoint{1.480894in}{3.854998in}}%
\pgfusepath{}%
\end{pgfscope}%
\begin{pgfscope}%
\pgfpathrectangle{\pgfqpoint{0.549740in}{0.463273in}}{\pgfqpoint{9.320225in}{4.495057in}}%
\pgfusepath{clip}%
\pgfsetbuttcap%
\pgfsetroundjoin%
\pgfsetlinewidth{0.000000pt}%
\definecolor{currentstroke}{rgb}{0.000000,0.000000,0.000000}%
\pgfsetstrokecolor{currentstroke}%
\pgfsetdash{}{0pt}%
\pgfpathmoveto{\pgfqpoint{1.667120in}{3.854998in}}%
\pgfpathlineto{\pgfqpoint{1.853347in}{3.854998in}}%
\pgfpathlineto{\pgfqpoint{1.853347in}{3.936726in}}%
\pgfpathlineto{\pgfqpoint{1.667120in}{3.936726in}}%
\pgfpathlineto{\pgfqpoint{1.667120in}{3.854998in}}%
\pgfusepath{}%
\end{pgfscope}%
\begin{pgfscope}%
\pgfpathrectangle{\pgfqpoint{0.549740in}{0.463273in}}{\pgfqpoint{9.320225in}{4.495057in}}%
\pgfusepath{clip}%
\pgfsetbuttcap%
\pgfsetroundjoin%
\pgfsetlinewidth{0.000000pt}%
\definecolor{currentstroke}{rgb}{0.000000,0.000000,0.000000}%
\pgfsetstrokecolor{currentstroke}%
\pgfsetdash{}{0pt}%
\pgfpathmoveto{\pgfqpoint{1.853347in}{3.854998in}}%
\pgfpathlineto{\pgfqpoint{2.039573in}{3.854998in}}%
\pgfpathlineto{\pgfqpoint{2.039573in}{3.936726in}}%
\pgfpathlineto{\pgfqpoint{1.853347in}{3.936726in}}%
\pgfpathlineto{\pgfqpoint{1.853347in}{3.854998in}}%
\pgfusepath{}%
\end{pgfscope}%
\begin{pgfscope}%
\pgfpathrectangle{\pgfqpoint{0.549740in}{0.463273in}}{\pgfqpoint{9.320225in}{4.495057in}}%
\pgfusepath{clip}%
\pgfsetbuttcap%
\pgfsetroundjoin%
\pgfsetlinewidth{0.000000pt}%
\definecolor{currentstroke}{rgb}{0.000000,0.000000,0.000000}%
\pgfsetstrokecolor{currentstroke}%
\pgfsetdash{}{0pt}%
\pgfpathmoveto{\pgfqpoint{2.039573in}{3.854998in}}%
\pgfpathlineto{\pgfqpoint{2.225800in}{3.854998in}}%
\pgfpathlineto{\pgfqpoint{2.225800in}{3.936726in}}%
\pgfpathlineto{\pgfqpoint{2.039573in}{3.936726in}}%
\pgfpathlineto{\pgfqpoint{2.039573in}{3.854998in}}%
\pgfusepath{}%
\end{pgfscope}%
\begin{pgfscope}%
\pgfpathrectangle{\pgfqpoint{0.549740in}{0.463273in}}{\pgfqpoint{9.320225in}{4.495057in}}%
\pgfusepath{clip}%
\pgfsetbuttcap%
\pgfsetroundjoin%
\pgfsetlinewidth{0.000000pt}%
\definecolor{currentstroke}{rgb}{0.000000,0.000000,0.000000}%
\pgfsetstrokecolor{currentstroke}%
\pgfsetdash{}{0pt}%
\pgfpathmoveto{\pgfqpoint{2.225800in}{3.854998in}}%
\pgfpathlineto{\pgfqpoint{2.412027in}{3.854998in}}%
\pgfpathlineto{\pgfqpoint{2.412027in}{3.936726in}}%
\pgfpathlineto{\pgfqpoint{2.225800in}{3.936726in}}%
\pgfpathlineto{\pgfqpoint{2.225800in}{3.854998in}}%
\pgfusepath{}%
\end{pgfscope}%
\begin{pgfscope}%
\pgfpathrectangle{\pgfqpoint{0.549740in}{0.463273in}}{\pgfqpoint{9.320225in}{4.495057in}}%
\pgfusepath{clip}%
\pgfsetbuttcap%
\pgfsetroundjoin%
\pgfsetlinewidth{0.000000pt}%
\definecolor{currentstroke}{rgb}{0.000000,0.000000,0.000000}%
\pgfsetstrokecolor{currentstroke}%
\pgfsetdash{}{0pt}%
\pgfpathmoveto{\pgfqpoint{2.412027in}{3.854998in}}%
\pgfpathlineto{\pgfqpoint{2.598253in}{3.854998in}}%
\pgfpathlineto{\pgfqpoint{2.598253in}{3.936726in}}%
\pgfpathlineto{\pgfqpoint{2.412027in}{3.936726in}}%
\pgfpathlineto{\pgfqpoint{2.412027in}{3.854998in}}%
\pgfusepath{}%
\end{pgfscope}%
\begin{pgfscope}%
\pgfpathrectangle{\pgfqpoint{0.549740in}{0.463273in}}{\pgfqpoint{9.320225in}{4.495057in}}%
\pgfusepath{clip}%
\pgfsetbuttcap%
\pgfsetroundjoin%
\pgfsetlinewidth{0.000000pt}%
\definecolor{currentstroke}{rgb}{0.000000,0.000000,0.000000}%
\pgfsetstrokecolor{currentstroke}%
\pgfsetdash{}{0pt}%
\pgfpathmoveto{\pgfqpoint{2.598253in}{3.854998in}}%
\pgfpathlineto{\pgfqpoint{2.784480in}{3.854998in}}%
\pgfpathlineto{\pgfqpoint{2.784480in}{3.936726in}}%
\pgfpathlineto{\pgfqpoint{2.598253in}{3.936726in}}%
\pgfpathlineto{\pgfqpoint{2.598253in}{3.854998in}}%
\pgfusepath{}%
\end{pgfscope}%
\begin{pgfscope}%
\pgfpathrectangle{\pgfqpoint{0.549740in}{0.463273in}}{\pgfqpoint{9.320225in}{4.495057in}}%
\pgfusepath{clip}%
\pgfsetbuttcap%
\pgfsetroundjoin%
\pgfsetlinewidth{0.000000pt}%
\definecolor{currentstroke}{rgb}{0.000000,0.000000,0.000000}%
\pgfsetstrokecolor{currentstroke}%
\pgfsetdash{}{0pt}%
\pgfpathmoveto{\pgfqpoint{2.784480in}{3.854998in}}%
\pgfpathlineto{\pgfqpoint{2.970706in}{3.854998in}}%
\pgfpathlineto{\pgfqpoint{2.970706in}{3.936726in}}%
\pgfpathlineto{\pgfqpoint{2.784480in}{3.936726in}}%
\pgfpathlineto{\pgfqpoint{2.784480in}{3.854998in}}%
\pgfusepath{}%
\end{pgfscope}%
\begin{pgfscope}%
\pgfpathrectangle{\pgfqpoint{0.549740in}{0.463273in}}{\pgfqpoint{9.320225in}{4.495057in}}%
\pgfusepath{clip}%
\pgfsetbuttcap%
\pgfsetroundjoin%
\pgfsetlinewidth{0.000000pt}%
\definecolor{currentstroke}{rgb}{0.000000,0.000000,0.000000}%
\pgfsetstrokecolor{currentstroke}%
\pgfsetdash{}{0pt}%
\pgfpathmoveto{\pgfqpoint{2.970706in}{3.854998in}}%
\pgfpathlineto{\pgfqpoint{3.156933in}{3.854998in}}%
\pgfpathlineto{\pgfqpoint{3.156933in}{3.936726in}}%
\pgfpathlineto{\pgfqpoint{2.970706in}{3.936726in}}%
\pgfpathlineto{\pgfqpoint{2.970706in}{3.854998in}}%
\pgfusepath{}%
\end{pgfscope}%
\begin{pgfscope}%
\pgfpathrectangle{\pgfqpoint{0.549740in}{0.463273in}}{\pgfqpoint{9.320225in}{4.495057in}}%
\pgfusepath{clip}%
\pgfsetbuttcap%
\pgfsetroundjoin%
\pgfsetlinewidth{0.000000pt}%
\definecolor{currentstroke}{rgb}{0.000000,0.000000,0.000000}%
\pgfsetstrokecolor{currentstroke}%
\pgfsetdash{}{0pt}%
\pgfpathmoveto{\pgfqpoint{3.156933in}{3.854998in}}%
\pgfpathlineto{\pgfqpoint{3.343159in}{3.854998in}}%
\pgfpathlineto{\pgfqpoint{3.343159in}{3.936726in}}%
\pgfpathlineto{\pgfqpoint{3.156933in}{3.936726in}}%
\pgfpathlineto{\pgfqpoint{3.156933in}{3.854998in}}%
\pgfusepath{}%
\end{pgfscope}%
\begin{pgfscope}%
\pgfpathrectangle{\pgfqpoint{0.549740in}{0.463273in}}{\pgfqpoint{9.320225in}{4.495057in}}%
\pgfusepath{clip}%
\pgfsetbuttcap%
\pgfsetroundjoin%
\pgfsetlinewidth{0.000000pt}%
\definecolor{currentstroke}{rgb}{0.000000,0.000000,0.000000}%
\pgfsetstrokecolor{currentstroke}%
\pgfsetdash{}{0pt}%
\pgfpathmoveto{\pgfqpoint{3.343159in}{3.854998in}}%
\pgfpathlineto{\pgfqpoint{3.529386in}{3.854998in}}%
\pgfpathlineto{\pgfqpoint{3.529386in}{3.936726in}}%
\pgfpathlineto{\pgfqpoint{3.343159in}{3.936726in}}%
\pgfpathlineto{\pgfqpoint{3.343159in}{3.854998in}}%
\pgfusepath{}%
\end{pgfscope}%
\begin{pgfscope}%
\pgfpathrectangle{\pgfqpoint{0.549740in}{0.463273in}}{\pgfqpoint{9.320225in}{4.495057in}}%
\pgfusepath{clip}%
\pgfsetbuttcap%
\pgfsetroundjoin%
\pgfsetlinewidth{0.000000pt}%
\definecolor{currentstroke}{rgb}{0.000000,0.000000,0.000000}%
\pgfsetstrokecolor{currentstroke}%
\pgfsetdash{}{0pt}%
\pgfpathmoveto{\pgfqpoint{3.529386in}{3.854998in}}%
\pgfpathlineto{\pgfqpoint{3.715612in}{3.854998in}}%
\pgfpathlineto{\pgfqpoint{3.715612in}{3.936726in}}%
\pgfpathlineto{\pgfqpoint{3.529386in}{3.936726in}}%
\pgfpathlineto{\pgfqpoint{3.529386in}{3.854998in}}%
\pgfusepath{}%
\end{pgfscope}%
\begin{pgfscope}%
\pgfpathrectangle{\pgfqpoint{0.549740in}{0.463273in}}{\pgfqpoint{9.320225in}{4.495057in}}%
\pgfusepath{clip}%
\pgfsetbuttcap%
\pgfsetroundjoin%
\pgfsetlinewidth{0.000000pt}%
\definecolor{currentstroke}{rgb}{0.000000,0.000000,0.000000}%
\pgfsetstrokecolor{currentstroke}%
\pgfsetdash{}{0pt}%
\pgfpathmoveto{\pgfqpoint{3.715612in}{3.854998in}}%
\pgfpathlineto{\pgfqpoint{3.901839in}{3.854998in}}%
\pgfpathlineto{\pgfqpoint{3.901839in}{3.936726in}}%
\pgfpathlineto{\pgfqpoint{3.715612in}{3.936726in}}%
\pgfpathlineto{\pgfqpoint{3.715612in}{3.854998in}}%
\pgfusepath{}%
\end{pgfscope}%
\begin{pgfscope}%
\pgfpathrectangle{\pgfqpoint{0.549740in}{0.463273in}}{\pgfqpoint{9.320225in}{4.495057in}}%
\pgfusepath{clip}%
\pgfsetbuttcap%
\pgfsetroundjoin%
\pgfsetlinewidth{0.000000pt}%
\definecolor{currentstroke}{rgb}{0.000000,0.000000,0.000000}%
\pgfsetstrokecolor{currentstroke}%
\pgfsetdash{}{0pt}%
\pgfpathmoveto{\pgfqpoint{3.901839in}{3.854998in}}%
\pgfpathlineto{\pgfqpoint{4.088065in}{3.854998in}}%
\pgfpathlineto{\pgfqpoint{4.088065in}{3.936726in}}%
\pgfpathlineto{\pgfqpoint{3.901839in}{3.936726in}}%
\pgfpathlineto{\pgfqpoint{3.901839in}{3.854998in}}%
\pgfusepath{}%
\end{pgfscope}%
\begin{pgfscope}%
\pgfpathrectangle{\pgfqpoint{0.549740in}{0.463273in}}{\pgfqpoint{9.320225in}{4.495057in}}%
\pgfusepath{clip}%
\pgfsetbuttcap%
\pgfsetroundjoin%
\pgfsetlinewidth{0.000000pt}%
\definecolor{currentstroke}{rgb}{0.000000,0.000000,0.000000}%
\pgfsetstrokecolor{currentstroke}%
\pgfsetdash{}{0pt}%
\pgfpathmoveto{\pgfqpoint{4.088065in}{3.854998in}}%
\pgfpathlineto{\pgfqpoint{4.274292in}{3.854998in}}%
\pgfpathlineto{\pgfqpoint{4.274292in}{3.936726in}}%
\pgfpathlineto{\pgfqpoint{4.088065in}{3.936726in}}%
\pgfpathlineto{\pgfqpoint{4.088065in}{3.854998in}}%
\pgfusepath{}%
\end{pgfscope}%
\begin{pgfscope}%
\pgfpathrectangle{\pgfqpoint{0.549740in}{0.463273in}}{\pgfqpoint{9.320225in}{4.495057in}}%
\pgfusepath{clip}%
\pgfsetbuttcap%
\pgfsetroundjoin%
\pgfsetlinewidth{0.000000pt}%
\definecolor{currentstroke}{rgb}{0.000000,0.000000,0.000000}%
\pgfsetstrokecolor{currentstroke}%
\pgfsetdash{}{0pt}%
\pgfpathmoveto{\pgfqpoint{4.274292in}{3.854998in}}%
\pgfpathlineto{\pgfqpoint{4.460519in}{3.854998in}}%
\pgfpathlineto{\pgfqpoint{4.460519in}{3.936726in}}%
\pgfpathlineto{\pgfqpoint{4.274292in}{3.936726in}}%
\pgfpathlineto{\pgfqpoint{4.274292in}{3.854998in}}%
\pgfusepath{}%
\end{pgfscope}%
\begin{pgfscope}%
\pgfpathrectangle{\pgfqpoint{0.549740in}{0.463273in}}{\pgfqpoint{9.320225in}{4.495057in}}%
\pgfusepath{clip}%
\pgfsetbuttcap%
\pgfsetroundjoin%
\pgfsetlinewidth{0.000000pt}%
\definecolor{currentstroke}{rgb}{0.000000,0.000000,0.000000}%
\pgfsetstrokecolor{currentstroke}%
\pgfsetdash{}{0pt}%
\pgfpathmoveto{\pgfqpoint{4.460519in}{3.854998in}}%
\pgfpathlineto{\pgfqpoint{4.646745in}{3.854998in}}%
\pgfpathlineto{\pgfqpoint{4.646745in}{3.936726in}}%
\pgfpathlineto{\pgfqpoint{4.460519in}{3.936726in}}%
\pgfpathlineto{\pgfqpoint{4.460519in}{3.854998in}}%
\pgfusepath{}%
\end{pgfscope}%
\begin{pgfscope}%
\pgfpathrectangle{\pgfqpoint{0.549740in}{0.463273in}}{\pgfqpoint{9.320225in}{4.495057in}}%
\pgfusepath{clip}%
\pgfsetbuttcap%
\pgfsetroundjoin%
\pgfsetlinewidth{0.000000pt}%
\definecolor{currentstroke}{rgb}{0.000000,0.000000,0.000000}%
\pgfsetstrokecolor{currentstroke}%
\pgfsetdash{}{0pt}%
\pgfpathmoveto{\pgfqpoint{4.646745in}{3.854998in}}%
\pgfpathlineto{\pgfqpoint{4.832972in}{3.854998in}}%
\pgfpathlineto{\pgfqpoint{4.832972in}{3.936726in}}%
\pgfpathlineto{\pgfqpoint{4.646745in}{3.936726in}}%
\pgfpathlineto{\pgfqpoint{4.646745in}{3.854998in}}%
\pgfusepath{}%
\end{pgfscope}%
\begin{pgfscope}%
\pgfpathrectangle{\pgfqpoint{0.549740in}{0.463273in}}{\pgfqpoint{9.320225in}{4.495057in}}%
\pgfusepath{clip}%
\pgfsetbuttcap%
\pgfsetroundjoin%
\pgfsetlinewidth{0.000000pt}%
\definecolor{currentstroke}{rgb}{0.000000,0.000000,0.000000}%
\pgfsetstrokecolor{currentstroke}%
\pgfsetdash{}{0pt}%
\pgfpathmoveto{\pgfqpoint{4.832972in}{3.854998in}}%
\pgfpathlineto{\pgfqpoint{5.019198in}{3.854998in}}%
\pgfpathlineto{\pgfqpoint{5.019198in}{3.936726in}}%
\pgfpathlineto{\pgfqpoint{4.832972in}{3.936726in}}%
\pgfpathlineto{\pgfqpoint{4.832972in}{3.854998in}}%
\pgfusepath{}%
\end{pgfscope}%
\begin{pgfscope}%
\pgfpathrectangle{\pgfqpoint{0.549740in}{0.463273in}}{\pgfqpoint{9.320225in}{4.495057in}}%
\pgfusepath{clip}%
\pgfsetbuttcap%
\pgfsetroundjoin%
\pgfsetlinewidth{0.000000pt}%
\definecolor{currentstroke}{rgb}{0.000000,0.000000,0.000000}%
\pgfsetstrokecolor{currentstroke}%
\pgfsetdash{}{0pt}%
\pgfpathmoveto{\pgfqpoint{5.019198in}{3.854998in}}%
\pgfpathlineto{\pgfqpoint{5.205425in}{3.854998in}}%
\pgfpathlineto{\pgfqpoint{5.205425in}{3.936726in}}%
\pgfpathlineto{\pgfqpoint{5.019198in}{3.936726in}}%
\pgfpathlineto{\pgfqpoint{5.019198in}{3.854998in}}%
\pgfusepath{}%
\end{pgfscope}%
\begin{pgfscope}%
\pgfpathrectangle{\pgfqpoint{0.549740in}{0.463273in}}{\pgfqpoint{9.320225in}{4.495057in}}%
\pgfusepath{clip}%
\pgfsetbuttcap%
\pgfsetroundjoin%
\pgfsetlinewidth{0.000000pt}%
\definecolor{currentstroke}{rgb}{0.000000,0.000000,0.000000}%
\pgfsetstrokecolor{currentstroke}%
\pgfsetdash{}{0pt}%
\pgfpathmoveto{\pgfqpoint{5.205425in}{3.854998in}}%
\pgfpathlineto{\pgfqpoint{5.391651in}{3.854998in}}%
\pgfpathlineto{\pgfqpoint{5.391651in}{3.936726in}}%
\pgfpathlineto{\pgfqpoint{5.205425in}{3.936726in}}%
\pgfpathlineto{\pgfqpoint{5.205425in}{3.854998in}}%
\pgfusepath{}%
\end{pgfscope}%
\begin{pgfscope}%
\pgfpathrectangle{\pgfqpoint{0.549740in}{0.463273in}}{\pgfqpoint{9.320225in}{4.495057in}}%
\pgfusepath{clip}%
\pgfsetbuttcap%
\pgfsetroundjoin%
\pgfsetlinewidth{0.000000pt}%
\definecolor{currentstroke}{rgb}{0.000000,0.000000,0.000000}%
\pgfsetstrokecolor{currentstroke}%
\pgfsetdash{}{0pt}%
\pgfpathmoveto{\pgfqpoint{5.391651in}{3.854998in}}%
\pgfpathlineto{\pgfqpoint{5.577878in}{3.854998in}}%
\pgfpathlineto{\pgfqpoint{5.577878in}{3.936726in}}%
\pgfpathlineto{\pgfqpoint{5.391651in}{3.936726in}}%
\pgfpathlineto{\pgfqpoint{5.391651in}{3.854998in}}%
\pgfusepath{}%
\end{pgfscope}%
\begin{pgfscope}%
\pgfpathrectangle{\pgfqpoint{0.549740in}{0.463273in}}{\pgfqpoint{9.320225in}{4.495057in}}%
\pgfusepath{clip}%
\pgfsetbuttcap%
\pgfsetroundjoin%
\pgfsetlinewidth{0.000000pt}%
\definecolor{currentstroke}{rgb}{0.000000,0.000000,0.000000}%
\pgfsetstrokecolor{currentstroke}%
\pgfsetdash{}{0pt}%
\pgfpathmoveto{\pgfqpoint{5.577878in}{3.854998in}}%
\pgfpathlineto{\pgfqpoint{5.764104in}{3.854998in}}%
\pgfpathlineto{\pgfqpoint{5.764104in}{3.936726in}}%
\pgfpathlineto{\pgfqpoint{5.577878in}{3.936726in}}%
\pgfpathlineto{\pgfqpoint{5.577878in}{3.854998in}}%
\pgfusepath{}%
\end{pgfscope}%
\begin{pgfscope}%
\pgfpathrectangle{\pgfqpoint{0.549740in}{0.463273in}}{\pgfqpoint{9.320225in}{4.495057in}}%
\pgfusepath{clip}%
\pgfsetbuttcap%
\pgfsetroundjoin%
\pgfsetlinewidth{0.000000pt}%
\definecolor{currentstroke}{rgb}{0.000000,0.000000,0.000000}%
\pgfsetstrokecolor{currentstroke}%
\pgfsetdash{}{0pt}%
\pgfpathmoveto{\pgfqpoint{5.764104in}{3.854998in}}%
\pgfpathlineto{\pgfqpoint{5.950331in}{3.854998in}}%
\pgfpathlineto{\pgfqpoint{5.950331in}{3.936726in}}%
\pgfpathlineto{\pgfqpoint{5.764104in}{3.936726in}}%
\pgfpathlineto{\pgfqpoint{5.764104in}{3.854998in}}%
\pgfusepath{}%
\end{pgfscope}%
\begin{pgfscope}%
\pgfpathrectangle{\pgfqpoint{0.549740in}{0.463273in}}{\pgfqpoint{9.320225in}{4.495057in}}%
\pgfusepath{clip}%
\pgfsetbuttcap%
\pgfsetroundjoin%
\pgfsetlinewidth{0.000000pt}%
\definecolor{currentstroke}{rgb}{0.000000,0.000000,0.000000}%
\pgfsetstrokecolor{currentstroke}%
\pgfsetdash{}{0pt}%
\pgfpathmoveto{\pgfqpoint{5.950331in}{3.854998in}}%
\pgfpathlineto{\pgfqpoint{6.136557in}{3.854998in}}%
\pgfpathlineto{\pgfqpoint{6.136557in}{3.936726in}}%
\pgfpathlineto{\pgfqpoint{5.950331in}{3.936726in}}%
\pgfpathlineto{\pgfqpoint{5.950331in}{3.854998in}}%
\pgfusepath{}%
\end{pgfscope}%
\begin{pgfscope}%
\pgfpathrectangle{\pgfqpoint{0.549740in}{0.463273in}}{\pgfqpoint{9.320225in}{4.495057in}}%
\pgfusepath{clip}%
\pgfsetbuttcap%
\pgfsetroundjoin%
\pgfsetlinewidth{0.000000pt}%
\definecolor{currentstroke}{rgb}{0.000000,0.000000,0.000000}%
\pgfsetstrokecolor{currentstroke}%
\pgfsetdash{}{0pt}%
\pgfpathmoveto{\pgfqpoint{6.136557in}{3.854998in}}%
\pgfpathlineto{\pgfqpoint{6.322784in}{3.854998in}}%
\pgfpathlineto{\pgfqpoint{6.322784in}{3.936726in}}%
\pgfpathlineto{\pgfqpoint{6.136557in}{3.936726in}}%
\pgfpathlineto{\pgfqpoint{6.136557in}{3.854998in}}%
\pgfusepath{}%
\end{pgfscope}%
\begin{pgfscope}%
\pgfpathrectangle{\pgfqpoint{0.549740in}{0.463273in}}{\pgfqpoint{9.320225in}{4.495057in}}%
\pgfusepath{clip}%
\pgfsetbuttcap%
\pgfsetroundjoin%
\pgfsetlinewidth{0.000000pt}%
\definecolor{currentstroke}{rgb}{0.000000,0.000000,0.000000}%
\pgfsetstrokecolor{currentstroke}%
\pgfsetdash{}{0pt}%
\pgfpathmoveto{\pgfqpoint{6.322784in}{3.854998in}}%
\pgfpathlineto{\pgfqpoint{6.509011in}{3.854998in}}%
\pgfpathlineto{\pgfqpoint{6.509011in}{3.936726in}}%
\pgfpathlineto{\pgfqpoint{6.322784in}{3.936726in}}%
\pgfpathlineto{\pgfqpoint{6.322784in}{3.854998in}}%
\pgfusepath{}%
\end{pgfscope}%
\begin{pgfscope}%
\pgfpathrectangle{\pgfqpoint{0.549740in}{0.463273in}}{\pgfqpoint{9.320225in}{4.495057in}}%
\pgfusepath{clip}%
\pgfsetbuttcap%
\pgfsetroundjoin%
\pgfsetlinewidth{0.000000pt}%
\definecolor{currentstroke}{rgb}{0.000000,0.000000,0.000000}%
\pgfsetstrokecolor{currentstroke}%
\pgfsetdash{}{0pt}%
\pgfpathmoveto{\pgfqpoint{6.509011in}{3.854998in}}%
\pgfpathlineto{\pgfqpoint{6.695237in}{3.854998in}}%
\pgfpathlineto{\pgfqpoint{6.695237in}{3.936726in}}%
\pgfpathlineto{\pgfqpoint{6.509011in}{3.936726in}}%
\pgfpathlineto{\pgfqpoint{6.509011in}{3.854998in}}%
\pgfusepath{}%
\end{pgfscope}%
\begin{pgfscope}%
\pgfpathrectangle{\pgfqpoint{0.549740in}{0.463273in}}{\pgfqpoint{9.320225in}{4.495057in}}%
\pgfusepath{clip}%
\pgfsetbuttcap%
\pgfsetroundjoin%
\pgfsetlinewidth{0.000000pt}%
\definecolor{currentstroke}{rgb}{0.000000,0.000000,0.000000}%
\pgfsetstrokecolor{currentstroke}%
\pgfsetdash{}{0pt}%
\pgfpathmoveto{\pgfqpoint{6.695237in}{3.854998in}}%
\pgfpathlineto{\pgfqpoint{6.881464in}{3.854998in}}%
\pgfpathlineto{\pgfqpoint{6.881464in}{3.936726in}}%
\pgfpathlineto{\pgfqpoint{6.695237in}{3.936726in}}%
\pgfpathlineto{\pgfqpoint{6.695237in}{3.854998in}}%
\pgfusepath{}%
\end{pgfscope}%
\begin{pgfscope}%
\pgfpathrectangle{\pgfqpoint{0.549740in}{0.463273in}}{\pgfqpoint{9.320225in}{4.495057in}}%
\pgfusepath{clip}%
\pgfsetbuttcap%
\pgfsetroundjoin%
\definecolor{currentfill}{rgb}{0.614330,0.761948,0.940009}%
\pgfsetfillcolor{currentfill}%
\pgfsetlinewidth{0.000000pt}%
\definecolor{currentstroke}{rgb}{0.000000,0.000000,0.000000}%
\pgfsetstrokecolor{currentstroke}%
\pgfsetdash{}{0pt}%
\pgfpathmoveto{\pgfqpoint{6.881464in}{3.854998in}}%
\pgfpathlineto{\pgfqpoint{7.067690in}{3.854998in}}%
\pgfpathlineto{\pgfqpoint{7.067690in}{3.936726in}}%
\pgfpathlineto{\pgfqpoint{6.881464in}{3.936726in}}%
\pgfpathlineto{\pgfqpoint{6.881464in}{3.854998in}}%
\pgfusepath{fill}%
\end{pgfscope}%
\begin{pgfscope}%
\pgfpathrectangle{\pgfqpoint{0.549740in}{0.463273in}}{\pgfqpoint{9.320225in}{4.495057in}}%
\pgfusepath{clip}%
\pgfsetbuttcap%
\pgfsetroundjoin%
\pgfsetlinewidth{0.000000pt}%
\definecolor{currentstroke}{rgb}{0.000000,0.000000,0.000000}%
\pgfsetstrokecolor{currentstroke}%
\pgfsetdash{}{0pt}%
\pgfpathmoveto{\pgfqpoint{7.067690in}{3.854998in}}%
\pgfpathlineto{\pgfqpoint{7.253917in}{3.854998in}}%
\pgfpathlineto{\pgfqpoint{7.253917in}{3.936726in}}%
\pgfpathlineto{\pgfqpoint{7.067690in}{3.936726in}}%
\pgfpathlineto{\pgfqpoint{7.067690in}{3.854998in}}%
\pgfusepath{}%
\end{pgfscope}%
\begin{pgfscope}%
\pgfpathrectangle{\pgfqpoint{0.549740in}{0.463273in}}{\pgfqpoint{9.320225in}{4.495057in}}%
\pgfusepath{clip}%
\pgfsetbuttcap%
\pgfsetroundjoin%
\pgfsetlinewidth{0.000000pt}%
\definecolor{currentstroke}{rgb}{0.000000,0.000000,0.000000}%
\pgfsetstrokecolor{currentstroke}%
\pgfsetdash{}{0pt}%
\pgfpathmoveto{\pgfqpoint{7.253917in}{3.854998in}}%
\pgfpathlineto{\pgfqpoint{7.440143in}{3.854998in}}%
\pgfpathlineto{\pgfqpoint{7.440143in}{3.936726in}}%
\pgfpathlineto{\pgfqpoint{7.253917in}{3.936726in}}%
\pgfpathlineto{\pgfqpoint{7.253917in}{3.854998in}}%
\pgfusepath{}%
\end{pgfscope}%
\begin{pgfscope}%
\pgfpathrectangle{\pgfqpoint{0.549740in}{0.463273in}}{\pgfqpoint{9.320225in}{4.495057in}}%
\pgfusepath{clip}%
\pgfsetbuttcap%
\pgfsetroundjoin%
\pgfsetlinewidth{0.000000pt}%
\definecolor{currentstroke}{rgb}{0.000000,0.000000,0.000000}%
\pgfsetstrokecolor{currentstroke}%
\pgfsetdash{}{0pt}%
\pgfpathmoveto{\pgfqpoint{7.440143in}{3.854998in}}%
\pgfpathlineto{\pgfqpoint{7.626370in}{3.854998in}}%
\pgfpathlineto{\pgfqpoint{7.626370in}{3.936726in}}%
\pgfpathlineto{\pgfqpoint{7.440143in}{3.936726in}}%
\pgfpathlineto{\pgfqpoint{7.440143in}{3.854998in}}%
\pgfusepath{}%
\end{pgfscope}%
\begin{pgfscope}%
\pgfpathrectangle{\pgfqpoint{0.549740in}{0.463273in}}{\pgfqpoint{9.320225in}{4.495057in}}%
\pgfusepath{clip}%
\pgfsetbuttcap%
\pgfsetroundjoin%
\pgfsetlinewidth{0.000000pt}%
\definecolor{currentstroke}{rgb}{0.000000,0.000000,0.000000}%
\pgfsetstrokecolor{currentstroke}%
\pgfsetdash{}{0pt}%
\pgfpathmoveto{\pgfqpoint{7.626370in}{3.854998in}}%
\pgfpathlineto{\pgfqpoint{7.812596in}{3.854998in}}%
\pgfpathlineto{\pgfqpoint{7.812596in}{3.936726in}}%
\pgfpathlineto{\pgfqpoint{7.626370in}{3.936726in}}%
\pgfpathlineto{\pgfqpoint{7.626370in}{3.854998in}}%
\pgfusepath{}%
\end{pgfscope}%
\begin{pgfscope}%
\pgfpathrectangle{\pgfqpoint{0.549740in}{0.463273in}}{\pgfqpoint{9.320225in}{4.495057in}}%
\pgfusepath{clip}%
\pgfsetbuttcap%
\pgfsetroundjoin%
\definecolor{currentfill}{rgb}{0.472869,0.711325,0.955316}%
\pgfsetfillcolor{currentfill}%
\pgfsetlinewidth{0.000000pt}%
\definecolor{currentstroke}{rgb}{0.000000,0.000000,0.000000}%
\pgfsetstrokecolor{currentstroke}%
\pgfsetdash{}{0pt}%
\pgfpathmoveto{\pgfqpoint{7.812596in}{3.854998in}}%
\pgfpathlineto{\pgfqpoint{7.998823in}{3.854998in}}%
\pgfpathlineto{\pgfqpoint{7.998823in}{3.936726in}}%
\pgfpathlineto{\pgfqpoint{7.812596in}{3.936726in}}%
\pgfpathlineto{\pgfqpoint{7.812596in}{3.854998in}}%
\pgfusepath{fill}%
\end{pgfscope}%
\begin{pgfscope}%
\pgfpathrectangle{\pgfqpoint{0.549740in}{0.463273in}}{\pgfqpoint{9.320225in}{4.495057in}}%
\pgfusepath{clip}%
\pgfsetbuttcap%
\pgfsetroundjoin%
\pgfsetlinewidth{0.000000pt}%
\definecolor{currentstroke}{rgb}{0.000000,0.000000,0.000000}%
\pgfsetstrokecolor{currentstroke}%
\pgfsetdash{}{0pt}%
\pgfpathmoveto{\pgfqpoint{7.998823in}{3.854998in}}%
\pgfpathlineto{\pgfqpoint{8.185049in}{3.854998in}}%
\pgfpathlineto{\pgfqpoint{8.185049in}{3.936726in}}%
\pgfpathlineto{\pgfqpoint{7.998823in}{3.936726in}}%
\pgfpathlineto{\pgfqpoint{7.998823in}{3.854998in}}%
\pgfusepath{}%
\end{pgfscope}%
\begin{pgfscope}%
\pgfpathrectangle{\pgfqpoint{0.549740in}{0.463273in}}{\pgfqpoint{9.320225in}{4.495057in}}%
\pgfusepath{clip}%
\pgfsetbuttcap%
\pgfsetroundjoin%
\pgfsetlinewidth{0.000000pt}%
\definecolor{currentstroke}{rgb}{0.000000,0.000000,0.000000}%
\pgfsetstrokecolor{currentstroke}%
\pgfsetdash{}{0pt}%
\pgfpathmoveto{\pgfqpoint{8.185049in}{3.854998in}}%
\pgfpathlineto{\pgfqpoint{8.371276in}{3.854998in}}%
\pgfpathlineto{\pgfqpoint{8.371276in}{3.936726in}}%
\pgfpathlineto{\pgfqpoint{8.185049in}{3.936726in}}%
\pgfpathlineto{\pgfqpoint{8.185049in}{3.854998in}}%
\pgfusepath{}%
\end{pgfscope}%
\begin{pgfscope}%
\pgfpathrectangle{\pgfqpoint{0.549740in}{0.463273in}}{\pgfqpoint{9.320225in}{4.495057in}}%
\pgfusepath{clip}%
\pgfsetbuttcap%
\pgfsetroundjoin%
\pgfsetlinewidth{0.000000pt}%
\definecolor{currentstroke}{rgb}{0.000000,0.000000,0.000000}%
\pgfsetstrokecolor{currentstroke}%
\pgfsetdash{}{0pt}%
\pgfpathmoveto{\pgfqpoint{8.371276in}{3.854998in}}%
\pgfpathlineto{\pgfqpoint{8.557503in}{3.854998in}}%
\pgfpathlineto{\pgfqpoint{8.557503in}{3.936726in}}%
\pgfpathlineto{\pgfqpoint{8.371276in}{3.936726in}}%
\pgfpathlineto{\pgfqpoint{8.371276in}{3.854998in}}%
\pgfusepath{}%
\end{pgfscope}%
\begin{pgfscope}%
\pgfpathrectangle{\pgfqpoint{0.549740in}{0.463273in}}{\pgfqpoint{9.320225in}{4.495057in}}%
\pgfusepath{clip}%
\pgfsetbuttcap%
\pgfsetroundjoin%
\pgfsetlinewidth{0.000000pt}%
\definecolor{currentstroke}{rgb}{0.000000,0.000000,0.000000}%
\pgfsetstrokecolor{currentstroke}%
\pgfsetdash{}{0pt}%
\pgfpathmoveto{\pgfqpoint{8.557503in}{3.854998in}}%
\pgfpathlineto{\pgfqpoint{8.743729in}{3.854998in}}%
\pgfpathlineto{\pgfqpoint{8.743729in}{3.936726in}}%
\pgfpathlineto{\pgfqpoint{8.557503in}{3.936726in}}%
\pgfpathlineto{\pgfqpoint{8.557503in}{3.854998in}}%
\pgfusepath{}%
\end{pgfscope}%
\begin{pgfscope}%
\pgfpathrectangle{\pgfqpoint{0.549740in}{0.463273in}}{\pgfqpoint{9.320225in}{4.495057in}}%
\pgfusepath{clip}%
\pgfsetbuttcap%
\pgfsetroundjoin%
\pgfsetlinewidth{0.000000pt}%
\definecolor{currentstroke}{rgb}{0.000000,0.000000,0.000000}%
\pgfsetstrokecolor{currentstroke}%
\pgfsetdash{}{0pt}%
\pgfpathmoveto{\pgfqpoint{8.743729in}{3.854998in}}%
\pgfpathlineto{\pgfqpoint{8.929956in}{3.854998in}}%
\pgfpathlineto{\pgfqpoint{8.929956in}{3.936726in}}%
\pgfpathlineto{\pgfqpoint{8.743729in}{3.936726in}}%
\pgfpathlineto{\pgfqpoint{8.743729in}{3.854998in}}%
\pgfusepath{}%
\end{pgfscope}%
\begin{pgfscope}%
\pgfpathrectangle{\pgfqpoint{0.549740in}{0.463273in}}{\pgfqpoint{9.320225in}{4.495057in}}%
\pgfusepath{clip}%
\pgfsetbuttcap%
\pgfsetroundjoin%
\pgfsetlinewidth{0.000000pt}%
\definecolor{currentstroke}{rgb}{0.000000,0.000000,0.000000}%
\pgfsetstrokecolor{currentstroke}%
\pgfsetdash{}{0pt}%
\pgfpathmoveto{\pgfqpoint{8.929956in}{3.854998in}}%
\pgfpathlineto{\pgfqpoint{9.116182in}{3.854998in}}%
\pgfpathlineto{\pgfqpoint{9.116182in}{3.936726in}}%
\pgfpathlineto{\pgfqpoint{8.929956in}{3.936726in}}%
\pgfpathlineto{\pgfqpoint{8.929956in}{3.854998in}}%
\pgfusepath{}%
\end{pgfscope}%
\begin{pgfscope}%
\pgfpathrectangle{\pgfqpoint{0.549740in}{0.463273in}}{\pgfqpoint{9.320225in}{4.495057in}}%
\pgfusepath{clip}%
\pgfsetbuttcap%
\pgfsetroundjoin%
\definecolor{currentfill}{rgb}{0.472869,0.711325,0.955316}%
\pgfsetfillcolor{currentfill}%
\pgfsetlinewidth{0.000000pt}%
\definecolor{currentstroke}{rgb}{0.000000,0.000000,0.000000}%
\pgfsetstrokecolor{currentstroke}%
\pgfsetdash{}{0pt}%
\pgfpathmoveto{\pgfqpoint{9.116182in}{3.854998in}}%
\pgfpathlineto{\pgfqpoint{9.302409in}{3.854998in}}%
\pgfpathlineto{\pgfqpoint{9.302409in}{3.936726in}}%
\pgfpathlineto{\pgfqpoint{9.116182in}{3.936726in}}%
\pgfpathlineto{\pgfqpoint{9.116182in}{3.854998in}}%
\pgfusepath{fill}%
\end{pgfscope}%
\begin{pgfscope}%
\pgfpathrectangle{\pgfqpoint{0.549740in}{0.463273in}}{\pgfqpoint{9.320225in}{4.495057in}}%
\pgfusepath{clip}%
\pgfsetbuttcap%
\pgfsetroundjoin%
\pgfsetlinewidth{0.000000pt}%
\definecolor{currentstroke}{rgb}{0.000000,0.000000,0.000000}%
\pgfsetstrokecolor{currentstroke}%
\pgfsetdash{}{0pt}%
\pgfpathmoveto{\pgfqpoint{9.302409in}{3.854998in}}%
\pgfpathlineto{\pgfqpoint{9.488635in}{3.854998in}}%
\pgfpathlineto{\pgfqpoint{9.488635in}{3.936726in}}%
\pgfpathlineto{\pgfqpoint{9.302409in}{3.936726in}}%
\pgfpathlineto{\pgfqpoint{9.302409in}{3.854998in}}%
\pgfusepath{}%
\end{pgfscope}%
\begin{pgfscope}%
\pgfpathrectangle{\pgfqpoint{0.549740in}{0.463273in}}{\pgfqpoint{9.320225in}{4.495057in}}%
\pgfusepath{clip}%
\pgfsetbuttcap%
\pgfsetroundjoin%
\pgfsetlinewidth{0.000000pt}%
\definecolor{currentstroke}{rgb}{0.000000,0.000000,0.000000}%
\pgfsetstrokecolor{currentstroke}%
\pgfsetdash{}{0pt}%
\pgfpathmoveto{\pgfqpoint{9.488635in}{3.854998in}}%
\pgfpathlineto{\pgfqpoint{9.674862in}{3.854998in}}%
\pgfpathlineto{\pgfqpoint{9.674862in}{3.936726in}}%
\pgfpathlineto{\pgfqpoint{9.488635in}{3.936726in}}%
\pgfpathlineto{\pgfqpoint{9.488635in}{3.854998in}}%
\pgfusepath{}%
\end{pgfscope}%
\begin{pgfscope}%
\pgfpathrectangle{\pgfqpoint{0.549740in}{0.463273in}}{\pgfqpoint{9.320225in}{4.495057in}}%
\pgfusepath{clip}%
\pgfsetbuttcap%
\pgfsetroundjoin%
\pgfsetlinewidth{0.000000pt}%
\definecolor{currentstroke}{rgb}{0.000000,0.000000,0.000000}%
\pgfsetstrokecolor{currentstroke}%
\pgfsetdash{}{0pt}%
\pgfpathmoveto{\pgfqpoint{9.674862in}{3.854998in}}%
\pgfpathlineto{\pgfqpoint{9.861088in}{3.854998in}}%
\pgfpathlineto{\pgfqpoint{9.861088in}{3.936726in}}%
\pgfpathlineto{\pgfqpoint{9.674862in}{3.936726in}}%
\pgfpathlineto{\pgfqpoint{9.674862in}{3.854998in}}%
\pgfusepath{}%
\end{pgfscope}%
\begin{pgfscope}%
\pgfpathrectangle{\pgfqpoint{0.549740in}{0.463273in}}{\pgfqpoint{9.320225in}{4.495057in}}%
\pgfusepath{clip}%
\pgfsetbuttcap%
\pgfsetroundjoin%
\pgfsetlinewidth{0.000000pt}%
\definecolor{currentstroke}{rgb}{0.000000,0.000000,0.000000}%
\pgfsetstrokecolor{currentstroke}%
\pgfsetdash{}{0pt}%
\pgfpathmoveto{\pgfqpoint{0.549761in}{3.936726in}}%
\pgfpathlineto{\pgfqpoint{0.735988in}{3.936726in}}%
\pgfpathlineto{\pgfqpoint{0.735988in}{4.018455in}}%
\pgfpathlineto{\pgfqpoint{0.549761in}{4.018455in}}%
\pgfpathlineto{\pgfqpoint{0.549761in}{3.936726in}}%
\pgfusepath{}%
\end{pgfscope}%
\begin{pgfscope}%
\pgfpathrectangle{\pgfqpoint{0.549740in}{0.463273in}}{\pgfqpoint{9.320225in}{4.495057in}}%
\pgfusepath{clip}%
\pgfsetbuttcap%
\pgfsetroundjoin%
\pgfsetlinewidth{0.000000pt}%
\definecolor{currentstroke}{rgb}{0.000000,0.000000,0.000000}%
\pgfsetstrokecolor{currentstroke}%
\pgfsetdash{}{0pt}%
\pgfpathmoveto{\pgfqpoint{0.735988in}{3.936726in}}%
\pgfpathlineto{\pgfqpoint{0.922214in}{3.936726in}}%
\pgfpathlineto{\pgfqpoint{0.922214in}{4.018455in}}%
\pgfpathlineto{\pgfqpoint{0.735988in}{4.018455in}}%
\pgfpathlineto{\pgfqpoint{0.735988in}{3.936726in}}%
\pgfusepath{}%
\end{pgfscope}%
\begin{pgfscope}%
\pgfpathrectangle{\pgfqpoint{0.549740in}{0.463273in}}{\pgfqpoint{9.320225in}{4.495057in}}%
\pgfusepath{clip}%
\pgfsetbuttcap%
\pgfsetroundjoin%
\pgfsetlinewidth{0.000000pt}%
\definecolor{currentstroke}{rgb}{0.000000,0.000000,0.000000}%
\pgfsetstrokecolor{currentstroke}%
\pgfsetdash{}{0pt}%
\pgfpathmoveto{\pgfqpoint{0.922214in}{3.936726in}}%
\pgfpathlineto{\pgfqpoint{1.108441in}{3.936726in}}%
\pgfpathlineto{\pgfqpoint{1.108441in}{4.018455in}}%
\pgfpathlineto{\pgfqpoint{0.922214in}{4.018455in}}%
\pgfpathlineto{\pgfqpoint{0.922214in}{3.936726in}}%
\pgfusepath{}%
\end{pgfscope}%
\begin{pgfscope}%
\pgfpathrectangle{\pgfqpoint{0.549740in}{0.463273in}}{\pgfqpoint{9.320225in}{4.495057in}}%
\pgfusepath{clip}%
\pgfsetbuttcap%
\pgfsetroundjoin%
\pgfsetlinewidth{0.000000pt}%
\definecolor{currentstroke}{rgb}{0.000000,0.000000,0.000000}%
\pgfsetstrokecolor{currentstroke}%
\pgfsetdash{}{0pt}%
\pgfpathmoveto{\pgfqpoint{1.108441in}{3.936726in}}%
\pgfpathlineto{\pgfqpoint{1.294667in}{3.936726in}}%
\pgfpathlineto{\pgfqpoint{1.294667in}{4.018455in}}%
\pgfpathlineto{\pgfqpoint{1.108441in}{4.018455in}}%
\pgfpathlineto{\pgfqpoint{1.108441in}{3.936726in}}%
\pgfusepath{}%
\end{pgfscope}%
\begin{pgfscope}%
\pgfpathrectangle{\pgfqpoint{0.549740in}{0.463273in}}{\pgfqpoint{9.320225in}{4.495057in}}%
\pgfusepath{clip}%
\pgfsetbuttcap%
\pgfsetroundjoin%
\pgfsetlinewidth{0.000000pt}%
\definecolor{currentstroke}{rgb}{0.000000,0.000000,0.000000}%
\pgfsetstrokecolor{currentstroke}%
\pgfsetdash{}{0pt}%
\pgfpathmoveto{\pgfqpoint{1.294667in}{3.936726in}}%
\pgfpathlineto{\pgfqpoint{1.480894in}{3.936726in}}%
\pgfpathlineto{\pgfqpoint{1.480894in}{4.018455in}}%
\pgfpathlineto{\pgfqpoint{1.294667in}{4.018455in}}%
\pgfpathlineto{\pgfqpoint{1.294667in}{3.936726in}}%
\pgfusepath{}%
\end{pgfscope}%
\begin{pgfscope}%
\pgfpathrectangle{\pgfqpoint{0.549740in}{0.463273in}}{\pgfqpoint{9.320225in}{4.495057in}}%
\pgfusepath{clip}%
\pgfsetbuttcap%
\pgfsetroundjoin%
\pgfsetlinewidth{0.000000pt}%
\definecolor{currentstroke}{rgb}{0.000000,0.000000,0.000000}%
\pgfsetstrokecolor{currentstroke}%
\pgfsetdash{}{0pt}%
\pgfpathmoveto{\pgfqpoint{1.480894in}{3.936726in}}%
\pgfpathlineto{\pgfqpoint{1.667120in}{3.936726in}}%
\pgfpathlineto{\pgfqpoint{1.667120in}{4.018455in}}%
\pgfpathlineto{\pgfqpoint{1.480894in}{4.018455in}}%
\pgfpathlineto{\pgfqpoint{1.480894in}{3.936726in}}%
\pgfusepath{}%
\end{pgfscope}%
\begin{pgfscope}%
\pgfpathrectangle{\pgfqpoint{0.549740in}{0.463273in}}{\pgfqpoint{9.320225in}{4.495057in}}%
\pgfusepath{clip}%
\pgfsetbuttcap%
\pgfsetroundjoin%
\pgfsetlinewidth{0.000000pt}%
\definecolor{currentstroke}{rgb}{0.000000,0.000000,0.000000}%
\pgfsetstrokecolor{currentstroke}%
\pgfsetdash{}{0pt}%
\pgfpathmoveto{\pgfqpoint{1.667120in}{3.936726in}}%
\pgfpathlineto{\pgfqpoint{1.853347in}{3.936726in}}%
\pgfpathlineto{\pgfqpoint{1.853347in}{4.018455in}}%
\pgfpathlineto{\pgfqpoint{1.667120in}{4.018455in}}%
\pgfpathlineto{\pgfqpoint{1.667120in}{3.936726in}}%
\pgfusepath{}%
\end{pgfscope}%
\begin{pgfscope}%
\pgfpathrectangle{\pgfqpoint{0.549740in}{0.463273in}}{\pgfqpoint{9.320225in}{4.495057in}}%
\pgfusepath{clip}%
\pgfsetbuttcap%
\pgfsetroundjoin%
\pgfsetlinewidth{0.000000pt}%
\definecolor{currentstroke}{rgb}{0.000000,0.000000,0.000000}%
\pgfsetstrokecolor{currentstroke}%
\pgfsetdash{}{0pt}%
\pgfpathmoveto{\pgfqpoint{1.853347in}{3.936726in}}%
\pgfpathlineto{\pgfqpoint{2.039573in}{3.936726in}}%
\pgfpathlineto{\pgfqpoint{2.039573in}{4.018455in}}%
\pgfpathlineto{\pgfqpoint{1.853347in}{4.018455in}}%
\pgfpathlineto{\pgfqpoint{1.853347in}{3.936726in}}%
\pgfusepath{}%
\end{pgfscope}%
\begin{pgfscope}%
\pgfpathrectangle{\pgfqpoint{0.549740in}{0.463273in}}{\pgfqpoint{9.320225in}{4.495057in}}%
\pgfusepath{clip}%
\pgfsetbuttcap%
\pgfsetroundjoin%
\pgfsetlinewidth{0.000000pt}%
\definecolor{currentstroke}{rgb}{0.000000,0.000000,0.000000}%
\pgfsetstrokecolor{currentstroke}%
\pgfsetdash{}{0pt}%
\pgfpathmoveto{\pgfqpoint{2.039573in}{3.936726in}}%
\pgfpathlineto{\pgfqpoint{2.225800in}{3.936726in}}%
\pgfpathlineto{\pgfqpoint{2.225800in}{4.018455in}}%
\pgfpathlineto{\pgfqpoint{2.039573in}{4.018455in}}%
\pgfpathlineto{\pgfqpoint{2.039573in}{3.936726in}}%
\pgfusepath{}%
\end{pgfscope}%
\begin{pgfscope}%
\pgfpathrectangle{\pgfqpoint{0.549740in}{0.463273in}}{\pgfqpoint{9.320225in}{4.495057in}}%
\pgfusepath{clip}%
\pgfsetbuttcap%
\pgfsetroundjoin%
\pgfsetlinewidth{0.000000pt}%
\definecolor{currentstroke}{rgb}{0.000000,0.000000,0.000000}%
\pgfsetstrokecolor{currentstroke}%
\pgfsetdash{}{0pt}%
\pgfpathmoveto{\pgfqpoint{2.225800in}{3.936726in}}%
\pgfpathlineto{\pgfqpoint{2.412027in}{3.936726in}}%
\pgfpathlineto{\pgfqpoint{2.412027in}{4.018455in}}%
\pgfpathlineto{\pgfqpoint{2.225800in}{4.018455in}}%
\pgfpathlineto{\pgfqpoint{2.225800in}{3.936726in}}%
\pgfusepath{}%
\end{pgfscope}%
\begin{pgfscope}%
\pgfpathrectangle{\pgfqpoint{0.549740in}{0.463273in}}{\pgfqpoint{9.320225in}{4.495057in}}%
\pgfusepath{clip}%
\pgfsetbuttcap%
\pgfsetroundjoin%
\pgfsetlinewidth{0.000000pt}%
\definecolor{currentstroke}{rgb}{0.000000,0.000000,0.000000}%
\pgfsetstrokecolor{currentstroke}%
\pgfsetdash{}{0pt}%
\pgfpathmoveto{\pgfqpoint{2.412027in}{3.936726in}}%
\pgfpathlineto{\pgfqpoint{2.598253in}{3.936726in}}%
\pgfpathlineto{\pgfqpoint{2.598253in}{4.018455in}}%
\pgfpathlineto{\pgfqpoint{2.412027in}{4.018455in}}%
\pgfpathlineto{\pgfqpoint{2.412027in}{3.936726in}}%
\pgfusepath{}%
\end{pgfscope}%
\begin{pgfscope}%
\pgfpathrectangle{\pgfqpoint{0.549740in}{0.463273in}}{\pgfqpoint{9.320225in}{4.495057in}}%
\pgfusepath{clip}%
\pgfsetbuttcap%
\pgfsetroundjoin%
\pgfsetlinewidth{0.000000pt}%
\definecolor{currentstroke}{rgb}{0.000000,0.000000,0.000000}%
\pgfsetstrokecolor{currentstroke}%
\pgfsetdash{}{0pt}%
\pgfpathmoveto{\pgfqpoint{2.598253in}{3.936726in}}%
\pgfpathlineto{\pgfqpoint{2.784480in}{3.936726in}}%
\pgfpathlineto{\pgfqpoint{2.784480in}{4.018455in}}%
\pgfpathlineto{\pgfqpoint{2.598253in}{4.018455in}}%
\pgfpathlineto{\pgfqpoint{2.598253in}{3.936726in}}%
\pgfusepath{}%
\end{pgfscope}%
\begin{pgfscope}%
\pgfpathrectangle{\pgfqpoint{0.549740in}{0.463273in}}{\pgfqpoint{9.320225in}{4.495057in}}%
\pgfusepath{clip}%
\pgfsetbuttcap%
\pgfsetroundjoin%
\pgfsetlinewidth{0.000000pt}%
\definecolor{currentstroke}{rgb}{0.000000,0.000000,0.000000}%
\pgfsetstrokecolor{currentstroke}%
\pgfsetdash{}{0pt}%
\pgfpathmoveto{\pgfqpoint{2.784480in}{3.936726in}}%
\pgfpathlineto{\pgfqpoint{2.970706in}{3.936726in}}%
\pgfpathlineto{\pgfqpoint{2.970706in}{4.018455in}}%
\pgfpathlineto{\pgfqpoint{2.784480in}{4.018455in}}%
\pgfpathlineto{\pgfqpoint{2.784480in}{3.936726in}}%
\pgfusepath{}%
\end{pgfscope}%
\begin{pgfscope}%
\pgfpathrectangle{\pgfqpoint{0.549740in}{0.463273in}}{\pgfqpoint{9.320225in}{4.495057in}}%
\pgfusepath{clip}%
\pgfsetbuttcap%
\pgfsetroundjoin%
\pgfsetlinewidth{0.000000pt}%
\definecolor{currentstroke}{rgb}{0.000000,0.000000,0.000000}%
\pgfsetstrokecolor{currentstroke}%
\pgfsetdash{}{0pt}%
\pgfpathmoveto{\pgfqpoint{2.970706in}{3.936726in}}%
\pgfpathlineto{\pgfqpoint{3.156933in}{3.936726in}}%
\pgfpathlineto{\pgfqpoint{3.156933in}{4.018455in}}%
\pgfpathlineto{\pgfqpoint{2.970706in}{4.018455in}}%
\pgfpathlineto{\pgfqpoint{2.970706in}{3.936726in}}%
\pgfusepath{}%
\end{pgfscope}%
\begin{pgfscope}%
\pgfpathrectangle{\pgfqpoint{0.549740in}{0.463273in}}{\pgfqpoint{9.320225in}{4.495057in}}%
\pgfusepath{clip}%
\pgfsetbuttcap%
\pgfsetroundjoin%
\pgfsetlinewidth{0.000000pt}%
\definecolor{currentstroke}{rgb}{0.000000,0.000000,0.000000}%
\pgfsetstrokecolor{currentstroke}%
\pgfsetdash{}{0pt}%
\pgfpathmoveto{\pgfqpoint{3.156933in}{3.936726in}}%
\pgfpathlineto{\pgfqpoint{3.343159in}{3.936726in}}%
\pgfpathlineto{\pgfqpoint{3.343159in}{4.018455in}}%
\pgfpathlineto{\pgfqpoint{3.156933in}{4.018455in}}%
\pgfpathlineto{\pgfqpoint{3.156933in}{3.936726in}}%
\pgfusepath{}%
\end{pgfscope}%
\begin{pgfscope}%
\pgfpathrectangle{\pgfqpoint{0.549740in}{0.463273in}}{\pgfqpoint{9.320225in}{4.495057in}}%
\pgfusepath{clip}%
\pgfsetbuttcap%
\pgfsetroundjoin%
\pgfsetlinewidth{0.000000pt}%
\definecolor{currentstroke}{rgb}{0.000000,0.000000,0.000000}%
\pgfsetstrokecolor{currentstroke}%
\pgfsetdash{}{0pt}%
\pgfpathmoveto{\pgfqpoint{3.343159in}{3.936726in}}%
\pgfpathlineto{\pgfqpoint{3.529386in}{3.936726in}}%
\pgfpathlineto{\pgfqpoint{3.529386in}{4.018455in}}%
\pgfpathlineto{\pgfqpoint{3.343159in}{4.018455in}}%
\pgfpathlineto{\pgfqpoint{3.343159in}{3.936726in}}%
\pgfusepath{}%
\end{pgfscope}%
\begin{pgfscope}%
\pgfpathrectangle{\pgfqpoint{0.549740in}{0.463273in}}{\pgfqpoint{9.320225in}{4.495057in}}%
\pgfusepath{clip}%
\pgfsetbuttcap%
\pgfsetroundjoin%
\pgfsetlinewidth{0.000000pt}%
\definecolor{currentstroke}{rgb}{0.000000,0.000000,0.000000}%
\pgfsetstrokecolor{currentstroke}%
\pgfsetdash{}{0pt}%
\pgfpathmoveto{\pgfqpoint{3.529386in}{3.936726in}}%
\pgfpathlineto{\pgfqpoint{3.715612in}{3.936726in}}%
\pgfpathlineto{\pgfqpoint{3.715612in}{4.018455in}}%
\pgfpathlineto{\pgfqpoint{3.529386in}{4.018455in}}%
\pgfpathlineto{\pgfqpoint{3.529386in}{3.936726in}}%
\pgfusepath{}%
\end{pgfscope}%
\begin{pgfscope}%
\pgfpathrectangle{\pgfqpoint{0.549740in}{0.463273in}}{\pgfqpoint{9.320225in}{4.495057in}}%
\pgfusepath{clip}%
\pgfsetbuttcap%
\pgfsetroundjoin%
\pgfsetlinewidth{0.000000pt}%
\definecolor{currentstroke}{rgb}{0.000000,0.000000,0.000000}%
\pgfsetstrokecolor{currentstroke}%
\pgfsetdash{}{0pt}%
\pgfpathmoveto{\pgfqpoint{3.715612in}{3.936726in}}%
\pgfpathlineto{\pgfqpoint{3.901839in}{3.936726in}}%
\pgfpathlineto{\pgfqpoint{3.901839in}{4.018455in}}%
\pgfpathlineto{\pgfqpoint{3.715612in}{4.018455in}}%
\pgfpathlineto{\pgfqpoint{3.715612in}{3.936726in}}%
\pgfusepath{}%
\end{pgfscope}%
\begin{pgfscope}%
\pgfpathrectangle{\pgfqpoint{0.549740in}{0.463273in}}{\pgfqpoint{9.320225in}{4.495057in}}%
\pgfusepath{clip}%
\pgfsetbuttcap%
\pgfsetroundjoin%
\pgfsetlinewidth{0.000000pt}%
\definecolor{currentstroke}{rgb}{0.000000,0.000000,0.000000}%
\pgfsetstrokecolor{currentstroke}%
\pgfsetdash{}{0pt}%
\pgfpathmoveto{\pgfqpoint{3.901839in}{3.936726in}}%
\pgfpathlineto{\pgfqpoint{4.088065in}{3.936726in}}%
\pgfpathlineto{\pgfqpoint{4.088065in}{4.018455in}}%
\pgfpathlineto{\pgfqpoint{3.901839in}{4.018455in}}%
\pgfpathlineto{\pgfqpoint{3.901839in}{3.936726in}}%
\pgfusepath{}%
\end{pgfscope}%
\begin{pgfscope}%
\pgfpathrectangle{\pgfqpoint{0.549740in}{0.463273in}}{\pgfqpoint{9.320225in}{4.495057in}}%
\pgfusepath{clip}%
\pgfsetbuttcap%
\pgfsetroundjoin%
\pgfsetlinewidth{0.000000pt}%
\definecolor{currentstroke}{rgb}{0.000000,0.000000,0.000000}%
\pgfsetstrokecolor{currentstroke}%
\pgfsetdash{}{0pt}%
\pgfpathmoveto{\pgfqpoint{4.088065in}{3.936726in}}%
\pgfpathlineto{\pgfqpoint{4.274292in}{3.936726in}}%
\pgfpathlineto{\pgfqpoint{4.274292in}{4.018455in}}%
\pgfpathlineto{\pgfqpoint{4.088065in}{4.018455in}}%
\pgfpathlineto{\pgfqpoint{4.088065in}{3.936726in}}%
\pgfusepath{}%
\end{pgfscope}%
\begin{pgfscope}%
\pgfpathrectangle{\pgfqpoint{0.549740in}{0.463273in}}{\pgfqpoint{9.320225in}{4.495057in}}%
\pgfusepath{clip}%
\pgfsetbuttcap%
\pgfsetroundjoin%
\pgfsetlinewidth{0.000000pt}%
\definecolor{currentstroke}{rgb}{0.000000,0.000000,0.000000}%
\pgfsetstrokecolor{currentstroke}%
\pgfsetdash{}{0pt}%
\pgfpathmoveto{\pgfqpoint{4.274292in}{3.936726in}}%
\pgfpathlineto{\pgfqpoint{4.460519in}{3.936726in}}%
\pgfpathlineto{\pgfqpoint{4.460519in}{4.018455in}}%
\pgfpathlineto{\pgfqpoint{4.274292in}{4.018455in}}%
\pgfpathlineto{\pgfqpoint{4.274292in}{3.936726in}}%
\pgfusepath{}%
\end{pgfscope}%
\begin{pgfscope}%
\pgfpathrectangle{\pgfqpoint{0.549740in}{0.463273in}}{\pgfqpoint{9.320225in}{4.495057in}}%
\pgfusepath{clip}%
\pgfsetbuttcap%
\pgfsetroundjoin%
\pgfsetlinewidth{0.000000pt}%
\definecolor{currentstroke}{rgb}{0.000000,0.000000,0.000000}%
\pgfsetstrokecolor{currentstroke}%
\pgfsetdash{}{0pt}%
\pgfpathmoveto{\pgfqpoint{4.460519in}{3.936726in}}%
\pgfpathlineto{\pgfqpoint{4.646745in}{3.936726in}}%
\pgfpathlineto{\pgfqpoint{4.646745in}{4.018455in}}%
\pgfpathlineto{\pgfqpoint{4.460519in}{4.018455in}}%
\pgfpathlineto{\pgfqpoint{4.460519in}{3.936726in}}%
\pgfusepath{}%
\end{pgfscope}%
\begin{pgfscope}%
\pgfpathrectangle{\pgfqpoint{0.549740in}{0.463273in}}{\pgfqpoint{9.320225in}{4.495057in}}%
\pgfusepath{clip}%
\pgfsetbuttcap%
\pgfsetroundjoin%
\pgfsetlinewidth{0.000000pt}%
\definecolor{currentstroke}{rgb}{0.000000,0.000000,0.000000}%
\pgfsetstrokecolor{currentstroke}%
\pgfsetdash{}{0pt}%
\pgfpathmoveto{\pgfqpoint{4.646745in}{3.936726in}}%
\pgfpathlineto{\pgfqpoint{4.832972in}{3.936726in}}%
\pgfpathlineto{\pgfqpoint{4.832972in}{4.018455in}}%
\pgfpathlineto{\pgfqpoint{4.646745in}{4.018455in}}%
\pgfpathlineto{\pgfqpoint{4.646745in}{3.936726in}}%
\pgfusepath{}%
\end{pgfscope}%
\begin{pgfscope}%
\pgfpathrectangle{\pgfqpoint{0.549740in}{0.463273in}}{\pgfqpoint{9.320225in}{4.495057in}}%
\pgfusepath{clip}%
\pgfsetbuttcap%
\pgfsetroundjoin%
\pgfsetlinewidth{0.000000pt}%
\definecolor{currentstroke}{rgb}{0.000000,0.000000,0.000000}%
\pgfsetstrokecolor{currentstroke}%
\pgfsetdash{}{0pt}%
\pgfpathmoveto{\pgfqpoint{4.832972in}{3.936726in}}%
\pgfpathlineto{\pgfqpoint{5.019198in}{3.936726in}}%
\pgfpathlineto{\pgfqpoint{5.019198in}{4.018455in}}%
\pgfpathlineto{\pgfqpoint{4.832972in}{4.018455in}}%
\pgfpathlineto{\pgfqpoint{4.832972in}{3.936726in}}%
\pgfusepath{}%
\end{pgfscope}%
\begin{pgfscope}%
\pgfpathrectangle{\pgfqpoint{0.549740in}{0.463273in}}{\pgfqpoint{9.320225in}{4.495057in}}%
\pgfusepath{clip}%
\pgfsetbuttcap%
\pgfsetroundjoin%
\pgfsetlinewidth{0.000000pt}%
\definecolor{currentstroke}{rgb}{0.000000,0.000000,0.000000}%
\pgfsetstrokecolor{currentstroke}%
\pgfsetdash{}{0pt}%
\pgfpathmoveto{\pgfqpoint{5.019198in}{3.936726in}}%
\pgfpathlineto{\pgfqpoint{5.205425in}{3.936726in}}%
\pgfpathlineto{\pgfqpoint{5.205425in}{4.018455in}}%
\pgfpathlineto{\pgfqpoint{5.019198in}{4.018455in}}%
\pgfpathlineto{\pgfqpoint{5.019198in}{3.936726in}}%
\pgfusepath{}%
\end{pgfscope}%
\begin{pgfscope}%
\pgfpathrectangle{\pgfqpoint{0.549740in}{0.463273in}}{\pgfqpoint{9.320225in}{4.495057in}}%
\pgfusepath{clip}%
\pgfsetbuttcap%
\pgfsetroundjoin%
\pgfsetlinewidth{0.000000pt}%
\definecolor{currentstroke}{rgb}{0.000000,0.000000,0.000000}%
\pgfsetstrokecolor{currentstroke}%
\pgfsetdash{}{0pt}%
\pgfpathmoveto{\pgfqpoint{5.205425in}{3.936726in}}%
\pgfpathlineto{\pgfqpoint{5.391651in}{3.936726in}}%
\pgfpathlineto{\pgfqpoint{5.391651in}{4.018455in}}%
\pgfpathlineto{\pgfqpoint{5.205425in}{4.018455in}}%
\pgfpathlineto{\pgfqpoint{5.205425in}{3.936726in}}%
\pgfusepath{}%
\end{pgfscope}%
\begin{pgfscope}%
\pgfpathrectangle{\pgfqpoint{0.549740in}{0.463273in}}{\pgfqpoint{9.320225in}{4.495057in}}%
\pgfusepath{clip}%
\pgfsetbuttcap%
\pgfsetroundjoin%
\pgfsetlinewidth{0.000000pt}%
\definecolor{currentstroke}{rgb}{0.000000,0.000000,0.000000}%
\pgfsetstrokecolor{currentstroke}%
\pgfsetdash{}{0pt}%
\pgfpathmoveto{\pgfqpoint{5.391651in}{3.936726in}}%
\pgfpathlineto{\pgfqpoint{5.577878in}{3.936726in}}%
\pgfpathlineto{\pgfqpoint{5.577878in}{4.018455in}}%
\pgfpathlineto{\pgfqpoint{5.391651in}{4.018455in}}%
\pgfpathlineto{\pgfqpoint{5.391651in}{3.936726in}}%
\pgfusepath{}%
\end{pgfscope}%
\begin{pgfscope}%
\pgfpathrectangle{\pgfqpoint{0.549740in}{0.463273in}}{\pgfqpoint{9.320225in}{4.495057in}}%
\pgfusepath{clip}%
\pgfsetbuttcap%
\pgfsetroundjoin%
\pgfsetlinewidth{0.000000pt}%
\definecolor{currentstroke}{rgb}{0.000000,0.000000,0.000000}%
\pgfsetstrokecolor{currentstroke}%
\pgfsetdash{}{0pt}%
\pgfpathmoveto{\pgfqpoint{5.577878in}{3.936726in}}%
\pgfpathlineto{\pgfqpoint{5.764104in}{3.936726in}}%
\pgfpathlineto{\pgfqpoint{5.764104in}{4.018455in}}%
\pgfpathlineto{\pgfqpoint{5.577878in}{4.018455in}}%
\pgfpathlineto{\pgfqpoint{5.577878in}{3.936726in}}%
\pgfusepath{}%
\end{pgfscope}%
\begin{pgfscope}%
\pgfpathrectangle{\pgfqpoint{0.549740in}{0.463273in}}{\pgfqpoint{9.320225in}{4.495057in}}%
\pgfusepath{clip}%
\pgfsetbuttcap%
\pgfsetroundjoin%
\pgfsetlinewidth{0.000000pt}%
\definecolor{currentstroke}{rgb}{0.000000,0.000000,0.000000}%
\pgfsetstrokecolor{currentstroke}%
\pgfsetdash{}{0pt}%
\pgfpathmoveto{\pgfqpoint{5.764104in}{3.936726in}}%
\pgfpathlineto{\pgfqpoint{5.950331in}{3.936726in}}%
\pgfpathlineto{\pgfqpoint{5.950331in}{4.018455in}}%
\pgfpathlineto{\pgfqpoint{5.764104in}{4.018455in}}%
\pgfpathlineto{\pgfqpoint{5.764104in}{3.936726in}}%
\pgfusepath{}%
\end{pgfscope}%
\begin{pgfscope}%
\pgfpathrectangle{\pgfqpoint{0.549740in}{0.463273in}}{\pgfqpoint{9.320225in}{4.495057in}}%
\pgfusepath{clip}%
\pgfsetbuttcap%
\pgfsetroundjoin%
\pgfsetlinewidth{0.000000pt}%
\definecolor{currentstroke}{rgb}{0.000000,0.000000,0.000000}%
\pgfsetstrokecolor{currentstroke}%
\pgfsetdash{}{0pt}%
\pgfpathmoveto{\pgfqpoint{5.950331in}{3.936726in}}%
\pgfpathlineto{\pgfqpoint{6.136557in}{3.936726in}}%
\pgfpathlineto{\pgfqpoint{6.136557in}{4.018455in}}%
\pgfpathlineto{\pgfqpoint{5.950331in}{4.018455in}}%
\pgfpathlineto{\pgfqpoint{5.950331in}{3.936726in}}%
\pgfusepath{}%
\end{pgfscope}%
\begin{pgfscope}%
\pgfpathrectangle{\pgfqpoint{0.549740in}{0.463273in}}{\pgfqpoint{9.320225in}{4.495057in}}%
\pgfusepath{clip}%
\pgfsetbuttcap%
\pgfsetroundjoin%
\pgfsetlinewidth{0.000000pt}%
\definecolor{currentstroke}{rgb}{0.000000,0.000000,0.000000}%
\pgfsetstrokecolor{currentstroke}%
\pgfsetdash{}{0pt}%
\pgfpathmoveto{\pgfqpoint{6.136557in}{3.936726in}}%
\pgfpathlineto{\pgfqpoint{6.322784in}{3.936726in}}%
\pgfpathlineto{\pgfqpoint{6.322784in}{4.018455in}}%
\pgfpathlineto{\pgfqpoint{6.136557in}{4.018455in}}%
\pgfpathlineto{\pgfqpoint{6.136557in}{3.936726in}}%
\pgfusepath{}%
\end{pgfscope}%
\begin{pgfscope}%
\pgfpathrectangle{\pgfqpoint{0.549740in}{0.463273in}}{\pgfqpoint{9.320225in}{4.495057in}}%
\pgfusepath{clip}%
\pgfsetbuttcap%
\pgfsetroundjoin%
\pgfsetlinewidth{0.000000pt}%
\definecolor{currentstroke}{rgb}{0.000000,0.000000,0.000000}%
\pgfsetstrokecolor{currentstroke}%
\pgfsetdash{}{0pt}%
\pgfpathmoveto{\pgfqpoint{6.322784in}{3.936726in}}%
\pgfpathlineto{\pgfqpoint{6.509011in}{3.936726in}}%
\pgfpathlineto{\pgfqpoint{6.509011in}{4.018455in}}%
\pgfpathlineto{\pgfqpoint{6.322784in}{4.018455in}}%
\pgfpathlineto{\pgfqpoint{6.322784in}{3.936726in}}%
\pgfusepath{}%
\end{pgfscope}%
\begin{pgfscope}%
\pgfpathrectangle{\pgfqpoint{0.549740in}{0.463273in}}{\pgfqpoint{9.320225in}{4.495057in}}%
\pgfusepath{clip}%
\pgfsetbuttcap%
\pgfsetroundjoin%
\pgfsetlinewidth{0.000000pt}%
\definecolor{currentstroke}{rgb}{0.000000,0.000000,0.000000}%
\pgfsetstrokecolor{currentstroke}%
\pgfsetdash{}{0pt}%
\pgfpathmoveto{\pgfqpoint{6.509011in}{3.936726in}}%
\pgfpathlineto{\pgfqpoint{6.695237in}{3.936726in}}%
\pgfpathlineto{\pgfqpoint{6.695237in}{4.018455in}}%
\pgfpathlineto{\pgfqpoint{6.509011in}{4.018455in}}%
\pgfpathlineto{\pgfqpoint{6.509011in}{3.936726in}}%
\pgfusepath{}%
\end{pgfscope}%
\begin{pgfscope}%
\pgfpathrectangle{\pgfqpoint{0.549740in}{0.463273in}}{\pgfqpoint{9.320225in}{4.495057in}}%
\pgfusepath{clip}%
\pgfsetbuttcap%
\pgfsetroundjoin%
\pgfsetlinewidth{0.000000pt}%
\definecolor{currentstroke}{rgb}{0.000000,0.000000,0.000000}%
\pgfsetstrokecolor{currentstroke}%
\pgfsetdash{}{0pt}%
\pgfpathmoveto{\pgfqpoint{6.695237in}{3.936726in}}%
\pgfpathlineto{\pgfqpoint{6.881464in}{3.936726in}}%
\pgfpathlineto{\pgfqpoint{6.881464in}{4.018455in}}%
\pgfpathlineto{\pgfqpoint{6.695237in}{4.018455in}}%
\pgfpathlineto{\pgfqpoint{6.695237in}{3.936726in}}%
\pgfusepath{}%
\end{pgfscope}%
\begin{pgfscope}%
\pgfpathrectangle{\pgfqpoint{0.549740in}{0.463273in}}{\pgfqpoint{9.320225in}{4.495057in}}%
\pgfusepath{clip}%
\pgfsetbuttcap%
\pgfsetroundjoin%
\pgfsetlinewidth{0.000000pt}%
\definecolor{currentstroke}{rgb}{0.000000,0.000000,0.000000}%
\pgfsetstrokecolor{currentstroke}%
\pgfsetdash{}{0pt}%
\pgfpathmoveto{\pgfqpoint{6.881464in}{3.936726in}}%
\pgfpathlineto{\pgfqpoint{7.067690in}{3.936726in}}%
\pgfpathlineto{\pgfqpoint{7.067690in}{4.018455in}}%
\pgfpathlineto{\pgfqpoint{6.881464in}{4.018455in}}%
\pgfpathlineto{\pgfqpoint{6.881464in}{3.936726in}}%
\pgfusepath{}%
\end{pgfscope}%
\begin{pgfscope}%
\pgfpathrectangle{\pgfqpoint{0.549740in}{0.463273in}}{\pgfqpoint{9.320225in}{4.495057in}}%
\pgfusepath{clip}%
\pgfsetbuttcap%
\pgfsetroundjoin%
\pgfsetlinewidth{0.000000pt}%
\definecolor{currentstroke}{rgb}{0.000000,0.000000,0.000000}%
\pgfsetstrokecolor{currentstroke}%
\pgfsetdash{}{0pt}%
\pgfpathmoveto{\pgfqpoint{7.067690in}{3.936726in}}%
\pgfpathlineto{\pgfqpoint{7.253917in}{3.936726in}}%
\pgfpathlineto{\pgfqpoint{7.253917in}{4.018455in}}%
\pgfpathlineto{\pgfqpoint{7.067690in}{4.018455in}}%
\pgfpathlineto{\pgfqpoint{7.067690in}{3.936726in}}%
\pgfusepath{}%
\end{pgfscope}%
\begin{pgfscope}%
\pgfpathrectangle{\pgfqpoint{0.549740in}{0.463273in}}{\pgfqpoint{9.320225in}{4.495057in}}%
\pgfusepath{clip}%
\pgfsetbuttcap%
\pgfsetroundjoin%
\pgfsetlinewidth{0.000000pt}%
\definecolor{currentstroke}{rgb}{0.000000,0.000000,0.000000}%
\pgfsetstrokecolor{currentstroke}%
\pgfsetdash{}{0pt}%
\pgfpathmoveto{\pgfqpoint{7.253917in}{3.936726in}}%
\pgfpathlineto{\pgfqpoint{7.440143in}{3.936726in}}%
\pgfpathlineto{\pgfqpoint{7.440143in}{4.018455in}}%
\pgfpathlineto{\pgfqpoint{7.253917in}{4.018455in}}%
\pgfpathlineto{\pgfqpoint{7.253917in}{3.936726in}}%
\pgfusepath{}%
\end{pgfscope}%
\begin{pgfscope}%
\pgfpathrectangle{\pgfqpoint{0.549740in}{0.463273in}}{\pgfqpoint{9.320225in}{4.495057in}}%
\pgfusepath{clip}%
\pgfsetbuttcap%
\pgfsetroundjoin%
\pgfsetlinewidth{0.000000pt}%
\definecolor{currentstroke}{rgb}{0.000000,0.000000,0.000000}%
\pgfsetstrokecolor{currentstroke}%
\pgfsetdash{}{0pt}%
\pgfpathmoveto{\pgfqpoint{7.440143in}{3.936726in}}%
\pgfpathlineto{\pgfqpoint{7.626370in}{3.936726in}}%
\pgfpathlineto{\pgfqpoint{7.626370in}{4.018455in}}%
\pgfpathlineto{\pgfqpoint{7.440143in}{4.018455in}}%
\pgfpathlineto{\pgfqpoint{7.440143in}{3.936726in}}%
\pgfusepath{}%
\end{pgfscope}%
\begin{pgfscope}%
\pgfpathrectangle{\pgfqpoint{0.549740in}{0.463273in}}{\pgfqpoint{9.320225in}{4.495057in}}%
\pgfusepath{clip}%
\pgfsetbuttcap%
\pgfsetroundjoin%
\pgfsetlinewidth{0.000000pt}%
\definecolor{currentstroke}{rgb}{0.000000,0.000000,0.000000}%
\pgfsetstrokecolor{currentstroke}%
\pgfsetdash{}{0pt}%
\pgfpathmoveto{\pgfqpoint{7.626370in}{3.936726in}}%
\pgfpathlineto{\pgfqpoint{7.812596in}{3.936726in}}%
\pgfpathlineto{\pgfqpoint{7.812596in}{4.018455in}}%
\pgfpathlineto{\pgfqpoint{7.626370in}{4.018455in}}%
\pgfpathlineto{\pgfqpoint{7.626370in}{3.936726in}}%
\pgfusepath{}%
\end{pgfscope}%
\begin{pgfscope}%
\pgfpathrectangle{\pgfqpoint{0.549740in}{0.463273in}}{\pgfqpoint{9.320225in}{4.495057in}}%
\pgfusepath{clip}%
\pgfsetbuttcap%
\pgfsetroundjoin%
\definecolor{currentfill}{rgb}{0.472869,0.711325,0.955316}%
\pgfsetfillcolor{currentfill}%
\pgfsetlinewidth{0.000000pt}%
\definecolor{currentstroke}{rgb}{0.000000,0.000000,0.000000}%
\pgfsetstrokecolor{currentstroke}%
\pgfsetdash{}{0pt}%
\pgfpathmoveto{\pgfqpoint{7.812596in}{3.936726in}}%
\pgfpathlineto{\pgfqpoint{7.998823in}{3.936726in}}%
\pgfpathlineto{\pgfqpoint{7.998823in}{4.018455in}}%
\pgfpathlineto{\pgfqpoint{7.812596in}{4.018455in}}%
\pgfpathlineto{\pgfqpoint{7.812596in}{3.936726in}}%
\pgfusepath{fill}%
\end{pgfscope}%
\begin{pgfscope}%
\pgfpathrectangle{\pgfqpoint{0.549740in}{0.463273in}}{\pgfqpoint{9.320225in}{4.495057in}}%
\pgfusepath{clip}%
\pgfsetbuttcap%
\pgfsetroundjoin%
\pgfsetlinewidth{0.000000pt}%
\definecolor{currentstroke}{rgb}{0.000000,0.000000,0.000000}%
\pgfsetstrokecolor{currentstroke}%
\pgfsetdash{}{0pt}%
\pgfpathmoveto{\pgfqpoint{7.998823in}{3.936726in}}%
\pgfpathlineto{\pgfqpoint{8.185049in}{3.936726in}}%
\pgfpathlineto{\pgfqpoint{8.185049in}{4.018455in}}%
\pgfpathlineto{\pgfqpoint{7.998823in}{4.018455in}}%
\pgfpathlineto{\pgfqpoint{7.998823in}{3.936726in}}%
\pgfusepath{}%
\end{pgfscope}%
\begin{pgfscope}%
\pgfpathrectangle{\pgfqpoint{0.549740in}{0.463273in}}{\pgfqpoint{9.320225in}{4.495057in}}%
\pgfusepath{clip}%
\pgfsetbuttcap%
\pgfsetroundjoin%
\pgfsetlinewidth{0.000000pt}%
\definecolor{currentstroke}{rgb}{0.000000,0.000000,0.000000}%
\pgfsetstrokecolor{currentstroke}%
\pgfsetdash{}{0pt}%
\pgfpathmoveto{\pgfqpoint{8.185049in}{3.936726in}}%
\pgfpathlineto{\pgfqpoint{8.371276in}{3.936726in}}%
\pgfpathlineto{\pgfqpoint{8.371276in}{4.018455in}}%
\pgfpathlineto{\pgfqpoint{8.185049in}{4.018455in}}%
\pgfpathlineto{\pgfqpoint{8.185049in}{3.936726in}}%
\pgfusepath{}%
\end{pgfscope}%
\begin{pgfscope}%
\pgfpathrectangle{\pgfqpoint{0.549740in}{0.463273in}}{\pgfqpoint{9.320225in}{4.495057in}}%
\pgfusepath{clip}%
\pgfsetbuttcap%
\pgfsetroundjoin%
\pgfsetlinewidth{0.000000pt}%
\definecolor{currentstroke}{rgb}{0.000000,0.000000,0.000000}%
\pgfsetstrokecolor{currentstroke}%
\pgfsetdash{}{0pt}%
\pgfpathmoveto{\pgfqpoint{8.371276in}{3.936726in}}%
\pgfpathlineto{\pgfqpoint{8.557503in}{3.936726in}}%
\pgfpathlineto{\pgfqpoint{8.557503in}{4.018455in}}%
\pgfpathlineto{\pgfqpoint{8.371276in}{4.018455in}}%
\pgfpathlineto{\pgfqpoint{8.371276in}{3.936726in}}%
\pgfusepath{}%
\end{pgfscope}%
\begin{pgfscope}%
\pgfpathrectangle{\pgfqpoint{0.549740in}{0.463273in}}{\pgfqpoint{9.320225in}{4.495057in}}%
\pgfusepath{clip}%
\pgfsetbuttcap%
\pgfsetroundjoin%
\pgfsetlinewidth{0.000000pt}%
\definecolor{currentstroke}{rgb}{0.000000,0.000000,0.000000}%
\pgfsetstrokecolor{currentstroke}%
\pgfsetdash{}{0pt}%
\pgfpathmoveto{\pgfqpoint{8.557503in}{3.936726in}}%
\pgfpathlineto{\pgfqpoint{8.743729in}{3.936726in}}%
\pgfpathlineto{\pgfqpoint{8.743729in}{4.018455in}}%
\pgfpathlineto{\pgfqpoint{8.557503in}{4.018455in}}%
\pgfpathlineto{\pgfqpoint{8.557503in}{3.936726in}}%
\pgfusepath{}%
\end{pgfscope}%
\begin{pgfscope}%
\pgfpathrectangle{\pgfqpoint{0.549740in}{0.463273in}}{\pgfqpoint{9.320225in}{4.495057in}}%
\pgfusepath{clip}%
\pgfsetbuttcap%
\pgfsetroundjoin%
\pgfsetlinewidth{0.000000pt}%
\definecolor{currentstroke}{rgb}{0.000000,0.000000,0.000000}%
\pgfsetstrokecolor{currentstroke}%
\pgfsetdash{}{0pt}%
\pgfpathmoveto{\pgfqpoint{8.743729in}{3.936726in}}%
\pgfpathlineto{\pgfqpoint{8.929956in}{3.936726in}}%
\pgfpathlineto{\pgfqpoint{8.929956in}{4.018455in}}%
\pgfpathlineto{\pgfqpoint{8.743729in}{4.018455in}}%
\pgfpathlineto{\pgfqpoint{8.743729in}{3.936726in}}%
\pgfusepath{}%
\end{pgfscope}%
\begin{pgfscope}%
\pgfpathrectangle{\pgfqpoint{0.549740in}{0.463273in}}{\pgfqpoint{9.320225in}{4.495057in}}%
\pgfusepath{clip}%
\pgfsetbuttcap%
\pgfsetroundjoin%
\pgfsetlinewidth{0.000000pt}%
\definecolor{currentstroke}{rgb}{0.000000,0.000000,0.000000}%
\pgfsetstrokecolor{currentstroke}%
\pgfsetdash{}{0pt}%
\pgfpathmoveto{\pgfqpoint{8.929956in}{3.936726in}}%
\pgfpathlineto{\pgfqpoint{9.116182in}{3.936726in}}%
\pgfpathlineto{\pgfqpoint{9.116182in}{4.018455in}}%
\pgfpathlineto{\pgfqpoint{8.929956in}{4.018455in}}%
\pgfpathlineto{\pgfqpoint{8.929956in}{3.936726in}}%
\pgfusepath{}%
\end{pgfscope}%
\begin{pgfscope}%
\pgfpathrectangle{\pgfqpoint{0.549740in}{0.463273in}}{\pgfqpoint{9.320225in}{4.495057in}}%
\pgfusepath{clip}%
\pgfsetbuttcap%
\pgfsetroundjoin%
\definecolor{currentfill}{rgb}{0.472869,0.711325,0.955316}%
\pgfsetfillcolor{currentfill}%
\pgfsetlinewidth{0.000000pt}%
\definecolor{currentstroke}{rgb}{0.000000,0.000000,0.000000}%
\pgfsetstrokecolor{currentstroke}%
\pgfsetdash{}{0pt}%
\pgfpathmoveto{\pgfqpoint{9.116182in}{3.936726in}}%
\pgfpathlineto{\pgfqpoint{9.302409in}{3.936726in}}%
\pgfpathlineto{\pgfqpoint{9.302409in}{4.018455in}}%
\pgfpathlineto{\pgfqpoint{9.116182in}{4.018455in}}%
\pgfpathlineto{\pgfqpoint{9.116182in}{3.936726in}}%
\pgfusepath{fill}%
\end{pgfscope}%
\begin{pgfscope}%
\pgfpathrectangle{\pgfqpoint{0.549740in}{0.463273in}}{\pgfqpoint{9.320225in}{4.495057in}}%
\pgfusepath{clip}%
\pgfsetbuttcap%
\pgfsetroundjoin%
\pgfsetlinewidth{0.000000pt}%
\definecolor{currentstroke}{rgb}{0.000000,0.000000,0.000000}%
\pgfsetstrokecolor{currentstroke}%
\pgfsetdash{}{0pt}%
\pgfpathmoveto{\pgfqpoint{9.302409in}{3.936726in}}%
\pgfpathlineto{\pgfqpoint{9.488635in}{3.936726in}}%
\pgfpathlineto{\pgfqpoint{9.488635in}{4.018455in}}%
\pgfpathlineto{\pgfqpoint{9.302409in}{4.018455in}}%
\pgfpathlineto{\pgfqpoint{9.302409in}{3.936726in}}%
\pgfusepath{}%
\end{pgfscope}%
\begin{pgfscope}%
\pgfpathrectangle{\pgfqpoint{0.549740in}{0.463273in}}{\pgfqpoint{9.320225in}{4.495057in}}%
\pgfusepath{clip}%
\pgfsetbuttcap%
\pgfsetroundjoin%
\pgfsetlinewidth{0.000000pt}%
\definecolor{currentstroke}{rgb}{0.000000,0.000000,0.000000}%
\pgfsetstrokecolor{currentstroke}%
\pgfsetdash{}{0pt}%
\pgfpathmoveto{\pgfqpoint{9.488635in}{3.936726in}}%
\pgfpathlineto{\pgfqpoint{9.674862in}{3.936726in}}%
\pgfpathlineto{\pgfqpoint{9.674862in}{4.018455in}}%
\pgfpathlineto{\pgfqpoint{9.488635in}{4.018455in}}%
\pgfpathlineto{\pgfqpoint{9.488635in}{3.936726in}}%
\pgfusepath{}%
\end{pgfscope}%
\begin{pgfscope}%
\pgfpathrectangle{\pgfqpoint{0.549740in}{0.463273in}}{\pgfqpoint{9.320225in}{4.495057in}}%
\pgfusepath{clip}%
\pgfsetbuttcap%
\pgfsetroundjoin%
\pgfsetlinewidth{0.000000pt}%
\definecolor{currentstroke}{rgb}{0.000000,0.000000,0.000000}%
\pgfsetstrokecolor{currentstroke}%
\pgfsetdash{}{0pt}%
\pgfpathmoveto{\pgfqpoint{9.674862in}{3.936726in}}%
\pgfpathlineto{\pgfqpoint{9.861088in}{3.936726in}}%
\pgfpathlineto{\pgfqpoint{9.861088in}{4.018455in}}%
\pgfpathlineto{\pgfqpoint{9.674862in}{4.018455in}}%
\pgfpathlineto{\pgfqpoint{9.674862in}{3.936726in}}%
\pgfusepath{}%
\end{pgfscope}%
\begin{pgfscope}%
\pgfpathrectangle{\pgfqpoint{0.549740in}{0.463273in}}{\pgfqpoint{9.320225in}{4.495057in}}%
\pgfusepath{clip}%
\pgfsetbuttcap%
\pgfsetroundjoin%
\pgfsetlinewidth{0.000000pt}%
\definecolor{currentstroke}{rgb}{0.000000,0.000000,0.000000}%
\pgfsetstrokecolor{currentstroke}%
\pgfsetdash{}{0pt}%
\pgfpathmoveto{\pgfqpoint{0.549761in}{4.018455in}}%
\pgfpathlineto{\pgfqpoint{0.735988in}{4.018455in}}%
\pgfpathlineto{\pgfqpoint{0.735988in}{4.100183in}}%
\pgfpathlineto{\pgfqpoint{0.549761in}{4.100183in}}%
\pgfpathlineto{\pgfqpoint{0.549761in}{4.018455in}}%
\pgfusepath{}%
\end{pgfscope}%
\begin{pgfscope}%
\pgfpathrectangle{\pgfqpoint{0.549740in}{0.463273in}}{\pgfqpoint{9.320225in}{4.495057in}}%
\pgfusepath{clip}%
\pgfsetbuttcap%
\pgfsetroundjoin%
\pgfsetlinewidth{0.000000pt}%
\definecolor{currentstroke}{rgb}{0.000000,0.000000,0.000000}%
\pgfsetstrokecolor{currentstroke}%
\pgfsetdash{}{0pt}%
\pgfpathmoveto{\pgfqpoint{0.735988in}{4.018455in}}%
\pgfpathlineto{\pgfqpoint{0.922214in}{4.018455in}}%
\pgfpathlineto{\pgfqpoint{0.922214in}{4.100183in}}%
\pgfpathlineto{\pgfqpoint{0.735988in}{4.100183in}}%
\pgfpathlineto{\pgfqpoint{0.735988in}{4.018455in}}%
\pgfusepath{}%
\end{pgfscope}%
\begin{pgfscope}%
\pgfpathrectangle{\pgfqpoint{0.549740in}{0.463273in}}{\pgfqpoint{9.320225in}{4.495057in}}%
\pgfusepath{clip}%
\pgfsetbuttcap%
\pgfsetroundjoin%
\pgfsetlinewidth{0.000000pt}%
\definecolor{currentstroke}{rgb}{0.000000,0.000000,0.000000}%
\pgfsetstrokecolor{currentstroke}%
\pgfsetdash{}{0pt}%
\pgfpathmoveto{\pgfqpoint{0.922214in}{4.018455in}}%
\pgfpathlineto{\pgfqpoint{1.108441in}{4.018455in}}%
\pgfpathlineto{\pgfqpoint{1.108441in}{4.100183in}}%
\pgfpathlineto{\pgfqpoint{0.922214in}{4.100183in}}%
\pgfpathlineto{\pgfqpoint{0.922214in}{4.018455in}}%
\pgfusepath{}%
\end{pgfscope}%
\begin{pgfscope}%
\pgfpathrectangle{\pgfqpoint{0.549740in}{0.463273in}}{\pgfqpoint{9.320225in}{4.495057in}}%
\pgfusepath{clip}%
\pgfsetbuttcap%
\pgfsetroundjoin%
\pgfsetlinewidth{0.000000pt}%
\definecolor{currentstroke}{rgb}{0.000000,0.000000,0.000000}%
\pgfsetstrokecolor{currentstroke}%
\pgfsetdash{}{0pt}%
\pgfpathmoveto{\pgfqpoint{1.108441in}{4.018455in}}%
\pgfpathlineto{\pgfqpoint{1.294667in}{4.018455in}}%
\pgfpathlineto{\pgfqpoint{1.294667in}{4.100183in}}%
\pgfpathlineto{\pgfqpoint{1.108441in}{4.100183in}}%
\pgfpathlineto{\pgfqpoint{1.108441in}{4.018455in}}%
\pgfusepath{}%
\end{pgfscope}%
\begin{pgfscope}%
\pgfpathrectangle{\pgfqpoint{0.549740in}{0.463273in}}{\pgfqpoint{9.320225in}{4.495057in}}%
\pgfusepath{clip}%
\pgfsetbuttcap%
\pgfsetroundjoin%
\pgfsetlinewidth{0.000000pt}%
\definecolor{currentstroke}{rgb}{0.000000,0.000000,0.000000}%
\pgfsetstrokecolor{currentstroke}%
\pgfsetdash{}{0pt}%
\pgfpathmoveto{\pgfqpoint{1.294667in}{4.018455in}}%
\pgfpathlineto{\pgfqpoint{1.480894in}{4.018455in}}%
\pgfpathlineto{\pgfqpoint{1.480894in}{4.100183in}}%
\pgfpathlineto{\pgfqpoint{1.294667in}{4.100183in}}%
\pgfpathlineto{\pgfqpoint{1.294667in}{4.018455in}}%
\pgfusepath{}%
\end{pgfscope}%
\begin{pgfscope}%
\pgfpathrectangle{\pgfqpoint{0.549740in}{0.463273in}}{\pgfqpoint{9.320225in}{4.495057in}}%
\pgfusepath{clip}%
\pgfsetbuttcap%
\pgfsetroundjoin%
\pgfsetlinewidth{0.000000pt}%
\definecolor{currentstroke}{rgb}{0.000000,0.000000,0.000000}%
\pgfsetstrokecolor{currentstroke}%
\pgfsetdash{}{0pt}%
\pgfpathmoveto{\pgfqpoint{1.480894in}{4.018455in}}%
\pgfpathlineto{\pgfqpoint{1.667120in}{4.018455in}}%
\pgfpathlineto{\pgfqpoint{1.667120in}{4.100183in}}%
\pgfpathlineto{\pgfqpoint{1.480894in}{4.100183in}}%
\pgfpathlineto{\pgfqpoint{1.480894in}{4.018455in}}%
\pgfusepath{}%
\end{pgfscope}%
\begin{pgfscope}%
\pgfpathrectangle{\pgfqpoint{0.549740in}{0.463273in}}{\pgfqpoint{9.320225in}{4.495057in}}%
\pgfusepath{clip}%
\pgfsetbuttcap%
\pgfsetroundjoin%
\pgfsetlinewidth{0.000000pt}%
\definecolor{currentstroke}{rgb}{0.000000,0.000000,0.000000}%
\pgfsetstrokecolor{currentstroke}%
\pgfsetdash{}{0pt}%
\pgfpathmoveto{\pgfqpoint{1.667120in}{4.018455in}}%
\pgfpathlineto{\pgfqpoint{1.853347in}{4.018455in}}%
\pgfpathlineto{\pgfqpoint{1.853347in}{4.100183in}}%
\pgfpathlineto{\pgfqpoint{1.667120in}{4.100183in}}%
\pgfpathlineto{\pgfqpoint{1.667120in}{4.018455in}}%
\pgfusepath{}%
\end{pgfscope}%
\begin{pgfscope}%
\pgfpathrectangle{\pgfqpoint{0.549740in}{0.463273in}}{\pgfqpoint{9.320225in}{4.495057in}}%
\pgfusepath{clip}%
\pgfsetbuttcap%
\pgfsetroundjoin%
\pgfsetlinewidth{0.000000pt}%
\definecolor{currentstroke}{rgb}{0.000000,0.000000,0.000000}%
\pgfsetstrokecolor{currentstroke}%
\pgfsetdash{}{0pt}%
\pgfpathmoveto{\pgfqpoint{1.853347in}{4.018455in}}%
\pgfpathlineto{\pgfqpoint{2.039573in}{4.018455in}}%
\pgfpathlineto{\pgfqpoint{2.039573in}{4.100183in}}%
\pgfpathlineto{\pgfqpoint{1.853347in}{4.100183in}}%
\pgfpathlineto{\pgfqpoint{1.853347in}{4.018455in}}%
\pgfusepath{}%
\end{pgfscope}%
\begin{pgfscope}%
\pgfpathrectangle{\pgfqpoint{0.549740in}{0.463273in}}{\pgfqpoint{9.320225in}{4.495057in}}%
\pgfusepath{clip}%
\pgfsetbuttcap%
\pgfsetroundjoin%
\pgfsetlinewidth{0.000000pt}%
\definecolor{currentstroke}{rgb}{0.000000,0.000000,0.000000}%
\pgfsetstrokecolor{currentstroke}%
\pgfsetdash{}{0pt}%
\pgfpathmoveto{\pgfqpoint{2.039573in}{4.018455in}}%
\pgfpathlineto{\pgfqpoint{2.225800in}{4.018455in}}%
\pgfpathlineto{\pgfqpoint{2.225800in}{4.100183in}}%
\pgfpathlineto{\pgfqpoint{2.039573in}{4.100183in}}%
\pgfpathlineto{\pgfqpoint{2.039573in}{4.018455in}}%
\pgfusepath{}%
\end{pgfscope}%
\begin{pgfscope}%
\pgfpathrectangle{\pgfqpoint{0.549740in}{0.463273in}}{\pgfqpoint{9.320225in}{4.495057in}}%
\pgfusepath{clip}%
\pgfsetbuttcap%
\pgfsetroundjoin%
\pgfsetlinewidth{0.000000pt}%
\definecolor{currentstroke}{rgb}{0.000000,0.000000,0.000000}%
\pgfsetstrokecolor{currentstroke}%
\pgfsetdash{}{0pt}%
\pgfpathmoveto{\pgfqpoint{2.225800in}{4.018455in}}%
\pgfpathlineto{\pgfqpoint{2.412027in}{4.018455in}}%
\pgfpathlineto{\pgfqpoint{2.412027in}{4.100183in}}%
\pgfpathlineto{\pgfqpoint{2.225800in}{4.100183in}}%
\pgfpathlineto{\pgfqpoint{2.225800in}{4.018455in}}%
\pgfusepath{}%
\end{pgfscope}%
\begin{pgfscope}%
\pgfpathrectangle{\pgfqpoint{0.549740in}{0.463273in}}{\pgfqpoint{9.320225in}{4.495057in}}%
\pgfusepath{clip}%
\pgfsetbuttcap%
\pgfsetroundjoin%
\pgfsetlinewidth{0.000000pt}%
\definecolor{currentstroke}{rgb}{0.000000,0.000000,0.000000}%
\pgfsetstrokecolor{currentstroke}%
\pgfsetdash{}{0pt}%
\pgfpathmoveto{\pgfqpoint{2.412027in}{4.018455in}}%
\pgfpathlineto{\pgfqpoint{2.598253in}{4.018455in}}%
\pgfpathlineto{\pgfqpoint{2.598253in}{4.100183in}}%
\pgfpathlineto{\pgfqpoint{2.412027in}{4.100183in}}%
\pgfpathlineto{\pgfqpoint{2.412027in}{4.018455in}}%
\pgfusepath{}%
\end{pgfscope}%
\begin{pgfscope}%
\pgfpathrectangle{\pgfqpoint{0.549740in}{0.463273in}}{\pgfqpoint{9.320225in}{4.495057in}}%
\pgfusepath{clip}%
\pgfsetbuttcap%
\pgfsetroundjoin%
\pgfsetlinewidth{0.000000pt}%
\definecolor{currentstroke}{rgb}{0.000000,0.000000,0.000000}%
\pgfsetstrokecolor{currentstroke}%
\pgfsetdash{}{0pt}%
\pgfpathmoveto{\pgfqpoint{2.598253in}{4.018455in}}%
\pgfpathlineto{\pgfqpoint{2.784480in}{4.018455in}}%
\pgfpathlineto{\pgfqpoint{2.784480in}{4.100183in}}%
\pgfpathlineto{\pgfqpoint{2.598253in}{4.100183in}}%
\pgfpathlineto{\pgfqpoint{2.598253in}{4.018455in}}%
\pgfusepath{}%
\end{pgfscope}%
\begin{pgfscope}%
\pgfpathrectangle{\pgfqpoint{0.549740in}{0.463273in}}{\pgfqpoint{9.320225in}{4.495057in}}%
\pgfusepath{clip}%
\pgfsetbuttcap%
\pgfsetroundjoin%
\pgfsetlinewidth{0.000000pt}%
\definecolor{currentstroke}{rgb}{0.000000,0.000000,0.000000}%
\pgfsetstrokecolor{currentstroke}%
\pgfsetdash{}{0pt}%
\pgfpathmoveto{\pgfqpoint{2.784480in}{4.018455in}}%
\pgfpathlineto{\pgfqpoint{2.970706in}{4.018455in}}%
\pgfpathlineto{\pgfqpoint{2.970706in}{4.100183in}}%
\pgfpathlineto{\pgfqpoint{2.784480in}{4.100183in}}%
\pgfpathlineto{\pgfqpoint{2.784480in}{4.018455in}}%
\pgfusepath{}%
\end{pgfscope}%
\begin{pgfscope}%
\pgfpathrectangle{\pgfqpoint{0.549740in}{0.463273in}}{\pgfqpoint{9.320225in}{4.495057in}}%
\pgfusepath{clip}%
\pgfsetbuttcap%
\pgfsetroundjoin%
\pgfsetlinewidth{0.000000pt}%
\definecolor{currentstroke}{rgb}{0.000000,0.000000,0.000000}%
\pgfsetstrokecolor{currentstroke}%
\pgfsetdash{}{0pt}%
\pgfpathmoveto{\pgfqpoint{2.970706in}{4.018455in}}%
\pgfpathlineto{\pgfqpoint{3.156933in}{4.018455in}}%
\pgfpathlineto{\pgfqpoint{3.156933in}{4.100183in}}%
\pgfpathlineto{\pgfqpoint{2.970706in}{4.100183in}}%
\pgfpathlineto{\pgfqpoint{2.970706in}{4.018455in}}%
\pgfusepath{}%
\end{pgfscope}%
\begin{pgfscope}%
\pgfpathrectangle{\pgfqpoint{0.549740in}{0.463273in}}{\pgfqpoint{9.320225in}{4.495057in}}%
\pgfusepath{clip}%
\pgfsetbuttcap%
\pgfsetroundjoin%
\pgfsetlinewidth{0.000000pt}%
\definecolor{currentstroke}{rgb}{0.000000,0.000000,0.000000}%
\pgfsetstrokecolor{currentstroke}%
\pgfsetdash{}{0pt}%
\pgfpathmoveto{\pgfqpoint{3.156933in}{4.018455in}}%
\pgfpathlineto{\pgfqpoint{3.343159in}{4.018455in}}%
\pgfpathlineto{\pgfqpoint{3.343159in}{4.100183in}}%
\pgfpathlineto{\pgfqpoint{3.156933in}{4.100183in}}%
\pgfpathlineto{\pgfqpoint{3.156933in}{4.018455in}}%
\pgfusepath{}%
\end{pgfscope}%
\begin{pgfscope}%
\pgfpathrectangle{\pgfqpoint{0.549740in}{0.463273in}}{\pgfqpoint{9.320225in}{4.495057in}}%
\pgfusepath{clip}%
\pgfsetbuttcap%
\pgfsetroundjoin%
\pgfsetlinewidth{0.000000pt}%
\definecolor{currentstroke}{rgb}{0.000000,0.000000,0.000000}%
\pgfsetstrokecolor{currentstroke}%
\pgfsetdash{}{0pt}%
\pgfpathmoveto{\pgfqpoint{3.343159in}{4.018455in}}%
\pgfpathlineto{\pgfqpoint{3.529386in}{4.018455in}}%
\pgfpathlineto{\pgfqpoint{3.529386in}{4.100183in}}%
\pgfpathlineto{\pgfqpoint{3.343159in}{4.100183in}}%
\pgfpathlineto{\pgfqpoint{3.343159in}{4.018455in}}%
\pgfusepath{}%
\end{pgfscope}%
\begin{pgfscope}%
\pgfpathrectangle{\pgfqpoint{0.549740in}{0.463273in}}{\pgfqpoint{9.320225in}{4.495057in}}%
\pgfusepath{clip}%
\pgfsetbuttcap%
\pgfsetroundjoin%
\pgfsetlinewidth{0.000000pt}%
\definecolor{currentstroke}{rgb}{0.000000,0.000000,0.000000}%
\pgfsetstrokecolor{currentstroke}%
\pgfsetdash{}{0pt}%
\pgfpathmoveto{\pgfqpoint{3.529386in}{4.018455in}}%
\pgfpathlineto{\pgfqpoint{3.715612in}{4.018455in}}%
\pgfpathlineto{\pgfqpoint{3.715612in}{4.100183in}}%
\pgfpathlineto{\pgfqpoint{3.529386in}{4.100183in}}%
\pgfpathlineto{\pgfqpoint{3.529386in}{4.018455in}}%
\pgfusepath{}%
\end{pgfscope}%
\begin{pgfscope}%
\pgfpathrectangle{\pgfqpoint{0.549740in}{0.463273in}}{\pgfqpoint{9.320225in}{4.495057in}}%
\pgfusepath{clip}%
\pgfsetbuttcap%
\pgfsetroundjoin%
\pgfsetlinewidth{0.000000pt}%
\definecolor{currentstroke}{rgb}{0.000000,0.000000,0.000000}%
\pgfsetstrokecolor{currentstroke}%
\pgfsetdash{}{0pt}%
\pgfpathmoveto{\pgfqpoint{3.715612in}{4.018455in}}%
\pgfpathlineto{\pgfqpoint{3.901839in}{4.018455in}}%
\pgfpathlineto{\pgfqpoint{3.901839in}{4.100183in}}%
\pgfpathlineto{\pgfqpoint{3.715612in}{4.100183in}}%
\pgfpathlineto{\pgfqpoint{3.715612in}{4.018455in}}%
\pgfusepath{}%
\end{pgfscope}%
\begin{pgfscope}%
\pgfpathrectangle{\pgfqpoint{0.549740in}{0.463273in}}{\pgfqpoint{9.320225in}{4.495057in}}%
\pgfusepath{clip}%
\pgfsetbuttcap%
\pgfsetroundjoin%
\pgfsetlinewidth{0.000000pt}%
\definecolor{currentstroke}{rgb}{0.000000,0.000000,0.000000}%
\pgfsetstrokecolor{currentstroke}%
\pgfsetdash{}{0pt}%
\pgfpathmoveto{\pgfqpoint{3.901839in}{4.018455in}}%
\pgfpathlineto{\pgfqpoint{4.088065in}{4.018455in}}%
\pgfpathlineto{\pgfqpoint{4.088065in}{4.100183in}}%
\pgfpathlineto{\pgfqpoint{3.901839in}{4.100183in}}%
\pgfpathlineto{\pgfqpoint{3.901839in}{4.018455in}}%
\pgfusepath{}%
\end{pgfscope}%
\begin{pgfscope}%
\pgfpathrectangle{\pgfqpoint{0.549740in}{0.463273in}}{\pgfqpoint{9.320225in}{4.495057in}}%
\pgfusepath{clip}%
\pgfsetbuttcap%
\pgfsetroundjoin%
\pgfsetlinewidth{0.000000pt}%
\definecolor{currentstroke}{rgb}{0.000000,0.000000,0.000000}%
\pgfsetstrokecolor{currentstroke}%
\pgfsetdash{}{0pt}%
\pgfpathmoveto{\pgfqpoint{4.088065in}{4.018455in}}%
\pgfpathlineto{\pgfqpoint{4.274292in}{4.018455in}}%
\pgfpathlineto{\pgfqpoint{4.274292in}{4.100183in}}%
\pgfpathlineto{\pgfqpoint{4.088065in}{4.100183in}}%
\pgfpathlineto{\pgfqpoint{4.088065in}{4.018455in}}%
\pgfusepath{}%
\end{pgfscope}%
\begin{pgfscope}%
\pgfpathrectangle{\pgfqpoint{0.549740in}{0.463273in}}{\pgfqpoint{9.320225in}{4.495057in}}%
\pgfusepath{clip}%
\pgfsetbuttcap%
\pgfsetroundjoin%
\pgfsetlinewidth{0.000000pt}%
\definecolor{currentstroke}{rgb}{0.000000,0.000000,0.000000}%
\pgfsetstrokecolor{currentstroke}%
\pgfsetdash{}{0pt}%
\pgfpathmoveto{\pgfqpoint{4.274292in}{4.018455in}}%
\pgfpathlineto{\pgfqpoint{4.460519in}{4.018455in}}%
\pgfpathlineto{\pgfqpoint{4.460519in}{4.100183in}}%
\pgfpathlineto{\pgfqpoint{4.274292in}{4.100183in}}%
\pgfpathlineto{\pgfqpoint{4.274292in}{4.018455in}}%
\pgfusepath{}%
\end{pgfscope}%
\begin{pgfscope}%
\pgfpathrectangle{\pgfqpoint{0.549740in}{0.463273in}}{\pgfqpoint{9.320225in}{4.495057in}}%
\pgfusepath{clip}%
\pgfsetbuttcap%
\pgfsetroundjoin%
\pgfsetlinewidth{0.000000pt}%
\definecolor{currentstroke}{rgb}{0.000000,0.000000,0.000000}%
\pgfsetstrokecolor{currentstroke}%
\pgfsetdash{}{0pt}%
\pgfpathmoveto{\pgfqpoint{4.460519in}{4.018455in}}%
\pgfpathlineto{\pgfqpoint{4.646745in}{4.018455in}}%
\pgfpathlineto{\pgfqpoint{4.646745in}{4.100183in}}%
\pgfpathlineto{\pgfqpoint{4.460519in}{4.100183in}}%
\pgfpathlineto{\pgfqpoint{4.460519in}{4.018455in}}%
\pgfusepath{}%
\end{pgfscope}%
\begin{pgfscope}%
\pgfpathrectangle{\pgfqpoint{0.549740in}{0.463273in}}{\pgfqpoint{9.320225in}{4.495057in}}%
\pgfusepath{clip}%
\pgfsetbuttcap%
\pgfsetroundjoin%
\pgfsetlinewidth{0.000000pt}%
\definecolor{currentstroke}{rgb}{0.000000,0.000000,0.000000}%
\pgfsetstrokecolor{currentstroke}%
\pgfsetdash{}{0pt}%
\pgfpathmoveto{\pgfqpoint{4.646745in}{4.018455in}}%
\pgfpathlineto{\pgfqpoint{4.832972in}{4.018455in}}%
\pgfpathlineto{\pgfqpoint{4.832972in}{4.100183in}}%
\pgfpathlineto{\pgfqpoint{4.646745in}{4.100183in}}%
\pgfpathlineto{\pgfqpoint{4.646745in}{4.018455in}}%
\pgfusepath{}%
\end{pgfscope}%
\begin{pgfscope}%
\pgfpathrectangle{\pgfqpoint{0.549740in}{0.463273in}}{\pgfqpoint{9.320225in}{4.495057in}}%
\pgfusepath{clip}%
\pgfsetbuttcap%
\pgfsetroundjoin%
\pgfsetlinewidth{0.000000pt}%
\definecolor{currentstroke}{rgb}{0.000000,0.000000,0.000000}%
\pgfsetstrokecolor{currentstroke}%
\pgfsetdash{}{0pt}%
\pgfpathmoveto{\pgfqpoint{4.832972in}{4.018455in}}%
\pgfpathlineto{\pgfqpoint{5.019198in}{4.018455in}}%
\pgfpathlineto{\pgfqpoint{5.019198in}{4.100183in}}%
\pgfpathlineto{\pgfqpoint{4.832972in}{4.100183in}}%
\pgfpathlineto{\pgfqpoint{4.832972in}{4.018455in}}%
\pgfusepath{}%
\end{pgfscope}%
\begin{pgfscope}%
\pgfpathrectangle{\pgfqpoint{0.549740in}{0.463273in}}{\pgfqpoint{9.320225in}{4.495057in}}%
\pgfusepath{clip}%
\pgfsetbuttcap%
\pgfsetroundjoin%
\pgfsetlinewidth{0.000000pt}%
\definecolor{currentstroke}{rgb}{0.000000,0.000000,0.000000}%
\pgfsetstrokecolor{currentstroke}%
\pgfsetdash{}{0pt}%
\pgfpathmoveto{\pgfqpoint{5.019198in}{4.018455in}}%
\pgfpathlineto{\pgfqpoint{5.205425in}{4.018455in}}%
\pgfpathlineto{\pgfqpoint{5.205425in}{4.100183in}}%
\pgfpathlineto{\pgfqpoint{5.019198in}{4.100183in}}%
\pgfpathlineto{\pgfqpoint{5.019198in}{4.018455in}}%
\pgfusepath{}%
\end{pgfscope}%
\begin{pgfscope}%
\pgfpathrectangle{\pgfqpoint{0.549740in}{0.463273in}}{\pgfqpoint{9.320225in}{4.495057in}}%
\pgfusepath{clip}%
\pgfsetbuttcap%
\pgfsetroundjoin%
\pgfsetlinewidth{0.000000pt}%
\definecolor{currentstroke}{rgb}{0.000000,0.000000,0.000000}%
\pgfsetstrokecolor{currentstroke}%
\pgfsetdash{}{0pt}%
\pgfpathmoveto{\pgfqpoint{5.205425in}{4.018455in}}%
\pgfpathlineto{\pgfqpoint{5.391651in}{4.018455in}}%
\pgfpathlineto{\pgfqpoint{5.391651in}{4.100183in}}%
\pgfpathlineto{\pgfqpoint{5.205425in}{4.100183in}}%
\pgfpathlineto{\pgfqpoint{5.205425in}{4.018455in}}%
\pgfusepath{}%
\end{pgfscope}%
\begin{pgfscope}%
\pgfpathrectangle{\pgfqpoint{0.549740in}{0.463273in}}{\pgfqpoint{9.320225in}{4.495057in}}%
\pgfusepath{clip}%
\pgfsetbuttcap%
\pgfsetroundjoin%
\pgfsetlinewidth{0.000000pt}%
\definecolor{currentstroke}{rgb}{0.000000,0.000000,0.000000}%
\pgfsetstrokecolor{currentstroke}%
\pgfsetdash{}{0pt}%
\pgfpathmoveto{\pgfqpoint{5.391651in}{4.018455in}}%
\pgfpathlineto{\pgfqpoint{5.577878in}{4.018455in}}%
\pgfpathlineto{\pgfqpoint{5.577878in}{4.100183in}}%
\pgfpathlineto{\pgfqpoint{5.391651in}{4.100183in}}%
\pgfpathlineto{\pgfqpoint{5.391651in}{4.018455in}}%
\pgfusepath{}%
\end{pgfscope}%
\begin{pgfscope}%
\pgfpathrectangle{\pgfqpoint{0.549740in}{0.463273in}}{\pgfqpoint{9.320225in}{4.495057in}}%
\pgfusepath{clip}%
\pgfsetbuttcap%
\pgfsetroundjoin%
\pgfsetlinewidth{0.000000pt}%
\definecolor{currentstroke}{rgb}{0.000000,0.000000,0.000000}%
\pgfsetstrokecolor{currentstroke}%
\pgfsetdash{}{0pt}%
\pgfpathmoveto{\pgfqpoint{5.577878in}{4.018455in}}%
\pgfpathlineto{\pgfqpoint{5.764104in}{4.018455in}}%
\pgfpathlineto{\pgfqpoint{5.764104in}{4.100183in}}%
\pgfpathlineto{\pgfqpoint{5.577878in}{4.100183in}}%
\pgfpathlineto{\pgfqpoint{5.577878in}{4.018455in}}%
\pgfusepath{}%
\end{pgfscope}%
\begin{pgfscope}%
\pgfpathrectangle{\pgfqpoint{0.549740in}{0.463273in}}{\pgfqpoint{9.320225in}{4.495057in}}%
\pgfusepath{clip}%
\pgfsetbuttcap%
\pgfsetroundjoin%
\pgfsetlinewidth{0.000000pt}%
\definecolor{currentstroke}{rgb}{0.000000,0.000000,0.000000}%
\pgfsetstrokecolor{currentstroke}%
\pgfsetdash{}{0pt}%
\pgfpathmoveto{\pgfqpoint{5.764104in}{4.018455in}}%
\pgfpathlineto{\pgfqpoint{5.950331in}{4.018455in}}%
\pgfpathlineto{\pgfqpoint{5.950331in}{4.100183in}}%
\pgfpathlineto{\pgfqpoint{5.764104in}{4.100183in}}%
\pgfpathlineto{\pgfqpoint{5.764104in}{4.018455in}}%
\pgfusepath{}%
\end{pgfscope}%
\begin{pgfscope}%
\pgfpathrectangle{\pgfqpoint{0.549740in}{0.463273in}}{\pgfqpoint{9.320225in}{4.495057in}}%
\pgfusepath{clip}%
\pgfsetbuttcap%
\pgfsetroundjoin%
\pgfsetlinewidth{0.000000pt}%
\definecolor{currentstroke}{rgb}{0.000000,0.000000,0.000000}%
\pgfsetstrokecolor{currentstroke}%
\pgfsetdash{}{0pt}%
\pgfpathmoveto{\pgfqpoint{5.950331in}{4.018455in}}%
\pgfpathlineto{\pgfqpoint{6.136557in}{4.018455in}}%
\pgfpathlineto{\pgfqpoint{6.136557in}{4.100183in}}%
\pgfpathlineto{\pgfqpoint{5.950331in}{4.100183in}}%
\pgfpathlineto{\pgfqpoint{5.950331in}{4.018455in}}%
\pgfusepath{}%
\end{pgfscope}%
\begin{pgfscope}%
\pgfpathrectangle{\pgfqpoint{0.549740in}{0.463273in}}{\pgfqpoint{9.320225in}{4.495057in}}%
\pgfusepath{clip}%
\pgfsetbuttcap%
\pgfsetroundjoin%
\pgfsetlinewidth{0.000000pt}%
\definecolor{currentstroke}{rgb}{0.000000,0.000000,0.000000}%
\pgfsetstrokecolor{currentstroke}%
\pgfsetdash{}{0pt}%
\pgfpathmoveto{\pgfqpoint{6.136557in}{4.018455in}}%
\pgfpathlineto{\pgfqpoint{6.322784in}{4.018455in}}%
\pgfpathlineto{\pgfqpoint{6.322784in}{4.100183in}}%
\pgfpathlineto{\pgfqpoint{6.136557in}{4.100183in}}%
\pgfpathlineto{\pgfqpoint{6.136557in}{4.018455in}}%
\pgfusepath{}%
\end{pgfscope}%
\begin{pgfscope}%
\pgfpathrectangle{\pgfqpoint{0.549740in}{0.463273in}}{\pgfqpoint{9.320225in}{4.495057in}}%
\pgfusepath{clip}%
\pgfsetbuttcap%
\pgfsetroundjoin%
\pgfsetlinewidth{0.000000pt}%
\definecolor{currentstroke}{rgb}{0.000000,0.000000,0.000000}%
\pgfsetstrokecolor{currentstroke}%
\pgfsetdash{}{0pt}%
\pgfpathmoveto{\pgfqpoint{6.322784in}{4.018455in}}%
\pgfpathlineto{\pgfqpoint{6.509011in}{4.018455in}}%
\pgfpathlineto{\pgfqpoint{6.509011in}{4.100183in}}%
\pgfpathlineto{\pgfqpoint{6.322784in}{4.100183in}}%
\pgfpathlineto{\pgfqpoint{6.322784in}{4.018455in}}%
\pgfusepath{}%
\end{pgfscope}%
\begin{pgfscope}%
\pgfpathrectangle{\pgfqpoint{0.549740in}{0.463273in}}{\pgfqpoint{9.320225in}{4.495057in}}%
\pgfusepath{clip}%
\pgfsetbuttcap%
\pgfsetroundjoin%
\pgfsetlinewidth{0.000000pt}%
\definecolor{currentstroke}{rgb}{0.000000,0.000000,0.000000}%
\pgfsetstrokecolor{currentstroke}%
\pgfsetdash{}{0pt}%
\pgfpathmoveto{\pgfqpoint{6.509011in}{4.018455in}}%
\pgfpathlineto{\pgfqpoint{6.695237in}{4.018455in}}%
\pgfpathlineto{\pgfqpoint{6.695237in}{4.100183in}}%
\pgfpathlineto{\pgfqpoint{6.509011in}{4.100183in}}%
\pgfpathlineto{\pgfqpoint{6.509011in}{4.018455in}}%
\pgfusepath{}%
\end{pgfscope}%
\begin{pgfscope}%
\pgfpathrectangle{\pgfqpoint{0.549740in}{0.463273in}}{\pgfqpoint{9.320225in}{4.495057in}}%
\pgfusepath{clip}%
\pgfsetbuttcap%
\pgfsetroundjoin%
\pgfsetlinewidth{0.000000pt}%
\definecolor{currentstroke}{rgb}{0.000000,0.000000,0.000000}%
\pgfsetstrokecolor{currentstroke}%
\pgfsetdash{}{0pt}%
\pgfpathmoveto{\pgfqpoint{6.695237in}{4.018455in}}%
\pgfpathlineto{\pgfqpoint{6.881464in}{4.018455in}}%
\pgfpathlineto{\pgfqpoint{6.881464in}{4.100183in}}%
\pgfpathlineto{\pgfqpoint{6.695237in}{4.100183in}}%
\pgfpathlineto{\pgfqpoint{6.695237in}{4.018455in}}%
\pgfusepath{}%
\end{pgfscope}%
\begin{pgfscope}%
\pgfpathrectangle{\pgfqpoint{0.549740in}{0.463273in}}{\pgfqpoint{9.320225in}{4.495057in}}%
\pgfusepath{clip}%
\pgfsetbuttcap%
\pgfsetroundjoin%
\pgfsetlinewidth{0.000000pt}%
\definecolor{currentstroke}{rgb}{0.000000,0.000000,0.000000}%
\pgfsetstrokecolor{currentstroke}%
\pgfsetdash{}{0pt}%
\pgfpathmoveto{\pgfqpoint{6.881464in}{4.018455in}}%
\pgfpathlineto{\pgfqpoint{7.067690in}{4.018455in}}%
\pgfpathlineto{\pgfqpoint{7.067690in}{4.100183in}}%
\pgfpathlineto{\pgfqpoint{6.881464in}{4.100183in}}%
\pgfpathlineto{\pgfqpoint{6.881464in}{4.018455in}}%
\pgfusepath{}%
\end{pgfscope}%
\begin{pgfscope}%
\pgfpathrectangle{\pgfqpoint{0.549740in}{0.463273in}}{\pgfqpoint{9.320225in}{4.495057in}}%
\pgfusepath{clip}%
\pgfsetbuttcap%
\pgfsetroundjoin%
\pgfsetlinewidth{0.000000pt}%
\definecolor{currentstroke}{rgb}{0.000000,0.000000,0.000000}%
\pgfsetstrokecolor{currentstroke}%
\pgfsetdash{}{0pt}%
\pgfpathmoveto{\pgfqpoint{7.067690in}{4.018455in}}%
\pgfpathlineto{\pgfqpoint{7.253917in}{4.018455in}}%
\pgfpathlineto{\pgfqpoint{7.253917in}{4.100183in}}%
\pgfpathlineto{\pgfqpoint{7.067690in}{4.100183in}}%
\pgfpathlineto{\pgfqpoint{7.067690in}{4.018455in}}%
\pgfusepath{}%
\end{pgfscope}%
\begin{pgfscope}%
\pgfpathrectangle{\pgfqpoint{0.549740in}{0.463273in}}{\pgfqpoint{9.320225in}{4.495057in}}%
\pgfusepath{clip}%
\pgfsetbuttcap%
\pgfsetroundjoin%
\pgfsetlinewidth{0.000000pt}%
\definecolor{currentstroke}{rgb}{0.000000,0.000000,0.000000}%
\pgfsetstrokecolor{currentstroke}%
\pgfsetdash{}{0pt}%
\pgfpathmoveto{\pgfqpoint{7.253917in}{4.018455in}}%
\pgfpathlineto{\pgfqpoint{7.440143in}{4.018455in}}%
\pgfpathlineto{\pgfqpoint{7.440143in}{4.100183in}}%
\pgfpathlineto{\pgfqpoint{7.253917in}{4.100183in}}%
\pgfpathlineto{\pgfqpoint{7.253917in}{4.018455in}}%
\pgfusepath{}%
\end{pgfscope}%
\begin{pgfscope}%
\pgfpathrectangle{\pgfqpoint{0.549740in}{0.463273in}}{\pgfqpoint{9.320225in}{4.495057in}}%
\pgfusepath{clip}%
\pgfsetbuttcap%
\pgfsetroundjoin%
\pgfsetlinewidth{0.000000pt}%
\definecolor{currentstroke}{rgb}{0.000000,0.000000,0.000000}%
\pgfsetstrokecolor{currentstroke}%
\pgfsetdash{}{0pt}%
\pgfpathmoveto{\pgfqpoint{7.440143in}{4.018455in}}%
\pgfpathlineto{\pgfqpoint{7.626370in}{4.018455in}}%
\pgfpathlineto{\pgfqpoint{7.626370in}{4.100183in}}%
\pgfpathlineto{\pgfqpoint{7.440143in}{4.100183in}}%
\pgfpathlineto{\pgfqpoint{7.440143in}{4.018455in}}%
\pgfusepath{}%
\end{pgfscope}%
\begin{pgfscope}%
\pgfpathrectangle{\pgfqpoint{0.549740in}{0.463273in}}{\pgfqpoint{9.320225in}{4.495057in}}%
\pgfusepath{clip}%
\pgfsetbuttcap%
\pgfsetroundjoin%
\pgfsetlinewidth{0.000000pt}%
\definecolor{currentstroke}{rgb}{0.000000,0.000000,0.000000}%
\pgfsetstrokecolor{currentstroke}%
\pgfsetdash{}{0pt}%
\pgfpathmoveto{\pgfqpoint{7.626370in}{4.018455in}}%
\pgfpathlineto{\pgfqpoint{7.812596in}{4.018455in}}%
\pgfpathlineto{\pgfqpoint{7.812596in}{4.100183in}}%
\pgfpathlineto{\pgfqpoint{7.626370in}{4.100183in}}%
\pgfpathlineto{\pgfqpoint{7.626370in}{4.018455in}}%
\pgfusepath{}%
\end{pgfscope}%
\begin{pgfscope}%
\pgfpathrectangle{\pgfqpoint{0.549740in}{0.463273in}}{\pgfqpoint{9.320225in}{4.495057in}}%
\pgfusepath{clip}%
\pgfsetbuttcap%
\pgfsetroundjoin%
\definecolor{currentfill}{rgb}{0.472869,0.711325,0.955316}%
\pgfsetfillcolor{currentfill}%
\pgfsetlinewidth{0.000000pt}%
\definecolor{currentstroke}{rgb}{0.000000,0.000000,0.000000}%
\pgfsetstrokecolor{currentstroke}%
\pgfsetdash{}{0pt}%
\pgfpathmoveto{\pgfqpoint{7.812596in}{4.018455in}}%
\pgfpathlineto{\pgfqpoint{7.998823in}{4.018455in}}%
\pgfpathlineto{\pgfqpoint{7.998823in}{4.100183in}}%
\pgfpathlineto{\pgfqpoint{7.812596in}{4.100183in}}%
\pgfpathlineto{\pgfqpoint{7.812596in}{4.018455in}}%
\pgfusepath{fill}%
\end{pgfscope}%
\begin{pgfscope}%
\pgfpathrectangle{\pgfqpoint{0.549740in}{0.463273in}}{\pgfqpoint{9.320225in}{4.495057in}}%
\pgfusepath{clip}%
\pgfsetbuttcap%
\pgfsetroundjoin%
\pgfsetlinewidth{0.000000pt}%
\definecolor{currentstroke}{rgb}{0.000000,0.000000,0.000000}%
\pgfsetstrokecolor{currentstroke}%
\pgfsetdash{}{0pt}%
\pgfpathmoveto{\pgfqpoint{7.998823in}{4.018455in}}%
\pgfpathlineto{\pgfqpoint{8.185049in}{4.018455in}}%
\pgfpathlineto{\pgfqpoint{8.185049in}{4.100183in}}%
\pgfpathlineto{\pgfqpoint{7.998823in}{4.100183in}}%
\pgfpathlineto{\pgfqpoint{7.998823in}{4.018455in}}%
\pgfusepath{}%
\end{pgfscope}%
\begin{pgfscope}%
\pgfpathrectangle{\pgfqpoint{0.549740in}{0.463273in}}{\pgfqpoint{9.320225in}{4.495057in}}%
\pgfusepath{clip}%
\pgfsetbuttcap%
\pgfsetroundjoin%
\pgfsetlinewidth{0.000000pt}%
\definecolor{currentstroke}{rgb}{0.000000,0.000000,0.000000}%
\pgfsetstrokecolor{currentstroke}%
\pgfsetdash{}{0pt}%
\pgfpathmoveto{\pgfqpoint{8.185049in}{4.018455in}}%
\pgfpathlineto{\pgfqpoint{8.371276in}{4.018455in}}%
\pgfpathlineto{\pgfqpoint{8.371276in}{4.100183in}}%
\pgfpathlineto{\pgfqpoint{8.185049in}{4.100183in}}%
\pgfpathlineto{\pgfqpoint{8.185049in}{4.018455in}}%
\pgfusepath{}%
\end{pgfscope}%
\begin{pgfscope}%
\pgfpathrectangle{\pgfqpoint{0.549740in}{0.463273in}}{\pgfqpoint{9.320225in}{4.495057in}}%
\pgfusepath{clip}%
\pgfsetbuttcap%
\pgfsetroundjoin%
\pgfsetlinewidth{0.000000pt}%
\definecolor{currentstroke}{rgb}{0.000000,0.000000,0.000000}%
\pgfsetstrokecolor{currentstroke}%
\pgfsetdash{}{0pt}%
\pgfpathmoveto{\pgfqpoint{8.371276in}{4.018455in}}%
\pgfpathlineto{\pgfqpoint{8.557503in}{4.018455in}}%
\pgfpathlineto{\pgfqpoint{8.557503in}{4.100183in}}%
\pgfpathlineto{\pgfqpoint{8.371276in}{4.100183in}}%
\pgfpathlineto{\pgfqpoint{8.371276in}{4.018455in}}%
\pgfusepath{}%
\end{pgfscope}%
\begin{pgfscope}%
\pgfpathrectangle{\pgfqpoint{0.549740in}{0.463273in}}{\pgfqpoint{9.320225in}{4.495057in}}%
\pgfusepath{clip}%
\pgfsetbuttcap%
\pgfsetroundjoin%
\pgfsetlinewidth{0.000000pt}%
\definecolor{currentstroke}{rgb}{0.000000,0.000000,0.000000}%
\pgfsetstrokecolor{currentstroke}%
\pgfsetdash{}{0pt}%
\pgfpathmoveto{\pgfqpoint{8.557503in}{4.018455in}}%
\pgfpathlineto{\pgfqpoint{8.743729in}{4.018455in}}%
\pgfpathlineto{\pgfqpoint{8.743729in}{4.100183in}}%
\pgfpathlineto{\pgfqpoint{8.557503in}{4.100183in}}%
\pgfpathlineto{\pgfqpoint{8.557503in}{4.018455in}}%
\pgfusepath{}%
\end{pgfscope}%
\begin{pgfscope}%
\pgfpathrectangle{\pgfqpoint{0.549740in}{0.463273in}}{\pgfqpoint{9.320225in}{4.495057in}}%
\pgfusepath{clip}%
\pgfsetbuttcap%
\pgfsetroundjoin%
\pgfsetlinewidth{0.000000pt}%
\definecolor{currentstroke}{rgb}{0.000000,0.000000,0.000000}%
\pgfsetstrokecolor{currentstroke}%
\pgfsetdash{}{0pt}%
\pgfpathmoveto{\pgfqpoint{8.743729in}{4.018455in}}%
\pgfpathlineto{\pgfqpoint{8.929956in}{4.018455in}}%
\pgfpathlineto{\pgfqpoint{8.929956in}{4.100183in}}%
\pgfpathlineto{\pgfqpoint{8.743729in}{4.100183in}}%
\pgfpathlineto{\pgfqpoint{8.743729in}{4.018455in}}%
\pgfusepath{}%
\end{pgfscope}%
\begin{pgfscope}%
\pgfpathrectangle{\pgfqpoint{0.549740in}{0.463273in}}{\pgfqpoint{9.320225in}{4.495057in}}%
\pgfusepath{clip}%
\pgfsetbuttcap%
\pgfsetroundjoin%
\pgfsetlinewidth{0.000000pt}%
\definecolor{currentstroke}{rgb}{0.000000,0.000000,0.000000}%
\pgfsetstrokecolor{currentstroke}%
\pgfsetdash{}{0pt}%
\pgfpathmoveto{\pgfqpoint{8.929956in}{4.018455in}}%
\pgfpathlineto{\pgfqpoint{9.116182in}{4.018455in}}%
\pgfpathlineto{\pgfqpoint{9.116182in}{4.100183in}}%
\pgfpathlineto{\pgfqpoint{8.929956in}{4.100183in}}%
\pgfpathlineto{\pgfqpoint{8.929956in}{4.018455in}}%
\pgfusepath{}%
\end{pgfscope}%
\begin{pgfscope}%
\pgfpathrectangle{\pgfqpoint{0.549740in}{0.463273in}}{\pgfqpoint{9.320225in}{4.495057in}}%
\pgfusepath{clip}%
\pgfsetbuttcap%
\pgfsetroundjoin%
\definecolor{currentfill}{rgb}{0.472869,0.711325,0.955316}%
\pgfsetfillcolor{currentfill}%
\pgfsetlinewidth{0.000000pt}%
\definecolor{currentstroke}{rgb}{0.000000,0.000000,0.000000}%
\pgfsetstrokecolor{currentstroke}%
\pgfsetdash{}{0pt}%
\pgfpathmoveto{\pgfqpoint{9.116182in}{4.018455in}}%
\pgfpathlineto{\pgfqpoint{9.302409in}{4.018455in}}%
\pgfpathlineto{\pgfqpoint{9.302409in}{4.100183in}}%
\pgfpathlineto{\pgfqpoint{9.116182in}{4.100183in}}%
\pgfpathlineto{\pgfqpoint{9.116182in}{4.018455in}}%
\pgfusepath{fill}%
\end{pgfscope}%
\begin{pgfscope}%
\pgfpathrectangle{\pgfqpoint{0.549740in}{0.463273in}}{\pgfqpoint{9.320225in}{4.495057in}}%
\pgfusepath{clip}%
\pgfsetbuttcap%
\pgfsetroundjoin%
\pgfsetlinewidth{0.000000pt}%
\definecolor{currentstroke}{rgb}{0.000000,0.000000,0.000000}%
\pgfsetstrokecolor{currentstroke}%
\pgfsetdash{}{0pt}%
\pgfpathmoveto{\pgfqpoint{9.302409in}{4.018455in}}%
\pgfpathlineto{\pgfqpoint{9.488635in}{4.018455in}}%
\pgfpathlineto{\pgfqpoint{9.488635in}{4.100183in}}%
\pgfpathlineto{\pgfqpoint{9.302409in}{4.100183in}}%
\pgfpathlineto{\pgfqpoint{9.302409in}{4.018455in}}%
\pgfusepath{}%
\end{pgfscope}%
\begin{pgfscope}%
\pgfpathrectangle{\pgfqpoint{0.549740in}{0.463273in}}{\pgfqpoint{9.320225in}{4.495057in}}%
\pgfusepath{clip}%
\pgfsetbuttcap%
\pgfsetroundjoin%
\pgfsetlinewidth{0.000000pt}%
\definecolor{currentstroke}{rgb}{0.000000,0.000000,0.000000}%
\pgfsetstrokecolor{currentstroke}%
\pgfsetdash{}{0pt}%
\pgfpathmoveto{\pgfqpoint{9.488635in}{4.018455in}}%
\pgfpathlineto{\pgfqpoint{9.674862in}{4.018455in}}%
\pgfpathlineto{\pgfqpoint{9.674862in}{4.100183in}}%
\pgfpathlineto{\pgfqpoint{9.488635in}{4.100183in}}%
\pgfpathlineto{\pgfqpoint{9.488635in}{4.018455in}}%
\pgfusepath{}%
\end{pgfscope}%
\begin{pgfscope}%
\pgfpathrectangle{\pgfqpoint{0.549740in}{0.463273in}}{\pgfqpoint{9.320225in}{4.495057in}}%
\pgfusepath{clip}%
\pgfsetbuttcap%
\pgfsetroundjoin%
\pgfsetlinewidth{0.000000pt}%
\definecolor{currentstroke}{rgb}{0.000000,0.000000,0.000000}%
\pgfsetstrokecolor{currentstroke}%
\pgfsetdash{}{0pt}%
\pgfpathmoveto{\pgfqpoint{9.674862in}{4.018455in}}%
\pgfpathlineto{\pgfqpoint{9.861088in}{4.018455in}}%
\pgfpathlineto{\pgfqpoint{9.861088in}{4.100183in}}%
\pgfpathlineto{\pgfqpoint{9.674862in}{4.100183in}}%
\pgfpathlineto{\pgfqpoint{9.674862in}{4.018455in}}%
\pgfusepath{}%
\end{pgfscope}%
\begin{pgfscope}%
\pgfpathrectangle{\pgfqpoint{0.549740in}{0.463273in}}{\pgfqpoint{9.320225in}{4.495057in}}%
\pgfusepath{clip}%
\pgfsetbuttcap%
\pgfsetroundjoin%
\pgfsetlinewidth{0.000000pt}%
\definecolor{currentstroke}{rgb}{0.000000,0.000000,0.000000}%
\pgfsetstrokecolor{currentstroke}%
\pgfsetdash{}{0pt}%
\pgfpathmoveto{\pgfqpoint{0.549761in}{4.100183in}}%
\pgfpathlineto{\pgfqpoint{0.735988in}{4.100183in}}%
\pgfpathlineto{\pgfqpoint{0.735988in}{4.181911in}}%
\pgfpathlineto{\pgfqpoint{0.549761in}{4.181911in}}%
\pgfpathlineto{\pgfqpoint{0.549761in}{4.100183in}}%
\pgfusepath{}%
\end{pgfscope}%
\begin{pgfscope}%
\pgfpathrectangle{\pgfqpoint{0.549740in}{0.463273in}}{\pgfqpoint{9.320225in}{4.495057in}}%
\pgfusepath{clip}%
\pgfsetbuttcap%
\pgfsetroundjoin%
\pgfsetlinewidth{0.000000pt}%
\definecolor{currentstroke}{rgb}{0.000000,0.000000,0.000000}%
\pgfsetstrokecolor{currentstroke}%
\pgfsetdash{}{0pt}%
\pgfpathmoveto{\pgfqpoint{0.735988in}{4.100183in}}%
\pgfpathlineto{\pgfqpoint{0.922214in}{4.100183in}}%
\pgfpathlineto{\pgfqpoint{0.922214in}{4.181911in}}%
\pgfpathlineto{\pgfqpoint{0.735988in}{4.181911in}}%
\pgfpathlineto{\pgfqpoint{0.735988in}{4.100183in}}%
\pgfusepath{}%
\end{pgfscope}%
\begin{pgfscope}%
\pgfpathrectangle{\pgfqpoint{0.549740in}{0.463273in}}{\pgfqpoint{9.320225in}{4.495057in}}%
\pgfusepath{clip}%
\pgfsetbuttcap%
\pgfsetroundjoin%
\pgfsetlinewidth{0.000000pt}%
\definecolor{currentstroke}{rgb}{0.000000,0.000000,0.000000}%
\pgfsetstrokecolor{currentstroke}%
\pgfsetdash{}{0pt}%
\pgfpathmoveto{\pgfqpoint{0.922214in}{4.100183in}}%
\pgfpathlineto{\pgfqpoint{1.108441in}{4.100183in}}%
\pgfpathlineto{\pgfqpoint{1.108441in}{4.181911in}}%
\pgfpathlineto{\pgfqpoint{0.922214in}{4.181911in}}%
\pgfpathlineto{\pgfqpoint{0.922214in}{4.100183in}}%
\pgfusepath{}%
\end{pgfscope}%
\begin{pgfscope}%
\pgfpathrectangle{\pgfqpoint{0.549740in}{0.463273in}}{\pgfqpoint{9.320225in}{4.495057in}}%
\pgfusepath{clip}%
\pgfsetbuttcap%
\pgfsetroundjoin%
\pgfsetlinewidth{0.000000pt}%
\definecolor{currentstroke}{rgb}{0.000000,0.000000,0.000000}%
\pgfsetstrokecolor{currentstroke}%
\pgfsetdash{}{0pt}%
\pgfpathmoveto{\pgfqpoint{1.108441in}{4.100183in}}%
\pgfpathlineto{\pgfqpoint{1.294667in}{4.100183in}}%
\pgfpathlineto{\pgfqpoint{1.294667in}{4.181911in}}%
\pgfpathlineto{\pgfqpoint{1.108441in}{4.181911in}}%
\pgfpathlineto{\pgfqpoint{1.108441in}{4.100183in}}%
\pgfusepath{}%
\end{pgfscope}%
\begin{pgfscope}%
\pgfpathrectangle{\pgfqpoint{0.549740in}{0.463273in}}{\pgfqpoint{9.320225in}{4.495057in}}%
\pgfusepath{clip}%
\pgfsetbuttcap%
\pgfsetroundjoin%
\pgfsetlinewidth{0.000000pt}%
\definecolor{currentstroke}{rgb}{0.000000,0.000000,0.000000}%
\pgfsetstrokecolor{currentstroke}%
\pgfsetdash{}{0pt}%
\pgfpathmoveto{\pgfqpoint{1.294667in}{4.100183in}}%
\pgfpathlineto{\pgfqpoint{1.480894in}{4.100183in}}%
\pgfpathlineto{\pgfqpoint{1.480894in}{4.181911in}}%
\pgfpathlineto{\pgfqpoint{1.294667in}{4.181911in}}%
\pgfpathlineto{\pgfqpoint{1.294667in}{4.100183in}}%
\pgfusepath{}%
\end{pgfscope}%
\begin{pgfscope}%
\pgfpathrectangle{\pgfqpoint{0.549740in}{0.463273in}}{\pgfqpoint{9.320225in}{4.495057in}}%
\pgfusepath{clip}%
\pgfsetbuttcap%
\pgfsetroundjoin%
\pgfsetlinewidth{0.000000pt}%
\definecolor{currentstroke}{rgb}{0.000000,0.000000,0.000000}%
\pgfsetstrokecolor{currentstroke}%
\pgfsetdash{}{0pt}%
\pgfpathmoveto{\pgfqpoint{1.480894in}{4.100183in}}%
\pgfpathlineto{\pgfqpoint{1.667120in}{4.100183in}}%
\pgfpathlineto{\pgfqpoint{1.667120in}{4.181911in}}%
\pgfpathlineto{\pgfqpoint{1.480894in}{4.181911in}}%
\pgfpathlineto{\pgfqpoint{1.480894in}{4.100183in}}%
\pgfusepath{}%
\end{pgfscope}%
\begin{pgfscope}%
\pgfpathrectangle{\pgfqpoint{0.549740in}{0.463273in}}{\pgfqpoint{9.320225in}{4.495057in}}%
\pgfusepath{clip}%
\pgfsetbuttcap%
\pgfsetroundjoin%
\pgfsetlinewidth{0.000000pt}%
\definecolor{currentstroke}{rgb}{0.000000,0.000000,0.000000}%
\pgfsetstrokecolor{currentstroke}%
\pgfsetdash{}{0pt}%
\pgfpathmoveto{\pgfqpoint{1.667120in}{4.100183in}}%
\pgfpathlineto{\pgfqpoint{1.853347in}{4.100183in}}%
\pgfpathlineto{\pgfqpoint{1.853347in}{4.181911in}}%
\pgfpathlineto{\pgfqpoint{1.667120in}{4.181911in}}%
\pgfpathlineto{\pgfqpoint{1.667120in}{4.100183in}}%
\pgfusepath{}%
\end{pgfscope}%
\begin{pgfscope}%
\pgfpathrectangle{\pgfqpoint{0.549740in}{0.463273in}}{\pgfqpoint{9.320225in}{4.495057in}}%
\pgfusepath{clip}%
\pgfsetbuttcap%
\pgfsetroundjoin%
\pgfsetlinewidth{0.000000pt}%
\definecolor{currentstroke}{rgb}{0.000000,0.000000,0.000000}%
\pgfsetstrokecolor{currentstroke}%
\pgfsetdash{}{0pt}%
\pgfpathmoveto{\pgfqpoint{1.853347in}{4.100183in}}%
\pgfpathlineto{\pgfqpoint{2.039573in}{4.100183in}}%
\pgfpathlineto{\pgfqpoint{2.039573in}{4.181911in}}%
\pgfpathlineto{\pgfqpoint{1.853347in}{4.181911in}}%
\pgfpathlineto{\pgfqpoint{1.853347in}{4.100183in}}%
\pgfusepath{}%
\end{pgfscope}%
\begin{pgfscope}%
\pgfpathrectangle{\pgfqpoint{0.549740in}{0.463273in}}{\pgfqpoint{9.320225in}{4.495057in}}%
\pgfusepath{clip}%
\pgfsetbuttcap%
\pgfsetroundjoin%
\pgfsetlinewidth{0.000000pt}%
\definecolor{currentstroke}{rgb}{0.000000,0.000000,0.000000}%
\pgfsetstrokecolor{currentstroke}%
\pgfsetdash{}{0pt}%
\pgfpathmoveto{\pgfqpoint{2.039573in}{4.100183in}}%
\pgfpathlineto{\pgfqpoint{2.225800in}{4.100183in}}%
\pgfpathlineto{\pgfqpoint{2.225800in}{4.181911in}}%
\pgfpathlineto{\pgfqpoint{2.039573in}{4.181911in}}%
\pgfpathlineto{\pgfqpoint{2.039573in}{4.100183in}}%
\pgfusepath{}%
\end{pgfscope}%
\begin{pgfscope}%
\pgfpathrectangle{\pgfqpoint{0.549740in}{0.463273in}}{\pgfqpoint{9.320225in}{4.495057in}}%
\pgfusepath{clip}%
\pgfsetbuttcap%
\pgfsetroundjoin%
\pgfsetlinewidth{0.000000pt}%
\definecolor{currentstroke}{rgb}{0.000000,0.000000,0.000000}%
\pgfsetstrokecolor{currentstroke}%
\pgfsetdash{}{0pt}%
\pgfpathmoveto{\pgfqpoint{2.225800in}{4.100183in}}%
\pgfpathlineto{\pgfqpoint{2.412027in}{4.100183in}}%
\pgfpathlineto{\pgfqpoint{2.412027in}{4.181911in}}%
\pgfpathlineto{\pgfqpoint{2.225800in}{4.181911in}}%
\pgfpathlineto{\pgfqpoint{2.225800in}{4.100183in}}%
\pgfusepath{}%
\end{pgfscope}%
\begin{pgfscope}%
\pgfpathrectangle{\pgfqpoint{0.549740in}{0.463273in}}{\pgfqpoint{9.320225in}{4.495057in}}%
\pgfusepath{clip}%
\pgfsetbuttcap%
\pgfsetroundjoin%
\pgfsetlinewidth{0.000000pt}%
\definecolor{currentstroke}{rgb}{0.000000,0.000000,0.000000}%
\pgfsetstrokecolor{currentstroke}%
\pgfsetdash{}{0pt}%
\pgfpathmoveto{\pgfqpoint{2.412027in}{4.100183in}}%
\pgfpathlineto{\pgfqpoint{2.598253in}{4.100183in}}%
\pgfpathlineto{\pgfqpoint{2.598253in}{4.181911in}}%
\pgfpathlineto{\pgfqpoint{2.412027in}{4.181911in}}%
\pgfpathlineto{\pgfqpoint{2.412027in}{4.100183in}}%
\pgfusepath{}%
\end{pgfscope}%
\begin{pgfscope}%
\pgfpathrectangle{\pgfqpoint{0.549740in}{0.463273in}}{\pgfqpoint{9.320225in}{4.495057in}}%
\pgfusepath{clip}%
\pgfsetbuttcap%
\pgfsetroundjoin%
\pgfsetlinewidth{0.000000pt}%
\definecolor{currentstroke}{rgb}{0.000000,0.000000,0.000000}%
\pgfsetstrokecolor{currentstroke}%
\pgfsetdash{}{0pt}%
\pgfpathmoveto{\pgfqpoint{2.598253in}{4.100183in}}%
\pgfpathlineto{\pgfqpoint{2.784480in}{4.100183in}}%
\pgfpathlineto{\pgfqpoint{2.784480in}{4.181911in}}%
\pgfpathlineto{\pgfqpoint{2.598253in}{4.181911in}}%
\pgfpathlineto{\pgfqpoint{2.598253in}{4.100183in}}%
\pgfusepath{}%
\end{pgfscope}%
\begin{pgfscope}%
\pgfpathrectangle{\pgfqpoint{0.549740in}{0.463273in}}{\pgfqpoint{9.320225in}{4.495057in}}%
\pgfusepath{clip}%
\pgfsetbuttcap%
\pgfsetroundjoin%
\pgfsetlinewidth{0.000000pt}%
\definecolor{currentstroke}{rgb}{0.000000,0.000000,0.000000}%
\pgfsetstrokecolor{currentstroke}%
\pgfsetdash{}{0pt}%
\pgfpathmoveto{\pgfqpoint{2.784480in}{4.100183in}}%
\pgfpathlineto{\pgfqpoint{2.970706in}{4.100183in}}%
\pgfpathlineto{\pgfqpoint{2.970706in}{4.181911in}}%
\pgfpathlineto{\pgfqpoint{2.784480in}{4.181911in}}%
\pgfpathlineto{\pgfqpoint{2.784480in}{4.100183in}}%
\pgfusepath{}%
\end{pgfscope}%
\begin{pgfscope}%
\pgfpathrectangle{\pgfqpoint{0.549740in}{0.463273in}}{\pgfqpoint{9.320225in}{4.495057in}}%
\pgfusepath{clip}%
\pgfsetbuttcap%
\pgfsetroundjoin%
\pgfsetlinewidth{0.000000pt}%
\definecolor{currentstroke}{rgb}{0.000000,0.000000,0.000000}%
\pgfsetstrokecolor{currentstroke}%
\pgfsetdash{}{0pt}%
\pgfpathmoveto{\pgfqpoint{2.970706in}{4.100183in}}%
\pgfpathlineto{\pgfqpoint{3.156933in}{4.100183in}}%
\pgfpathlineto{\pgfqpoint{3.156933in}{4.181911in}}%
\pgfpathlineto{\pgfqpoint{2.970706in}{4.181911in}}%
\pgfpathlineto{\pgfqpoint{2.970706in}{4.100183in}}%
\pgfusepath{}%
\end{pgfscope}%
\begin{pgfscope}%
\pgfpathrectangle{\pgfqpoint{0.549740in}{0.463273in}}{\pgfqpoint{9.320225in}{4.495057in}}%
\pgfusepath{clip}%
\pgfsetbuttcap%
\pgfsetroundjoin%
\pgfsetlinewidth{0.000000pt}%
\definecolor{currentstroke}{rgb}{0.000000,0.000000,0.000000}%
\pgfsetstrokecolor{currentstroke}%
\pgfsetdash{}{0pt}%
\pgfpathmoveto{\pgfqpoint{3.156933in}{4.100183in}}%
\pgfpathlineto{\pgfqpoint{3.343159in}{4.100183in}}%
\pgfpathlineto{\pgfqpoint{3.343159in}{4.181911in}}%
\pgfpathlineto{\pgfqpoint{3.156933in}{4.181911in}}%
\pgfpathlineto{\pgfqpoint{3.156933in}{4.100183in}}%
\pgfusepath{}%
\end{pgfscope}%
\begin{pgfscope}%
\pgfpathrectangle{\pgfqpoint{0.549740in}{0.463273in}}{\pgfqpoint{9.320225in}{4.495057in}}%
\pgfusepath{clip}%
\pgfsetbuttcap%
\pgfsetroundjoin%
\pgfsetlinewidth{0.000000pt}%
\definecolor{currentstroke}{rgb}{0.000000,0.000000,0.000000}%
\pgfsetstrokecolor{currentstroke}%
\pgfsetdash{}{0pt}%
\pgfpathmoveto{\pgfqpoint{3.343159in}{4.100183in}}%
\pgfpathlineto{\pgfqpoint{3.529386in}{4.100183in}}%
\pgfpathlineto{\pgfqpoint{3.529386in}{4.181911in}}%
\pgfpathlineto{\pgfqpoint{3.343159in}{4.181911in}}%
\pgfpathlineto{\pgfqpoint{3.343159in}{4.100183in}}%
\pgfusepath{}%
\end{pgfscope}%
\begin{pgfscope}%
\pgfpathrectangle{\pgfqpoint{0.549740in}{0.463273in}}{\pgfqpoint{9.320225in}{4.495057in}}%
\pgfusepath{clip}%
\pgfsetbuttcap%
\pgfsetroundjoin%
\pgfsetlinewidth{0.000000pt}%
\definecolor{currentstroke}{rgb}{0.000000,0.000000,0.000000}%
\pgfsetstrokecolor{currentstroke}%
\pgfsetdash{}{0pt}%
\pgfpathmoveto{\pgfqpoint{3.529386in}{4.100183in}}%
\pgfpathlineto{\pgfqpoint{3.715612in}{4.100183in}}%
\pgfpathlineto{\pgfqpoint{3.715612in}{4.181911in}}%
\pgfpathlineto{\pgfqpoint{3.529386in}{4.181911in}}%
\pgfpathlineto{\pgfqpoint{3.529386in}{4.100183in}}%
\pgfusepath{}%
\end{pgfscope}%
\begin{pgfscope}%
\pgfpathrectangle{\pgfqpoint{0.549740in}{0.463273in}}{\pgfqpoint{9.320225in}{4.495057in}}%
\pgfusepath{clip}%
\pgfsetbuttcap%
\pgfsetroundjoin%
\pgfsetlinewidth{0.000000pt}%
\definecolor{currentstroke}{rgb}{0.000000,0.000000,0.000000}%
\pgfsetstrokecolor{currentstroke}%
\pgfsetdash{}{0pt}%
\pgfpathmoveto{\pgfqpoint{3.715612in}{4.100183in}}%
\pgfpathlineto{\pgfqpoint{3.901839in}{4.100183in}}%
\pgfpathlineto{\pgfqpoint{3.901839in}{4.181911in}}%
\pgfpathlineto{\pgfqpoint{3.715612in}{4.181911in}}%
\pgfpathlineto{\pgfqpoint{3.715612in}{4.100183in}}%
\pgfusepath{}%
\end{pgfscope}%
\begin{pgfscope}%
\pgfpathrectangle{\pgfqpoint{0.549740in}{0.463273in}}{\pgfqpoint{9.320225in}{4.495057in}}%
\pgfusepath{clip}%
\pgfsetbuttcap%
\pgfsetroundjoin%
\pgfsetlinewidth{0.000000pt}%
\definecolor{currentstroke}{rgb}{0.000000,0.000000,0.000000}%
\pgfsetstrokecolor{currentstroke}%
\pgfsetdash{}{0pt}%
\pgfpathmoveto{\pgfqpoint{3.901839in}{4.100183in}}%
\pgfpathlineto{\pgfqpoint{4.088065in}{4.100183in}}%
\pgfpathlineto{\pgfqpoint{4.088065in}{4.181911in}}%
\pgfpathlineto{\pgfqpoint{3.901839in}{4.181911in}}%
\pgfpathlineto{\pgfqpoint{3.901839in}{4.100183in}}%
\pgfusepath{}%
\end{pgfscope}%
\begin{pgfscope}%
\pgfpathrectangle{\pgfqpoint{0.549740in}{0.463273in}}{\pgfqpoint{9.320225in}{4.495057in}}%
\pgfusepath{clip}%
\pgfsetbuttcap%
\pgfsetroundjoin%
\pgfsetlinewidth{0.000000pt}%
\definecolor{currentstroke}{rgb}{0.000000,0.000000,0.000000}%
\pgfsetstrokecolor{currentstroke}%
\pgfsetdash{}{0pt}%
\pgfpathmoveto{\pgfqpoint{4.088065in}{4.100183in}}%
\pgfpathlineto{\pgfqpoint{4.274292in}{4.100183in}}%
\pgfpathlineto{\pgfqpoint{4.274292in}{4.181911in}}%
\pgfpathlineto{\pgfqpoint{4.088065in}{4.181911in}}%
\pgfpathlineto{\pgfqpoint{4.088065in}{4.100183in}}%
\pgfusepath{}%
\end{pgfscope}%
\begin{pgfscope}%
\pgfpathrectangle{\pgfqpoint{0.549740in}{0.463273in}}{\pgfqpoint{9.320225in}{4.495057in}}%
\pgfusepath{clip}%
\pgfsetbuttcap%
\pgfsetroundjoin%
\pgfsetlinewidth{0.000000pt}%
\definecolor{currentstroke}{rgb}{0.000000,0.000000,0.000000}%
\pgfsetstrokecolor{currentstroke}%
\pgfsetdash{}{0pt}%
\pgfpathmoveto{\pgfqpoint{4.274292in}{4.100183in}}%
\pgfpathlineto{\pgfqpoint{4.460519in}{4.100183in}}%
\pgfpathlineto{\pgfqpoint{4.460519in}{4.181911in}}%
\pgfpathlineto{\pgfqpoint{4.274292in}{4.181911in}}%
\pgfpathlineto{\pgfqpoint{4.274292in}{4.100183in}}%
\pgfusepath{}%
\end{pgfscope}%
\begin{pgfscope}%
\pgfpathrectangle{\pgfqpoint{0.549740in}{0.463273in}}{\pgfqpoint{9.320225in}{4.495057in}}%
\pgfusepath{clip}%
\pgfsetbuttcap%
\pgfsetroundjoin%
\pgfsetlinewidth{0.000000pt}%
\definecolor{currentstroke}{rgb}{0.000000,0.000000,0.000000}%
\pgfsetstrokecolor{currentstroke}%
\pgfsetdash{}{0pt}%
\pgfpathmoveto{\pgfqpoint{4.460519in}{4.100183in}}%
\pgfpathlineto{\pgfqpoint{4.646745in}{4.100183in}}%
\pgfpathlineto{\pgfqpoint{4.646745in}{4.181911in}}%
\pgfpathlineto{\pgfqpoint{4.460519in}{4.181911in}}%
\pgfpathlineto{\pgfqpoint{4.460519in}{4.100183in}}%
\pgfusepath{}%
\end{pgfscope}%
\begin{pgfscope}%
\pgfpathrectangle{\pgfqpoint{0.549740in}{0.463273in}}{\pgfqpoint{9.320225in}{4.495057in}}%
\pgfusepath{clip}%
\pgfsetbuttcap%
\pgfsetroundjoin%
\pgfsetlinewidth{0.000000pt}%
\definecolor{currentstroke}{rgb}{0.000000,0.000000,0.000000}%
\pgfsetstrokecolor{currentstroke}%
\pgfsetdash{}{0pt}%
\pgfpathmoveto{\pgfqpoint{4.646745in}{4.100183in}}%
\pgfpathlineto{\pgfqpoint{4.832972in}{4.100183in}}%
\pgfpathlineto{\pgfqpoint{4.832972in}{4.181911in}}%
\pgfpathlineto{\pgfqpoint{4.646745in}{4.181911in}}%
\pgfpathlineto{\pgfqpoint{4.646745in}{4.100183in}}%
\pgfusepath{}%
\end{pgfscope}%
\begin{pgfscope}%
\pgfpathrectangle{\pgfqpoint{0.549740in}{0.463273in}}{\pgfqpoint{9.320225in}{4.495057in}}%
\pgfusepath{clip}%
\pgfsetbuttcap%
\pgfsetroundjoin%
\pgfsetlinewidth{0.000000pt}%
\definecolor{currentstroke}{rgb}{0.000000,0.000000,0.000000}%
\pgfsetstrokecolor{currentstroke}%
\pgfsetdash{}{0pt}%
\pgfpathmoveto{\pgfqpoint{4.832972in}{4.100183in}}%
\pgfpathlineto{\pgfqpoint{5.019198in}{4.100183in}}%
\pgfpathlineto{\pgfqpoint{5.019198in}{4.181911in}}%
\pgfpathlineto{\pgfqpoint{4.832972in}{4.181911in}}%
\pgfpathlineto{\pgfqpoint{4.832972in}{4.100183in}}%
\pgfusepath{}%
\end{pgfscope}%
\begin{pgfscope}%
\pgfpathrectangle{\pgfqpoint{0.549740in}{0.463273in}}{\pgfqpoint{9.320225in}{4.495057in}}%
\pgfusepath{clip}%
\pgfsetbuttcap%
\pgfsetroundjoin%
\pgfsetlinewidth{0.000000pt}%
\definecolor{currentstroke}{rgb}{0.000000,0.000000,0.000000}%
\pgfsetstrokecolor{currentstroke}%
\pgfsetdash{}{0pt}%
\pgfpathmoveto{\pgfqpoint{5.019198in}{4.100183in}}%
\pgfpathlineto{\pgfqpoint{5.205425in}{4.100183in}}%
\pgfpathlineto{\pgfqpoint{5.205425in}{4.181911in}}%
\pgfpathlineto{\pgfqpoint{5.019198in}{4.181911in}}%
\pgfpathlineto{\pgfqpoint{5.019198in}{4.100183in}}%
\pgfusepath{}%
\end{pgfscope}%
\begin{pgfscope}%
\pgfpathrectangle{\pgfqpoint{0.549740in}{0.463273in}}{\pgfqpoint{9.320225in}{4.495057in}}%
\pgfusepath{clip}%
\pgfsetbuttcap%
\pgfsetroundjoin%
\pgfsetlinewidth{0.000000pt}%
\definecolor{currentstroke}{rgb}{0.000000,0.000000,0.000000}%
\pgfsetstrokecolor{currentstroke}%
\pgfsetdash{}{0pt}%
\pgfpathmoveto{\pgfqpoint{5.205425in}{4.100183in}}%
\pgfpathlineto{\pgfqpoint{5.391651in}{4.100183in}}%
\pgfpathlineto{\pgfqpoint{5.391651in}{4.181911in}}%
\pgfpathlineto{\pgfqpoint{5.205425in}{4.181911in}}%
\pgfpathlineto{\pgfqpoint{5.205425in}{4.100183in}}%
\pgfusepath{}%
\end{pgfscope}%
\begin{pgfscope}%
\pgfpathrectangle{\pgfqpoint{0.549740in}{0.463273in}}{\pgfqpoint{9.320225in}{4.495057in}}%
\pgfusepath{clip}%
\pgfsetbuttcap%
\pgfsetroundjoin%
\pgfsetlinewidth{0.000000pt}%
\definecolor{currentstroke}{rgb}{0.000000,0.000000,0.000000}%
\pgfsetstrokecolor{currentstroke}%
\pgfsetdash{}{0pt}%
\pgfpathmoveto{\pgfqpoint{5.391651in}{4.100183in}}%
\pgfpathlineto{\pgfqpoint{5.577878in}{4.100183in}}%
\pgfpathlineto{\pgfqpoint{5.577878in}{4.181911in}}%
\pgfpathlineto{\pgfqpoint{5.391651in}{4.181911in}}%
\pgfpathlineto{\pgfqpoint{5.391651in}{4.100183in}}%
\pgfusepath{}%
\end{pgfscope}%
\begin{pgfscope}%
\pgfpathrectangle{\pgfqpoint{0.549740in}{0.463273in}}{\pgfqpoint{9.320225in}{4.495057in}}%
\pgfusepath{clip}%
\pgfsetbuttcap%
\pgfsetroundjoin%
\pgfsetlinewidth{0.000000pt}%
\definecolor{currentstroke}{rgb}{0.000000,0.000000,0.000000}%
\pgfsetstrokecolor{currentstroke}%
\pgfsetdash{}{0pt}%
\pgfpathmoveto{\pgfqpoint{5.577878in}{4.100183in}}%
\pgfpathlineto{\pgfqpoint{5.764104in}{4.100183in}}%
\pgfpathlineto{\pgfqpoint{5.764104in}{4.181911in}}%
\pgfpathlineto{\pgfqpoint{5.577878in}{4.181911in}}%
\pgfpathlineto{\pgfqpoint{5.577878in}{4.100183in}}%
\pgfusepath{}%
\end{pgfscope}%
\begin{pgfscope}%
\pgfpathrectangle{\pgfqpoint{0.549740in}{0.463273in}}{\pgfqpoint{9.320225in}{4.495057in}}%
\pgfusepath{clip}%
\pgfsetbuttcap%
\pgfsetroundjoin%
\pgfsetlinewidth{0.000000pt}%
\definecolor{currentstroke}{rgb}{0.000000,0.000000,0.000000}%
\pgfsetstrokecolor{currentstroke}%
\pgfsetdash{}{0pt}%
\pgfpathmoveto{\pgfqpoint{5.764104in}{4.100183in}}%
\pgfpathlineto{\pgfqpoint{5.950331in}{4.100183in}}%
\pgfpathlineto{\pgfqpoint{5.950331in}{4.181911in}}%
\pgfpathlineto{\pgfqpoint{5.764104in}{4.181911in}}%
\pgfpathlineto{\pgfqpoint{5.764104in}{4.100183in}}%
\pgfusepath{}%
\end{pgfscope}%
\begin{pgfscope}%
\pgfpathrectangle{\pgfqpoint{0.549740in}{0.463273in}}{\pgfqpoint{9.320225in}{4.495057in}}%
\pgfusepath{clip}%
\pgfsetbuttcap%
\pgfsetroundjoin%
\pgfsetlinewidth{0.000000pt}%
\definecolor{currentstroke}{rgb}{0.000000,0.000000,0.000000}%
\pgfsetstrokecolor{currentstroke}%
\pgfsetdash{}{0pt}%
\pgfpathmoveto{\pgfqpoint{5.950331in}{4.100183in}}%
\pgfpathlineto{\pgfqpoint{6.136557in}{4.100183in}}%
\pgfpathlineto{\pgfqpoint{6.136557in}{4.181911in}}%
\pgfpathlineto{\pgfqpoint{5.950331in}{4.181911in}}%
\pgfpathlineto{\pgfqpoint{5.950331in}{4.100183in}}%
\pgfusepath{}%
\end{pgfscope}%
\begin{pgfscope}%
\pgfpathrectangle{\pgfqpoint{0.549740in}{0.463273in}}{\pgfqpoint{9.320225in}{4.495057in}}%
\pgfusepath{clip}%
\pgfsetbuttcap%
\pgfsetroundjoin%
\pgfsetlinewidth{0.000000pt}%
\definecolor{currentstroke}{rgb}{0.000000,0.000000,0.000000}%
\pgfsetstrokecolor{currentstroke}%
\pgfsetdash{}{0pt}%
\pgfpathmoveto{\pgfqpoint{6.136557in}{4.100183in}}%
\pgfpathlineto{\pgfqpoint{6.322784in}{4.100183in}}%
\pgfpathlineto{\pgfqpoint{6.322784in}{4.181911in}}%
\pgfpathlineto{\pgfqpoint{6.136557in}{4.181911in}}%
\pgfpathlineto{\pgfqpoint{6.136557in}{4.100183in}}%
\pgfusepath{}%
\end{pgfscope}%
\begin{pgfscope}%
\pgfpathrectangle{\pgfqpoint{0.549740in}{0.463273in}}{\pgfqpoint{9.320225in}{4.495057in}}%
\pgfusepath{clip}%
\pgfsetbuttcap%
\pgfsetroundjoin%
\pgfsetlinewidth{0.000000pt}%
\definecolor{currentstroke}{rgb}{0.000000,0.000000,0.000000}%
\pgfsetstrokecolor{currentstroke}%
\pgfsetdash{}{0pt}%
\pgfpathmoveto{\pgfqpoint{6.322784in}{4.100183in}}%
\pgfpathlineto{\pgfqpoint{6.509011in}{4.100183in}}%
\pgfpathlineto{\pgfqpoint{6.509011in}{4.181911in}}%
\pgfpathlineto{\pgfqpoint{6.322784in}{4.181911in}}%
\pgfpathlineto{\pgfqpoint{6.322784in}{4.100183in}}%
\pgfusepath{}%
\end{pgfscope}%
\begin{pgfscope}%
\pgfpathrectangle{\pgfqpoint{0.549740in}{0.463273in}}{\pgfqpoint{9.320225in}{4.495057in}}%
\pgfusepath{clip}%
\pgfsetbuttcap%
\pgfsetroundjoin%
\pgfsetlinewidth{0.000000pt}%
\definecolor{currentstroke}{rgb}{0.000000,0.000000,0.000000}%
\pgfsetstrokecolor{currentstroke}%
\pgfsetdash{}{0pt}%
\pgfpathmoveto{\pgfqpoint{6.509011in}{4.100183in}}%
\pgfpathlineto{\pgfqpoint{6.695237in}{4.100183in}}%
\pgfpathlineto{\pgfqpoint{6.695237in}{4.181911in}}%
\pgfpathlineto{\pgfqpoint{6.509011in}{4.181911in}}%
\pgfpathlineto{\pgfqpoint{6.509011in}{4.100183in}}%
\pgfusepath{}%
\end{pgfscope}%
\begin{pgfscope}%
\pgfpathrectangle{\pgfqpoint{0.549740in}{0.463273in}}{\pgfqpoint{9.320225in}{4.495057in}}%
\pgfusepath{clip}%
\pgfsetbuttcap%
\pgfsetroundjoin%
\pgfsetlinewidth{0.000000pt}%
\definecolor{currentstroke}{rgb}{0.000000,0.000000,0.000000}%
\pgfsetstrokecolor{currentstroke}%
\pgfsetdash{}{0pt}%
\pgfpathmoveto{\pgfqpoint{6.695237in}{4.100183in}}%
\pgfpathlineto{\pgfqpoint{6.881464in}{4.100183in}}%
\pgfpathlineto{\pgfqpoint{6.881464in}{4.181911in}}%
\pgfpathlineto{\pgfqpoint{6.695237in}{4.181911in}}%
\pgfpathlineto{\pgfqpoint{6.695237in}{4.100183in}}%
\pgfusepath{}%
\end{pgfscope}%
\begin{pgfscope}%
\pgfpathrectangle{\pgfqpoint{0.549740in}{0.463273in}}{\pgfqpoint{9.320225in}{4.495057in}}%
\pgfusepath{clip}%
\pgfsetbuttcap%
\pgfsetroundjoin%
\pgfsetlinewidth{0.000000pt}%
\definecolor{currentstroke}{rgb}{0.000000,0.000000,0.000000}%
\pgfsetstrokecolor{currentstroke}%
\pgfsetdash{}{0pt}%
\pgfpathmoveto{\pgfqpoint{6.881464in}{4.100183in}}%
\pgfpathlineto{\pgfqpoint{7.067690in}{4.100183in}}%
\pgfpathlineto{\pgfqpoint{7.067690in}{4.181911in}}%
\pgfpathlineto{\pgfqpoint{6.881464in}{4.181911in}}%
\pgfpathlineto{\pgfqpoint{6.881464in}{4.100183in}}%
\pgfusepath{}%
\end{pgfscope}%
\begin{pgfscope}%
\pgfpathrectangle{\pgfqpoint{0.549740in}{0.463273in}}{\pgfqpoint{9.320225in}{4.495057in}}%
\pgfusepath{clip}%
\pgfsetbuttcap%
\pgfsetroundjoin%
\pgfsetlinewidth{0.000000pt}%
\definecolor{currentstroke}{rgb}{0.000000,0.000000,0.000000}%
\pgfsetstrokecolor{currentstroke}%
\pgfsetdash{}{0pt}%
\pgfpathmoveto{\pgfqpoint{7.067690in}{4.100183in}}%
\pgfpathlineto{\pgfqpoint{7.253917in}{4.100183in}}%
\pgfpathlineto{\pgfqpoint{7.253917in}{4.181911in}}%
\pgfpathlineto{\pgfqpoint{7.067690in}{4.181911in}}%
\pgfpathlineto{\pgfqpoint{7.067690in}{4.100183in}}%
\pgfusepath{}%
\end{pgfscope}%
\begin{pgfscope}%
\pgfpathrectangle{\pgfqpoint{0.549740in}{0.463273in}}{\pgfqpoint{9.320225in}{4.495057in}}%
\pgfusepath{clip}%
\pgfsetbuttcap%
\pgfsetroundjoin%
\pgfsetlinewidth{0.000000pt}%
\definecolor{currentstroke}{rgb}{0.000000,0.000000,0.000000}%
\pgfsetstrokecolor{currentstroke}%
\pgfsetdash{}{0pt}%
\pgfpathmoveto{\pgfqpoint{7.253917in}{4.100183in}}%
\pgfpathlineto{\pgfqpoint{7.440143in}{4.100183in}}%
\pgfpathlineto{\pgfqpoint{7.440143in}{4.181911in}}%
\pgfpathlineto{\pgfqpoint{7.253917in}{4.181911in}}%
\pgfpathlineto{\pgfqpoint{7.253917in}{4.100183in}}%
\pgfusepath{}%
\end{pgfscope}%
\begin{pgfscope}%
\pgfpathrectangle{\pgfqpoint{0.549740in}{0.463273in}}{\pgfqpoint{9.320225in}{4.495057in}}%
\pgfusepath{clip}%
\pgfsetbuttcap%
\pgfsetroundjoin%
\pgfsetlinewidth{0.000000pt}%
\definecolor{currentstroke}{rgb}{0.000000,0.000000,0.000000}%
\pgfsetstrokecolor{currentstroke}%
\pgfsetdash{}{0pt}%
\pgfpathmoveto{\pgfqpoint{7.440143in}{4.100183in}}%
\pgfpathlineto{\pgfqpoint{7.626370in}{4.100183in}}%
\pgfpathlineto{\pgfqpoint{7.626370in}{4.181911in}}%
\pgfpathlineto{\pgfqpoint{7.440143in}{4.181911in}}%
\pgfpathlineto{\pgfqpoint{7.440143in}{4.100183in}}%
\pgfusepath{}%
\end{pgfscope}%
\begin{pgfscope}%
\pgfpathrectangle{\pgfqpoint{0.549740in}{0.463273in}}{\pgfqpoint{9.320225in}{4.495057in}}%
\pgfusepath{clip}%
\pgfsetbuttcap%
\pgfsetroundjoin%
\pgfsetlinewidth{0.000000pt}%
\definecolor{currentstroke}{rgb}{0.000000,0.000000,0.000000}%
\pgfsetstrokecolor{currentstroke}%
\pgfsetdash{}{0pt}%
\pgfpathmoveto{\pgfqpoint{7.626370in}{4.100183in}}%
\pgfpathlineto{\pgfqpoint{7.812596in}{4.100183in}}%
\pgfpathlineto{\pgfqpoint{7.812596in}{4.181911in}}%
\pgfpathlineto{\pgfqpoint{7.626370in}{4.181911in}}%
\pgfpathlineto{\pgfqpoint{7.626370in}{4.100183in}}%
\pgfusepath{}%
\end{pgfscope}%
\begin{pgfscope}%
\pgfpathrectangle{\pgfqpoint{0.549740in}{0.463273in}}{\pgfqpoint{9.320225in}{4.495057in}}%
\pgfusepath{clip}%
\pgfsetbuttcap%
\pgfsetroundjoin%
\definecolor{currentfill}{rgb}{0.614330,0.761948,0.940009}%
\pgfsetfillcolor{currentfill}%
\pgfsetlinewidth{0.000000pt}%
\definecolor{currentstroke}{rgb}{0.000000,0.000000,0.000000}%
\pgfsetstrokecolor{currentstroke}%
\pgfsetdash{}{0pt}%
\pgfpathmoveto{\pgfqpoint{7.812596in}{4.100183in}}%
\pgfpathlineto{\pgfqpoint{7.998823in}{4.100183in}}%
\pgfpathlineto{\pgfqpoint{7.998823in}{4.181911in}}%
\pgfpathlineto{\pgfqpoint{7.812596in}{4.181911in}}%
\pgfpathlineto{\pgfqpoint{7.812596in}{4.100183in}}%
\pgfusepath{fill}%
\end{pgfscope}%
\begin{pgfscope}%
\pgfpathrectangle{\pgfqpoint{0.549740in}{0.463273in}}{\pgfqpoint{9.320225in}{4.495057in}}%
\pgfusepath{clip}%
\pgfsetbuttcap%
\pgfsetroundjoin%
\pgfsetlinewidth{0.000000pt}%
\definecolor{currentstroke}{rgb}{0.000000,0.000000,0.000000}%
\pgfsetstrokecolor{currentstroke}%
\pgfsetdash{}{0pt}%
\pgfpathmoveto{\pgfqpoint{7.998823in}{4.100183in}}%
\pgfpathlineto{\pgfqpoint{8.185049in}{4.100183in}}%
\pgfpathlineto{\pgfqpoint{8.185049in}{4.181911in}}%
\pgfpathlineto{\pgfqpoint{7.998823in}{4.181911in}}%
\pgfpathlineto{\pgfqpoint{7.998823in}{4.100183in}}%
\pgfusepath{}%
\end{pgfscope}%
\begin{pgfscope}%
\pgfpathrectangle{\pgfqpoint{0.549740in}{0.463273in}}{\pgfqpoint{9.320225in}{4.495057in}}%
\pgfusepath{clip}%
\pgfsetbuttcap%
\pgfsetroundjoin%
\pgfsetlinewidth{0.000000pt}%
\definecolor{currentstroke}{rgb}{0.000000,0.000000,0.000000}%
\pgfsetstrokecolor{currentstroke}%
\pgfsetdash{}{0pt}%
\pgfpathmoveto{\pgfqpoint{8.185049in}{4.100183in}}%
\pgfpathlineto{\pgfqpoint{8.371276in}{4.100183in}}%
\pgfpathlineto{\pgfqpoint{8.371276in}{4.181911in}}%
\pgfpathlineto{\pgfqpoint{8.185049in}{4.181911in}}%
\pgfpathlineto{\pgfqpoint{8.185049in}{4.100183in}}%
\pgfusepath{}%
\end{pgfscope}%
\begin{pgfscope}%
\pgfpathrectangle{\pgfqpoint{0.549740in}{0.463273in}}{\pgfqpoint{9.320225in}{4.495057in}}%
\pgfusepath{clip}%
\pgfsetbuttcap%
\pgfsetroundjoin%
\pgfsetlinewidth{0.000000pt}%
\definecolor{currentstroke}{rgb}{0.000000,0.000000,0.000000}%
\pgfsetstrokecolor{currentstroke}%
\pgfsetdash{}{0pt}%
\pgfpathmoveto{\pgfqpoint{8.371276in}{4.100183in}}%
\pgfpathlineto{\pgfqpoint{8.557503in}{4.100183in}}%
\pgfpathlineto{\pgfqpoint{8.557503in}{4.181911in}}%
\pgfpathlineto{\pgfqpoint{8.371276in}{4.181911in}}%
\pgfpathlineto{\pgfqpoint{8.371276in}{4.100183in}}%
\pgfusepath{}%
\end{pgfscope}%
\begin{pgfscope}%
\pgfpathrectangle{\pgfqpoint{0.549740in}{0.463273in}}{\pgfqpoint{9.320225in}{4.495057in}}%
\pgfusepath{clip}%
\pgfsetbuttcap%
\pgfsetroundjoin%
\pgfsetlinewidth{0.000000pt}%
\definecolor{currentstroke}{rgb}{0.000000,0.000000,0.000000}%
\pgfsetstrokecolor{currentstroke}%
\pgfsetdash{}{0pt}%
\pgfpathmoveto{\pgfqpoint{8.557503in}{4.100183in}}%
\pgfpathlineto{\pgfqpoint{8.743729in}{4.100183in}}%
\pgfpathlineto{\pgfqpoint{8.743729in}{4.181911in}}%
\pgfpathlineto{\pgfqpoint{8.557503in}{4.181911in}}%
\pgfpathlineto{\pgfqpoint{8.557503in}{4.100183in}}%
\pgfusepath{}%
\end{pgfscope}%
\begin{pgfscope}%
\pgfpathrectangle{\pgfqpoint{0.549740in}{0.463273in}}{\pgfqpoint{9.320225in}{4.495057in}}%
\pgfusepath{clip}%
\pgfsetbuttcap%
\pgfsetroundjoin%
\pgfsetlinewidth{0.000000pt}%
\definecolor{currentstroke}{rgb}{0.000000,0.000000,0.000000}%
\pgfsetstrokecolor{currentstroke}%
\pgfsetdash{}{0pt}%
\pgfpathmoveto{\pgfqpoint{8.743729in}{4.100183in}}%
\pgfpathlineto{\pgfqpoint{8.929956in}{4.100183in}}%
\pgfpathlineto{\pgfqpoint{8.929956in}{4.181911in}}%
\pgfpathlineto{\pgfqpoint{8.743729in}{4.181911in}}%
\pgfpathlineto{\pgfqpoint{8.743729in}{4.100183in}}%
\pgfusepath{}%
\end{pgfscope}%
\begin{pgfscope}%
\pgfpathrectangle{\pgfqpoint{0.549740in}{0.463273in}}{\pgfqpoint{9.320225in}{4.495057in}}%
\pgfusepath{clip}%
\pgfsetbuttcap%
\pgfsetroundjoin%
\pgfsetlinewidth{0.000000pt}%
\definecolor{currentstroke}{rgb}{0.000000,0.000000,0.000000}%
\pgfsetstrokecolor{currentstroke}%
\pgfsetdash{}{0pt}%
\pgfpathmoveto{\pgfqpoint{8.929956in}{4.100183in}}%
\pgfpathlineto{\pgfqpoint{9.116182in}{4.100183in}}%
\pgfpathlineto{\pgfqpoint{9.116182in}{4.181911in}}%
\pgfpathlineto{\pgfqpoint{8.929956in}{4.181911in}}%
\pgfpathlineto{\pgfqpoint{8.929956in}{4.100183in}}%
\pgfusepath{}%
\end{pgfscope}%
\begin{pgfscope}%
\pgfpathrectangle{\pgfqpoint{0.549740in}{0.463273in}}{\pgfqpoint{9.320225in}{4.495057in}}%
\pgfusepath{clip}%
\pgfsetbuttcap%
\pgfsetroundjoin%
\definecolor{currentfill}{rgb}{0.472869,0.711325,0.955316}%
\pgfsetfillcolor{currentfill}%
\pgfsetlinewidth{0.000000pt}%
\definecolor{currentstroke}{rgb}{0.000000,0.000000,0.000000}%
\pgfsetstrokecolor{currentstroke}%
\pgfsetdash{}{0pt}%
\pgfpathmoveto{\pgfqpoint{9.116182in}{4.100183in}}%
\pgfpathlineto{\pgfqpoint{9.302409in}{4.100183in}}%
\pgfpathlineto{\pgfqpoint{9.302409in}{4.181911in}}%
\pgfpathlineto{\pgfqpoint{9.116182in}{4.181911in}}%
\pgfpathlineto{\pgfqpoint{9.116182in}{4.100183in}}%
\pgfusepath{fill}%
\end{pgfscope}%
\begin{pgfscope}%
\pgfpathrectangle{\pgfqpoint{0.549740in}{0.463273in}}{\pgfqpoint{9.320225in}{4.495057in}}%
\pgfusepath{clip}%
\pgfsetbuttcap%
\pgfsetroundjoin%
\pgfsetlinewidth{0.000000pt}%
\definecolor{currentstroke}{rgb}{0.000000,0.000000,0.000000}%
\pgfsetstrokecolor{currentstroke}%
\pgfsetdash{}{0pt}%
\pgfpathmoveto{\pgfqpoint{9.302409in}{4.100183in}}%
\pgfpathlineto{\pgfqpoint{9.488635in}{4.100183in}}%
\pgfpathlineto{\pgfqpoint{9.488635in}{4.181911in}}%
\pgfpathlineto{\pgfqpoint{9.302409in}{4.181911in}}%
\pgfpathlineto{\pgfqpoint{9.302409in}{4.100183in}}%
\pgfusepath{}%
\end{pgfscope}%
\begin{pgfscope}%
\pgfpathrectangle{\pgfqpoint{0.549740in}{0.463273in}}{\pgfqpoint{9.320225in}{4.495057in}}%
\pgfusepath{clip}%
\pgfsetbuttcap%
\pgfsetroundjoin%
\pgfsetlinewidth{0.000000pt}%
\definecolor{currentstroke}{rgb}{0.000000,0.000000,0.000000}%
\pgfsetstrokecolor{currentstroke}%
\pgfsetdash{}{0pt}%
\pgfpathmoveto{\pgfqpoint{9.488635in}{4.100183in}}%
\pgfpathlineto{\pgfqpoint{9.674862in}{4.100183in}}%
\pgfpathlineto{\pgfqpoint{9.674862in}{4.181911in}}%
\pgfpathlineto{\pgfqpoint{9.488635in}{4.181911in}}%
\pgfpathlineto{\pgfqpoint{9.488635in}{4.100183in}}%
\pgfusepath{}%
\end{pgfscope}%
\begin{pgfscope}%
\pgfpathrectangle{\pgfqpoint{0.549740in}{0.463273in}}{\pgfqpoint{9.320225in}{4.495057in}}%
\pgfusepath{clip}%
\pgfsetbuttcap%
\pgfsetroundjoin%
\pgfsetlinewidth{0.000000pt}%
\definecolor{currentstroke}{rgb}{0.000000,0.000000,0.000000}%
\pgfsetstrokecolor{currentstroke}%
\pgfsetdash{}{0pt}%
\pgfpathmoveto{\pgfqpoint{9.674862in}{4.100183in}}%
\pgfpathlineto{\pgfqpoint{9.861088in}{4.100183in}}%
\pgfpathlineto{\pgfqpoint{9.861088in}{4.181911in}}%
\pgfpathlineto{\pgfqpoint{9.674862in}{4.181911in}}%
\pgfpathlineto{\pgfqpoint{9.674862in}{4.100183in}}%
\pgfusepath{}%
\end{pgfscope}%
\begin{pgfscope}%
\pgfpathrectangle{\pgfqpoint{0.549740in}{0.463273in}}{\pgfqpoint{9.320225in}{4.495057in}}%
\pgfusepath{clip}%
\pgfsetbuttcap%
\pgfsetroundjoin%
\pgfsetlinewidth{0.000000pt}%
\definecolor{currentstroke}{rgb}{0.000000,0.000000,0.000000}%
\pgfsetstrokecolor{currentstroke}%
\pgfsetdash{}{0pt}%
\pgfpathmoveto{\pgfqpoint{0.549761in}{4.181911in}}%
\pgfpathlineto{\pgfqpoint{0.735988in}{4.181911in}}%
\pgfpathlineto{\pgfqpoint{0.735988in}{4.263639in}}%
\pgfpathlineto{\pgfqpoint{0.549761in}{4.263639in}}%
\pgfpathlineto{\pgfqpoint{0.549761in}{4.181911in}}%
\pgfusepath{}%
\end{pgfscope}%
\begin{pgfscope}%
\pgfpathrectangle{\pgfqpoint{0.549740in}{0.463273in}}{\pgfqpoint{9.320225in}{4.495057in}}%
\pgfusepath{clip}%
\pgfsetbuttcap%
\pgfsetroundjoin%
\pgfsetlinewidth{0.000000pt}%
\definecolor{currentstroke}{rgb}{0.000000,0.000000,0.000000}%
\pgfsetstrokecolor{currentstroke}%
\pgfsetdash{}{0pt}%
\pgfpathmoveto{\pgfqpoint{0.735988in}{4.181911in}}%
\pgfpathlineto{\pgfqpoint{0.922214in}{4.181911in}}%
\pgfpathlineto{\pgfqpoint{0.922214in}{4.263639in}}%
\pgfpathlineto{\pgfqpoint{0.735988in}{4.263639in}}%
\pgfpathlineto{\pgfqpoint{0.735988in}{4.181911in}}%
\pgfusepath{}%
\end{pgfscope}%
\begin{pgfscope}%
\pgfpathrectangle{\pgfqpoint{0.549740in}{0.463273in}}{\pgfqpoint{9.320225in}{4.495057in}}%
\pgfusepath{clip}%
\pgfsetbuttcap%
\pgfsetroundjoin%
\pgfsetlinewidth{0.000000pt}%
\definecolor{currentstroke}{rgb}{0.000000,0.000000,0.000000}%
\pgfsetstrokecolor{currentstroke}%
\pgfsetdash{}{0pt}%
\pgfpathmoveto{\pgfqpoint{0.922214in}{4.181911in}}%
\pgfpathlineto{\pgfqpoint{1.108441in}{4.181911in}}%
\pgfpathlineto{\pgfqpoint{1.108441in}{4.263639in}}%
\pgfpathlineto{\pgfqpoint{0.922214in}{4.263639in}}%
\pgfpathlineto{\pgfqpoint{0.922214in}{4.181911in}}%
\pgfusepath{}%
\end{pgfscope}%
\begin{pgfscope}%
\pgfpathrectangle{\pgfqpoint{0.549740in}{0.463273in}}{\pgfqpoint{9.320225in}{4.495057in}}%
\pgfusepath{clip}%
\pgfsetbuttcap%
\pgfsetroundjoin%
\pgfsetlinewidth{0.000000pt}%
\definecolor{currentstroke}{rgb}{0.000000,0.000000,0.000000}%
\pgfsetstrokecolor{currentstroke}%
\pgfsetdash{}{0pt}%
\pgfpathmoveto{\pgfqpoint{1.108441in}{4.181911in}}%
\pgfpathlineto{\pgfqpoint{1.294667in}{4.181911in}}%
\pgfpathlineto{\pgfqpoint{1.294667in}{4.263639in}}%
\pgfpathlineto{\pgfqpoint{1.108441in}{4.263639in}}%
\pgfpathlineto{\pgfqpoint{1.108441in}{4.181911in}}%
\pgfusepath{}%
\end{pgfscope}%
\begin{pgfscope}%
\pgfpathrectangle{\pgfqpoint{0.549740in}{0.463273in}}{\pgfqpoint{9.320225in}{4.495057in}}%
\pgfusepath{clip}%
\pgfsetbuttcap%
\pgfsetroundjoin%
\pgfsetlinewidth{0.000000pt}%
\definecolor{currentstroke}{rgb}{0.000000,0.000000,0.000000}%
\pgfsetstrokecolor{currentstroke}%
\pgfsetdash{}{0pt}%
\pgfpathmoveto{\pgfqpoint{1.294667in}{4.181911in}}%
\pgfpathlineto{\pgfqpoint{1.480894in}{4.181911in}}%
\pgfpathlineto{\pgfqpoint{1.480894in}{4.263639in}}%
\pgfpathlineto{\pgfqpoint{1.294667in}{4.263639in}}%
\pgfpathlineto{\pgfqpoint{1.294667in}{4.181911in}}%
\pgfusepath{}%
\end{pgfscope}%
\begin{pgfscope}%
\pgfpathrectangle{\pgfqpoint{0.549740in}{0.463273in}}{\pgfqpoint{9.320225in}{4.495057in}}%
\pgfusepath{clip}%
\pgfsetbuttcap%
\pgfsetroundjoin%
\pgfsetlinewidth{0.000000pt}%
\definecolor{currentstroke}{rgb}{0.000000,0.000000,0.000000}%
\pgfsetstrokecolor{currentstroke}%
\pgfsetdash{}{0pt}%
\pgfpathmoveto{\pgfqpoint{1.480894in}{4.181911in}}%
\pgfpathlineto{\pgfqpoint{1.667120in}{4.181911in}}%
\pgfpathlineto{\pgfqpoint{1.667120in}{4.263639in}}%
\pgfpathlineto{\pgfqpoint{1.480894in}{4.263639in}}%
\pgfpathlineto{\pgfqpoint{1.480894in}{4.181911in}}%
\pgfusepath{}%
\end{pgfscope}%
\begin{pgfscope}%
\pgfpathrectangle{\pgfqpoint{0.549740in}{0.463273in}}{\pgfqpoint{9.320225in}{4.495057in}}%
\pgfusepath{clip}%
\pgfsetbuttcap%
\pgfsetroundjoin%
\pgfsetlinewidth{0.000000pt}%
\definecolor{currentstroke}{rgb}{0.000000,0.000000,0.000000}%
\pgfsetstrokecolor{currentstroke}%
\pgfsetdash{}{0pt}%
\pgfpathmoveto{\pgfqpoint{1.667120in}{4.181911in}}%
\pgfpathlineto{\pgfqpoint{1.853347in}{4.181911in}}%
\pgfpathlineto{\pgfqpoint{1.853347in}{4.263639in}}%
\pgfpathlineto{\pgfqpoint{1.667120in}{4.263639in}}%
\pgfpathlineto{\pgfqpoint{1.667120in}{4.181911in}}%
\pgfusepath{}%
\end{pgfscope}%
\begin{pgfscope}%
\pgfpathrectangle{\pgfqpoint{0.549740in}{0.463273in}}{\pgfqpoint{9.320225in}{4.495057in}}%
\pgfusepath{clip}%
\pgfsetbuttcap%
\pgfsetroundjoin%
\pgfsetlinewidth{0.000000pt}%
\definecolor{currentstroke}{rgb}{0.000000,0.000000,0.000000}%
\pgfsetstrokecolor{currentstroke}%
\pgfsetdash{}{0pt}%
\pgfpathmoveto{\pgfqpoint{1.853347in}{4.181911in}}%
\pgfpathlineto{\pgfqpoint{2.039573in}{4.181911in}}%
\pgfpathlineto{\pgfqpoint{2.039573in}{4.263639in}}%
\pgfpathlineto{\pgfqpoint{1.853347in}{4.263639in}}%
\pgfpathlineto{\pgfqpoint{1.853347in}{4.181911in}}%
\pgfusepath{}%
\end{pgfscope}%
\begin{pgfscope}%
\pgfpathrectangle{\pgfqpoint{0.549740in}{0.463273in}}{\pgfqpoint{9.320225in}{4.495057in}}%
\pgfusepath{clip}%
\pgfsetbuttcap%
\pgfsetroundjoin%
\pgfsetlinewidth{0.000000pt}%
\definecolor{currentstroke}{rgb}{0.000000,0.000000,0.000000}%
\pgfsetstrokecolor{currentstroke}%
\pgfsetdash{}{0pt}%
\pgfpathmoveto{\pgfqpoint{2.039573in}{4.181911in}}%
\pgfpathlineto{\pgfqpoint{2.225800in}{4.181911in}}%
\pgfpathlineto{\pgfqpoint{2.225800in}{4.263639in}}%
\pgfpathlineto{\pgfqpoint{2.039573in}{4.263639in}}%
\pgfpathlineto{\pgfqpoint{2.039573in}{4.181911in}}%
\pgfusepath{}%
\end{pgfscope}%
\begin{pgfscope}%
\pgfpathrectangle{\pgfqpoint{0.549740in}{0.463273in}}{\pgfqpoint{9.320225in}{4.495057in}}%
\pgfusepath{clip}%
\pgfsetbuttcap%
\pgfsetroundjoin%
\pgfsetlinewidth{0.000000pt}%
\definecolor{currentstroke}{rgb}{0.000000,0.000000,0.000000}%
\pgfsetstrokecolor{currentstroke}%
\pgfsetdash{}{0pt}%
\pgfpathmoveto{\pgfqpoint{2.225800in}{4.181911in}}%
\pgfpathlineto{\pgfqpoint{2.412027in}{4.181911in}}%
\pgfpathlineto{\pgfqpoint{2.412027in}{4.263639in}}%
\pgfpathlineto{\pgfqpoint{2.225800in}{4.263639in}}%
\pgfpathlineto{\pgfqpoint{2.225800in}{4.181911in}}%
\pgfusepath{}%
\end{pgfscope}%
\begin{pgfscope}%
\pgfpathrectangle{\pgfqpoint{0.549740in}{0.463273in}}{\pgfqpoint{9.320225in}{4.495057in}}%
\pgfusepath{clip}%
\pgfsetbuttcap%
\pgfsetroundjoin%
\pgfsetlinewidth{0.000000pt}%
\definecolor{currentstroke}{rgb}{0.000000,0.000000,0.000000}%
\pgfsetstrokecolor{currentstroke}%
\pgfsetdash{}{0pt}%
\pgfpathmoveto{\pgfqpoint{2.412027in}{4.181911in}}%
\pgfpathlineto{\pgfqpoint{2.598253in}{4.181911in}}%
\pgfpathlineto{\pgfqpoint{2.598253in}{4.263639in}}%
\pgfpathlineto{\pgfqpoint{2.412027in}{4.263639in}}%
\pgfpathlineto{\pgfqpoint{2.412027in}{4.181911in}}%
\pgfusepath{}%
\end{pgfscope}%
\begin{pgfscope}%
\pgfpathrectangle{\pgfqpoint{0.549740in}{0.463273in}}{\pgfqpoint{9.320225in}{4.495057in}}%
\pgfusepath{clip}%
\pgfsetbuttcap%
\pgfsetroundjoin%
\pgfsetlinewidth{0.000000pt}%
\definecolor{currentstroke}{rgb}{0.000000,0.000000,0.000000}%
\pgfsetstrokecolor{currentstroke}%
\pgfsetdash{}{0pt}%
\pgfpathmoveto{\pgfqpoint{2.598253in}{4.181911in}}%
\pgfpathlineto{\pgfqpoint{2.784480in}{4.181911in}}%
\pgfpathlineto{\pgfqpoint{2.784480in}{4.263639in}}%
\pgfpathlineto{\pgfqpoint{2.598253in}{4.263639in}}%
\pgfpathlineto{\pgfqpoint{2.598253in}{4.181911in}}%
\pgfusepath{}%
\end{pgfscope}%
\begin{pgfscope}%
\pgfpathrectangle{\pgfqpoint{0.549740in}{0.463273in}}{\pgfqpoint{9.320225in}{4.495057in}}%
\pgfusepath{clip}%
\pgfsetbuttcap%
\pgfsetroundjoin%
\pgfsetlinewidth{0.000000pt}%
\definecolor{currentstroke}{rgb}{0.000000,0.000000,0.000000}%
\pgfsetstrokecolor{currentstroke}%
\pgfsetdash{}{0pt}%
\pgfpathmoveto{\pgfqpoint{2.784480in}{4.181911in}}%
\pgfpathlineto{\pgfqpoint{2.970706in}{4.181911in}}%
\pgfpathlineto{\pgfqpoint{2.970706in}{4.263639in}}%
\pgfpathlineto{\pgfqpoint{2.784480in}{4.263639in}}%
\pgfpathlineto{\pgfqpoint{2.784480in}{4.181911in}}%
\pgfusepath{}%
\end{pgfscope}%
\begin{pgfscope}%
\pgfpathrectangle{\pgfqpoint{0.549740in}{0.463273in}}{\pgfqpoint{9.320225in}{4.495057in}}%
\pgfusepath{clip}%
\pgfsetbuttcap%
\pgfsetroundjoin%
\pgfsetlinewidth{0.000000pt}%
\definecolor{currentstroke}{rgb}{0.000000,0.000000,0.000000}%
\pgfsetstrokecolor{currentstroke}%
\pgfsetdash{}{0pt}%
\pgfpathmoveto{\pgfqpoint{2.970706in}{4.181911in}}%
\pgfpathlineto{\pgfqpoint{3.156933in}{4.181911in}}%
\pgfpathlineto{\pgfqpoint{3.156933in}{4.263639in}}%
\pgfpathlineto{\pgfqpoint{2.970706in}{4.263639in}}%
\pgfpathlineto{\pgfqpoint{2.970706in}{4.181911in}}%
\pgfusepath{}%
\end{pgfscope}%
\begin{pgfscope}%
\pgfpathrectangle{\pgfqpoint{0.549740in}{0.463273in}}{\pgfqpoint{9.320225in}{4.495057in}}%
\pgfusepath{clip}%
\pgfsetbuttcap%
\pgfsetroundjoin%
\pgfsetlinewidth{0.000000pt}%
\definecolor{currentstroke}{rgb}{0.000000,0.000000,0.000000}%
\pgfsetstrokecolor{currentstroke}%
\pgfsetdash{}{0pt}%
\pgfpathmoveto{\pgfqpoint{3.156933in}{4.181911in}}%
\pgfpathlineto{\pgfqpoint{3.343159in}{4.181911in}}%
\pgfpathlineto{\pgfqpoint{3.343159in}{4.263639in}}%
\pgfpathlineto{\pgfqpoint{3.156933in}{4.263639in}}%
\pgfpathlineto{\pgfqpoint{3.156933in}{4.181911in}}%
\pgfusepath{}%
\end{pgfscope}%
\begin{pgfscope}%
\pgfpathrectangle{\pgfqpoint{0.549740in}{0.463273in}}{\pgfqpoint{9.320225in}{4.495057in}}%
\pgfusepath{clip}%
\pgfsetbuttcap%
\pgfsetroundjoin%
\pgfsetlinewidth{0.000000pt}%
\definecolor{currentstroke}{rgb}{0.000000,0.000000,0.000000}%
\pgfsetstrokecolor{currentstroke}%
\pgfsetdash{}{0pt}%
\pgfpathmoveto{\pgfqpoint{3.343159in}{4.181911in}}%
\pgfpathlineto{\pgfqpoint{3.529386in}{4.181911in}}%
\pgfpathlineto{\pgfqpoint{3.529386in}{4.263639in}}%
\pgfpathlineto{\pgfqpoint{3.343159in}{4.263639in}}%
\pgfpathlineto{\pgfqpoint{3.343159in}{4.181911in}}%
\pgfusepath{}%
\end{pgfscope}%
\begin{pgfscope}%
\pgfpathrectangle{\pgfqpoint{0.549740in}{0.463273in}}{\pgfqpoint{9.320225in}{4.495057in}}%
\pgfusepath{clip}%
\pgfsetbuttcap%
\pgfsetroundjoin%
\pgfsetlinewidth{0.000000pt}%
\definecolor{currentstroke}{rgb}{0.000000,0.000000,0.000000}%
\pgfsetstrokecolor{currentstroke}%
\pgfsetdash{}{0pt}%
\pgfpathmoveto{\pgfqpoint{3.529386in}{4.181911in}}%
\pgfpathlineto{\pgfqpoint{3.715612in}{4.181911in}}%
\pgfpathlineto{\pgfqpoint{3.715612in}{4.263639in}}%
\pgfpathlineto{\pgfqpoint{3.529386in}{4.263639in}}%
\pgfpathlineto{\pgfqpoint{3.529386in}{4.181911in}}%
\pgfusepath{}%
\end{pgfscope}%
\begin{pgfscope}%
\pgfpathrectangle{\pgfqpoint{0.549740in}{0.463273in}}{\pgfqpoint{9.320225in}{4.495057in}}%
\pgfusepath{clip}%
\pgfsetbuttcap%
\pgfsetroundjoin%
\pgfsetlinewidth{0.000000pt}%
\definecolor{currentstroke}{rgb}{0.000000,0.000000,0.000000}%
\pgfsetstrokecolor{currentstroke}%
\pgfsetdash{}{0pt}%
\pgfpathmoveto{\pgfqpoint{3.715612in}{4.181911in}}%
\pgfpathlineto{\pgfqpoint{3.901839in}{4.181911in}}%
\pgfpathlineto{\pgfqpoint{3.901839in}{4.263639in}}%
\pgfpathlineto{\pgfqpoint{3.715612in}{4.263639in}}%
\pgfpathlineto{\pgfqpoint{3.715612in}{4.181911in}}%
\pgfusepath{}%
\end{pgfscope}%
\begin{pgfscope}%
\pgfpathrectangle{\pgfqpoint{0.549740in}{0.463273in}}{\pgfqpoint{9.320225in}{4.495057in}}%
\pgfusepath{clip}%
\pgfsetbuttcap%
\pgfsetroundjoin%
\pgfsetlinewidth{0.000000pt}%
\definecolor{currentstroke}{rgb}{0.000000,0.000000,0.000000}%
\pgfsetstrokecolor{currentstroke}%
\pgfsetdash{}{0pt}%
\pgfpathmoveto{\pgfqpoint{3.901839in}{4.181911in}}%
\pgfpathlineto{\pgfqpoint{4.088065in}{4.181911in}}%
\pgfpathlineto{\pgfqpoint{4.088065in}{4.263639in}}%
\pgfpathlineto{\pgfqpoint{3.901839in}{4.263639in}}%
\pgfpathlineto{\pgfqpoint{3.901839in}{4.181911in}}%
\pgfusepath{}%
\end{pgfscope}%
\begin{pgfscope}%
\pgfpathrectangle{\pgfqpoint{0.549740in}{0.463273in}}{\pgfqpoint{9.320225in}{4.495057in}}%
\pgfusepath{clip}%
\pgfsetbuttcap%
\pgfsetroundjoin%
\pgfsetlinewidth{0.000000pt}%
\definecolor{currentstroke}{rgb}{0.000000,0.000000,0.000000}%
\pgfsetstrokecolor{currentstroke}%
\pgfsetdash{}{0pt}%
\pgfpathmoveto{\pgfqpoint{4.088065in}{4.181911in}}%
\pgfpathlineto{\pgfqpoint{4.274292in}{4.181911in}}%
\pgfpathlineto{\pgfqpoint{4.274292in}{4.263639in}}%
\pgfpathlineto{\pgfqpoint{4.088065in}{4.263639in}}%
\pgfpathlineto{\pgfqpoint{4.088065in}{4.181911in}}%
\pgfusepath{}%
\end{pgfscope}%
\begin{pgfscope}%
\pgfpathrectangle{\pgfqpoint{0.549740in}{0.463273in}}{\pgfqpoint{9.320225in}{4.495057in}}%
\pgfusepath{clip}%
\pgfsetbuttcap%
\pgfsetroundjoin%
\pgfsetlinewidth{0.000000pt}%
\definecolor{currentstroke}{rgb}{0.000000,0.000000,0.000000}%
\pgfsetstrokecolor{currentstroke}%
\pgfsetdash{}{0pt}%
\pgfpathmoveto{\pgfqpoint{4.274292in}{4.181911in}}%
\pgfpathlineto{\pgfqpoint{4.460519in}{4.181911in}}%
\pgfpathlineto{\pgfqpoint{4.460519in}{4.263639in}}%
\pgfpathlineto{\pgfqpoint{4.274292in}{4.263639in}}%
\pgfpathlineto{\pgfqpoint{4.274292in}{4.181911in}}%
\pgfusepath{}%
\end{pgfscope}%
\begin{pgfscope}%
\pgfpathrectangle{\pgfqpoint{0.549740in}{0.463273in}}{\pgfqpoint{9.320225in}{4.495057in}}%
\pgfusepath{clip}%
\pgfsetbuttcap%
\pgfsetroundjoin%
\pgfsetlinewidth{0.000000pt}%
\definecolor{currentstroke}{rgb}{0.000000,0.000000,0.000000}%
\pgfsetstrokecolor{currentstroke}%
\pgfsetdash{}{0pt}%
\pgfpathmoveto{\pgfqpoint{4.460519in}{4.181911in}}%
\pgfpathlineto{\pgfqpoint{4.646745in}{4.181911in}}%
\pgfpathlineto{\pgfqpoint{4.646745in}{4.263639in}}%
\pgfpathlineto{\pgfqpoint{4.460519in}{4.263639in}}%
\pgfpathlineto{\pgfqpoint{4.460519in}{4.181911in}}%
\pgfusepath{}%
\end{pgfscope}%
\begin{pgfscope}%
\pgfpathrectangle{\pgfqpoint{0.549740in}{0.463273in}}{\pgfqpoint{9.320225in}{4.495057in}}%
\pgfusepath{clip}%
\pgfsetbuttcap%
\pgfsetroundjoin%
\pgfsetlinewidth{0.000000pt}%
\definecolor{currentstroke}{rgb}{0.000000,0.000000,0.000000}%
\pgfsetstrokecolor{currentstroke}%
\pgfsetdash{}{0pt}%
\pgfpathmoveto{\pgfqpoint{4.646745in}{4.181911in}}%
\pgfpathlineto{\pgfqpoint{4.832972in}{4.181911in}}%
\pgfpathlineto{\pgfqpoint{4.832972in}{4.263639in}}%
\pgfpathlineto{\pgfqpoint{4.646745in}{4.263639in}}%
\pgfpathlineto{\pgfqpoint{4.646745in}{4.181911in}}%
\pgfusepath{}%
\end{pgfscope}%
\begin{pgfscope}%
\pgfpathrectangle{\pgfqpoint{0.549740in}{0.463273in}}{\pgfqpoint{9.320225in}{4.495057in}}%
\pgfusepath{clip}%
\pgfsetbuttcap%
\pgfsetroundjoin%
\pgfsetlinewidth{0.000000pt}%
\definecolor{currentstroke}{rgb}{0.000000,0.000000,0.000000}%
\pgfsetstrokecolor{currentstroke}%
\pgfsetdash{}{0pt}%
\pgfpathmoveto{\pgfqpoint{4.832972in}{4.181911in}}%
\pgfpathlineto{\pgfqpoint{5.019198in}{4.181911in}}%
\pgfpathlineto{\pgfqpoint{5.019198in}{4.263639in}}%
\pgfpathlineto{\pgfqpoint{4.832972in}{4.263639in}}%
\pgfpathlineto{\pgfqpoint{4.832972in}{4.181911in}}%
\pgfusepath{}%
\end{pgfscope}%
\begin{pgfscope}%
\pgfpathrectangle{\pgfqpoint{0.549740in}{0.463273in}}{\pgfqpoint{9.320225in}{4.495057in}}%
\pgfusepath{clip}%
\pgfsetbuttcap%
\pgfsetroundjoin%
\pgfsetlinewidth{0.000000pt}%
\definecolor{currentstroke}{rgb}{0.000000,0.000000,0.000000}%
\pgfsetstrokecolor{currentstroke}%
\pgfsetdash{}{0pt}%
\pgfpathmoveto{\pgfqpoint{5.019198in}{4.181911in}}%
\pgfpathlineto{\pgfqpoint{5.205425in}{4.181911in}}%
\pgfpathlineto{\pgfqpoint{5.205425in}{4.263639in}}%
\pgfpathlineto{\pgfqpoint{5.019198in}{4.263639in}}%
\pgfpathlineto{\pgfqpoint{5.019198in}{4.181911in}}%
\pgfusepath{}%
\end{pgfscope}%
\begin{pgfscope}%
\pgfpathrectangle{\pgfqpoint{0.549740in}{0.463273in}}{\pgfqpoint{9.320225in}{4.495057in}}%
\pgfusepath{clip}%
\pgfsetbuttcap%
\pgfsetroundjoin%
\pgfsetlinewidth{0.000000pt}%
\definecolor{currentstroke}{rgb}{0.000000,0.000000,0.000000}%
\pgfsetstrokecolor{currentstroke}%
\pgfsetdash{}{0pt}%
\pgfpathmoveto{\pgfqpoint{5.205425in}{4.181911in}}%
\pgfpathlineto{\pgfqpoint{5.391651in}{4.181911in}}%
\pgfpathlineto{\pgfqpoint{5.391651in}{4.263639in}}%
\pgfpathlineto{\pgfqpoint{5.205425in}{4.263639in}}%
\pgfpathlineto{\pgfqpoint{5.205425in}{4.181911in}}%
\pgfusepath{}%
\end{pgfscope}%
\begin{pgfscope}%
\pgfpathrectangle{\pgfqpoint{0.549740in}{0.463273in}}{\pgfqpoint{9.320225in}{4.495057in}}%
\pgfusepath{clip}%
\pgfsetbuttcap%
\pgfsetroundjoin%
\pgfsetlinewidth{0.000000pt}%
\definecolor{currentstroke}{rgb}{0.000000,0.000000,0.000000}%
\pgfsetstrokecolor{currentstroke}%
\pgfsetdash{}{0pt}%
\pgfpathmoveto{\pgfqpoint{5.391651in}{4.181911in}}%
\pgfpathlineto{\pgfqpoint{5.577878in}{4.181911in}}%
\pgfpathlineto{\pgfqpoint{5.577878in}{4.263639in}}%
\pgfpathlineto{\pgfqpoint{5.391651in}{4.263639in}}%
\pgfpathlineto{\pgfqpoint{5.391651in}{4.181911in}}%
\pgfusepath{}%
\end{pgfscope}%
\begin{pgfscope}%
\pgfpathrectangle{\pgfqpoint{0.549740in}{0.463273in}}{\pgfqpoint{9.320225in}{4.495057in}}%
\pgfusepath{clip}%
\pgfsetbuttcap%
\pgfsetroundjoin%
\pgfsetlinewidth{0.000000pt}%
\definecolor{currentstroke}{rgb}{0.000000,0.000000,0.000000}%
\pgfsetstrokecolor{currentstroke}%
\pgfsetdash{}{0pt}%
\pgfpathmoveto{\pgfqpoint{5.577878in}{4.181911in}}%
\pgfpathlineto{\pgfqpoint{5.764104in}{4.181911in}}%
\pgfpathlineto{\pgfqpoint{5.764104in}{4.263639in}}%
\pgfpathlineto{\pgfqpoint{5.577878in}{4.263639in}}%
\pgfpathlineto{\pgfqpoint{5.577878in}{4.181911in}}%
\pgfusepath{}%
\end{pgfscope}%
\begin{pgfscope}%
\pgfpathrectangle{\pgfqpoint{0.549740in}{0.463273in}}{\pgfqpoint{9.320225in}{4.495057in}}%
\pgfusepath{clip}%
\pgfsetbuttcap%
\pgfsetroundjoin%
\pgfsetlinewidth{0.000000pt}%
\definecolor{currentstroke}{rgb}{0.000000,0.000000,0.000000}%
\pgfsetstrokecolor{currentstroke}%
\pgfsetdash{}{0pt}%
\pgfpathmoveto{\pgfqpoint{5.764104in}{4.181911in}}%
\pgfpathlineto{\pgfqpoint{5.950331in}{4.181911in}}%
\pgfpathlineto{\pgfqpoint{5.950331in}{4.263639in}}%
\pgfpathlineto{\pgfqpoint{5.764104in}{4.263639in}}%
\pgfpathlineto{\pgfqpoint{5.764104in}{4.181911in}}%
\pgfusepath{}%
\end{pgfscope}%
\begin{pgfscope}%
\pgfpathrectangle{\pgfqpoint{0.549740in}{0.463273in}}{\pgfqpoint{9.320225in}{4.495057in}}%
\pgfusepath{clip}%
\pgfsetbuttcap%
\pgfsetroundjoin%
\pgfsetlinewidth{0.000000pt}%
\definecolor{currentstroke}{rgb}{0.000000,0.000000,0.000000}%
\pgfsetstrokecolor{currentstroke}%
\pgfsetdash{}{0pt}%
\pgfpathmoveto{\pgfqpoint{5.950331in}{4.181911in}}%
\pgfpathlineto{\pgfqpoint{6.136557in}{4.181911in}}%
\pgfpathlineto{\pgfqpoint{6.136557in}{4.263639in}}%
\pgfpathlineto{\pgfqpoint{5.950331in}{4.263639in}}%
\pgfpathlineto{\pgfqpoint{5.950331in}{4.181911in}}%
\pgfusepath{}%
\end{pgfscope}%
\begin{pgfscope}%
\pgfpathrectangle{\pgfqpoint{0.549740in}{0.463273in}}{\pgfqpoint{9.320225in}{4.495057in}}%
\pgfusepath{clip}%
\pgfsetbuttcap%
\pgfsetroundjoin%
\pgfsetlinewidth{0.000000pt}%
\definecolor{currentstroke}{rgb}{0.000000,0.000000,0.000000}%
\pgfsetstrokecolor{currentstroke}%
\pgfsetdash{}{0pt}%
\pgfpathmoveto{\pgfqpoint{6.136557in}{4.181911in}}%
\pgfpathlineto{\pgfqpoint{6.322784in}{4.181911in}}%
\pgfpathlineto{\pgfqpoint{6.322784in}{4.263639in}}%
\pgfpathlineto{\pgfqpoint{6.136557in}{4.263639in}}%
\pgfpathlineto{\pgfqpoint{6.136557in}{4.181911in}}%
\pgfusepath{}%
\end{pgfscope}%
\begin{pgfscope}%
\pgfpathrectangle{\pgfqpoint{0.549740in}{0.463273in}}{\pgfqpoint{9.320225in}{4.495057in}}%
\pgfusepath{clip}%
\pgfsetbuttcap%
\pgfsetroundjoin%
\pgfsetlinewidth{0.000000pt}%
\definecolor{currentstroke}{rgb}{0.000000,0.000000,0.000000}%
\pgfsetstrokecolor{currentstroke}%
\pgfsetdash{}{0pt}%
\pgfpathmoveto{\pgfqpoint{6.322784in}{4.181911in}}%
\pgfpathlineto{\pgfqpoint{6.509011in}{4.181911in}}%
\pgfpathlineto{\pgfqpoint{6.509011in}{4.263639in}}%
\pgfpathlineto{\pgfqpoint{6.322784in}{4.263639in}}%
\pgfpathlineto{\pgfqpoint{6.322784in}{4.181911in}}%
\pgfusepath{}%
\end{pgfscope}%
\begin{pgfscope}%
\pgfpathrectangle{\pgfqpoint{0.549740in}{0.463273in}}{\pgfqpoint{9.320225in}{4.495057in}}%
\pgfusepath{clip}%
\pgfsetbuttcap%
\pgfsetroundjoin%
\pgfsetlinewidth{0.000000pt}%
\definecolor{currentstroke}{rgb}{0.000000,0.000000,0.000000}%
\pgfsetstrokecolor{currentstroke}%
\pgfsetdash{}{0pt}%
\pgfpathmoveto{\pgfqpoint{6.509011in}{4.181911in}}%
\pgfpathlineto{\pgfqpoint{6.695237in}{4.181911in}}%
\pgfpathlineto{\pgfqpoint{6.695237in}{4.263639in}}%
\pgfpathlineto{\pgfqpoint{6.509011in}{4.263639in}}%
\pgfpathlineto{\pgfqpoint{6.509011in}{4.181911in}}%
\pgfusepath{}%
\end{pgfscope}%
\begin{pgfscope}%
\pgfpathrectangle{\pgfqpoint{0.549740in}{0.463273in}}{\pgfqpoint{9.320225in}{4.495057in}}%
\pgfusepath{clip}%
\pgfsetbuttcap%
\pgfsetroundjoin%
\pgfsetlinewidth{0.000000pt}%
\definecolor{currentstroke}{rgb}{0.000000,0.000000,0.000000}%
\pgfsetstrokecolor{currentstroke}%
\pgfsetdash{}{0pt}%
\pgfpathmoveto{\pgfqpoint{6.695237in}{4.181911in}}%
\pgfpathlineto{\pgfqpoint{6.881464in}{4.181911in}}%
\pgfpathlineto{\pgfqpoint{6.881464in}{4.263639in}}%
\pgfpathlineto{\pgfqpoint{6.695237in}{4.263639in}}%
\pgfpathlineto{\pgfqpoint{6.695237in}{4.181911in}}%
\pgfusepath{}%
\end{pgfscope}%
\begin{pgfscope}%
\pgfpathrectangle{\pgfqpoint{0.549740in}{0.463273in}}{\pgfqpoint{9.320225in}{4.495057in}}%
\pgfusepath{clip}%
\pgfsetbuttcap%
\pgfsetroundjoin%
\pgfsetlinewidth{0.000000pt}%
\definecolor{currentstroke}{rgb}{0.000000,0.000000,0.000000}%
\pgfsetstrokecolor{currentstroke}%
\pgfsetdash{}{0pt}%
\pgfpathmoveto{\pgfqpoint{6.881464in}{4.181911in}}%
\pgfpathlineto{\pgfqpoint{7.067690in}{4.181911in}}%
\pgfpathlineto{\pgfqpoint{7.067690in}{4.263639in}}%
\pgfpathlineto{\pgfqpoint{6.881464in}{4.263639in}}%
\pgfpathlineto{\pgfqpoint{6.881464in}{4.181911in}}%
\pgfusepath{}%
\end{pgfscope}%
\begin{pgfscope}%
\pgfpathrectangle{\pgfqpoint{0.549740in}{0.463273in}}{\pgfqpoint{9.320225in}{4.495057in}}%
\pgfusepath{clip}%
\pgfsetbuttcap%
\pgfsetroundjoin%
\pgfsetlinewidth{0.000000pt}%
\definecolor{currentstroke}{rgb}{0.000000,0.000000,0.000000}%
\pgfsetstrokecolor{currentstroke}%
\pgfsetdash{}{0pt}%
\pgfpathmoveto{\pgfqpoint{7.067690in}{4.181911in}}%
\pgfpathlineto{\pgfqpoint{7.253917in}{4.181911in}}%
\pgfpathlineto{\pgfqpoint{7.253917in}{4.263639in}}%
\pgfpathlineto{\pgfqpoint{7.067690in}{4.263639in}}%
\pgfpathlineto{\pgfqpoint{7.067690in}{4.181911in}}%
\pgfusepath{}%
\end{pgfscope}%
\begin{pgfscope}%
\pgfpathrectangle{\pgfqpoint{0.549740in}{0.463273in}}{\pgfqpoint{9.320225in}{4.495057in}}%
\pgfusepath{clip}%
\pgfsetbuttcap%
\pgfsetroundjoin%
\pgfsetlinewidth{0.000000pt}%
\definecolor{currentstroke}{rgb}{0.000000,0.000000,0.000000}%
\pgfsetstrokecolor{currentstroke}%
\pgfsetdash{}{0pt}%
\pgfpathmoveto{\pgfqpoint{7.253917in}{4.181911in}}%
\pgfpathlineto{\pgfqpoint{7.440143in}{4.181911in}}%
\pgfpathlineto{\pgfqpoint{7.440143in}{4.263639in}}%
\pgfpathlineto{\pgfqpoint{7.253917in}{4.263639in}}%
\pgfpathlineto{\pgfqpoint{7.253917in}{4.181911in}}%
\pgfusepath{}%
\end{pgfscope}%
\begin{pgfscope}%
\pgfpathrectangle{\pgfqpoint{0.549740in}{0.463273in}}{\pgfqpoint{9.320225in}{4.495057in}}%
\pgfusepath{clip}%
\pgfsetbuttcap%
\pgfsetroundjoin%
\pgfsetlinewidth{0.000000pt}%
\definecolor{currentstroke}{rgb}{0.000000,0.000000,0.000000}%
\pgfsetstrokecolor{currentstroke}%
\pgfsetdash{}{0pt}%
\pgfpathmoveto{\pgfqpoint{7.440143in}{4.181911in}}%
\pgfpathlineto{\pgfqpoint{7.626370in}{4.181911in}}%
\pgfpathlineto{\pgfqpoint{7.626370in}{4.263639in}}%
\pgfpathlineto{\pgfqpoint{7.440143in}{4.263639in}}%
\pgfpathlineto{\pgfqpoint{7.440143in}{4.181911in}}%
\pgfusepath{}%
\end{pgfscope}%
\begin{pgfscope}%
\pgfpathrectangle{\pgfqpoint{0.549740in}{0.463273in}}{\pgfqpoint{9.320225in}{4.495057in}}%
\pgfusepath{clip}%
\pgfsetbuttcap%
\pgfsetroundjoin%
\pgfsetlinewidth{0.000000pt}%
\definecolor{currentstroke}{rgb}{0.000000,0.000000,0.000000}%
\pgfsetstrokecolor{currentstroke}%
\pgfsetdash{}{0pt}%
\pgfpathmoveto{\pgfqpoint{7.626370in}{4.181911in}}%
\pgfpathlineto{\pgfqpoint{7.812596in}{4.181911in}}%
\pgfpathlineto{\pgfqpoint{7.812596in}{4.263639in}}%
\pgfpathlineto{\pgfqpoint{7.626370in}{4.263639in}}%
\pgfpathlineto{\pgfqpoint{7.626370in}{4.181911in}}%
\pgfusepath{}%
\end{pgfscope}%
\begin{pgfscope}%
\pgfpathrectangle{\pgfqpoint{0.549740in}{0.463273in}}{\pgfqpoint{9.320225in}{4.495057in}}%
\pgfusepath{clip}%
\pgfsetbuttcap%
\pgfsetroundjoin%
\pgfsetlinewidth{0.000000pt}%
\definecolor{currentstroke}{rgb}{0.000000,0.000000,0.000000}%
\pgfsetstrokecolor{currentstroke}%
\pgfsetdash{}{0pt}%
\pgfpathmoveto{\pgfqpoint{7.812596in}{4.181911in}}%
\pgfpathlineto{\pgfqpoint{7.998823in}{4.181911in}}%
\pgfpathlineto{\pgfqpoint{7.998823in}{4.263639in}}%
\pgfpathlineto{\pgfqpoint{7.812596in}{4.263639in}}%
\pgfpathlineto{\pgfqpoint{7.812596in}{4.181911in}}%
\pgfusepath{}%
\end{pgfscope}%
\begin{pgfscope}%
\pgfpathrectangle{\pgfqpoint{0.549740in}{0.463273in}}{\pgfqpoint{9.320225in}{4.495057in}}%
\pgfusepath{clip}%
\pgfsetbuttcap%
\pgfsetroundjoin%
\pgfsetlinewidth{0.000000pt}%
\definecolor{currentstroke}{rgb}{0.000000,0.000000,0.000000}%
\pgfsetstrokecolor{currentstroke}%
\pgfsetdash{}{0pt}%
\pgfpathmoveto{\pgfqpoint{7.998823in}{4.181911in}}%
\pgfpathlineto{\pgfqpoint{8.185049in}{4.181911in}}%
\pgfpathlineto{\pgfqpoint{8.185049in}{4.263639in}}%
\pgfpathlineto{\pgfqpoint{7.998823in}{4.263639in}}%
\pgfpathlineto{\pgfqpoint{7.998823in}{4.181911in}}%
\pgfusepath{}%
\end{pgfscope}%
\begin{pgfscope}%
\pgfpathrectangle{\pgfqpoint{0.549740in}{0.463273in}}{\pgfqpoint{9.320225in}{4.495057in}}%
\pgfusepath{clip}%
\pgfsetbuttcap%
\pgfsetroundjoin%
\pgfsetlinewidth{0.000000pt}%
\definecolor{currentstroke}{rgb}{0.000000,0.000000,0.000000}%
\pgfsetstrokecolor{currentstroke}%
\pgfsetdash{}{0pt}%
\pgfpathmoveto{\pgfqpoint{8.185049in}{4.181911in}}%
\pgfpathlineto{\pgfqpoint{8.371276in}{4.181911in}}%
\pgfpathlineto{\pgfqpoint{8.371276in}{4.263639in}}%
\pgfpathlineto{\pgfqpoint{8.185049in}{4.263639in}}%
\pgfpathlineto{\pgfqpoint{8.185049in}{4.181911in}}%
\pgfusepath{}%
\end{pgfscope}%
\begin{pgfscope}%
\pgfpathrectangle{\pgfqpoint{0.549740in}{0.463273in}}{\pgfqpoint{9.320225in}{4.495057in}}%
\pgfusepath{clip}%
\pgfsetbuttcap%
\pgfsetroundjoin%
\pgfsetlinewidth{0.000000pt}%
\definecolor{currentstroke}{rgb}{0.000000,0.000000,0.000000}%
\pgfsetstrokecolor{currentstroke}%
\pgfsetdash{}{0pt}%
\pgfpathmoveto{\pgfqpoint{8.371276in}{4.181911in}}%
\pgfpathlineto{\pgfqpoint{8.557503in}{4.181911in}}%
\pgfpathlineto{\pgfqpoint{8.557503in}{4.263639in}}%
\pgfpathlineto{\pgfqpoint{8.371276in}{4.263639in}}%
\pgfpathlineto{\pgfqpoint{8.371276in}{4.181911in}}%
\pgfusepath{}%
\end{pgfscope}%
\begin{pgfscope}%
\pgfpathrectangle{\pgfqpoint{0.549740in}{0.463273in}}{\pgfqpoint{9.320225in}{4.495057in}}%
\pgfusepath{clip}%
\pgfsetbuttcap%
\pgfsetroundjoin%
\pgfsetlinewidth{0.000000pt}%
\definecolor{currentstroke}{rgb}{0.000000,0.000000,0.000000}%
\pgfsetstrokecolor{currentstroke}%
\pgfsetdash{}{0pt}%
\pgfpathmoveto{\pgfqpoint{8.557503in}{4.181911in}}%
\pgfpathlineto{\pgfqpoint{8.743729in}{4.181911in}}%
\pgfpathlineto{\pgfqpoint{8.743729in}{4.263639in}}%
\pgfpathlineto{\pgfqpoint{8.557503in}{4.263639in}}%
\pgfpathlineto{\pgfqpoint{8.557503in}{4.181911in}}%
\pgfusepath{}%
\end{pgfscope}%
\begin{pgfscope}%
\pgfpathrectangle{\pgfqpoint{0.549740in}{0.463273in}}{\pgfqpoint{9.320225in}{4.495057in}}%
\pgfusepath{clip}%
\pgfsetbuttcap%
\pgfsetroundjoin%
\pgfsetlinewidth{0.000000pt}%
\definecolor{currentstroke}{rgb}{0.000000,0.000000,0.000000}%
\pgfsetstrokecolor{currentstroke}%
\pgfsetdash{}{0pt}%
\pgfpathmoveto{\pgfqpoint{8.743729in}{4.181911in}}%
\pgfpathlineto{\pgfqpoint{8.929956in}{4.181911in}}%
\pgfpathlineto{\pgfqpoint{8.929956in}{4.263639in}}%
\pgfpathlineto{\pgfqpoint{8.743729in}{4.263639in}}%
\pgfpathlineto{\pgfqpoint{8.743729in}{4.181911in}}%
\pgfusepath{}%
\end{pgfscope}%
\begin{pgfscope}%
\pgfpathrectangle{\pgfqpoint{0.549740in}{0.463273in}}{\pgfqpoint{9.320225in}{4.495057in}}%
\pgfusepath{clip}%
\pgfsetbuttcap%
\pgfsetroundjoin%
\definecolor{currentfill}{rgb}{0.614330,0.761948,0.940009}%
\pgfsetfillcolor{currentfill}%
\pgfsetlinewidth{0.000000pt}%
\definecolor{currentstroke}{rgb}{0.000000,0.000000,0.000000}%
\pgfsetstrokecolor{currentstroke}%
\pgfsetdash{}{0pt}%
\pgfpathmoveto{\pgfqpoint{8.929956in}{4.181911in}}%
\pgfpathlineto{\pgfqpoint{9.116182in}{4.181911in}}%
\pgfpathlineto{\pgfqpoint{9.116182in}{4.263639in}}%
\pgfpathlineto{\pgfqpoint{8.929956in}{4.263639in}}%
\pgfpathlineto{\pgfqpoint{8.929956in}{4.181911in}}%
\pgfusepath{fill}%
\end{pgfscope}%
\begin{pgfscope}%
\pgfpathrectangle{\pgfqpoint{0.549740in}{0.463273in}}{\pgfqpoint{9.320225in}{4.495057in}}%
\pgfusepath{clip}%
\pgfsetbuttcap%
\pgfsetroundjoin%
\definecolor{currentfill}{rgb}{0.547810,0.736432,0.947518}%
\pgfsetfillcolor{currentfill}%
\pgfsetlinewidth{0.000000pt}%
\definecolor{currentstroke}{rgb}{0.000000,0.000000,0.000000}%
\pgfsetstrokecolor{currentstroke}%
\pgfsetdash{}{0pt}%
\pgfpathmoveto{\pgfqpoint{9.116182in}{4.181911in}}%
\pgfpathlineto{\pgfqpoint{9.302409in}{4.181911in}}%
\pgfpathlineto{\pgfqpoint{9.302409in}{4.263639in}}%
\pgfpathlineto{\pgfqpoint{9.116182in}{4.263639in}}%
\pgfpathlineto{\pgfqpoint{9.116182in}{4.181911in}}%
\pgfusepath{fill}%
\end{pgfscope}%
\begin{pgfscope}%
\pgfpathrectangle{\pgfqpoint{0.549740in}{0.463273in}}{\pgfqpoint{9.320225in}{4.495057in}}%
\pgfusepath{clip}%
\pgfsetbuttcap%
\pgfsetroundjoin%
\pgfsetlinewidth{0.000000pt}%
\definecolor{currentstroke}{rgb}{0.000000,0.000000,0.000000}%
\pgfsetstrokecolor{currentstroke}%
\pgfsetdash{}{0pt}%
\pgfpathmoveto{\pgfqpoint{9.302409in}{4.181911in}}%
\pgfpathlineto{\pgfqpoint{9.488635in}{4.181911in}}%
\pgfpathlineto{\pgfqpoint{9.488635in}{4.263639in}}%
\pgfpathlineto{\pgfqpoint{9.302409in}{4.263639in}}%
\pgfpathlineto{\pgfqpoint{9.302409in}{4.181911in}}%
\pgfusepath{}%
\end{pgfscope}%
\begin{pgfscope}%
\pgfpathrectangle{\pgfqpoint{0.549740in}{0.463273in}}{\pgfqpoint{9.320225in}{4.495057in}}%
\pgfusepath{clip}%
\pgfsetbuttcap%
\pgfsetroundjoin%
\pgfsetlinewidth{0.000000pt}%
\definecolor{currentstroke}{rgb}{0.000000,0.000000,0.000000}%
\pgfsetstrokecolor{currentstroke}%
\pgfsetdash{}{0pt}%
\pgfpathmoveto{\pgfqpoint{9.488635in}{4.181911in}}%
\pgfpathlineto{\pgfqpoint{9.674862in}{4.181911in}}%
\pgfpathlineto{\pgfqpoint{9.674862in}{4.263639in}}%
\pgfpathlineto{\pgfqpoint{9.488635in}{4.263639in}}%
\pgfpathlineto{\pgfqpoint{9.488635in}{4.181911in}}%
\pgfusepath{}%
\end{pgfscope}%
\begin{pgfscope}%
\pgfpathrectangle{\pgfqpoint{0.549740in}{0.463273in}}{\pgfqpoint{9.320225in}{4.495057in}}%
\pgfusepath{clip}%
\pgfsetbuttcap%
\pgfsetroundjoin%
\pgfsetlinewidth{0.000000pt}%
\definecolor{currentstroke}{rgb}{0.000000,0.000000,0.000000}%
\pgfsetstrokecolor{currentstroke}%
\pgfsetdash{}{0pt}%
\pgfpathmoveto{\pgfqpoint{9.674862in}{4.181911in}}%
\pgfpathlineto{\pgfqpoint{9.861088in}{4.181911in}}%
\pgfpathlineto{\pgfqpoint{9.861088in}{4.263639in}}%
\pgfpathlineto{\pgfqpoint{9.674862in}{4.263639in}}%
\pgfpathlineto{\pgfqpoint{9.674862in}{4.181911in}}%
\pgfusepath{}%
\end{pgfscope}%
\begin{pgfscope}%
\pgfpathrectangle{\pgfqpoint{0.549740in}{0.463273in}}{\pgfqpoint{9.320225in}{4.495057in}}%
\pgfusepath{clip}%
\pgfsetbuttcap%
\pgfsetroundjoin%
\pgfsetlinewidth{0.000000pt}%
\definecolor{currentstroke}{rgb}{0.000000,0.000000,0.000000}%
\pgfsetstrokecolor{currentstroke}%
\pgfsetdash{}{0pt}%
\pgfpathmoveto{\pgfqpoint{0.549761in}{4.263639in}}%
\pgfpathlineto{\pgfqpoint{0.735988in}{4.263639in}}%
\pgfpathlineto{\pgfqpoint{0.735988in}{4.345368in}}%
\pgfpathlineto{\pgfqpoint{0.549761in}{4.345368in}}%
\pgfpathlineto{\pgfqpoint{0.549761in}{4.263639in}}%
\pgfusepath{}%
\end{pgfscope}%
\begin{pgfscope}%
\pgfpathrectangle{\pgfqpoint{0.549740in}{0.463273in}}{\pgfqpoint{9.320225in}{4.495057in}}%
\pgfusepath{clip}%
\pgfsetbuttcap%
\pgfsetroundjoin%
\pgfsetlinewidth{0.000000pt}%
\definecolor{currentstroke}{rgb}{0.000000,0.000000,0.000000}%
\pgfsetstrokecolor{currentstroke}%
\pgfsetdash{}{0pt}%
\pgfpathmoveto{\pgfqpoint{0.735988in}{4.263639in}}%
\pgfpathlineto{\pgfqpoint{0.922214in}{4.263639in}}%
\pgfpathlineto{\pgfqpoint{0.922214in}{4.345368in}}%
\pgfpathlineto{\pgfqpoint{0.735988in}{4.345368in}}%
\pgfpathlineto{\pgfqpoint{0.735988in}{4.263639in}}%
\pgfusepath{}%
\end{pgfscope}%
\begin{pgfscope}%
\pgfpathrectangle{\pgfqpoint{0.549740in}{0.463273in}}{\pgfqpoint{9.320225in}{4.495057in}}%
\pgfusepath{clip}%
\pgfsetbuttcap%
\pgfsetroundjoin%
\pgfsetlinewidth{0.000000pt}%
\definecolor{currentstroke}{rgb}{0.000000,0.000000,0.000000}%
\pgfsetstrokecolor{currentstroke}%
\pgfsetdash{}{0pt}%
\pgfpathmoveto{\pgfqpoint{0.922214in}{4.263639in}}%
\pgfpathlineto{\pgfqpoint{1.108441in}{4.263639in}}%
\pgfpathlineto{\pgfqpoint{1.108441in}{4.345368in}}%
\pgfpathlineto{\pgfqpoint{0.922214in}{4.345368in}}%
\pgfpathlineto{\pgfqpoint{0.922214in}{4.263639in}}%
\pgfusepath{}%
\end{pgfscope}%
\begin{pgfscope}%
\pgfpathrectangle{\pgfqpoint{0.549740in}{0.463273in}}{\pgfqpoint{9.320225in}{4.495057in}}%
\pgfusepath{clip}%
\pgfsetbuttcap%
\pgfsetroundjoin%
\pgfsetlinewidth{0.000000pt}%
\definecolor{currentstroke}{rgb}{0.000000,0.000000,0.000000}%
\pgfsetstrokecolor{currentstroke}%
\pgfsetdash{}{0pt}%
\pgfpathmoveto{\pgfqpoint{1.108441in}{4.263639in}}%
\pgfpathlineto{\pgfqpoint{1.294667in}{4.263639in}}%
\pgfpathlineto{\pgfqpoint{1.294667in}{4.345368in}}%
\pgfpathlineto{\pgfqpoint{1.108441in}{4.345368in}}%
\pgfpathlineto{\pgfqpoint{1.108441in}{4.263639in}}%
\pgfusepath{}%
\end{pgfscope}%
\begin{pgfscope}%
\pgfpathrectangle{\pgfqpoint{0.549740in}{0.463273in}}{\pgfqpoint{9.320225in}{4.495057in}}%
\pgfusepath{clip}%
\pgfsetbuttcap%
\pgfsetroundjoin%
\pgfsetlinewidth{0.000000pt}%
\definecolor{currentstroke}{rgb}{0.000000,0.000000,0.000000}%
\pgfsetstrokecolor{currentstroke}%
\pgfsetdash{}{0pt}%
\pgfpathmoveto{\pgfqpoint{1.294667in}{4.263639in}}%
\pgfpathlineto{\pgfqpoint{1.480894in}{4.263639in}}%
\pgfpathlineto{\pgfqpoint{1.480894in}{4.345368in}}%
\pgfpathlineto{\pgfqpoint{1.294667in}{4.345368in}}%
\pgfpathlineto{\pgfqpoint{1.294667in}{4.263639in}}%
\pgfusepath{}%
\end{pgfscope}%
\begin{pgfscope}%
\pgfpathrectangle{\pgfqpoint{0.549740in}{0.463273in}}{\pgfqpoint{9.320225in}{4.495057in}}%
\pgfusepath{clip}%
\pgfsetbuttcap%
\pgfsetroundjoin%
\pgfsetlinewidth{0.000000pt}%
\definecolor{currentstroke}{rgb}{0.000000,0.000000,0.000000}%
\pgfsetstrokecolor{currentstroke}%
\pgfsetdash{}{0pt}%
\pgfpathmoveto{\pgfqpoint{1.480894in}{4.263639in}}%
\pgfpathlineto{\pgfqpoint{1.667120in}{4.263639in}}%
\pgfpathlineto{\pgfqpoint{1.667120in}{4.345368in}}%
\pgfpathlineto{\pgfqpoint{1.480894in}{4.345368in}}%
\pgfpathlineto{\pgfqpoint{1.480894in}{4.263639in}}%
\pgfusepath{}%
\end{pgfscope}%
\begin{pgfscope}%
\pgfpathrectangle{\pgfqpoint{0.549740in}{0.463273in}}{\pgfqpoint{9.320225in}{4.495057in}}%
\pgfusepath{clip}%
\pgfsetbuttcap%
\pgfsetroundjoin%
\pgfsetlinewidth{0.000000pt}%
\definecolor{currentstroke}{rgb}{0.000000,0.000000,0.000000}%
\pgfsetstrokecolor{currentstroke}%
\pgfsetdash{}{0pt}%
\pgfpathmoveto{\pgfqpoint{1.667120in}{4.263639in}}%
\pgfpathlineto{\pgfqpoint{1.853347in}{4.263639in}}%
\pgfpathlineto{\pgfqpoint{1.853347in}{4.345368in}}%
\pgfpathlineto{\pgfqpoint{1.667120in}{4.345368in}}%
\pgfpathlineto{\pgfqpoint{1.667120in}{4.263639in}}%
\pgfusepath{}%
\end{pgfscope}%
\begin{pgfscope}%
\pgfpathrectangle{\pgfqpoint{0.549740in}{0.463273in}}{\pgfqpoint{9.320225in}{4.495057in}}%
\pgfusepath{clip}%
\pgfsetbuttcap%
\pgfsetroundjoin%
\pgfsetlinewidth{0.000000pt}%
\definecolor{currentstroke}{rgb}{0.000000,0.000000,0.000000}%
\pgfsetstrokecolor{currentstroke}%
\pgfsetdash{}{0pt}%
\pgfpathmoveto{\pgfqpoint{1.853347in}{4.263639in}}%
\pgfpathlineto{\pgfqpoint{2.039573in}{4.263639in}}%
\pgfpathlineto{\pgfqpoint{2.039573in}{4.345368in}}%
\pgfpathlineto{\pgfqpoint{1.853347in}{4.345368in}}%
\pgfpathlineto{\pgfqpoint{1.853347in}{4.263639in}}%
\pgfusepath{}%
\end{pgfscope}%
\begin{pgfscope}%
\pgfpathrectangle{\pgfqpoint{0.549740in}{0.463273in}}{\pgfqpoint{9.320225in}{4.495057in}}%
\pgfusepath{clip}%
\pgfsetbuttcap%
\pgfsetroundjoin%
\pgfsetlinewidth{0.000000pt}%
\definecolor{currentstroke}{rgb}{0.000000,0.000000,0.000000}%
\pgfsetstrokecolor{currentstroke}%
\pgfsetdash{}{0pt}%
\pgfpathmoveto{\pgfqpoint{2.039573in}{4.263639in}}%
\pgfpathlineto{\pgfqpoint{2.225800in}{4.263639in}}%
\pgfpathlineto{\pgfqpoint{2.225800in}{4.345368in}}%
\pgfpathlineto{\pgfqpoint{2.039573in}{4.345368in}}%
\pgfpathlineto{\pgfqpoint{2.039573in}{4.263639in}}%
\pgfusepath{}%
\end{pgfscope}%
\begin{pgfscope}%
\pgfpathrectangle{\pgfqpoint{0.549740in}{0.463273in}}{\pgfqpoint{9.320225in}{4.495057in}}%
\pgfusepath{clip}%
\pgfsetbuttcap%
\pgfsetroundjoin%
\pgfsetlinewidth{0.000000pt}%
\definecolor{currentstroke}{rgb}{0.000000,0.000000,0.000000}%
\pgfsetstrokecolor{currentstroke}%
\pgfsetdash{}{0pt}%
\pgfpathmoveto{\pgfqpoint{2.225800in}{4.263639in}}%
\pgfpathlineto{\pgfqpoint{2.412027in}{4.263639in}}%
\pgfpathlineto{\pgfqpoint{2.412027in}{4.345368in}}%
\pgfpathlineto{\pgfqpoint{2.225800in}{4.345368in}}%
\pgfpathlineto{\pgfqpoint{2.225800in}{4.263639in}}%
\pgfusepath{}%
\end{pgfscope}%
\begin{pgfscope}%
\pgfpathrectangle{\pgfqpoint{0.549740in}{0.463273in}}{\pgfqpoint{9.320225in}{4.495057in}}%
\pgfusepath{clip}%
\pgfsetbuttcap%
\pgfsetroundjoin%
\pgfsetlinewidth{0.000000pt}%
\definecolor{currentstroke}{rgb}{0.000000,0.000000,0.000000}%
\pgfsetstrokecolor{currentstroke}%
\pgfsetdash{}{0pt}%
\pgfpathmoveto{\pgfqpoint{2.412027in}{4.263639in}}%
\pgfpathlineto{\pgfqpoint{2.598253in}{4.263639in}}%
\pgfpathlineto{\pgfqpoint{2.598253in}{4.345368in}}%
\pgfpathlineto{\pgfqpoint{2.412027in}{4.345368in}}%
\pgfpathlineto{\pgfqpoint{2.412027in}{4.263639in}}%
\pgfusepath{}%
\end{pgfscope}%
\begin{pgfscope}%
\pgfpathrectangle{\pgfqpoint{0.549740in}{0.463273in}}{\pgfqpoint{9.320225in}{4.495057in}}%
\pgfusepath{clip}%
\pgfsetbuttcap%
\pgfsetroundjoin%
\pgfsetlinewidth{0.000000pt}%
\definecolor{currentstroke}{rgb}{0.000000,0.000000,0.000000}%
\pgfsetstrokecolor{currentstroke}%
\pgfsetdash{}{0pt}%
\pgfpathmoveto{\pgfqpoint{2.598253in}{4.263639in}}%
\pgfpathlineto{\pgfqpoint{2.784480in}{4.263639in}}%
\pgfpathlineto{\pgfqpoint{2.784480in}{4.345368in}}%
\pgfpathlineto{\pgfqpoint{2.598253in}{4.345368in}}%
\pgfpathlineto{\pgfqpoint{2.598253in}{4.263639in}}%
\pgfusepath{}%
\end{pgfscope}%
\begin{pgfscope}%
\pgfpathrectangle{\pgfqpoint{0.549740in}{0.463273in}}{\pgfqpoint{9.320225in}{4.495057in}}%
\pgfusepath{clip}%
\pgfsetbuttcap%
\pgfsetroundjoin%
\pgfsetlinewidth{0.000000pt}%
\definecolor{currentstroke}{rgb}{0.000000,0.000000,0.000000}%
\pgfsetstrokecolor{currentstroke}%
\pgfsetdash{}{0pt}%
\pgfpathmoveto{\pgfqpoint{2.784480in}{4.263639in}}%
\pgfpathlineto{\pgfqpoint{2.970706in}{4.263639in}}%
\pgfpathlineto{\pgfqpoint{2.970706in}{4.345368in}}%
\pgfpathlineto{\pgfqpoint{2.784480in}{4.345368in}}%
\pgfpathlineto{\pgfqpoint{2.784480in}{4.263639in}}%
\pgfusepath{}%
\end{pgfscope}%
\begin{pgfscope}%
\pgfpathrectangle{\pgfqpoint{0.549740in}{0.463273in}}{\pgfqpoint{9.320225in}{4.495057in}}%
\pgfusepath{clip}%
\pgfsetbuttcap%
\pgfsetroundjoin%
\pgfsetlinewidth{0.000000pt}%
\definecolor{currentstroke}{rgb}{0.000000,0.000000,0.000000}%
\pgfsetstrokecolor{currentstroke}%
\pgfsetdash{}{0pt}%
\pgfpathmoveto{\pgfqpoint{2.970706in}{4.263639in}}%
\pgfpathlineto{\pgfqpoint{3.156933in}{4.263639in}}%
\pgfpathlineto{\pgfqpoint{3.156933in}{4.345368in}}%
\pgfpathlineto{\pgfqpoint{2.970706in}{4.345368in}}%
\pgfpathlineto{\pgfqpoint{2.970706in}{4.263639in}}%
\pgfusepath{}%
\end{pgfscope}%
\begin{pgfscope}%
\pgfpathrectangle{\pgfqpoint{0.549740in}{0.463273in}}{\pgfqpoint{9.320225in}{4.495057in}}%
\pgfusepath{clip}%
\pgfsetbuttcap%
\pgfsetroundjoin%
\pgfsetlinewidth{0.000000pt}%
\definecolor{currentstroke}{rgb}{0.000000,0.000000,0.000000}%
\pgfsetstrokecolor{currentstroke}%
\pgfsetdash{}{0pt}%
\pgfpathmoveto{\pgfqpoint{3.156933in}{4.263639in}}%
\pgfpathlineto{\pgfqpoint{3.343159in}{4.263639in}}%
\pgfpathlineto{\pgfqpoint{3.343159in}{4.345368in}}%
\pgfpathlineto{\pgfqpoint{3.156933in}{4.345368in}}%
\pgfpathlineto{\pgfqpoint{3.156933in}{4.263639in}}%
\pgfusepath{}%
\end{pgfscope}%
\begin{pgfscope}%
\pgfpathrectangle{\pgfqpoint{0.549740in}{0.463273in}}{\pgfqpoint{9.320225in}{4.495057in}}%
\pgfusepath{clip}%
\pgfsetbuttcap%
\pgfsetroundjoin%
\pgfsetlinewidth{0.000000pt}%
\definecolor{currentstroke}{rgb}{0.000000,0.000000,0.000000}%
\pgfsetstrokecolor{currentstroke}%
\pgfsetdash{}{0pt}%
\pgfpathmoveto{\pgfqpoint{3.343159in}{4.263639in}}%
\pgfpathlineto{\pgfqpoint{3.529386in}{4.263639in}}%
\pgfpathlineto{\pgfqpoint{3.529386in}{4.345368in}}%
\pgfpathlineto{\pgfqpoint{3.343159in}{4.345368in}}%
\pgfpathlineto{\pgfqpoint{3.343159in}{4.263639in}}%
\pgfusepath{}%
\end{pgfscope}%
\begin{pgfscope}%
\pgfpathrectangle{\pgfqpoint{0.549740in}{0.463273in}}{\pgfqpoint{9.320225in}{4.495057in}}%
\pgfusepath{clip}%
\pgfsetbuttcap%
\pgfsetroundjoin%
\pgfsetlinewidth{0.000000pt}%
\definecolor{currentstroke}{rgb}{0.000000,0.000000,0.000000}%
\pgfsetstrokecolor{currentstroke}%
\pgfsetdash{}{0pt}%
\pgfpathmoveto{\pgfqpoint{3.529386in}{4.263639in}}%
\pgfpathlineto{\pgfqpoint{3.715612in}{4.263639in}}%
\pgfpathlineto{\pgfqpoint{3.715612in}{4.345368in}}%
\pgfpathlineto{\pgfqpoint{3.529386in}{4.345368in}}%
\pgfpathlineto{\pgfqpoint{3.529386in}{4.263639in}}%
\pgfusepath{}%
\end{pgfscope}%
\begin{pgfscope}%
\pgfpathrectangle{\pgfqpoint{0.549740in}{0.463273in}}{\pgfqpoint{9.320225in}{4.495057in}}%
\pgfusepath{clip}%
\pgfsetbuttcap%
\pgfsetroundjoin%
\pgfsetlinewidth{0.000000pt}%
\definecolor{currentstroke}{rgb}{0.000000,0.000000,0.000000}%
\pgfsetstrokecolor{currentstroke}%
\pgfsetdash{}{0pt}%
\pgfpathmoveto{\pgfqpoint{3.715612in}{4.263639in}}%
\pgfpathlineto{\pgfqpoint{3.901839in}{4.263639in}}%
\pgfpathlineto{\pgfqpoint{3.901839in}{4.345368in}}%
\pgfpathlineto{\pgfqpoint{3.715612in}{4.345368in}}%
\pgfpathlineto{\pgfqpoint{3.715612in}{4.263639in}}%
\pgfusepath{}%
\end{pgfscope}%
\begin{pgfscope}%
\pgfpathrectangle{\pgfqpoint{0.549740in}{0.463273in}}{\pgfqpoint{9.320225in}{4.495057in}}%
\pgfusepath{clip}%
\pgfsetbuttcap%
\pgfsetroundjoin%
\pgfsetlinewidth{0.000000pt}%
\definecolor{currentstroke}{rgb}{0.000000,0.000000,0.000000}%
\pgfsetstrokecolor{currentstroke}%
\pgfsetdash{}{0pt}%
\pgfpathmoveto{\pgfqpoint{3.901839in}{4.263639in}}%
\pgfpathlineto{\pgfqpoint{4.088065in}{4.263639in}}%
\pgfpathlineto{\pgfqpoint{4.088065in}{4.345368in}}%
\pgfpathlineto{\pgfqpoint{3.901839in}{4.345368in}}%
\pgfpathlineto{\pgfqpoint{3.901839in}{4.263639in}}%
\pgfusepath{}%
\end{pgfscope}%
\begin{pgfscope}%
\pgfpathrectangle{\pgfqpoint{0.549740in}{0.463273in}}{\pgfqpoint{9.320225in}{4.495057in}}%
\pgfusepath{clip}%
\pgfsetbuttcap%
\pgfsetroundjoin%
\pgfsetlinewidth{0.000000pt}%
\definecolor{currentstroke}{rgb}{0.000000,0.000000,0.000000}%
\pgfsetstrokecolor{currentstroke}%
\pgfsetdash{}{0pt}%
\pgfpathmoveto{\pgfqpoint{4.088065in}{4.263639in}}%
\pgfpathlineto{\pgfqpoint{4.274292in}{4.263639in}}%
\pgfpathlineto{\pgfqpoint{4.274292in}{4.345368in}}%
\pgfpathlineto{\pgfqpoint{4.088065in}{4.345368in}}%
\pgfpathlineto{\pgfqpoint{4.088065in}{4.263639in}}%
\pgfusepath{}%
\end{pgfscope}%
\begin{pgfscope}%
\pgfpathrectangle{\pgfqpoint{0.549740in}{0.463273in}}{\pgfqpoint{9.320225in}{4.495057in}}%
\pgfusepath{clip}%
\pgfsetbuttcap%
\pgfsetroundjoin%
\pgfsetlinewidth{0.000000pt}%
\definecolor{currentstroke}{rgb}{0.000000,0.000000,0.000000}%
\pgfsetstrokecolor{currentstroke}%
\pgfsetdash{}{0pt}%
\pgfpathmoveto{\pgfqpoint{4.274292in}{4.263639in}}%
\pgfpathlineto{\pgfqpoint{4.460519in}{4.263639in}}%
\pgfpathlineto{\pgfqpoint{4.460519in}{4.345368in}}%
\pgfpathlineto{\pgfqpoint{4.274292in}{4.345368in}}%
\pgfpathlineto{\pgfqpoint{4.274292in}{4.263639in}}%
\pgfusepath{}%
\end{pgfscope}%
\begin{pgfscope}%
\pgfpathrectangle{\pgfqpoint{0.549740in}{0.463273in}}{\pgfqpoint{9.320225in}{4.495057in}}%
\pgfusepath{clip}%
\pgfsetbuttcap%
\pgfsetroundjoin%
\pgfsetlinewidth{0.000000pt}%
\definecolor{currentstroke}{rgb}{0.000000,0.000000,0.000000}%
\pgfsetstrokecolor{currentstroke}%
\pgfsetdash{}{0pt}%
\pgfpathmoveto{\pgfqpoint{4.460519in}{4.263639in}}%
\pgfpathlineto{\pgfqpoint{4.646745in}{4.263639in}}%
\pgfpathlineto{\pgfqpoint{4.646745in}{4.345368in}}%
\pgfpathlineto{\pgfqpoint{4.460519in}{4.345368in}}%
\pgfpathlineto{\pgfqpoint{4.460519in}{4.263639in}}%
\pgfusepath{}%
\end{pgfscope}%
\begin{pgfscope}%
\pgfpathrectangle{\pgfqpoint{0.549740in}{0.463273in}}{\pgfqpoint{9.320225in}{4.495057in}}%
\pgfusepath{clip}%
\pgfsetbuttcap%
\pgfsetroundjoin%
\pgfsetlinewidth{0.000000pt}%
\definecolor{currentstroke}{rgb}{0.000000,0.000000,0.000000}%
\pgfsetstrokecolor{currentstroke}%
\pgfsetdash{}{0pt}%
\pgfpathmoveto{\pgfqpoint{4.646745in}{4.263639in}}%
\pgfpathlineto{\pgfqpoint{4.832972in}{4.263639in}}%
\pgfpathlineto{\pgfqpoint{4.832972in}{4.345368in}}%
\pgfpathlineto{\pgfqpoint{4.646745in}{4.345368in}}%
\pgfpathlineto{\pgfqpoint{4.646745in}{4.263639in}}%
\pgfusepath{}%
\end{pgfscope}%
\begin{pgfscope}%
\pgfpathrectangle{\pgfqpoint{0.549740in}{0.463273in}}{\pgfqpoint{9.320225in}{4.495057in}}%
\pgfusepath{clip}%
\pgfsetbuttcap%
\pgfsetroundjoin%
\pgfsetlinewidth{0.000000pt}%
\definecolor{currentstroke}{rgb}{0.000000,0.000000,0.000000}%
\pgfsetstrokecolor{currentstroke}%
\pgfsetdash{}{0pt}%
\pgfpathmoveto{\pgfqpoint{4.832972in}{4.263639in}}%
\pgfpathlineto{\pgfqpoint{5.019198in}{4.263639in}}%
\pgfpathlineto{\pgfqpoint{5.019198in}{4.345368in}}%
\pgfpathlineto{\pgfqpoint{4.832972in}{4.345368in}}%
\pgfpathlineto{\pgfqpoint{4.832972in}{4.263639in}}%
\pgfusepath{}%
\end{pgfscope}%
\begin{pgfscope}%
\pgfpathrectangle{\pgfqpoint{0.549740in}{0.463273in}}{\pgfqpoint{9.320225in}{4.495057in}}%
\pgfusepath{clip}%
\pgfsetbuttcap%
\pgfsetroundjoin%
\pgfsetlinewidth{0.000000pt}%
\definecolor{currentstroke}{rgb}{0.000000,0.000000,0.000000}%
\pgfsetstrokecolor{currentstroke}%
\pgfsetdash{}{0pt}%
\pgfpathmoveto{\pgfqpoint{5.019198in}{4.263639in}}%
\pgfpathlineto{\pgfqpoint{5.205425in}{4.263639in}}%
\pgfpathlineto{\pgfqpoint{5.205425in}{4.345368in}}%
\pgfpathlineto{\pgfqpoint{5.019198in}{4.345368in}}%
\pgfpathlineto{\pgfqpoint{5.019198in}{4.263639in}}%
\pgfusepath{}%
\end{pgfscope}%
\begin{pgfscope}%
\pgfpathrectangle{\pgfqpoint{0.549740in}{0.463273in}}{\pgfqpoint{9.320225in}{4.495057in}}%
\pgfusepath{clip}%
\pgfsetbuttcap%
\pgfsetroundjoin%
\pgfsetlinewidth{0.000000pt}%
\definecolor{currentstroke}{rgb}{0.000000,0.000000,0.000000}%
\pgfsetstrokecolor{currentstroke}%
\pgfsetdash{}{0pt}%
\pgfpathmoveto{\pgfqpoint{5.205425in}{4.263639in}}%
\pgfpathlineto{\pgfqpoint{5.391651in}{4.263639in}}%
\pgfpathlineto{\pgfqpoint{5.391651in}{4.345368in}}%
\pgfpathlineto{\pgfqpoint{5.205425in}{4.345368in}}%
\pgfpathlineto{\pgfqpoint{5.205425in}{4.263639in}}%
\pgfusepath{}%
\end{pgfscope}%
\begin{pgfscope}%
\pgfpathrectangle{\pgfqpoint{0.549740in}{0.463273in}}{\pgfqpoint{9.320225in}{4.495057in}}%
\pgfusepath{clip}%
\pgfsetbuttcap%
\pgfsetroundjoin%
\pgfsetlinewidth{0.000000pt}%
\definecolor{currentstroke}{rgb}{0.000000,0.000000,0.000000}%
\pgfsetstrokecolor{currentstroke}%
\pgfsetdash{}{0pt}%
\pgfpathmoveto{\pgfqpoint{5.391651in}{4.263639in}}%
\pgfpathlineto{\pgfqpoint{5.577878in}{4.263639in}}%
\pgfpathlineto{\pgfqpoint{5.577878in}{4.345368in}}%
\pgfpathlineto{\pgfqpoint{5.391651in}{4.345368in}}%
\pgfpathlineto{\pgfqpoint{5.391651in}{4.263639in}}%
\pgfusepath{}%
\end{pgfscope}%
\begin{pgfscope}%
\pgfpathrectangle{\pgfqpoint{0.549740in}{0.463273in}}{\pgfqpoint{9.320225in}{4.495057in}}%
\pgfusepath{clip}%
\pgfsetbuttcap%
\pgfsetroundjoin%
\pgfsetlinewidth{0.000000pt}%
\definecolor{currentstroke}{rgb}{0.000000,0.000000,0.000000}%
\pgfsetstrokecolor{currentstroke}%
\pgfsetdash{}{0pt}%
\pgfpathmoveto{\pgfqpoint{5.577878in}{4.263639in}}%
\pgfpathlineto{\pgfqpoint{5.764104in}{4.263639in}}%
\pgfpathlineto{\pgfqpoint{5.764104in}{4.345368in}}%
\pgfpathlineto{\pgfqpoint{5.577878in}{4.345368in}}%
\pgfpathlineto{\pgfqpoint{5.577878in}{4.263639in}}%
\pgfusepath{}%
\end{pgfscope}%
\begin{pgfscope}%
\pgfpathrectangle{\pgfqpoint{0.549740in}{0.463273in}}{\pgfqpoint{9.320225in}{4.495057in}}%
\pgfusepath{clip}%
\pgfsetbuttcap%
\pgfsetroundjoin%
\pgfsetlinewidth{0.000000pt}%
\definecolor{currentstroke}{rgb}{0.000000,0.000000,0.000000}%
\pgfsetstrokecolor{currentstroke}%
\pgfsetdash{}{0pt}%
\pgfpathmoveto{\pgfqpoint{5.764104in}{4.263639in}}%
\pgfpathlineto{\pgfqpoint{5.950331in}{4.263639in}}%
\pgfpathlineto{\pgfqpoint{5.950331in}{4.345368in}}%
\pgfpathlineto{\pgfqpoint{5.764104in}{4.345368in}}%
\pgfpathlineto{\pgfqpoint{5.764104in}{4.263639in}}%
\pgfusepath{}%
\end{pgfscope}%
\begin{pgfscope}%
\pgfpathrectangle{\pgfqpoint{0.549740in}{0.463273in}}{\pgfqpoint{9.320225in}{4.495057in}}%
\pgfusepath{clip}%
\pgfsetbuttcap%
\pgfsetroundjoin%
\pgfsetlinewidth{0.000000pt}%
\definecolor{currentstroke}{rgb}{0.000000,0.000000,0.000000}%
\pgfsetstrokecolor{currentstroke}%
\pgfsetdash{}{0pt}%
\pgfpathmoveto{\pgfqpoint{5.950331in}{4.263639in}}%
\pgfpathlineto{\pgfqpoint{6.136557in}{4.263639in}}%
\pgfpathlineto{\pgfqpoint{6.136557in}{4.345368in}}%
\pgfpathlineto{\pgfqpoint{5.950331in}{4.345368in}}%
\pgfpathlineto{\pgfqpoint{5.950331in}{4.263639in}}%
\pgfusepath{}%
\end{pgfscope}%
\begin{pgfscope}%
\pgfpathrectangle{\pgfqpoint{0.549740in}{0.463273in}}{\pgfqpoint{9.320225in}{4.495057in}}%
\pgfusepath{clip}%
\pgfsetbuttcap%
\pgfsetroundjoin%
\pgfsetlinewidth{0.000000pt}%
\definecolor{currentstroke}{rgb}{0.000000,0.000000,0.000000}%
\pgfsetstrokecolor{currentstroke}%
\pgfsetdash{}{0pt}%
\pgfpathmoveto{\pgfqpoint{6.136557in}{4.263639in}}%
\pgfpathlineto{\pgfqpoint{6.322784in}{4.263639in}}%
\pgfpathlineto{\pgfqpoint{6.322784in}{4.345368in}}%
\pgfpathlineto{\pgfqpoint{6.136557in}{4.345368in}}%
\pgfpathlineto{\pgfqpoint{6.136557in}{4.263639in}}%
\pgfusepath{}%
\end{pgfscope}%
\begin{pgfscope}%
\pgfpathrectangle{\pgfqpoint{0.549740in}{0.463273in}}{\pgfqpoint{9.320225in}{4.495057in}}%
\pgfusepath{clip}%
\pgfsetbuttcap%
\pgfsetroundjoin%
\pgfsetlinewidth{0.000000pt}%
\definecolor{currentstroke}{rgb}{0.000000,0.000000,0.000000}%
\pgfsetstrokecolor{currentstroke}%
\pgfsetdash{}{0pt}%
\pgfpathmoveto{\pgfqpoint{6.322784in}{4.263639in}}%
\pgfpathlineto{\pgfqpoint{6.509011in}{4.263639in}}%
\pgfpathlineto{\pgfqpoint{6.509011in}{4.345368in}}%
\pgfpathlineto{\pgfqpoint{6.322784in}{4.345368in}}%
\pgfpathlineto{\pgfqpoint{6.322784in}{4.263639in}}%
\pgfusepath{}%
\end{pgfscope}%
\begin{pgfscope}%
\pgfpathrectangle{\pgfqpoint{0.549740in}{0.463273in}}{\pgfqpoint{9.320225in}{4.495057in}}%
\pgfusepath{clip}%
\pgfsetbuttcap%
\pgfsetroundjoin%
\pgfsetlinewidth{0.000000pt}%
\definecolor{currentstroke}{rgb}{0.000000,0.000000,0.000000}%
\pgfsetstrokecolor{currentstroke}%
\pgfsetdash{}{0pt}%
\pgfpathmoveto{\pgfqpoint{6.509011in}{4.263639in}}%
\pgfpathlineto{\pgfqpoint{6.695237in}{4.263639in}}%
\pgfpathlineto{\pgfqpoint{6.695237in}{4.345368in}}%
\pgfpathlineto{\pgfqpoint{6.509011in}{4.345368in}}%
\pgfpathlineto{\pgfqpoint{6.509011in}{4.263639in}}%
\pgfusepath{}%
\end{pgfscope}%
\begin{pgfscope}%
\pgfpathrectangle{\pgfqpoint{0.549740in}{0.463273in}}{\pgfqpoint{9.320225in}{4.495057in}}%
\pgfusepath{clip}%
\pgfsetbuttcap%
\pgfsetroundjoin%
\pgfsetlinewidth{0.000000pt}%
\definecolor{currentstroke}{rgb}{0.000000,0.000000,0.000000}%
\pgfsetstrokecolor{currentstroke}%
\pgfsetdash{}{0pt}%
\pgfpathmoveto{\pgfqpoint{6.695237in}{4.263639in}}%
\pgfpathlineto{\pgfqpoint{6.881464in}{4.263639in}}%
\pgfpathlineto{\pgfqpoint{6.881464in}{4.345368in}}%
\pgfpathlineto{\pgfqpoint{6.695237in}{4.345368in}}%
\pgfpathlineto{\pgfqpoint{6.695237in}{4.263639in}}%
\pgfusepath{}%
\end{pgfscope}%
\begin{pgfscope}%
\pgfpathrectangle{\pgfqpoint{0.549740in}{0.463273in}}{\pgfqpoint{9.320225in}{4.495057in}}%
\pgfusepath{clip}%
\pgfsetbuttcap%
\pgfsetroundjoin%
\pgfsetlinewidth{0.000000pt}%
\definecolor{currentstroke}{rgb}{0.000000,0.000000,0.000000}%
\pgfsetstrokecolor{currentstroke}%
\pgfsetdash{}{0pt}%
\pgfpathmoveto{\pgfqpoint{6.881464in}{4.263639in}}%
\pgfpathlineto{\pgfqpoint{7.067690in}{4.263639in}}%
\pgfpathlineto{\pgfqpoint{7.067690in}{4.345368in}}%
\pgfpathlineto{\pgfqpoint{6.881464in}{4.345368in}}%
\pgfpathlineto{\pgfqpoint{6.881464in}{4.263639in}}%
\pgfusepath{}%
\end{pgfscope}%
\begin{pgfscope}%
\pgfpathrectangle{\pgfqpoint{0.549740in}{0.463273in}}{\pgfqpoint{9.320225in}{4.495057in}}%
\pgfusepath{clip}%
\pgfsetbuttcap%
\pgfsetroundjoin%
\pgfsetlinewidth{0.000000pt}%
\definecolor{currentstroke}{rgb}{0.000000,0.000000,0.000000}%
\pgfsetstrokecolor{currentstroke}%
\pgfsetdash{}{0pt}%
\pgfpathmoveto{\pgfqpoint{7.067690in}{4.263639in}}%
\pgfpathlineto{\pgfqpoint{7.253917in}{4.263639in}}%
\pgfpathlineto{\pgfqpoint{7.253917in}{4.345368in}}%
\pgfpathlineto{\pgfqpoint{7.067690in}{4.345368in}}%
\pgfpathlineto{\pgfqpoint{7.067690in}{4.263639in}}%
\pgfusepath{}%
\end{pgfscope}%
\begin{pgfscope}%
\pgfpathrectangle{\pgfqpoint{0.549740in}{0.463273in}}{\pgfqpoint{9.320225in}{4.495057in}}%
\pgfusepath{clip}%
\pgfsetbuttcap%
\pgfsetroundjoin%
\pgfsetlinewidth{0.000000pt}%
\definecolor{currentstroke}{rgb}{0.000000,0.000000,0.000000}%
\pgfsetstrokecolor{currentstroke}%
\pgfsetdash{}{0pt}%
\pgfpathmoveto{\pgfqpoint{7.253917in}{4.263639in}}%
\pgfpathlineto{\pgfqpoint{7.440143in}{4.263639in}}%
\pgfpathlineto{\pgfqpoint{7.440143in}{4.345368in}}%
\pgfpathlineto{\pgfqpoint{7.253917in}{4.345368in}}%
\pgfpathlineto{\pgfqpoint{7.253917in}{4.263639in}}%
\pgfusepath{}%
\end{pgfscope}%
\begin{pgfscope}%
\pgfpathrectangle{\pgfqpoint{0.549740in}{0.463273in}}{\pgfqpoint{9.320225in}{4.495057in}}%
\pgfusepath{clip}%
\pgfsetbuttcap%
\pgfsetroundjoin%
\pgfsetlinewidth{0.000000pt}%
\definecolor{currentstroke}{rgb}{0.000000,0.000000,0.000000}%
\pgfsetstrokecolor{currentstroke}%
\pgfsetdash{}{0pt}%
\pgfpathmoveto{\pgfqpoint{7.440143in}{4.263639in}}%
\pgfpathlineto{\pgfqpoint{7.626370in}{4.263639in}}%
\pgfpathlineto{\pgfqpoint{7.626370in}{4.345368in}}%
\pgfpathlineto{\pgfqpoint{7.440143in}{4.345368in}}%
\pgfpathlineto{\pgfqpoint{7.440143in}{4.263639in}}%
\pgfusepath{}%
\end{pgfscope}%
\begin{pgfscope}%
\pgfpathrectangle{\pgfqpoint{0.549740in}{0.463273in}}{\pgfqpoint{9.320225in}{4.495057in}}%
\pgfusepath{clip}%
\pgfsetbuttcap%
\pgfsetroundjoin%
\pgfsetlinewidth{0.000000pt}%
\definecolor{currentstroke}{rgb}{0.000000,0.000000,0.000000}%
\pgfsetstrokecolor{currentstroke}%
\pgfsetdash{}{0pt}%
\pgfpathmoveto{\pgfqpoint{7.626370in}{4.263639in}}%
\pgfpathlineto{\pgfqpoint{7.812596in}{4.263639in}}%
\pgfpathlineto{\pgfqpoint{7.812596in}{4.345368in}}%
\pgfpathlineto{\pgfqpoint{7.626370in}{4.345368in}}%
\pgfpathlineto{\pgfqpoint{7.626370in}{4.263639in}}%
\pgfusepath{}%
\end{pgfscope}%
\begin{pgfscope}%
\pgfpathrectangle{\pgfqpoint{0.549740in}{0.463273in}}{\pgfqpoint{9.320225in}{4.495057in}}%
\pgfusepath{clip}%
\pgfsetbuttcap%
\pgfsetroundjoin%
\pgfsetlinewidth{0.000000pt}%
\definecolor{currentstroke}{rgb}{0.000000,0.000000,0.000000}%
\pgfsetstrokecolor{currentstroke}%
\pgfsetdash{}{0pt}%
\pgfpathmoveto{\pgfqpoint{7.812596in}{4.263639in}}%
\pgfpathlineto{\pgfqpoint{7.998823in}{4.263639in}}%
\pgfpathlineto{\pgfqpoint{7.998823in}{4.345368in}}%
\pgfpathlineto{\pgfqpoint{7.812596in}{4.345368in}}%
\pgfpathlineto{\pgfqpoint{7.812596in}{4.263639in}}%
\pgfusepath{}%
\end{pgfscope}%
\begin{pgfscope}%
\pgfpathrectangle{\pgfqpoint{0.549740in}{0.463273in}}{\pgfqpoint{9.320225in}{4.495057in}}%
\pgfusepath{clip}%
\pgfsetbuttcap%
\pgfsetroundjoin%
\pgfsetlinewidth{0.000000pt}%
\definecolor{currentstroke}{rgb}{0.000000,0.000000,0.000000}%
\pgfsetstrokecolor{currentstroke}%
\pgfsetdash{}{0pt}%
\pgfpathmoveto{\pgfqpoint{7.998823in}{4.263639in}}%
\pgfpathlineto{\pgfqpoint{8.185049in}{4.263639in}}%
\pgfpathlineto{\pgfqpoint{8.185049in}{4.345368in}}%
\pgfpathlineto{\pgfqpoint{7.998823in}{4.345368in}}%
\pgfpathlineto{\pgfqpoint{7.998823in}{4.263639in}}%
\pgfusepath{}%
\end{pgfscope}%
\begin{pgfscope}%
\pgfpathrectangle{\pgfqpoint{0.549740in}{0.463273in}}{\pgfqpoint{9.320225in}{4.495057in}}%
\pgfusepath{clip}%
\pgfsetbuttcap%
\pgfsetroundjoin%
\pgfsetlinewidth{0.000000pt}%
\definecolor{currentstroke}{rgb}{0.000000,0.000000,0.000000}%
\pgfsetstrokecolor{currentstroke}%
\pgfsetdash{}{0pt}%
\pgfpathmoveto{\pgfqpoint{8.185049in}{4.263639in}}%
\pgfpathlineto{\pgfqpoint{8.371276in}{4.263639in}}%
\pgfpathlineto{\pgfqpoint{8.371276in}{4.345368in}}%
\pgfpathlineto{\pgfqpoint{8.185049in}{4.345368in}}%
\pgfpathlineto{\pgfqpoint{8.185049in}{4.263639in}}%
\pgfusepath{}%
\end{pgfscope}%
\begin{pgfscope}%
\pgfpathrectangle{\pgfqpoint{0.549740in}{0.463273in}}{\pgfqpoint{9.320225in}{4.495057in}}%
\pgfusepath{clip}%
\pgfsetbuttcap%
\pgfsetroundjoin%
\pgfsetlinewidth{0.000000pt}%
\definecolor{currentstroke}{rgb}{0.000000,0.000000,0.000000}%
\pgfsetstrokecolor{currentstroke}%
\pgfsetdash{}{0pt}%
\pgfpathmoveto{\pgfqpoint{8.371276in}{4.263639in}}%
\pgfpathlineto{\pgfqpoint{8.557503in}{4.263639in}}%
\pgfpathlineto{\pgfqpoint{8.557503in}{4.345368in}}%
\pgfpathlineto{\pgfqpoint{8.371276in}{4.345368in}}%
\pgfpathlineto{\pgfqpoint{8.371276in}{4.263639in}}%
\pgfusepath{}%
\end{pgfscope}%
\begin{pgfscope}%
\pgfpathrectangle{\pgfqpoint{0.549740in}{0.463273in}}{\pgfqpoint{9.320225in}{4.495057in}}%
\pgfusepath{clip}%
\pgfsetbuttcap%
\pgfsetroundjoin%
\pgfsetlinewidth{0.000000pt}%
\definecolor{currentstroke}{rgb}{0.000000,0.000000,0.000000}%
\pgfsetstrokecolor{currentstroke}%
\pgfsetdash{}{0pt}%
\pgfpathmoveto{\pgfqpoint{8.557503in}{4.263639in}}%
\pgfpathlineto{\pgfqpoint{8.743729in}{4.263639in}}%
\pgfpathlineto{\pgfqpoint{8.743729in}{4.345368in}}%
\pgfpathlineto{\pgfqpoint{8.557503in}{4.345368in}}%
\pgfpathlineto{\pgfqpoint{8.557503in}{4.263639in}}%
\pgfusepath{}%
\end{pgfscope}%
\begin{pgfscope}%
\pgfpathrectangle{\pgfqpoint{0.549740in}{0.463273in}}{\pgfqpoint{9.320225in}{4.495057in}}%
\pgfusepath{clip}%
\pgfsetbuttcap%
\pgfsetroundjoin%
\pgfsetlinewidth{0.000000pt}%
\definecolor{currentstroke}{rgb}{0.000000,0.000000,0.000000}%
\pgfsetstrokecolor{currentstroke}%
\pgfsetdash{}{0pt}%
\pgfpathmoveto{\pgfqpoint{8.743729in}{4.263639in}}%
\pgfpathlineto{\pgfqpoint{8.929956in}{4.263639in}}%
\pgfpathlineto{\pgfqpoint{8.929956in}{4.345368in}}%
\pgfpathlineto{\pgfqpoint{8.743729in}{4.345368in}}%
\pgfpathlineto{\pgfqpoint{8.743729in}{4.263639in}}%
\pgfusepath{}%
\end{pgfscope}%
\begin{pgfscope}%
\pgfpathrectangle{\pgfqpoint{0.549740in}{0.463273in}}{\pgfqpoint{9.320225in}{4.495057in}}%
\pgfusepath{clip}%
\pgfsetbuttcap%
\pgfsetroundjoin%
\definecolor{currentfill}{rgb}{0.472869,0.711325,0.955316}%
\pgfsetfillcolor{currentfill}%
\pgfsetlinewidth{0.000000pt}%
\definecolor{currentstroke}{rgb}{0.000000,0.000000,0.000000}%
\pgfsetstrokecolor{currentstroke}%
\pgfsetdash{}{0pt}%
\pgfpathmoveto{\pgfqpoint{8.929956in}{4.263639in}}%
\pgfpathlineto{\pgfqpoint{9.116182in}{4.263639in}}%
\pgfpathlineto{\pgfqpoint{9.116182in}{4.345368in}}%
\pgfpathlineto{\pgfqpoint{8.929956in}{4.345368in}}%
\pgfpathlineto{\pgfqpoint{8.929956in}{4.263639in}}%
\pgfusepath{fill}%
\end{pgfscope}%
\begin{pgfscope}%
\pgfpathrectangle{\pgfqpoint{0.549740in}{0.463273in}}{\pgfqpoint{9.320225in}{4.495057in}}%
\pgfusepath{clip}%
\pgfsetbuttcap%
\pgfsetroundjoin%
\pgfsetlinewidth{0.000000pt}%
\definecolor{currentstroke}{rgb}{0.000000,0.000000,0.000000}%
\pgfsetstrokecolor{currentstroke}%
\pgfsetdash{}{0pt}%
\pgfpathmoveto{\pgfqpoint{9.116182in}{4.263639in}}%
\pgfpathlineto{\pgfqpoint{9.302409in}{4.263639in}}%
\pgfpathlineto{\pgfqpoint{9.302409in}{4.345368in}}%
\pgfpathlineto{\pgfqpoint{9.116182in}{4.345368in}}%
\pgfpathlineto{\pgfqpoint{9.116182in}{4.263639in}}%
\pgfusepath{}%
\end{pgfscope}%
\begin{pgfscope}%
\pgfpathrectangle{\pgfqpoint{0.549740in}{0.463273in}}{\pgfqpoint{9.320225in}{4.495057in}}%
\pgfusepath{clip}%
\pgfsetbuttcap%
\pgfsetroundjoin%
\pgfsetlinewidth{0.000000pt}%
\definecolor{currentstroke}{rgb}{0.000000,0.000000,0.000000}%
\pgfsetstrokecolor{currentstroke}%
\pgfsetdash{}{0pt}%
\pgfpathmoveto{\pgfqpoint{9.302409in}{4.263639in}}%
\pgfpathlineto{\pgfqpoint{9.488635in}{4.263639in}}%
\pgfpathlineto{\pgfqpoint{9.488635in}{4.345368in}}%
\pgfpathlineto{\pgfqpoint{9.302409in}{4.345368in}}%
\pgfpathlineto{\pgfqpoint{9.302409in}{4.263639in}}%
\pgfusepath{}%
\end{pgfscope}%
\begin{pgfscope}%
\pgfpathrectangle{\pgfqpoint{0.549740in}{0.463273in}}{\pgfqpoint{9.320225in}{4.495057in}}%
\pgfusepath{clip}%
\pgfsetbuttcap%
\pgfsetroundjoin%
\pgfsetlinewidth{0.000000pt}%
\definecolor{currentstroke}{rgb}{0.000000,0.000000,0.000000}%
\pgfsetstrokecolor{currentstroke}%
\pgfsetdash{}{0pt}%
\pgfpathmoveto{\pgfqpoint{9.488635in}{4.263639in}}%
\pgfpathlineto{\pgfqpoint{9.674862in}{4.263639in}}%
\pgfpathlineto{\pgfqpoint{9.674862in}{4.345368in}}%
\pgfpathlineto{\pgfqpoint{9.488635in}{4.345368in}}%
\pgfpathlineto{\pgfqpoint{9.488635in}{4.263639in}}%
\pgfusepath{}%
\end{pgfscope}%
\begin{pgfscope}%
\pgfpathrectangle{\pgfqpoint{0.549740in}{0.463273in}}{\pgfqpoint{9.320225in}{4.495057in}}%
\pgfusepath{clip}%
\pgfsetbuttcap%
\pgfsetroundjoin%
\pgfsetlinewidth{0.000000pt}%
\definecolor{currentstroke}{rgb}{0.000000,0.000000,0.000000}%
\pgfsetstrokecolor{currentstroke}%
\pgfsetdash{}{0pt}%
\pgfpathmoveto{\pgfqpoint{9.674862in}{4.263639in}}%
\pgfpathlineto{\pgfqpoint{9.861088in}{4.263639in}}%
\pgfpathlineto{\pgfqpoint{9.861088in}{4.345368in}}%
\pgfpathlineto{\pgfqpoint{9.674862in}{4.345368in}}%
\pgfpathlineto{\pgfqpoint{9.674862in}{4.263639in}}%
\pgfusepath{}%
\end{pgfscope}%
\begin{pgfscope}%
\pgfpathrectangle{\pgfqpoint{0.549740in}{0.463273in}}{\pgfqpoint{9.320225in}{4.495057in}}%
\pgfusepath{clip}%
\pgfsetbuttcap%
\pgfsetroundjoin%
\pgfsetlinewidth{0.000000pt}%
\definecolor{currentstroke}{rgb}{0.000000,0.000000,0.000000}%
\pgfsetstrokecolor{currentstroke}%
\pgfsetdash{}{0pt}%
\pgfpathmoveto{\pgfqpoint{0.549761in}{4.345368in}}%
\pgfpathlineto{\pgfqpoint{0.735988in}{4.345368in}}%
\pgfpathlineto{\pgfqpoint{0.735988in}{4.427096in}}%
\pgfpathlineto{\pgfqpoint{0.549761in}{4.427096in}}%
\pgfpathlineto{\pgfqpoint{0.549761in}{4.345368in}}%
\pgfusepath{}%
\end{pgfscope}%
\begin{pgfscope}%
\pgfpathrectangle{\pgfqpoint{0.549740in}{0.463273in}}{\pgfqpoint{9.320225in}{4.495057in}}%
\pgfusepath{clip}%
\pgfsetbuttcap%
\pgfsetroundjoin%
\pgfsetlinewidth{0.000000pt}%
\definecolor{currentstroke}{rgb}{0.000000,0.000000,0.000000}%
\pgfsetstrokecolor{currentstroke}%
\pgfsetdash{}{0pt}%
\pgfpathmoveto{\pgfqpoint{0.735988in}{4.345368in}}%
\pgfpathlineto{\pgfqpoint{0.922214in}{4.345368in}}%
\pgfpathlineto{\pgfqpoint{0.922214in}{4.427096in}}%
\pgfpathlineto{\pgfqpoint{0.735988in}{4.427096in}}%
\pgfpathlineto{\pgfqpoint{0.735988in}{4.345368in}}%
\pgfusepath{}%
\end{pgfscope}%
\begin{pgfscope}%
\pgfpathrectangle{\pgfqpoint{0.549740in}{0.463273in}}{\pgfqpoint{9.320225in}{4.495057in}}%
\pgfusepath{clip}%
\pgfsetbuttcap%
\pgfsetroundjoin%
\pgfsetlinewidth{0.000000pt}%
\definecolor{currentstroke}{rgb}{0.000000,0.000000,0.000000}%
\pgfsetstrokecolor{currentstroke}%
\pgfsetdash{}{0pt}%
\pgfpathmoveto{\pgfqpoint{0.922214in}{4.345368in}}%
\pgfpathlineto{\pgfqpoint{1.108441in}{4.345368in}}%
\pgfpathlineto{\pgfqpoint{1.108441in}{4.427096in}}%
\pgfpathlineto{\pgfqpoint{0.922214in}{4.427096in}}%
\pgfpathlineto{\pgfqpoint{0.922214in}{4.345368in}}%
\pgfusepath{}%
\end{pgfscope}%
\begin{pgfscope}%
\pgfpathrectangle{\pgfqpoint{0.549740in}{0.463273in}}{\pgfqpoint{9.320225in}{4.495057in}}%
\pgfusepath{clip}%
\pgfsetbuttcap%
\pgfsetroundjoin%
\pgfsetlinewidth{0.000000pt}%
\definecolor{currentstroke}{rgb}{0.000000,0.000000,0.000000}%
\pgfsetstrokecolor{currentstroke}%
\pgfsetdash{}{0pt}%
\pgfpathmoveto{\pgfqpoint{1.108441in}{4.345368in}}%
\pgfpathlineto{\pgfqpoint{1.294667in}{4.345368in}}%
\pgfpathlineto{\pgfqpoint{1.294667in}{4.427096in}}%
\pgfpathlineto{\pgfqpoint{1.108441in}{4.427096in}}%
\pgfpathlineto{\pgfqpoint{1.108441in}{4.345368in}}%
\pgfusepath{}%
\end{pgfscope}%
\begin{pgfscope}%
\pgfpathrectangle{\pgfqpoint{0.549740in}{0.463273in}}{\pgfqpoint{9.320225in}{4.495057in}}%
\pgfusepath{clip}%
\pgfsetbuttcap%
\pgfsetroundjoin%
\pgfsetlinewidth{0.000000pt}%
\definecolor{currentstroke}{rgb}{0.000000,0.000000,0.000000}%
\pgfsetstrokecolor{currentstroke}%
\pgfsetdash{}{0pt}%
\pgfpathmoveto{\pgfqpoint{1.294667in}{4.345368in}}%
\pgfpathlineto{\pgfqpoint{1.480894in}{4.345368in}}%
\pgfpathlineto{\pgfqpoint{1.480894in}{4.427096in}}%
\pgfpathlineto{\pgfqpoint{1.294667in}{4.427096in}}%
\pgfpathlineto{\pgfqpoint{1.294667in}{4.345368in}}%
\pgfusepath{}%
\end{pgfscope}%
\begin{pgfscope}%
\pgfpathrectangle{\pgfqpoint{0.549740in}{0.463273in}}{\pgfqpoint{9.320225in}{4.495057in}}%
\pgfusepath{clip}%
\pgfsetbuttcap%
\pgfsetroundjoin%
\pgfsetlinewidth{0.000000pt}%
\definecolor{currentstroke}{rgb}{0.000000,0.000000,0.000000}%
\pgfsetstrokecolor{currentstroke}%
\pgfsetdash{}{0pt}%
\pgfpathmoveto{\pgfqpoint{1.480894in}{4.345368in}}%
\pgfpathlineto{\pgfqpoint{1.667120in}{4.345368in}}%
\pgfpathlineto{\pgfqpoint{1.667120in}{4.427096in}}%
\pgfpathlineto{\pgfqpoint{1.480894in}{4.427096in}}%
\pgfpathlineto{\pgfqpoint{1.480894in}{4.345368in}}%
\pgfusepath{}%
\end{pgfscope}%
\begin{pgfscope}%
\pgfpathrectangle{\pgfqpoint{0.549740in}{0.463273in}}{\pgfqpoint{9.320225in}{4.495057in}}%
\pgfusepath{clip}%
\pgfsetbuttcap%
\pgfsetroundjoin%
\pgfsetlinewidth{0.000000pt}%
\definecolor{currentstroke}{rgb}{0.000000,0.000000,0.000000}%
\pgfsetstrokecolor{currentstroke}%
\pgfsetdash{}{0pt}%
\pgfpathmoveto{\pgfqpoint{1.667120in}{4.345368in}}%
\pgfpathlineto{\pgfqpoint{1.853347in}{4.345368in}}%
\pgfpathlineto{\pgfqpoint{1.853347in}{4.427096in}}%
\pgfpathlineto{\pgfqpoint{1.667120in}{4.427096in}}%
\pgfpathlineto{\pgfqpoint{1.667120in}{4.345368in}}%
\pgfusepath{}%
\end{pgfscope}%
\begin{pgfscope}%
\pgfpathrectangle{\pgfqpoint{0.549740in}{0.463273in}}{\pgfqpoint{9.320225in}{4.495057in}}%
\pgfusepath{clip}%
\pgfsetbuttcap%
\pgfsetroundjoin%
\pgfsetlinewidth{0.000000pt}%
\definecolor{currentstroke}{rgb}{0.000000,0.000000,0.000000}%
\pgfsetstrokecolor{currentstroke}%
\pgfsetdash{}{0pt}%
\pgfpathmoveto{\pgfqpoint{1.853347in}{4.345368in}}%
\pgfpathlineto{\pgfqpoint{2.039573in}{4.345368in}}%
\pgfpathlineto{\pgfqpoint{2.039573in}{4.427096in}}%
\pgfpathlineto{\pgfqpoint{1.853347in}{4.427096in}}%
\pgfpathlineto{\pgfqpoint{1.853347in}{4.345368in}}%
\pgfusepath{}%
\end{pgfscope}%
\begin{pgfscope}%
\pgfpathrectangle{\pgfqpoint{0.549740in}{0.463273in}}{\pgfqpoint{9.320225in}{4.495057in}}%
\pgfusepath{clip}%
\pgfsetbuttcap%
\pgfsetroundjoin%
\pgfsetlinewidth{0.000000pt}%
\definecolor{currentstroke}{rgb}{0.000000,0.000000,0.000000}%
\pgfsetstrokecolor{currentstroke}%
\pgfsetdash{}{0pt}%
\pgfpathmoveto{\pgfqpoint{2.039573in}{4.345368in}}%
\pgfpathlineto{\pgfqpoint{2.225800in}{4.345368in}}%
\pgfpathlineto{\pgfqpoint{2.225800in}{4.427096in}}%
\pgfpathlineto{\pgfqpoint{2.039573in}{4.427096in}}%
\pgfpathlineto{\pgfqpoint{2.039573in}{4.345368in}}%
\pgfusepath{}%
\end{pgfscope}%
\begin{pgfscope}%
\pgfpathrectangle{\pgfqpoint{0.549740in}{0.463273in}}{\pgfqpoint{9.320225in}{4.495057in}}%
\pgfusepath{clip}%
\pgfsetbuttcap%
\pgfsetroundjoin%
\pgfsetlinewidth{0.000000pt}%
\definecolor{currentstroke}{rgb}{0.000000,0.000000,0.000000}%
\pgfsetstrokecolor{currentstroke}%
\pgfsetdash{}{0pt}%
\pgfpathmoveto{\pgfqpoint{2.225800in}{4.345368in}}%
\pgfpathlineto{\pgfqpoint{2.412027in}{4.345368in}}%
\pgfpathlineto{\pgfqpoint{2.412027in}{4.427096in}}%
\pgfpathlineto{\pgfqpoint{2.225800in}{4.427096in}}%
\pgfpathlineto{\pgfqpoint{2.225800in}{4.345368in}}%
\pgfusepath{}%
\end{pgfscope}%
\begin{pgfscope}%
\pgfpathrectangle{\pgfqpoint{0.549740in}{0.463273in}}{\pgfqpoint{9.320225in}{4.495057in}}%
\pgfusepath{clip}%
\pgfsetbuttcap%
\pgfsetroundjoin%
\pgfsetlinewidth{0.000000pt}%
\definecolor{currentstroke}{rgb}{0.000000,0.000000,0.000000}%
\pgfsetstrokecolor{currentstroke}%
\pgfsetdash{}{0pt}%
\pgfpathmoveto{\pgfqpoint{2.412027in}{4.345368in}}%
\pgfpathlineto{\pgfqpoint{2.598253in}{4.345368in}}%
\pgfpathlineto{\pgfqpoint{2.598253in}{4.427096in}}%
\pgfpathlineto{\pgfqpoint{2.412027in}{4.427096in}}%
\pgfpathlineto{\pgfqpoint{2.412027in}{4.345368in}}%
\pgfusepath{}%
\end{pgfscope}%
\begin{pgfscope}%
\pgfpathrectangle{\pgfqpoint{0.549740in}{0.463273in}}{\pgfqpoint{9.320225in}{4.495057in}}%
\pgfusepath{clip}%
\pgfsetbuttcap%
\pgfsetroundjoin%
\pgfsetlinewidth{0.000000pt}%
\definecolor{currentstroke}{rgb}{0.000000,0.000000,0.000000}%
\pgfsetstrokecolor{currentstroke}%
\pgfsetdash{}{0pt}%
\pgfpathmoveto{\pgfqpoint{2.598253in}{4.345368in}}%
\pgfpathlineto{\pgfqpoint{2.784480in}{4.345368in}}%
\pgfpathlineto{\pgfqpoint{2.784480in}{4.427096in}}%
\pgfpathlineto{\pgfqpoint{2.598253in}{4.427096in}}%
\pgfpathlineto{\pgfqpoint{2.598253in}{4.345368in}}%
\pgfusepath{}%
\end{pgfscope}%
\begin{pgfscope}%
\pgfpathrectangle{\pgfqpoint{0.549740in}{0.463273in}}{\pgfqpoint{9.320225in}{4.495057in}}%
\pgfusepath{clip}%
\pgfsetbuttcap%
\pgfsetroundjoin%
\pgfsetlinewidth{0.000000pt}%
\definecolor{currentstroke}{rgb}{0.000000,0.000000,0.000000}%
\pgfsetstrokecolor{currentstroke}%
\pgfsetdash{}{0pt}%
\pgfpathmoveto{\pgfqpoint{2.784480in}{4.345368in}}%
\pgfpathlineto{\pgfqpoint{2.970706in}{4.345368in}}%
\pgfpathlineto{\pgfqpoint{2.970706in}{4.427096in}}%
\pgfpathlineto{\pgfqpoint{2.784480in}{4.427096in}}%
\pgfpathlineto{\pgfqpoint{2.784480in}{4.345368in}}%
\pgfusepath{}%
\end{pgfscope}%
\begin{pgfscope}%
\pgfpathrectangle{\pgfqpoint{0.549740in}{0.463273in}}{\pgfqpoint{9.320225in}{4.495057in}}%
\pgfusepath{clip}%
\pgfsetbuttcap%
\pgfsetroundjoin%
\pgfsetlinewidth{0.000000pt}%
\definecolor{currentstroke}{rgb}{0.000000,0.000000,0.000000}%
\pgfsetstrokecolor{currentstroke}%
\pgfsetdash{}{0pt}%
\pgfpathmoveto{\pgfqpoint{2.970706in}{4.345368in}}%
\pgfpathlineto{\pgfqpoint{3.156933in}{4.345368in}}%
\pgfpathlineto{\pgfqpoint{3.156933in}{4.427096in}}%
\pgfpathlineto{\pgfqpoint{2.970706in}{4.427096in}}%
\pgfpathlineto{\pgfqpoint{2.970706in}{4.345368in}}%
\pgfusepath{}%
\end{pgfscope}%
\begin{pgfscope}%
\pgfpathrectangle{\pgfqpoint{0.549740in}{0.463273in}}{\pgfqpoint{9.320225in}{4.495057in}}%
\pgfusepath{clip}%
\pgfsetbuttcap%
\pgfsetroundjoin%
\pgfsetlinewidth{0.000000pt}%
\definecolor{currentstroke}{rgb}{0.000000,0.000000,0.000000}%
\pgfsetstrokecolor{currentstroke}%
\pgfsetdash{}{0pt}%
\pgfpathmoveto{\pgfqpoint{3.156933in}{4.345368in}}%
\pgfpathlineto{\pgfqpoint{3.343159in}{4.345368in}}%
\pgfpathlineto{\pgfqpoint{3.343159in}{4.427096in}}%
\pgfpathlineto{\pgfqpoint{3.156933in}{4.427096in}}%
\pgfpathlineto{\pgfqpoint{3.156933in}{4.345368in}}%
\pgfusepath{}%
\end{pgfscope}%
\begin{pgfscope}%
\pgfpathrectangle{\pgfqpoint{0.549740in}{0.463273in}}{\pgfqpoint{9.320225in}{4.495057in}}%
\pgfusepath{clip}%
\pgfsetbuttcap%
\pgfsetroundjoin%
\pgfsetlinewidth{0.000000pt}%
\definecolor{currentstroke}{rgb}{0.000000,0.000000,0.000000}%
\pgfsetstrokecolor{currentstroke}%
\pgfsetdash{}{0pt}%
\pgfpathmoveto{\pgfqpoint{3.343159in}{4.345368in}}%
\pgfpathlineto{\pgfqpoint{3.529386in}{4.345368in}}%
\pgfpathlineto{\pgfqpoint{3.529386in}{4.427096in}}%
\pgfpathlineto{\pgfqpoint{3.343159in}{4.427096in}}%
\pgfpathlineto{\pgfqpoint{3.343159in}{4.345368in}}%
\pgfusepath{}%
\end{pgfscope}%
\begin{pgfscope}%
\pgfpathrectangle{\pgfqpoint{0.549740in}{0.463273in}}{\pgfqpoint{9.320225in}{4.495057in}}%
\pgfusepath{clip}%
\pgfsetbuttcap%
\pgfsetroundjoin%
\pgfsetlinewidth{0.000000pt}%
\definecolor{currentstroke}{rgb}{0.000000,0.000000,0.000000}%
\pgfsetstrokecolor{currentstroke}%
\pgfsetdash{}{0pt}%
\pgfpathmoveto{\pgfqpoint{3.529386in}{4.345368in}}%
\pgfpathlineto{\pgfqpoint{3.715612in}{4.345368in}}%
\pgfpathlineto{\pgfqpoint{3.715612in}{4.427096in}}%
\pgfpathlineto{\pgfqpoint{3.529386in}{4.427096in}}%
\pgfpathlineto{\pgfqpoint{3.529386in}{4.345368in}}%
\pgfusepath{}%
\end{pgfscope}%
\begin{pgfscope}%
\pgfpathrectangle{\pgfqpoint{0.549740in}{0.463273in}}{\pgfqpoint{9.320225in}{4.495057in}}%
\pgfusepath{clip}%
\pgfsetbuttcap%
\pgfsetroundjoin%
\pgfsetlinewidth{0.000000pt}%
\definecolor{currentstroke}{rgb}{0.000000,0.000000,0.000000}%
\pgfsetstrokecolor{currentstroke}%
\pgfsetdash{}{0pt}%
\pgfpathmoveto{\pgfqpoint{3.715612in}{4.345368in}}%
\pgfpathlineto{\pgfqpoint{3.901839in}{4.345368in}}%
\pgfpathlineto{\pgfqpoint{3.901839in}{4.427096in}}%
\pgfpathlineto{\pgfqpoint{3.715612in}{4.427096in}}%
\pgfpathlineto{\pgfqpoint{3.715612in}{4.345368in}}%
\pgfusepath{}%
\end{pgfscope}%
\begin{pgfscope}%
\pgfpathrectangle{\pgfqpoint{0.549740in}{0.463273in}}{\pgfqpoint{9.320225in}{4.495057in}}%
\pgfusepath{clip}%
\pgfsetbuttcap%
\pgfsetroundjoin%
\pgfsetlinewidth{0.000000pt}%
\definecolor{currentstroke}{rgb}{0.000000,0.000000,0.000000}%
\pgfsetstrokecolor{currentstroke}%
\pgfsetdash{}{0pt}%
\pgfpathmoveto{\pgfqpoint{3.901839in}{4.345368in}}%
\pgfpathlineto{\pgfqpoint{4.088065in}{4.345368in}}%
\pgfpathlineto{\pgfqpoint{4.088065in}{4.427096in}}%
\pgfpathlineto{\pgfqpoint{3.901839in}{4.427096in}}%
\pgfpathlineto{\pgfqpoint{3.901839in}{4.345368in}}%
\pgfusepath{}%
\end{pgfscope}%
\begin{pgfscope}%
\pgfpathrectangle{\pgfqpoint{0.549740in}{0.463273in}}{\pgfqpoint{9.320225in}{4.495057in}}%
\pgfusepath{clip}%
\pgfsetbuttcap%
\pgfsetroundjoin%
\pgfsetlinewidth{0.000000pt}%
\definecolor{currentstroke}{rgb}{0.000000,0.000000,0.000000}%
\pgfsetstrokecolor{currentstroke}%
\pgfsetdash{}{0pt}%
\pgfpathmoveto{\pgfqpoint{4.088065in}{4.345368in}}%
\pgfpathlineto{\pgfqpoint{4.274292in}{4.345368in}}%
\pgfpathlineto{\pgfqpoint{4.274292in}{4.427096in}}%
\pgfpathlineto{\pgfqpoint{4.088065in}{4.427096in}}%
\pgfpathlineto{\pgfqpoint{4.088065in}{4.345368in}}%
\pgfusepath{}%
\end{pgfscope}%
\begin{pgfscope}%
\pgfpathrectangle{\pgfqpoint{0.549740in}{0.463273in}}{\pgfqpoint{9.320225in}{4.495057in}}%
\pgfusepath{clip}%
\pgfsetbuttcap%
\pgfsetroundjoin%
\pgfsetlinewidth{0.000000pt}%
\definecolor{currentstroke}{rgb}{0.000000,0.000000,0.000000}%
\pgfsetstrokecolor{currentstroke}%
\pgfsetdash{}{0pt}%
\pgfpathmoveto{\pgfqpoint{4.274292in}{4.345368in}}%
\pgfpathlineto{\pgfqpoint{4.460519in}{4.345368in}}%
\pgfpathlineto{\pgfqpoint{4.460519in}{4.427096in}}%
\pgfpathlineto{\pgfqpoint{4.274292in}{4.427096in}}%
\pgfpathlineto{\pgfqpoint{4.274292in}{4.345368in}}%
\pgfusepath{}%
\end{pgfscope}%
\begin{pgfscope}%
\pgfpathrectangle{\pgfqpoint{0.549740in}{0.463273in}}{\pgfqpoint{9.320225in}{4.495057in}}%
\pgfusepath{clip}%
\pgfsetbuttcap%
\pgfsetroundjoin%
\pgfsetlinewidth{0.000000pt}%
\definecolor{currentstroke}{rgb}{0.000000,0.000000,0.000000}%
\pgfsetstrokecolor{currentstroke}%
\pgfsetdash{}{0pt}%
\pgfpathmoveto{\pgfqpoint{4.460519in}{4.345368in}}%
\pgfpathlineto{\pgfqpoint{4.646745in}{4.345368in}}%
\pgfpathlineto{\pgfqpoint{4.646745in}{4.427096in}}%
\pgfpathlineto{\pgfqpoint{4.460519in}{4.427096in}}%
\pgfpathlineto{\pgfqpoint{4.460519in}{4.345368in}}%
\pgfusepath{}%
\end{pgfscope}%
\begin{pgfscope}%
\pgfpathrectangle{\pgfqpoint{0.549740in}{0.463273in}}{\pgfqpoint{9.320225in}{4.495057in}}%
\pgfusepath{clip}%
\pgfsetbuttcap%
\pgfsetroundjoin%
\pgfsetlinewidth{0.000000pt}%
\definecolor{currentstroke}{rgb}{0.000000,0.000000,0.000000}%
\pgfsetstrokecolor{currentstroke}%
\pgfsetdash{}{0pt}%
\pgfpathmoveto{\pgfqpoint{4.646745in}{4.345368in}}%
\pgfpathlineto{\pgfqpoint{4.832972in}{4.345368in}}%
\pgfpathlineto{\pgfqpoint{4.832972in}{4.427096in}}%
\pgfpathlineto{\pgfqpoint{4.646745in}{4.427096in}}%
\pgfpathlineto{\pgfqpoint{4.646745in}{4.345368in}}%
\pgfusepath{}%
\end{pgfscope}%
\begin{pgfscope}%
\pgfpathrectangle{\pgfqpoint{0.549740in}{0.463273in}}{\pgfqpoint{9.320225in}{4.495057in}}%
\pgfusepath{clip}%
\pgfsetbuttcap%
\pgfsetroundjoin%
\pgfsetlinewidth{0.000000pt}%
\definecolor{currentstroke}{rgb}{0.000000,0.000000,0.000000}%
\pgfsetstrokecolor{currentstroke}%
\pgfsetdash{}{0pt}%
\pgfpathmoveto{\pgfqpoint{4.832972in}{4.345368in}}%
\pgfpathlineto{\pgfqpoint{5.019198in}{4.345368in}}%
\pgfpathlineto{\pgfqpoint{5.019198in}{4.427096in}}%
\pgfpathlineto{\pgfqpoint{4.832972in}{4.427096in}}%
\pgfpathlineto{\pgfqpoint{4.832972in}{4.345368in}}%
\pgfusepath{}%
\end{pgfscope}%
\begin{pgfscope}%
\pgfpathrectangle{\pgfqpoint{0.549740in}{0.463273in}}{\pgfqpoint{9.320225in}{4.495057in}}%
\pgfusepath{clip}%
\pgfsetbuttcap%
\pgfsetroundjoin%
\pgfsetlinewidth{0.000000pt}%
\definecolor{currentstroke}{rgb}{0.000000,0.000000,0.000000}%
\pgfsetstrokecolor{currentstroke}%
\pgfsetdash{}{0pt}%
\pgfpathmoveto{\pgfqpoint{5.019198in}{4.345368in}}%
\pgfpathlineto{\pgfqpoint{5.205425in}{4.345368in}}%
\pgfpathlineto{\pgfqpoint{5.205425in}{4.427096in}}%
\pgfpathlineto{\pgfqpoint{5.019198in}{4.427096in}}%
\pgfpathlineto{\pgfqpoint{5.019198in}{4.345368in}}%
\pgfusepath{}%
\end{pgfscope}%
\begin{pgfscope}%
\pgfpathrectangle{\pgfqpoint{0.549740in}{0.463273in}}{\pgfqpoint{9.320225in}{4.495057in}}%
\pgfusepath{clip}%
\pgfsetbuttcap%
\pgfsetroundjoin%
\pgfsetlinewidth{0.000000pt}%
\definecolor{currentstroke}{rgb}{0.000000,0.000000,0.000000}%
\pgfsetstrokecolor{currentstroke}%
\pgfsetdash{}{0pt}%
\pgfpathmoveto{\pgfqpoint{5.205425in}{4.345368in}}%
\pgfpathlineto{\pgfqpoint{5.391651in}{4.345368in}}%
\pgfpathlineto{\pgfqpoint{5.391651in}{4.427096in}}%
\pgfpathlineto{\pgfqpoint{5.205425in}{4.427096in}}%
\pgfpathlineto{\pgfqpoint{5.205425in}{4.345368in}}%
\pgfusepath{}%
\end{pgfscope}%
\begin{pgfscope}%
\pgfpathrectangle{\pgfqpoint{0.549740in}{0.463273in}}{\pgfqpoint{9.320225in}{4.495057in}}%
\pgfusepath{clip}%
\pgfsetbuttcap%
\pgfsetroundjoin%
\pgfsetlinewidth{0.000000pt}%
\definecolor{currentstroke}{rgb}{0.000000,0.000000,0.000000}%
\pgfsetstrokecolor{currentstroke}%
\pgfsetdash{}{0pt}%
\pgfpathmoveto{\pgfqpoint{5.391651in}{4.345368in}}%
\pgfpathlineto{\pgfqpoint{5.577878in}{4.345368in}}%
\pgfpathlineto{\pgfqpoint{5.577878in}{4.427096in}}%
\pgfpathlineto{\pgfqpoint{5.391651in}{4.427096in}}%
\pgfpathlineto{\pgfqpoint{5.391651in}{4.345368in}}%
\pgfusepath{}%
\end{pgfscope}%
\begin{pgfscope}%
\pgfpathrectangle{\pgfqpoint{0.549740in}{0.463273in}}{\pgfqpoint{9.320225in}{4.495057in}}%
\pgfusepath{clip}%
\pgfsetbuttcap%
\pgfsetroundjoin%
\pgfsetlinewidth{0.000000pt}%
\definecolor{currentstroke}{rgb}{0.000000,0.000000,0.000000}%
\pgfsetstrokecolor{currentstroke}%
\pgfsetdash{}{0pt}%
\pgfpathmoveto{\pgfqpoint{5.577878in}{4.345368in}}%
\pgfpathlineto{\pgfqpoint{5.764104in}{4.345368in}}%
\pgfpathlineto{\pgfqpoint{5.764104in}{4.427096in}}%
\pgfpathlineto{\pgfqpoint{5.577878in}{4.427096in}}%
\pgfpathlineto{\pgfqpoint{5.577878in}{4.345368in}}%
\pgfusepath{}%
\end{pgfscope}%
\begin{pgfscope}%
\pgfpathrectangle{\pgfqpoint{0.549740in}{0.463273in}}{\pgfqpoint{9.320225in}{4.495057in}}%
\pgfusepath{clip}%
\pgfsetbuttcap%
\pgfsetroundjoin%
\pgfsetlinewidth{0.000000pt}%
\definecolor{currentstroke}{rgb}{0.000000,0.000000,0.000000}%
\pgfsetstrokecolor{currentstroke}%
\pgfsetdash{}{0pt}%
\pgfpathmoveto{\pgfqpoint{5.764104in}{4.345368in}}%
\pgfpathlineto{\pgfqpoint{5.950331in}{4.345368in}}%
\pgfpathlineto{\pgfqpoint{5.950331in}{4.427096in}}%
\pgfpathlineto{\pgfqpoint{5.764104in}{4.427096in}}%
\pgfpathlineto{\pgfqpoint{5.764104in}{4.345368in}}%
\pgfusepath{}%
\end{pgfscope}%
\begin{pgfscope}%
\pgfpathrectangle{\pgfqpoint{0.549740in}{0.463273in}}{\pgfqpoint{9.320225in}{4.495057in}}%
\pgfusepath{clip}%
\pgfsetbuttcap%
\pgfsetroundjoin%
\pgfsetlinewidth{0.000000pt}%
\definecolor{currentstroke}{rgb}{0.000000,0.000000,0.000000}%
\pgfsetstrokecolor{currentstroke}%
\pgfsetdash{}{0pt}%
\pgfpathmoveto{\pgfqpoint{5.950331in}{4.345368in}}%
\pgfpathlineto{\pgfqpoint{6.136557in}{4.345368in}}%
\pgfpathlineto{\pgfqpoint{6.136557in}{4.427096in}}%
\pgfpathlineto{\pgfqpoint{5.950331in}{4.427096in}}%
\pgfpathlineto{\pgfqpoint{5.950331in}{4.345368in}}%
\pgfusepath{}%
\end{pgfscope}%
\begin{pgfscope}%
\pgfpathrectangle{\pgfqpoint{0.549740in}{0.463273in}}{\pgfqpoint{9.320225in}{4.495057in}}%
\pgfusepath{clip}%
\pgfsetbuttcap%
\pgfsetroundjoin%
\pgfsetlinewidth{0.000000pt}%
\definecolor{currentstroke}{rgb}{0.000000,0.000000,0.000000}%
\pgfsetstrokecolor{currentstroke}%
\pgfsetdash{}{0pt}%
\pgfpathmoveto{\pgfqpoint{6.136557in}{4.345368in}}%
\pgfpathlineto{\pgfqpoint{6.322784in}{4.345368in}}%
\pgfpathlineto{\pgfqpoint{6.322784in}{4.427096in}}%
\pgfpathlineto{\pgfqpoint{6.136557in}{4.427096in}}%
\pgfpathlineto{\pgfqpoint{6.136557in}{4.345368in}}%
\pgfusepath{}%
\end{pgfscope}%
\begin{pgfscope}%
\pgfpathrectangle{\pgfqpoint{0.549740in}{0.463273in}}{\pgfqpoint{9.320225in}{4.495057in}}%
\pgfusepath{clip}%
\pgfsetbuttcap%
\pgfsetroundjoin%
\pgfsetlinewidth{0.000000pt}%
\definecolor{currentstroke}{rgb}{0.000000,0.000000,0.000000}%
\pgfsetstrokecolor{currentstroke}%
\pgfsetdash{}{0pt}%
\pgfpathmoveto{\pgfqpoint{6.322784in}{4.345368in}}%
\pgfpathlineto{\pgfqpoint{6.509011in}{4.345368in}}%
\pgfpathlineto{\pgfqpoint{6.509011in}{4.427096in}}%
\pgfpathlineto{\pgfqpoint{6.322784in}{4.427096in}}%
\pgfpathlineto{\pgfqpoint{6.322784in}{4.345368in}}%
\pgfusepath{}%
\end{pgfscope}%
\begin{pgfscope}%
\pgfpathrectangle{\pgfqpoint{0.549740in}{0.463273in}}{\pgfqpoint{9.320225in}{4.495057in}}%
\pgfusepath{clip}%
\pgfsetbuttcap%
\pgfsetroundjoin%
\pgfsetlinewidth{0.000000pt}%
\definecolor{currentstroke}{rgb}{0.000000,0.000000,0.000000}%
\pgfsetstrokecolor{currentstroke}%
\pgfsetdash{}{0pt}%
\pgfpathmoveto{\pgfqpoint{6.509011in}{4.345368in}}%
\pgfpathlineto{\pgfqpoint{6.695237in}{4.345368in}}%
\pgfpathlineto{\pgfqpoint{6.695237in}{4.427096in}}%
\pgfpathlineto{\pgfqpoint{6.509011in}{4.427096in}}%
\pgfpathlineto{\pgfqpoint{6.509011in}{4.345368in}}%
\pgfusepath{}%
\end{pgfscope}%
\begin{pgfscope}%
\pgfpathrectangle{\pgfqpoint{0.549740in}{0.463273in}}{\pgfqpoint{9.320225in}{4.495057in}}%
\pgfusepath{clip}%
\pgfsetbuttcap%
\pgfsetroundjoin%
\pgfsetlinewidth{0.000000pt}%
\definecolor{currentstroke}{rgb}{0.000000,0.000000,0.000000}%
\pgfsetstrokecolor{currentstroke}%
\pgfsetdash{}{0pt}%
\pgfpathmoveto{\pgfqpoint{6.695237in}{4.345368in}}%
\pgfpathlineto{\pgfqpoint{6.881464in}{4.345368in}}%
\pgfpathlineto{\pgfqpoint{6.881464in}{4.427096in}}%
\pgfpathlineto{\pgfqpoint{6.695237in}{4.427096in}}%
\pgfpathlineto{\pgfqpoint{6.695237in}{4.345368in}}%
\pgfusepath{}%
\end{pgfscope}%
\begin{pgfscope}%
\pgfpathrectangle{\pgfqpoint{0.549740in}{0.463273in}}{\pgfqpoint{9.320225in}{4.495057in}}%
\pgfusepath{clip}%
\pgfsetbuttcap%
\pgfsetroundjoin%
\pgfsetlinewidth{0.000000pt}%
\definecolor{currentstroke}{rgb}{0.000000,0.000000,0.000000}%
\pgfsetstrokecolor{currentstroke}%
\pgfsetdash{}{0pt}%
\pgfpathmoveto{\pgfqpoint{6.881464in}{4.345368in}}%
\pgfpathlineto{\pgfqpoint{7.067690in}{4.345368in}}%
\pgfpathlineto{\pgfqpoint{7.067690in}{4.427096in}}%
\pgfpathlineto{\pgfqpoint{6.881464in}{4.427096in}}%
\pgfpathlineto{\pgfqpoint{6.881464in}{4.345368in}}%
\pgfusepath{}%
\end{pgfscope}%
\begin{pgfscope}%
\pgfpathrectangle{\pgfqpoint{0.549740in}{0.463273in}}{\pgfqpoint{9.320225in}{4.495057in}}%
\pgfusepath{clip}%
\pgfsetbuttcap%
\pgfsetroundjoin%
\pgfsetlinewidth{0.000000pt}%
\definecolor{currentstroke}{rgb}{0.000000,0.000000,0.000000}%
\pgfsetstrokecolor{currentstroke}%
\pgfsetdash{}{0pt}%
\pgfpathmoveto{\pgfqpoint{7.067690in}{4.345368in}}%
\pgfpathlineto{\pgfqpoint{7.253917in}{4.345368in}}%
\pgfpathlineto{\pgfqpoint{7.253917in}{4.427096in}}%
\pgfpathlineto{\pgfqpoint{7.067690in}{4.427096in}}%
\pgfpathlineto{\pgfqpoint{7.067690in}{4.345368in}}%
\pgfusepath{}%
\end{pgfscope}%
\begin{pgfscope}%
\pgfpathrectangle{\pgfqpoint{0.549740in}{0.463273in}}{\pgfqpoint{9.320225in}{4.495057in}}%
\pgfusepath{clip}%
\pgfsetbuttcap%
\pgfsetroundjoin%
\pgfsetlinewidth{0.000000pt}%
\definecolor{currentstroke}{rgb}{0.000000,0.000000,0.000000}%
\pgfsetstrokecolor{currentstroke}%
\pgfsetdash{}{0pt}%
\pgfpathmoveto{\pgfqpoint{7.253917in}{4.345368in}}%
\pgfpathlineto{\pgfqpoint{7.440143in}{4.345368in}}%
\pgfpathlineto{\pgfqpoint{7.440143in}{4.427096in}}%
\pgfpathlineto{\pgfqpoint{7.253917in}{4.427096in}}%
\pgfpathlineto{\pgfqpoint{7.253917in}{4.345368in}}%
\pgfusepath{}%
\end{pgfscope}%
\begin{pgfscope}%
\pgfpathrectangle{\pgfqpoint{0.549740in}{0.463273in}}{\pgfqpoint{9.320225in}{4.495057in}}%
\pgfusepath{clip}%
\pgfsetbuttcap%
\pgfsetroundjoin%
\pgfsetlinewidth{0.000000pt}%
\definecolor{currentstroke}{rgb}{0.000000,0.000000,0.000000}%
\pgfsetstrokecolor{currentstroke}%
\pgfsetdash{}{0pt}%
\pgfpathmoveto{\pgfqpoint{7.440143in}{4.345368in}}%
\pgfpathlineto{\pgfqpoint{7.626370in}{4.345368in}}%
\pgfpathlineto{\pgfqpoint{7.626370in}{4.427096in}}%
\pgfpathlineto{\pgfqpoint{7.440143in}{4.427096in}}%
\pgfpathlineto{\pgfqpoint{7.440143in}{4.345368in}}%
\pgfusepath{}%
\end{pgfscope}%
\begin{pgfscope}%
\pgfpathrectangle{\pgfqpoint{0.549740in}{0.463273in}}{\pgfqpoint{9.320225in}{4.495057in}}%
\pgfusepath{clip}%
\pgfsetbuttcap%
\pgfsetroundjoin%
\pgfsetlinewidth{0.000000pt}%
\definecolor{currentstroke}{rgb}{0.000000,0.000000,0.000000}%
\pgfsetstrokecolor{currentstroke}%
\pgfsetdash{}{0pt}%
\pgfpathmoveto{\pgfqpoint{7.626370in}{4.345368in}}%
\pgfpathlineto{\pgfqpoint{7.812596in}{4.345368in}}%
\pgfpathlineto{\pgfqpoint{7.812596in}{4.427096in}}%
\pgfpathlineto{\pgfqpoint{7.626370in}{4.427096in}}%
\pgfpathlineto{\pgfqpoint{7.626370in}{4.345368in}}%
\pgfusepath{}%
\end{pgfscope}%
\begin{pgfscope}%
\pgfpathrectangle{\pgfqpoint{0.549740in}{0.463273in}}{\pgfqpoint{9.320225in}{4.495057in}}%
\pgfusepath{clip}%
\pgfsetbuttcap%
\pgfsetroundjoin%
\pgfsetlinewidth{0.000000pt}%
\definecolor{currentstroke}{rgb}{0.000000,0.000000,0.000000}%
\pgfsetstrokecolor{currentstroke}%
\pgfsetdash{}{0pt}%
\pgfpathmoveto{\pgfqpoint{7.812596in}{4.345368in}}%
\pgfpathlineto{\pgfqpoint{7.998823in}{4.345368in}}%
\pgfpathlineto{\pgfqpoint{7.998823in}{4.427096in}}%
\pgfpathlineto{\pgfqpoint{7.812596in}{4.427096in}}%
\pgfpathlineto{\pgfqpoint{7.812596in}{4.345368in}}%
\pgfusepath{}%
\end{pgfscope}%
\begin{pgfscope}%
\pgfpathrectangle{\pgfqpoint{0.549740in}{0.463273in}}{\pgfqpoint{9.320225in}{4.495057in}}%
\pgfusepath{clip}%
\pgfsetbuttcap%
\pgfsetroundjoin%
\pgfsetlinewidth{0.000000pt}%
\definecolor{currentstroke}{rgb}{0.000000,0.000000,0.000000}%
\pgfsetstrokecolor{currentstroke}%
\pgfsetdash{}{0pt}%
\pgfpathmoveto{\pgfqpoint{7.998823in}{4.345368in}}%
\pgfpathlineto{\pgfqpoint{8.185049in}{4.345368in}}%
\pgfpathlineto{\pgfqpoint{8.185049in}{4.427096in}}%
\pgfpathlineto{\pgfqpoint{7.998823in}{4.427096in}}%
\pgfpathlineto{\pgfqpoint{7.998823in}{4.345368in}}%
\pgfusepath{}%
\end{pgfscope}%
\begin{pgfscope}%
\pgfpathrectangle{\pgfqpoint{0.549740in}{0.463273in}}{\pgfqpoint{9.320225in}{4.495057in}}%
\pgfusepath{clip}%
\pgfsetbuttcap%
\pgfsetroundjoin%
\pgfsetlinewidth{0.000000pt}%
\definecolor{currentstroke}{rgb}{0.000000,0.000000,0.000000}%
\pgfsetstrokecolor{currentstroke}%
\pgfsetdash{}{0pt}%
\pgfpathmoveto{\pgfqpoint{8.185049in}{4.345368in}}%
\pgfpathlineto{\pgfqpoint{8.371276in}{4.345368in}}%
\pgfpathlineto{\pgfqpoint{8.371276in}{4.427096in}}%
\pgfpathlineto{\pgfqpoint{8.185049in}{4.427096in}}%
\pgfpathlineto{\pgfqpoint{8.185049in}{4.345368in}}%
\pgfusepath{}%
\end{pgfscope}%
\begin{pgfscope}%
\pgfpathrectangle{\pgfqpoint{0.549740in}{0.463273in}}{\pgfqpoint{9.320225in}{4.495057in}}%
\pgfusepath{clip}%
\pgfsetbuttcap%
\pgfsetroundjoin%
\pgfsetlinewidth{0.000000pt}%
\definecolor{currentstroke}{rgb}{0.000000,0.000000,0.000000}%
\pgfsetstrokecolor{currentstroke}%
\pgfsetdash{}{0pt}%
\pgfpathmoveto{\pgfqpoint{8.371276in}{4.345368in}}%
\pgfpathlineto{\pgfqpoint{8.557503in}{4.345368in}}%
\pgfpathlineto{\pgfqpoint{8.557503in}{4.427096in}}%
\pgfpathlineto{\pgfqpoint{8.371276in}{4.427096in}}%
\pgfpathlineto{\pgfqpoint{8.371276in}{4.345368in}}%
\pgfusepath{}%
\end{pgfscope}%
\begin{pgfscope}%
\pgfpathrectangle{\pgfqpoint{0.549740in}{0.463273in}}{\pgfqpoint{9.320225in}{4.495057in}}%
\pgfusepath{clip}%
\pgfsetbuttcap%
\pgfsetroundjoin%
\pgfsetlinewidth{0.000000pt}%
\definecolor{currentstroke}{rgb}{0.000000,0.000000,0.000000}%
\pgfsetstrokecolor{currentstroke}%
\pgfsetdash{}{0pt}%
\pgfpathmoveto{\pgfqpoint{8.557503in}{4.345368in}}%
\pgfpathlineto{\pgfqpoint{8.743729in}{4.345368in}}%
\pgfpathlineto{\pgfqpoint{8.743729in}{4.427096in}}%
\pgfpathlineto{\pgfqpoint{8.557503in}{4.427096in}}%
\pgfpathlineto{\pgfqpoint{8.557503in}{4.345368in}}%
\pgfusepath{}%
\end{pgfscope}%
\begin{pgfscope}%
\pgfpathrectangle{\pgfqpoint{0.549740in}{0.463273in}}{\pgfqpoint{9.320225in}{4.495057in}}%
\pgfusepath{clip}%
\pgfsetbuttcap%
\pgfsetroundjoin%
\pgfsetlinewidth{0.000000pt}%
\definecolor{currentstroke}{rgb}{0.000000,0.000000,0.000000}%
\pgfsetstrokecolor{currentstroke}%
\pgfsetdash{}{0pt}%
\pgfpathmoveto{\pgfqpoint{8.743729in}{4.345368in}}%
\pgfpathlineto{\pgfqpoint{8.929956in}{4.345368in}}%
\pgfpathlineto{\pgfqpoint{8.929956in}{4.427096in}}%
\pgfpathlineto{\pgfqpoint{8.743729in}{4.427096in}}%
\pgfpathlineto{\pgfqpoint{8.743729in}{4.345368in}}%
\pgfusepath{}%
\end{pgfscope}%
\begin{pgfscope}%
\pgfpathrectangle{\pgfqpoint{0.549740in}{0.463273in}}{\pgfqpoint{9.320225in}{4.495057in}}%
\pgfusepath{clip}%
\pgfsetbuttcap%
\pgfsetroundjoin%
\definecolor{currentfill}{rgb}{0.472869,0.711325,0.955316}%
\pgfsetfillcolor{currentfill}%
\pgfsetlinewidth{0.000000pt}%
\definecolor{currentstroke}{rgb}{0.000000,0.000000,0.000000}%
\pgfsetstrokecolor{currentstroke}%
\pgfsetdash{}{0pt}%
\pgfpathmoveto{\pgfqpoint{8.929956in}{4.345368in}}%
\pgfpathlineto{\pgfqpoint{9.116182in}{4.345368in}}%
\pgfpathlineto{\pgfqpoint{9.116182in}{4.427096in}}%
\pgfpathlineto{\pgfqpoint{8.929956in}{4.427096in}}%
\pgfpathlineto{\pgfqpoint{8.929956in}{4.345368in}}%
\pgfusepath{fill}%
\end{pgfscope}%
\begin{pgfscope}%
\pgfpathrectangle{\pgfqpoint{0.549740in}{0.463273in}}{\pgfqpoint{9.320225in}{4.495057in}}%
\pgfusepath{clip}%
\pgfsetbuttcap%
\pgfsetroundjoin%
\pgfsetlinewidth{0.000000pt}%
\definecolor{currentstroke}{rgb}{0.000000,0.000000,0.000000}%
\pgfsetstrokecolor{currentstroke}%
\pgfsetdash{}{0pt}%
\pgfpathmoveto{\pgfqpoint{9.116182in}{4.345368in}}%
\pgfpathlineto{\pgfqpoint{9.302409in}{4.345368in}}%
\pgfpathlineto{\pgfqpoint{9.302409in}{4.427096in}}%
\pgfpathlineto{\pgfqpoint{9.116182in}{4.427096in}}%
\pgfpathlineto{\pgfqpoint{9.116182in}{4.345368in}}%
\pgfusepath{}%
\end{pgfscope}%
\begin{pgfscope}%
\pgfpathrectangle{\pgfqpoint{0.549740in}{0.463273in}}{\pgfqpoint{9.320225in}{4.495057in}}%
\pgfusepath{clip}%
\pgfsetbuttcap%
\pgfsetroundjoin%
\pgfsetlinewidth{0.000000pt}%
\definecolor{currentstroke}{rgb}{0.000000,0.000000,0.000000}%
\pgfsetstrokecolor{currentstroke}%
\pgfsetdash{}{0pt}%
\pgfpathmoveto{\pgfqpoint{9.302409in}{4.345368in}}%
\pgfpathlineto{\pgfqpoint{9.488635in}{4.345368in}}%
\pgfpathlineto{\pgfqpoint{9.488635in}{4.427096in}}%
\pgfpathlineto{\pgfqpoint{9.302409in}{4.427096in}}%
\pgfpathlineto{\pgfqpoint{9.302409in}{4.345368in}}%
\pgfusepath{}%
\end{pgfscope}%
\begin{pgfscope}%
\pgfpathrectangle{\pgfqpoint{0.549740in}{0.463273in}}{\pgfqpoint{9.320225in}{4.495057in}}%
\pgfusepath{clip}%
\pgfsetbuttcap%
\pgfsetroundjoin%
\pgfsetlinewidth{0.000000pt}%
\definecolor{currentstroke}{rgb}{0.000000,0.000000,0.000000}%
\pgfsetstrokecolor{currentstroke}%
\pgfsetdash{}{0pt}%
\pgfpathmoveto{\pgfqpoint{9.488635in}{4.345368in}}%
\pgfpathlineto{\pgfqpoint{9.674862in}{4.345368in}}%
\pgfpathlineto{\pgfqpoint{9.674862in}{4.427096in}}%
\pgfpathlineto{\pgfqpoint{9.488635in}{4.427096in}}%
\pgfpathlineto{\pgfqpoint{9.488635in}{4.345368in}}%
\pgfusepath{}%
\end{pgfscope}%
\begin{pgfscope}%
\pgfpathrectangle{\pgfqpoint{0.549740in}{0.463273in}}{\pgfqpoint{9.320225in}{4.495057in}}%
\pgfusepath{clip}%
\pgfsetbuttcap%
\pgfsetroundjoin%
\pgfsetlinewidth{0.000000pt}%
\definecolor{currentstroke}{rgb}{0.000000,0.000000,0.000000}%
\pgfsetstrokecolor{currentstroke}%
\pgfsetdash{}{0pt}%
\pgfpathmoveto{\pgfqpoint{9.674862in}{4.345368in}}%
\pgfpathlineto{\pgfqpoint{9.861088in}{4.345368in}}%
\pgfpathlineto{\pgfqpoint{9.861088in}{4.427096in}}%
\pgfpathlineto{\pgfqpoint{9.674862in}{4.427096in}}%
\pgfpathlineto{\pgfqpoint{9.674862in}{4.345368in}}%
\pgfusepath{}%
\end{pgfscope}%
\begin{pgfscope}%
\pgfpathrectangle{\pgfqpoint{0.549740in}{0.463273in}}{\pgfqpoint{9.320225in}{4.495057in}}%
\pgfusepath{clip}%
\pgfsetbuttcap%
\pgfsetroundjoin%
\pgfsetlinewidth{0.000000pt}%
\definecolor{currentstroke}{rgb}{0.000000,0.000000,0.000000}%
\pgfsetstrokecolor{currentstroke}%
\pgfsetdash{}{0pt}%
\pgfpathmoveto{\pgfqpoint{0.549761in}{4.427096in}}%
\pgfpathlineto{\pgfqpoint{0.735988in}{4.427096in}}%
\pgfpathlineto{\pgfqpoint{0.735988in}{4.508824in}}%
\pgfpathlineto{\pgfqpoint{0.549761in}{4.508824in}}%
\pgfpathlineto{\pgfqpoint{0.549761in}{4.427096in}}%
\pgfusepath{}%
\end{pgfscope}%
\begin{pgfscope}%
\pgfpathrectangle{\pgfqpoint{0.549740in}{0.463273in}}{\pgfqpoint{9.320225in}{4.495057in}}%
\pgfusepath{clip}%
\pgfsetbuttcap%
\pgfsetroundjoin%
\pgfsetlinewidth{0.000000pt}%
\definecolor{currentstroke}{rgb}{0.000000,0.000000,0.000000}%
\pgfsetstrokecolor{currentstroke}%
\pgfsetdash{}{0pt}%
\pgfpathmoveto{\pgfqpoint{0.735988in}{4.427096in}}%
\pgfpathlineto{\pgfqpoint{0.922214in}{4.427096in}}%
\pgfpathlineto{\pgfqpoint{0.922214in}{4.508824in}}%
\pgfpathlineto{\pgfqpoint{0.735988in}{4.508824in}}%
\pgfpathlineto{\pgfqpoint{0.735988in}{4.427096in}}%
\pgfusepath{}%
\end{pgfscope}%
\begin{pgfscope}%
\pgfpathrectangle{\pgfqpoint{0.549740in}{0.463273in}}{\pgfqpoint{9.320225in}{4.495057in}}%
\pgfusepath{clip}%
\pgfsetbuttcap%
\pgfsetroundjoin%
\pgfsetlinewidth{0.000000pt}%
\definecolor{currentstroke}{rgb}{0.000000,0.000000,0.000000}%
\pgfsetstrokecolor{currentstroke}%
\pgfsetdash{}{0pt}%
\pgfpathmoveto{\pgfqpoint{0.922214in}{4.427096in}}%
\pgfpathlineto{\pgfqpoint{1.108441in}{4.427096in}}%
\pgfpathlineto{\pgfqpoint{1.108441in}{4.508824in}}%
\pgfpathlineto{\pgfqpoint{0.922214in}{4.508824in}}%
\pgfpathlineto{\pgfqpoint{0.922214in}{4.427096in}}%
\pgfusepath{}%
\end{pgfscope}%
\begin{pgfscope}%
\pgfpathrectangle{\pgfqpoint{0.549740in}{0.463273in}}{\pgfqpoint{9.320225in}{4.495057in}}%
\pgfusepath{clip}%
\pgfsetbuttcap%
\pgfsetroundjoin%
\pgfsetlinewidth{0.000000pt}%
\definecolor{currentstroke}{rgb}{0.000000,0.000000,0.000000}%
\pgfsetstrokecolor{currentstroke}%
\pgfsetdash{}{0pt}%
\pgfpathmoveto{\pgfqpoint{1.108441in}{4.427096in}}%
\pgfpathlineto{\pgfqpoint{1.294667in}{4.427096in}}%
\pgfpathlineto{\pgfqpoint{1.294667in}{4.508824in}}%
\pgfpathlineto{\pgfqpoint{1.108441in}{4.508824in}}%
\pgfpathlineto{\pgfqpoint{1.108441in}{4.427096in}}%
\pgfusepath{}%
\end{pgfscope}%
\begin{pgfscope}%
\pgfpathrectangle{\pgfqpoint{0.549740in}{0.463273in}}{\pgfqpoint{9.320225in}{4.495057in}}%
\pgfusepath{clip}%
\pgfsetbuttcap%
\pgfsetroundjoin%
\pgfsetlinewidth{0.000000pt}%
\definecolor{currentstroke}{rgb}{0.000000,0.000000,0.000000}%
\pgfsetstrokecolor{currentstroke}%
\pgfsetdash{}{0pt}%
\pgfpathmoveto{\pgfqpoint{1.294667in}{4.427096in}}%
\pgfpathlineto{\pgfqpoint{1.480894in}{4.427096in}}%
\pgfpathlineto{\pgfqpoint{1.480894in}{4.508824in}}%
\pgfpathlineto{\pgfqpoint{1.294667in}{4.508824in}}%
\pgfpathlineto{\pgfqpoint{1.294667in}{4.427096in}}%
\pgfusepath{}%
\end{pgfscope}%
\begin{pgfscope}%
\pgfpathrectangle{\pgfqpoint{0.549740in}{0.463273in}}{\pgfqpoint{9.320225in}{4.495057in}}%
\pgfusepath{clip}%
\pgfsetbuttcap%
\pgfsetroundjoin%
\pgfsetlinewidth{0.000000pt}%
\definecolor{currentstroke}{rgb}{0.000000,0.000000,0.000000}%
\pgfsetstrokecolor{currentstroke}%
\pgfsetdash{}{0pt}%
\pgfpathmoveto{\pgfqpoint{1.480894in}{4.427096in}}%
\pgfpathlineto{\pgfqpoint{1.667120in}{4.427096in}}%
\pgfpathlineto{\pgfqpoint{1.667120in}{4.508824in}}%
\pgfpathlineto{\pgfqpoint{1.480894in}{4.508824in}}%
\pgfpathlineto{\pgfqpoint{1.480894in}{4.427096in}}%
\pgfusepath{}%
\end{pgfscope}%
\begin{pgfscope}%
\pgfpathrectangle{\pgfqpoint{0.549740in}{0.463273in}}{\pgfqpoint{9.320225in}{4.495057in}}%
\pgfusepath{clip}%
\pgfsetbuttcap%
\pgfsetroundjoin%
\pgfsetlinewidth{0.000000pt}%
\definecolor{currentstroke}{rgb}{0.000000,0.000000,0.000000}%
\pgfsetstrokecolor{currentstroke}%
\pgfsetdash{}{0pt}%
\pgfpathmoveto{\pgfqpoint{1.667120in}{4.427096in}}%
\pgfpathlineto{\pgfqpoint{1.853347in}{4.427096in}}%
\pgfpathlineto{\pgfqpoint{1.853347in}{4.508824in}}%
\pgfpathlineto{\pgfqpoint{1.667120in}{4.508824in}}%
\pgfpathlineto{\pgfqpoint{1.667120in}{4.427096in}}%
\pgfusepath{}%
\end{pgfscope}%
\begin{pgfscope}%
\pgfpathrectangle{\pgfqpoint{0.549740in}{0.463273in}}{\pgfqpoint{9.320225in}{4.495057in}}%
\pgfusepath{clip}%
\pgfsetbuttcap%
\pgfsetroundjoin%
\pgfsetlinewidth{0.000000pt}%
\definecolor{currentstroke}{rgb}{0.000000,0.000000,0.000000}%
\pgfsetstrokecolor{currentstroke}%
\pgfsetdash{}{0pt}%
\pgfpathmoveto{\pgfqpoint{1.853347in}{4.427096in}}%
\pgfpathlineto{\pgfqpoint{2.039573in}{4.427096in}}%
\pgfpathlineto{\pgfqpoint{2.039573in}{4.508824in}}%
\pgfpathlineto{\pgfqpoint{1.853347in}{4.508824in}}%
\pgfpathlineto{\pgfqpoint{1.853347in}{4.427096in}}%
\pgfusepath{}%
\end{pgfscope}%
\begin{pgfscope}%
\pgfpathrectangle{\pgfqpoint{0.549740in}{0.463273in}}{\pgfqpoint{9.320225in}{4.495057in}}%
\pgfusepath{clip}%
\pgfsetbuttcap%
\pgfsetroundjoin%
\pgfsetlinewidth{0.000000pt}%
\definecolor{currentstroke}{rgb}{0.000000,0.000000,0.000000}%
\pgfsetstrokecolor{currentstroke}%
\pgfsetdash{}{0pt}%
\pgfpathmoveto{\pgfqpoint{2.039573in}{4.427096in}}%
\pgfpathlineto{\pgfqpoint{2.225800in}{4.427096in}}%
\pgfpathlineto{\pgfqpoint{2.225800in}{4.508824in}}%
\pgfpathlineto{\pgfqpoint{2.039573in}{4.508824in}}%
\pgfpathlineto{\pgfqpoint{2.039573in}{4.427096in}}%
\pgfusepath{}%
\end{pgfscope}%
\begin{pgfscope}%
\pgfpathrectangle{\pgfqpoint{0.549740in}{0.463273in}}{\pgfqpoint{9.320225in}{4.495057in}}%
\pgfusepath{clip}%
\pgfsetbuttcap%
\pgfsetroundjoin%
\pgfsetlinewidth{0.000000pt}%
\definecolor{currentstroke}{rgb}{0.000000,0.000000,0.000000}%
\pgfsetstrokecolor{currentstroke}%
\pgfsetdash{}{0pt}%
\pgfpathmoveto{\pgfqpoint{2.225800in}{4.427096in}}%
\pgfpathlineto{\pgfqpoint{2.412027in}{4.427096in}}%
\pgfpathlineto{\pgfqpoint{2.412027in}{4.508824in}}%
\pgfpathlineto{\pgfqpoint{2.225800in}{4.508824in}}%
\pgfpathlineto{\pgfqpoint{2.225800in}{4.427096in}}%
\pgfusepath{}%
\end{pgfscope}%
\begin{pgfscope}%
\pgfpathrectangle{\pgfqpoint{0.549740in}{0.463273in}}{\pgfqpoint{9.320225in}{4.495057in}}%
\pgfusepath{clip}%
\pgfsetbuttcap%
\pgfsetroundjoin%
\pgfsetlinewidth{0.000000pt}%
\definecolor{currentstroke}{rgb}{0.000000,0.000000,0.000000}%
\pgfsetstrokecolor{currentstroke}%
\pgfsetdash{}{0pt}%
\pgfpathmoveto{\pgfqpoint{2.412027in}{4.427096in}}%
\pgfpathlineto{\pgfqpoint{2.598253in}{4.427096in}}%
\pgfpathlineto{\pgfqpoint{2.598253in}{4.508824in}}%
\pgfpathlineto{\pgfqpoint{2.412027in}{4.508824in}}%
\pgfpathlineto{\pgfqpoint{2.412027in}{4.427096in}}%
\pgfusepath{}%
\end{pgfscope}%
\begin{pgfscope}%
\pgfpathrectangle{\pgfqpoint{0.549740in}{0.463273in}}{\pgfqpoint{9.320225in}{4.495057in}}%
\pgfusepath{clip}%
\pgfsetbuttcap%
\pgfsetroundjoin%
\pgfsetlinewidth{0.000000pt}%
\definecolor{currentstroke}{rgb}{0.000000,0.000000,0.000000}%
\pgfsetstrokecolor{currentstroke}%
\pgfsetdash{}{0pt}%
\pgfpathmoveto{\pgfqpoint{2.598253in}{4.427096in}}%
\pgfpathlineto{\pgfqpoint{2.784480in}{4.427096in}}%
\pgfpathlineto{\pgfqpoint{2.784480in}{4.508824in}}%
\pgfpathlineto{\pgfqpoint{2.598253in}{4.508824in}}%
\pgfpathlineto{\pgfqpoint{2.598253in}{4.427096in}}%
\pgfusepath{}%
\end{pgfscope}%
\begin{pgfscope}%
\pgfpathrectangle{\pgfqpoint{0.549740in}{0.463273in}}{\pgfqpoint{9.320225in}{4.495057in}}%
\pgfusepath{clip}%
\pgfsetbuttcap%
\pgfsetroundjoin%
\pgfsetlinewidth{0.000000pt}%
\definecolor{currentstroke}{rgb}{0.000000,0.000000,0.000000}%
\pgfsetstrokecolor{currentstroke}%
\pgfsetdash{}{0pt}%
\pgfpathmoveto{\pgfqpoint{2.784480in}{4.427096in}}%
\pgfpathlineto{\pgfqpoint{2.970706in}{4.427096in}}%
\pgfpathlineto{\pgfqpoint{2.970706in}{4.508824in}}%
\pgfpathlineto{\pgfqpoint{2.784480in}{4.508824in}}%
\pgfpathlineto{\pgfqpoint{2.784480in}{4.427096in}}%
\pgfusepath{}%
\end{pgfscope}%
\begin{pgfscope}%
\pgfpathrectangle{\pgfqpoint{0.549740in}{0.463273in}}{\pgfqpoint{9.320225in}{4.495057in}}%
\pgfusepath{clip}%
\pgfsetbuttcap%
\pgfsetroundjoin%
\pgfsetlinewidth{0.000000pt}%
\definecolor{currentstroke}{rgb}{0.000000,0.000000,0.000000}%
\pgfsetstrokecolor{currentstroke}%
\pgfsetdash{}{0pt}%
\pgfpathmoveto{\pgfqpoint{2.970706in}{4.427096in}}%
\pgfpathlineto{\pgfqpoint{3.156933in}{4.427096in}}%
\pgfpathlineto{\pgfqpoint{3.156933in}{4.508824in}}%
\pgfpathlineto{\pgfqpoint{2.970706in}{4.508824in}}%
\pgfpathlineto{\pgfqpoint{2.970706in}{4.427096in}}%
\pgfusepath{}%
\end{pgfscope}%
\begin{pgfscope}%
\pgfpathrectangle{\pgfqpoint{0.549740in}{0.463273in}}{\pgfqpoint{9.320225in}{4.495057in}}%
\pgfusepath{clip}%
\pgfsetbuttcap%
\pgfsetroundjoin%
\pgfsetlinewidth{0.000000pt}%
\definecolor{currentstroke}{rgb}{0.000000,0.000000,0.000000}%
\pgfsetstrokecolor{currentstroke}%
\pgfsetdash{}{0pt}%
\pgfpathmoveto{\pgfqpoint{3.156933in}{4.427096in}}%
\pgfpathlineto{\pgfqpoint{3.343159in}{4.427096in}}%
\pgfpathlineto{\pgfqpoint{3.343159in}{4.508824in}}%
\pgfpathlineto{\pgfqpoint{3.156933in}{4.508824in}}%
\pgfpathlineto{\pgfqpoint{3.156933in}{4.427096in}}%
\pgfusepath{}%
\end{pgfscope}%
\begin{pgfscope}%
\pgfpathrectangle{\pgfqpoint{0.549740in}{0.463273in}}{\pgfqpoint{9.320225in}{4.495057in}}%
\pgfusepath{clip}%
\pgfsetbuttcap%
\pgfsetroundjoin%
\pgfsetlinewidth{0.000000pt}%
\definecolor{currentstroke}{rgb}{0.000000,0.000000,0.000000}%
\pgfsetstrokecolor{currentstroke}%
\pgfsetdash{}{0pt}%
\pgfpathmoveto{\pgfqpoint{3.343159in}{4.427096in}}%
\pgfpathlineto{\pgfqpoint{3.529386in}{4.427096in}}%
\pgfpathlineto{\pgfqpoint{3.529386in}{4.508824in}}%
\pgfpathlineto{\pgfqpoint{3.343159in}{4.508824in}}%
\pgfpathlineto{\pgfqpoint{3.343159in}{4.427096in}}%
\pgfusepath{}%
\end{pgfscope}%
\begin{pgfscope}%
\pgfpathrectangle{\pgfqpoint{0.549740in}{0.463273in}}{\pgfqpoint{9.320225in}{4.495057in}}%
\pgfusepath{clip}%
\pgfsetbuttcap%
\pgfsetroundjoin%
\pgfsetlinewidth{0.000000pt}%
\definecolor{currentstroke}{rgb}{0.000000,0.000000,0.000000}%
\pgfsetstrokecolor{currentstroke}%
\pgfsetdash{}{0pt}%
\pgfpathmoveto{\pgfqpoint{3.529386in}{4.427096in}}%
\pgfpathlineto{\pgfqpoint{3.715612in}{4.427096in}}%
\pgfpathlineto{\pgfqpoint{3.715612in}{4.508824in}}%
\pgfpathlineto{\pgfqpoint{3.529386in}{4.508824in}}%
\pgfpathlineto{\pgfqpoint{3.529386in}{4.427096in}}%
\pgfusepath{}%
\end{pgfscope}%
\begin{pgfscope}%
\pgfpathrectangle{\pgfqpoint{0.549740in}{0.463273in}}{\pgfqpoint{9.320225in}{4.495057in}}%
\pgfusepath{clip}%
\pgfsetbuttcap%
\pgfsetroundjoin%
\pgfsetlinewidth{0.000000pt}%
\definecolor{currentstroke}{rgb}{0.000000,0.000000,0.000000}%
\pgfsetstrokecolor{currentstroke}%
\pgfsetdash{}{0pt}%
\pgfpathmoveto{\pgfqpoint{3.715612in}{4.427096in}}%
\pgfpathlineto{\pgfqpoint{3.901839in}{4.427096in}}%
\pgfpathlineto{\pgfqpoint{3.901839in}{4.508824in}}%
\pgfpathlineto{\pgfqpoint{3.715612in}{4.508824in}}%
\pgfpathlineto{\pgfqpoint{3.715612in}{4.427096in}}%
\pgfusepath{}%
\end{pgfscope}%
\begin{pgfscope}%
\pgfpathrectangle{\pgfqpoint{0.549740in}{0.463273in}}{\pgfqpoint{9.320225in}{4.495057in}}%
\pgfusepath{clip}%
\pgfsetbuttcap%
\pgfsetroundjoin%
\pgfsetlinewidth{0.000000pt}%
\definecolor{currentstroke}{rgb}{0.000000,0.000000,0.000000}%
\pgfsetstrokecolor{currentstroke}%
\pgfsetdash{}{0pt}%
\pgfpathmoveto{\pgfqpoint{3.901839in}{4.427096in}}%
\pgfpathlineto{\pgfqpoint{4.088065in}{4.427096in}}%
\pgfpathlineto{\pgfqpoint{4.088065in}{4.508824in}}%
\pgfpathlineto{\pgfqpoint{3.901839in}{4.508824in}}%
\pgfpathlineto{\pgfqpoint{3.901839in}{4.427096in}}%
\pgfusepath{}%
\end{pgfscope}%
\begin{pgfscope}%
\pgfpathrectangle{\pgfqpoint{0.549740in}{0.463273in}}{\pgfqpoint{9.320225in}{4.495057in}}%
\pgfusepath{clip}%
\pgfsetbuttcap%
\pgfsetroundjoin%
\pgfsetlinewidth{0.000000pt}%
\definecolor{currentstroke}{rgb}{0.000000,0.000000,0.000000}%
\pgfsetstrokecolor{currentstroke}%
\pgfsetdash{}{0pt}%
\pgfpathmoveto{\pgfqpoint{4.088065in}{4.427096in}}%
\pgfpathlineto{\pgfqpoint{4.274292in}{4.427096in}}%
\pgfpathlineto{\pgfqpoint{4.274292in}{4.508824in}}%
\pgfpathlineto{\pgfqpoint{4.088065in}{4.508824in}}%
\pgfpathlineto{\pgfqpoint{4.088065in}{4.427096in}}%
\pgfusepath{}%
\end{pgfscope}%
\begin{pgfscope}%
\pgfpathrectangle{\pgfqpoint{0.549740in}{0.463273in}}{\pgfqpoint{9.320225in}{4.495057in}}%
\pgfusepath{clip}%
\pgfsetbuttcap%
\pgfsetroundjoin%
\pgfsetlinewidth{0.000000pt}%
\definecolor{currentstroke}{rgb}{0.000000,0.000000,0.000000}%
\pgfsetstrokecolor{currentstroke}%
\pgfsetdash{}{0pt}%
\pgfpathmoveto{\pgfqpoint{4.274292in}{4.427096in}}%
\pgfpathlineto{\pgfqpoint{4.460519in}{4.427096in}}%
\pgfpathlineto{\pgfqpoint{4.460519in}{4.508824in}}%
\pgfpathlineto{\pgfqpoint{4.274292in}{4.508824in}}%
\pgfpathlineto{\pgfqpoint{4.274292in}{4.427096in}}%
\pgfusepath{}%
\end{pgfscope}%
\begin{pgfscope}%
\pgfpathrectangle{\pgfqpoint{0.549740in}{0.463273in}}{\pgfqpoint{9.320225in}{4.495057in}}%
\pgfusepath{clip}%
\pgfsetbuttcap%
\pgfsetroundjoin%
\pgfsetlinewidth{0.000000pt}%
\definecolor{currentstroke}{rgb}{0.000000,0.000000,0.000000}%
\pgfsetstrokecolor{currentstroke}%
\pgfsetdash{}{0pt}%
\pgfpathmoveto{\pgfqpoint{4.460519in}{4.427096in}}%
\pgfpathlineto{\pgfqpoint{4.646745in}{4.427096in}}%
\pgfpathlineto{\pgfqpoint{4.646745in}{4.508824in}}%
\pgfpathlineto{\pgfqpoint{4.460519in}{4.508824in}}%
\pgfpathlineto{\pgfqpoint{4.460519in}{4.427096in}}%
\pgfusepath{}%
\end{pgfscope}%
\begin{pgfscope}%
\pgfpathrectangle{\pgfqpoint{0.549740in}{0.463273in}}{\pgfqpoint{9.320225in}{4.495057in}}%
\pgfusepath{clip}%
\pgfsetbuttcap%
\pgfsetroundjoin%
\pgfsetlinewidth{0.000000pt}%
\definecolor{currentstroke}{rgb}{0.000000,0.000000,0.000000}%
\pgfsetstrokecolor{currentstroke}%
\pgfsetdash{}{0pt}%
\pgfpathmoveto{\pgfqpoint{4.646745in}{4.427096in}}%
\pgfpathlineto{\pgfqpoint{4.832972in}{4.427096in}}%
\pgfpathlineto{\pgfqpoint{4.832972in}{4.508824in}}%
\pgfpathlineto{\pgfqpoint{4.646745in}{4.508824in}}%
\pgfpathlineto{\pgfqpoint{4.646745in}{4.427096in}}%
\pgfusepath{}%
\end{pgfscope}%
\begin{pgfscope}%
\pgfpathrectangle{\pgfqpoint{0.549740in}{0.463273in}}{\pgfqpoint{9.320225in}{4.495057in}}%
\pgfusepath{clip}%
\pgfsetbuttcap%
\pgfsetroundjoin%
\pgfsetlinewidth{0.000000pt}%
\definecolor{currentstroke}{rgb}{0.000000,0.000000,0.000000}%
\pgfsetstrokecolor{currentstroke}%
\pgfsetdash{}{0pt}%
\pgfpathmoveto{\pgfqpoint{4.832972in}{4.427096in}}%
\pgfpathlineto{\pgfqpoint{5.019198in}{4.427096in}}%
\pgfpathlineto{\pgfqpoint{5.019198in}{4.508824in}}%
\pgfpathlineto{\pgfqpoint{4.832972in}{4.508824in}}%
\pgfpathlineto{\pgfqpoint{4.832972in}{4.427096in}}%
\pgfusepath{}%
\end{pgfscope}%
\begin{pgfscope}%
\pgfpathrectangle{\pgfqpoint{0.549740in}{0.463273in}}{\pgfqpoint{9.320225in}{4.495057in}}%
\pgfusepath{clip}%
\pgfsetbuttcap%
\pgfsetroundjoin%
\pgfsetlinewidth{0.000000pt}%
\definecolor{currentstroke}{rgb}{0.000000,0.000000,0.000000}%
\pgfsetstrokecolor{currentstroke}%
\pgfsetdash{}{0pt}%
\pgfpathmoveto{\pgfqpoint{5.019198in}{4.427096in}}%
\pgfpathlineto{\pgfqpoint{5.205425in}{4.427096in}}%
\pgfpathlineto{\pgfqpoint{5.205425in}{4.508824in}}%
\pgfpathlineto{\pgfqpoint{5.019198in}{4.508824in}}%
\pgfpathlineto{\pgfqpoint{5.019198in}{4.427096in}}%
\pgfusepath{}%
\end{pgfscope}%
\begin{pgfscope}%
\pgfpathrectangle{\pgfqpoint{0.549740in}{0.463273in}}{\pgfqpoint{9.320225in}{4.495057in}}%
\pgfusepath{clip}%
\pgfsetbuttcap%
\pgfsetroundjoin%
\pgfsetlinewidth{0.000000pt}%
\definecolor{currentstroke}{rgb}{0.000000,0.000000,0.000000}%
\pgfsetstrokecolor{currentstroke}%
\pgfsetdash{}{0pt}%
\pgfpathmoveto{\pgfqpoint{5.205425in}{4.427096in}}%
\pgfpathlineto{\pgfqpoint{5.391651in}{4.427096in}}%
\pgfpathlineto{\pgfqpoint{5.391651in}{4.508824in}}%
\pgfpathlineto{\pgfqpoint{5.205425in}{4.508824in}}%
\pgfpathlineto{\pgfqpoint{5.205425in}{4.427096in}}%
\pgfusepath{}%
\end{pgfscope}%
\begin{pgfscope}%
\pgfpathrectangle{\pgfqpoint{0.549740in}{0.463273in}}{\pgfqpoint{9.320225in}{4.495057in}}%
\pgfusepath{clip}%
\pgfsetbuttcap%
\pgfsetroundjoin%
\pgfsetlinewidth{0.000000pt}%
\definecolor{currentstroke}{rgb}{0.000000,0.000000,0.000000}%
\pgfsetstrokecolor{currentstroke}%
\pgfsetdash{}{0pt}%
\pgfpathmoveto{\pgfqpoint{5.391651in}{4.427096in}}%
\pgfpathlineto{\pgfqpoint{5.577878in}{4.427096in}}%
\pgfpathlineto{\pgfqpoint{5.577878in}{4.508824in}}%
\pgfpathlineto{\pgfqpoint{5.391651in}{4.508824in}}%
\pgfpathlineto{\pgfqpoint{5.391651in}{4.427096in}}%
\pgfusepath{}%
\end{pgfscope}%
\begin{pgfscope}%
\pgfpathrectangle{\pgfqpoint{0.549740in}{0.463273in}}{\pgfqpoint{9.320225in}{4.495057in}}%
\pgfusepath{clip}%
\pgfsetbuttcap%
\pgfsetroundjoin%
\pgfsetlinewidth{0.000000pt}%
\definecolor{currentstroke}{rgb}{0.000000,0.000000,0.000000}%
\pgfsetstrokecolor{currentstroke}%
\pgfsetdash{}{0pt}%
\pgfpathmoveto{\pgfqpoint{5.577878in}{4.427096in}}%
\pgfpathlineto{\pgfqpoint{5.764104in}{4.427096in}}%
\pgfpathlineto{\pgfqpoint{5.764104in}{4.508824in}}%
\pgfpathlineto{\pgfqpoint{5.577878in}{4.508824in}}%
\pgfpathlineto{\pgfqpoint{5.577878in}{4.427096in}}%
\pgfusepath{}%
\end{pgfscope}%
\begin{pgfscope}%
\pgfpathrectangle{\pgfqpoint{0.549740in}{0.463273in}}{\pgfqpoint{9.320225in}{4.495057in}}%
\pgfusepath{clip}%
\pgfsetbuttcap%
\pgfsetroundjoin%
\pgfsetlinewidth{0.000000pt}%
\definecolor{currentstroke}{rgb}{0.000000,0.000000,0.000000}%
\pgfsetstrokecolor{currentstroke}%
\pgfsetdash{}{0pt}%
\pgfpathmoveto{\pgfqpoint{5.764104in}{4.427096in}}%
\pgfpathlineto{\pgfqpoint{5.950331in}{4.427096in}}%
\pgfpathlineto{\pgfqpoint{5.950331in}{4.508824in}}%
\pgfpathlineto{\pgfqpoint{5.764104in}{4.508824in}}%
\pgfpathlineto{\pgfqpoint{5.764104in}{4.427096in}}%
\pgfusepath{}%
\end{pgfscope}%
\begin{pgfscope}%
\pgfpathrectangle{\pgfqpoint{0.549740in}{0.463273in}}{\pgfqpoint{9.320225in}{4.495057in}}%
\pgfusepath{clip}%
\pgfsetbuttcap%
\pgfsetroundjoin%
\pgfsetlinewidth{0.000000pt}%
\definecolor{currentstroke}{rgb}{0.000000,0.000000,0.000000}%
\pgfsetstrokecolor{currentstroke}%
\pgfsetdash{}{0pt}%
\pgfpathmoveto{\pgfqpoint{5.950331in}{4.427096in}}%
\pgfpathlineto{\pgfqpoint{6.136557in}{4.427096in}}%
\pgfpathlineto{\pgfqpoint{6.136557in}{4.508824in}}%
\pgfpathlineto{\pgfqpoint{5.950331in}{4.508824in}}%
\pgfpathlineto{\pgfqpoint{5.950331in}{4.427096in}}%
\pgfusepath{}%
\end{pgfscope}%
\begin{pgfscope}%
\pgfpathrectangle{\pgfqpoint{0.549740in}{0.463273in}}{\pgfqpoint{9.320225in}{4.495057in}}%
\pgfusepath{clip}%
\pgfsetbuttcap%
\pgfsetroundjoin%
\pgfsetlinewidth{0.000000pt}%
\definecolor{currentstroke}{rgb}{0.000000,0.000000,0.000000}%
\pgfsetstrokecolor{currentstroke}%
\pgfsetdash{}{0pt}%
\pgfpathmoveto{\pgfqpoint{6.136557in}{4.427096in}}%
\pgfpathlineto{\pgfqpoint{6.322784in}{4.427096in}}%
\pgfpathlineto{\pgfqpoint{6.322784in}{4.508824in}}%
\pgfpathlineto{\pgfqpoint{6.136557in}{4.508824in}}%
\pgfpathlineto{\pgfqpoint{6.136557in}{4.427096in}}%
\pgfusepath{}%
\end{pgfscope}%
\begin{pgfscope}%
\pgfpathrectangle{\pgfqpoint{0.549740in}{0.463273in}}{\pgfqpoint{9.320225in}{4.495057in}}%
\pgfusepath{clip}%
\pgfsetbuttcap%
\pgfsetroundjoin%
\pgfsetlinewidth{0.000000pt}%
\definecolor{currentstroke}{rgb}{0.000000,0.000000,0.000000}%
\pgfsetstrokecolor{currentstroke}%
\pgfsetdash{}{0pt}%
\pgfpathmoveto{\pgfqpoint{6.322784in}{4.427096in}}%
\pgfpathlineto{\pgfqpoint{6.509011in}{4.427096in}}%
\pgfpathlineto{\pgfqpoint{6.509011in}{4.508824in}}%
\pgfpathlineto{\pgfqpoint{6.322784in}{4.508824in}}%
\pgfpathlineto{\pgfqpoint{6.322784in}{4.427096in}}%
\pgfusepath{}%
\end{pgfscope}%
\begin{pgfscope}%
\pgfpathrectangle{\pgfqpoint{0.549740in}{0.463273in}}{\pgfqpoint{9.320225in}{4.495057in}}%
\pgfusepath{clip}%
\pgfsetbuttcap%
\pgfsetroundjoin%
\pgfsetlinewidth{0.000000pt}%
\definecolor{currentstroke}{rgb}{0.000000,0.000000,0.000000}%
\pgfsetstrokecolor{currentstroke}%
\pgfsetdash{}{0pt}%
\pgfpathmoveto{\pgfqpoint{6.509011in}{4.427096in}}%
\pgfpathlineto{\pgfqpoint{6.695237in}{4.427096in}}%
\pgfpathlineto{\pgfqpoint{6.695237in}{4.508824in}}%
\pgfpathlineto{\pgfqpoint{6.509011in}{4.508824in}}%
\pgfpathlineto{\pgfqpoint{6.509011in}{4.427096in}}%
\pgfusepath{}%
\end{pgfscope}%
\begin{pgfscope}%
\pgfpathrectangle{\pgfqpoint{0.549740in}{0.463273in}}{\pgfqpoint{9.320225in}{4.495057in}}%
\pgfusepath{clip}%
\pgfsetbuttcap%
\pgfsetroundjoin%
\pgfsetlinewidth{0.000000pt}%
\definecolor{currentstroke}{rgb}{0.000000,0.000000,0.000000}%
\pgfsetstrokecolor{currentstroke}%
\pgfsetdash{}{0pt}%
\pgfpathmoveto{\pgfqpoint{6.695237in}{4.427096in}}%
\pgfpathlineto{\pgfqpoint{6.881464in}{4.427096in}}%
\pgfpathlineto{\pgfqpoint{6.881464in}{4.508824in}}%
\pgfpathlineto{\pgfqpoint{6.695237in}{4.508824in}}%
\pgfpathlineto{\pgfqpoint{6.695237in}{4.427096in}}%
\pgfusepath{}%
\end{pgfscope}%
\begin{pgfscope}%
\pgfpathrectangle{\pgfqpoint{0.549740in}{0.463273in}}{\pgfqpoint{9.320225in}{4.495057in}}%
\pgfusepath{clip}%
\pgfsetbuttcap%
\pgfsetroundjoin%
\pgfsetlinewidth{0.000000pt}%
\definecolor{currentstroke}{rgb}{0.000000,0.000000,0.000000}%
\pgfsetstrokecolor{currentstroke}%
\pgfsetdash{}{0pt}%
\pgfpathmoveto{\pgfqpoint{6.881464in}{4.427096in}}%
\pgfpathlineto{\pgfqpoint{7.067690in}{4.427096in}}%
\pgfpathlineto{\pgfqpoint{7.067690in}{4.508824in}}%
\pgfpathlineto{\pgfqpoint{6.881464in}{4.508824in}}%
\pgfpathlineto{\pgfqpoint{6.881464in}{4.427096in}}%
\pgfusepath{}%
\end{pgfscope}%
\begin{pgfscope}%
\pgfpathrectangle{\pgfqpoint{0.549740in}{0.463273in}}{\pgfqpoint{9.320225in}{4.495057in}}%
\pgfusepath{clip}%
\pgfsetbuttcap%
\pgfsetroundjoin%
\pgfsetlinewidth{0.000000pt}%
\definecolor{currentstroke}{rgb}{0.000000,0.000000,0.000000}%
\pgfsetstrokecolor{currentstroke}%
\pgfsetdash{}{0pt}%
\pgfpathmoveto{\pgfqpoint{7.067690in}{4.427096in}}%
\pgfpathlineto{\pgfqpoint{7.253917in}{4.427096in}}%
\pgfpathlineto{\pgfqpoint{7.253917in}{4.508824in}}%
\pgfpathlineto{\pgfqpoint{7.067690in}{4.508824in}}%
\pgfpathlineto{\pgfqpoint{7.067690in}{4.427096in}}%
\pgfusepath{}%
\end{pgfscope}%
\begin{pgfscope}%
\pgfpathrectangle{\pgfqpoint{0.549740in}{0.463273in}}{\pgfqpoint{9.320225in}{4.495057in}}%
\pgfusepath{clip}%
\pgfsetbuttcap%
\pgfsetroundjoin%
\pgfsetlinewidth{0.000000pt}%
\definecolor{currentstroke}{rgb}{0.000000,0.000000,0.000000}%
\pgfsetstrokecolor{currentstroke}%
\pgfsetdash{}{0pt}%
\pgfpathmoveto{\pgfqpoint{7.253917in}{4.427096in}}%
\pgfpathlineto{\pgfqpoint{7.440143in}{4.427096in}}%
\pgfpathlineto{\pgfqpoint{7.440143in}{4.508824in}}%
\pgfpathlineto{\pgfqpoint{7.253917in}{4.508824in}}%
\pgfpathlineto{\pgfqpoint{7.253917in}{4.427096in}}%
\pgfusepath{}%
\end{pgfscope}%
\begin{pgfscope}%
\pgfpathrectangle{\pgfqpoint{0.549740in}{0.463273in}}{\pgfqpoint{9.320225in}{4.495057in}}%
\pgfusepath{clip}%
\pgfsetbuttcap%
\pgfsetroundjoin%
\pgfsetlinewidth{0.000000pt}%
\definecolor{currentstroke}{rgb}{0.000000,0.000000,0.000000}%
\pgfsetstrokecolor{currentstroke}%
\pgfsetdash{}{0pt}%
\pgfpathmoveto{\pgfqpoint{7.440143in}{4.427096in}}%
\pgfpathlineto{\pgfqpoint{7.626370in}{4.427096in}}%
\pgfpathlineto{\pgfqpoint{7.626370in}{4.508824in}}%
\pgfpathlineto{\pgfqpoint{7.440143in}{4.508824in}}%
\pgfpathlineto{\pgfqpoint{7.440143in}{4.427096in}}%
\pgfusepath{}%
\end{pgfscope}%
\begin{pgfscope}%
\pgfpathrectangle{\pgfqpoint{0.549740in}{0.463273in}}{\pgfqpoint{9.320225in}{4.495057in}}%
\pgfusepath{clip}%
\pgfsetbuttcap%
\pgfsetroundjoin%
\pgfsetlinewidth{0.000000pt}%
\definecolor{currentstroke}{rgb}{0.000000,0.000000,0.000000}%
\pgfsetstrokecolor{currentstroke}%
\pgfsetdash{}{0pt}%
\pgfpathmoveto{\pgfqpoint{7.626370in}{4.427096in}}%
\pgfpathlineto{\pgfqpoint{7.812596in}{4.427096in}}%
\pgfpathlineto{\pgfqpoint{7.812596in}{4.508824in}}%
\pgfpathlineto{\pgfqpoint{7.626370in}{4.508824in}}%
\pgfpathlineto{\pgfqpoint{7.626370in}{4.427096in}}%
\pgfusepath{}%
\end{pgfscope}%
\begin{pgfscope}%
\pgfpathrectangle{\pgfqpoint{0.549740in}{0.463273in}}{\pgfqpoint{9.320225in}{4.495057in}}%
\pgfusepath{clip}%
\pgfsetbuttcap%
\pgfsetroundjoin%
\pgfsetlinewidth{0.000000pt}%
\definecolor{currentstroke}{rgb}{0.000000,0.000000,0.000000}%
\pgfsetstrokecolor{currentstroke}%
\pgfsetdash{}{0pt}%
\pgfpathmoveto{\pgfqpoint{7.812596in}{4.427096in}}%
\pgfpathlineto{\pgfqpoint{7.998823in}{4.427096in}}%
\pgfpathlineto{\pgfqpoint{7.998823in}{4.508824in}}%
\pgfpathlineto{\pgfqpoint{7.812596in}{4.508824in}}%
\pgfpathlineto{\pgfqpoint{7.812596in}{4.427096in}}%
\pgfusepath{}%
\end{pgfscope}%
\begin{pgfscope}%
\pgfpathrectangle{\pgfqpoint{0.549740in}{0.463273in}}{\pgfqpoint{9.320225in}{4.495057in}}%
\pgfusepath{clip}%
\pgfsetbuttcap%
\pgfsetroundjoin%
\pgfsetlinewidth{0.000000pt}%
\definecolor{currentstroke}{rgb}{0.000000,0.000000,0.000000}%
\pgfsetstrokecolor{currentstroke}%
\pgfsetdash{}{0pt}%
\pgfpathmoveto{\pgfqpoint{7.998823in}{4.427096in}}%
\pgfpathlineto{\pgfqpoint{8.185049in}{4.427096in}}%
\pgfpathlineto{\pgfqpoint{8.185049in}{4.508824in}}%
\pgfpathlineto{\pgfqpoint{7.998823in}{4.508824in}}%
\pgfpathlineto{\pgfqpoint{7.998823in}{4.427096in}}%
\pgfusepath{}%
\end{pgfscope}%
\begin{pgfscope}%
\pgfpathrectangle{\pgfqpoint{0.549740in}{0.463273in}}{\pgfqpoint{9.320225in}{4.495057in}}%
\pgfusepath{clip}%
\pgfsetbuttcap%
\pgfsetroundjoin%
\pgfsetlinewidth{0.000000pt}%
\definecolor{currentstroke}{rgb}{0.000000,0.000000,0.000000}%
\pgfsetstrokecolor{currentstroke}%
\pgfsetdash{}{0pt}%
\pgfpathmoveto{\pgfqpoint{8.185049in}{4.427096in}}%
\pgfpathlineto{\pgfqpoint{8.371276in}{4.427096in}}%
\pgfpathlineto{\pgfqpoint{8.371276in}{4.508824in}}%
\pgfpathlineto{\pgfqpoint{8.185049in}{4.508824in}}%
\pgfpathlineto{\pgfqpoint{8.185049in}{4.427096in}}%
\pgfusepath{}%
\end{pgfscope}%
\begin{pgfscope}%
\pgfpathrectangle{\pgfqpoint{0.549740in}{0.463273in}}{\pgfqpoint{9.320225in}{4.495057in}}%
\pgfusepath{clip}%
\pgfsetbuttcap%
\pgfsetroundjoin%
\pgfsetlinewidth{0.000000pt}%
\definecolor{currentstroke}{rgb}{0.000000,0.000000,0.000000}%
\pgfsetstrokecolor{currentstroke}%
\pgfsetdash{}{0pt}%
\pgfpathmoveto{\pgfqpoint{8.371276in}{4.427096in}}%
\pgfpathlineto{\pgfqpoint{8.557503in}{4.427096in}}%
\pgfpathlineto{\pgfqpoint{8.557503in}{4.508824in}}%
\pgfpathlineto{\pgfqpoint{8.371276in}{4.508824in}}%
\pgfpathlineto{\pgfqpoint{8.371276in}{4.427096in}}%
\pgfusepath{}%
\end{pgfscope}%
\begin{pgfscope}%
\pgfpathrectangle{\pgfqpoint{0.549740in}{0.463273in}}{\pgfqpoint{9.320225in}{4.495057in}}%
\pgfusepath{clip}%
\pgfsetbuttcap%
\pgfsetroundjoin%
\pgfsetlinewidth{0.000000pt}%
\definecolor{currentstroke}{rgb}{0.000000,0.000000,0.000000}%
\pgfsetstrokecolor{currentstroke}%
\pgfsetdash{}{0pt}%
\pgfpathmoveto{\pgfqpoint{8.557503in}{4.427096in}}%
\pgfpathlineto{\pgfqpoint{8.743729in}{4.427096in}}%
\pgfpathlineto{\pgfqpoint{8.743729in}{4.508824in}}%
\pgfpathlineto{\pgfqpoint{8.557503in}{4.508824in}}%
\pgfpathlineto{\pgfqpoint{8.557503in}{4.427096in}}%
\pgfusepath{}%
\end{pgfscope}%
\begin{pgfscope}%
\pgfpathrectangle{\pgfqpoint{0.549740in}{0.463273in}}{\pgfqpoint{9.320225in}{4.495057in}}%
\pgfusepath{clip}%
\pgfsetbuttcap%
\pgfsetroundjoin%
\pgfsetlinewidth{0.000000pt}%
\definecolor{currentstroke}{rgb}{0.000000,0.000000,0.000000}%
\pgfsetstrokecolor{currentstroke}%
\pgfsetdash{}{0pt}%
\pgfpathmoveto{\pgfqpoint{8.743729in}{4.427096in}}%
\pgfpathlineto{\pgfqpoint{8.929956in}{4.427096in}}%
\pgfpathlineto{\pgfqpoint{8.929956in}{4.508824in}}%
\pgfpathlineto{\pgfqpoint{8.743729in}{4.508824in}}%
\pgfpathlineto{\pgfqpoint{8.743729in}{4.427096in}}%
\pgfusepath{}%
\end{pgfscope}%
\begin{pgfscope}%
\pgfpathrectangle{\pgfqpoint{0.549740in}{0.463273in}}{\pgfqpoint{9.320225in}{4.495057in}}%
\pgfusepath{clip}%
\pgfsetbuttcap%
\pgfsetroundjoin%
\definecolor{currentfill}{rgb}{0.472869,0.711325,0.955316}%
\pgfsetfillcolor{currentfill}%
\pgfsetlinewidth{0.000000pt}%
\definecolor{currentstroke}{rgb}{0.000000,0.000000,0.000000}%
\pgfsetstrokecolor{currentstroke}%
\pgfsetdash{}{0pt}%
\pgfpathmoveto{\pgfqpoint{8.929956in}{4.427096in}}%
\pgfpathlineto{\pgfqpoint{9.116182in}{4.427096in}}%
\pgfpathlineto{\pgfqpoint{9.116182in}{4.508824in}}%
\pgfpathlineto{\pgfqpoint{8.929956in}{4.508824in}}%
\pgfpathlineto{\pgfqpoint{8.929956in}{4.427096in}}%
\pgfusepath{fill}%
\end{pgfscope}%
\begin{pgfscope}%
\pgfpathrectangle{\pgfqpoint{0.549740in}{0.463273in}}{\pgfqpoint{9.320225in}{4.495057in}}%
\pgfusepath{clip}%
\pgfsetbuttcap%
\pgfsetroundjoin%
\pgfsetlinewidth{0.000000pt}%
\definecolor{currentstroke}{rgb}{0.000000,0.000000,0.000000}%
\pgfsetstrokecolor{currentstroke}%
\pgfsetdash{}{0pt}%
\pgfpathmoveto{\pgfqpoint{9.116182in}{4.427096in}}%
\pgfpathlineto{\pgfqpoint{9.302409in}{4.427096in}}%
\pgfpathlineto{\pgfqpoint{9.302409in}{4.508824in}}%
\pgfpathlineto{\pgfqpoint{9.116182in}{4.508824in}}%
\pgfpathlineto{\pgfqpoint{9.116182in}{4.427096in}}%
\pgfusepath{}%
\end{pgfscope}%
\begin{pgfscope}%
\pgfpathrectangle{\pgfqpoint{0.549740in}{0.463273in}}{\pgfqpoint{9.320225in}{4.495057in}}%
\pgfusepath{clip}%
\pgfsetbuttcap%
\pgfsetroundjoin%
\pgfsetlinewidth{0.000000pt}%
\definecolor{currentstroke}{rgb}{0.000000,0.000000,0.000000}%
\pgfsetstrokecolor{currentstroke}%
\pgfsetdash{}{0pt}%
\pgfpathmoveto{\pgfqpoint{9.302409in}{4.427096in}}%
\pgfpathlineto{\pgfqpoint{9.488635in}{4.427096in}}%
\pgfpathlineto{\pgfqpoint{9.488635in}{4.508824in}}%
\pgfpathlineto{\pgfqpoint{9.302409in}{4.508824in}}%
\pgfpathlineto{\pgfqpoint{9.302409in}{4.427096in}}%
\pgfusepath{}%
\end{pgfscope}%
\begin{pgfscope}%
\pgfpathrectangle{\pgfqpoint{0.549740in}{0.463273in}}{\pgfqpoint{9.320225in}{4.495057in}}%
\pgfusepath{clip}%
\pgfsetbuttcap%
\pgfsetroundjoin%
\pgfsetlinewidth{0.000000pt}%
\definecolor{currentstroke}{rgb}{0.000000,0.000000,0.000000}%
\pgfsetstrokecolor{currentstroke}%
\pgfsetdash{}{0pt}%
\pgfpathmoveto{\pgfqpoint{9.488635in}{4.427096in}}%
\pgfpathlineto{\pgfqpoint{9.674862in}{4.427096in}}%
\pgfpathlineto{\pgfqpoint{9.674862in}{4.508824in}}%
\pgfpathlineto{\pgfqpoint{9.488635in}{4.508824in}}%
\pgfpathlineto{\pgfqpoint{9.488635in}{4.427096in}}%
\pgfusepath{}%
\end{pgfscope}%
\begin{pgfscope}%
\pgfpathrectangle{\pgfqpoint{0.549740in}{0.463273in}}{\pgfqpoint{9.320225in}{4.495057in}}%
\pgfusepath{clip}%
\pgfsetbuttcap%
\pgfsetroundjoin%
\pgfsetlinewidth{0.000000pt}%
\definecolor{currentstroke}{rgb}{0.000000,0.000000,0.000000}%
\pgfsetstrokecolor{currentstroke}%
\pgfsetdash{}{0pt}%
\pgfpathmoveto{\pgfqpoint{9.674862in}{4.427096in}}%
\pgfpathlineto{\pgfqpoint{9.861088in}{4.427096in}}%
\pgfpathlineto{\pgfqpoint{9.861088in}{4.508824in}}%
\pgfpathlineto{\pgfqpoint{9.674862in}{4.508824in}}%
\pgfpathlineto{\pgfqpoint{9.674862in}{4.427096in}}%
\pgfusepath{}%
\end{pgfscope}%
\begin{pgfscope}%
\pgfpathrectangle{\pgfqpoint{0.549740in}{0.463273in}}{\pgfqpoint{9.320225in}{4.495057in}}%
\pgfusepath{clip}%
\pgfsetbuttcap%
\pgfsetroundjoin%
\pgfsetlinewidth{0.000000pt}%
\definecolor{currentstroke}{rgb}{0.000000,0.000000,0.000000}%
\pgfsetstrokecolor{currentstroke}%
\pgfsetdash{}{0pt}%
\pgfpathmoveto{\pgfqpoint{0.549761in}{4.508824in}}%
\pgfpathlineto{\pgfqpoint{0.735988in}{4.508824in}}%
\pgfpathlineto{\pgfqpoint{0.735988in}{4.590553in}}%
\pgfpathlineto{\pgfqpoint{0.549761in}{4.590553in}}%
\pgfpathlineto{\pgfqpoint{0.549761in}{4.508824in}}%
\pgfusepath{}%
\end{pgfscope}%
\begin{pgfscope}%
\pgfpathrectangle{\pgfqpoint{0.549740in}{0.463273in}}{\pgfqpoint{9.320225in}{4.495057in}}%
\pgfusepath{clip}%
\pgfsetbuttcap%
\pgfsetroundjoin%
\pgfsetlinewidth{0.000000pt}%
\definecolor{currentstroke}{rgb}{0.000000,0.000000,0.000000}%
\pgfsetstrokecolor{currentstroke}%
\pgfsetdash{}{0pt}%
\pgfpathmoveto{\pgfqpoint{0.735988in}{4.508824in}}%
\pgfpathlineto{\pgfqpoint{0.922214in}{4.508824in}}%
\pgfpathlineto{\pgfqpoint{0.922214in}{4.590553in}}%
\pgfpathlineto{\pgfqpoint{0.735988in}{4.590553in}}%
\pgfpathlineto{\pgfqpoint{0.735988in}{4.508824in}}%
\pgfusepath{}%
\end{pgfscope}%
\begin{pgfscope}%
\pgfpathrectangle{\pgfqpoint{0.549740in}{0.463273in}}{\pgfqpoint{9.320225in}{4.495057in}}%
\pgfusepath{clip}%
\pgfsetbuttcap%
\pgfsetroundjoin%
\pgfsetlinewidth{0.000000pt}%
\definecolor{currentstroke}{rgb}{0.000000,0.000000,0.000000}%
\pgfsetstrokecolor{currentstroke}%
\pgfsetdash{}{0pt}%
\pgfpathmoveto{\pgfqpoint{0.922214in}{4.508824in}}%
\pgfpathlineto{\pgfqpoint{1.108441in}{4.508824in}}%
\pgfpathlineto{\pgfqpoint{1.108441in}{4.590553in}}%
\pgfpathlineto{\pgfqpoint{0.922214in}{4.590553in}}%
\pgfpathlineto{\pgfqpoint{0.922214in}{4.508824in}}%
\pgfusepath{}%
\end{pgfscope}%
\begin{pgfscope}%
\pgfpathrectangle{\pgfqpoint{0.549740in}{0.463273in}}{\pgfqpoint{9.320225in}{4.495057in}}%
\pgfusepath{clip}%
\pgfsetbuttcap%
\pgfsetroundjoin%
\pgfsetlinewidth{0.000000pt}%
\definecolor{currentstroke}{rgb}{0.000000,0.000000,0.000000}%
\pgfsetstrokecolor{currentstroke}%
\pgfsetdash{}{0pt}%
\pgfpathmoveto{\pgfqpoint{1.108441in}{4.508824in}}%
\pgfpathlineto{\pgfqpoint{1.294667in}{4.508824in}}%
\pgfpathlineto{\pgfqpoint{1.294667in}{4.590553in}}%
\pgfpathlineto{\pgfqpoint{1.108441in}{4.590553in}}%
\pgfpathlineto{\pgfqpoint{1.108441in}{4.508824in}}%
\pgfusepath{}%
\end{pgfscope}%
\begin{pgfscope}%
\pgfpathrectangle{\pgfqpoint{0.549740in}{0.463273in}}{\pgfqpoint{9.320225in}{4.495057in}}%
\pgfusepath{clip}%
\pgfsetbuttcap%
\pgfsetroundjoin%
\pgfsetlinewidth{0.000000pt}%
\definecolor{currentstroke}{rgb}{0.000000,0.000000,0.000000}%
\pgfsetstrokecolor{currentstroke}%
\pgfsetdash{}{0pt}%
\pgfpathmoveto{\pgfqpoint{1.294667in}{4.508824in}}%
\pgfpathlineto{\pgfqpoint{1.480894in}{4.508824in}}%
\pgfpathlineto{\pgfqpoint{1.480894in}{4.590553in}}%
\pgfpathlineto{\pgfqpoint{1.294667in}{4.590553in}}%
\pgfpathlineto{\pgfqpoint{1.294667in}{4.508824in}}%
\pgfusepath{}%
\end{pgfscope}%
\begin{pgfscope}%
\pgfpathrectangle{\pgfqpoint{0.549740in}{0.463273in}}{\pgfqpoint{9.320225in}{4.495057in}}%
\pgfusepath{clip}%
\pgfsetbuttcap%
\pgfsetroundjoin%
\pgfsetlinewidth{0.000000pt}%
\definecolor{currentstroke}{rgb}{0.000000,0.000000,0.000000}%
\pgfsetstrokecolor{currentstroke}%
\pgfsetdash{}{0pt}%
\pgfpathmoveto{\pgfqpoint{1.480894in}{4.508824in}}%
\pgfpathlineto{\pgfqpoint{1.667120in}{4.508824in}}%
\pgfpathlineto{\pgfqpoint{1.667120in}{4.590553in}}%
\pgfpathlineto{\pgfqpoint{1.480894in}{4.590553in}}%
\pgfpathlineto{\pgfqpoint{1.480894in}{4.508824in}}%
\pgfusepath{}%
\end{pgfscope}%
\begin{pgfscope}%
\pgfpathrectangle{\pgfqpoint{0.549740in}{0.463273in}}{\pgfqpoint{9.320225in}{4.495057in}}%
\pgfusepath{clip}%
\pgfsetbuttcap%
\pgfsetroundjoin%
\pgfsetlinewidth{0.000000pt}%
\definecolor{currentstroke}{rgb}{0.000000,0.000000,0.000000}%
\pgfsetstrokecolor{currentstroke}%
\pgfsetdash{}{0pt}%
\pgfpathmoveto{\pgfqpoint{1.667120in}{4.508824in}}%
\pgfpathlineto{\pgfqpoint{1.853347in}{4.508824in}}%
\pgfpathlineto{\pgfqpoint{1.853347in}{4.590553in}}%
\pgfpathlineto{\pgfqpoint{1.667120in}{4.590553in}}%
\pgfpathlineto{\pgfqpoint{1.667120in}{4.508824in}}%
\pgfusepath{}%
\end{pgfscope}%
\begin{pgfscope}%
\pgfpathrectangle{\pgfqpoint{0.549740in}{0.463273in}}{\pgfqpoint{9.320225in}{4.495057in}}%
\pgfusepath{clip}%
\pgfsetbuttcap%
\pgfsetroundjoin%
\pgfsetlinewidth{0.000000pt}%
\definecolor{currentstroke}{rgb}{0.000000,0.000000,0.000000}%
\pgfsetstrokecolor{currentstroke}%
\pgfsetdash{}{0pt}%
\pgfpathmoveto{\pgfqpoint{1.853347in}{4.508824in}}%
\pgfpathlineto{\pgfqpoint{2.039573in}{4.508824in}}%
\pgfpathlineto{\pgfqpoint{2.039573in}{4.590553in}}%
\pgfpathlineto{\pgfqpoint{1.853347in}{4.590553in}}%
\pgfpathlineto{\pgfqpoint{1.853347in}{4.508824in}}%
\pgfusepath{}%
\end{pgfscope}%
\begin{pgfscope}%
\pgfpathrectangle{\pgfqpoint{0.549740in}{0.463273in}}{\pgfqpoint{9.320225in}{4.495057in}}%
\pgfusepath{clip}%
\pgfsetbuttcap%
\pgfsetroundjoin%
\pgfsetlinewidth{0.000000pt}%
\definecolor{currentstroke}{rgb}{0.000000,0.000000,0.000000}%
\pgfsetstrokecolor{currentstroke}%
\pgfsetdash{}{0pt}%
\pgfpathmoveto{\pgfqpoint{2.039573in}{4.508824in}}%
\pgfpathlineto{\pgfqpoint{2.225800in}{4.508824in}}%
\pgfpathlineto{\pgfqpoint{2.225800in}{4.590553in}}%
\pgfpathlineto{\pgfqpoint{2.039573in}{4.590553in}}%
\pgfpathlineto{\pgfqpoint{2.039573in}{4.508824in}}%
\pgfusepath{}%
\end{pgfscope}%
\begin{pgfscope}%
\pgfpathrectangle{\pgfqpoint{0.549740in}{0.463273in}}{\pgfqpoint{9.320225in}{4.495057in}}%
\pgfusepath{clip}%
\pgfsetbuttcap%
\pgfsetroundjoin%
\pgfsetlinewidth{0.000000pt}%
\definecolor{currentstroke}{rgb}{0.000000,0.000000,0.000000}%
\pgfsetstrokecolor{currentstroke}%
\pgfsetdash{}{0pt}%
\pgfpathmoveto{\pgfqpoint{2.225800in}{4.508824in}}%
\pgfpathlineto{\pgfqpoint{2.412027in}{4.508824in}}%
\pgfpathlineto{\pgfqpoint{2.412027in}{4.590553in}}%
\pgfpathlineto{\pgfqpoint{2.225800in}{4.590553in}}%
\pgfpathlineto{\pgfqpoint{2.225800in}{4.508824in}}%
\pgfusepath{}%
\end{pgfscope}%
\begin{pgfscope}%
\pgfpathrectangle{\pgfqpoint{0.549740in}{0.463273in}}{\pgfqpoint{9.320225in}{4.495057in}}%
\pgfusepath{clip}%
\pgfsetbuttcap%
\pgfsetroundjoin%
\pgfsetlinewidth{0.000000pt}%
\definecolor{currentstroke}{rgb}{0.000000,0.000000,0.000000}%
\pgfsetstrokecolor{currentstroke}%
\pgfsetdash{}{0pt}%
\pgfpathmoveto{\pgfqpoint{2.412027in}{4.508824in}}%
\pgfpathlineto{\pgfqpoint{2.598253in}{4.508824in}}%
\pgfpathlineto{\pgfqpoint{2.598253in}{4.590553in}}%
\pgfpathlineto{\pgfqpoint{2.412027in}{4.590553in}}%
\pgfpathlineto{\pgfqpoint{2.412027in}{4.508824in}}%
\pgfusepath{}%
\end{pgfscope}%
\begin{pgfscope}%
\pgfpathrectangle{\pgfqpoint{0.549740in}{0.463273in}}{\pgfqpoint{9.320225in}{4.495057in}}%
\pgfusepath{clip}%
\pgfsetbuttcap%
\pgfsetroundjoin%
\pgfsetlinewidth{0.000000pt}%
\definecolor{currentstroke}{rgb}{0.000000,0.000000,0.000000}%
\pgfsetstrokecolor{currentstroke}%
\pgfsetdash{}{0pt}%
\pgfpathmoveto{\pgfqpoint{2.598253in}{4.508824in}}%
\pgfpathlineto{\pgfqpoint{2.784480in}{4.508824in}}%
\pgfpathlineto{\pgfqpoint{2.784480in}{4.590553in}}%
\pgfpathlineto{\pgfqpoint{2.598253in}{4.590553in}}%
\pgfpathlineto{\pgfqpoint{2.598253in}{4.508824in}}%
\pgfusepath{}%
\end{pgfscope}%
\begin{pgfscope}%
\pgfpathrectangle{\pgfqpoint{0.549740in}{0.463273in}}{\pgfqpoint{9.320225in}{4.495057in}}%
\pgfusepath{clip}%
\pgfsetbuttcap%
\pgfsetroundjoin%
\pgfsetlinewidth{0.000000pt}%
\definecolor{currentstroke}{rgb}{0.000000,0.000000,0.000000}%
\pgfsetstrokecolor{currentstroke}%
\pgfsetdash{}{0pt}%
\pgfpathmoveto{\pgfqpoint{2.784480in}{4.508824in}}%
\pgfpathlineto{\pgfqpoint{2.970706in}{4.508824in}}%
\pgfpathlineto{\pgfqpoint{2.970706in}{4.590553in}}%
\pgfpathlineto{\pgfqpoint{2.784480in}{4.590553in}}%
\pgfpathlineto{\pgfqpoint{2.784480in}{4.508824in}}%
\pgfusepath{}%
\end{pgfscope}%
\begin{pgfscope}%
\pgfpathrectangle{\pgfqpoint{0.549740in}{0.463273in}}{\pgfqpoint{9.320225in}{4.495057in}}%
\pgfusepath{clip}%
\pgfsetbuttcap%
\pgfsetroundjoin%
\pgfsetlinewidth{0.000000pt}%
\definecolor{currentstroke}{rgb}{0.000000,0.000000,0.000000}%
\pgfsetstrokecolor{currentstroke}%
\pgfsetdash{}{0pt}%
\pgfpathmoveto{\pgfqpoint{2.970706in}{4.508824in}}%
\pgfpathlineto{\pgfqpoint{3.156933in}{4.508824in}}%
\pgfpathlineto{\pgfqpoint{3.156933in}{4.590553in}}%
\pgfpathlineto{\pgfqpoint{2.970706in}{4.590553in}}%
\pgfpathlineto{\pgfqpoint{2.970706in}{4.508824in}}%
\pgfusepath{}%
\end{pgfscope}%
\begin{pgfscope}%
\pgfpathrectangle{\pgfqpoint{0.549740in}{0.463273in}}{\pgfqpoint{9.320225in}{4.495057in}}%
\pgfusepath{clip}%
\pgfsetbuttcap%
\pgfsetroundjoin%
\pgfsetlinewidth{0.000000pt}%
\definecolor{currentstroke}{rgb}{0.000000,0.000000,0.000000}%
\pgfsetstrokecolor{currentstroke}%
\pgfsetdash{}{0pt}%
\pgfpathmoveto{\pgfqpoint{3.156933in}{4.508824in}}%
\pgfpathlineto{\pgfqpoint{3.343159in}{4.508824in}}%
\pgfpathlineto{\pgfqpoint{3.343159in}{4.590553in}}%
\pgfpathlineto{\pgfqpoint{3.156933in}{4.590553in}}%
\pgfpathlineto{\pgfqpoint{3.156933in}{4.508824in}}%
\pgfusepath{}%
\end{pgfscope}%
\begin{pgfscope}%
\pgfpathrectangle{\pgfqpoint{0.549740in}{0.463273in}}{\pgfqpoint{9.320225in}{4.495057in}}%
\pgfusepath{clip}%
\pgfsetbuttcap%
\pgfsetroundjoin%
\pgfsetlinewidth{0.000000pt}%
\definecolor{currentstroke}{rgb}{0.000000,0.000000,0.000000}%
\pgfsetstrokecolor{currentstroke}%
\pgfsetdash{}{0pt}%
\pgfpathmoveto{\pgfqpoint{3.343159in}{4.508824in}}%
\pgfpathlineto{\pgfqpoint{3.529386in}{4.508824in}}%
\pgfpathlineto{\pgfqpoint{3.529386in}{4.590553in}}%
\pgfpathlineto{\pgfqpoint{3.343159in}{4.590553in}}%
\pgfpathlineto{\pgfqpoint{3.343159in}{4.508824in}}%
\pgfusepath{}%
\end{pgfscope}%
\begin{pgfscope}%
\pgfpathrectangle{\pgfqpoint{0.549740in}{0.463273in}}{\pgfqpoint{9.320225in}{4.495057in}}%
\pgfusepath{clip}%
\pgfsetbuttcap%
\pgfsetroundjoin%
\pgfsetlinewidth{0.000000pt}%
\definecolor{currentstroke}{rgb}{0.000000,0.000000,0.000000}%
\pgfsetstrokecolor{currentstroke}%
\pgfsetdash{}{0pt}%
\pgfpathmoveto{\pgfqpoint{3.529386in}{4.508824in}}%
\pgfpathlineto{\pgfqpoint{3.715612in}{4.508824in}}%
\pgfpathlineto{\pgfqpoint{3.715612in}{4.590553in}}%
\pgfpathlineto{\pgfqpoint{3.529386in}{4.590553in}}%
\pgfpathlineto{\pgfqpoint{3.529386in}{4.508824in}}%
\pgfusepath{}%
\end{pgfscope}%
\begin{pgfscope}%
\pgfpathrectangle{\pgfqpoint{0.549740in}{0.463273in}}{\pgfqpoint{9.320225in}{4.495057in}}%
\pgfusepath{clip}%
\pgfsetbuttcap%
\pgfsetroundjoin%
\pgfsetlinewidth{0.000000pt}%
\definecolor{currentstroke}{rgb}{0.000000,0.000000,0.000000}%
\pgfsetstrokecolor{currentstroke}%
\pgfsetdash{}{0pt}%
\pgfpathmoveto{\pgfqpoint{3.715612in}{4.508824in}}%
\pgfpathlineto{\pgfqpoint{3.901839in}{4.508824in}}%
\pgfpathlineto{\pgfqpoint{3.901839in}{4.590553in}}%
\pgfpathlineto{\pgfqpoint{3.715612in}{4.590553in}}%
\pgfpathlineto{\pgfqpoint{3.715612in}{4.508824in}}%
\pgfusepath{}%
\end{pgfscope}%
\begin{pgfscope}%
\pgfpathrectangle{\pgfqpoint{0.549740in}{0.463273in}}{\pgfqpoint{9.320225in}{4.495057in}}%
\pgfusepath{clip}%
\pgfsetbuttcap%
\pgfsetroundjoin%
\pgfsetlinewidth{0.000000pt}%
\definecolor{currentstroke}{rgb}{0.000000,0.000000,0.000000}%
\pgfsetstrokecolor{currentstroke}%
\pgfsetdash{}{0pt}%
\pgfpathmoveto{\pgfqpoint{3.901839in}{4.508824in}}%
\pgfpathlineto{\pgfqpoint{4.088065in}{4.508824in}}%
\pgfpathlineto{\pgfqpoint{4.088065in}{4.590553in}}%
\pgfpathlineto{\pgfqpoint{3.901839in}{4.590553in}}%
\pgfpathlineto{\pgfqpoint{3.901839in}{4.508824in}}%
\pgfusepath{}%
\end{pgfscope}%
\begin{pgfscope}%
\pgfpathrectangle{\pgfqpoint{0.549740in}{0.463273in}}{\pgfqpoint{9.320225in}{4.495057in}}%
\pgfusepath{clip}%
\pgfsetbuttcap%
\pgfsetroundjoin%
\pgfsetlinewidth{0.000000pt}%
\definecolor{currentstroke}{rgb}{0.000000,0.000000,0.000000}%
\pgfsetstrokecolor{currentstroke}%
\pgfsetdash{}{0pt}%
\pgfpathmoveto{\pgfqpoint{4.088065in}{4.508824in}}%
\pgfpathlineto{\pgfqpoint{4.274292in}{4.508824in}}%
\pgfpathlineto{\pgfqpoint{4.274292in}{4.590553in}}%
\pgfpathlineto{\pgfqpoint{4.088065in}{4.590553in}}%
\pgfpathlineto{\pgfqpoint{4.088065in}{4.508824in}}%
\pgfusepath{}%
\end{pgfscope}%
\begin{pgfscope}%
\pgfpathrectangle{\pgfqpoint{0.549740in}{0.463273in}}{\pgfqpoint{9.320225in}{4.495057in}}%
\pgfusepath{clip}%
\pgfsetbuttcap%
\pgfsetroundjoin%
\pgfsetlinewidth{0.000000pt}%
\definecolor{currentstroke}{rgb}{0.000000,0.000000,0.000000}%
\pgfsetstrokecolor{currentstroke}%
\pgfsetdash{}{0pt}%
\pgfpathmoveto{\pgfqpoint{4.274292in}{4.508824in}}%
\pgfpathlineto{\pgfqpoint{4.460519in}{4.508824in}}%
\pgfpathlineto{\pgfqpoint{4.460519in}{4.590553in}}%
\pgfpathlineto{\pgfqpoint{4.274292in}{4.590553in}}%
\pgfpathlineto{\pgfqpoint{4.274292in}{4.508824in}}%
\pgfusepath{}%
\end{pgfscope}%
\begin{pgfscope}%
\pgfpathrectangle{\pgfqpoint{0.549740in}{0.463273in}}{\pgfqpoint{9.320225in}{4.495057in}}%
\pgfusepath{clip}%
\pgfsetbuttcap%
\pgfsetroundjoin%
\pgfsetlinewidth{0.000000pt}%
\definecolor{currentstroke}{rgb}{0.000000,0.000000,0.000000}%
\pgfsetstrokecolor{currentstroke}%
\pgfsetdash{}{0pt}%
\pgfpathmoveto{\pgfqpoint{4.460519in}{4.508824in}}%
\pgfpathlineto{\pgfqpoint{4.646745in}{4.508824in}}%
\pgfpathlineto{\pgfqpoint{4.646745in}{4.590553in}}%
\pgfpathlineto{\pgfqpoint{4.460519in}{4.590553in}}%
\pgfpathlineto{\pgfqpoint{4.460519in}{4.508824in}}%
\pgfusepath{}%
\end{pgfscope}%
\begin{pgfscope}%
\pgfpathrectangle{\pgfqpoint{0.549740in}{0.463273in}}{\pgfqpoint{9.320225in}{4.495057in}}%
\pgfusepath{clip}%
\pgfsetbuttcap%
\pgfsetroundjoin%
\pgfsetlinewidth{0.000000pt}%
\definecolor{currentstroke}{rgb}{0.000000,0.000000,0.000000}%
\pgfsetstrokecolor{currentstroke}%
\pgfsetdash{}{0pt}%
\pgfpathmoveto{\pgfqpoint{4.646745in}{4.508824in}}%
\pgfpathlineto{\pgfqpoint{4.832972in}{4.508824in}}%
\pgfpathlineto{\pgfqpoint{4.832972in}{4.590553in}}%
\pgfpathlineto{\pgfqpoint{4.646745in}{4.590553in}}%
\pgfpathlineto{\pgfqpoint{4.646745in}{4.508824in}}%
\pgfusepath{}%
\end{pgfscope}%
\begin{pgfscope}%
\pgfpathrectangle{\pgfqpoint{0.549740in}{0.463273in}}{\pgfqpoint{9.320225in}{4.495057in}}%
\pgfusepath{clip}%
\pgfsetbuttcap%
\pgfsetroundjoin%
\pgfsetlinewidth{0.000000pt}%
\definecolor{currentstroke}{rgb}{0.000000,0.000000,0.000000}%
\pgfsetstrokecolor{currentstroke}%
\pgfsetdash{}{0pt}%
\pgfpathmoveto{\pgfqpoint{4.832972in}{4.508824in}}%
\pgfpathlineto{\pgfqpoint{5.019198in}{4.508824in}}%
\pgfpathlineto{\pgfqpoint{5.019198in}{4.590553in}}%
\pgfpathlineto{\pgfqpoint{4.832972in}{4.590553in}}%
\pgfpathlineto{\pgfqpoint{4.832972in}{4.508824in}}%
\pgfusepath{}%
\end{pgfscope}%
\begin{pgfscope}%
\pgfpathrectangle{\pgfqpoint{0.549740in}{0.463273in}}{\pgfqpoint{9.320225in}{4.495057in}}%
\pgfusepath{clip}%
\pgfsetbuttcap%
\pgfsetroundjoin%
\pgfsetlinewidth{0.000000pt}%
\definecolor{currentstroke}{rgb}{0.000000,0.000000,0.000000}%
\pgfsetstrokecolor{currentstroke}%
\pgfsetdash{}{0pt}%
\pgfpathmoveto{\pgfqpoint{5.019198in}{4.508824in}}%
\pgfpathlineto{\pgfqpoint{5.205425in}{4.508824in}}%
\pgfpathlineto{\pgfqpoint{5.205425in}{4.590553in}}%
\pgfpathlineto{\pgfqpoint{5.019198in}{4.590553in}}%
\pgfpathlineto{\pgfqpoint{5.019198in}{4.508824in}}%
\pgfusepath{}%
\end{pgfscope}%
\begin{pgfscope}%
\pgfpathrectangle{\pgfqpoint{0.549740in}{0.463273in}}{\pgfqpoint{9.320225in}{4.495057in}}%
\pgfusepath{clip}%
\pgfsetbuttcap%
\pgfsetroundjoin%
\pgfsetlinewidth{0.000000pt}%
\definecolor{currentstroke}{rgb}{0.000000,0.000000,0.000000}%
\pgfsetstrokecolor{currentstroke}%
\pgfsetdash{}{0pt}%
\pgfpathmoveto{\pgfqpoint{5.205425in}{4.508824in}}%
\pgfpathlineto{\pgfqpoint{5.391651in}{4.508824in}}%
\pgfpathlineto{\pgfqpoint{5.391651in}{4.590553in}}%
\pgfpathlineto{\pgfqpoint{5.205425in}{4.590553in}}%
\pgfpathlineto{\pgfqpoint{5.205425in}{4.508824in}}%
\pgfusepath{}%
\end{pgfscope}%
\begin{pgfscope}%
\pgfpathrectangle{\pgfqpoint{0.549740in}{0.463273in}}{\pgfqpoint{9.320225in}{4.495057in}}%
\pgfusepath{clip}%
\pgfsetbuttcap%
\pgfsetroundjoin%
\pgfsetlinewidth{0.000000pt}%
\definecolor{currentstroke}{rgb}{0.000000,0.000000,0.000000}%
\pgfsetstrokecolor{currentstroke}%
\pgfsetdash{}{0pt}%
\pgfpathmoveto{\pgfqpoint{5.391651in}{4.508824in}}%
\pgfpathlineto{\pgfqpoint{5.577878in}{4.508824in}}%
\pgfpathlineto{\pgfqpoint{5.577878in}{4.590553in}}%
\pgfpathlineto{\pgfqpoint{5.391651in}{4.590553in}}%
\pgfpathlineto{\pgfqpoint{5.391651in}{4.508824in}}%
\pgfusepath{}%
\end{pgfscope}%
\begin{pgfscope}%
\pgfpathrectangle{\pgfqpoint{0.549740in}{0.463273in}}{\pgfqpoint{9.320225in}{4.495057in}}%
\pgfusepath{clip}%
\pgfsetbuttcap%
\pgfsetroundjoin%
\pgfsetlinewidth{0.000000pt}%
\definecolor{currentstroke}{rgb}{0.000000,0.000000,0.000000}%
\pgfsetstrokecolor{currentstroke}%
\pgfsetdash{}{0pt}%
\pgfpathmoveto{\pgfqpoint{5.577878in}{4.508824in}}%
\pgfpathlineto{\pgfqpoint{5.764104in}{4.508824in}}%
\pgfpathlineto{\pgfqpoint{5.764104in}{4.590553in}}%
\pgfpathlineto{\pgfqpoint{5.577878in}{4.590553in}}%
\pgfpathlineto{\pgfqpoint{5.577878in}{4.508824in}}%
\pgfusepath{}%
\end{pgfscope}%
\begin{pgfscope}%
\pgfpathrectangle{\pgfqpoint{0.549740in}{0.463273in}}{\pgfqpoint{9.320225in}{4.495057in}}%
\pgfusepath{clip}%
\pgfsetbuttcap%
\pgfsetroundjoin%
\pgfsetlinewidth{0.000000pt}%
\definecolor{currentstroke}{rgb}{0.000000,0.000000,0.000000}%
\pgfsetstrokecolor{currentstroke}%
\pgfsetdash{}{0pt}%
\pgfpathmoveto{\pgfqpoint{5.764104in}{4.508824in}}%
\pgfpathlineto{\pgfqpoint{5.950331in}{4.508824in}}%
\pgfpathlineto{\pgfqpoint{5.950331in}{4.590553in}}%
\pgfpathlineto{\pgfqpoint{5.764104in}{4.590553in}}%
\pgfpathlineto{\pgfqpoint{5.764104in}{4.508824in}}%
\pgfusepath{}%
\end{pgfscope}%
\begin{pgfscope}%
\pgfpathrectangle{\pgfqpoint{0.549740in}{0.463273in}}{\pgfqpoint{9.320225in}{4.495057in}}%
\pgfusepath{clip}%
\pgfsetbuttcap%
\pgfsetroundjoin%
\pgfsetlinewidth{0.000000pt}%
\definecolor{currentstroke}{rgb}{0.000000,0.000000,0.000000}%
\pgfsetstrokecolor{currentstroke}%
\pgfsetdash{}{0pt}%
\pgfpathmoveto{\pgfqpoint{5.950331in}{4.508824in}}%
\pgfpathlineto{\pgfqpoint{6.136557in}{4.508824in}}%
\pgfpathlineto{\pgfqpoint{6.136557in}{4.590553in}}%
\pgfpathlineto{\pgfqpoint{5.950331in}{4.590553in}}%
\pgfpathlineto{\pgfqpoint{5.950331in}{4.508824in}}%
\pgfusepath{}%
\end{pgfscope}%
\begin{pgfscope}%
\pgfpathrectangle{\pgfqpoint{0.549740in}{0.463273in}}{\pgfqpoint{9.320225in}{4.495057in}}%
\pgfusepath{clip}%
\pgfsetbuttcap%
\pgfsetroundjoin%
\pgfsetlinewidth{0.000000pt}%
\definecolor{currentstroke}{rgb}{0.000000,0.000000,0.000000}%
\pgfsetstrokecolor{currentstroke}%
\pgfsetdash{}{0pt}%
\pgfpathmoveto{\pgfqpoint{6.136557in}{4.508824in}}%
\pgfpathlineto{\pgfqpoint{6.322784in}{4.508824in}}%
\pgfpathlineto{\pgfqpoint{6.322784in}{4.590553in}}%
\pgfpathlineto{\pgfqpoint{6.136557in}{4.590553in}}%
\pgfpathlineto{\pgfqpoint{6.136557in}{4.508824in}}%
\pgfusepath{}%
\end{pgfscope}%
\begin{pgfscope}%
\pgfpathrectangle{\pgfqpoint{0.549740in}{0.463273in}}{\pgfqpoint{9.320225in}{4.495057in}}%
\pgfusepath{clip}%
\pgfsetbuttcap%
\pgfsetroundjoin%
\pgfsetlinewidth{0.000000pt}%
\definecolor{currentstroke}{rgb}{0.000000,0.000000,0.000000}%
\pgfsetstrokecolor{currentstroke}%
\pgfsetdash{}{0pt}%
\pgfpathmoveto{\pgfqpoint{6.322784in}{4.508824in}}%
\pgfpathlineto{\pgfqpoint{6.509011in}{4.508824in}}%
\pgfpathlineto{\pgfqpoint{6.509011in}{4.590553in}}%
\pgfpathlineto{\pgfqpoint{6.322784in}{4.590553in}}%
\pgfpathlineto{\pgfqpoint{6.322784in}{4.508824in}}%
\pgfusepath{}%
\end{pgfscope}%
\begin{pgfscope}%
\pgfpathrectangle{\pgfqpoint{0.549740in}{0.463273in}}{\pgfqpoint{9.320225in}{4.495057in}}%
\pgfusepath{clip}%
\pgfsetbuttcap%
\pgfsetroundjoin%
\pgfsetlinewidth{0.000000pt}%
\definecolor{currentstroke}{rgb}{0.000000,0.000000,0.000000}%
\pgfsetstrokecolor{currentstroke}%
\pgfsetdash{}{0pt}%
\pgfpathmoveto{\pgfqpoint{6.509011in}{4.508824in}}%
\pgfpathlineto{\pgfqpoint{6.695237in}{4.508824in}}%
\pgfpathlineto{\pgfqpoint{6.695237in}{4.590553in}}%
\pgfpathlineto{\pgfqpoint{6.509011in}{4.590553in}}%
\pgfpathlineto{\pgfqpoint{6.509011in}{4.508824in}}%
\pgfusepath{}%
\end{pgfscope}%
\begin{pgfscope}%
\pgfpathrectangle{\pgfqpoint{0.549740in}{0.463273in}}{\pgfqpoint{9.320225in}{4.495057in}}%
\pgfusepath{clip}%
\pgfsetbuttcap%
\pgfsetroundjoin%
\pgfsetlinewidth{0.000000pt}%
\definecolor{currentstroke}{rgb}{0.000000,0.000000,0.000000}%
\pgfsetstrokecolor{currentstroke}%
\pgfsetdash{}{0pt}%
\pgfpathmoveto{\pgfqpoint{6.695237in}{4.508824in}}%
\pgfpathlineto{\pgfqpoint{6.881464in}{4.508824in}}%
\pgfpathlineto{\pgfqpoint{6.881464in}{4.590553in}}%
\pgfpathlineto{\pgfqpoint{6.695237in}{4.590553in}}%
\pgfpathlineto{\pgfqpoint{6.695237in}{4.508824in}}%
\pgfusepath{}%
\end{pgfscope}%
\begin{pgfscope}%
\pgfpathrectangle{\pgfqpoint{0.549740in}{0.463273in}}{\pgfqpoint{9.320225in}{4.495057in}}%
\pgfusepath{clip}%
\pgfsetbuttcap%
\pgfsetroundjoin%
\pgfsetlinewidth{0.000000pt}%
\definecolor{currentstroke}{rgb}{0.000000,0.000000,0.000000}%
\pgfsetstrokecolor{currentstroke}%
\pgfsetdash{}{0pt}%
\pgfpathmoveto{\pgfqpoint{6.881464in}{4.508824in}}%
\pgfpathlineto{\pgfqpoint{7.067690in}{4.508824in}}%
\pgfpathlineto{\pgfqpoint{7.067690in}{4.590553in}}%
\pgfpathlineto{\pgfqpoint{6.881464in}{4.590553in}}%
\pgfpathlineto{\pgfqpoint{6.881464in}{4.508824in}}%
\pgfusepath{}%
\end{pgfscope}%
\begin{pgfscope}%
\pgfpathrectangle{\pgfqpoint{0.549740in}{0.463273in}}{\pgfqpoint{9.320225in}{4.495057in}}%
\pgfusepath{clip}%
\pgfsetbuttcap%
\pgfsetroundjoin%
\pgfsetlinewidth{0.000000pt}%
\definecolor{currentstroke}{rgb}{0.000000,0.000000,0.000000}%
\pgfsetstrokecolor{currentstroke}%
\pgfsetdash{}{0pt}%
\pgfpathmoveto{\pgfqpoint{7.067690in}{4.508824in}}%
\pgfpathlineto{\pgfqpoint{7.253917in}{4.508824in}}%
\pgfpathlineto{\pgfqpoint{7.253917in}{4.590553in}}%
\pgfpathlineto{\pgfqpoint{7.067690in}{4.590553in}}%
\pgfpathlineto{\pgfqpoint{7.067690in}{4.508824in}}%
\pgfusepath{}%
\end{pgfscope}%
\begin{pgfscope}%
\pgfpathrectangle{\pgfqpoint{0.549740in}{0.463273in}}{\pgfqpoint{9.320225in}{4.495057in}}%
\pgfusepath{clip}%
\pgfsetbuttcap%
\pgfsetroundjoin%
\pgfsetlinewidth{0.000000pt}%
\definecolor{currentstroke}{rgb}{0.000000,0.000000,0.000000}%
\pgfsetstrokecolor{currentstroke}%
\pgfsetdash{}{0pt}%
\pgfpathmoveto{\pgfqpoint{7.253917in}{4.508824in}}%
\pgfpathlineto{\pgfqpoint{7.440143in}{4.508824in}}%
\pgfpathlineto{\pgfqpoint{7.440143in}{4.590553in}}%
\pgfpathlineto{\pgfqpoint{7.253917in}{4.590553in}}%
\pgfpathlineto{\pgfqpoint{7.253917in}{4.508824in}}%
\pgfusepath{}%
\end{pgfscope}%
\begin{pgfscope}%
\pgfpathrectangle{\pgfqpoint{0.549740in}{0.463273in}}{\pgfqpoint{9.320225in}{4.495057in}}%
\pgfusepath{clip}%
\pgfsetbuttcap%
\pgfsetroundjoin%
\pgfsetlinewidth{0.000000pt}%
\definecolor{currentstroke}{rgb}{0.000000,0.000000,0.000000}%
\pgfsetstrokecolor{currentstroke}%
\pgfsetdash{}{0pt}%
\pgfpathmoveto{\pgfqpoint{7.440143in}{4.508824in}}%
\pgfpathlineto{\pgfqpoint{7.626370in}{4.508824in}}%
\pgfpathlineto{\pgfqpoint{7.626370in}{4.590553in}}%
\pgfpathlineto{\pgfqpoint{7.440143in}{4.590553in}}%
\pgfpathlineto{\pgfqpoint{7.440143in}{4.508824in}}%
\pgfusepath{}%
\end{pgfscope}%
\begin{pgfscope}%
\pgfpathrectangle{\pgfqpoint{0.549740in}{0.463273in}}{\pgfqpoint{9.320225in}{4.495057in}}%
\pgfusepath{clip}%
\pgfsetbuttcap%
\pgfsetroundjoin%
\pgfsetlinewidth{0.000000pt}%
\definecolor{currentstroke}{rgb}{0.000000,0.000000,0.000000}%
\pgfsetstrokecolor{currentstroke}%
\pgfsetdash{}{0pt}%
\pgfpathmoveto{\pgfqpoint{7.626370in}{4.508824in}}%
\pgfpathlineto{\pgfqpoint{7.812596in}{4.508824in}}%
\pgfpathlineto{\pgfqpoint{7.812596in}{4.590553in}}%
\pgfpathlineto{\pgfqpoint{7.626370in}{4.590553in}}%
\pgfpathlineto{\pgfqpoint{7.626370in}{4.508824in}}%
\pgfusepath{}%
\end{pgfscope}%
\begin{pgfscope}%
\pgfpathrectangle{\pgfqpoint{0.549740in}{0.463273in}}{\pgfqpoint{9.320225in}{4.495057in}}%
\pgfusepath{clip}%
\pgfsetbuttcap%
\pgfsetroundjoin%
\pgfsetlinewidth{0.000000pt}%
\definecolor{currentstroke}{rgb}{0.000000,0.000000,0.000000}%
\pgfsetstrokecolor{currentstroke}%
\pgfsetdash{}{0pt}%
\pgfpathmoveto{\pgfqpoint{7.812596in}{4.508824in}}%
\pgfpathlineto{\pgfqpoint{7.998823in}{4.508824in}}%
\pgfpathlineto{\pgfqpoint{7.998823in}{4.590553in}}%
\pgfpathlineto{\pgfqpoint{7.812596in}{4.590553in}}%
\pgfpathlineto{\pgfqpoint{7.812596in}{4.508824in}}%
\pgfusepath{}%
\end{pgfscope}%
\begin{pgfscope}%
\pgfpathrectangle{\pgfqpoint{0.549740in}{0.463273in}}{\pgfqpoint{9.320225in}{4.495057in}}%
\pgfusepath{clip}%
\pgfsetbuttcap%
\pgfsetroundjoin%
\pgfsetlinewidth{0.000000pt}%
\definecolor{currentstroke}{rgb}{0.000000,0.000000,0.000000}%
\pgfsetstrokecolor{currentstroke}%
\pgfsetdash{}{0pt}%
\pgfpathmoveto{\pgfqpoint{7.998823in}{4.508824in}}%
\pgfpathlineto{\pgfqpoint{8.185049in}{4.508824in}}%
\pgfpathlineto{\pgfqpoint{8.185049in}{4.590553in}}%
\pgfpathlineto{\pgfqpoint{7.998823in}{4.590553in}}%
\pgfpathlineto{\pgfqpoint{7.998823in}{4.508824in}}%
\pgfusepath{}%
\end{pgfscope}%
\begin{pgfscope}%
\pgfpathrectangle{\pgfqpoint{0.549740in}{0.463273in}}{\pgfqpoint{9.320225in}{4.495057in}}%
\pgfusepath{clip}%
\pgfsetbuttcap%
\pgfsetroundjoin%
\pgfsetlinewidth{0.000000pt}%
\definecolor{currentstroke}{rgb}{0.000000,0.000000,0.000000}%
\pgfsetstrokecolor{currentstroke}%
\pgfsetdash{}{0pt}%
\pgfpathmoveto{\pgfqpoint{8.185049in}{4.508824in}}%
\pgfpathlineto{\pgfqpoint{8.371276in}{4.508824in}}%
\pgfpathlineto{\pgfqpoint{8.371276in}{4.590553in}}%
\pgfpathlineto{\pgfqpoint{8.185049in}{4.590553in}}%
\pgfpathlineto{\pgfqpoint{8.185049in}{4.508824in}}%
\pgfusepath{}%
\end{pgfscope}%
\begin{pgfscope}%
\pgfpathrectangle{\pgfqpoint{0.549740in}{0.463273in}}{\pgfqpoint{9.320225in}{4.495057in}}%
\pgfusepath{clip}%
\pgfsetbuttcap%
\pgfsetroundjoin%
\pgfsetlinewidth{0.000000pt}%
\definecolor{currentstroke}{rgb}{0.000000,0.000000,0.000000}%
\pgfsetstrokecolor{currentstroke}%
\pgfsetdash{}{0pt}%
\pgfpathmoveto{\pgfqpoint{8.371276in}{4.508824in}}%
\pgfpathlineto{\pgfqpoint{8.557503in}{4.508824in}}%
\pgfpathlineto{\pgfqpoint{8.557503in}{4.590553in}}%
\pgfpathlineto{\pgfqpoint{8.371276in}{4.590553in}}%
\pgfpathlineto{\pgfqpoint{8.371276in}{4.508824in}}%
\pgfusepath{}%
\end{pgfscope}%
\begin{pgfscope}%
\pgfpathrectangle{\pgfqpoint{0.549740in}{0.463273in}}{\pgfqpoint{9.320225in}{4.495057in}}%
\pgfusepath{clip}%
\pgfsetbuttcap%
\pgfsetroundjoin%
\pgfsetlinewidth{0.000000pt}%
\definecolor{currentstroke}{rgb}{0.000000,0.000000,0.000000}%
\pgfsetstrokecolor{currentstroke}%
\pgfsetdash{}{0pt}%
\pgfpathmoveto{\pgfqpoint{8.557503in}{4.508824in}}%
\pgfpathlineto{\pgfqpoint{8.743729in}{4.508824in}}%
\pgfpathlineto{\pgfqpoint{8.743729in}{4.590553in}}%
\pgfpathlineto{\pgfqpoint{8.557503in}{4.590553in}}%
\pgfpathlineto{\pgfqpoint{8.557503in}{4.508824in}}%
\pgfusepath{}%
\end{pgfscope}%
\begin{pgfscope}%
\pgfpathrectangle{\pgfqpoint{0.549740in}{0.463273in}}{\pgfqpoint{9.320225in}{4.495057in}}%
\pgfusepath{clip}%
\pgfsetbuttcap%
\pgfsetroundjoin%
\pgfsetlinewidth{0.000000pt}%
\definecolor{currentstroke}{rgb}{0.000000,0.000000,0.000000}%
\pgfsetstrokecolor{currentstroke}%
\pgfsetdash{}{0pt}%
\pgfpathmoveto{\pgfqpoint{8.743729in}{4.508824in}}%
\pgfpathlineto{\pgfqpoint{8.929956in}{4.508824in}}%
\pgfpathlineto{\pgfqpoint{8.929956in}{4.590553in}}%
\pgfpathlineto{\pgfqpoint{8.743729in}{4.590553in}}%
\pgfpathlineto{\pgfqpoint{8.743729in}{4.508824in}}%
\pgfusepath{}%
\end{pgfscope}%
\begin{pgfscope}%
\pgfpathrectangle{\pgfqpoint{0.549740in}{0.463273in}}{\pgfqpoint{9.320225in}{4.495057in}}%
\pgfusepath{clip}%
\pgfsetbuttcap%
\pgfsetroundjoin%
\definecolor{currentfill}{rgb}{0.472869,0.711325,0.955316}%
\pgfsetfillcolor{currentfill}%
\pgfsetlinewidth{0.000000pt}%
\definecolor{currentstroke}{rgb}{0.000000,0.000000,0.000000}%
\pgfsetstrokecolor{currentstroke}%
\pgfsetdash{}{0pt}%
\pgfpathmoveto{\pgfqpoint{8.929956in}{4.508824in}}%
\pgfpathlineto{\pgfqpoint{9.116182in}{4.508824in}}%
\pgfpathlineto{\pgfqpoint{9.116182in}{4.590553in}}%
\pgfpathlineto{\pgfqpoint{8.929956in}{4.590553in}}%
\pgfpathlineto{\pgfqpoint{8.929956in}{4.508824in}}%
\pgfusepath{fill}%
\end{pgfscope}%
\begin{pgfscope}%
\pgfpathrectangle{\pgfqpoint{0.549740in}{0.463273in}}{\pgfqpoint{9.320225in}{4.495057in}}%
\pgfusepath{clip}%
\pgfsetbuttcap%
\pgfsetroundjoin%
\pgfsetlinewidth{0.000000pt}%
\definecolor{currentstroke}{rgb}{0.000000,0.000000,0.000000}%
\pgfsetstrokecolor{currentstroke}%
\pgfsetdash{}{0pt}%
\pgfpathmoveto{\pgfqpoint{9.116182in}{4.508824in}}%
\pgfpathlineto{\pgfqpoint{9.302409in}{4.508824in}}%
\pgfpathlineto{\pgfqpoint{9.302409in}{4.590553in}}%
\pgfpathlineto{\pgfqpoint{9.116182in}{4.590553in}}%
\pgfpathlineto{\pgfqpoint{9.116182in}{4.508824in}}%
\pgfusepath{}%
\end{pgfscope}%
\begin{pgfscope}%
\pgfpathrectangle{\pgfqpoint{0.549740in}{0.463273in}}{\pgfqpoint{9.320225in}{4.495057in}}%
\pgfusepath{clip}%
\pgfsetbuttcap%
\pgfsetroundjoin%
\pgfsetlinewidth{0.000000pt}%
\definecolor{currentstroke}{rgb}{0.000000,0.000000,0.000000}%
\pgfsetstrokecolor{currentstroke}%
\pgfsetdash{}{0pt}%
\pgfpathmoveto{\pgfqpoint{9.302409in}{4.508824in}}%
\pgfpathlineto{\pgfqpoint{9.488635in}{4.508824in}}%
\pgfpathlineto{\pgfqpoint{9.488635in}{4.590553in}}%
\pgfpathlineto{\pgfqpoint{9.302409in}{4.590553in}}%
\pgfpathlineto{\pgfqpoint{9.302409in}{4.508824in}}%
\pgfusepath{}%
\end{pgfscope}%
\begin{pgfscope}%
\pgfpathrectangle{\pgfqpoint{0.549740in}{0.463273in}}{\pgfqpoint{9.320225in}{4.495057in}}%
\pgfusepath{clip}%
\pgfsetbuttcap%
\pgfsetroundjoin%
\pgfsetlinewidth{0.000000pt}%
\definecolor{currentstroke}{rgb}{0.000000,0.000000,0.000000}%
\pgfsetstrokecolor{currentstroke}%
\pgfsetdash{}{0pt}%
\pgfpathmoveto{\pgfqpoint{9.488635in}{4.508824in}}%
\pgfpathlineto{\pgfqpoint{9.674862in}{4.508824in}}%
\pgfpathlineto{\pgfqpoint{9.674862in}{4.590553in}}%
\pgfpathlineto{\pgfqpoint{9.488635in}{4.590553in}}%
\pgfpathlineto{\pgfqpoint{9.488635in}{4.508824in}}%
\pgfusepath{}%
\end{pgfscope}%
\begin{pgfscope}%
\pgfpathrectangle{\pgfqpoint{0.549740in}{0.463273in}}{\pgfqpoint{9.320225in}{4.495057in}}%
\pgfusepath{clip}%
\pgfsetbuttcap%
\pgfsetroundjoin%
\pgfsetlinewidth{0.000000pt}%
\definecolor{currentstroke}{rgb}{0.000000,0.000000,0.000000}%
\pgfsetstrokecolor{currentstroke}%
\pgfsetdash{}{0pt}%
\pgfpathmoveto{\pgfqpoint{9.674862in}{4.508824in}}%
\pgfpathlineto{\pgfqpoint{9.861088in}{4.508824in}}%
\pgfpathlineto{\pgfqpoint{9.861088in}{4.590553in}}%
\pgfpathlineto{\pgfqpoint{9.674862in}{4.590553in}}%
\pgfpathlineto{\pgfqpoint{9.674862in}{4.508824in}}%
\pgfusepath{}%
\end{pgfscope}%
\begin{pgfscope}%
\pgfpathrectangle{\pgfqpoint{0.549740in}{0.463273in}}{\pgfqpoint{9.320225in}{4.495057in}}%
\pgfusepath{clip}%
\pgfsetbuttcap%
\pgfsetroundjoin%
\pgfsetlinewidth{0.000000pt}%
\definecolor{currentstroke}{rgb}{0.000000,0.000000,0.000000}%
\pgfsetstrokecolor{currentstroke}%
\pgfsetdash{}{0pt}%
\pgfpathmoveto{\pgfqpoint{0.549761in}{4.590553in}}%
\pgfpathlineto{\pgfqpoint{0.735988in}{4.590553in}}%
\pgfpathlineto{\pgfqpoint{0.735988in}{4.672281in}}%
\pgfpathlineto{\pgfqpoint{0.549761in}{4.672281in}}%
\pgfpathlineto{\pgfqpoint{0.549761in}{4.590553in}}%
\pgfusepath{}%
\end{pgfscope}%
\begin{pgfscope}%
\pgfpathrectangle{\pgfqpoint{0.549740in}{0.463273in}}{\pgfqpoint{9.320225in}{4.495057in}}%
\pgfusepath{clip}%
\pgfsetbuttcap%
\pgfsetroundjoin%
\pgfsetlinewidth{0.000000pt}%
\definecolor{currentstroke}{rgb}{0.000000,0.000000,0.000000}%
\pgfsetstrokecolor{currentstroke}%
\pgfsetdash{}{0pt}%
\pgfpathmoveto{\pgfqpoint{0.735988in}{4.590553in}}%
\pgfpathlineto{\pgfqpoint{0.922214in}{4.590553in}}%
\pgfpathlineto{\pgfqpoint{0.922214in}{4.672281in}}%
\pgfpathlineto{\pgfqpoint{0.735988in}{4.672281in}}%
\pgfpathlineto{\pgfqpoint{0.735988in}{4.590553in}}%
\pgfusepath{}%
\end{pgfscope}%
\begin{pgfscope}%
\pgfpathrectangle{\pgfqpoint{0.549740in}{0.463273in}}{\pgfqpoint{9.320225in}{4.495057in}}%
\pgfusepath{clip}%
\pgfsetbuttcap%
\pgfsetroundjoin%
\pgfsetlinewidth{0.000000pt}%
\definecolor{currentstroke}{rgb}{0.000000,0.000000,0.000000}%
\pgfsetstrokecolor{currentstroke}%
\pgfsetdash{}{0pt}%
\pgfpathmoveto{\pgfqpoint{0.922214in}{4.590553in}}%
\pgfpathlineto{\pgfqpoint{1.108441in}{4.590553in}}%
\pgfpathlineto{\pgfqpoint{1.108441in}{4.672281in}}%
\pgfpathlineto{\pgfqpoint{0.922214in}{4.672281in}}%
\pgfpathlineto{\pgfqpoint{0.922214in}{4.590553in}}%
\pgfusepath{}%
\end{pgfscope}%
\begin{pgfscope}%
\pgfpathrectangle{\pgfqpoint{0.549740in}{0.463273in}}{\pgfqpoint{9.320225in}{4.495057in}}%
\pgfusepath{clip}%
\pgfsetbuttcap%
\pgfsetroundjoin%
\pgfsetlinewidth{0.000000pt}%
\definecolor{currentstroke}{rgb}{0.000000,0.000000,0.000000}%
\pgfsetstrokecolor{currentstroke}%
\pgfsetdash{}{0pt}%
\pgfpathmoveto{\pgfqpoint{1.108441in}{4.590553in}}%
\pgfpathlineto{\pgfqpoint{1.294667in}{4.590553in}}%
\pgfpathlineto{\pgfqpoint{1.294667in}{4.672281in}}%
\pgfpathlineto{\pgfqpoint{1.108441in}{4.672281in}}%
\pgfpathlineto{\pgfqpoint{1.108441in}{4.590553in}}%
\pgfusepath{}%
\end{pgfscope}%
\begin{pgfscope}%
\pgfpathrectangle{\pgfqpoint{0.549740in}{0.463273in}}{\pgfqpoint{9.320225in}{4.495057in}}%
\pgfusepath{clip}%
\pgfsetbuttcap%
\pgfsetroundjoin%
\pgfsetlinewidth{0.000000pt}%
\definecolor{currentstroke}{rgb}{0.000000,0.000000,0.000000}%
\pgfsetstrokecolor{currentstroke}%
\pgfsetdash{}{0pt}%
\pgfpathmoveto{\pgfqpoint{1.294667in}{4.590553in}}%
\pgfpathlineto{\pgfqpoint{1.480894in}{4.590553in}}%
\pgfpathlineto{\pgfqpoint{1.480894in}{4.672281in}}%
\pgfpathlineto{\pgfqpoint{1.294667in}{4.672281in}}%
\pgfpathlineto{\pgfqpoint{1.294667in}{4.590553in}}%
\pgfusepath{}%
\end{pgfscope}%
\begin{pgfscope}%
\pgfpathrectangle{\pgfqpoint{0.549740in}{0.463273in}}{\pgfqpoint{9.320225in}{4.495057in}}%
\pgfusepath{clip}%
\pgfsetbuttcap%
\pgfsetroundjoin%
\pgfsetlinewidth{0.000000pt}%
\definecolor{currentstroke}{rgb}{0.000000,0.000000,0.000000}%
\pgfsetstrokecolor{currentstroke}%
\pgfsetdash{}{0pt}%
\pgfpathmoveto{\pgfqpoint{1.480894in}{4.590553in}}%
\pgfpathlineto{\pgfqpoint{1.667120in}{4.590553in}}%
\pgfpathlineto{\pgfqpoint{1.667120in}{4.672281in}}%
\pgfpathlineto{\pgfqpoint{1.480894in}{4.672281in}}%
\pgfpathlineto{\pgfqpoint{1.480894in}{4.590553in}}%
\pgfusepath{}%
\end{pgfscope}%
\begin{pgfscope}%
\pgfpathrectangle{\pgfqpoint{0.549740in}{0.463273in}}{\pgfqpoint{9.320225in}{4.495057in}}%
\pgfusepath{clip}%
\pgfsetbuttcap%
\pgfsetroundjoin%
\pgfsetlinewidth{0.000000pt}%
\definecolor{currentstroke}{rgb}{0.000000,0.000000,0.000000}%
\pgfsetstrokecolor{currentstroke}%
\pgfsetdash{}{0pt}%
\pgfpathmoveto{\pgfqpoint{1.667120in}{4.590553in}}%
\pgfpathlineto{\pgfqpoint{1.853347in}{4.590553in}}%
\pgfpathlineto{\pgfqpoint{1.853347in}{4.672281in}}%
\pgfpathlineto{\pgfqpoint{1.667120in}{4.672281in}}%
\pgfpathlineto{\pgfqpoint{1.667120in}{4.590553in}}%
\pgfusepath{}%
\end{pgfscope}%
\begin{pgfscope}%
\pgfpathrectangle{\pgfqpoint{0.549740in}{0.463273in}}{\pgfqpoint{9.320225in}{4.495057in}}%
\pgfusepath{clip}%
\pgfsetbuttcap%
\pgfsetroundjoin%
\pgfsetlinewidth{0.000000pt}%
\definecolor{currentstroke}{rgb}{0.000000,0.000000,0.000000}%
\pgfsetstrokecolor{currentstroke}%
\pgfsetdash{}{0pt}%
\pgfpathmoveto{\pgfqpoint{1.853347in}{4.590553in}}%
\pgfpathlineto{\pgfqpoint{2.039573in}{4.590553in}}%
\pgfpathlineto{\pgfqpoint{2.039573in}{4.672281in}}%
\pgfpathlineto{\pgfqpoint{1.853347in}{4.672281in}}%
\pgfpathlineto{\pgfqpoint{1.853347in}{4.590553in}}%
\pgfusepath{}%
\end{pgfscope}%
\begin{pgfscope}%
\pgfpathrectangle{\pgfqpoint{0.549740in}{0.463273in}}{\pgfqpoint{9.320225in}{4.495057in}}%
\pgfusepath{clip}%
\pgfsetbuttcap%
\pgfsetroundjoin%
\pgfsetlinewidth{0.000000pt}%
\definecolor{currentstroke}{rgb}{0.000000,0.000000,0.000000}%
\pgfsetstrokecolor{currentstroke}%
\pgfsetdash{}{0pt}%
\pgfpathmoveto{\pgfqpoint{2.039573in}{4.590553in}}%
\pgfpathlineto{\pgfqpoint{2.225800in}{4.590553in}}%
\pgfpathlineto{\pgfqpoint{2.225800in}{4.672281in}}%
\pgfpathlineto{\pgfqpoint{2.039573in}{4.672281in}}%
\pgfpathlineto{\pgfqpoint{2.039573in}{4.590553in}}%
\pgfusepath{}%
\end{pgfscope}%
\begin{pgfscope}%
\pgfpathrectangle{\pgfqpoint{0.549740in}{0.463273in}}{\pgfqpoint{9.320225in}{4.495057in}}%
\pgfusepath{clip}%
\pgfsetbuttcap%
\pgfsetroundjoin%
\pgfsetlinewidth{0.000000pt}%
\definecolor{currentstroke}{rgb}{0.000000,0.000000,0.000000}%
\pgfsetstrokecolor{currentstroke}%
\pgfsetdash{}{0pt}%
\pgfpathmoveto{\pgfqpoint{2.225800in}{4.590553in}}%
\pgfpathlineto{\pgfqpoint{2.412027in}{4.590553in}}%
\pgfpathlineto{\pgfqpoint{2.412027in}{4.672281in}}%
\pgfpathlineto{\pgfqpoint{2.225800in}{4.672281in}}%
\pgfpathlineto{\pgfqpoint{2.225800in}{4.590553in}}%
\pgfusepath{}%
\end{pgfscope}%
\begin{pgfscope}%
\pgfpathrectangle{\pgfqpoint{0.549740in}{0.463273in}}{\pgfqpoint{9.320225in}{4.495057in}}%
\pgfusepath{clip}%
\pgfsetbuttcap%
\pgfsetroundjoin%
\pgfsetlinewidth{0.000000pt}%
\definecolor{currentstroke}{rgb}{0.000000,0.000000,0.000000}%
\pgfsetstrokecolor{currentstroke}%
\pgfsetdash{}{0pt}%
\pgfpathmoveto{\pgfqpoint{2.412027in}{4.590553in}}%
\pgfpathlineto{\pgfqpoint{2.598253in}{4.590553in}}%
\pgfpathlineto{\pgfqpoint{2.598253in}{4.672281in}}%
\pgfpathlineto{\pgfqpoint{2.412027in}{4.672281in}}%
\pgfpathlineto{\pgfqpoint{2.412027in}{4.590553in}}%
\pgfusepath{}%
\end{pgfscope}%
\begin{pgfscope}%
\pgfpathrectangle{\pgfqpoint{0.549740in}{0.463273in}}{\pgfqpoint{9.320225in}{4.495057in}}%
\pgfusepath{clip}%
\pgfsetbuttcap%
\pgfsetroundjoin%
\pgfsetlinewidth{0.000000pt}%
\definecolor{currentstroke}{rgb}{0.000000,0.000000,0.000000}%
\pgfsetstrokecolor{currentstroke}%
\pgfsetdash{}{0pt}%
\pgfpathmoveto{\pgfqpoint{2.598253in}{4.590553in}}%
\pgfpathlineto{\pgfqpoint{2.784480in}{4.590553in}}%
\pgfpathlineto{\pgfqpoint{2.784480in}{4.672281in}}%
\pgfpathlineto{\pgfqpoint{2.598253in}{4.672281in}}%
\pgfpathlineto{\pgfqpoint{2.598253in}{4.590553in}}%
\pgfusepath{}%
\end{pgfscope}%
\begin{pgfscope}%
\pgfpathrectangle{\pgfqpoint{0.549740in}{0.463273in}}{\pgfqpoint{9.320225in}{4.495057in}}%
\pgfusepath{clip}%
\pgfsetbuttcap%
\pgfsetroundjoin%
\pgfsetlinewidth{0.000000pt}%
\definecolor{currentstroke}{rgb}{0.000000,0.000000,0.000000}%
\pgfsetstrokecolor{currentstroke}%
\pgfsetdash{}{0pt}%
\pgfpathmoveto{\pgfqpoint{2.784480in}{4.590553in}}%
\pgfpathlineto{\pgfqpoint{2.970706in}{4.590553in}}%
\pgfpathlineto{\pgfqpoint{2.970706in}{4.672281in}}%
\pgfpathlineto{\pgfqpoint{2.784480in}{4.672281in}}%
\pgfpathlineto{\pgfqpoint{2.784480in}{4.590553in}}%
\pgfusepath{}%
\end{pgfscope}%
\begin{pgfscope}%
\pgfpathrectangle{\pgfqpoint{0.549740in}{0.463273in}}{\pgfqpoint{9.320225in}{4.495057in}}%
\pgfusepath{clip}%
\pgfsetbuttcap%
\pgfsetroundjoin%
\pgfsetlinewidth{0.000000pt}%
\definecolor{currentstroke}{rgb}{0.000000,0.000000,0.000000}%
\pgfsetstrokecolor{currentstroke}%
\pgfsetdash{}{0pt}%
\pgfpathmoveto{\pgfqpoint{2.970706in}{4.590553in}}%
\pgfpathlineto{\pgfqpoint{3.156933in}{4.590553in}}%
\pgfpathlineto{\pgfqpoint{3.156933in}{4.672281in}}%
\pgfpathlineto{\pgfqpoint{2.970706in}{4.672281in}}%
\pgfpathlineto{\pgfqpoint{2.970706in}{4.590553in}}%
\pgfusepath{}%
\end{pgfscope}%
\begin{pgfscope}%
\pgfpathrectangle{\pgfqpoint{0.549740in}{0.463273in}}{\pgfqpoint{9.320225in}{4.495057in}}%
\pgfusepath{clip}%
\pgfsetbuttcap%
\pgfsetroundjoin%
\pgfsetlinewidth{0.000000pt}%
\definecolor{currentstroke}{rgb}{0.000000,0.000000,0.000000}%
\pgfsetstrokecolor{currentstroke}%
\pgfsetdash{}{0pt}%
\pgfpathmoveto{\pgfqpoint{3.156933in}{4.590553in}}%
\pgfpathlineto{\pgfqpoint{3.343159in}{4.590553in}}%
\pgfpathlineto{\pgfqpoint{3.343159in}{4.672281in}}%
\pgfpathlineto{\pgfqpoint{3.156933in}{4.672281in}}%
\pgfpathlineto{\pgfqpoint{3.156933in}{4.590553in}}%
\pgfusepath{}%
\end{pgfscope}%
\begin{pgfscope}%
\pgfpathrectangle{\pgfqpoint{0.549740in}{0.463273in}}{\pgfqpoint{9.320225in}{4.495057in}}%
\pgfusepath{clip}%
\pgfsetbuttcap%
\pgfsetroundjoin%
\pgfsetlinewidth{0.000000pt}%
\definecolor{currentstroke}{rgb}{0.000000,0.000000,0.000000}%
\pgfsetstrokecolor{currentstroke}%
\pgfsetdash{}{0pt}%
\pgfpathmoveto{\pgfqpoint{3.343159in}{4.590553in}}%
\pgfpathlineto{\pgfqpoint{3.529386in}{4.590553in}}%
\pgfpathlineto{\pgfqpoint{3.529386in}{4.672281in}}%
\pgfpathlineto{\pgfqpoint{3.343159in}{4.672281in}}%
\pgfpathlineto{\pgfqpoint{3.343159in}{4.590553in}}%
\pgfusepath{}%
\end{pgfscope}%
\begin{pgfscope}%
\pgfpathrectangle{\pgfqpoint{0.549740in}{0.463273in}}{\pgfqpoint{9.320225in}{4.495057in}}%
\pgfusepath{clip}%
\pgfsetbuttcap%
\pgfsetroundjoin%
\pgfsetlinewidth{0.000000pt}%
\definecolor{currentstroke}{rgb}{0.000000,0.000000,0.000000}%
\pgfsetstrokecolor{currentstroke}%
\pgfsetdash{}{0pt}%
\pgfpathmoveto{\pgfqpoint{3.529386in}{4.590553in}}%
\pgfpathlineto{\pgfqpoint{3.715612in}{4.590553in}}%
\pgfpathlineto{\pgfqpoint{3.715612in}{4.672281in}}%
\pgfpathlineto{\pgfqpoint{3.529386in}{4.672281in}}%
\pgfpathlineto{\pgfqpoint{3.529386in}{4.590553in}}%
\pgfusepath{}%
\end{pgfscope}%
\begin{pgfscope}%
\pgfpathrectangle{\pgfqpoint{0.549740in}{0.463273in}}{\pgfqpoint{9.320225in}{4.495057in}}%
\pgfusepath{clip}%
\pgfsetbuttcap%
\pgfsetroundjoin%
\pgfsetlinewidth{0.000000pt}%
\definecolor{currentstroke}{rgb}{0.000000,0.000000,0.000000}%
\pgfsetstrokecolor{currentstroke}%
\pgfsetdash{}{0pt}%
\pgfpathmoveto{\pgfqpoint{3.715612in}{4.590553in}}%
\pgfpathlineto{\pgfqpoint{3.901839in}{4.590553in}}%
\pgfpathlineto{\pgfqpoint{3.901839in}{4.672281in}}%
\pgfpathlineto{\pgfqpoint{3.715612in}{4.672281in}}%
\pgfpathlineto{\pgfqpoint{3.715612in}{4.590553in}}%
\pgfusepath{}%
\end{pgfscope}%
\begin{pgfscope}%
\pgfpathrectangle{\pgfqpoint{0.549740in}{0.463273in}}{\pgfqpoint{9.320225in}{4.495057in}}%
\pgfusepath{clip}%
\pgfsetbuttcap%
\pgfsetroundjoin%
\pgfsetlinewidth{0.000000pt}%
\definecolor{currentstroke}{rgb}{0.000000,0.000000,0.000000}%
\pgfsetstrokecolor{currentstroke}%
\pgfsetdash{}{0pt}%
\pgfpathmoveto{\pgfqpoint{3.901839in}{4.590553in}}%
\pgfpathlineto{\pgfqpoint{4.088065in}{4.590553in}}%
\pgfpathlineto{\pgfqpoint{4.088065in}{4.672281in}}%
\pgfpathlineto{\pgfqpoint{3.901839in}{4.672281in}}%
\pgfpathlineto{\pgfqpoint{3.901839in}{4.590553in}}%
\pgfusepath{}%
\end{pgfscope}%
\begin{pgfscope}%
\pgfpathrectangle{\pgfqpoint{0.549740in}{0.463273in}}{\pgfqpoint{9.320225in}{4.495057in}}%
\pgfusepath{clip}%
\pgfsetbuttcap%
\pgfsetroundjoin%
\pgfsetlinewidth{0.000000pt}%
\definecolor{currentstroke}{rgb}{0.000000,0.000000,0.000000}%
\pgfsetstrokecolor{currentstroke}%
\pgfsetdash{}{0pt}%
\pgfpathmoveto{\pgfqpoint{4.088065in}{4.590553in}}%
\pgfpathlineto{\pgfqpoint{4.274292in}{4.590553in}}%
\pgfpathlineto{\pgfqpoint{4.274292in}{4.672281in}}%
\pgfpathlineto{\pgfqpoint{4.088065in}{4.672281in}}%
\pgfpathlineto{\pgfqpoint{4.088065in}{4.590553in}}%
\pgfusepath{}%
\end{pgfscope}%
\begin{pgfscope}%
\pgfpathrectangle{\pgfqpoint{0.549740in}{0.463273in}}{\pgfqpoint{9.320225in}{4.495057in}}%
\pgfusepath{clip}%
\pgfsetbuttcap%
\pgfsetroundjoin%
\pgfsetlinewidth{0.000000pt}%
\definecolor{currentstroke}{rgb}{0.000000,0.000000,0.000000}%
\pgfsetstrokecolor{currentstroke}%
\pgfsetdash{}{0pt}%
\pgfpathmoveto{\pgfqpoint{4.274292in}{4.590553in}}%
\pgfpathlineto{\pgfqpoint{4.460519in}{4.590553in}}%
\pgfpathlineto{\pgfqpoint{4.460519in}{4.672281in}}%
\pgfpathlineto{\pgfqpoint{4.274292in}{4.672281in}}%
\pgfpathlineto{\pgfqpoint{4.274292in}{4.590553in}}%
\pgfusepath{}%
\end{pgfscope}%
\begin{pgfscope}%
\pgfpathrectangle{\pgfqpoint{0.549740in}{0.463273in}}{\pgfqpoint{9.320225in}{4.495057in}}%
\pgfusepath{clip}%
\pgfsetbuttcap%
\pgfsetroundjoin%
\pgfsetlinewidth{0.000000pt}%
\definecolor{currentstroke}{rgb}{0.000000,0.000000,0.000000}%
\pgfsetstrokecolor{currentstroke}%
\pgfsetdash{}{0pt}%
\pgfpathmoveto{\pgfqpoint{4.460519in}{4.590553in}}%
\pgfpathlineto{\pgfqpoint{4.646745in}{4.590553in}}%
\pgfpathlineto{\pgfqpoint{4.646745in}{4.672281in}}%
\pgfpathlineto{\pgfqpoint{4.460519in}{4.672281in}}%
\pgfpathlineto{\pgfqpoint{4.460519in}{4.590553in}}%
\pgfusepath{}%
\end{pgfscope}%
\begin{pgfscope}%
\pgfpathrectangle{\pgfqpoint{0.549740in}{0.463273in}}{\pgfqpoint{9.320225in}{4.495057in}}%
\pgfusepath{clip}%
\pgfsetbuttcap%
\pgfsetroundjoin%
\pgfsetlinewidth{0.000000pt}%
\definecolor{currentstroke}{rgb}{0.000000,0.000000,0.000000}%
\pgfsetstrokecolor{currentstroke}%
\pgfsetdash{}{0pt}%
\pgfpathmoveto{\pgfqpoint{4.646745in}{4.590553in}}%
\pgfpathlineto{\pgfqpoint{4.832972in}{4.590553in}}%
\pgfpathlineto{\pgfqpoint{4.832972in}{4.672281in}}%
\pgfpathlineto{\pgfqpoint{4.646745in}{4.672281in}}%
\pgfpathlineto{\pgfqpoint{4.646745in}{4.590553in}}%
\pgfusepath{}%
\end{pgfscope}%
\begin{pgfscope}%
\pgfpathrectangle{\pgfqpoint{0.549740in}{0.463273in}}{\pgfqpoint{9.320225in}{4.495057in}}%
\pgfusepath{clip}%
\pgfsetbuttcap%
\pgfsetroundjoin%
\pgfsetlinewidth{0.000000pt}%
\definecolor{currentstroke}{rgb}{0.000000,0.000000,0.000000}%
\pgfsetstrokecolor{currentstroke}%
\pgfsetdash{}{0pt}%
\pgfpathmoveto{\pgfqpoint{4.832972in}{4.590553in}}%
\pgfpathlineto{\pgfqpoint{5.019198in}{4.590553in}}%
\pgfpathlineto{\pgfqpoint{5.019198in}{4.672281in}}%
\pgfpathlineto{\pgfqpoint{4.832972in}{4.672281in}}%
\pgfpathlineto{\pgfqpoint{4.832972in}{4.590553in}}%
\pgfusepath{}%
\end{pgfscope}%
\begin{pgfscope}%
\pgfpathrectangle{\pgfqpoint{0.549740in}{0.463273in}}{\pgfqpoint{9.320225in}{4.495057in}}%
\pgfusepath{clip}%
\pgfsetbuttcap%
\pgfsetroundjoin%
\pgfsetlinewidth{0.000000pt}%
\definecolor{currentstroke}{rgb}{0.000000,0.000000,0.000000}%
\pgfsetstrokecolor{currentstroke}%
\pgfsetdash{}{0pt}%
\pgfpathmoveto{\pgfqpoint{5.019198in}{4.590553in}}%
\pgfpathlineto{\pgfqpoint{5.205425in}{4.590553in}}%
\pgfpathlineto{\pgfqpoint{5.205425in}{4.672281in}}%
\pgfpathlineto{\pgfqpoint{5.019198in}{4.672281in}}%
\pgfpathlineto{\pgfqpoint{5.019198in}{4.590553in}}%
\pgfusepath{}%
\end{pgfscope}%
\begin{pgfscope}%
\pgfpathrectangle{\pgfqpoint{0.549740in}{0.463273in}}{\pgfqpoint{9.320225in}{4.495057in}}%
\pgfusepath{clip}%
\pgfsetbuttcap%
\pgfsetroundjoin%
\pgfsetlinewidth{0.000000pt}%
\definecolor{currentstroke}{rgb}{0.000000,0.000000,0.000000}%
\pgfsetstrokecolor{currentstroke}%
\pgfsetdash{}{0pt}%
\pgfpathmoveto{\pgfqpoint{5.205425in}{4.590553in}}%
\pgfpathlineto{\pgfqpoint{5.391651in}{4.590553in}}%
\pgfpathlineto{\pgfqpoint{5.391651in}{4.672281in}}%
\pgfpathlineto{\pgfqpoint{5.205425in}{4.672281in}}%
\pgfpathlineto{\pgfqpoint{5.205425in}{4.590553in}}%
\pgfusepath{}%
\end{pgfscope}%
\begin{pgfscope}%
\pgfpathrectangle{\pgfqpoint{0.549740in}{0.463273in}}{\pgfqpoint{9.320225in}{4.495057in}}%
\pgfusepath{clip}%
\pgfsetbuttcap%
\pgfsetroundjoin%
\pgfsetlinewidth{0.000000pt}%
\definecolor{currentstroke}{rgb}{0.000000,0.000000,0.000000}%
\pgfsetstrokecolor{currentstroke}%
\pgfsetdash{}{0pt}%
\pgfpathmoveto{\pgfqpoint{5.391651in}{4.590553in}}%
\pgfpathlineto{\pgfqpoint{5.577878in}{4.590553in}}%
\pgfpathlineto{\pgfqpoint{5.577878in}{4.672281in}}%
\pgfpathlineto{\pgfqpoint{5.391651in}{4.672281in}}%
\pgfpathlineto{\pgfqpoint{5.391651in}{4.590553in}}%
\pgfusepath{}%
\end{pgfscope}%
\begin{pgfscope}%
\pgfpathrectangle{\pgfqpoint{0.549740in}{0.463273in}}{\pgfqpoint{9.320225in}{4.495057in}}%
\pgfusepath{clip}%
\pgfsetbuttcap%
\pgfsetroundjoin%
\pgfsetlinewidth{0.000000pt}%
\definecolor{currentstroke}{rgb}{0.000000,0.000000,0.000000}%
\pgfsetstrokecolor{currentstroke}%
\pgfsetdash{}{0pt}%
\pgfpathmoveto{\pgfqpoint{5.577878in}{4.590553in}}%
\pgfpathlineto{\pgfqpoint{5.764104in}{4.590553in}}%
\pgfpathlineto{\pgfqpoint{5.764104in}{4.672281in}}%
\pgfpathlineto{\pgfqpoint{5.577878in}{4.672281in}}%
\pgfpathlineto{\pgfqpoint{5.577878in}{4.590553in}}%
\pgfusepath{}%
\end{pgfscope}%
\begin{pgfscope}%
\pgfpathrectangle{\pgfqpoint{0.549740in}{0.463273in}}{\pgfqpoint{9.320225in}{4.495057in}}%
\pgfusepath{clip}%
\pgfsetbuttcap%
\pgfsetroundjoin%
\pgfsetlinewidth{0.000000pt}%
\definecolor{currentstroke}{rgb}{0.000000,0.000000,0.000000}%
\pgfsetstrokecolor{currentstroke}%
\pgfsetdash{}{0pt}%
\pgfpathmoveto{\pgfqpoint{5.764104in}{4.590553in}}%
\pgfpathlineto{\pgfqpoint{5.950331in}{4.590553in}}%
\pgfpathlineto{\pgfqpoint{5.950331in}{4.672281in}}%
\pgfpathlineto{\pgfqpoint{5.764104in}{4.672281in}}%
\pgfpathlineto{\pgfqpoint{5.764104in}{4.590553in}}%
\pgfusepath{}%
\end{pgfscope}%
\begin{pgfscope}%
\pgfpathrectangle{\pgfqpoint{0.549740in}{0.463273in}}{\pgfqpoint{9.320225in}{4.495057in}}%
\pgfusepath{clip}%
\pgfsetbuttcap%
\pgfsetroundjoin%
\pgfsetlinewidth{0.000000pt}%
\definecolor{currentstroke}{rgb}{0.000000,0.000000,0.000000}%
\pgfsetstrokecolor{currentstroke}%
\pgfsetdash{}{0pt}%
\pgfpathmoveto{\pgfqpoint{5.950331in}{4.590553in}}%
\pgfpathlineto{\pgfqpoint{6.136557in}{4.590553in}}%
\pgfpathlineto{\pgfqpoint{6.136557in}{4.672281in}}%
\pgfpathlineto{\pgfqpoint{5.950331in}{4.672281in}}%
\pgfpathlineto{\pgfqpoint{5.950331in}{4.590553in}}%
\pgfusepath{}%
\end{pgfscope}%
\begin{pgfscope}%
\pgfpathrectangle{\pgfqpoint{0.549740in}{0.463273in}}{\pgfqpoint{9.320225in}{4.495057in}}%
\pgfusepath{clip}%
\pgfsetbuttcap%
\pgfsetroundjoin%
\pgfsetlinewidth{0.000000pt}%
\definecolor{currentstroke}{rgb}{0.000000,0.000000,0.000000}%
\pgfsetstrokecolor{currentstroke}%
\pgfsetdash{}{0pt}%
\pgfpathmoveto{\pgfqpoint{6.136557in}{4.590553in}}%
\pgfpathlineto{\pgfqpoint{6.322784in}{4.590553in}}%
\pgfpathlineto{\pgfqpoint{6.322784in}{4.672281in}}%
\pgfpathlineto{\pgfqpoint{6.136557in}{4.672281in}}%
\pgfpathlineto{\pgfqpoint{6.136557in}{4.590553in}}%
\pgfusepath{}%
\end{pgfscope}%
\begin{pgfscope}%
\pgfpathrectangle{\pgfqpoint{0.549740in}{0.463273in}}{\pgfqpoint{9.320225in}{4.495057in}}%
\pgfusepath{clip}%
\pgfsetbuttcap%
\pgfsetroundjoin%
\pgfsetlinewidth{0.000000pt}%
\definecolor{currentstroke}{rgb}{0.000000,0.000000,0.000000}%
\pgfsetstrokecolor{currentstroke}%
\pgfsetdash{}{0pt}%
\pgfpathmoveto{\pgfqpoint{6.322784in}{4.590553in}}%
\pgfpathlineto{\pgfqpoint{6.509011in}{4.590553in}}%
\pgfpathlineto{\pgfqpoint{6.509011in}{4.672281in}}%
\pgfpathlineto{\pgfqpoint{6.322784in}{4.672281in}}%
\pgfpathlineto{\pgfqpoint{6.322784in}{4.590553in}}%
\pgfusepath{}%
\end{pgfscope}%
\begin{pgfscope}%
\pgfpathrectangle{\pgfqpoint{0.549740in}{0.463273in}}{\pgfqpoint{9.320225in}{4.495057in}}%
\pgfusepath{clip}%
\pgfsetbuttcap%
\pgfsetroundjoin%
\pgfsetlinewidth{0.000000pt}%
\definecolor{currentstroke}{rgb}{0.000000,0.000000,0.000000}%
\pgfsetstrokecolor{currentstroke}%
\pgfsetdash{}{0pt}%
\pgfpathmoveto{\pgfqpoint{6.509011in}{4.590553in}}%
\pgfpathlineto{\pgfqpoint{6.695237in}{4.590553in}}%
\pgfpathlineto{\pgfqpoint{6.695237in}{4.672281in}}%
\pgfpathlineto{\pgfqpoint{6.509011in}{4.672281in}}%
\pgfpathlineto{\pgfqpoint{6.509011in}{4.590553in}}%
\pgfusepath{}%
\end{pgfscope}%
\begin{pgfscope}%
\pgfpathrectangle{\pgfqpoint{0.549740in}{0.463273in}}{\pgfqpoint{9.320225in}{4.495057in}}%
\pgfusepath{clip}%
\pgfsetbuttcap%
\pgfsetroundjoin%
\pgfsetlinewidth{0.000000pt}%
\definecolor{currentstroke}{rgb}{0.000000,0.000000,0.000000}%
\pgfsetstrokecolor{currentstroke}%
\pgfsetdash{}{0pt}%
\pgfpathmoveto{\pgfqpoint{6.695237in}{4.590553in}}%
\pgfpathlineto{\pgfqpoint{6.881464in}{4.590553in}}%
\pgfpathlineto{\pgfqpoint{6.881464in}{4.672281in}}%
\pgfpathlineto{\pgfqpoint{6.695237in}{4.672281in}}%
\pgfpathlineto{\pgfqpoint{6.695237in}{4.590553in}}%
\pgfusepath{}%
\end{pgfscope}%
\begin{pgfscope}%
\pgfpathrectangle{\pgfqpoint{0.549740in}{0.463273in}}{\pgfqpoint{9.320225in}{4.495057in}}%
\pgfusepath{clip}%
\pgfsetbuttcap%
\pgfsetroundjoin%
\pgfsetlinewidth{0.000000pt}%
\definecolor{currentstroke}{rgb}{0.000000,0.000000,0.000000}%
\pgfsetstrokecolor{currentstroke}%
\pgfsetdash{}{0pt}%
\pgfpathmoveto{\pgfqpoint{6.881464in}{4.590553in}}%
\pgfpathlineto{\pgfqpoint{7.067690in}{4.590553in}}%
\pgfpathlineto{\pgfqpoint{7.067690in}{4.672281in}}%
\pgfpathlineto{\pgfqpoint{6.881464in}{4.672281in}}%
\pgfpathlineto{\pgfqpoint{6.881464in}{4.590553in}}%
\pgfusepath{}%
\end{pgfscope}%
\begin{pgfscope}%
\pgfpathrectangle{\pgfqpoint{0.549740in}{0.463273in}}{\pgfqpoint{9.320225in}{4.495057in}}%
\pgfusepath{clip}%
\pgfsetbuttcap%
\pgfsetroundjoin%
\pgfsetlinewidth{0.000000pt}%
\definecolor{currentstroke}{rgb}{0.000000,0.000000,0.000000}%
\pgfsetstrokecolor{currentstroke}%
\pgfsetdash{}{0pt}%
\pgfpathmoveto{\pgfqpoint{7.067690in}{4.590553in}}%
\pgfpathlineto{\pgfqpoint{7.253917in}{4.590553in}}%
\pgfpathlineto{\pgfqpoint{7.253917in}{4.672281in}}%
\pgfpathlineto{\pgfqpoint{7.067690in}{4.672281in}}%
\pgfpathlineto{\pgfqpoint{7.067690in}{4.590553in}}%
\pgfusepath{}%
\end{pgfscope}%
\begin{pgfscope}%
\pgfpathrectangle{\pgfqpoint{0.549740in}{0.463273in}}{\pgfqpoint{9.320225in}{4.495057in}}%
\pgfusepath{clip}%
\pgfsetbuttcap%
\pgfsetroundjoin%
\pgfsetlinewidth{0.000000pt}%
\definecolor{currentstroke}{rgb}{0.000000,0.000000,0.000000}%
\pgfsetstrokecolor{currentstroke}%
\pgfsetdash{}{0pt}%
\pgfpathmoveto{\pgfqpoint{7.253917in}{4.590553in}}%
\pgfpathlineto{\pgfqpoint{7.440143in}{4.590553in}}%
\pgfpathlineto{\pgfqpoint{7.440143in}{4.672281in}}%
\pgfpathlineto{\pgfqpoint{7.253917in}{4.672281in}}%
\pgfpathlineto{\pgfqpoint{7.253917in}{4.590553in}}%
\pgfusepath{}%
\end{pgfscope}%
\begin{pgfscope}%
\pgfpathrectangle{\pgfqpoint{0.549740in}{0.463273in}}{\pgfqpoint{9.320225in}{4.495057in}}%
\pgfusepath{clip}%
\pgfsetbuttcap%
\pgfsetroundjoin%
\pgfsetlinewidth{0.000000pt}%
\definecolor{currentstroke}{rgb}{0.000000,0.000000,0.000000}%
\pgfsetstrokecolor{currentstroke}%
\pgfsetdash{}{0pt}%
\pgfpathmoveto{\pgfqpoint{7.440143in}{4.590553in}}%
\pgfpathlineto{\pgfqpoint{7.626370in}{4.590553in}}%
\pgfpathlineto{\pgfqpoint{7.626370in}{4.672281in}}%
\pgfpathlineto{\pgfqpoint{7.440143in}{4.672281in}}%
\pgfpathlineto{\pgfqpoint{7.440143in}{4.590553in}}%
\pgfusepath{}%
\end{pgfscope}%
\begin{pgfscope}%
\pgfpathrectangle{\pgfqpoint{0.549740in}{0.463273in}}{\pgfqpoint{9.320225in}{4.495057in}}%
\pgfusepath{clip}%
\pgfsetbuttcap%
\pgfsetroundjoin%
\pgfsetlinewidth{0.000000pt}%
\definecolor{currentstroke}{rgb}{0.000000,0.000000,0.000000}%
\pgfsetstrokecolor{currentstroke}%
\pgfsetdash{}{0pt}%
\pgfpathmoveto{\pgfqpoint{7.626370in}{4.590553in}}%
\pgfpathlineto{\pgfqpoint{7.812596in}{4.590553in}}%
\pgfpathlineto{\pgfqpoint{7.812596in}{4.672281in}}%
\pgfpathlineto{\pgfqpoint{7.626370in}{4.672281in}}%
\pgfpathlineto{\pgfqpoint{7.626370in}{4.590553in}}%
\pgfusepath{}%
\end{pgfscope}%
\begin{pgfscope}%
\pgfpathrectangle{\pgfqpoint{0.549740in}{0.463273in}}{\pgfqpoint{9.320225in}{4.495057in}}%
\pgfusepath{clip}%
\pgfsetbuttcap%
\pgfsetroundjoin%
\pgfsetlinewidth{0.000000pt}%
\definecolor{currentstroke}{rgb}{0.000000,0.000000,0.000000}%
\pgfsetstrokecolor{currentstroke}%
\pgfsetdash{}{0pt}%
\pgfpathmoveto{\pgfqpoint{7.812596in}{4.590553in}}%
\pgfpathlineto{\pgfqpoint{7.998823in}{4.590553in}}%
\pgfpathlineto{\pgfqpoint{7.998823in}{4.672281in}}%
\pgfpathlineto{\pgfqpoint{7.812596in}{4.672281in}}%
\pgfpathlineto{\pgfqpoint{7.812596in}{4.590553in}}%
\pgfusepath{}%
\end{pgfscope}%
\begin{pgfscope}%
\pgfpathrectangle{\pgfqpoint{0.549740in}{0.463273in}}{\pgfqpoint{9.320225in}{4.495057in}}%
\pgfusepath{clip}%
\pgfsetbuttcap%
\pgfsetroundjoin%
\pgfsetlinewidth{0.000000pt}%
\definecolor{currentstroke}{rgb}{0.000000,0.000000,0.000000}%
\pgfsetstrokecolor{currentstroke}%
\pgfsetdash{}{0pt}%
\pgfpathmoveto{\pgfqpoint{7.998823in}{4.590553in}}%
\pgfpathlineto{\pgfqpoint{8.185049in}{4.590553in}}%
\pgfpathlineto{\pgfqpoint{8.185049in}{4.672281in}}%
\pgfpathlineto{\pgfqpoint{7.998823in}{4.672281in}}%
\pgfpathlineto{\pgfqpoint{7.998823in}{4.590553in}}%
\pgfusepath{}%
\end{pgfscope}%
\begin{pgfscope}%
\pgfpathrectangle{\pgfqpoint{0.549740in}{0.463273in}}{\pgfqpoint{9.320225in}{4.495057in}}%
\pgfusepath{clip}%
\pgfsetbuttcap%
\pgfsetroundjoin%
\pgfsetlinewidth{0.000000pt}%
\definecolor{currentstroke}{rgb}{0.000000,0.000000,0.000000}%
\pgfsetstrokecolor{currentstroke}%
\pgfsetdash{}{0pt}%
\pgfpathmoveto{\pgfqpoint{8.185049in}{4.590553in}}%
\pgfpathlineto{\pgfqpoint{8.371276in}{4.590553in}}%
\pgfpathlineto{\pgfqpoint{8.371276in}{4.672281in}}%
\pgfpathlineto{\pgfqpoint{8.185049in}{4.672281in}}%
\pgfpathlineto{\pgfqpoint{8.185049in}{4.590553in}}%
\pgfusepath{}%
\end{pgfscope}%
\begin{pgfscope}%
\pgfpathrectangle{\pgfqpoint{0.549740in}{0.463273in}}{\pgfqpoint{9.320225in}{4.495057in}}%
\pgfusepath{clip}%
\pgfsetbuttcap%
\pgfsetroundjoin%
\pgfsetlinewidth{0.000000pt}%
\definecolor{currentstroke}{rgb}{0.000000,0.000000,0.000000}%
\pgfsetstrokecolor{currentstroke}%
\pgfsetdash{}{0pt}%
\pgfpathmoveto{\pgfqpoint{8.371276in}{4.590553in}}%
\pgfpathlineto{\pgfqpoint{8.557503in}{4.590553in}}%
\pgfpathlineto{\pgfqpoint{8.557503in}{4.672281in}}%
\pgfpathlineto{\pgfqpoint{8.371276in}{4.672281in}}%
\pgfpathlineto{\pgfqpoint{8.371276in}{4.590553in}}%
\pgfusepath{}%
\end{pgfscope}%
\begin{pgfscope}%
\pgfpathrectangle{\pgfqpoint{0.549740in}{0.463273in}}{\pgfqpoint{9.320225in}{4.495057in}}%
\pgfusepath{clip}%
\pgfsetbuttcap%
\pgfsetroundjoin%
\pgfsetlinewidth{0.000000pt}%
\definecolor{currentstroke}{rgb}{0.000000,0.000000,0.000000}%
\pgfsetstrokecolor{currentstroke}%
\pgfsetdash{}{0pt}%
\pgfpathmoveto{\pgfqpoint{8.557503in}{4.590553in}}%
\pgfpathlineto{\pgfqpoint{8.743729in}{4.590553in}}%
\pgfpathlineto{\pgfqpoint{8.743729in}{4.672281in}}%
\pgfpathlineto{\pgfqpoint{8.557503in}{4.672281in}}%
\pgfpathlineto{\pgfqpoint{8.557503in}{4.590553in}}%
\pgfusepath{}%
\end{pgfscope}%
\begin{pgfscope}%
\pgfpathrectangle{\pgfqpoint{0.549740in}{0.463273in}}{\pgfqpoint{9.320225in}{4.495057in}}%
\pgfusepath{clip}%
\pgfsetbuttcap%
\pgfsetroundjoin%
\pgfsetlinewidth{0.000000pt}%
\definecolor{currentstroke}{rgb}{0.000000,0.000000,0.000000}%
\pgfsetstrokecolor{currentstroke}%
\pgfsetdash{}{0pt}%
\pgfpathmoveto{\pgfqpoint{8.743729in}{4.590553in}}%
\pgfpathlineto{\pgfqpoint{8.929956in}{4.590553in}}%
\pgfpathlineto{\pgfqpoint{8.929956in}{4.672281in}}%
\pgfpathlineto{\pgfqpoint{8.743729in}{4.672281in}}%
\pgfpathlineto{\pgfqpoint{8.743729in}{4.590553in}}%
\pgfusepath{}%
\end{pgfscope}%
\begin{pgfscope}%
\pgfpathrectangle{\pgfqpoint{0.549740in}{0.463273in}}{\pgfqpoint{9.320225in}{4.495057in}}%
\pgfusepath{clip}%
\pgfsetbuttcap%
\pgfsetroundjoin%
\definecolor{currentfill}{rgb}{0.472869,0.711325,0.955316}%
\pgfsetfillcolor{currentfill}%
\pgfsetlinewidth{0.000000pt}%
\definecolor{currentstroke}{rgb}{0.000000,0.000000,0.000000}%
\pgfsetstrokecolor{currentstroke}%
\pgfsetdash{}{0pt}%
\pgfpathmoveto{\pgfqpoint{8.929956in}{4.590553in}}%
\pgfpathlineto{\pgfqpoint{9.116182in}{4.590553in}}%
\pgfpathlineto{\pgfqpoint{9.116182in}{4.672281in}}%
\pgfpathlineto{\pgfqpoint{8.929956in}{4.672281in}}%
\pgfpathlineto{\pgfqpoint{8.929956in}{4.590553in}}%
\pgfusepath{fill}%
\end{pgfscope}%
\begin{pgfscope}%
\pgfpathrectangle{\pgfqpoint{0.549740in}{0.463273in}}{\pgfqpoint{9.320225in}{4.495057in}}%
\pgfusepath{clip}%
\pgfsetbuttcap%
\pgfsetroundjoin%
\pgfsetlinewidth{0.000000pt}%
\definecolor{currentstroke}{rgb}{0.000000,0.000000,0.000000}%
\pgfsetstrokecolor{currentstroke}%
\pgfsetdash{}{0pt}%
\pgfpathmoveto{\pgfqpoint{9.116182in}{4.590553in}}%
\pgfpathlineto{\pgfqpoint{9.302409in}{4.590553in}}%
\pgfpathlineto{\pgfqpoint{9.302409in}{4.672281in}}%
\pgfpathlineto{\pgfqpoint{9.116182in}{4.672281in}}%
\pgfpathlineto{\pgfqpoint{9.116182in}{4.590553in}}%
\pgfusepath{}%
\end{pgfscope}%
\begin{pgfscope}%
\pgfpathrectangle{\pgfqpoint{0.549740in}{0.463273in}}{\pgfqpoint{9.320225in}{4.495057in}}%
\pgfusepath{clip}%
\pgfsetbuttcap%
\pgfsetroundjoin%
\pgfsetlinewidth{0.000000pt}%
\definecolor{currentstroke}{rgb}{0.000000,0.000000,0.000000}%
\pgfsetstrokecolor{currentstroke}%
\pgfsetdash{}{0pt}%
\pgfpathmoveto{\pgfqpoint{9.302409in}{4.590553in}}%
\pgfpathlineto{\pgfqpoint{9.488635in}{4.590553in}}%
\pgfpathlineto{\pgfqpoint{9.488635in}{4.672281in}}%
\pgfpathlineto{\pgfqpoint{9.302409in}{4.672281in}}%
\pgfpathlineto{\pgfqpoint{9.302409in}{4.590553in}}%
\pgfusepath{}%
\end{pgfscope}%
\begin{pgfscope}%
\pgfpathrectangle{\pgfqpoint{0.549740in}{0.463273in}}{\pgfqpoint{9.320225in}{4.495057in}}%
\pgfusepath{clip}%
\pgfsetbuttcap%
\pgfsetroundjoin%
\pgfsetlinewidth{0.000000pt}%
\definecolor{currentstroke}{rgb}{0.000000,0.000000,0.000000}%
\pgfsetstrokecolor{currentstroke}%
\pgfsetdash{}{0pt}%
\pgfpathmoveto{\pgfqpoint{9.488635in}{4.590553in}}%
\pgfpathlineto{\pgfqpoint{9.674862in}{4.590553in}}%
\pgfpathlineto{\pgfqpoint{9.674862in}{4.672281in}}%
\pgfpathlineto{\pgfqpoint{9.488635in}{4.672281in}}%
\pgfpathlineto{\pgfqpoint{9.488635in}{4.590553in}}%
\pgfusepath{}%
\end{pgfscope}%
\begin{pgfscope}%
\pgfpathrectangle{\pgfqpoint{0.549740in}{0.463273in}}{\pgfqpoint{9.320225in}{4.495057in}}%
\pgfusepath{clip}%
\pgfsetbuttcap%
\pgfsetroundjoin%
\pgfsetlinewidth{0.000000pt}%
\definecolor{currentstroke}{rgb}{0.000000,0.000000,0.000000}%
\pgfsetstrokecolor{currentstroke}%
\pgfsetdash{}{0pt}%
\pgfpathmoveto{\pgfqpoint{9.674862in}{4.590553in}}%
\pgfpathlineto{\pgfqpoint{9.861088in}{4.590553in}}%
\pgfpathlineto{\pgfqpoint{9.861088in}{4.672281in}}%
\pgfpathlineto{\pgfqpoint{9.674862in}{4.672281in}}%
\pgfpathlineto{\pgfqpoint{9.674862in}{4.590553in}}%
\pgfusepath{}%
\end{pgfscope}%
\begin{pgfscope}%
\pgfpathrectangle{\pgfqpoint{0.549740in}{0.463273in}}{\pgfqpoint{9.320225in}{4.495057in}}%
\pgfusepath{clip}%
\pgfsetbuttcap%
\pgfsetroundjoin%
\pgfsetlinewidth{0.000000pt}%
\definecolor{currentstroke}{rgb}{0.000000,0.000000,0.000000}%
\pgfsetstrokecolor{currentstroke}%
\pgfsetdash{}{0pt}%
\pgfpathmoveto{\pgfqpoint{0.549761in}{4.672281in}}%
\pgfpathlineto{\pgfqpoint{0.735988in}{4.672281in}}%
\pgfpathlineto{\pgfqpoint{0.735988in}{4.754009in}}%
\pgfpathlineto{\pgfqpoint{0.549761in}{4.754009in}}%
\pgfpathlineto{\pgfqpoint{0.549761in}{4.672281in}}%
\pgfusepath{}%
\end{pgfscope}%
\begin{pgfscope}%
\pgfpathrectangle{\pgfqpoint{0.549740in}{0.463273in}}{\pgfqpoint{9.320225in}{4.495057in}}%
\pgfusepath{clip}%
\pgfsetbuttcap%
\pgfsetroundjoin%
\pgfsetlinewidth{0.000000pt}%
\definecolor{currentstroke}{rgb}{0.000000,0.000000,0.000000}%
\pgfsetstrokecolor{currentstroke}%
\pgfsetdash{}{0pt}%
\pgfpathmoveto{\pgfqpoint{0.735988in}{4.672281in}}%
\pgfpathlineto{\pgfqpoint{0.922214in}{4.672281in}}%
\pgfpathlineto{\pgfqpoint{0.922214in}{4.754009in}}%
\pgfpathlineto{\pgfqpoint{0.735988in}{4.754009in}}%
\pgfpathlineto{\pgfqpoint{0.735988in}{4.672281in}}%
\pgfusepath{}%
\end{pgfscope}%
\begin{pgfscope}%
\pgfpathrectangle{\pgfqpoint{0.549740in}{0.463273in}}{\pgfqpoint{9.320225in}{4.495057in}}%
\pgfusepath{clip}%
\pgfsetbuttcap%
\pgfsetroundjoin%
\pgfsetlinewidth{0.000000pt}%
\definecolor{currentstroke}{rgb}{0.000000,0.000000,0.000000}%
\pgfsetstrokecolor{currentstroke}%
\pgfsetdash{}{0pt}%
\pgfpathmoveto{\pgfqpoint{0.922214in}{4.672281in}}%
\pgfpathlineto{\pgfqpoint{1.108441in}{4.672281in}}%
\pgfpathlineto{\pgfqpoint{1.108441in}{4.754009in}}%
\pgfpathlineto{\pgfqpoint{0.922214in}{4.754009in}}%
\pgfpathlineto{\pgfqpoint{0.922214in}{4.672281in}}%
\pgfusepath{}%
\end{pgfscope}%
\begin{pgfscope}%
\pgfpathrectangle{\pgfqpoint{0.549740in}{0.463273in}}{\pgfqpoint{9.320225in}{4.495057in}}%
\pgfusepath{clip}%
\pgfsetbuttcap%
\pgfsetroundjoin%
\pgfsetlinewidth{0.000000pt}%
\definecolor{currentstroke}{rgb}{0.000000,0.000000,0.000000}%
\pgfsetstrokecolor{currentstroke}%
\pgfsetdash{}{0pt}%
\pgfpathmoveto{\pgfqpoint{1.108441in}{4.672281in}}%
\pgfpathlineto{\pgfqpoint{1.294667in}{4.672281in}}%
\pgfpathlineto{\pgfqpoint{1.294667in}{4.754009in}}%
\pgfpathlineto{\pgfqpoint{1.108441in}{4.754009in}}%
\pgfpathlineto{\pgfqpoint{1.108441in}{4.672281in}}%
\pgfusepath{}%
\end{pgfscope}%
\begin{pgfscope}%
\pgfpathrectangle{\pgfqpoint{0.549740in}{0.463273in}}{\pgfqpoint{9.320225in}{4.495057in}}%
\pgfusepath{clip}%
\pgfsetbuttcap%
\pgfsetroundjoin%
\pgfsetlinewidth{0.000000pt}%
\definecolor{currentstroke}{rgb}{0.000000,0.000000,0.000000}%
\pgfsetstrokecolor{currentstroke}%
\pgfsetdash{}{0pt}%
\pgfpathmoveto{\pgfqpoint{1.294667in}{4.672281in}}%
\pgfpathlineto{\pgfqpoint{1.480894in}{4.672281in}}%
\pgfpathlineto{\pgfqpoint{1.480894in}{4.754009in}}%
\pgfpathlineto{\pgfqpoint{1.294667in}{4.754009in}}%
\pgfpathlineto{\pgfqpoint{1.294667in}{4.672281in}}%
\pgfusepath{}%
\end{pgfscope}%
\begin{pgfscope}%
\pgfpathrectangle{\pgfqpoint{0.549740in}{0.463273in}}{\pgfqpoint{9.320225in}{4.495057in}}%
\pgfusepath{clip}%
\pgfsetbuttcap%
\pgfsetroundjoin%
\pgfsetlinewidth{0.000000pt}%
\definecolor{currentstroke}{rgb}{0.000000,0.000000,0.000000}%
\pgfsetstrokecolor{currentstroke}%
\pgfsetdash{}{0pt}%
\pgfpathmoveto{\pgfqpoint{1.480894in}{4.672281in}}%
\pgfpathlineto{\pgfqpoint{1.667120in}{4.672281in}}%
\pgfpathlineto{\pgfqpoint{1.667120in}{4.754009in}}%
\pgfpathlineto{\pgfqpoint{1.480894in}{4.754009in}}%
\pgfpathlineto{\pgfqpoint{1.480894in}{4.672281in}}%
\pgfusepath{}%
\end{pgfscope}%
\begin{pgfscope}%
\pgfpathrectangle{\pgfqpoint{0.549740in}{0.463273in}}{\pgfqpoint{9.320225in}{4.495057in}}%
\pgfusepath{clip}%
\pgfsetbuttcap%
\pgfsetroundjoin%
\pgfsetlinewidth{0.000000pt}%
\definecolor{currentstroke}{rgb}{0.000000,0.000000,0.000000}%
\pgfsetstrokecolor{currentstroke}%
\pgfsetdash{}{0pt}%
\pgfpathmoveto{\pgfqpoint{1.667120in}{4.672281in}}%
\pgfpathlineto{\pgfqpoint{1.853347in}{4.672281in}}%
\pgfpathlineto{\pgfqpoint{1.853347in}{4.754009in}}%
\pgfpathlineto{\pgfqpoint{1.667120in}{4.754009in}}%
\pgfpathlineto{\pgfqpoint{1.667120in}{4.672281in}}%
\pgfusepath{}%
\end{pgfscope}%
\begin{pgfscope}%
\pgfpathrectangle{\pgfqpoint{0.549740in}{0.463273in}}{\pgfqpoint{9.320225in}{4.495057in}}%
\pgfusepath{clip}%
\pgfsetbuttcap%
\pgfsetroundjoin%
\pgfsetlinewidth{0.000000pt}%
\definecolor{currentstroke}{rgb}{0.000000,0.000000,0.000000}%
\pgfsetstrokecolor{currentstroke}%
\pgfsetdash{}{0pt}%
\pgfpathmoveto{\pgfqpoint{1.853347in}{4.672281in}}%
\pgfpathlineto{\pgfqpoint{2.039573in}{4.672281in}}%
\pgfpathlineto{\pgfqpoint{2.039573in}{4.754009in}}%
\pgfpathlineto{\pgfqpoint{1.853347in}{4.754009in}}%
\pgfpathlineto{\pgfqpoint{1.853347in}{4.672281in}}%
\pgfusepath{}%
\end{pgfscope}%
\begin{pgfscope}%
\pgfpathrectangle{\pgfqpoint{0.549740in}{0.463273in}}{\pgfqpoint{9.320225in}{4.495057in}}%
\pgfusepath{clip}%
\pgfsetbuttcap%
\pgfsetroundjoin%
\pgfsetlinewidth{0.000000pt}%
\definecolor{currentstroke}{rgb}{0.000000,0.000000,0.000000}%
\pgfsetstrokecolor{currentstroke}%
\pgfsetdash{}{0pt}%
\pgfpathmoveto{\pgfqpoint{2.039573in}{4.672281in}}%
\pgfpathlineto{\pgfqpoint{2.225800in}{4.672281in}}%
\pgfpathlineto{\pgfqpoint{2.225800in}{4.754009in}}%
\pgfpathlineto{\pgfqpoint{2.039573in}{4.754009in}}%
\pgfpathlineto{\pgfqpoint{2.039573in}{4.672281in}}%
\pgfusepath{}%
\end{pgfscope}%
\begin{pgfscope}%
\pgfpathrectangle{\pgfqpoint{0.549740in}{0.463273in}}{\pgfqpoint{9.320225in}{4.495057in}}%
\pgfusepath{clip}%
\pgfsetbuttcap%
\pgfsetroundjoin%
\pgfsetlinewidth{0.000000pt}%
\definecolor{currentstroke}{rgb}{0.000000,0.000000,0.000000}%
\pgfsetstrokecolor{currentstroke}%
\pgfsetdash{}{0pt}%
\pgfpathmoveto{\pgfqpoint{2.225800in}{4.672281in}}%
\pgfpathlineto{\pgfqpoint{2.412027in}{4.672281in}}%
\pgfpathlineto{\pgfqpoint{2.412027in}{4.754009in}}%
\pgfpathlineto{\pgfqpoint{2.225800in}{4.754009in}}%
\pgfpathlineto{\pgfqpoint{2.225800in}{4.672281in}}%
\pgfusepath{}%
\end{pgfscope}%
\begin{pgfscope}%
\pgfpathrectangle{\pgfqpoint{0.549740in}{0.463273in}}{\pgfqpoint{9.320225in}{4.495057in}}%
\pgfusepath{clip}%
\pgfsetbuttcap%
\pgfsetroundjoin%
\pgfsetlinewidth{0.000000pt}%
\definecolor{currentstroke}{rgb}{0.000000,0.000000,0.000000}%
\pgfsetstrokecolor{currentstroke}%
\pgfsetdash{}{0pt}%
\pgfpathmoveto{\pgfqpoint{2.412027in}{4.672281in}}%
\pgfpathlineto{\pgfqpoint{2.598253in}{4.672281in}}%
\pgfpathlineto{\pgfqpoint{2.598253in}{4.754009in}}%
\pgfpathlineto{\pgfqpoint{2.412027in}{4.754009in}}%
\pgfpathlineto{\pgfqpoint{2.412027in}{4.672281in}}%
\pgfusepath{}%
\end{pgfscope}%
\begin{pgfscope}%
\pgfpathrectangle{\pgfqpoint{0.549740in}{0.463273in}}{\pgfqpoint{9.320225in}{4.495057in}}%
\pgfusepath{clip}%
\pgfsetbuttcap%
\pgfsetroundjoin%
\pgfsetlinewidth{0.000000pt}%
\definecolor{currentstroke}{rgb}{0.000000,0.000000,0.000000}%
\pgfsetstrokecolor{currentstroke}%
\pgfsetdash{}{0pt}%
\pgfpathmoveto{\pgfqpoint{2.598253in}{4.672281in}}%
\pgfpathlineto{\pgfqpoint{2.784480in}{4.672281in}}%
\pgfpathlineto{\pgfqpoint{2.784480in}{4.754009in}}%
\pgfpathlineto{\pgfqpoint{2.598253in}{4.754009in}}%
\pgfpathlineto{\pgfqpoint{2.598253in}{4.672281in}}%
\pgfusepath{}%
\end{pgfscope}%
\begin{pgfscope}%
\pgfpathrectangle{\pgfqpoint{0.549740in}{0.463273in}}{\pgfqpoint{9.320225in}{4.495057in}}%
\pgfusepath{clip}%
\pgfsetbuttcap%
\pgfsetroundjoin%
\pgfsetlinewidth{0.000000pt}%
\definecolor{currentstroke}{rgb}{0.000000,0.000000,0.000000}%
\pgfsetstrokecolor{currentstroke}%
\pgfsetdash{}{0pt}%
\pgfpathmoveto{\pgfqpoint{2.784480in}{4.672281in}}%
\pgfpathlineto{\pgfqpoint{2.970706in}{4.672281in}}%
\pgfpathlineto{\pgfqpoint{2.970706in}{4.754009in}}%
\pgfpathlineto{\pgfqpoint{2.784480in}{4.754009in}}%
\pgfpathlineto{\pgfqpoint{2.784480in}{4.672281in}}%
\pgfusepath{}%
\end{pgfscope}%
\begin{pgfscope}%
\pgfpathrectangle{\pgfqpoint{0.549740in}{0.463273in}}{\pgfqpoint{9.320225in}{4.495057in}}%
\pgfusepath{clip}%
\pgfsetbuttcap%
\pgfsetroundjoin%
\pgfsetlinewidth{0.000000pt}%
\definecolor{currentstroke}{rgb}{0.000000,0.000000,0.000000}%
\pgfsetstrokecolor{currentstroke}%
\pgfsetdash{}{0pt}%
\pgfpathmoveto{\pgfqpoint{2.970706in}{4.672281in}}%
\pgfpathlineto{\pgfqpoint{3.156933in}{4.672281in}}%
\pgfpathlineto{\pgfqpoint{3.156933in}{4.754009in}}%
\pgfpathlineto{\pgfqpoint{2.970706in}{4.754009in}}%
\pgfpathlineto{\pgfqpoint{2.970706in}{4.672281in}}%
\pgfusepath{}%
\end{pgfscope}%
\begin{pgfscope}%
\pgfpathrectangle{\pgfqpoint{0.549740in}{0.463273in}}{\pgfqpoint{9.320225in}{4.495057in}}%
\pgfusepath{clip}%
\pgfsetbuttcap%
\pgfsetroundjoin%
\pgfsetlinewidth{0.000000pt}%
\definecolor{currentstroke}{rgb}{0.000000,0.000000,0.000000}%
\pgfsetstrokecolor{currentstroke}%
\pgfsetdash{}{0pt}%
\pgfpathmoveto{\pgfqpoint{3.156933in}{4.672281in}}%
\pgfpathlineto{\pgfqpoint{3.343159in}{4.672281in}}%
\pgfpathlineto{\pgfqpoint{3.343159in}{4.754009in}}%
\pgfpathlineto{\pgfqpoint{3.156933in}{4.754009in}}%
\pgfpathlineto{\pgfqpoint{3.156933in}{4.672281in}}%
\pgfusepath{}%
\end{pgfscope}%
\begin{pgfscope}%
\pgfpathrectangle{\pgfqpoint{0.549740in}{0.463273in}}{\pgfqpoint{9.320225in}{4.495057in}}%
\pgfusepath{clip}%
\pgfsetbuttcap%
\pgfsetroundjoin%
\pgfsetlinewidth{0.000000pt}%
\definecolor{currentstroke}{rgb}{0.000000,0.000000,0.000000}%
\pgfsetstrokecolor{currentstroke}%
\pgfsetdash{}{0pt}%
\pgfpathmoveto{\pgfqpoint{3.343159in}{4.672281in}}%
\pgfpathlineto{\pgfqpoint{3.529386in}{4.672281in}}%
\pgfpathlineto{\pgfqpoint{3.529386in}{4.754009in}}%
\pgfpathlineto{\pgfqpoint{3.343159in}{4.754009in}}%
\pgfpathlineto{\pgfqpoint{3.343159in}{4.672281in}}%
\pgfusepath{}%
\end{pgfscope}%
\begin{pgfscope}%
\pgfpathrectangle{\pgfqpoint{0.549740in}{0.463273in}}{\pgfqpoint{9.320225in}{4.495057in}}%
\pgfusepath{clip}%
\pgfsetbuttcap%
\pgfsetroundjoin%
\pgfsetlinewidth{0.000000pt}%
\definecolor{currentstroke}{rgb}{0.000000,0.000000,0.000000}%
\pgfsetstrokecolor{currentstroke}%
\pgfsetdash{}{0pt}%
\pgfpathmoveto{\pgfqpoint{3.529386in}{4.672281in}}%
\pgfpathlineto{\pgfqpoint{3.715612in}{4.672281in}}%
\pgfpathlineto{\pgfqpoint{3.715612in}{4.754009in}}%
\pgfpathlineto{\pgfqpoint{3.529386in}{4.754009in}}%
\pgfpathlineto{\pgfqpoint{3.529386in}{4.672281in}}%
\pgfusepath{}%
\end{pgfscope}%
\begin{pgfscope}%
\pgfpathrectangle{\pgfqpoint{0.549740in}{0.463273in}}{\pgfqpoint{9.320225in}{4.495057in}}%
\pgfusepath{clip}%
\pgfsetbuttcap%
\pgfsetroundjoin%
\pgfsetlinewidth{0.000000pt}%
\definecolor{currentstroke}{rgb}{0.000000,0.000000,0.000000}%
\pgfsetstrokecolor{currentstroke}%
\pgfsetdash{}{0pt}%
\pgfpathmoveto{\pgfqpoint{3.715612in}{4.672281in}}%
\pgfpathlineto{\pgfqpoint{3.901839in}{4.672281in}}%
\pgfpathlineto{\pgfqpoint{3.901839in}{4.754009in}}%
\pgfpathlineto{\pgfqpoint{3.715612in}{4.754009in}}%
\pgfpathlineto{\pgfqpoint{3.715612in}{4.672281in}}%
\pgfusepath{}%
\end{pgfscope}%
\begin{pgfscope}%
\pgfpathrectangle{\pgfqpoint{0.549740in}{0.463273in}}{\pgfqpoint{9.320225in}{4.495057in}}%
\pgfusepath{clip}%
\pgfsetbuttcap%
\pgfsetroundjoin%
\pgfsetlinewidth{0.000000pt}%
\definecolor{currentstroke}{rgb}{0.000000,0.000000,0.000000}%
\pgfsetstrokecolor{currentstroke}%
\pgfsetdash{}{0pt}%
\pgfpathmoveto{\pgfqpoint{3.901839in}{4.672281in}}%
\pgfpathlineto{\pgfqpoint{4.088065in}{4.672281in}}%
\pgfpathlineto{\pgfqpoint{4.088065in}{4.754009in}}%
\pgfpathlineto{\pgfqpoint{3.901839in}{4.754009in}}%
\pgfpathlineto{\pgfqpoint{3.901839in}{4.672281in}}%
\pgfusepath{}%
\end{pgfscope}%
\begin{pgfscope}%
\pgfpathrectangle{\pgfqpoint{0.549740in}{0.463273in}}{\pgfqpoint{9.320225in}{4.495057in}}%
\pgfusepath{clip}%
\pgfsetbuttcap%
\pgfsetroundjoin%
\pgfsetlinewidth{0.000000pt}%
\definecolor{currentstroke}{rgb}{0.000000,0.000000,0.000000}%
\pgfsetstrokecolor{currentstroke}%
\pgfsetdash{}{0pt}%
\pgfpathmoveto{\pgfqpoint{4.088065in}{4.672281in}}%
\pgfpathlineto{\pgfqpoint{4.274292in}{4.672281in}}%
\pgfpathlineto{\pgfqpoint{4.274292in}{4.754009in}}%
\pgfpathlineto{\pgfqpoint{4.088065in}{4.754009in}}%
\pgfpathlineto{\pgfqpoint{4.088065in}{4.672281in}}%
\pgfusepath{}%
\end{pgfscope}%
\begin{pgfscope}%
\pgfpathrectangle{\pgfqpoint{0.549740in}{0.463273in}}{\pgfqpoint{9.320225in}{4.495057in}}%
\pgfusepath{clip}%
\pgfsetbuttcap%
\pgfsetroundjoin%
\pgfsetlinewidth{0.000000pt}%
\definecolor{currentstroke}{rgb}{0.000000,0.000000,0.000000}%
\pgfsetstrokecolor{currentstroke}%
\pgfsetdash{}{0pt}%
\pgfpathmoveto{\pgfqpoint{4.274292in}{4.672281in}}%
\pgfpathlineto{\pgfqpoint{4.460519in}{4.672281in}}%
\pgfpathlineto{\pgfqpoint{4.460519in}{4.754009in}}%
\pgfpathlineto{\pgfqpoint{4.274292in}{4.754009in}}%
\pgfpathlineto{\pgfqpoint{4.274292in}{4.672281in}}%
\pgfusepath{}%
\end{pgfscope}%
\begin{pgfscope}%
\pgfpathrectangle{\pgfqpoint{0.549740in}{0.463273in}}{\pgfqpoint{9.320225in}{4.495057in}}%
\pgfusepath{clip}%
\pgfsetbuttcap%
\pgfsetroundjoin%
\pgfsetlinewidth{0.000000pt}%
\definecolor{currentstroke}{rgb}{0.000000,0.000000,0.000000}%
\pgfsetstrokecolor{currentstroke}%
\pgfsetdash{}{0pt}%
\pgfpathmoveto{\pgfqpoint{4.460519in}{4.672281in}}%
\pgfpathlineto{\pgfqpoint{4.646745in}{4.672281in}}%
\pgfpathlineto{\pgfqpoint{4.646745in}{4.754009in}}%
\pgfpathlineto{\pgfqpoint{4.460519in}{4.754009in}}%
\pgfpathlineto{\pgfqpoint{4.460519in}{4.672281in}}%
\pgfusepath{}%
\end{pgfscope}%
\begin{pgfscope}%
\pgfpathrectangle{\pgfqpoint{0.549740in}{0.463273in}}{\pgfqpoint{9.320225in}{4.495057in}}%
\pgfusepath{clip}%
\pgfsetbuttcap%
\pgfsetroundjoin%
\pgfsetlinewidth{0.000000pt}%
\definecolor{currentstroke}{rgb}{0.000000,0.000000,0.000000}%
\pgfsetstrokecolor{currentstroke}%
\pgfsetdash{}{0pt}%
\pgfpathmoveto{\pgfqpoint{4.646745in}{4.672281in}}%
\pgfpathlineto{\pgfqpoint{4.832972in}{4.672281in}}%
\pgfpathlineto{\pgfqpoint{4.832972in}{4.754009in}}%
\pgfpathlineto{\pgfqpoint{4.646745in}{4.754009in}}%
\pgfpathlineto{\pgfqpoint{4.646745in}{4.672281in}}%
\pgfusepath{}%
\end{pgfscope}%
\begin{pgfscope}%
\pgfpathrectangle{\pgfqpoint{0.549740in}{0.463273in}}{\pgfqpoint{9.320225in}{4.495057in}}%
\pgfusepath{clip}%
\pgfsetbuttcap%
\pgfsetroundjoin%
\pgfsetlinewidth{0.000000pt}%
\definecolor{currentstroke}{rgb}{0.000000,0.000000,0.000000}%
\pgfsetstrokecolor{currentstroke}%
\pgfsetdash{}{0pt}%
\pgfpathmoveto{\pgfqpoint{4.832972in}{4.672281in}}%
\pgfpathlineto{\pgfqpoint{5.019198in}{4.672281in}}%
\pgfpathlineto{\pgfqpoint{5.019198in}{4.754009in}}%
\pgfpathlineto{\pgfqpoint{4.832972in}{4.754009in}}%
\pgfpathlineto{\pgfqpoint{4.832972in}{4.672281in}}%
\pgfusepath{}%
\end{pgfscope}%
\begin{pgfscope}%
\pgfpathrectangle{\pgfqpoint{0.549740in}{0.463273in}}{\pgfqpoint{9.320225in}{4.495057in}}%
\pgfusepath{clip}%
\pgfsetbuttcap%
\pgfsetroundjoin%
\pgfsetlinewidth{0.000000pt}%
\definecolor{currentstroke}{rgb}{0.000000,0.000000,0.000000}%
\pgfsetstrokecolor{currentstroke}%
\pgfsetdash{}{0pt}%
\pgfpathmoveto{\pgfqpoint{5.019198in}{4.672281in}}%
\pgfpathlineto{\pgfqpoint{5.205425in}{4.672281in}}%
\pgfpathlineto{\pgfqpoint{5.205425in}{4.754009in}}%
\pgfpathlineto{\pgfqpoint{5.019198in}{4.754009in}}%
\pgfpathlineto{\pgfqpoint{5.019198in}{4.672281in}}%
\pgfusepath{}%
\end{pgfscope}%
\begin{pgfscope}%
\pgfpathrectangle{\pgfqpoint{0.549740in}{0.463273in}}{\pgfqpoint{9.320225in}{4.495057in}}%
\pgfusepath{clip}%
\pgfsetbuttcap%
\pgfsetroundjoin%
\pgfsetlinewidth{0.000000pt}%
\definecolor{currentstroke}{rgb}{0.000000,0.000000,0.000000}%
\pgfsetstrokecolor{currentstroke}%
\pgfsetdash{}{0pt}%
\pgfpathmoveto{\pgfqpoint{5.205425in}{4.672281in}}%
\pgfpathlineto{\pgfqpoint{5.391651in}{4.672281in}}%
\pgfpathlineto{\pgfqpoint{5.391651in}{4.754009in}}%
\pgfpathlineto{\pgfqpoint{5.205425in}{4.754009in}}%
\pgfpathlineto{\pgfqpoint{5.205425in}{4.672281in}}%
\pgfusepath{}%
\end{pgfscope}%
\begin{pgfscope}%
\pgfpathrectangle{\pgfqpoint{0.549740in}{0.463273in}}{\pgfqpoint{9.320225in}{4.495057in}}%
\pgfusepath{clip}%
\pgfsetbuttcap%
\pgfsetroundjoin%
\pgfsetlinewidth{0.000000pt}%
\definecolor{currentstroke}{rgb}{0.000000,0.000000,0.000000}%
\pgfsetstrokecolor{currentstroke}%
\pgfsetdash{}{0pt}%
\pgfpathmoveto{\pgfqpoint{5.391651in}{4.672281in}}%
\pgfpathlineto{\pgfqpoint{5.577878in}{4.672281in}}%
\pgfpathlineto{\pgfqpoint{5.577878in}{4.754009in}}%
\pgfpathlineto{\pgfqpoint{5.391651in}{4.754009in}}%
\pgfpathlineto{\pgfqpoint{5.391651in}{4.672281in}}%
\pgfusepath{}%
\end{pgfscope}%
\begin{pgfscope}%
\pgfpathrectangle{\pgfqpoint{0.549740in}{0.463273in}}{\pgfqpoint{9.320225in}{4.495057in}}%
\pgfusepath{clip}%
\pgfsetbuttcap%
\pgfsetroundjoin%
\pgfsetlinewidth{0.000000pt}%
\definecolor{currentstroke}{rgb}{0.000000,0.000000,0.000000}%
\pgfsetstrokecolor{currentstroke}%
\pgfsetdash{}{0pt}%
\pgfpathmoveto{\pgfqpoint{5.577878in}{4.672281in}}%
\pgfpathlineto{\pgfqpoint{5.764104in}{4.672281in}}%
\pgfpathlineto{\pgfqpoint{5.764104in}{4.754009in}}%
\pgfpathlineto{\pgfqpoint{5.577878in}{4.754009in}}%
\pgfpathlineto{\pgfqpoint{5.577878in}{4.672281in}}%
\pgfusepath{}%
\end{pgfscope}%
\begin{pgfscope}%
\pgfpathrectangle{\pgfqpoint{0.549740in}{0.463273in}}{\pgfqpoint{9.320225in}{4.495057in}}%
\pgfusepath{clip}%
\pgfsetbuttcap%
\pgfsetroundjoin%
\pgfsetlinewidth{0.000000pt}%
\definecolor{currentstroke}{rgb}{0.000000,0.000000,0.000000}%
\pgfsetstrokecolor{currentstroke}%
\pgfsetdash{}{0pt}%
\pgfpathmoveto{\pgfqpoint{5.764104in}{4.672281in}}%
\pgfpathlineto{\pgfqpoint{5.950331in}{4.672281in}}%
\pgfpathlineto{\pgfqpoint{5.950331in}{4.754009in}}%
\pgfpathlineto{\pgfqpoint{5.764104in}{4.754009in}}%
\pgfpathlineto{\pgfqpoint{5.764104in}{4.672281in}}%
\pgfusepath{}%
\end{pgfscope}%
\begin{pgfscope}%
\pgfpathrectangle{\pgfqpoint{0.549740in}{0.463273in}}{\pgfqpoint{9.320225in}{4.495057in}}%
\pgfusepath{clip}%
\pgfsetbuttcap%
\pgfsetroundjoin%
\pgfsetlinewidth{0.000000pt}%
\definecolor{currentstroke}{rgb}{0.000000,0.000000,0.000000}%
\pgfsetstrokecolor{currentstroke}%
\pgfsetdash{}{0pt}%
\pgfpathmoveto{\pgfqpoint{5.950331in}{4.672281in}}%
\pgfpathlineto{\pgfqpoint{6.136557in}{4.672281in}}%
\pgfpathlineto{\pgfqpoint{6.136557in}{4.754009in}}%
\pgfpathlineto{\pgfqpoint{5.950331in}{4.754009in}}%
\pgfpathlineto{\pgfqpoint{5.950331in}{4.672281in}}%
\pgfusepath{}%
\end{pgfscope}%
\begin{pgfscope}%
\pgfpathrectangle{\pgfqpoint{0.549740in}{0.463273in}}{\pgfqpoint{9.320225in}{4.495057in}}%
\pgfusepath{clip}%
\pgfsetbuttcap%
\pgfsetroundjoin%
\pgfsetlinewidth{0.000000pt}%
\definecolor{currentstroke}{rgb}{0.000000,0.000000,0.000000}%
\pgfsetstrokecolor{currentstroke}%
\pgfsetdash{}{0pt}%
\pgfpathmoveto{\pgfqpoint{6.136557in}{4.672281in}}%
\pgfpathlineto{\pgfqpoint{6.322784in}{4.672281in}}%
\pgfpathlineto{\pgfqpoint{6.322784in}{4.754009in}}%
\pgfpathlineto{\pgfqpoint{6.136557in}{4.754009in}}%
\pgfpathlineto{\pgfqpoint{6.136557in}{4.672281in}}%
\pgfusepath{}%
\end{pgfscope}%
\begin{pgfscope}%
\pgfpathrectangle{\pgfqpoint{0.549740in}{0.463273in}}{\pgfqpoint{9.320225in}{4.495057in}}%
\pgfusepath{clip}%
\pgfsetbuttcap%
\pgfsetroundjoin%
\pgfsetlinewidth{0.000000pt}%
\definecolor{currentstroke}{rgb}{0.000000,0.000000,0.000000}%
\pgfsetstrokecolor{currentstroke}%
\pgfsetdash{}{0pt}%
\pgfpathmoveto{\pgfqpoint{6.322784in}{4.672281in}}%
\pgfpathlineto{\pgfqpoint{6.509011in}{4.672281in}}%
\pgfpathlineto{\pgfqpoint{6.509011in}{4.754009in}}%
\pgfpathlineto{\pgfqpoint{6.322784in}{4.754009in}}%
\pgfpathlineto{\pgfqpoint{6.322784in}{4.672281in}}%
\pgfusepath{}%
\end{pgfscope}%
\begin{pgfscope}%
\pgfpathrectangle{\pgfqpoint{0.549740in}{0.463273in}}{\pgfqpoint{9.320225in}{4.495057in}}%
\pgfusepath{clip}%
\pgfsetbuttcap%
\pgfsetroundjoin%
\pgfsetlinewidth{0.000000pt}%
\definecolor{currentstroke}{rgb}{0.000000,0.000000,0.000000}%
\pgfsetstrokecolor{currentstroke}%
\pgfsetdash{}{0pt}%
\pgfpathmoveto{\pgfqpoint{6.509011in}{4.672281in}}%
\pgfpathlineto{\pgfqpoint{6.695237in}{4.672281in}}%
\pgfpathlineto{\pgfqpoint{6.695237in}{4.754009in}}%
\pgfpathlineto{\pgfqpoint{6.509011in}{4.754009in}}%
\pgfpathlineto{\pgfqpoint{6.509011in}{4.672281in}}%
\pgfusepath{}%
\end{pgfscope}%
\begin{pgfscope}%
\pgfpathrectangle{\pgfqpoint{0.549740in}{0.463273in}}{\pgfqpoint{9.320225in}{4.495057in}}%
\pgfusepath{clip}%
\pgfsetbuttcap%
\pgfsetroundjoin%
\pgfsetlinewidth{0.000000pt}%
\definecolor{currentstroke}{rgb}{0.000000,0.000000,0.000000}%
\pgfsetstrokecolor{currentstroke}%
\pgfsetdash{}{0pt}%
\pgfpathmoveto{\pgfqpoint{6.695237in}{4.672281in}}%
\pgfpathlineto{\pgfqpoint{6.881464in}{4.672281in}}%
\pgfpathlineto{\pgfqpoint{6.881464in}{4.754009in}}%
\pgfpathlineto{\pgfqpoint{6.695237in}{4.754009in}}%
\pgfpathlineto{\pgfqpoint{6.695237in}{4.672281in}}%
\pgfusepath{}%
\end{pgfscope}%
\begin{pgfscope}%
\pgfpathrectangle{\pgfqpoint{0.549740in}{0.463273in}}{\pgfqpoint{9.320225in}{4.495057in}}%
\pgfusepath{clip}%
\pgfsetbuttcap%
\pgfsetroundjoin%
\pgfsetlinewidth{0.000000pt}%
\definecolor{currentstroke}{rgb}{0.000000,0.000000,0.000000}%
\pgfsetstrokecolor{currentstroke}%
\pgfsetdash{}{0pt}%
\pgfpathmoveto{\pgfqpoint{6.881464in}{4.672281in}}%
\pgfpathlineto{\pgfqpoint{7.067690in}{4.672281in}}%
\pgfpathlineto{\pgfqpoint{7.067690in}{4.754009in}}%
\pgfpathlineto{\pgfqpoint{6.881464in}{4.754009in}}%
\pgfpathlineto{\pgfqpoint{6.881464in}{4.672281in}}%
\pgfusepath{}%
\end{pgfscope}%
\begin{pgfscope}%
\pgfpathrectangle{\pgfqpoint{0.549740in}{0.463273in}}{\pgfqpoint{9.320225in}{4.495057in}}%
\pgfusepath{clip}%
\pgfsetbuttcap%
\pgfsetroundjoin%
\pgfsetlinewidth{0.000000pt}%
\definecolor{currentstroke}{rgb}{0.000000,0.000000,0.000000}%
\pgfsetstrokecolor{currentstroke}%
\pgfsetdash{}{0pt}%
\pgfpathmoveto{\pgfqpoint{7.067690in}{4.672281in}}%
\pgfpathlineto{\pgfqpoint{7.253917in}{4.672281in}}%
\pgfpathlineto{\pgfqpoint{7.253917in}{4.754009in}}%
\pgfpathlineto{\pgfqpoint{7.067690in}{4.754009in}}%
\pgfpathlineto{\pgfqpoint{7.067690in}{4.672281in}}%
\pgfusepath{}%
\end{pgfscope}%
\begin{pgfscope}%
\pgfpathrectangle{\pgfqpoint{0.549740in}{0.463273in}}{\pgfqpoint{9.320225in}{4.495057in}}%
\pgfusepath{clip}%
\pgfsetbuttcap%
\pgfsetroundjoin%
\pgfsetlinewidth{0.000000pt}%
\definecolor{currentstroke}{rgb}{0.000000,0.000000,0.000000}%
\pgfsetstrokecolor{currentstroke}%
\pgfsetdash{}{0pt}%
\pgfpathmoveto{\pgfqpoint{7.253917in}{4.672281in}}%
\pgfpathlineto{\pgfqpoint{7.440143in}{4.672281in}}%
\pgfpathlineto{\pgfqpoint{7.440143in}{4.754009in}}%
\pgfpathlineto{\pgfqpoint{7.253917in}{4.754009in}}%
\pgfpathlineto{\pgfqpoint{7.253917in}{4.672281in}}%
\pgfusepath{}%
\end{pgfscope}%
\begin{pgfscope}%
\pgfpathrectangle{\pgfqpoint{0.549740in}{0.463273in}}{\pgfqpoint{9.320225in}{4.495057in}}%
\pgfusepath{clip}%
\pgfsetbuttcap%
\pgfsetroundjoin%
\pgfsetlinewidth{0.000000pt}%
\definecolor{currentstroke}{rgb}{0.000000,0.000000,0.000000}%
\pgfsetstrokecolor{currentstroke}%
\pgfsetdash{}{0pt}%
\pgfpathmoveto{\pgfqpoint{7.440143in}{4.672281in}}%
\pgfpathlineto{\pgfqpoint{7.626370in}{4.672281in}}%
\pgfpathlineto{\pgfqpoint{7.626370in}{4.754009in}}%
\pgfpathlineto{\pgfqpoint{7.440143in}{4.754009in}}%
\pgfpathlineto{\pgfqpoint{7.440143in}{4.672281in}}%
\pgfusepath{}%
\end{pgfscope}%
\begin{pgfscope}%
\pgfpathrectangle{\pgfqpoint{0.549740in}{0.463273in}}{\pgfqpoint{9.320225in}{4.495057in}}%
\pgfusepath{clip}%
\pgfsetbuttcap%
\pgfsetroundjoin%
\pgfsetlinewidth{0.000000pt}%
\definecolor{currentstroke}{rgb}{0.000000,0.000000,0.000000}%
\pgfsetstrokecolor{currentstroke}%
\pgfsetdash{}{0pt}%
\pgfpathmoveto{\pgfqpoint{7.626370in}{4.672281in}}%
\pgfpathlineto{\pgfqpoint{7.812596in}{4.672281in}}%
\pgfpathlineto{\pgfqpoint{7.812596in}{4.754009in}}%
\pgfpathlineto{\pgfqpoint{7.626370in}{4.754009in}}%
\pgfpathlineto{\pgfqpoint{7.626370in}{4.672281in}}%
\pgfusepath{}%
\end{pgfscope}%
\begin{pgfscope}%
\pgfpathrectangle{\pgfqpoint{0.549740in}{0.463273in}}{\pgfqpoint{9.320225in}{4.495057in}}%
\pgfusepath{clip}%
\pgfsetbuttcap%
\pgfsetroundjoin%
\pgfsetlinewidth{0.000000pt}%
\definecolor{currentstroke}{rgb}{0.000000,0.000000,0.000000}%
\pgfsetstrokecolor{currentstroke}%
\pgfsetdash{}{0pt}%
\pgfpathmoveto{\pgfqpoint{7.812596in}{4.672281in}}%
\pgfpathlineto{\pgfqpoint{7.998823in}{4.672281in}}%
\pgfpathlineto{\pgfqpoint{7.998823in}{4.754009in}}%
\pgfpathlineto{\pgfqpoint{7.812596in}{4.754009in}}%
\pgfpathlineto{\pgfqpoint{7.812596in}{4.672281in}}%
\pgfusepath{}%
\end{pgfscope}%
\begin{pgfscope}%
\pgfpathrectangle{\pgfqpoint{0.549740in}{0.463273in}}{\pgfqpoint{9.320225in}{4.495057in}}%
\pgfusepath{clip}%
\pgfsetbuttcap%
\pgfsetroundjoin%
\pgfsetlinewidth{0.000000pt}%
\definecolor{currentstroke}{rgb}{0.000000,0.000000,0.000000}%
\pgfsetstrokecolor{currentstroke}%
\pgfsetdash{}{0pt}%
\pgfpathmoveto{\pgfqpoint{7.998823in}{4.672281in}}%
\pgfpathlineto{\pgfqpoint{8.185049in}{4.672281in}}%
\pgfpathlineto{\pgfqpoint{8.185049in}{4.754009in}}%
\pgfpathlineto{\pgfqpoint{7.998823in}{4.754009in}}%
\pgfpathlineto{\pgfqpoint{7.998823in}{4.672281in}}%
\pgfusepath{}%
\end{pgfscope}%
\begin{pgfscope}%
\pgfpathrectangle{\pgfqpoint{0.549740in}{0.463273in}}{\pgfqpoint{9.320225in}{4.495057in}}%
\pgfusepath{clip}%
\pgfsetbuttcap%
\pgfsetroundjoin%
\pgfsetlinewidth{0.000000pt}%
\definecolor{currentstroke}{rgb}{0.000000,0.000000,0.000000}%
\pgfsetstrokecolor{currentstroke}%
\pgfsetdash{}{0pt}%
\pgfpathmoveto{\pgfqpoint{8.185049in}{4.672281in}}%
\pgfpathlineto{\pgfqpoint{8.371276in}{4.672281in}}%
\pgfpathlineto{\pgfqpoint{8.371276in}{4.754009in}}%
\pgfpathlineto{\pgfqpoint{8.185049in}{4.754009in}}%
\pgfpathlineto{\pgfqpoint{8.185049in}{4.672281in}}%
\pgfusepath{}%
\end{pgfscope}%
\begin{pgfscope}%
\pgfpathrectangle{\pgfqpoint{0.549740in}{0.463273in}}{\pgfqpoint{9.320225in}{4.495057in}}%
\pgfusepath{clip}%
\pgfsetbuttcap%
\pgfsetroundjoin%
\pgfsetlinewidth{0.000000pt}%
\definecolor{currentstroke}{rgb}{0.000000,0.000000,0.000000}%
\pgfsetstrokecolor{currentstroke}%
\pgfsetdash{}{0pt}%
\pgfpathmoveto{\pgfqpoint{8.371276in}{4.672281in}}%
\pgfpathlineto{\pgfqpoint{8.557503in}{4.672281in}}%
\pgfpathlineto{\pgfqpoint{8.557503in}{4.754009in}}%
\pgfpathlineto{\pgfqpoint{8.371276in}{4.754009in}}%
\pgfpathlineto{\pgfqpoint{8.371276in}{4.672281in}}%
\pgfusepath{}%
\end{pgfscope}%
\begin{pgfscope}%
\pgfpathrectangle{\pgfqpoint{0.549740in}{0.463273in}}{\pgfqpoint{9.320225in}{4.495057in}}%
\pgfusepath{clip}%
\pgfsetbuttcap%
\pgfsetroundjoin%
\pgfsetlinewidth{0.000000pt}%
\definecolor{currentstroke}{rgb}{0.000000,0.000000,0.000000}%
\pgfsetstrokecolor{currentstroke}%
\pgfsetdash{}{0pt}%
\pgfpathmoveto{\pgfqpoint{8.557503in}{4.672281in}}%
\pgfpathlineto{\pgfqpoint{8.743729in}{4.672281in}}%
\pgfpathlineto{\pgfqpoint{8.743729in}{4.754009in}}%
\pgfpathlineto{\pgfqpoint{8.557503in}{4.754009in}}%
\pgfpathlineto{\pgfqpoint{8.557503in}{4.672281in}}%
\pgfusepath{}%
\end{pgfscope}%
\begin{pgfscope}%
\pgfpathrectangle{\pgfqpoint{0.549740in}{0.463273in}}{\pgfqpoint{9.320225in}{4.495057in}}%
\pgfusepath{clip}%
\pgfsetbuttcap%
\pgfsetroundjoin%
\pgfsetlinewidth{0.000000pt}%
\definecolor{currentstroke}{rgb}{0.000000,0.000000,0.000000}%
\pgfsetstrokecolor{currentstroke}%
\pgfsetdash{}{0pt}%
\pgfpathmoveto{\pgfqpoint{8.743729in}{4.672281in}}%
\pgfpathlineto{\pgfqpoint{8.929956in}{4.672281in}}%
\pgfpathlineto{\pgfqpoint{8.929956in}{4.754009in}}%
\pgfpathlineto{\pgfqpoint{8.743729in}{4.754009in}}%
\pgfpathlineto{\pgfqpoint{8.743729in}{4.672281in}}%
\pgfusepath{}%
\end{pgfscope}%
\begin{pgfscope}%
\pgfpathrectangle{\pgfqpoint{0.549740in}{0.463273in}}{\pgfqpoint{9.320225in}{4.495057in}}%
\pgfusepath{clip}%
\pgfsetbuttcap%
\pgfsetroundjoin%
\definecolor{currentfill}{rgb}{0.472869,0.711325,0.955316}%
\pgfsetfillcolor{currentfill}%
\pgfsetlinewidth{0.000000pt}%
\definecolor{currentstroke}{rgb}{0.000000,0.000000,0.000000}%
\pgfsetstrokecolor{currentstroke}%
\pgfsetdash{}{0pt}%
\pgfpathmoveto{\pgfqpoint{8.929956in}{4.672281in}}%
\pgfpathlineto{\pgfqpoint{9.116182in}{4.672281in}}%
\pgfpathlineto{\pgfqpoint{9.116182in}{4.754009in}}%
\pgfpathlineto{\pgfqpoint{8.929956in}{4.754009in}}%
\pgfpathlineto{\pgfqpoint{8.929956in}{4.672281in}}%
\pgfusepath{fill}%
\end{pgfscope}%
\begin{pgfscope}%
\pgfpathrectangle{\pgfqpoint{0.549740in}{0.463273in}}{\pgfqpoint{9.320225in}{4.495057in}}%
\pgfusepath{clip}%
\pgfsetbuttcap%
\pgfsetroundjoin%
\pgfsetlinewidth{0.000000pt}%
\definecolor{currentstroke}{rgb}{0.000000,0.000000,0.000000}%
\pgfsetstrokecolor{currentstroke}%
\pgfsetdash{}{0pt}%
\pgfpathmoveto{\pgfqpoint{9.116182in}{4.672281in}}%
\pgfpathlineto{\pgfqpoint{9.302409in}{4.672281in}}%
\pgfpathlineto{\pgfqpoint{9.302409in}{4.754009in}}%
\pgfpathlineto{\pgfqpoint{9.116182in}{4.754009in}}%
\pgfpathlineto{\pgfqpoint{9.116182in}{4.672281in}}%
\pgfusepath{}%
\end{pgfscope}%
\begin{pgfscope}%
\pgfpathrectangle{\pgfqpoint{0.549740in}{0.463273in}}{\pgfqpoint{9.320225in}{4.495057in}}%
\pgfusepath{clip}%
\pgfsetbuttcap%
\pgfsetroundjoin%
\pgfsetlinewidth{0.000000pt}%
\definecolor{currentstroke}{rgb}{0.000000,0.000000,0.000000}%
\pgfsetstrokecolor{currentstroke}%
\pgfsetdash{}{0pt}%
\pgfpathmoveto{\pgfqpoint{9.302409in}{4.672281in}}%
\pgfpathlineto{\pgfqpoint{9.488635in}{4.672281in}}%
\pgfpathlineto{\pgfqpoint{9.488635in}{4.754009in}}%
\pgfpathlineto{\pgfqpoint{9.302409in}{4.754009in}}%
\pgfpathlineto{\pgfqpoint{9.302409in}{4.672281in}}%
\pgfusepath{}%
\end{pgfscope}%
\begin{pgfscope}%
\pgfpathrectangle{\pgfqpoint{0.549740in}{0.463273in}}{\pgfqpoint{9.320225in}{4.495057in}}%
\pgfusepath{clip}%
\pgfsetbuttcap%
\pgfsetroundjoin%
\pgfsetlinewidth{0.000000pt}%
\definecolor{currentstroke}{rgb}{0.000000,0.000000,0.000000}%
\pgfsetstrokecolor{currentstroke}%
\pgfsetdash{}{0pt}%
\pgfpathmoveto{\pgfqpoint{9.488635in}{4.672281in}}%
\pgfpathlineto{\pgfqpoint{9.674862in}{4.672281in}}%
\pgfpathlineto{\pgfqpoint{9.674862in}{4.754009in}}%
\pgfpathlineto{\pgfqpoint{9.488635in}{4.754009in}}%
\pgfpathlineto{\pgfqpoint{9.488635in}{4.672281in}}%
\pgfusepath{}%
\end{pgfscope}%
\begin{pgfscope}%
\pgfpathrectangle{\pgfqpoint{0.549740in}{0.463273in}}{\pgfqpoint{9.320225in}{4.495057in}}%
\pgfusepath{clip}%
\pgfsetbuttcap%
\pgfsetroundjoin%
\pgfsetlinewidth{0.000000pt}%
\definecolor{currentstroke}{rgb}{0.000000,0.000000,0.000000}%
\pgfsetstrokecolor{currentstroke}%
\pgfsetdash{}{0pt}%
\pgfpathmoveto{\pgfqpoint{9.674862in}{4.672281in}}%
\pgfpathlineto{\pgfqpoint{9.861088in}{4.672281in}}%
\pgfpathlineto{\pgfqpoint{9.861088in}{4.754009in}}%
\pgfpathlineto{\pgfqpoint{9.674862in}{4.754009in}}%
\pgfpathlineto{\pgfqpoint{9.674862in}{4.672281in}}%
\pgfusepath{}%
\end{pgfscope}%
\begin{pgfscope}%
\pgfsetbuttcap%
\pgfsetroundjoin%
\definecolor{currentfill}{rgb}{0.000000,0.000000,0.000000}%
\pgfsetfillcolor{currentfill}%
\pgfsetlinewidth{0.803000pt}%
\definecolor{currentstroke}{rgb}{0.000000,0.000000,0.000000}%
\pgfsetstrokecolor{currentstroke}%
\pgfsetdash{}{0pt}%
\pgfsys@defobject{currentmarker}{\pgfqpoint{0.000000in}{-0.048611in}}{\pgfqpoint{0.000000in}{0.000000in}}{%
\pgfpathmoveto{\pgfqpoint{0.000000in}{0.000000in}}%
\pgfpathlineto{\pgfqpoint{0.000000in}{-0.048611in}}%
\pgfusepath{stroke,fill}%
}%
\begin{pgfscope}%
\pgfsys@transformshift{0.549740in}{0.463273in}%
\pgfsys@useobject{currentmarker}{}%
\end{pgfscope}%
\end{pgfscope}%
\begin{pgfscope}%
\definecolor{textcolor}{rgb}{0.000000,0.000000,0.000000}%
\pgfsetstrokecolor{textcolor}%
\pgfsetfillcolor{textcolor}%
\pgftext[x=0.549740in,y=0.366051in,,top]{\color{textcolor}\sffamily\fontsize{10.000000}{12.000000}\selectfont 0}%
\end{pgfscope}%
\begin{pgfscope}%
\pgfsetbuttcap%
\pgfsetroundjoin%
\definecolor{currentfill}{rgb}{0.000000,0.000000,0.000000}%
\pgfsetfillcolor{currentfill}%
\pgfsetlinewidth{0.803000pt}%
\definecolor{currentstroke}{rgb}{0.000000,0.000000,0.000000}%
\pgfsetstrokecolor{currentstroke}%
\pgfsetdash{}{0pt}%
\pgfsys@defobject{currentmarker}{\pgfqpoint{0.000000in}{-0.048611in}}{\pgfqpoint{0.000000in}{0.000000in}}{%
\pgfpathmoveto{\pgfqpoint{0.000000in}{0.000000in}}%
\pgfpathlineto{\pgfqpoint{0.000000in}{-0.048611in}}%
\pgfusepath{stroke,fill}%
}%
\begin{pgfscope}%
\pgfsys@transformshift{2.413785in}{0.463273in}%
\pgfsys@useobject{currentmarker}{}%
\end{pgfscope}%
\end{pgfscope}%
\begin{pgfscope}%
\definecolor{textcolor}{rgb}{0.000000,0.000000,0.000000}%
\pgfsetstrokecolor{textcolor}%
\pgfsetfillcolor{textcolor}%
\pgftext[x=2.413785in,y=0.366051in,,top]{\color{textcolor}\sffamily\fontsize{10.000000}{12.000000}\selectfont 2}%
\end{pgfscope}%
\begin{pgfscope}%
\pgfsetbuttcap%
\pgfsetroundjoin%
\definecolor{currentfill}{rgb}{0.000000,0.000000,0.000000}%
\pgfsetfillcolor{currentfill}%
\pgfsetlinewidth{0.803000pt}%
\definecolor{currentstroke}{rgb}{0.000000,0.000000,0.000000}%
\pgfsetstrokecolor{currentstroke}%
\pgfsetdash{}{0pt}%
\pgfsys@defobject{currentmarker}{\pgfqpoint{0.000000in}{-0.048611in}}{\pgfqpoint{0.000000in}{0.000000in}}{%
\pgfpathmoveto{\pgfqpoint{0.000000in}{0.000000in}}%
\pgfpathlineto{\pgfqpoint{0.000000in}{-0.048611in}}%
\pgfusepath{stroke,fill}%
}%
\begin{pgfscope}%
\pgfsys@transformshift{4.277830in}{0.463273in}%
\pgfsys@useobject{currentmarker}{}%
\end{pgfscope}%
\end{pgfscope}%
\begin{pgfscope}%
\definecolor{textcolor}{rgb}{0.000000,0.000000,0.000000}%
\pgfsetstrokecolor{textcolor}%
\pgfsetfillcolor{textcolor}%
\pgftext[x=4.277830in,y=0.366051in,,top]{\color{textcolor}\sffamily\fontsize{10.000000}{12.000000}\selectfont 4}%
\end{pgfscope}%
\begin{pgfscope}%
\pgfsetbuttcap%
\pgfsetroundjoin%
\definecolor{currentfill}{rgb}{0.000000,0.000000,0.000000}%
\pgfsetfillcolor{currentfill}%
\pgfsetlinewidth{0.803000pt}%
\definecolor{currentstroke}{rgb}{0.000000,0.000000,0.000000}%
\pgfsetstrokecolor{currentstroke}%
\pgfsetdash{}{0pt}%
\pgfsys@defobject{currentmarker}{\pgfqpoint{0.000000in}{-0.048611in}}{\pgfqpoint{0.000000in}{0.000000in}}{%
\pgfpathmoveto{\pgfqpoint{0.000000in}{0.000000in}}%
\pgfpathlineto{\pgfqpoint{0.000000in}{-0.048611in}}%
\pgfusepath{stroke,fill}%
}%
\begin{pgfscope}%
\pgfsys@transformshift{6.141875in}{0.463273in}%
\pgfsys@useobject{currentmarker}{}%
\end{pgfscope}%
\end{pgfscope}%
\begin{pgfscope}%
\definecolor{textcolor}{rgb}{0.000000,0.000000,0.000000}%
\pgfsetstrokecolor{textcolor}%
\pgfsetfillcolor{textcolor}%
\pgftext[x=6.141875in,y=0.366051in,,top]{\color{textcolor}\sffamily\fontsize{10.000000}{12.000000}\selectfont 6}%
\end{pgfscope}%
\begin{pgfscope}%
\pgfsetbuttcap%
\pgfsetroundjoin%
\definecolor{currentfill}{rgb}{0.000000,0.000000,0.000000}%
\pgfsetfillcolor{currentfill}%
\pgfsetlinewidth{0.803000pt}%
\definecolor{currentstroke}{rgb}{0.000000,0.000000,0.000000}%
\pgfsetstrokecolor{currentstroke}%
\pgfsetdash{}{0pt}%
\pgfsys@defobject{currentmarker}{\pgfqpoint{0.000000in}{-0.048611in}}{\pgfqpoint{0.000000in}{0.000000in}}{%
\pgfpathmoveto{\pgfqpoint{0.000000in}{0.000000in}}%
\pgfpathlineto{\pgfqpoint{0.000000in}{-0.048611in}}%
\pgfusepath{stroke,fill}%
}%
\begin{pgfscope}%
\pgfsys@transformshift{8.005920in}{0.463273in}%
\pgfsys@useobject{currentmarker}{}%
\end{pgfscope}%
\end{pgfscope}%
\begin{pgfscope}%
\definecolor{textcolor}{rgb}{0.000000,0.000000,0.000000}%
\pgfsetstrokecolor{textcolor}%
\pgfsetfillcolor{textcolor}%
\pgftext[x=8.005920in,y=0.366051in,,top]{\color{textcolor}\sffamily\fontsize{10.000000}{12.000000}\selectfont 8}%
\end{pgfscope}%
\begin{pgfscope}%
\pgfsetbuttcap%
\pgfsetroundjoin%
\definecolor{currentfill}{rgb}{0.000000,0.000000,0.000000}%
\pgfsetfillcolor{currentfill}%
\pgfsetlinewidth{0.803000pt}%
\definecolor{currentstroke}{rgb}{0.000000,0.000000,0.000000}%
\pgfsetstrokecolor{currentstroke}%
\pgfsetdash{}{0pt}%
\pgfsys@defobject{currentmarker}{\pgfqpoint{0.000000in}{-0.048611in}}{\pgfqpoint{0.000000in}{0.000000in}}{%
\pgfpathmoveto{\pgfqpoint{0.000000in}{0.000000in}}%
\pgfpathlineto{\pgfqpoint{0.000000in}{-0.048611in}}%
\pgfusepath{stroke,fill}%
}%
\begin{pgfscope}%
\pgfsys@transformshift{9.869965in}{0.463273in}%
\pgfsys@useobject{currentmarker}{}%
\end{pgfscope}%
\end{pgfscope}%
\begin{pgfscope}%
\definecolor{textcolor}{rgb}{0.000000,0.000000,0.000000}%
\pgfsetstrokecolor{textcolor}%
\pgfsetfillcolor{textcolor}%
\pgftext[x=9.869965in,y=0.366051in,,top]{\color{textcolor}\sffamily\fontsize{10.000000}{12.000000}\selectfont 10}%
\end{pgfscope}%
\begin{pgfscope}%
\definecolor{textcolor}{rgb}{0.000000,0.000000,0.000000}%
\pgfsetstrokecolor{textcolor}%
\pgfsetfillcolor{textcolor}%
\pgftext[x=5.209852in,y=0.176083in,,top]{\color{textcolor}\sffamily\fontsize{10.000000}{12.000000}\selectfont time (s)}%
\end{pgfscope}%
\begin{pgfscope}%
\pgfsetbuttcap%
\pgfsetroundjoin%
\definecolor{currentfill}{rgb}{0.000000,0.000000,0.000000}%
\pgfsetfillcolor{currentfill}%
\pgfsetlinewidth{0.803000pt}%
\definecolor{currentstroke}{rgb}{0.000000,0.000000,0.000000}%
\pgfsetstrokecolor{currentstroke}%
\pgfsetdash{}{0pt}%
\pgfsys@defobject{currentmarker}{\pgfqpoint{-0.048611in}{0.000000in}}{\pgfqpoint{-0.000000in}{0.000000in}}{%
\pgfpathmoveto{\pgfqpoint{-0.000000in}{0.000000in}}%
\pgfpathlineto{\pgfqpoint{-0.048611in}{0.000000in}}%
\pgfusepath{stroke,fill}%
}%
\begin{pgfscope}%
\pgfsys@transformshift{0.549740in}{0.466598in}%
\pgfsys@useobject{currentmarker}{}%
\end{pgfscope}%
\end{pgfscope}%
\begin{pgfscope}%
\definecolor{textcolor}{rgb}{0.000000,0.000000,0.000000}%
\pgfsetstrokecolor{textcolor}%
\pgfsetfillcolor{textcolor}%
\pgftext[x=0.231638in, y=0.413836in, left, base]{\color{textcolor}\sffamily\fontsize{10.000000}{12.000000}\selectfont 0.0}%
\end{pgfscope}%
\begin{pgfscope}%
\pgfsetbuttcap%
\pgfsetroundjoin%
\definecolor{currentfill}{rgb}{0.000000,0.000000,0.000000}%
\pgfsetfillcolor{currentfill}%
\pgfsetlinewidth{0.803000pt}%
\definecolor{currentstroke}{rgb}{0.000000,0.000000,0.000000}%
\pgfsetstrokecolor{currentstroke}%
\pgfsetdash{}{0pt}%
\pgfsys@defobject{currentmarker}{\pgfqpoint{-0.048611in}{0.000000in}}{\pgfqpoint{-0.000000in}{0.000000in}}{%
\pgfpathmoveto{\pgfqpoint{-0.000000in}{0.000000in}}%
\pgfpathlineto{\pgfqpoint{-0.048611in}{0.000000in}}%
\pgfusepath{stroke,fill}%
}%
\begin{pgfscope}%
\pgfsys@transformshift{0.549740in}{1.021408in}%
\pgfsys@useobject{currentmarker}{}%
\end{pgfscope}%
\end{pgfscope}%
\begin{pgfscope}%
\definecolor{textcolor}{rgb}{0.000000,0.000000,0.000000}%
\pgfsetstrokecolor{textcolor}%
\pgfsetfillcolor{textcolor}%
\pgftext[x=0.231638in, y=0.968647in, left, base]{\color{textcolor}\sffamily\fontsize{10.000000}{12.000000}\selectfont 0.2}%
\end{pgfscope}%
\begin{pgfscope}%
\pgfsetbuttcap%
\pgfsetroundjoin%
\definecolor{currentfill}{rgb}{0.000000,0.000000,0.000000}%
\pgfsetfillcolor{currentfill}%
\pgfsetlinewidth{0.803000pt}%
\definecolor{currentstroke}{rgb}{0.000000,0.000000,0.000000}%
\pgfsetstrokecolor{currentstroke}%
\pgfsetdash{}{0pt}%
\pgfsys@defobject{currentmarker}{\pgfqpoint{-0.048611in}{0.000000in}}{\pgfqpoint{-0.000000in}{0.000000in}}{%
\pgfpathmoveto{\pgfqpoint{-0.000000in}{0.000000in}}%
\pgfpathlineto{\pgfqpoint{-0.048611in}{0.000000in}}%
\pgfusepath{stroke,fill}%
}%
\begin{pgfscope}%
\pgfsys@transformshift{0.549740in}{1.576218in}%
\pgfsys@useobject{currentmarker}{}%
\end{pgfscope}%
\end{pgfscope}%
\begin{pgfscope}%
\definecolor{textcolor}{rgb}{0.000000,0.000000,0.000000}%
\pgfsetstrokecolor{textcolor}%
\pgfsetfillcolor{textcolor}%
\pgftext[x=0.231638in, y=1.523457in, left, base]{\color{textcolor}\sffamily\fontsize{10.000000}{12.000000}\selectfont 0.4}%
\end{pgfscope}%
\begin{pgfscope}%
\pgfsetbuttcap%
\pgfsetroundjoin%
\definecolor{currentfill}{rgb}{0.000000,0.000000,0.000000}%
\pgfsetfillcolor{currentfill}%
\pgfsetlinewidth{0.803000pt}%
\definecolor{currentstroke}{rgb}{0.000000,0.000000,0.000000}%
\pgfsetstrokecolor{currentstroke}%
\pgfsetdash{}{0pt}%
\pgfsys@defobject{currentmarker}{\pgfqpoint{-0.048611in}{0.000000in}}{\pgfqpoint{-0.000000in}{0.000000in}}{%
\pgfpathmoveto{\pgfqpoint{-0.000000in}{0.000000in}}%
\pgfpathlineto{\pgfqpoint{-0.048611in}{0.000000in}}%
\pgfusepath{stroke,fill}%
}%
\begin{pgfscope}%
\pgfsys@transformshift{0.549740in}{2.131029in}%
\pgfsys@useobject{currentmarker}{}%
\end{pgfscope}%
\end{pgfscope}%
\begin{pgfscope}%
\definecolor{textcolor}{rgb}{0.000000,0.000000,0.000000}%
\pgfsetstrokecolor{textcolor}%
\pgfsetfillcolor{textcolor}%
\pgftext[x=0.231638in, y=2.078267in, left, base]{\color{textcolor}\sffamily\fontsize{10.000000}{12.000000}\selectfont 0.6}%
\end{pgfscope}%
\begin{pgfscope}%
\pgfsetbuttcap%
\pgfsetroundjoin%
\definecolor{currentfill}{rgb}{0.000000,0.000000,0.000000}%
\pgfsetfillcolor{currentfill}%
\pgfsetlinewidth{0.803000pt}%
\definecolor{currentstroke}{rgb}{0.000000,0.000000,0.000000}%
\pgfsetstrokecolor{currentstroke}%
\pgfsetdash{}{0pt}%
\pgfsys@defobject{currentmarker}{\pgfqpoint{-0.048611in}{0.000000in}}{\pgfqpoint{-0.000000in}{0.000000in}}{%
\pgfpathmoveto{\pgfqpoint{-0.000000in}{0.000000in}}%
\pgfpathlineto{\pgfqpoint{-0.048611in}{0.000000in}}%
\pgfusepath{stroke,fill}%
}%
\begin{pgfscope}%
\pgfsys@transformshift{0.549740in}{2.685839in}%
\pgfsys@useobject{currentmarker}{}%
\end{pgfscope}%
\end{pgfscope}%
\begin{pgfscope}%
\definecolor{textcolor}{rgb}{0.000000,0.000000,0.000000}%
\pgfsetstrokecolor{textcolor}%
\pgfsetfillcolor{textcolor}%
\pgftext[x=0.231638in, y=2.633078in, left, base]{\color{textcolor}\sffamily\fontsize{10.000000}{12.000000}\selectfont 0.8}%
\end{pgfscope}%
\begin{pgfscope}%
\pgfsetbuttcap%
\pgfsetroundjoin%
\definecolor{currentfill}{rgb}{0.000000,0.000000,0.000000}%
\pgfsetfillcolor{currentfill}%
\pgfsetlinewidth{0.803000pt}%
\definecolor{currentstroke}{rgb}{0.000000,0.000000,0.000000}%
\pgfsetstrokecolor{currentstroke}%
\pgfsetdash{}{0pt}%
\pgfsys@defobject{currentmarker}{\pgfqpoint{-0.048611in}{0.000000in}}{\pgfqpoint{-0.000000in}{0.000000in}}{%
\pgfpathmoveto{\pgfqpoint{-0.000000in}{0.000000in}}%
\pgfpathlineto{\pgfqpoint{-0.048611in}{0.000000in}}%
\pgfusepath{stroke,fill}%
}%
\begin{pgfscope}%
\pgfsys@transformshift{0.549740in}{3.240650in}%
\pgfsys@useobject{currentmarker}{}%
\end{pgfscope}%
\end{pgfscope}%
\begin{pgfscope}%
\definecolor{textcolor}{rgb}{0.000000,0.000000,0.000000}%
\pgfsetstrokecolor{textcolor}%
\pgfsetfillcolor{textcolor}%
\pgftext[x=0.231638in, y=3.187888in, left, base]{\color{textcolor}\sffamily\fontsize{10.000000}{12.000000}\selectfont 1.0}%
\end{pgfscope}%
\begin{pgfscope}%
\pgfsetbuttcap%
\pgfsetroundjoin%
\definecolor{currentfill}{rgb}{0.000000,0.000000,0.000000}%
\pgfsetfillcolor{currentfill}%
\pgfsetlinewidth{0.803000pt}%
\definecolor{currentstroke}{rgb}{0.000000,0.000000,0.000000}%
\pgfsetstrokecolor{currentstroke}%
\pgfsetdash{}{0pt}%
\pgfsys@defobject{currentmarker}{\pgfqpoint{-0.048611in}{0.000000in}}{\pgfqpoint{-0.000000in}{0.000000in}}{%
\pgfpathmoveto{\pgfqpoint{-0.000000in}{0.000000in}}%
\pgfpathlineto{\pgfqpoint{-0.048611in}{0.000000in}}%
\pgfusepath{stroke,fill}%
}%
\begin{pgfscope}%
\pgfsys@transformshift{0.549740in}{3.795460in}%
\pgfsys@useobject{currentmarker}{}%
\end{pgfscope}%
\end{pgfscope}%
\begin{pgfscope}%
\definecolor{textcolor}{rgb}{0.000000,0.000000,0.000000}%
\pgfsetstrokecolor{textcolor}%
\pgfsetfillcolor{textcolor}%
\pgftext[x=0.231638in, y=3.742699in, left, base]{\color{textcolor}\sffamily\fontsize{10.000000}{12.000000}\selectfont 1.2}%
\end{pgfscope}%
\begin{pgfscope}%
\pgfsetbuttcap%
\pgfsetroundjoin%
\definecolor{currentfill}{rgb}{0.000000,0.000000,0.000000}%
\pgfsetfillcolor{currentfill}%
\pgfsetlinewidth{0.803000pt}%
\definecolor{currentstroke}{rgb}{0.000000,0.000000,0.000000}%
\pgfsetstrokecolor{currentstroke}%
\pgfsetdash{}{0pt}%
\pgfsys@defobject{currentmarker}{\pgfqpoint{-0.048611in}{0.000000in}}{\pgfqpoint{-0.000000in}{0.000000in}}{%
\pgfpathmoveto{\pgfqpoint{-0.000000in}{0.000000in}}%
\pgfpathlineto{\pgfqpoint{-0.048611in}{0.000000in}}%
\pgfusepath{stroke,fill}%
}%
\begin{pgfscope}%
\pgfsys@transformshift{0.549740in}{4.350271in}%
\pgfsys@useobject{currentmarker}{}%
\end{pgfscope}%
\end{pgfscope}%
\begin{pgfscope}%
\definecolor{textcolor}{rgb}{0.000000,0.000000,0.000000}%
\pgfsetstrokecolor{textcolor}%
\pgfsetfillcolor{textcolor}%
\pgftext[x=0.231638in, y=4.297509in, left, base]{\color{textcolor}\sffamily\fontsize{10.000000}{12.000000}\selectfont 1.4}%
\end{pgfscope}%
\begin{pgfscope}%
\pgfsetbuttcap%
\pgfsetroundjoin%
\definecolor{currentfill}{rgb}{0.000000,0.000000,0.000000}%
\pgfsetfillcolor{currentfill}%
\pgfsetlinewidth{0.803000pt}%
\definecolor{currentstroke}{rgb}{0.000000,0.000000,0.000000}%
\pgfsetstrokecolor{currentstroke}%
\pgfsetdash{}{0pt}%
\pgfsys@defobject{currentmarker}{\pgfqpoint{-0.048611in}{0.000000in}}{\pgfqpoint{-0.000000in}{0.000000in}}{%
\pgfpathmoveto{\pgfqpoint{-0.000000in}{0.000000in}}%
\pgfpathlineto{\pgfqpoint{-0.048611in}{0.000000in}}%
\pgfusepath{stroke,fill}%
}%
\begin{pgfscope}%
\pgfsys@transformshift{0.549740in}{4.905081in}%
\pgfsys@useobject{currentmarker}{}%
\end{pgfscope}%
\end{pgfscope}%
\begin{pgfscope}%
\definecolor{textcolor}{rgb}{0.000000,0.000000,0.000000}%
\pgfsetstrokecolor{textcolor}%
\pgfsetfillcolor{textcolor}%
\pgftext[x=0.231638in, y=4.852320in, left, base]{\color{textcolor}\sffamily\fontsize{10.000000}{12.000000}\selectfont 1.6}%
\end{pgfscope}%
\begin{pgfscope}%
\definecolor{textcolor}{rgb}{0.000000,0.000000,0.000000}%
\pgfsetstrokecolor{textcolor}%
\pgfsetfillcolor{textcolor}%
\pgftext[x=0.176083in,y=2.710802in,,bottom,rotate=90.000000]{\color{textcolor}\sffamily\fontsize{10.000000}{12.000000}\selectfont latency (s)}%
\end{pgfscope}%
\begin{pgfscope}%
\pgfsetrectcap%
\pgfsetmiterjoin%
\pgfsetlinewidth{0.803000pt}%
\definecolor{currentstroke}{rgb}{0.000000,0.000000,0.000000}%
\pgfsetstrokecolor{currentstroke}%
\pgfsetdash{}{0pt}%
\pgfpathmoveto{\pgfqpoint{0.549740in}{0.463273in}}%
\pgfpathlineto{\pgfqpoint{0.549740in}{4.958330in}}%
\pgfusepath{stroke}%
\end{pgfscope}%
\begin{pgfscope}%
\pgfsetrectcap%
\pgfsetmiterjoin%
\pgfsetlinewidth{0.803000pt}%
\definecolor{currentstroke}{rgb}{0.000000,0.000000,0.000000}%
\pgfsetstrokecolor{currentstroke}%
\pgfsetdash{}{0pt}%
\pgfpathmoveto{\pgfqpoint{9.869965in}{0.463273in}}%
\pgfpathlineto{\pgfqpoint{9.869965in}{4.958330in}}%
\pgfusepath{stroke}%
\end{pgfscope}%
\begin{pgfscope}%
\pgfsetrectcap%
\pgfsetmiterjoin%
\pgfsetlinewidth{0.803000pt}%
\definecolor{currentstroke}{rgb}{0.000000,0.000000,0.000000}%
\pgfsetstrokecolor{currentstroke}%
\pgfsetdash{}{0pt}%
\pgfpathmoveto{\pgfqpoint{0.549740in}{0.463273in}}%
\pgfpathlineto{\pgfqpoint{9.869965in}{0.463273in}}%
\pgfusepath{stroke}%
\end{pgfscope}%
\begin{pgfscope}%
\pgfsetrectcap%
\pgfsetmiterjoin%
\pgfsetlinewidth{0.803000pt}%
\definecolor{currentstroke}{rgb}{0.000000,0.000000,0.000000}%
\pgfsetstrokecolor{currentstroke}%
\pgfsetdash{}{0pt}%
\pgfpathmoveto{\pgfqpoint{0.549740in}{4.958330in}}%
\pgfpathlineto{\pgfqpoint{9.869965in}{4.958330in}}%
\pgfusepath{stroke}%
\end{pgfscope}%
\end{pgfpicture}%
\makeatother%
\endgroup%
}
\caption{Heatmap of an experiment where a node is killed.}
\label{viewchangeheatmap}
\end{figure}

This study explores the behaviour of the system in the event of a node failing, where the view-change protocol (Section~\ref{viewchange}) is needed to skip the faulty leader's view and make progress.

Figure~\ref{viewchangeheatmap} shows an experiment that was run for 10s on a network of 7 nodes with a batch size of 300. A node was killed 5s into the experiment. The view timeout was set to 0.5s.

There is a clear jump in latency every time the killed node is the leader of the view, and the nodes must wait for the 0.5s timeout to elapse before the next view begins. The latency jump of roughly 0.7s is about what one would expect; 0.5s timeout and 0.2s of latency (the same as before the node was killed). Latency gradually increases after this point as requests begin to queue on the nodes, incurring some overhead.

The view-change protocol is successful in allowing the system to make progress, albeit with a significant increase in latency. This is an inherent problem with the view-change protocol, although a failure-detector could help to detect that a node has failed and skip its view without waiting for a timeout to elapse, allowing the latency to return to a stable value.

% Clone github.com/cjen1/reckon

% ```
% # This is likely to take a while
% make reckon-mininet

% docker run -it --privileged -e DISPLAY --network host --name reckon-mininet cjen1/reckon:latest bash

% # Set up mininet net with a single switch and 3 nodes
% # drops you into a cli (you can also use python scripting)
% mn --topo single,3

% # observe no delay between nodes
% mininet> h1 ping h2
% mininet> <Ctrl-C>/<Ctrl-D> to exit

% # syntax is `mininet> <node> <command>`
% # I run screens on each node and then attach to those from outside mininet to run the tests in different terminal screens. (Tmux doesn't work correctly afaicr)

% mininet> h1 screen -dmS node_h1 bash

% #Then in another terminal session
% docker exec -it reckon-mininet bash
% screen -r node_h1
% <whatever commands you want to run on that emulated node>

% #Similarly for the other nodes
% ```