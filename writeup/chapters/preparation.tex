% Principally, this chapter should describe the work which was undertaken before code was written, hardware built or theories worked on. It should show how the project proposal was further refined and clarified, so that the implementation stage could go smoothly rather than by trial and error.

% Throughout this chapter and indeed the whole dissertation, it is essential to demonstrate that a proper professional approach was employed.

% The nature of this chapter will vary greatly from one dissertation to another but, underlining the professional approach, this chapter will very likely include a section headed "Requirements Analysis" and refer to appropriate software engineering techniques used in the dissertation. The chapter will also cite any new programming languages and systems which had to be learnt and will mention complicated theories or algorithms which required understanding.

% It is essential to declare the starting point. This states any existing codebase or materials that your project builds on. The text here can commonly be identical to the text in your proposal, but it may enlarge on it or report variations. For instance, the true starting point may have turned out to be different from that declared in the proposal and such discrepancies must be explained.

% hotstuff algorithm
% why Ocaml - mirage etc.
% chubby (Google) - consensus is hard
% related works

\textit{In this chapter we disclose my knowledge and experience prior to beginning this project [\ref{start}], give a theoretical basis for understanding the HotStuff algorithm [\ref{hotstufftheory}], outline the tools, libraries [\ref{tools}], and methodology [\ref{softwareeng}] deployed in implementation, and highlight the requirements that the implementation should meet [\ref{requirements}].}

\section{Starting point} \label{start}
I had some experience using OCaml from the IA Foundations of Computer Science course but had never used it in a project. The IB Distributed Systems course also provided some useful background knowledge, particularly as it briefly covered Raft \cite{ongaro_search_nodate}, a non-byzantine consensus algorithm. I had some understanding of byzantine consensus from my own reading into Nakamoto consensus \cite{nakamoto_bitcoin_nodate} and from developing a wallet application for Ethereum \cite{wood_ethereum_nodate}; neither of these was directly useful to implementing HotStuff, but they gave me some wider context of the field.

% the whole implementation was written from scratch??

\section{HotStuff algorithm} \label{hotstufftheory}
HotStuff is a byzantine consensus algorithm; it allows a group of nodes to reach agreement on a log of values under adverse conditions, such as messages being lost, or some nodes being byzantine. In each \textit{view} a \textit{leader} node proposes some value by sending it the \textit{replicas} (another word for nodes). After several messages are exchanged the log may be committed, meaning that there is consensus on the committed part of the log, and it is now immutable.

HotStuff has the following properties:
\begin{itemize}
	\item Safety --- once a log has been committed up to some point that part of the log is immutable, it can only appended to.
	\item Liveness --- the system is guaranteed to make progress once a non-faulty leader is elected.
	\item Responsiveness\footnote{To be precise HotStuff is \textit{optimistically} responsive, we will describe this in section \ref{optresponsive}.} --- the system is able to make progress as fast as network conditions allow once a non-faulty leader is elected; it does not have to wait for some timeout to elapse.
\end{itemize}

The system model describes the adverse conditions under which HotStuff can operate:
\begin{itemize}
	\item Partially Synchronous --- messages sent by one party will always be delivered to another within some bounded amount of time ($\delta$) after global synchronisation time (GST) has been reached.
	\item Authenticated --- a message source cannot be spoofed. We assume that all messages are signed, providing authentication.
	\item Byzantine --- a maximum of $f$ faulty nodes may be controlled by an adversary that is actively trying to prevent the nodes from reaching consensus, where $n = 3f + 1$ and $n$ is the total number of nodes. This is the maximum number of byzantine nodes for which consensus can theoretically be reached \cite{pease_reaching_1980}\cite{fischer_easy_nodate}.
\end{itemize}

Each view is composed of several \textit{phases}. In each phase the leader broadcasts to the replicas, that respond with an acknowledgement (\textit{ack}). The leader waits until it receives a \textit{quorum} of $n - f$ acks\footnote{N.B. the size of a quorum is different in the non-byzantine case.} before proceeding to the next phase, meaning that at least $f + 1$ of the acks were from honest nodes. The basic HotStuff algorithm has five phases in each view.

The final phase of a consensus algorithm is \textit{decide}. We assume that a \textit{client} sends commands to the nodes, the nodes reach consensus on a log of commands, and then execute them in order; this ensures all nodes will be in the same state. The decide phase commences after the log is committed, the nodes can execute the new commands and respond to the client that the command was successfully committed.

The HotStuff algorithm consists of a responsive byzantine consensus algorithm (section \ref{optresponsive}), with a view-change protocol (section \ref{viewchange}) that allows the system to skip the view of a faulty leader. We will start by describing a simpler consensus algorithm that is not responsive, and cannot handle byzantine nodes (section \ref{nonbyzconsensus}). We then extend this algorithm to handle byzantine threats (section \ref{byzconsensus}). Finally, we extend this algorithm to be responsive by removing the need for a timeout to progress (section \ref{optresponsive}).

\subsection{Non-byzantine consensus} \label{nonbyzconsensus}
We will start by describing an algorithm to solve the simpler problem of reaching consensus with the stronger assumption of a crash-stop model instead of a byzantine one; this means that we assume nodes cannot be malicious but they can still crash and never come back online. Examples of similar algorithms include Raft \cite{ongaro_search_nodate} and multi-shot Paxos \cite{lamport_part-time_1998}\cite{lamport_paxos_2001}.

Each view consists of three phases:

\textit{new-view} --- The leader learns about previously committed logs. All replicas send a \textit{new-view} message to the leader, containing their longest previously committed log.

\textit{commit} --- The leader waits until it receives a quorum of greater than $\frac{n}{2}$ \textit{new-view} messages, and then picks the longest log it received ($\lambda$) to propose. It then broadcasts a \textit{commit} message to the replicas, proposing $\lambda'$ to the nodes, where $\lambda'$ is $\lambda$ optionally extended with the leader's new value.

\textit{decide} --- Once the leader receives a quorum of acks, the log has been successfully committed. The leader can broadcast ``decide'' to the replicas, who can execute the new commands and respond to the client.

Crucially, this algorithm has the \textit{safety} property. Since the leader waits to receive a quorum of greater than $\frac{n}{2}$ \textit{new-view} messages, $\lambda$ is guaranteed to be the longest log that has previously been committed. This is because the the quorum of \textit{new-view} messages must share at least one node with the past quorum of \textit{commit} acks for $\lambda$. The new proposal $\lambda'$ will never conflict with $\lambda$, hence the algorithm is safe.
%Each view is broken into two phases; the \textit{new-view} phase allows the leader to learn of previously committed values, and in \textit{commit} phase the leader proposes a log extended with new values that the replicas can reach consensus on. The \textit{new-view} phase is initiated by the leader broadcasting the current view number to the replicas [we don't actually do this, is it necessary???], which respond by sending their longest accepted log (the one with the highest corresponding view number). Once the leader has a quorum of responses, the \textit{commit} phase commences: the leader selects the longest log that has been sent to it and broadcasts it to the replicas. The leader may also extend the log at this point with its own values, or create a new log if it did not receive anything. Finally, each replica updates its log to the value sent by the leader and sends an acknowledgement. Once the leader receives a quorum of acknowledgements it can commit the new log. This algorithm satisfies our requirement that a committed log can only be extended.

%We will refer to phase 1 as the  stage, and phase 2 as the \textit{commit} phase to use the terminology of the HotStuff paper. These are followed by the \textit{decide} phase, in which the leader sends a decide message to replicas. Once they receive this message they can consider the log decided, and execute the new commands which have been added to the log.
% See an example of a non-byzantine consensus algorithm in figure \ref{nonbyzantineexample}.
% \begin{figure}[h!] 
% \texttt{
% \begin{enumerate}
% \item New-view: 
% 	\begin{enumerate}
% 	\item leader $\to$ replicas: ``view = 3" 
% 	\item replicas $\to$ leader: ``view = 2, log = [`hello', `world']", ...
% 	\end{enumerate}
% \item Commit:
% 	\begin{enumerate}
% 	\item leader $\to$ replicas: propose ``log = [`hello', `world', `!']"
% 	\item replicas $\to$ leader: ``ack", \dots
% 	\end{enumerate}
% \item Decide:
% 	\begin{enumerate}
% 	\item leader $\to$ replicas: ``decide"
% 	\item replicas: execute log
% 	\end{enumerate}
% \end{enumerate}
% }
% \caption{Example of crash-stop consensus algorithm.}
% \label{nonbyzantineexample}
% \end{figure}

\subsection{Byzantine consensus} \label{byzconsensus}
In this section we extend our non-byzantine algorithm (section \ref{nonbyzconsensus}) to achieve consensus under a byzantine system model. To do this we must first introduce the \textit{quorum certificate} (\textit{QC}), a cryptographic proof that a leader has received a quorum of acks. We then consider the threats posed by byzantine nodes, give an algorithm that solves these problems, and make an argument for safety. Examples of similar algorithms include xyz

\subsubsection{Quorum certificates}

A QC is a quorum of $n - f$ acks with a matching \textit{threshold signature}. A threshold signature combines several signatures of the same message into one \cite{goos_practical_2000}\cite{cachin_random_2005}; in this case the signature from each ack is combined\footnote{Recall that our messages are signed to provide authenticated delivery.}. QCs have a key property that our byzantine algorithm will rely on:

[add a diagram of intersecting QCs!]

\begin{property} \label{qcproperty}
There will always be at least one honest node in the intersection of any two QCs.
\end{property}

Recall that $n = 3f + 1$. The property holds since a QC contains a quorum of $n - f$ acks, at least $f + 1$ of which must be from honest nodes. To have two quorums that \textit{do not} share an honest node would require at least $2(f + 1) = 2f + 2$ honest nodes, but our system only has $2f + 1$.

\subsubsection{Threats}
Byzantine nodes introduce two threats that we will deal with in turn. We present ideas for solutions to these threats; it will become clear why these solutions are effective in our argument that the algorithm is safe.

\begin{enumerate}
\item Threat (Equivocation) --- a faulty leader proposes one value to some replicas and a different value to others. For example, in the case of a cryptocurrency a malicious actor (Mallory) could carry out a double-spend attack by proposing ``Mallory transfers Alice £10" to some nodes and ``Mallory transfers Bob £10" to others, even if Mallory's account contains less than £20. \label{threat1}

Solution idea --- add a new stage \textit{prepare} which happens just before the \textit{commit} phase, where the leader pre-proposes a log before proposing it in the \textit{commit} phase.

\item Threat --- a faulty leader proposes a log that conflicts with one that has previously been committed. \label{threat2}

Solution idea --- when replicas receive a proposal for some log they must \textit{lock} on it, and not accept a pre-proposal for a conflicting log.
\end{enumerate}

\subsubsection{Algorithm}
% See an example of a byzantine consensus algorithm in figure \ref{byzantineexample}.
Modifying the non-byzantine algorithm to include our solutions to threats \ref{threat1} and \ref{threat2} results in the following:

\textit{new-view} --- the leader learns about previously committed logs. All replicas send a \textit{new-view} message to the leader, containing their longest previously committed log.

\textit{prepare} --- the leader waits $\delta$ until it receives a \textit{new-view} message from all replicas (we will revisit this in section \ref{optresponsive}), and then picks the longest log it received ($\lambda$) to pre-propose. It then broadcasts a \textit{prepare} message to the replicas, pre-proposing $\lambda'$ to the nodes, where $\lambda'$ is $\lambda$ optionally extended with the leader's new value. The replicas ensure that $\lambda'$ does not conflict with their \textit{lock} before sending an ack.

\textit{commit} --- the leader waits until it receives a quorum of at least $n - f$ \textit{prepare} acks. It then broadcasts a \textit{commit} message to the replicas, proposing $\lambda'$ to the nodes, and includes a QC of \textit{prepare} acks. The replicas then \textit{lock} on $\lambda'$ and send a \textit{commit} ack.

\textit{decide} --- once the leader receives a quorum of acks, $\lambda'$ has been successfully committed. The leader can broadcast ``decide'' to the replicas, who can execute the new commands and respond to the client.

\subsubsection{Argument for safety}
We give an informal inductive argument of the safety of this algorithm based on it solving threats \ref{threat1} and \ref{threat2}. To be safe, we must have that if some log $\lambda$ is committed in view $v$, at no point in future will some conflicting log be committed.

Base case --- in view $v$ no log that conflicts with $\lambda$ may be committed, in other words equivocation (threat \ref{threat1}) is not possible. For a view to commit a value, there must have been a QC of \textit{prepare} acks, and a QC of \textit{commit} acks. By property \ref{qcproperty}, there must be at least one honest replica in the intersection of these QCs that would not have acknowledged conflicting proposals in the same view.

Inductive step --- no conflicting log can be proposed in view $v'$ where $v' > v$ (threat \ref{threat2}). This holds because any log pre-proposed in view $v'$ must receive a QC of \textit{prepare} acks; by property \ref{qcproperty}, there must be at least one honest node in the intersection between this QC, and the QC of \textit{commit} acks $\lambda$ in view $v$. This honest replica is locked on $\lambda$, so would not accept a proposal that conflicts it.

From this it follows that in no view from $v$ onwards will a log conflicting with $\lambda$ be committed, so the algorithm is safe.

% \begin{figure}[h!] 
% \texttt{
% \begin{enumerate}
% \item New-view: 
% 	\begin{enumerate}
% 	\item leader $\to$ replicas: ``view = 3" 
% 	\item replicas $\to$ leader: ``view = 2, log = [`hello', `world']", \dots
% 	\item leader: waits $\Delta$ to hear from all non-faulty replicas
% 	\end{enumerate}
% \item Prepare:
% 	\begin{enumerate}
% 	\item leader $\to$ replicas: pre-propose ``log = [`hello', `world', `!']"
% 	\item replicas verify the pre-proposal does not conflict with their \textit{lock}
% 	\item replicas $\to$ leader: ``log = [`hello', `world', `!'] ack", \dots
% 	\end{enumerate}
% \item Commit (lock):
% 	\begin{enumerate}
% 	\item leader $\to$ replicas: propose ``log = [`hello', `world', `!']", qc = (\textit{prepare} acks from previous stage)"
% 	\item replicas \textit{lock} on the proposed log
% 	\item replicas $\to$ leader: ``ack", \dots
% 	\end{enumerate}
% \item Decide:
% 	\begin{enumerate}
% 	\item leader $\to$ replicas: ``decide, qc = (\textit{commit} acks from previous stage)"
% 	\item replicas: execute log
% 	\end{enumerate}
% \end{enumerate}
% }
% \caption{Example of byzantine fault-tolerant consensus algorithm.}
% \label{byzantineexample}
% \end{figure}

\subsection{Optimistic responsiveness} \label{optresponsive}
The HotStuff protocol has the \textit{optimistic responsiveness} property, which means that it can make progress as fast as network conditions allow without waiting for a timeout [cite thunderella]. It is called \textit{optimistic} as it is only guaranteed to have this property once GST has been reached.

The algorithm described in section \ref{byzconsensus} is not responsive, as it has a timeout in the \textit{prepare} phase. We first describe the problem that means this timeout is needed, present an algorithm that solves it, and give an informal argument of its effectiveness.

[rewrite all this!]

\subsubsection{Problem}
Consider again the \textit{prepare} phase; the leader includes in its pre-proposal the QC from the highest view that it hears about in the \textit{new-view} messages it receives. However, there may be some honest replica [how???] that the leader did not hear from (perhaps their message was lost) that is locked on a higher-view proposal than the one that the leader chooses to pre-propose. When this replica receives the pre-proposal, it will reject it as it is locked on a higher-view proposal, and the system will not make progress. This is why it is necessary to wait for a timeout $\Delta$, so that the leader waits a sufficient amount of time to receive a \textit{new-view} from all replicas.

Solution idea --- blah blah

\subsubsection{Algorithm}
%[HotStuff does \textit{optimistic} responsiveness]In order to achieve responsiveness we can modify our algorithm by adding a \textit{pre-commit} phase in between \textit{prepare} and \textit{commit}, and removing the requirement for the leader to wait for a timeout in the \textit{new-view} phase. As before, there may be some honest replica that becomes locked on some proposal $x$ during the \textit{commit} phase. However, now there is a \textit{pre-commit} phase before this, where a quorum of replicas all store a \textit{key} for $x$, with this \textit{key} being a QC of \textit{prepare} acks. We now consider the next \textit{new-view} phase; the replicas now send their \textit{key} to the new leader, and the leader chooses what to propose based on the highest view \textit{key} it sees. Even if the leader does not receive a \textit{new-view} from the honest replica itself, it is guaranteed to hear about the proposal $x$ [because the quorums intersect!!!], so it will be able to make progress. See an example in figure \ref{responsiveeexample}.
\textit{new-view} --- the leader learns about previously committed logs. All replicas send a \textit{new-view} message to the leader, containing their longest previously committed log.

\textit{prepare} --- the leader waits $\delta$ until it receives a \textit{new-view} message from all replicas (we will revisit this in section \ref{optresponsive}), and then picks the longest log it received ($\lambda$) to pre-propose. It then broadcasts a \textit{prepare} message to the replicas, pre-proposing $\lambda'$ to the nodes, where $\lambda'$ is $\lambda$ optionally extended with the leader's new value. The replicas ensure that $\lambda'$ does not conflict with their \textit{lock} before sending an ack.

\textit{pre-commit} --- blah blah

\textit{commit} --- the leader waits until it receives a quorum of at least $n - f$ \textit{prepare} acks. It then broadcasts a \textit{commit} message to the replicas, proposing $\lambda'$ to the nodes, and includes a QC of \textit{prepare} acks. The replicas then \textit{lock} on $\lambda'$ and send a \textit{commit} ack.

\textit{decide} --- once the leader receives a quorum of acks, $\lambda'$ has been successfully committed. The leader can broadcast ``decide'' to the replicas, who can execute the new commands and respond to the client.

\subsubsection{Argument for responsiveness}
This ensures that if some honest replica becomes locked on a value in the \textit{commit} phase, then there are at least $f + 1$ honest nodes that have a `key' for that value from the \textit{pre-commit} phase. More specifically, they have a `key-proof' composed of a QC over \textit{pre-commit} acks. Replicas then send this key-proof with their \textit{new-view} message when a new view begins. The new leader selects the key-proof with the highest view proposal and sends it along with their \textit{prepare} message. Even if the leader does not receive a \textit{new-view} message from the replica that is locked on the highest view proposal, they must have received the key to this proposal from some honest replica, so their \textit{prepare} message will make progress.

% \begin{figure}[h!]
% \texttt{
% \begin{enumerate}
% \item New view: 
% 	\begin{enumerate}
% 	\item leader $\to$ replicas: ``view = 3" 
% 	\item replicas $\to$ leader: ``qc = (\textit{key} for view = 2, log = [`hello', `world'])", \dots
% 	\end{enumerate}
% \item Prepare:
% 	\begin{enumerate}
% 	\item leader $\to$ replicas: pre-propose ``log = [`hello', `world', `!'], qc = (key for view = 2, log = [`hello', `world'])"
% 	\item replicas verify the pre-proposal does not conflict with their \textit{lock}
% 	\item replicas $\to$ leader: ``log = [`hello', `world', `!'] ack", \dots
% 	\end{enumerate}
% \item Pre-commit (key):
% 	\begin{enumerate}
% 	\item leader $\to$ replicas: ``qc = (\textit{prepare} acks from previous stage)"
% 	\item replicas store qc as a \textit{key}
% 	\item replicas $\to$ leader: ``log = [`hello', `world', `!'] ack", \dots
% 	\end{enumerate}
% \item Commit (lock):
% 	\begin{enumerate}
% 	\item leader $\to$ replicas: propose ``qc = (\textit{pre-commit} acks from previous stage)"
% 	\item replicas \textit{lock} on proposed log
% 	\item replicas $\to$ leader: ``ack", \dots
% 	\end{enumerate}
% \item Decide:
% 	\begin{enumerate}
% 	\item leader $\to$ replicas: ``decide, qc = (\textit{commit} acks from previous stage)"
% 	\item replicas: execute log
% 	\end{enumerate}
% \end{enumerate}
% }
% \caption{Example of responsive consensus algorithm.}
% \label{responsiveeexample}
% \end{figure}

\subsection{View-changes} \label{viewchange}
If a leader fails to make progress within some timeout a view-change takes place and the next view begins. The HotStuff paper does not go into detail on how view-changes take place, so this explanation is based on a talk by one of its authors, Ittai Abraham \cite{ittai}, and is similar to the approach used by LibraBFT \cite{baudet_state_nodate}. [mention cogsworth]

Once the view times out, nodes send a \textit{complain} message to the next leader and start a new timeout for the next view. Once the next leader achieves a quorum of \textit{complain} messages it collects them into a QC known as a \textit{view-change proof}. This leader can then send a \textit{view-change} message containing the \textit{view-change proof} to all replicas, who will respond by transitioning to the next view and sending a \textit{new-view} message to the new leader. The inclusion of the \textit{view-change proof} prevents liveness attacks by byzantine nodes that could otherwise attack the system by constantly causing view-changes to take place and preventing non-faulty leaders from making progress.

The use of timeouts in this way is a type of failure detector, which is a system that facilitates the detection of failed nodes \cite{chandra_weakest_1996}\cite{chandra_unreliable_1996}.

\section{Tools \& libraries} \label{tools}
\textit{In this section we outline the languages and libraries used in implementation, and justify why they were appropriate for this project.}

\subsection{OCaml}
I chose OCaml \cite{noauthor_ocaml_nodate} for this project due to its high-level nature, static type system, ability to blend functional and imperative paradigms, and good library support. OCaml's multi-paradigm nature is suitable for implementing HotStuff, as the core state machine can be elegantly expressed in a functional way, whereas interacting with the RPC library to send messages is better suited to an imperative paradigm. Additionally, the Tezos cryptocurrency [cite!!!] is written in OCaml and contains a cryptography library (section \ref{tezos}) that provides the functionality needed by HotStuff.

The performance bottlenecks for distributed byzantine algorithms are generally cryptography, message serialisation and network delays. This means that it is more important to choose a language with suitable features to aid implementation, rather than picking a `high-performance' language like C++. [mention multicore OCaml, it was not ready when this project began]

OCaml has a powerful module system that facilitates writing highly reusable code. The module system was only briefly touched upon in the tripos (in Concepts in Programming Languages from IB), so I spent time learning about these features. Modules provide an elegant interface for the core state machine to interact with the imperative parts of the program that actually send messages over the network.

There is no existing reference implementation of HotStuff in OCaml, so my project contributes to the growing OCaml ecosystem\footnote{Since our project is implemented purely in OCaml, it could be deployed in a MirageOS unikernel \cite{noauthor_mirageos_nodate}}.

\subsection{Lwt}
Lwt \cite{noauthor_lwt_2023} is a concurrent programming library for OCaml. It allows the creation of promises, which are values that will become determined in the future; these promises may spawn threads that perform computation and I/O in parallel. In order to use Lwt I had to learn about monads, which are ways of sequencing effects in functional languages [verify!!!] and are used by asynchronous promises in Lwt. Lwt is useful to this project as promises provide a way to asynchronously dispatch messages over the network and wait for their responses in different threads. Promises are cheap to create in Lwt, so one can create many lightweight threads with good performance [citation needed***].

One alternative I could have used is Jane Street's Async library \cite{noauthor_async_nodate}, which has similar features but better performance; however, I chose not to due to its poor documentation. Another alternative that also has better performance than Lwt is EIO \cite{noauthor_eio_2023}, but this library is new and not yet in a stable state.

\subsection{Cap'n Proto}
Cap'n Proto  \cite{noauthor_capn_nodate} is an RPC framework that includes a library for sending and receiving RPCs [mention message serialisation!!!], and a schema language for designing the format of RPCs that can be sent. Benchmarks for the library are presented in section \ref{capnpbenchmark}.

[some people use xyz fancy library that is much faster but this was clearly not feasible for this project\dots]

\subsection{Tezos cryptography} \label{tezos}
The Tezos cryptography library \cite{noauthor_tezos_nodate} provides aggregate signatures using the BLS12-381 elliptic curve construction [cite crypto paper???]. It provides functions to sign some data using a private key, aggregate several signatures into a single one, and check whether an aggregate signature is valid. The only difference from the threshold signatures needed by HotStuff is that each individual signature in an aggregate signature can sign different data, whereas with threshold signatures each individual signature is over the same data. It is trivial to implement threshold signatures using this library by checking that the data is the same for all signatures inside the aggregate signature. Benchmarks for the library are presented in section \ref{tezosbenchmark}.

\section{Requirements analysis} \label{requirements}

In order to be successful the implementation should conform to the following requirements:
\begin{itemize}
	\item Correctness --- the consensus algorithm should be implemented as it is described in the paper \cite{yin2019hotstuff}. This can be established by testing the program trace for compliance with the algorithm specification.
	\item Evaluation --- analysis of system throughput and latency should be carried out by testing the program locally, analysing the trace, and testing in simulated network.
	\item Optimisation --- implement features to improve transaction throughput and reduce latency over the naive implementation. This can be achieved through architectural decisions, tuning the scheduler, and ensuring cryptographic libraries are being used efficiently.
\end{itemize}

% These requirements are similar to those presented in my proposal (Appendix X***) with a few differences. The first difference is to evaluate on 4 nodes rather than 32. Benchmarking of Cap'n Proto has revealed its limitations when sending large messages (\ref{capnpbenchmark}). Once batching of requests is implemented the internal messages sent between nodes will be large and could cause a performance bottleneck for the state machine progressing. Because of this, it may not be feasible to get reasonable performance with more nodes, as more nodes result in more internal messages being sent. In practice, permissioned blockchains are often run with a small number of nodes, so this limitation may not be important [citation needed *** honeybadger BFT?].
% https://github.com/heidihoward/distributed-consensus-reading-list#bft-surveys

% Additionally the extension has been changed from adding support for network reconfiguration to describing how to implement verifiable anonymous identities. This is because this extension seemed like a more interesting direction for the project with more exciting applications.
% mention how the reconfig extension was dropped for focusing on performance improvements

\section{Software engineering practices} \label{softwareeng}

\textit{In this section we describe the professional software engineering methodology deployed during implementation.}

\subsection{Development methodology} \label{devmethods}

For this project we used an iterative spiral development methodology. Objectives were chosen in accordance with the timetable set out in the proposal [reference appendix!]. Development then proceeded in cycles of implementation, testing, and analysis of the program trace and timing statements. This approach was particularly useful during the optimisation of our system (section \ref{performance}), which required extensive analysis of the logs and rapid development of different prototypes to compare performance.

\subsection{Testing \& debugging methodology} \label{testing}

Unit testing was carried out using `expect tests', which compare a program trace to the correct output. A testing suite of expect tests verifies that the program behaves as specified in the HotStuff paper. This suite has 100\% code coverage\footnote{The report says 97.89\% coverage, but the only uncovered code is the testing code itself.} of the consensus state machine code, the coverage report is at \textit{\_coverage/index.html}.

The Memtrace library and viewer \cite{noauthor_memtrace_nodate} were used to profile the memory usage of the program. One can generate a flame graph of memory allocations to see which parts of the program are using the most memory.

[talk about mininet!] \cite{noauthor_mininet_nodate}

Due to the distributed nature of the program, normal debugging tools and profilers are not useful for debugging deadlocks and performance issues. This is because the cause of deadlocks and performance issues is often some process waiting or a backlog of work forming, but this cannot be detected by tools that just track things like CPU usage. Instead, I had to rely on manual inspection of the program trace and commands that measure the real time taken for some part of the program to run.

[integrate this with the above to make one paragraph!!!]
As mentioned in section \ref{testing}, the nature of the project meant that debugging had to be carried out by manual inspection of the program trace and timing sections of the program. To overcome this I carried out tests in a scientific manner, constructing a hypothesis for why the program was slow based on analysing the program trace and timing statements, then attempting to test my hypothesis while controlling other variables, and finally implementing a solution.

[CI???]

\subsection{Source code management}

I used Git for version control and regularly pushed my local changes to a private GitHub repository.