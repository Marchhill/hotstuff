% This chapter is likely to be very short and it may well refer back to the Introduction. It might offer a reflection on the lessons learned and explain how you would have planned the project if starting again with the benefit of hindsight.

% general-use framework ??
\subsection{Future Work}
Further work would port to using the async library which is known to have better performance. It was out of scope to rewrite in async or use it in the first place due to poor documentation (although now I could look at the type signatures and understand the documentation). Hopefully reimplementing would give better performance and avoid the bugs of capnrpc. I would carry out more extensive tests on the message sending capabilities before diving into implementation. I would be more aware beforehand of the whole algorithm (including the pacemaker code) and implement based on the new pseudocode we have presented and proven correct. This would allow for better structuring of the code.

We have presented a potential path for implementing verifiable anonymous identities and reconfiguration using our permissioned blockchain, future work could consist of a practical implementation of this.