% The introduction should explain the principal motivation for the project and show how the work fits into the broad area of surrounding computer science and give a brief survey of previous related work. It should generally be unnecessary to quote at length from technical papers or textbooks. If a simple bibliographic reference is insufficient, consign any lengthy quotation to an appendix.

% 	importance of Blockchains
% 	My contributions:
% 	reference implementation in OCaml
% 	problems I faced
% 	more complete specification of HS

Blockchains are exciting because of their potential to decentralise many applications that are traditionally centralised. These include:
Central banks being replaced with decentralised cryptocurrencies 
Centralised DNS servers being replaced with distributed ones
Any possible algorithm that could run some algorithm on a centralised machine. This can be replaced with a decentralised virtual machine like the EVM that is Turing complete
One example of this is an NFT which is a 'smart contract' implemented on the EVM. This is a program that maintains a list of unique (fungible) tokens and their corresponding owners. These tokens could correspond to a file stored on the internet (images, music, videos), possibly in a decentralised manner using IPFS, or with a physical object like a car that could be unlocked with the token. The program allows requests from owners to transfer to others.

There are permissioned and permissionless blockchains. *Give examples of both and cases where they may be preferable to one another*

The underlying algorithm of blockchains is known as consensus. *Informally describe the problem and system model and formalise later*

They are pseudonymous, so can be used by people to hide their identity. They can be made completely anonymous like in currencies like Monero and Zcash by using zero knowledge succinct proofs. They have the benefits of protecting people's internet freedom, hiding their activities from oppressive governments, and removing the need to trust centralised authorities that could possibly be corrupt. They have the disadvantage of facilitating money laundering, and the trade of illicit materials (cite government’s position on blockchains and the silk road case study). They tend towards anarcho-capitalist systems with no regulation (cite play to earn games and darknet markets).

One way to combat this is with verifiable anonymous identities, which could combine the benefits of anonymity with some level of control, accountability, (and democracy?) without the need of a centralised authority. Could verify that an individual has the right to vote eg. They can only vote once, that they are a member of some federated area (voting constituency), they are an eligible citizen of the correct age - without compromising voting anonymity. Could facilitate decentralised direct democracy and anarchic governance in an organisation, community, or society, such as: ownership of a company by a large number of people, voting on policy decisions, running an organisation with a flat hierarchy (like a union, cooperative?)

Live testing revealed many subtle bugs, deadlocks, and performance issues (memory usage, messages dropped due to bugs preventing leaders from always advancing) that were time consuming to debug. However they helped me to flesh out a pacemaker algorithm based on these ad-hoc fixes that we have proven both correctness and liveness for. I also improved the ease of implementability based on practicalities I discovered during implementation (eg. using node offset and hashes to compare nodes)